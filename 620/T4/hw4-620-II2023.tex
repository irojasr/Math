\documentclass[12pt]{memoir}

\def\nsemestre {II}
\def\nterm {Fall}
\def\nyear {2023}
\def\nprofesor {Wolfgang Bangerth}
\def\nsigla {MATH620}
\def\nsiglahead {Variational Methods and Optimization I}
\def\nextra {HW4}
\def\nlang {ENG}
\let\footruleskip\relax %%FADIR
\input{../../headerVarillyDiff}

\begin{document}
\begin{Ej}
    On finite dimensional spaces, all norms are equivalent. Let us test
    this for a subset of all possible norms on $\bR^n$. For this, first
    recall the definition of the three norms we care about:
    \begin{align*}
      \| x \|_{l_1} &:= \sum_{i=1}^n |x_i|, \\
      \| x \|_{l_2} &:= \left(\sum_{i=1}^n |x_i|^2 \right)^{1/2}, \\
      \| x \|_{l_\infty} &:= \max_{i=1}^n |x_i|.
    \end{align*}
    Now show that there
    exist constants $0<c_{12}\le C_{12} <\infty$ so that
    \begin{align*}
      c_{12} \| x \|_{l_1} \le \| x \|_{l_2} \le C_{12} \| x \|_{l_1}
    \end{align*}
    for all vectors $x\in \bR^n$. In other words, that the $l_1$ and $l_2$
    norms on $\bR^n$ are ``equivalent''.
    
    Then show the same for the $l_2$ and $l_\infty$ norms: Show that there
    exist constants $0<c_{2\infty}\le C_{2\infty} <\infty$ so that
    \begin{align*}
      c_{2\infty} \| x \|_{l_2} \le \| x \|_{l_\infty} \le C_{2\infty} \| x \|_{l_2}
    \end{align*}
    for all vectors $x\in \bR^n$.
    
    Combine these estimates to show that the $l_1$ and the $l_\infty$
    norms are equivalent.
\end{Ej}

\begin{ptcbr}
\end{ptcbr}

\begin{Ej}
    Consider the (infinite dimensional) vector space $C^1([0,1])$.
Define on this space the same three norms as before:
\begin{itemize}
\item $\|u\|_1 = \max_{x \in [0,1]} |u(x)|$,
\item $\|u\|_2 = \max_{x \in [0,1]} |u(x)| + \max_{x \in [0,1]}
  |u'(x)|$,
\item $\|u\|_3 = \left(\int_0^1 |u(x)|^2 \, dx\right)^{1/2}$.
\end{itemize}
(You will notice that these are the $L^\infty$, $W^{1,\infty}$, and
$L^2$ norms when taking into account that the members of $X$ are all
functions that are continuous and continuously differentiable.) 
Then answer the following questions:
\begin{itemize}
\item[(a)] For each of these three norms, show that it is indeed a
  norm on $X$, i.e., that it satisfies the norm axioms.
\item[(b)] Are any of these norms equivalent to each other? If you
  can't show that they are, demonstrate that they are \textit{not}
  equivalent by showing that no constants $0<c\le C<\infty$ can exist as are
  necessary for equivalence of norms.
\end{itemize}
\end{Ej}
\begin{nonum-Rmk}
    Let's say you want to show that two norms are
    \textit{not} equivalent, say the first two. Then you need to show that either
    there is no constant $c_{12}>0$ so that $c_{12} \| u \|_{1} \le \| u
    \|_{2}$ for all $u$, or that there is no constant $C_{12}<\infty$ so
    that $\|u\|_2 \le C_{12} \| u \|_{1}$ for all $u$. For this, it is
    not sufficient to think about a single $u$, because for one specific
    $u$ the inequalities can always be made to work with an appropriate
    constant. What you have to do instead is to work with
    \textit{sequences of functions $u_k$} appropriately chosen, and to
    show that the inequalities cannot be true for all elements of a
    sequence at the same time. A different perspective is to see that
    the statement ``there exists $c_{12}>0$ so that $c_{12} \| u \|_{1} \le \| u
    \|_{2}$ for all $u$'' is equivalent to saying that
    $$
      0 < \inf_{u} \frac{\| u \|_{2}}{\| u \|_{1}}.
    $$
    To disprove that this is the case, you will want to find a sequence
    of functions $u_k$ so that
    $$
      \frac{\| u_k \|_{2}}{\| u_k \|_{1}} \rightarrow 0.
    $$
    If you can find such a sequence, then clearly $\inf_{u} \frac{\| u
      \|_{2}}{\| u \|_{1}}=0$, which then implies that there is no
    $c_{12}>0$ so that $c_{12} \| u \|_{1} \le \| u
    \|_{2}$ for all $u$.
  
    For the other inequality, consider that the statement ``there exists
    $C_{12}<\infty$ so that $\|u\|_2 \le C_{12} \| u \|_{1}$'' is
    equivalent to
    $$
      \sup_{u} \frac{\| u \|_{2}}{\| u \|_{1}} < \infty.
    $$
    To disprove this statement, you want to find a sequence so that
    $$
      \frac{\| u_k \|_{2}}{\| u_k \|_{1}} \rightarrow \infty.
    $$
\end{nonum-Rmk}

\begin{ptcbr}
\begin{enumerate}[(a)]
    \item First observe that all the norms are positive. The absolute value is always positive, sum of positive terms is positive as well and the integral of a positive function is positive.\par 
    Assume now that $\norm*{u(x)}_1=0$, so for all $x\in\bonj{0,1}$, $|u(x)|$ is sandwiched between $0$ and $0$ this means that $u(x)$ must be identically zero.\par 
    In a similar fashion $\norm{\.}_2$ is a sum of positive terms which we assume to be zero. This means that both of them must be zero and by a similar argument we can deduce that $u(x)$ is also zero.\par 
    Finally if $\norm*{u}_3$ is zero then $\int_{[0,1]}|u(x)|^2\dd x$ is also zero. We know that if the integral of a positive function is zero, then the function itself must've been zero to begin with. This means that 
    $$|u(x)|^2=0\To u(x)=0,\word{for all}x.$$
    Now suppose $c$ is a scalar. Then 
    $$\norm*{cu}_1=\max_{\bonj{0,1}}|cu(x)|=\max_{\bonj{0,1}}|c||u(x)|=|c|\max_{\bonj{0,1}}|u(x)|=|c|\norm*{u}_1,$$
    while for the second case we have
    $$\norm*{cu}_2=\max_{\bonj{0,1}}|cu(x)|+\max_{\bonj{0,1}}|cu'(x)|=\max_{\bonj{0,1}}|c||u(x)|+\max_{\bonj{0,1}}|c||u'(x)|=|c|\norm*{u}_2.$$
    In the final case we take advantage of the linearity of the integral:
    $$\left(\int_{\bonj{0,1}}|cu|^2\dd x\right)^{1/2}=\left(|c|^2\int_{\bonj{0,1}}|u|^2\dd x\right)^{1/2}=|c|\left(\int_{\bonj{0,1}}|u|^2\dd x\right)^{1/2}=|c|\norm*{u}_3.$$
    Now, take another continuous function $v$, it holds that
    $$|u(x)|\leq\sup_{\bonj{0,1}}|u(x)|,\word{and}|v(x)|\leq\sup_{\bonj{0,1}}|v(x)|.$$
    This means that $\sup_{\bonj{0,1}}|u(x)|+\sup_{\bonj{0,1}}|v(x)|$ is an upper bound for $|u(x)|+|v(x)|$. From this, using the usual triangle inequality we have 
    $$\sup_{\bonj{0,1}}|u(x)+v(x)|\leq\sup_{\bonj{0,1}}(|u(x)|+|v(x)|)\leq\sup_{\bonj{0,1}}|u(x)|+\sup_{\bonj{0,1}}|v(x)|.$$
    As all of the functions are defined on the compact set $[0,1]$, then we may exchange all supremums by maximums which verifies the triangle inequality for $\norm*{\.}_1$.\par 
    Using this fact, we may conclude $\norm*{\.}_2$ is a norm, because
    $$\norm*{u+v}_2=\norm*{u+v}_1+\norm*{u'+v'}_1$$
    so applying the triangle inequality for $\norm*{\.}_1$ and arranging terms we get 
    $$\norm*{u+v}_2\leq\norm*{u}_1+\norm*{v}_1+\norm*{u'}_1+\norm*{v'}_1=\norm*{u}_2+\norm*{v}_2.$$
    Without appealing to the Minkowski inequality, but using the Cauchy-Schwarz inequality we have 
    $$\int_{\bonj{0,1}}|u+v|^2\dd x\leq \int_{\bonj{0,1}}(|u|+|v|)^2\dd x=\int_{\bonj{0,1}}|u|^2+|v|^2+2|u||v|\dd x$$
    and by the Cauchy-Schwarz inequality we have 
    $$\int_{\bonj{0,1}}|u||v|\dd x\leq\left(\int_{\bonj{0,1}}|u|^2\dd x\right)^{1/2}\left(\int_{\bonj{0,1}}|v|^2\dd x\right)^{1/2}.$$
    Combining this with the previous inequality we have
    \begin{align*}
    \int_{\bonj{0,1}}|u+v|^2\dd x&\leq \int_{\bonj{0,1}}|u|^2+|v|^2+2\left(\int_{\bonj{0,1}}|u|^2\dd x\right)^{1/2}\left(\int_{\bonj{0,1}}|v|^2\dd x\right)^{1/2}\\
    &=\bonj{\left(\int_{\bonj{0,1}}|u|^2\dd x\right)^{1/2}+\left(\int_{\bonj{0,1}}|v|^2\dd x\right)^{1/2}}^2
    \end{align*}%https://math.stackexchange.com/questions/710254/triangle-inequality-for-l2-norm
    \item \red{FINISH COUNTEREXAMPLES}
\end{enumerate}
\end{ptcbr}
\begin{Ej}
    Consider the ($n+1$ dimensional) space of polynomials of degree at most $n$:
\begin{align*}
  X := \left\{ 
   u : [0,1] \rightarrow \bR \; :
   u=\sum_{k=0}^n a_k x^k
  \right\}.
\end{align*}
Define on this space the following norms (the same as in the previous problem):
\begin{itemize}
\item $\|u\|_1 = \max_{x \in [0,1]} |u(x)|$,
\item $\|u\|_2 = \max_{x \in [0,1]} |u(x)| + \max_{x \in [0,1]}
  |u'(x)|$,
\item $\|u\|_3 = \left(\int_0^1 |u(x)|^2 \, dx\right)^{1/2}$.
\end{itemize}
Then answer the following questions:
\begin{itemize}
\item[(a)] For each of these three norms, show that it is indeed a
  norm, i.e., that it satisfies the norm axioms. You should be able to
  infer this from the work you have done for the previous problem;
  instead of repeating the work for the special case here, feel free
  to argue why the work for the previous problem \textit{implies} what you are
  asked to show here.
\item[(b)] Show that all of these norms are equivalent on $X$ -- as they
  ought to be because the space $X$ is finite dimensional.
\end{itemize}
\end{Ej}
\begin{nonum-Rmk}
    It turns out to actually be quite complicated to
  show the equivalence of norms, even though we know that it is true. While the equivalence of
  $\|\cdot\|_1$ and  $\|\cdot\|_2$ is not difficult, the one with the
  $\|\cdot\|_3$ norm is. In the end, I would be surprised if any of you figure
  out all pieces of the proof without substantial time reading up on
  the material and spending more time on the problem than it is likely
  worth given your other obligations (including leisure time). That's
  ok -- give it a shot and see how far you get, I will accept partial results!
\end{nonum-Rmk}

\begin{ptcbr}
\begin{enumerate}[(a)]
    \item Observe that every polynomial is a smooth in the whole of $\bR$, so in particular it is of class $C^1$ across $\bR$. In this fashion, the polynomials form a subspace of the $C^1$ functions.\par 
    Now, a norm restricted to a subspace continues being a norm. So this means that the previous functions do define norms on the space of polynomials we have.
    \item 
\end{enumerate}
\end{ptcbr}
\begin{Ej}
    
Let's build some intuition. If one thinks of functions that have $k$
derivatives, i.e., are in the set of functions usually denoted by
$C^k(\Omega)$, then we think of them as quite smooth and definitely
not singular. Is this true also for the functions in $W^{k,p}$ that
have $k$ \textit{weak} derivatives whose $p$th power is integrable?
Recall that
\begin{align*}
  W^{k,p}(\Omega) = \left\{ \varphi: \ \int |u|^p < \infty,
  \quad
  \int |\nabla u|^p < \infty,
  \quad
  \ldots,
  \quad
  \int |\nabla^k u|^p < \infty
  \right\}.
\end{align*}
To test whether a given function $u$ is in $W^{k,p}$ for some $k,p$ on
\textit{bounded} domains $\Omega$,
we really only have to check that the highest order term $\int |\nabla^k u|^p < \infty$
because if this is the case then all of the lower order derivatives
also have finite integrals.

Now think about singular functions $u : B_1(0)\subset \bR^d \rightarrow \bR\cup\{\pm\infty\}$
of the form
\begin{align*}
  u(x) = \frac{1}{\|x\|^s}
\end{align*}
with $s>0$. (The domain $\Omega=B_1(0)$ is the ball of radius 1
around the origin 0. For simplicity of calculations, use the Euclidean norm
for $\|x\|$ -- although the equivalence of norms on finite dimensional
spaces implies that none of the results below actually depend on which
norm $\|x\|$ you choose.)

For given values of the space dimension $d\ge 1$, the degree $k\ge 0$,
and the exponent $1\le p\le \infty$, state for which values $s$ the
function satisfies $u \in W^{k,p}(B_1(0))$.

The spaces $H^k=W^{k,2}$ have special importance in the theory of
partial differential equations.
Does the space $H^1 = W^{1,2}$ contain any singular functions with $s>0$ for
$d=1$? For $d=2$? For $d>3$? How about the space $H^2=W^{2,2}$?
\end{Ej}

\begin{ptcbr}
To check membership in $W^{k,p}$, it suffices to analyze $\nb^k u$. It may appear as an insurmountable task to analyze $|\nb^k u|$, however we may take it as 
$$|\nb^k u|=\sup_{\substack{|\al|\leq k\\ |x|\leq 1}}|\del^\al u(x)|,\word{where}\al\in[k]^d.$$
\end{ptcbr}
\begin{Ej}
    The previous problem let you explore whether functions of the form
  \begin{align*}
    u_s(x) = \frac{1}{\|x\|^s}
  \end{align*}
  are in $W^{k,p}$. (The result also depends on the space dimension
  $d$.) One can turn this around: Given $k,p,d$, for which set
  $S(k,p,d)\subset \bR$ of exponents $s$ do these functions lie in $W^{k,p}$?

  We can use this to speculate whether a space $W^{k',p'}$ may be a
  subset or superset of a space $W^{k,p}$ in the following way: Let's
  say that $u_s \in W^{k,p}$ for all $s\in S(k,p,d)$ and 
  $u_s \in W^{k',p'}$ for all $s\in S(k',p',d)$, and that 
  $S(k,p,d) \supset S(k',p',d)$, then we might believe that 
  $W^{k,p} \subset W^{k',p'}$ in this space dimension $d$.

  \begin{itemize}
  \item[(a)] Explain why the speculative inference
  $$
    S(k,p,d) \supset S(k',p',d)
    \qquad\To\qquad
    W^{k,p} \subset W^{k',p'}
  $$
  makes sense.

  \item[(b)] In the previous problem, you found what values $s$ lead
    $u_s$ to be in $W^{k,p}$. That is, you have an explicit
    representation for the set $S(k,p,d)$. Express the statement
    $$
    S(k,p,d) \supset S(k',p',d)
    \qquad\To\qquad
    W^{k,p} \subset W^{k',p'}
  $$
  in a way where the left side of the inference is simply an
  inequality involving $k,k',p,p',d$.

  \item[(c)] Explain why the statement in (a) is just speculation, rather than a
    proof.

  \item[(d)] Statements about the inclusion $W^{k,p} \subset
    W^{k',p'}$ are collectively known as
    \href{https://en.wikipedia.org/wiki/Sobolev_inequality}{Sobolev
      inequalities}. Look up the Sobolev inequalities for $W^{k,p} \subset
    W^{k',p'}$ and compare with the speculation above. 
  \end{itemize}
\end{Ej}

\subsection*{Project}
\end{document} 
