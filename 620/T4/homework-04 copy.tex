\documentclass{article}
\usepackage{amsmath}
\usepackage{amsthm}
\usepackage{amssymb}
\usepackage{amsfonts}
\usepackage{mathtools}
\usepackage{listings}
\title{MATH 620: Homework 4}
\author{Fernando}
\date{\today}
\begin{document}
\maketitle
\section*{Problem 1}
%\includegraphics[width=0.99\textwidth]{prob1Part1hw4.png}

%\includegraphics[width=0.95\textwidth]{prob1Part2hw4.png}
\section*{Problem 2}
\subsection*{Norm axioms}
\subsubsection*{$||\cdot||_1$}
All the axioms follow from the properties of the $\max$
\begin{itemize}
\item $\max_{x\in[0.1]}|au(x)|=|a|\max_{x\in[0.1]}|u(x)|$
\item $\max_{x\in[0.1]}|u(x)|=0 \implies \forall x \in [0,1] (|u(x)|\leq 0) \implies u\equiv0$
\item $\max_{x\in[0.1]}|u(x)+v(x)| \leq \max_{x\in[0.1]}|u(x)|+\max_{x\in[0.1]}|v(x)|$
\end{itemize}
The last statement is true because if we find $x_M$ that maximizes the sum $u+v$
then 
\begin{align*}
\max_{x\in[0.1]}|(u+v)(x)|= |(u+v)(x_M)|&\leq|u(x_M)|+|v(x_M)|\\
					&\leq\max_{x\in[0.1]}|u(x)|+\max_{x\in[0.1]}|v(x)|.
\end{align*}
\subsubsection*{$||\cdot||_2$}
Notice that $||u||_2=||u||_1+||u'||_1$. So all the norm requirements follow
from the fact that $||\cdot||_1$ is a norm and that the derivative is linear.
\subsubsection*{$||\cdot||_3$}
In these case the only property that is not immediate is the triangular
inequality. So let's proof that.

By Cauchy-Schwarz:
\[
	\int_0^1|u||v|\leq \left( \int_0^1|u|^2\right)^{1/2} 
	\left(\int_0^1|v|^2\right)^{1/2}
\]
Then:
\begin{align*}
	\int_0^1|u+v|^2&=\int_0^1|u|^2+\int_0^1|v|^2+2\int_0^1uv\\
		&\leq\int_0^1|u|^2+\int_0^1|v|^2+2\int_0^1|u||v|\\
		&\leq\int_0^1|u|^2+\int_0^1|v|^2+2\left(
		\int_0^1|u|^2\right)^{1/2}\left(\int_0^1|v|^2\right)^{1/2}\\
		&=\left(\left(\int_0^1|u|^2\right)^{1/2}+\left(\int_0^1|v|^2\right)^{1/2}\right)^2.
\end{align*}
Taking the square root we obtain the result.
\subsection*{Equivalence of norms}
No two norms are equivalent in this case. Notice that a consequence of the definition of
equivalence of norms is the following:

If $||\cdot||_a$ and $||\cdot||_b$ are equivalent then any sequence that is
convergent in $||\cdot||_a$ must also be convergent to the same limit point in
$||\cdot||_b$. This is because $x_n\to x$ in norm $||\cdot||_a$ means
$||x_n-x||_a\to 0$, and due to the equivalence of norms
\[
	||x_n-x||_b\leq c ||x_n-x||_a
\]
we have
\[
	||x_n-x||_b\to 0.
\]
\subsubsection*{$||\cdot||_1$ vs $||\cdot||_2$}
Since $||u||_2=||u||_1+||u'||_1$ it is clear that convergence in $||\cdot||_2$
implies convergence in $||\cdot||_1$ but the converse is not true.
Pick for example $x_n=x^n/n$. It is easy to see that $x_n\to 0$ in
$||\cdot||_1$ but because $\frac{d}{dx}(x^n/n)=x^{n-1}$ then $||x_n||_2=1$.
So $x_n\nrightarrow 0$ in $||\cdot||_2$. So these norms are not equivalent.
\subsubsection*{$||\cdot||_1$ vs $||\cdot||_3$}
Notice that:
\[
||\cdot||_3^2=\int_0^1|u|^2\leq \int_0^1 ||u||_1^2=||u||_1^2.
\]
So convergence in $||\cdot||_1$ implies convergence in $||\cdot||_3$. But the
converse is not true. Consider $x_n=x^n$. Clearly $||\cdot||_1=1$ but
$||x_n||_3=(1/(2n+1))^{1/2}$. So $x_n \to 0$ in $||\cdot||_3$ but
$x_n\nrightarrow 0$ in $||\cdot||_1$. So these norms are not equivalent.
\subsubsection*{$||\cdot||_2$ vs $||\cdot||_3$}
Convergence in $||\cdot||_2$ implies convergence in $||\cdot||_1$ which in turn
implies convergence in $||\cdot||_3$, but the converse is not true. Take the
same example as before. Again these norms are not equivalent.
\section*{Problem 3}
\subsection*{Norm axioms}
In the previous section we proved this for any continuously differentiable
function on [0,1] so in particular this is true for all polynomials.
\subsection*{Norm equivalence}
In order to prove that these norms are equivalent we can just repeat the proof
that all norms are equivalent which only uses the fact that the norm is
continuous and the unit ball is compact, but given the conversation we had in
class we want to find an explicit constant for each case.
\subsubsection*{$||\cdot||_1$ vs $||\cdot||_3$}
We want to find $c$ and $C$ so that
\[
	c||p||_3\leq ||p||_1 \leq C||p||_3.
\]
Notice that we can take $c=1$ because
\[
||p||_3^2=\int_0^1|p|^2\leq \int_0^1||p||_1^2=||p||_1^2.
\]
Now let's find $C$.

Notice that this is equivalent to finding $C$ such that
\[
	\bigg|\bigg|\frac{p}{||p||_3}\bigg|\bigg|_1\leq C,
\]
so without loss of generality we can consider $p$ with $||p||_3=1$. And show that
\[
	||p||_1\leq C.
\]

In order to continue with the proof consider the shifted Lengedre polynomials
(the usual Legendre polynomials are defined on [-1,1] but the shifted Legendre
polynomials are defined on [0,1]). We will denote the $i-th$ shifted Legendre
polynomial by $L_i$. These polynomials have the following properties which I
will not proof because it would be too long:
\begin{enumerate}
	\item $\{L_i\}_{i=0,\dots,n}$ forms a basis for $X$
	\item $\int_0^1 L_iL_j=\frac{\delta_{ij}}{2i+1}$
	\item $||L_i||_1=\max_{x\in[0,1]}|L_i|\leq 1 \quad \forall i$
\end{enumerate}

\textbf{Proof:}

Take $p$ with $||p||_3=1$. By property (1) we can write $p$ as:
\[
	p=\sum_0^n a_iL_i,
\]
then using property (2) we get
\[
	1=\int_0^1|p|^2=\int_0^1\left(\sum_0^na_iL_i\right)^2=\sum_0^n\frac{a_i^2}{2i+1},
\]
which implies that $\frac{|a_i|}{\sqrt{2i+1}}\leq 1$ for $i=0,\dots,n$.

Now notice that
\[
	||p||_1\leq\sum_0^n|a_i|||L_i||_1 \leq \sum_0^n\sqrt{2i+1}.
\]
Where we used the previous inequality and property (3). So our explicit
constant is:
\[
	C=\sum_0^n\sqrt{2i+1}.
\]
\section*{Problem 4}
\end{document}