\documentclass[12pt]{memoir}

\def\nsemestre {II}
\def\nterm {Fall}
\def\nyear {2023}
\def\nprofesor {Wolfgang Bangerth}
\def\nsigla {MATH620}
\def\nsiglahead {Variational Methods and Optimization I}
\def\nextra {HW5}
\def\nlang {ENG}
\let\footruleskip\relax %%FADIR
\input{../../headerVarillyDiff}

\begin{document}
\begin{Ej}[Weak derivatives and membership in $W^{1,2}$]
  In talking about the definition of weak derivatives, we have
  discussed that functions with kinks and, in some cases, functions
  that are discontinuous can have weak derivatives even though they
  clearly are not differentiable in the ``classical'' sense.
  
But not all discontinuous functions have a weak derivative. For example,
the following function of one variable,
\begin{align*}
  u(x) = \begin{cases}
    -1 & \text{if}\; x<0
    \\
    0 & \text{if}\; x\ge 0,
  \end{cases}
\end{align*}
does not have a weak derivative. Convince yourself that that is true,
and explain your thinking.

Next, consider the function $u : \bR^2 \rightarrow \bR$ of two arguments,
\begin{align*}
  u(\mathbf x) = \sin(\arctan(x_2/x_1))
\end{align*}
that is discontinuous at the origin?%
\footnote{For the given expression for $u$ to make sense, we need to
  make sure that you define $\arctan$ with the right branch cut. It is
  here to be understood as $\arctan(x_2/x_1)=\theta$ where $\theta$ is
  the angle a point $\mathbf x\in{\mathbb R}^2$ makes against the
  positive $x_1$ axis.}
Can you guess a weak gradient for
this function $u$ (which is of course a two-dimensional vector field) and
prove that it really is \textit{the} weak gradient?

If so, what spaces $W^{1,p}(B_1(0))$ is $u$ in if we take
the unit ball in $\bR^2$ as the domain?

(Hint: Plot the function. Then think about whether there is possibly a
coordinate system better suitable to the task.)
\end{Ej}

\begin{ptcbr}
  
\end{ptcbr}

\begin{Ej}
    The dual space $X'$ of a vector space $X$ is the set of all linear,
continuous functionals $\varphi: X \rightarrow \bR$.

We say that a functional $\varphi$ is
bounded if it satisfies the condition
\begin{align*}
  |\varphi(x)| \le c \|x\|_X
\end{align*}
for all $x\in X$ and with some constant $c=c(\varphi)<\infty$.

Show that (i)~if $\varphi\in X'$, then it is bounded; and
(ii)~that if a linear functional is bounded, then it is also
continuous. (In other words, a linear functional $\varphi$ is in $X'$
\textit{if and only if} it is also bounded.)
\end{Ej}

\begin{ptcbr}
  Suppose $\vf$ is bounded, then there is a $c>0$ such that 
  $$|\vf(x)|\leq c\norm{x}.$$
  For $\eps>0$, let $\dl=\eps/c$ and suppose $\norm{x-y}\leq\dl$ for $x,y\in X$. Then 
  $$|\vf(x)-\vf(y)|=|\vf(x-y)|\leq c\norm{x-y}\leq c\dl=\eps$$
  and so $\vf$ is continuous.\par 
  If $\vf$ were continuous, it would be continuous at $y=0$. So for all $\eps$, in particular $\eps=1$, there exists a $\dl>0$ such that $\norm{x}\leq\dl$ implies 
  $$|\vf(x)|=|\vf(x)-\vf(0)|\leq 1.$$
  About zero, $\vf$ is bounded by any value as $\vf(0)=0$, so now take $x\neq 0$. The vector $\dl(x/\norm{x})$ has norm $\dl$, so by our previous observation it holds that
  $$\vf\left(\dl\frac{x}{\norm{x}}\right)\leq 1\To \vf(x)\leq\frac{1}{\dl}\norm{x}.$$
  Thus $\vf$ is bounded and therefore, boundedness is equivalent to continuity here.
\end{ptcbr}
\begin{Ej}
    Every $a\in L^q$ induces a
bounded linear functional $\varphi:L^p\rightarrow \bR$ of the form
\begin{align*}
  \varphi(u) = \int_\Omega a(x) u(x) \, \text{d}x.
\end{align*}
Here and below, we will always assume that $\frac 1p + \frac 1q = 1$,
and that $\Omega\subset\bR^d$ is a bounded domain.

Show in a first step that this functional is linear and bounded (and
consequently continuous), i.e., that indeed we have $\varphi\in (L^p(\Omega))'$.

It is of course conceivable that $X'$ is indeed larger than just the
functionals introduced above. One of the possibilities would be that
one could choose a larger class of functions $a$ than just the $L^q$
functions above. Show that this is not the case, i.e., that a function
$a\not\in L^q$ (for example if $a\in L^r$ with $r<q$ but $a\not\in
L^q$) does not induce a functional $\varphi\in X'$.

This statement is not easy to show in its full generality -- though as
often, it's all about finding the right approach. Here, it means
showing that for such an $a$, $\varphi$ can not be linear and
continuous (or linear and bounded). If you
don't see how to do this, create an example: Pick a particular $p$ and
corresponding $q$, then choose a specific $a\in L^r\backslash L^q$
(for example, a function with a singularity) and show that the
corresponding $\varphi$ is either not linear, not continuous, or not
bounded by playing with functions $u\in L^p$ and investigating what
$\varphi(u)$ is.

(It is worth noting that this argument only shows that linear
functionals of the form shown above with $a \in L^r\backslash L^q$ do
not give rise to functionals $\varphi$ in $(L^p(\Omega))'$. It does
not show that there are no \textit{completely different} ways to
construct linear and continuous functionals. It is the
\href{https://en.wikipedia.org/wiki/Riesz_representation_theorem}{Riesz
representation
theorem}
that states that such other ways do not, in fact, exist.)
\end{Ej}

\begin{ptcbr}
  To show that $\vf$ is linear, consider $u,v\in L^p$ and $c\in\bR$, then 
  \begin{align*}
  \vf(cu+v)&=\int_\Om a(x)(cu+v)(x)\dd x\\
  &=\int_\Om ca(x)u(x)+a(x)v(x)\dd x\\
  &=c\int_\Om a(x)u(x)\dd x+\int_\Om a(x)v(x)\dd x\\
  &=c\vf(u)+\vf(v)
  \end{align*}
  which shows $\vf$ is linear. On the other hand observe that using H\"older's inequality we have 
\begin{align*}
  &|\vf(u)|=\left|\int_\Om a(x)u(x)\dd x\right|\leq\norm{a}_{L^q}\norm{u}_{L^p}\\
  \To& \frac{|\vf(u)|}{\norm{u}_{L^p}}\leq\norm{a}_{L^q}\\
  \To&\sup_{u\neq 0}\frac{|\vf(u)|}{\norm{u}_{L^p}}\leq\norm{a}_{L^q}
\end{align*}
  and this last quantity is the operator norm of $\vf$. This means that $\norm{\vf}_{\text{op}}\leq\norm{a}_{L^q}$ and so $\vf$ is a bounded functional.\par
  Let us consider the function $1/\sqrt{x}$ which is in $L^1([0,1])$, its integral is $2$, but not in $L^2([0,1])$. Observe that linearity doesn't depend on our choice of function, what should fail is boundedness of the functional.
\end{ptcbr}
\begin{Ej}
    The previous problem claimed that every (dual) functional $\varphi\in (L^p(\Omega))'$
can be written in the form
\begin{align*}
  \varphi(u) = \int_\Omega a(x) u(x) \, \text{d}x
\end{align*}
for some $a\in L^q$. Furthermore, we have defined the norm on the dual
space as
\begin{align*}
  \|\varphi\|_{(L^p(\Omega))'} =
  \sup_{u\in L^p(\Omega)} \frac{|\varphi(u)|}{\|u\|_{L^p(\Omega)}}.
\end{align*}
Prove that
\begin{align*}
  \|\varphi\|_{(L^p(\Omega))'} = \|a\|_{L^q(\Omega)}.
\end{align*}

(Hint: It is not difficult to show that $\|\varphi\|_{(L^p(\Omega))'}
\le \|a\|_{L^q(\Omega)}$ using H\"older's inequality. To complete the proof, find a $u\in L^p$ so
  that $|\varphi(u)|=\|a\|_{L^q(\Omega)}\|u\|_{L^p(\Omega)}$ for which
  a good approach is to try $u(x)=|a(x)|^s \,\text{sign}(a(x))$ with some exponent $s$ to
  be chosen conveniently. You'll then have to check that this $u$ is
  in $L^p$, as well as what $\varphi(u)$ is.)
\end{Ej}

\begin{Ej}
    Define the function $w$ on $[0,1]$ as
\begin{align*}
  w(x) = \begin{cases}
    \alpha & \text{if $x\le\frac 12$} \\
    \beta & \text{if $x>\frac 12$},
  \end{cases}
\end{align*}
and let $\bar w$ be its periodic extension to all of $\bR$.

Next consider the following sequence of functions on the set $\Omega=[0,1]$:
\begin{align*}
  u_n(x) = \bar w(nx).
\end{align*}
In other words, $u_n(x)$ takes the periodic extension $\bar w(x)$,
compresses it by a factor of $n$ in $x$ direction, and then restricts
consideration to the domain $[0,1]$.

Show the following statements
\begin{itemize}
\item[(a)] The sequence $u_n$ does not converge strongly to any $u$ in
  any of the $L^p$ spaces, $1\le p \le \infty$.

\item[(b)] We have
\begin{align*}
  u_n(x) \rightharpoonup \frac{\alpha+\beta}{2} \qquad \text{in $L^p$}
\end{align*}
for any $1\le p<\infty$.
  
\item[(c)] We have
\begin{align*}
  u_n(x) \overset{\ast}{\rightharpoonup} \frac{\alpha+\beta}{2}  \qquad \text{in $L^\infty$}.
\end{align*}
\end{itemize}

(A similar result is actually true for more general cases: If you
start with \textit{any} bounded function $w:[0,1]\rightarrow \bR$ and its
periodic extension $\bar w$, then the $u_n$ defined above converges
weakly or weak-* to the \textit{mean value} $\int_0^1 w(x)\,
\text{d}x$. The proof is not much different but does not shed any
further light on the issue, so we are content with the simpler case above.)
\end{Ej}

\begin{ptcbr}
  \begin{enumerate}[(a)]
    \item Let us assume that $u_n$ converges weakly to $1/2(\al+\bt)$. Observe that if $u_n$ were to converge strongly, then it should converge to its weak limit and for sake of argument assume $\bt=-\al$, in this case the limit is zero.\par 
    If we let $m=\min(\al,-\al)$ then taking the $L^p$ norm of $u_n$ we get
    $$\norm{u_n}_{L^p}^p=\int_0^1 |u_n(x)|^p\dd x=\int_0^1 |\ov{w}(nx)|^p\dd x\geq \int_0^1 |m|^p\dd x=|\al|^p$$
    which means that if $\al\neq 0$ (and we are assuming ($\bt=-\al$)), then the sequence is always bounded below and thus can't go to zero in $L^p$ norm. On the other hand if it were the case that $\al=0$ then the result would be true and $u_n$ would converge to zero.
    \item We must show that $\int_0^1u_nv\dd x\to0$ for $v\in L^q$ where $1/p+1/q=1$. Let $0=a_0<a_1<\dots<a_r=1$, by density of step functions in $L^q$ we have that
    $$s(x)=\sum_{k=1}^r \al_k\ind_{A_k}(x),\word{where}A_k=\obonj{a_{k-1},a_k}\footnote{I know this is the other notation, but I just get too confused when reading it with parenthesis because I read it as an ordered pair.}$$
    approximates $v$: $\norm{v-s}_{L^q}\leq\eps$ for $\eps>0$. Then 
    $$\int_0^1u_nv\dd x=\int_0^1u_n(v-s+s)\dd x=\int_0^1u_n(v-s)\dd x+\int_0^1u_ns\dd x.$$
    We may bound this quantity using the triangle inequality as follows:
    $$\left|\int_0^1u_nv\dd x\right|\leq\int_0^1|u_n||v-s|\dd x+\left|\int_0^1u_ns\dd x\right|.$$
    The first integral we bound using H\"older's inequality as follows: 
    $$\int_0^1|u_n||v-s|\dd x\leq\norm{u_n}_{L^p}\norm{v-s}_{L^q}$$
    and observe that from our first discussion about $u_n$ we may deduce that its norm is actually $w$'s norm. We may use a change of variables to see that 
    $$\int_0^1 |\ov{w}(nx)|^p\dd x=\int_{0}^{n}|\ov{w}(y)|^p(\dd y/n),\word{where}\left(\substack{y=nx\ \To\ \dd y=n\dd x\\ x=0,x=1\ \To\ y=0,y=n}\right).$$
    But the last integral contains a periodic function of period 1. So this is actually 
    $$\int_{0}^{n}|\ov{w}(y)|^p(\dd y/n)=n\int_{0}^{1}|\ov{w}(y)|^p(\dd y/n)=\int_{0}^{1}|\ov{w}(y)|^p\dd y=\norm{w}_{L^p}^p.$$
    This lets us conclude that $\norm{u_n}_{L^p}=\norm{w}_{L^p}$ and as $w$ is a step function, we can find its integral to be
    $$\norm{w}_{L^p}^p=\int_0^{1/2}|\al|^p\dd x+\int_{1/2}^1|\bt|^p\dd x=\frac{|\al|^p+|\bt|^p}{2}<\infty.$$
    So in this way we may finish bounding our original first integral as $\int_0^1|u_n||v-s|\dd x\leq \eps\norm{w}_{L^p}$.\par 
    The second integral we rewrite as 
    $$\int_0^1u_ns\dd x=\sum_{k=1}^{r}\al_k\int_0^1u_n\ind_{A_k}\dd x=\sum_{k=1}^{r}\al_k\int_{A_k}u_n\dd x$$
    and we treat the integrals over $A_k$ as follows:
    \begin{align*}
    \int_{A_k}u_n\dd x&=\int_{a_{k-1}}^{a_k}u_n\dd x\\
    &=\int_{a_{k-1}}^{a_k}\ov{w}(nx)\dd x\\  
    \left(\substack{y=nx\ \To\ \dd y=n\dd x\\ y=na_{k-1},y=na_k}\right)&=\int_{na_{k-1}}^{na_k}\ov{w}(y)(\dd y/n).  
    \end{align*}
    This last integral we chop into pieces:
    \begin{align*}
      &\int_{na_{k-1}}^{na_k}\ov{w}(y)(\dd y/n)\\
      =&\frac1n\left(\int_{na_{k-1}}^{\floor{na_{k-1}}+1}\ov{w}(y)\dd y+\int_{\floor{na_{k-1}}+1}^{\floor{na_k}}\ov{w}(y)\dd y+\int_{\floor{na_k}}^{na_k}\ov{w}(y)\dd y\right)  
    \end{align*}
    and observe that the quantities
    $$\bonj{(\floor{na_{k-1}}+1)-na_{k-1}}\word{and}(na_k-\floor{na_k})$$
    are both between $0$ and $1$. Also the second integral is between two integer values. We may apply periodicity of $\ov{w}$ to conclude that 
    $$\left|\int_0^1u_ns\dd x\right|\leq \frac{2}{n}\int_{0}^{1}|\ov w|\dd y+\frac{\floor{na_k}-\floor{na_{k-1}}+1}{n}\left|\int_{0}^{1}\ov w\dd y\right|.$$
    The first integral is $(2/n)\norm{\ov w}_{L^1}$ and the second one is zero because we have assumed that $\bt=-\al$. In total, we have 
    $$\left|\int_0^1u_nv\dd x\right|\leq \eps\norm{w}_{L^p}+\frac2n\norm{\ov w}_{L^1}$$
    and letting $n$ grow we get the desired result.
    \item Observe that the previous argument also applies for weak * convergence as we may take $v\in L^1$.
  \end{enumerate}
\end{ptcbr}
\end{document} 
