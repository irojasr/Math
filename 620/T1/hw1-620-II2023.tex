\documentclass[12pt]{memoir}

\def\nsemestre {II}
\def\nterm {Fall}
\def\nyear {2023}
\def\nprofesor {Wolfgang Bangerth}
\def\nsigla {MATH620}
\def\nsiglahead {Variational Methods and Optimization I}
\def\nextra {HW1}
\def\nlang {ENG}
\input{../../headerVarillyDiff}

\begin{document}

\begin{Ej}
    In class, we proved that any continuous function $f\: D\subseteq \bR^n\to\bR$ has (at least one) local minimum in $D$ if $D$ is compact. We also convinced ourselves that all three conditions - boundedness and closedness of $D$ (which together constitute compactness in finite
dimensional spaces) and continuity of $f$ - were in fact necessary to guarantee the existence of a minimum.
\begin{enumerate}[i)]
    \itemsep=-0.4em
    \item Show one example each of domains $D$ and functions $f$, for each of the three conditions that violate that one condition and that do not have a minimum. In other words, show that omitting any of the conditions does not result in a situation where existence of a minimum is guaranteed.
    \item In truth, the statement above is not quite optimal. Continuity of the function is not actually necessary, even though it is easy to find discontinuous functions that do not have a minimum on a compact set
    $D$. Indeed, it is not difficult to find discontinuous functions that do have a minimum on a compact set $D$. Give a one and a two-dimensional example.
    \item The resolution to this conundrum is that obviously the set of continuous functions is too small, and the
    set of potentially discontinuous functions too large.\par
    We need to seek another set of function that lies
    between. This set is the class of \emph{lower semicontinuous functions}. A function $f\: D\subseteq \bR^n\to\bR$ is called
    lower semicontinuous at $x\in D$ if $f(x)\leq \lim_{k\to\infty}f(x_k)$ for all sequences $x_k\to x$; more generally, $f$
    is called lower semicontinuous if it is lower semicontinuous at all $x\in D$. \aside{Obviously, if the statement
    holds with equality, then the function is continuous; furthermore, a function that is both lower and
    upper semicontinuous is of course also continuous.}
    Repeat the proof of the existence of a minimum for functions that only satisfy this weaker condition.
    Point out, in particular, where the proof deviates or is different from the one we have seen in class. %Zygmund 4.7
\end{enumerate}
\end{Ej}

\begin{ptcbr}
\begin{enumerate}[i)]
    \itemsep=-0.4em
    \item Consider the identity function on $\bR$. The real line is closed, the identity function is continuous, however the set is unbounded.\par 
    If we suppose by contradiction that the identity function has a minimum, then for some $x_0\in\bR$ 
    $$x_0\leq x,\quad x\in\bR.$$
    This is impossible as $x_0-1\leq x_0$ and $x_0-1$ is in the range of the identity function. Therefore the identity function has no minimum.\par
    Consider now the identity function on the set $\obonj{0,1}$. This is once again a continuous function on a bounded set which is not closed. The unit interval maps to itself via the identity function, so finding a minimum value for the function equates to finding a minimum value for the set $\obonj{0,1}$.\par 
    Suppose by contradiction that $x_0$ is such a minimal element, as $\obonj{0,1}$ is open, there exists an $r>0$ such that $\obonj{x_0-r,x_0+r}\subseteq\obonj{0,1}$. Take the element $x_0=\frac{r}{2}$, this element is smaller than $x_0$ and still in the image of the identity function. It follows that our minimum is actually not a minimum, so our assumption must've been wrong to begin with. It follows that our function achieves no minimum.\par
    Finally consider the function $\frac{1}{2x-1}$ on $[0,1]$. This function has a simple pole at $x=1/2$ so it is a discontinuous function on a closed and bounded set. The image of $[0,1]$ under our function is $\rbonj{-\infty,-1}\cup\lbonj{1,\infty}$ and this set has no minimum element by a similar consideration to previous arguments.
    \item Let us now consider the piecewise function 
    $$\begin{cases}
        \half-x,\quad 0\leq x\leq\half\\
        \frac32-x,\quad \half<x\leq1
    \end{cases}$$
    defined on $\bonj{0,1}$. Observe that the first derivative test doesn't yield information, so evaluating at the endpoints of each subinterval we see that $f(\half)=0$ is the minimum value.\par 
    In $\bR^2$ consider the indicator function $f(x,y)=\ind_{(\bQ\cap\bonj{0,1})^2}(x,y)$. However let us redefine $f$ at $(0,0)$ as $-1$. Then $f$ is discontinuous everywhere on $[0,1]^2$ but has a minimum value at the origin and it's $z=-1$.
    \item Let us assume that $f$ is lower semicontinuous and call $\ga=\inf_D f$. There are two possibilities, either $\ga$ is infinite or its not. No finite minimum exists when $f$ is not bounded so let us assume that $\ga$ is finite.\par 
    By definition of $\inf$, there exists $x_k\in D$ such that 
    $$f(x_k)<\ga+\frac1k.$$
    The collection $(x_k)\subseteq D$ forms a sequence inside a compact set, given this we can find $(x_{k_\l})_{\l\in\bN}$ such that $x_{k_\l}\xrightarrow[\l\to\infty]{}x\in D$. This means that 
    $$f(x_{k_\l})<\ga+\frac{1}{k_\l}\To f(x)=f(\lim x_{k_\l})\leq \lim f(x_{k_l})\leq \ga+\lim\frac{1}{k_\l}=\ga.$$
    This means that $f(x)\leq \ga$, but since $\ga$ is the $\inf$, it must happen that $f$ reaches $\ga$ at $x$.\par 
    The difference between this proof and the one in the class is that for continuous functions, the limit interchanges with the function. In this case we only have the inequality between $f(\lim x_{k_\l})$ and $\lim f(x_{k_\l})$. Also, in the class we have assumed that there's a sequence $(x_k)$ such that $f(x_k)$ converges to $\ga$, but here the sequence is constructed, albeit not explicitly. 
\end{enumerate}
\end{ptcbr}

\begin{Ej}[Compactness]
    Do the following:
    \begin{enumerate}[i)]
        \itemsep=-0.4em
        \item We have sketched in class how one shows that a bounded and closed set in a finite dimensional space
        $\bR^n$ is compact. \aside{Here, let us use the ``sequential compactness'' we defined in class, rather than the
        topological one mentioned as an aside.} Work out the proof of this statement in detail and rigor. You
        will, in particular, need to work out the volume of the sets we consider in each step of the iteration, and
        how that affects the possible distance of any two points in it; then use this maximal possible distance
        rigorously to establish convergence. The key step in the proof is to show that if you make the volume
        smaller by bisecting the volume, the maximal distance must also decrease \aside{perhaps not in each step
        individually, but after a fixed number of go-arounds}.
        \item Show in detail and rigor why this proof does not work in infinite dimensional spaces.
        \item One could think of other ways of proving the statement, but fundamentally they fail because of a
        slightly surprising fact: \emph{The volume of a ball of radius 1 goes to zero as the dimension goes to infinity.}\par
        In other words, ensuring that a sequence is entirely enclosed in a sequence of smaller and smaller
        volumes does not guarantee that it actually converges because that no longer implies that points are
        closer and closer to each other in large space dimensions.\par
        Confirm that the fact above is indeed true. You could look up the volume of the unit ball in $n$ space
dimension, but showing some kind of proof would be better :-)
    \end{enumerate}
\end{Ej}

\begin{ptcbr}
    \begin{enumerate}[i)]
        \item Let $F\subseteq\bR^n$ be our closed and bounded set, we wish to see that every sequence in $F$ has a convergent subsequence.\par 
        To that effect, let $(x_n)_{n\in\bN}\subseteq F$ be a sequence in $F$ and consider the $1^{\text{st}}$ coordinate of each $x_n$. Observe that $(x_n)$ is a bounded sequence so it holds that 
        $$|x^1_n|\leq\norm{x_n}\leq M\word{for some}M>0.$$
        This means that the real sequence $(x^1_n)_{n\in\bN}$ is also bounded and we may use the Bolzano-Weierstrass theorem in one dimension to find a subsequence $(x^1_{n_k})$ such that $x^1_{n_k}\xrightarrow[k\to\infty]{}x^1$. We now consider the $2^{\text{nd}}$
        \item \red{FINISH}
        \item To prove that the volume of the unit ball goes to zero as dimension grows we will calculate it using a Gaussian integral. First observe that 
        $$\int_{\bR^n}e^{-\half\norm{x}^2}\dd x=(2\pi)^{n/2}$$
        because we may separate the integral into a product, each equal to $\sqrt{2\pi}$. The function in question is radial, so we exploit that symmetry by calculating the last integral through the first dimension and then through an $n-1$ sphere. 
        $$\int_{\bR^n}e^{-\half\norm{x}^2}\dd x=\int_0^\infty\int_{\del B(0,r)}e^{-\half r^2}\dd\sg\dd r$$
        where $\sg$ is the Lebesgue measure on the surface of the sphere. Observe two facts, the function in question is independent of the inner integral so we may take it out and leave only the measure of $\del B(0,r)$. We can compare this to the area of the unit ball's boundary via the formula 
        $$\sg(\del B(0,r))=r^{n-1}\sg(\del B(0,1))$$
        which converts the integral to 
        $$\sg(\del B(0,1))\int_0^\infty e^{-\half r^2}r^{n-1}\dd r$$
    \end{enumerate}
\end{ptcbr}

\begin{Ej}[Compactness in finite and infinite dimensional spaces]
    The spaces $\l_2$ and $\l_2$ are examples of spaces where closed and bounded sets are not sequentially compact.\par 
    Consider $L_\infty[0,1]$ and the set of bounded linear operators over $L_\infty$: 
    $$\cA=\set{A\: L_\infty\to L_\infty\:\ A\ \text{is linear},\ \norm{A}_\cA<\infty}$$
    where the operator norm is defined as 
    $$\norm{A}_{\cA}=\sup_{\norm{f}_{\infty}=1}\norm{Af}_{\infty}$$
    Consider $D=\set{A\in\cA\:\ \norm{A}_\cA\leq 1}$ and construct a sequence of operators $(A_n)\subseteq D$ for which no subsequence converges.
\end{Ej}
\end{document} 
