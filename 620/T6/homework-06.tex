\documentclass{article}
\usepackage{comment}
\excludecomment{answers}
\include{macros}

\begin{document}

\homework{6}

\problem{A small variation for the Dirichlet problem}{40}{
In class, we have gone through the details of a proof for guaranteeing
that a minimizer exists for the functional
\begin{align*}
  I(u) = \int_\Omega \frac 12 |\nabla u|^2
\end{align*}
over the (affine) space
\begin{align*}
  X_g = \left\{ u \in W^{1,2}(\Omega) : u|_{\partial\Omega} = g \right\}.
\end{align*}
Among the other consequences of the theorem were that the (unique)
minimizer $\bar u$ had to satisfy the weak Euler-Lagrange equation
\begin{align*}
  \int_\Omega \nabla \bar u \cdot \nabla \varphi = 0
  \qquad
  \forall \varphi\in X_0,
\end{align*}
where $X_0$ is the tangent space to $X_g$ (i.e., consists of functions
with zero boundary values), and that if $\bar u$ happens to be smooth
enough, that it then has to satisfy the partial differential equation
\begin{align*}
  -\Delta \bar u &= 0 && \text{in $\Omega$},
  \\
  \bar u &= g \qquad && \text{on $\partial\Omega$},
\end{align*}
i.e., it has to solve the Laplace equation.

Repeat some of the steps of the proof for the following variation:
\begin{align*}
  I(u) &= \int_\Omega \frac 12 |\nabla u|^2 - hu,
\end{align*}
where $h\in L^2(\Omega)$ is a given function. For simplicity take
$X_0=W^{1,2}_0$ as the set to find a minimum over, i.e., $g=0$.

In particular, do the
following:
\begin{itemize}
  \item Repeat the first step of showing that a minimizer exists.
    Namely, we needed to show that for a minimizing sequence $\{u_n\}\subset X_g$ so that
    $I(u_n)\rightarrow m=\inf_{u\in X_g} I(u)$, there exists an
    $N$ and $\gamma<\infty$ so that for all $n\ge N$, we have that
    $\|u_n\|_{W^{1,2}} \le \gamma$.
    \begin{align*}
      \|u\|_{W^{1,2}} \le \gamma.
    \end{align*}
    The key to this was to show that
    \begin{align*}
      \|u\|_{W^{1,2}}^2 \le c_1 I(u) + c_2.
    \end{align*}
    If this is true, then we know -- because $u_n$ is a \textit{minimizing
    sequence} -- that there are $N<\infty, |a|<\infty, b<\infty$ so that
    \begin{align*}
      I(u_n) \le am + b
    \end{align*}
    for all sufficiently large $n\ge N$. As a consequence, we know
    that after that point in the sequence, $\|u\|_{W^{1,2}} \le
    \sqrt{c_1(am+b)+c_2}=\gamma$ and the weak compactness of the ball
    of radius $\gamma$ in $W^{1,2}$ then guarantees that there is a
    weakly convergent subsequence.

    Show a similar proof with the variation of the functional $I(u)$ above.

  \item Show the weak Euler-Lagrange equation a minimizer has to
    satisfy.

  \item Show the strong Euler-Lagrange equation a minimizer has to
    satisfy if it is regular (smooth) enough.
\end{itemize}
}


The remainder of the homework is concerned with finding
counter-examples for extensions of the general theorem we have
mentioned in class. It reads as follows:

\textit{\textbf{Theorem:} Let $\Omega\subset \R^n$ be a bounded open
  set with a Lipschitz boundary. Let $f\in C^0(\Omega\times\R\times
  \R^n)$, $f=f(x,u,\xi)$ be a function that satisfies the following
  conditions:
  \begin{enumerate}
    \item[(i)] $\xi \mapsto f(x,u,\xi)$ is convex for all $x\in
      \Omega,u\in \R$;
    \item[(ii)] there exist $p>q\ge 1$ and $\alpha_1>0$,
      $\alpha_2,\alpha_3\in\R$ (i.e., they must be finite) so that
      \begin{align*}
        f(x,u,\xi) \ge \alpha_1 |\xi|^p + \alpha_2|u|^q + \alpha_3
      \end{align*}
      for all $x\in \Omega,u\in \R,\xi\in\R^n$.
  \end{enumerate}
  Then the functional
  \begin{align*}
    I(u) = \int_\Omega f(x,u(x),\nabla u(x)) \, \text{d}x
  \end{align*}
  has a minimizer $\bar u$ in 
  \begin{align*}
    X_g = \left\{ u \in W^{1,p}(\Omega) : u|_{\partial\Omega} = g \right\},
  \end{align*}
  where $g$ is the restriction of some $\tilde g\in W^{1,p}(\Omega)$
  to $\partial \Omega$. (Or viewed differently, $g$ are prescribed boundary
  values that are nice enough so that we can find an
  \textit{extension} of $g$ called $\tilde g$ so that $\tilde g\in
  W^{1,p}(\Omega)$ and so that $\tilde g|_{\partial\Omega}=g$.)
}

\textit{
  If, furthermore, 
  \begin{enumerate}
    \item[(iii)] $f\in C^1$ and if there is a $\beta\ge 0$ so that
      \begin{align*}
        |f_u(x,u,\xi)| &\le  \beta(1+|u|^{p-1} + |\xi|^{p-1}),
        \\
        |f_\xi(x,u,\xi)| &\le  \beta(1+|u|^{p-1} + |\xi|^{p-1}),
      \end{align*}
      for all $x\in \Omega,u\in \R,\xi\in\R^n$,
  \end{enumerate}
  then $\bar u$ satisfies the weak Euler-Lagrange equations
  \begin{align*}
    \int_\Omega (f_u(x,\bar u(x),\nabla\bar u(x))\varphi +
    f_\xi(x,\bar u(x),\nabla\bar u(x))\cdot\nabla\varphi) \text{d}x = 0
  \end{align*}
  for all $\varphi\in X_0$.
}

The theorem as stated seems to have a lot of restrictions, but it
turns out that they all seem necessary since one can find
counter-examples without too much trouble. The following exercises are
therefore meant to probe the applicability of the theorem.

\problem{Application 1 of the general theorem}{20}{
Consider the function $f(x,u,\xi)=\frac{1}{4}|\xi|^4+g(x,u)$ where
$g\in C^{0,1}(\Omega\times\R)$. Show that the theorem applies.
}

\problem{Application 2 of the general theorem}{20}{
Consider the function $f(x,u,\xi)=\frac12 |\xi|^2-\frac12\lambda^2u^2$ where
$\lambda$ is large -- say, larger than the constant in the Poincar\'e
inequality for functions in $W^{1,2}_0(\Omega)$. Show that the
theorem does not apply by checking each condition individually. Then try to construct a sequence $u_n$ so that
$I(u_n)\rightarrow -\infty$, i.e., show that $I(u)$ is not bounded
from below on $X_0=W^{1,2}_0$. For this part of the example, choose
$\Omega=(0,1)$ and $\lambda>\pi$.
}


\problem{Application 3 of the general theorem}{20}{
Consider the function $f(x,u,\xi)=(|\xi|^2-1)^2$ on
$\Omega=(0,1)\subset \R$ and with $X_g=W^{1,4}_0(0,1)$. Show that the
theorem does not apply by checking each condition individually.

Derive the weak and strong Euler-Lagrange equations for this
case. Show that $u=0$ satisfies both of these equations; then show
that it is not a minimizer of $I(u)$, for example by finding another
function $v\in X_g$ so that $I(v)<I(u)$.
}


\end{document}
