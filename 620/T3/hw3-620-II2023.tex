\documentclass[12pt]{memoir}

\def\nsemestre {II}
\def\nterm {Fall}
\def\nyear {2023}
\def\nprofesor {Wolfgang Bangerth}
\def\nsigla {MATH620}
\def\nsiglahead {Variational Methods and Optimization I}
\def\nextra {HW2}
\def\nlang {ENG}
\let\footruleskip\relax %%FADIR
\input{../../headerVarillyDiff}

\begin{document}
\begin{Ej}
    We have proven rigorously that any succinctly smooth minimizer (say, if the minimizer happens to be in $C^2$) of the functional
    $$I\bonj{u}=\int\limits_a^b f(x,u(x),u'(x))\dd x $$
    has to satisfy the Euler-Lagrange equations
    $$f_u(x,u(x),u'(x))=\dv{x}f_\xi(x,u(x),u'(x))$$
    for all $x\in\obonj{a,b}$. The energy related to the vertical defection of a thin (one-dimensional) beam subject to a vertical gravity geld with strength $g$ is given by
    $$I\bonj{u}=\int\limits_a^b \bonj{\mu\left(\sqrt{1+u'(x)^2}-1\right)-gu(x)}\dd x $$
    where $\mu$ is related to the elasticity constants of the material.
    \begin{enumerate}[i)]
        \item Derive the Euler-Lagrange equations (including boundary conditions) for $u(x)$ for this problem if you
        try to minimize the energy over the set of functions
        $$D=\set{\vf\in C^2\:\ \vf(a)=\vf(b)=1}.$$
        \item Can you prove that the solution $\ov u$ of the (Euler-Lagrange) deferential equations is (i) a minimizer of
        $I\bonj{u}$, and (ii) is unique?
    \end{enumerate}
\end{Ej}

\begin{ptcbr}
    \begin{enumerate}[i)]
        \item Our function in question is 
        $$f(x,u,\xi)=\mu\left(\sqrt{1+\xi^2}-1\right)-gu$$
        so differentiating with respect to $u$ and $\xi$ we get 
        $$
        \begin{cases}
            f_u(x,u,\xi)=-g\\
            f_\xi(x,u,\xi)=\mu\frac{\xi}{\sqrt{1+\xi^2}}
        \end{cases}
        $$
        Now taking the derivative with respect to $x$ of $f_{\xi}(x,u(x),u'(x))$ we get 
        \begin{align*}
            \dv{x}f_{\xi}(x,u(x),u'(x))&=\mu\dv{x}\frac{u'(x)}{\sqrt{1+u'(x)^2}}\\
            &=\mu\frac{u''(x)\sqrt{1+u'(x)^2}-u'(x)\frac{u'(x)}{\sqrt{1+u'(x)^2}}}{1+u'(x)^2}\\
            &=\mu\frac{u''(x)(1+u'(x)^2)-u'(x)^2}{(1+u'(x)^2)^{3/2}}
        \end{align*}
    So the Euler-Lagrange equations in this case are given by
    $$\mu\frac{u''(x)(1+u'(x)^2)-u'(x)^2}{(1+u'(x)^2)^{3/2}}=-g$$
    and the boundary conditions will be satisfied when $u\in D$. This is $u(a)=u(b)=1$. \red{IS THIS THE CORRECT INTERPRETATION FOR BDRY CONDS?}
    \item \red{LOOK FOR THEOREM}
    \end{enumerate}
\end{ptcbr}

\begin{Ej}
    In general, the Euler-Lagrange equations will be a (system
    of) nonlinear ordinary deferential equation. Most often, they will not be exactly solvable. But occasionally,
    we can solve simplices problems.\par
    If you take Problem 1, consider the case of stir materials that do not deform very much. In that case, $u'$
    will be small, and we can use the approximation
    $$\sqrt{1+y}\approx 1+\half y.$$
    (This is just Taylor expansion around $y = 0$.) Use this to dene an approximate energy functional $\ov I\bonj{u}$.
    \begin{enumerate}[i)]
        \item Derive the Euler-Lagrange equations (including boundary conditions) for $u(x)$ for this approximate
        problem with the same $D$ as before.
        \item Can you prove that the solution $\ov u$ of the (Euler-Lagrange) deferential equations is the unique minimizer
        of $\ov I\bonj{u}$? (In other words, that the solution of the Euler-Lagrange equation is not just a stationary
        point of $\ov I\bonj{u}$, but in fact a minimizer?)
        \item Actually solve this problem, i.e., find $\ov u$ that satires the Euler-Lagrange equations.
    \end{enumerate}
\end{Ej}


\begin{ptcbr}
    \begin{enumerate}[i)]
        \item Observe that our new functional is 
        $$\ov I\bonj{u}=\int\limits_a^b \bonj{\mu\left(1+\half u'(x)^2-1\right)-gu(x)}\dd x=\int\limits_a^b \left(\frac{\mu}{2}u'(x)^2-gu(x)\right)\dd x$$
        so the new function for us is 
        $$f(x,u,\xi)=\frac{\mu}{2}\xi^2-gu\To f_u=-g,\quad f_\xi=\mu\xi.$$
        Now we can see that
        $$\dv{x}f_{\xi}(x,u(x),u'(x))=\dv{x}\mu u'(x)=\mu u''(x).$$
        The corresponding equation is now $\mu u''(x)=-g$ and the conditions remain the same, $u(a)=u(b)=1$
        \item \red{ASK FOR THM}
        \item The equation in question can be solved as follows:
        $$u''(x)=\frac{-g}{\mu}\To u'(x)=\frac{-g}{\la}x+c_1\To u(x)=\frac{-g}{2\la}x^2+c_1x+c_2.$$
        Applying the initial conditions we get a system of the form 
        $$
        \left\lbrace
        \begin{aligned}
            1=\frac{-g}{2\la}a^2+c_1a+c_2\\
            1=\frac{-g}{2\la}b^2+c_1b+c_2
        \end{aligned}
        \right.\To
        \twobythree{a}{1}{1+ga^2/2\la}{b}{1}{1+gb^2/2\la}
        $$
    \end{enumerate}
\end{ptcbr}

\begin{Ej}
    We have started the semester by considering Newton's
    minimal resistance problem and Bernoulli's brachystochrone problem (the ``bead on a wire''). The former is
    a bit more complicated because the right integration bound $ b$ depends on the solution. But for the former,
    the case is easy: We have
    $$I\bonj{u}=\int\limits_0^L\sqrt{\frac{\half(1+u'(x)^2)}{gH-gu(x)}}\dd x$$
    and
    $$D=\set{\vf\in C^2\:\ \vf(0)=H,\ \vf(L)=0}.$$
    State the Euler-Lagrange equations and boundary conditions any scent smooth minimizer $\ov u$ would
have to satisfy. Is the solution unique?
\end{Ej}

\begin{ptcbr}
Observe that in this case, the function in question is 
$$f(x,u,\xi)=\sqrt{\frac{\half(1+\xi^2)}{gH-gu}}=\sqrt{\frac{1}{2g}}\sqrt{\frac{1+\xi^2}{H-u}}.$$
Call $\la=\sqrt{\frac{1}{2g}}$ so when differentiating we obtain 
$$\left\lbrace
\begin{aligned}
    &f_u(x,u,\xi)=\la\sqrt{1+\xi^2}\frac{-1}{(\sqrt{H-u})^2}\frac{1}{2\sqrt{H-u}}(-1)=\frac{\la}{2}\sqrt{\frac{1+\xi^2}{(H-u)^3}}\\
    &f_\xi(x,u,\xi)=\frac{\la}{\sqrt{H-u}}\frac{\xi}{\sqrt{1+\xi^2}}
\end{aligned}
\right.
$$
We now must differentiate $f_\xi$ with respect to $x$. This is 
\begin{align*}
    \dv{x}f_{\xi}(x,u(x),u'(x))&=\dv{x}\frac{\la}{\sqrt{H-u(x)}}\frac{u'(x)}{\sqrt{1+u'(x)^2}}.
\end{align*}
Using the product rule and the fact that we have already calculated the derivative of a term similar to the second one in the previous exercise, we obtain:
$$\frac{\la u'(x)}{2\sqrt{(H-u(x))^3}}\frac{u'(x)}{\sqrt{1+u'(x)^2}}+\frac{\la}{\sqrt{H-u(x)}}\frac{u''(x)(1+u'(x)^2)-u'(x)^2}{\sqrt{(1+u'(x)^2)^3}}.$$
We may factor out some terms to clean up the expression as follows:
$$\frac{\la}{\sqrt{(H-u(x))(1+u'(x)^2)}}\left(\frac{u'(x)^2}{2(H-u(x))}+\frac{u''(x)(1+u'(x)^2)-u'(x)^2}{1+u'(x)^2}\right).$$
We now homogenize the terms inside the parenthesis. The numerator is an expression of the form 
\begin{gather*}
u'(x)^2(1+u'(x)^2)+2(H-u(x))\bonj{u''(x)(1+u'(x)^2)-u'(x)^2}\\
=u'(x)^2+u'(x)^4+2(Hu''(x)+Hu''(x)u'(x)^2-Hu'(x)^2\\
-u''(x)u(x)-u''(x)u'(x)^2u(x)+u'(x)^2u(x))
\end{gather*}
\end{ptcbr}

\begin{nonum-Ej}
    If you can, also solve the Euler-Lagrange equations. The solution can of course be found on the internet
or in any number of books, but if you want to get these bonus points, you will need to show step by step
how you solve the equations - this is going to be non-trivial.
\end{nonum-Ej}

\begin{Ej}
    Everything we have done in class was based on functions $u(x)$ of a
single argument $x\in\bonj{a,b}\subseteq\bR$. But in reality, it is not very dict to derive the same kind of Euler-Lagrange
equations also for functions of multiple arguments $u(\vec x),\ \vec x\in\Om\subseteq\bR^n$.
To this end, let us assume that we want to find a minimizer of
$$I\bonj{u}=\int\limits_\Om f(\vec x,u(\vec x),\nb u(\vec x))\dd x$$
and
$$D=\set{\vf\in C^2(\Om)\:\ \vf(\vec x)=g(\vec x),\ \vec x\in\del\Om}.$$
Go through the one-dimensional derivation of the Euler-Lagrange equations and adapt it as appropriate to
derive the (now partial) deferential equation any succinctly smooth minimizer $\ov u\in D$ has to satisfy. The
key step is to remember your integration-by-parts rules and use what you know about the boundary values
of the variations. You will also have to keep in mind that you now really have $f(x, u,\xi_1,\dots,\xi_n)$.
State both the Euler-Lagrange equations and boundary conditions.
\end{Ej}

\begin{Ej}
    The generalization to higher dimensions of the erst problem is to
look for the minimizer of the functional
$$I\bonj{u}=\int\limits_\Om\bonj{\mu\left(\sqrt{1+|\nb u(\vec x)|^2}-1\right)-gu(\vec x)}\dd x,$$
state the Euler-Lagrange equations for this problem.
Next, apply the same simplification we considered in Problem 2 and again derive the corresponding
Euler-Lagrange equations.
\end{Ej}
\end{document} 
