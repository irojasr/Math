\documentclass[12pt]{memoir}

\def\nsemestre {II}
\def\nterm {Fall}
\def\nyear {2023}
\def\nprofesor {Wolfgang Bangerth}
\def\nsigla {MATH620}
\def\nsiglahead {Variational Methods and Optimization I}
\def\nextra {HW2}
\def\nlang {ENG}
\let\footruleskip\relax %%FADIR
\input{../../headerVarillyDiff}

\begin{document}
\begin{Ej}
    Consider the space $X=C^k([0,1])$ of $k$ times continuously differentiable functions. We often impose boundary conditions so we take the subset 
    $$D=\Set{\vf\in X\:\ \vf(0)=2,\pdv{x}\vf(1)=3}\subseteq X$$
    of functions with a prescribed function value on the left, and prescribed derivative on the right end of the
    interval. (This clearly only makes sense if $k\geq 1$.)
    \begin{enumerate}[i)]
        \item Show that this is an affine subspace of $X$.
        \item Show that $D$ is convex.
        \item State what the tangent space $D'$ of $D$ is. 
    \end{enumerate}
\end{Ej}

\begin{ptcbr}
    To show that $D$ is affine, we must show it is of the form $v+D_0$ where $D_0$ is indeed a subspace of $X$. To that effect, consider the subspace 
    $$D_0=\Set{\vf\in X\:\ \vf(0)=0,\pdv{x}\vf(1)=0}.$$
    For $\vf,\psi\in D_0$ we have
    $$c\vf(0)+\psi(0)=c\.0+0=0\word{and}\pdv{x}(c\vf(x)+\psi(x))\!\mid_{x=1}=c\vf_x(1)+\psi_x(1)=0,$$
    so it is indeed a subspace. Now if we consider $v=3x+2$, then when exploring the conditions imposed on our functions we have that for $\vf\in D_0$
    $$3(0)+2+\vf(0)=2\word{and}\pdv{x}(3x+2+\vf(x))\!\mid_{x=1}=3+0=3.$$
    So it holds that $D=(3x+2)+D_0$ and therefore, it is an affine subspace.\par 
    Consider now two functions $\vf,\psi\in D$, we are to show that any convex combination of them is in $D$. To that effect note that for any $t\in[0,1]$ we have 
    $$t\psi(0)+(1-t)vf(0)=2t+2(1-t)=2t+2-2t=2.$$
    When differentiating we get 
    $$\pdv{x}\left(t\psi(x)+(1-t)\vf(x)\right)\!\mid_{x=1}=t\psi_x(1)+(1-t)\vf_x(1)=3t+3-3t=3.$$
    In conclusion, convex combinations of functions in our space are still in our space. It follows that $D$ is convex.\par 
    Finally, reacall that the tangent space of $D$ at a point is 
    $$D'(x_0)=\set{v\in X\:\ v+x_0\in D}$$
    and as $D$ is a linear affine subspace, $D'$ is independent of the selection of $x_0$. Regardless, we claim that $D'=D_0$. Observe that if we add any element of $D_0$ to an element of $D$ we don't switch the boundary conditions. This means that $D_0\subseteq D'$, and if we take such a $v\in D'$, then there is a $\vf\in D$ such that $v+x_0=\vf$. This means that $v=\vf-x_0$ so when considering the boundary conditions we see that $v$'s boundary conditions are precisely those of $D_0$. It follows that the tangent space is precisely $D_0$. \emph{This makes sense, as $D$ is a linear space, the tangent space must be its slopes. But that's itself. However, $D'$ must be a subspace, so that's why it's $D_0$.} 
\end{ptcbr}

\begin{Ej}
    
    Take the function 
    $$f(x,y)=\min\set{|x|,|y|}\sgn(x).$$
    In class, we talked about the fact that the directional (Gateaux) derivative satisfies
    $$Df((x,y);v)=\braket{\nb f(x,y)}{v}.$$
    For the current function, at the origin, we have $\nb f(0)=0$\footnote{That is because the gradient is defined as $\nb f=(\del_x f,\del_y f)^{\sT}$ and because the function $f$ is constant (equal to zero) along both the $x$ and $y$ axes}, but the directional derivative is not zero for all $v$.
    \begin{enumerate}[i)]
        \item Let's build intuition. Plot this function. Then use this visualization to convince yourself (and the
        reader of your answer) graphically that the statement about $\nb f(0)=0$ is true, as well as that the
        statement that the directional derivative is not zero in other directions is also true.
        \item Explain the discrepancy. What does this imply for the viability of the idea that we can look for points
        with $\nb f=0$ when searching for minima of functions?
    \end{enumerate}
   
\end{Ej}

\begin{figure}[h!]
    \centering
    \includegraphics[width=1\textwidth, trim= 0.3cm 19.5cm 0.6cm 0.1cm,clip]{fig1.pdf}
\end{figure}

\begin{ptcbr}
    Imagine first the function $z=|x|$. When viewing the slice at $y=0$ we get a $\bigvee$ shaped curve. So as there's no dependence on the $y$ variable, the shape of the surface $z=|x|$ is a $\bigvee$ shaped trough follwing the $y$ axis. The surface $z=|y|$ has the same kind of story but rotated $90^\circ$. If we reduce the opacity of one of the troughs and increase it for the other, we can see that the section of the dim trough is covered above by the more visible one. This can't happen if we are taking the minimum, so we must eliminate that part.\par 
    If we cut the shape by a plane $z=c$, then the resulting surface is an $\bigtimes$ shaped trough. The final detail is considering what happens we multiply by $\sgn(x)$. On the right half of plane we still have our trough cut in half so it is $\vdash$ shaped now. However on the left, we take the same mirror image of our trough but flip it $180^\circ$ with respect to the $yz$ plane\footnote{I just realized I messed up in my drawing, I flipped it with respect to the $xz$ plane and not the $yz$ plane. I hope you will excuse that mistake. If it helps, imagine rotating the figure in question along the $xy$ plane $90^\circ$ counter-clockwise and exchanging the red blue coloring. The function I have drawn by mistake is instead $\sgn(y)\min(|x|,|y|)$}.\par 
    Now, observe that at the origin the gradient is zero as it is only contemplating the $x$ and $y$ directions. However we can see that when taking the limit definition of the Gateaux derivative at zero we get
    $$\lim_{\eps\to 0}\frac{f(0+\eps v)-f(0)}{\eps}=\lim_{\eps\to0}\frac{f(\eps v)}{\eps}.$$
    For directions like $v=(1,1)$ and $(-1,1)$ we get 
    $$\eps^{-1}f(\eps,\eps)=1\word{and}\eps^{-1}f(-\eps,\eps)=-1$$
    which means that we don't have a Gateaux derivative in every direction.\par 
    The issue is that our function is not continuously differentiable at every point. In order to find minima using the necessary conditions, we will have to work around this either by asking the whole function to be $C^1$ or search inside the region where our function is $C^1$ to find global minima. 
\end{ptcbr}

\begin{Ej}
    Take the following variation of the function of the previous
problem:
$$f(x,y)=\min(|x|,|y|).$$
Visualize this function. Then show rigorously that this function still has $\nb f(0)=0$, but that it does not
have a Gateaux derivative $Df(x;v)$ for all $v$.
\end{Ej}

\begin{ptcbr}
    We have already seen this function in the previous exercise. In order to find the derivative at zero, we take the partial derivatives by definition:
    $$\lim_{h\to 0}\frac{f(0+h,0)-f(0,0)}{h}=\lim_{h\to 0}\frac{\min(|h|,0)}{h}=0$$
    and similarly $k^{-1}f(0,k)$ tends to $0$ so it holds that $\nb f(0)=0$.\par 
    With a similar argument as before, we can see that there's no Gateaux derivative in the direction $(1,1)$:
    $$\lim_{\eps\to 0}\frac{f(0+\eps(1,1))}{\eps}=\lim_{\eps\to0}\frac{|\eps|}{\eps}=\text{two-sided limit}.$$
    As the limit doesn't exist, there's no Gateaux derivative in the direction of $(1,1)$ so $f$ isn't Gateaux differentiable in \emph{all} directions.
\end{ptcbr}
\end{document} 
