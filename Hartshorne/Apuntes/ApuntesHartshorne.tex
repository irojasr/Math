\documentclass[12pt]{memoir}

\def\nsemestre {I}
\def\nterm {Primer Semestre}
\def\nyear {2022}
\def\nprofesor {Javier Carvajal y Roberto Ulloa}
\def\nsigla {SP1313}
\def\nsiglahead {Geometr\'ia Algebraica}
\def\darktheme{}
\input{../../headerVarillyDiff}

\begin{document}
%\begin{multicols}{2}

\chapter{Variedades}

\section{Variedades Afines}

\begin{Def}\label{def:espacio-afin}
El \term{espacio afín} $\bA^n(k)$ sobre $k$, un cuerpo algebraicamente cerrado es el conjunto de todas las $n$-tuplas de elementos de $k$.
\end{Def}

Los elementos del espacio afín son puntos $x$ que se representan con coordenadas $\vec x=(x_1,\dots,x_n)$. No debemos confundir el punto con sus coordenadas.

\begin{Def}\label{def:conj-algebraico}
   Un subconjunto $Y\subseteq\bA^n(k)$ se dice ser un \term{conjunto algebraico} si existe un subconjunto $T\subseteq k[\vec x]$ tal que 
   $$Y=V(T)\defeq\set{\vec a\in\bA^n(k):\ \forall p\in T(p(\vec a)=0)}.$$
\end{Def}

\begin{Prop}\label{prop:propiedades-variedades}
  Supongamos que $M,N\subseteq\bA^n(k)$, vale lo siguiente:
  \begin{enumerate}
    \item Si $M\subseteq N$, entonces $V(N)\subseteq V(M)$.
    \item Si $\lie i =\gen(M)$, entonces $V(\lie i)=V(M)$.
    \item La unión de finita de variedades es una variedad. Más específicamente 
    $$V(M)\cup V(N)=V(M\cap N)=V(M\.N).$$
    \item La intersección de variedades es variedad. Específicamente vale:
    $$\bigcap_{i\in I}V(M_i)=V\left(\bigcup_{i\in I}M_i\right).$$
    \item El vacío y el espacio afín son variedades.
  \end{enumerate}
\end{Prop}

\begin{ptcbp}
\begin{enumerate}
  \item Si $\vec x\in V(N)$, entonces $\vec{x}$ anula a cualquier polinomio de $N$. En particular, anula a cualquier polinomio de $M$. Es decir $\vec{x}\in V(M)$.
  \item Como $M\subseteq\lie i$, entonces $V(\lie i)\subseteq V(M)$. Por otro lado, si $\vec{a}\in V(M)$ entonces $\vec{a}$ anula a cualquier polinomio de $M$. Como $M$ genera a $\lie i$, entonces todo elemento $g\in\lie i$ es de la forma
  $$g=\sum_{f\in M} hf\To g(\vec{a})=\sum_{f\in M}h(\vec a)f(\vec a)=0.$$
  Por lo tanto tenemos la otra inclusión.
  \item Como $M\.N\subseteq M\cap N\subseteq M,N$ entonces valen las inclusiones de izquierda a derecha.\par 
  Si $\vec a\in V(M\.N)$ pero $\vec a\not\in V(M)$ entonces para algún $f\in M$ vale que $f(\vec{a})\neq 0$. Sin embargo para $g\in N$ vale $(f\.g)(\vec{a})=0$. Como $g$ es arbitrario, entonces $\vec a\in V(N)$. Así concluimos la igualdad.
  \item Como $M_i\subseteq\bigcup_{i\in I}M_i$ para todo $i$, entonces vale la inclusión $\supseteq$.\par 
  Mientras tanto si $\vec{a}\in\bigcap_{i\in I}V(M_i)$, entonces
  $$\forall i\in I\left(\vec{a}\in V(M_i)\right)\To\forall i\in I\forall f\in M_i(f(\vec{a})=0).$$
  Es decir, $\vec{a}$ anula a cualquier polinomio dentro de cualquier $M_i$. Entonces $\vec{a}$ anula a cualquier polinomio en $\bigcup_{i\in I}M_i$. \yelo{Esta justificación no me gusta.}
  \item Finalmente vale que $V(\set{0})=\bA^n(k)$ y $V(k[\vec{x}])=\emptyset$.
\end{enumerate}
\end{ptcbp}

\begin{Def}\label{def:top-zariski}
  La \term{topología de Zariski} se define tomando como cerrados a las variedades. Los abiertos son los complementos de las variedades. De acuerdo con la proposición anterior, esta colección forma una topología.
\end{Def}

\begin{Ex}{Topología de Zariski en dimensión 1}
Recordemos que en $k[x]$ todo ideal es principal. Entonces vale para cualquier variedad:
$$V(M)=V(\gen(M))=V(p),$$
donde $p$ es el único generador del ideal en cuestión. Es decir, cualquier variedad es el conjunto de ceros de un único polinomio. A su vez
$$p(x)=c(x-a_1)\cdots(x-a_n),\ c,a_j\in k\To V(p)=\set{a_j}_{j=1}^n.$$
Así cualquier variedad en $\bA^1(k)$ es un conjunto finito.\par 
Esto nos dice que la topología de Zariski en dimensión 1 corresponde con la topología cofinita que de hecho no es de Hausdorff.
\end{Ex}

Para hablar de ejemplos en dimensión superior precisamos otros conceptos topológicos.

\begin{Def}\label{def:conj-irreducible}
  Un espacio topológico no vacío es \term{irreducible} si no puede ser expresado como la unión de dos subconjuntos cerrados propios. Un subconjunto es irreducible si visto con la topología inducida es irreducible.
\end{Def}

Viendo la definición extraemos lo siguiente:
\begin{itemize}
  \item El conjunto vacío \emph{no} es irreducible.
  \item Si $Y\subseteq X$ es no vacío, entonces es irreducible si no es posible escribir $Y=F_1\cup F_2$ con $F_1,F_2\subset Y$ cerrados en $Y$.
\end{itemize}

\begin{Ex}
  A manera de ejemplo, $\bA^1(k)$ es irreducible. Todo cerrado de $\bA^1(k)$ es un conjunto finito.\par 
  Si quisiéramos que $\bA^1(k)=F_1\cup F_2$, precisamos que alguno de estos conjuntos sea infinito. Pero esto no es posible, pues todos los cerrados son finitos.
\end{Ex}

A esta definición un poco nueva le agregamos algunas equivalencias que aún no conocíamos.

\begin{Prop}\label{prop:equivs-irreducible}
  Si $X$ es un espacio topológico no vacío entonces lo siguiente es equivalente:
  \begin{enumerate}
    \item $X$ es irreducible.
    \item Cualesquiera dos abiertos no vacíos de $X$ tienen intersección no vacía.
    \item Cualquier abierto no vacío de $X$ es denso, conexo e irreducible.
  \end{enumerate}
\end{Prop}

\begin{ptcbp}
  Las primeras dos condiciones son equivalentes pues
  $$X=F_1\cup F_2\iff \emptyset=G_1\cap G_2.$$
  Si no existen los cerrados $F_1,F_2$ que satisfacen la primera igualdad, entonces cualesquiera abiertos $G_1,G_2$ satisfacen la segunda igualdad.\par 
  Inmediatamente cualquier abierto es denso pues todos los abiertos se tocan. Para conexidad si dos abiertos tienen intersección vacía, su unión es un abierto disconexo. Esto prueba que:
  \begin{significant}
    Si todo abierto es conexo, entonces todos los abiertos se tocan.
  \end{significant}
  Por otro lado, si hay un abierto disconexo, se puede particionar en dos subconjuntos abiertos que no se tocan cuya unión es todo el conjunto. Esto prueba la otra dirección.\par 
  Finalmente, si todo abierto es irreducible, todo el espacio es irreducible. Tomemos ahora un abierto $G$, y $H\subseteq G$ abierto. Queremos ver que $H$ es denso en $G$, pero en efecto vale que $H$ es denso en $X$. Esto porque si 
  $$\ov H = X\To \ov H\cap G=X\cap G=G.$$
  Concluimos que $G$ es irreducible.
\end{ptcbp}

\begin{Ex}%https://math.stackexchange.com/questions/1101542/show-that-a-subset-a-is-irreducible-if-and-only-if-its-closure-is-irreducible
  Las clausuras de los conjuntos irreducibles son irreducibles. La otra dirección también vale pero es inmediata.
\end{Ex}

Si $\ov Y$ fuese reducible, entonces $\ov Y=F_1\cup F_2$ con $F_1,F_2$ cerrados en $\ov Y$. Inmediatamente vale que 
$$Y=(F_1\cap Y)\cup(F_2\cap Y)$$
%\end{multicols}
%%%%%%%%%%%% Contents end %%%%%%%%%%%%%%%%
\ifx\nextra\undefined
\printindex
\else\fi
\nocite{*}
\bibliographystyle{plain}
\bibliography{bibiGeomAlg}
\end{document} 