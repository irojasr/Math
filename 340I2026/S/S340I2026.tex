%----------------------------------------------------------------------------------------
%	PACKAGES AND OTHER DOCUMENT CONFIGURATIONS
%----------------------------------------------------------------------------------------
\DocumentMetadata{
pdfversion=2.0,  
lang=en-US,   
pdfstandard={ua-2, a-4f},
tagging = on, 
tagging-setup={math/setup={mathml-SE}} ,
}
\documentclass[12pt]{article}
\usepackage{amsmath}
%\usepackage[spanish]{babel}
\usepackage[extreme]{savetrees}
\usepackage[utf8]{inputenc}
\usepackage{fancyhdr} % Required for custom headers
\usepackage{lastpage} % Required to determine the last page for the footer
\usepackage{setspace}
\usepackage{hyperref}
%\usepackage[theoremfont,largesc,tighter,osf]{newpxtext}
\usepackage{newpxtext}
\usepackage{xcolor}
\usepackage{graphicx}
\usepackage{booktabs}
\usepackage{multicol}
\def\nlang{}

%%%%%%%%% === Document Configuration === %%%%%%%%%%%%%%

\hypersetup{
pdftitle={MATH 340 -- Letter to the students Spring 2026},
pdfauthor={Ignacio Rojas},
pdfkeywords={Grades and policies, schedule, classrom environment},%
}

\pagestyle{plain}
\lhead{} % Top left header
\chead{} % Top center header
\rhead{} % Top right header
\lfoot{} % Bottom left footer
\ifx\nlang \undefined
\cfoot{\thepage} % Bottom center footer
\rfoot{} % Bottom right footer
\else
\cfoot{\thepage} % Bottom center footer
\rfoot{}% Bottom right footer
\fi

\renewcommand{\footrulewidth}{0.3pt}% default is 0pt
\renewcommand{\headrulewidth}{0.3pt}% default is 0pt
\def\baselinestretch{1.5}% Interlineado
\setlength{\parindent}{0.09\linewidth}% Sangria
\setlength{\footskip}{14.79999pt}

\newskip\smallskipamount \smallskipamount=6pt plus 2pt minus 2pt
\newskip\medskipamount   \medskipamount  =12pt plus 4pt minus 4pt
\newskip\bigskipamount   \bigskipamount =18pt plus 6pt minus 6pt

\newcommand{\MONTH}{%
\ifx\nlang \undefined
  \ifcase\the\month
  \or enero% 1
  \or febrero% 2
  \or marzo% 3
  \or abril% 4
  \or mayo% 5
  \or junio% 6
  \or julio% 7
  \or agosto% 8
  \or septiembre% 9
  \or octubre% 10
  \or noviembre% 11
  \or diciembre% 12
  \fi
\else
  \ifcase\the\month
  \or January% 1
  \or February% 2
  \or March% 3
  \or April% 4
  \or May% 5
  \or June% 6
  \or July% 7
  \or August% 8
  \or September% 9
  \or October% 10
  \or November% 11
  \or December% 12
  \fi
\fi
  }

%----------------------------------------------------------------------------------------
%	ARTICLE CONTENTS
%----------------------------------------------------------------------------------------
\begin{document}

\begin{center}
    {\LARGE MATH 340 -- Letter to the students\\
    Spring 2026}
\end{center}

Welcome to the Differential Equations course for this spring semester of 2026. In this document, you will find information about the course aspects you should know. It is both your right and your responsibility to be informed about what you are expected to learn in this course and how your learning will be assessed. It is advisable to read this letter carefully and ask any questions you may have.

\nocite{*}
\bibliographystyle{plain}
\bibliography{bibi340.bib}

The course will follow the book \cite{BDPM11} Elementary Differential Equations and Boundary Value Problems 11th edition, William E. Boyce, Richard C. Diprima, Douglas B. Meade. It is available in the coordinator's website at: \href{https://www.math.colostate.edu/~liu/MATH340_S26Crd/}{https://www.math.colostate.edu/~liu/MATH340\_S26Crd/}. The previous selection of books will be available in our Canvas shell: \href{https://colostate.instructure.com/courses/215527/modules}{https://colostate.instructure.com/courses/215527/modules} under the ``Books and resources'' module. Other books may be procured through a quick internet search.

\section{Basic information}

\begin{itemize}
    \item \textbf{Coordinator:} Dr. Jiangguo ``James'' Liu, \href{mailto:jiangguo.liu@colostate.edu}{jiangguo.liu@colostate.edu}
    \item \textbf{Instructor:} Ignacio Rojas, \href{mailto:ir@colostate.edu}{ir@colostate.edu}, \href{https://t.me/irojas}{t.me/irojas}, 9708898979\\
Official communications should be sent to my email (i.e. absence notifications, doctor's notes, etc.). I use telegram for instant messaging but I know here in the US SMS is more common. My expectation for you is that you will ask me questions as soon as you have them, this can be during weekends, at 1 am or at any point. I may not answer immediately so I'll do my best to answer in a timely manner. 
\item\textbf{Office Hours:} Tuesdays 9am to 11am, TILT Great Hall. The full schedule is available at \href{https://col.st/UqGUh}{col.st/UqGUh}. Feel free to attend any of the possible help hours, not just mine.\\
Arrangements for meeting outside office hours can be made by contacting me.
\item\textbf{Class time and location:} MTWF 8am at Engineering E103.

\end{itemize}

\section{Grades and policies}

The following table summarizes how the course will be graded.

\tagpdfsetup{table/header-rows={1}}
\begin{table*}[h]
    \centering
\begin{tabular}{lll}\toprule
Activity  & Percentage of grade & Date                        \\
\midrule
Homework  & 25                  & Due Fridays in class        \\
Midterm 1 & 25                  & Thursday 02/26, 5pm-6:50pm  \\
Midterm 2 & 25                  & Thursday 04/09, 5pm-6:50pm  \\
Final     & 25                  & Monday 05/11, 7:30am-9:30am\\
\bottomrule
\end{tabular}
\end{table*}

\textbf{Homework} will be posted each Monday and will be due the Friday of the following week in class (on paper, or printed if you typed it in \LaTeX\ or written on a tablet, or by email if you are out sick or hand it in late). There will not be homework due on exam weeks. Each homework problem will be assigned a number of points based on difficulty ratings, which are listed as:
\begin{multicols}{2}
\begin{enumerate}
    \itemsep=-0.8em
    \item routine, straightforward
    \item somewhat difficult or tricky
    \item difficult
    \item extraordinarily difficult
    \item unsolved\\\phantom{Ahhh}
\end{enumerate}
\end{multicols}

Modifiers of (+) and (-) are used to further distinguish between difficulties.
Therefore a problem ranked 1- is rather easy, whereas 2+ is a hard undergraduate-level homework problem.
The number of points you can earn for each rank of problem is as follows:
\tagpdfsetup{table/header-rows={1}}
\begin{table*}[h]
    \centering
\begin{tabular}{llllllllllllll}\toprule
Rank   & 1- & 1 & 1+ & 2- & 2 & 2+ & 3- & 3 & 3+ & 4- & 4  & 4+ & 5                     \\
Points & 1  & 1 & 2  & 3  & 3 & 4  & 8  & 9 & 10 & 10 & 10 & 10 & $\infty$\\
\bottomrule
\end{tabular}
\end{table*}\\
The points you earn are cumulative, and each homework is graded out of a maximum of 10 points. You can choose any problems of the appropriate difficulties in order to score all 10 points. For instance, if you hand in correct solutions to two 2+ level problems and one 1+, that will be a score of 10.\par
You may hand in a set of problems whose total score is greater than 10 if and only if removing any one of the problems will make the total less than 10. For instance, you may hand in three 2- problems and a 1+, because the total number of points is 11 but removing any one of them will reduce the total to either 8 or 9. But you may not hand in three 2- problems and two 1-, because removing either of the 1- problems will make the total score be 10. \textbf{If you hand in an invalid set of problems, you get an automatic zero for that assignment.}\par
Your score on the homework will be
$$\min(T,10)$$
where $T$ is the total number of points of the problems you handed in correct solutions to. There may be partial credit given on harder problems. \textbf{Make sure you clearly indicate which problems you are handing in!}\par

\textbf{Late homework policy:} You may hand in homework late, but one point will be deducted for each day it is late. In particular, if it is handed in between 9 am on the Friday it is due and 8:59:59 am on Saturday, it is counted as one day late. If it is handed in between 9 on Saturday and 8:59:59 am on Sunday, it is counted as two days late, and so on.\par
These deductions will bottom out at -10 points, and if the total score is negative then the official homework grade is simply a 0.

\textbf{Collaboration} on homeworks is permitted and encouraged. You must list all collaborators' names on a problem's solution, and any external references at the bottom.

\textbf{AI policy}: I will assume that all work you hand in has been revised by an AI model. Such work will be graded normally, and such grade will be a reflection of your ability to harness technology as you prepare for your future in a workforce that will increasingly require your proficiency with AI assisted work.

\textbf{Exam policy}: There are three exams for this course. They are coordinated across all sections of MATH340. One handwritten, two sided crib-sheet is allowed for each midterm, and two sheets are allowed on the final. All exams will be ``closed book/notes''. No calculators/computers/tablets/phones are allowed. For the two midterms, there is no class the day after a midterm. The final exam is comprehensive and covers all the topics in the course.

\section{Topics and tentative schedule}
In the list of topics, chapters and section numbers without reference will refer to the main book of the course \cite{BDPM11} by Boyce, Di Prima and Meade. 
\begin{itemize}
    \itemsep=-0.8em
    \item Week 1: What is a differential equation, new vocabulary, classification, examples and non-examples. Modeling. Qualitative analysis and direction fields. Equilibria and stability theorem. (Chapter 1)
    [see also \href{https://www.youtube.com/watch?v=p_di4Zn4wz4&list=PLZHQObOWTQDNPOjrT6KVlfJuKtYTftqH6&index=1}{3B1B DE1}, \cite{ahmad2015textbook} 1.1-1.3, \cite{schaum} Chapters 1-3, \cite{dobrushkinspiegel} Ch.1, \cite{edwardsandpenney} 1.3, 2.2, and \cite{zillcullen} Ch.1 and 2.1]
    \item Week 2: Solution of first order ODE. Separable, Exact. (Chapter 2.1 through 2.6, omit 2.3)
    [see also \cite{ahmad2015textbook} 3.1-3.3, \cite{schaum} Chs. 4-6, \cite{edwardsandpenney} 1.4-1.6, and \cite{zillcullen} 2.1-2.4]
    \item Week 3: Examples, modeling using first order ODE and more solving. (Chapter 2.3, 2.5)
    [see also \cite{ahmad2015textbook} 1.3, \cite{schaum} Ch. 7, \cite{dobrushkinspiegel} 2.1, \cite{edwardsandpenney} 2.1, 2.3, and \cite{zillcullen} 3.1-3.2]
    \item Week 4: Second and higher order ODE, characteristic equation, undetermined coefficients. Linearity of solutions and Wronskian theorem. (3.1-3.5)
    [see also \cite{ahmad2015textbook} 5.1-5.5 and 5.8, \cite{schaum} Ch. 9, Ch. 11, \cite{edwardsandpenney} 3.1-3.3, and \cite{zillcullen} 4.3-4.5]
    \item Week 5: Applications and modeling via second order. Mechanical vibrations and forced periodic vibrations. (3.7, 3.8, 4.1-4.3)
    [see also \cite{ahmad2015textbook} 5.6.1, 5.7, \cite{schaum} Ch. 14, \cite{edwardsandpenney} 3.6, \cite{zillcullen} 5.1]
    \item Week 6: Review, \textbf{Midterm 1 (02/26, 5pm), no class on Friday}.
    \item Week 7: Crash course on linear systems, eigenvalues, eigenvectors and generalized eigenvectors (7.1-7.3)
    [see also \href{https://www.youtube.com/playlist?list=PL8erL0pXF3JYm7VaTdKDaWc8Q3FuP8Sa7}{CalcBLUE Ch.11}, \href{https://www.youtube.com/playlist?list=PLZHQObOWTQDPD3MizzM2xVFitgF8hE_ab}{3B1B Ch. 14 and 15}, \cite{ahmad2015textbook} 7.1.4, 7.1.5]
    \item Week 8: $2\times2$ ODE systems with matrices, autonomous systems and equilibria, analysis via eigenvalues (7.5-7.8)
    [see also \cite{ahmad2015textbook} 7.2-7.4, \cite{edwardsandpenney} Ch. 5, 6.1, \cite{zillcullen} 8.1, 8.2]
    \item Week 9: \textbf{Spring Break}.
    \item Week 10: Continue with higher order ODE systems. Non homogenous ODE systems (7.9)
    [see also \cite{ahmad2015textbook} 7.5, \cite{dobrushkinspiegel} Ch.9, \cite{edwardsandpenney} 5.7, \cite{zillcullen} 8.3.1]
    \item Week 11: Some theory plus fundamental matrices. (Focus on 7.4 and 7.7)
    [see also \cite{edwardsandpenney} 5.6, \cite{zillcullen} 8.3.2]
    \item Week 12: Review \textbf{Midterm 2 (04/09, 5pm), no class on Friday}.
    \item Week 13: Introduction to Laplace transform, properties, examples, tables. (6.1)
    [see also \href{https://www.youtube.com/watch?v=j0wJBEZdwLs}{3B1B DefLaplace}, \cite{ahmad2015textbook} 11.1-11.3, \cite{dobrushkinspiegel} 5.1-5.4, skip 5.3, \cite{schaum} Ch. 21-23, \cite{edwardsandpenney} 7.1, \cite{zillcullen} 7.1-7.4]
    \item Week 14: Applications of the Laplace transform to first and second order ODE. (6.2)
    [see also \href{https://www.youtube.com/watch?v=FE-hM1kRK4Y}{3B1B Laplace2}, \cite{ahmad2015textbook} 11.4, 11.5, \cite{dobrushkinspiegel} 5.5, 5.6, \cite{schaum} Ch. 24, 25, \cite{edwardsandpenney} 7.2-7.5, \cite{zillcullen} 7.6]
    \item Week 15: Nonlinear systems, equilibria and stability. Applications of systems. (9.1-9.5, skip 9.4).
    [see also \cite{dobrushkinspiegel} 10.1-10.3, \cite{edwardsandpenney} 6.1-6.3, \cite{zillcullen} Ch. 10]
    \item Week 16: Optional topics: KdV equation, Abel theorem, numerical methods, PDE. Final review.
    \item Week 17: Final exam on Monday 05/11, 7:30am.
\end{itemize}
\section{Classroom environment}

In order to help make our classroom an excellent place to be in and learn mathematics, please keep in mind the following principles:
\begin{itemize}
    \itemsep=-0.8em
    \item \textbf{Speaking up in class is encouraged!} If you don't understand something, no matter how small, chances are someone else in the class doesn't understand it either, and asking will help me address the confusion and make things clearer. Off-the-wall ideas and comments are also always encouraged.
 \item \textbf{Talk to each other!} If I give a problem for you all to think about during class, chatting with your classmates is encouraged; explaining helps both the explainer and the listener to understand better.
\item \textbf{Kindness} The students in this class will be coming from many different backgrounds, both mathematically and as human beings. Please be respectful and encouraging towards each other.
\end{itemize}

\section{Academic integrity}

This course will adhere to the CSU Academic Integrity Policy as found on the Student Responsibilities page of the CSU General Catalog and in the Student Conduct Code. At a minimum, violations will result in a grading penalty in this course and a report to the Student Resolution Center.

\section{Disabilities}

This is a disability-inclusive classroom. Students with disabilities who need accommodations can ask me directly, though I may ask you to contact the Student Disability Center (SDC) before approving accommodations such as extra time on exams. If you have already done so, it is very likely that I have received the accommodations letter by now. However, if there are other accommodations that are not related to course/grading policy that could help you participate more fully in class, please let me know.

\section{QR codes and external links}
The following QR codes will direct you to the course coordination webpage: \href{https://www.math.colostate.edu/~liu/MATH340_S26Crd/ColoState_MATH340_S26Crd_Info.html}{math.colostate.edu/liu/MATH340\_S26Crd/} and \href{https://col.st/2FA2g}{col.st/2FA2g}. The second linked page provides policies relevant to courses and resources to help with various challenges you may encounter.\\
\begin{center}
\includegraphics[width=0.25\textwidth, alt={Coordination website QR.}]{qrcoord.png}\includegraphics[width=0.25\textwidth, alt={TILT Syllabus resources QR.}]{QRCodeSyllabusResources.png}
\end{center}

The academic calendar containing important dates can be found in \href{https://calendar.colostate.edu/academic/}{calendar.colostate.edu/academic/} along with final exam dates at \href{https://thehub.colostate.edu/registration-records/final-exams/}{thehub.colostate.edu/registration-records/final-exams/}.\par
Without more to say besides wishing you all the success this semester, and at your disposal for any inquires you may have remains; \par
Yours very truly,\par 
\textit{\textbf{José Ignacio Rojas Rojas}}%\par


%%%%%%%%%%%% Contents end %%%%%%%%%%%%%%%%
%Indent whole paragrah: https://tex.stackexchange.com/questions/35933/indenting-a-whole-paragraph
%Dates: https://tex.stackexchange.com/questions/185548/the-year-in-roman-and-the-month-in-text
%SKIPS:https://tex.stackexchange.com/questions/41476/lengths-and-when-to-use-them/41488
%SKIPS2: https://tex.stackexchange.com/questions/476/what-if-anything-is-the-advantage-of-bigskip-and-friends-over-vspace
%%%%%%%%%%%%%%%%%%%%%%%%%%%%%%%%%%%%%%%%%%

\end{document} 