\documentclass[12pt]{memoir}

\def\nsemestre {II}
\def\nterm {Fall}
\def\nyear {2022}
\def\nprofesor {Maria Gillespie}
\def\nsigla {MATH501}
\def\nsiglahead {Combinatorics}
\def\nextra {HW1}
\def\nlang {ENG}
\input{../../headerVarillyDiff}

\begin{document}

\begin{Ej}[Exercise 1]
 Let $a_n$ be the number of ways to hand in a subset of the problems on this homework (Homework 1) whose total score is $n$, even if handing in such a homework would be against the rules. Note that different problems of the same rating are now thought of as distinct; for instance, if you hand in parts (a) through (e) of the next problem, that is different from handing in parts (h) through (l), even though they all have difficulty level 1+.
\begin{enumerate}[i)]
    \item Find a factorization for the generating function for the sequence $a_n$. Explain how you found your answer.
    \item Compute $a_{10}$
\end{enumerate}
\end{Ej}

\begin{ptcbr}
Let us begin by enumerating all the problems and their respective scores in permutation notation (even though it's not a permutation).
$$
\left(
\begin{matrix}
    1&2&3&4&5_{5a}&6_{5b}&\dots&16_{5\l}& 17_6 & 18_7 \\
    4&2&2&2&2&2&\dots&2& 3 & 2 
\end{matrix}\hspace{0.67em}
\begin{matrix}
    19_8&20_9&21_{10}&22_{11}\\
    8&8&3&4
\end{matrix}
\right)
$$
So in total we have 22 problems, we'll arrange these once more now by score and by the amount of problems which have such a score:
$$
\begin{pmatrix}
    2&3&4&8\\
    16&2&2&2
\end{pmatrix}
$$
Now let's call $x$ the variable which counts the amount of points for a given problem and $y_j$ with $j\in[22]$ the variable which counts the amount of problems\footnote{I was having issues with this, since the polynomial with only $x$'s only cares about the amount of ways to hand in problems to get an specific score and not the problems themselves. When I count I like to count the objects themselves, so I kindly asked \textbf{Kyle} to help me out with my conundrum. He suggested that I use more variables to count the number of problems. I know that this was not asked but I simply felt more comfortable including them in the beginning.}. So, for example the monomial which represents the empty hand-in is $1y_1^0\dots y_{22}^0x^0$ since there's only one way to obtain zero points and that is handing in no problems at all. On the other hand, if we hand in all the problems, the total number of points would be $62$ so the monomial which represents that score is 
$$1y_1y_2\dots y_{22}x^{62}.$$
Having the variable $y_j$ present in the monomial indicates whether problem $j$ was handed in or not, it's a binary choice. So the polynomial which encodes problem $j$'s information is 
$$(1+y_jx^{s_j})$$
where $s_j$ is the score assigned to problem $j$. It follows that the generating function has the factorization
$$\prod_{j=1}^{22}(1+y_jx^{s_j})=(1+y_1x^4)(1+y_2x^2)(1+y_3x^2)\cdots(1-y_{17}x^3)\cdots(1+y_{22}x^4).$$
Not considering the problems themselves, looking only for the scord, we would count without the $y_j$'s so the polynomial would be
$$(1+x^2)^{16}(1+x^3)^2(1+x^4)^2(1+x^8)^2.$$
This is the factorization for the generating function for $a_n$.\par 
Expanding out\footnote{This was done using Wolfram|Alpha.} the whole thing, we would get that the monomial with power 10 is \un{$5658x^{10}$} and so the number of combinations of problems that we could hand in to get 10 points is $5658$.
\end{ptcbr}

\begin{Ej}[Exercise 8, Stanley 1.3.c]
    Give a combinatorial proof of the following identity
    $$\sum_{k=0}^n\binom{2k}{k}\binom{2(n-k)}{n-k}=4^n.$$
\end{Ej}
%https://math.stackexchange.com/questions/687221/proving-sum-k-0n2k-choose-k2n-2k-choose-n-k-4n
%https://math.stackexchange.com/questions/37971/identity-for-convolution-of-central-binomial-coefficients-sum-limits-k-0n
%https://math.stackexchange.com/questions/1064216/generating-functions-and-central-binomial-coefficient
%https://en.wikipedia.org/wiki/Catalan_number
%https://en.wikipedia.org/wiki/Central_binomial_coefficient#Generating_function

\begin{ptcbr}
    We will proceed considering generating functions. Call $(a_n)_{n\in\bN}$ and $(b_n)_{n\in\bN}$ the sequences in question. We can see that the generating function for $b_n$ is 
    $$\sum_{n=0}^\infty b_nx^n=\sum_{n=0}^\infty (4x)^n=\frac{1}{1-4x}.$$
    To see that $(a_n)=(b_n)$ we will see that $(a_n)$ generates the same function. The terms of the sequence $a_n$ are a convolution of central binomial coefficients $c_n=\binom{2n}{n}$ which indicates the idea of a squaring a series. Thus, we are set out to prove that
    $$\left(\sum_{n=0}^\infty c_nx^n\right)^2=\sum_{n=0}^\infty a_nx^n=\frac{1}{1-4x}\iff \sum_{n=0}^\infty c_nx^n=\frac{1}{\sqrt[]{1-4x}}.$$
    Finding the central binomial coefficients' generating function can be done through the use of the binomial theorem. We have that 
    $$\frac{1}{\sqrt{1-4x}}=\sum_{n=0}^\infty\binom{-1/2}{n}(-4x)^n$$
    and we can manipulate the generalized binomial coefficient as follows:
    \begin{align*}
        \binom{-1/2}{n}&=\frac{(-1/2)(-1/2-1)(-1/2-2)(-1/2-3)\cdots(-1/2-(n-1))}{n!}\\
        &=\frac{(-1/2)(-3/2)(-5/2)(-7/2)\cdots((-2n+1)/2)}{n!}\\
        &=\frac{(-1)^n}{2^n}\.\frac{1\.3\.5\.\cdots\.(2n-1)}{n!}\\
        &=\frac{(-1)^n}{2^n}\.\frac{1\.3\.5\.\cdots\.(2n-1)}{n!}\cdot\frac{2\.4\.6\.\cdots\.2n}{2^nn!}\\
&=\frac{(-1)^n}{2^{2n}}\.\frac{(2n)!}{(n!)^2}=\frac{1}{(-4)^n}\binom{2n}{n}
    \end{align*}
    Therefore 
    $$\sum_{n=0}^\infty\binom{-1/2}{n}(-4x)^n=\sum_{n=0}^\infty\binom{2n}{n}x^n$$
    and by uniqueness of coefficients our identity follows.
\end{ptcbr}
\end{document} 
