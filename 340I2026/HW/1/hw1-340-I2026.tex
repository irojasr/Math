%----------------------------------------------------------------------------------------
%	PACKAGES AND OTHER DOCUMENT CONFIGURATIONS
%----------------------------------------------------------------------------------------
\DocumentMetadata{
pdfversion=2.0,  
lang=en-US,   
pdfstandard={ua-2, a-4f},
tagging = on, 
tagging-setup={math/setup={mathml-SE}} ,
}
\documentclass[12pt]{article}
\usepackage{amsmath}
\usepackage[extreme]{savetrees}
\usepackage[utf8]{inputenc}
\usepackage{newpxtext}
\usepackage{hyperref}
\usepackage{multicol}
\usepackage{physics}

%%%%%%%%% === Document Configuration === %%%%%%%%%%%%%%

\hypersetup{
pdftitle={MATH 340 -- Homework 1 Spring 2026},
pdfauthor={Ignacio Rojas},
pdfkeywords={first homework},%
}

%----------------------------------------------------------------------------------------
%	ARTICLE CONTENTS
%----------------------------------------------------------------------------------------
\begin{document}

\begin{center}
    {\Large MATH 340 -- Homework 1,\quad Spring 2026}
\end{center}

Recall that you must hand in a subset of the problems for which deleting any problem makes the total score less than 10. The maximum possible score on this homework is 10 points. See the syllabus for scoring details. NOTE: All problems from Boyce, Di Prima and Meade are from the \(11^{\text{th}}\)  edition, which is available online in the coordinator's website: \href{https://www.math.colostate.edu/~liu/MATH340_S26Crd/}{https://www.math.colostate.edu/~liu/MATH340\_S26Crd/}.

\begin{enumerate}
  \item (2) [3 points] Understand the following concepts by writing out their definitions:
  \begin{multicols}{3}
  \begin{enumerate}
    \itemsep=0em
    \item Slope/direction field
    \item Phase line
    \item Integral/solution curve
    \item Isocline
    \item Nullcline
    \item Equilibrium point
  \end{enumerate}
  \end{multicols}
  \item (2+) [4 points] Classify the following equations by answering: Is it implicit or can be made explicit? What is the order? Is it linear or nonlinear, and if it is linear, is it homogeneous, constant-coefficients? Is it autonomous? Is it exact?
  \begin{multicols}{2}
  \begin{enumerate}
    \itemsep=0em
    \item \(\sin(t)\ddot{x}+\cos(t)x=t^2\)
    \item \(((y^{(3)})^2-\sin(x))'=0\)
    \item \((y')^2-(y')^5=2x^3y^5\)
    \item \(y^{(5)}+y^6=y^{(4)}-y''+y\)
    \item \(\ddot{x}+tx^2=t\)
    \item \(x^{(4)}=0\)
  \end{enumerate}
  \end{multicols}
  
  \item (1-) [1 point] For which \(k\) is \(\dot{x}+x^k=t^{k+2}\) linear? [Hint: There are two answers.]
  
  \item (2+) [4 points] Give an explicit example of a third order linear ODE with variable coefficients that is non autonomous and non homogeneous, and in which all derivatives up to third order appear.

  \item (1+) [2 points] A first order ODE is written as \(P(x,y)\dd x+Q(x,y)\dd y=0\). Write  down the condition for this ODE to be exact. Consider the ODE
  \[
  (e^xy^2)\dd x+(2e^xy+x\cos(y))\dd y=0,\quad (x+3x^2\sin(y))\dd x+x^4\cos(y)\dd y=0.
  \]
  Determine whether they are exact.

  \item (1+) [2 points] Boyce, di Prima, and Meade chapter 1 section 1 problems 11, 12 and 15.

  \item (1) [1 point] For the differential equation
  \[
    x^{(3)}-12\ddot{x}+48\dot{x}-64x=0.
  \]
  Verify whether \(x_1(t)=e^{4t}\) and \(x_2(t)=e^t\) satisfy the equation.

  \item (2-) [2 points] For a real valued function \(f\), show that 
\[
y(x)=\int_{0}^{x}tf(t)\dd t
\]
satisfies the differential equation \(xy''+(x-1)y'=0\).
  
  \item (3-) [8 points] Suppose \(y(x)\) solves the autonomous equation \(\dv{y}{x}=f(y)\) and is bounded above and below by two consecutive equilibrium points \(c_1<c_2\). If \(f(y)>0\), then it happens that \(\lim_{x\to\infty}y(x)=c_2\). Explain why its impossible for the limit to equal \(L<c_2\). In your argument, consider what happens to \(y'(x)\) as \(x\to\infty\).
  
  \item (2+) [4 points] Give an example of an autonomous differential equation with exactly three equilibrium points, two stable and one unstable. Draw the phase line indicating stability, and sketch the slope field. In the slope field, draw four distinct nonconstant solution solutions between the equilibria. Analyze the behaviour of the solutions as \(x\to\pm\infty\).\par 
  [Use \href{https://aeb019.hosted.uark.edu/dfield.html}{https://aeb019.hosted.uark.edu/dfield.html} to help yourself.]

  \item (1) [1 point] Consider the following autonomous equations
  \[
  y'=y^3-y,\quad y'=(y^4-16)(y^2-16).  
  \]
  List the equilibria for both.

  \item (1) [1 point] Consider the differnetial equation 
  \[
  y'=y^n,\quad n>0
  \]
  Use \href{https://aeb019.hosted.uark.edu/dfield.html}{https://aeb019.hosted.uark.edu/dfield.html} and describe what happens to the equilibrium point \(0\)  as \(n\)  varies. Support your statements using the stability theorem.

  \item (2+) [4 points] Consider the equation \(\dot{x}+kx=1\). Is there any value of \(k\) for which there is a nonconstant solution \(x\) such that \(x\to-3\) as \(t\to\infty\)? Explain and support your claims via the stability theorem.

  \item (4) [10 points] Consider the initial value problem
  \[
  \dot{x}(t)=1+t^2x(t)^4,\quad x(0)=0.
  \]
  Show that the solution to this equation is necessarily an odd function, this is, \(x(-t)=-x(t)\). [Hint: Use the Picard-Lindelöf theorem.]
\end{enumerate}

\end{document} 