%----------------------------------------------------------------------------------------
%	PACKAGES AND OTHER DOCUMENT CONFIGURATIONS
%----------------------------------------------------------------------------------------
\DocumentMetadata{
pdfversion=2.0,  
lang=en-US,   
pdfstandard={ua-2, a-4f},
tagging = on, 
tagging-setup={math/setup={mathml-SE}} ,
}
\documentclass[12pt]{article}
\usepackage{amsmath}
\usepackage[extreme]{savetrees}
\usepackage[utf8]{inputenc}
\usepackage{newpxtext}
\usepackage{hyperref}
\usepackage{multicol}
\usepackage{physics}

%%%%%%%%% === Document Configuration === %%%%%%%%%%%%%%

\hypersetup{
pdftitle={MATH 340 -- Homework 2 Spring 2026},
pdfauthor={Ignacio Rojas},
pdfkeywords={second homework},%
}

%----------------------------------------------------------------------------------------
%	ARTICLE CONTENTS
%----------------------------------------------------------------------------------------
\begin{document}

\begin{center}
    {\Large MATH 340 -- Homework 2,\quad Spring 2026}
\end{center}

Recall that you must hand in a subset of the problems for which deleting any problem makes the total score less than 10. The maximum possible score on this homework is 10 points. See the syllabus for scoring details. NOTE: All problems from Boyce, Di Prima and Meade are from the \(11^{\text{th}}\)  edition, which is available online in the coordinator's website: \href{https://www.math.colostate.edu/~liu/MATH340_S26Crd/}{https://www.math.colostate.edu/~liu/MATH340\_S26Crd/}.

\begin{enumerate}

  \item (2) [3 points] Boyce, di Prima, and Meade chapter 2 section 1 problems 3 and 7. 
  
  \item (2) [3 points] Boyce, di Prima, and Meade chapter 2 section 1 problems 9 and 11. 
  
  \item (2+) [4 points] Boyce, di Prima, and Meade chapter 2 section 1 problem 23.
  
  \item (2-) [2 points] Boyce, di Prima, and Meade chapter 2 section 2 problems 10 and 12.
  
  \item (2-) [2 points] Boyce, di Prima, and Meade chapter 2 section 2 problems 10 and 12.
  
  \item (2) [3 points] Boyce, di Prima, and Meade chapter 2 section 2 problems 17 and 18.
  
  \item (2+) [4 points] Boyce, di Prima, and Meade chapter 2 section 2 problem 27.

  
  \item (1) [1 point] Verify the following:
  \begin{enumerate}
    \item The constant function \(x=-1\) is a solution of \(\dot{x}=(x-1)\ln(x+2)\).
    \item The constant function \(y=\pi/2\) is a solution of \(y'=2x\cos^2(y)\).
    \item The IVP \(\dot{x}=x+e^t,\quad x(0)=1\) has \(x(t)=(1+t)e^{-t}\) as a solution.
  \end{enumerate}

  \item (2) [3 points] Solve the following IVP 
  \begin{multicols}{2}
    \begin{enumerate}
      \item \(\dot{x}=\cos^2(x)/(1+t^2),\quad x(0)=\pi/4\)
      \item \(\cos(t)\dot{x}-\sin(t)x=e^t,\quad x(0)=2025\)
    \end{enumerate}
  \end{multicols}

  \item (3-) [8 points] Solve the differential equation 
  \[
  y'(x)+y(x)=f'(x)+f(x)
  \]
  for any differentiable function \(f\). Conclude that all solutions of this equation tend to \(f(x)\) as \(x\to\infty\). [Hint: Put the derivatives on one side and leave the rest on the other side.]
  
  \item (2+) [4 points] Solve the following equations via separation of variables:
  \begin{multicols}{3}
    \begin{enumerate}
      \item \(e^{f(x)}g(y)\dd x=\frac{\ln(g(y)^{g'(y)})}{f'(x)}\dd y\)
      \item \(2xy'=y^2\log_{\sqrt{x}}(2)\)
      \item \(\left(1+\frac1x+\frac1y+\frac1{xy}\right)\dv{y}{x}=1\)
    \end{enumerate}
  \end{multicols}

  \item (2+) [4 points] In each case, determine the values of \(a,\dots,d\) for the equation to be exact. Then, solve the IVP.
  \begin{enumerate}
    \item \((x^2+y^2)\dd x+(axy+y^4)\dd y=0\) with \(y(0)=1\)
    \item \((x^2+axy+by^2)\dd x+(x^2+cxy+dy^2)\dd y=0\) with \(y(0)=0\)
  \end{enumerate}


  \item (3-) [8 points] Verify that \(y_c(x)=Ae^{-\int P(x)\dd x}\) is a general solution of \(y'+P(x)y=0\). Then verify that 
\[
y_p(x)=e^{-\int P(x)\dd x}\left(\int Q(x)e^{\int P(x)\dd x}\dd x\right) 
\]
is a particular solution of \(y'+P(x)y=Q(x)\). Finally, if \(y_c,y_p\) solve the previous equations, verify that \(y_c+y_p\) is a \textbf{general} solution to \(y'+P(x)y=Q(x)\).
\end{enumerate}
\end{document} 