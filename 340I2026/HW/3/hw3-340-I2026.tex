%----------------------------------------------------------------------------------------
%	PACKAGES AND OTHER DOCUMENT CONFIGURATIONS
%----------------------------------------------------------------------------------------
\DocumentMetadata{
pdfversion=2.0,  
lang=en-US,   
pdfstandard={ua-2, a-4f},
tagging = on, 
tagging-setup={math/setup={mathml-SE}} ,
}
\documentclass[12pt]{article}
\usepackage{amsmath}
\usepackage[extreme]{savetrees}
\usepackage[utf8]{inputenc}
\usepackage{newpxtext}
\usepackage{hyperref}
\usepackage{multicol}
\usepackage{physics}

%%%%%%%%% === Document Configuration === %%%%%%%%%%%%%%

\hypersetup{
pdftitle={MATH 340 -- Homework 3 Spring 2026},
pdfauthor={Ignacio Rojas},
pdfkeywords={third homework},%
}

%----------------------------------------------------------------------------------------
%	ARTICLE CONTENTS
%----------------------------------------------------------------------------------------
\begin{document}

\begin{center}
    {\Large MATH 340 -- Homework 3,\quad Spring 2026}
\end{center}

Recall that you must hand in a subset of the problems for which deleting any problem makes the total score less than 10. The maximum possible score on this homework is 10 points. See the syllabus for scoring details. NOTE: All problems from Boyce, Di Prima and Meade are from the \(11^{\text{th}}\)  edition, which is available online in the coordinator's website: \href{https://www.math.colostate.edu/~liu/MATH340_S26Crd/}{https://www.math.colostate.edu/~liu/MATH340\_S26Crd/}.

\begin{enumerate}

  \item (1+) [2 points] A tank initially contains 120 liters of pure water. A mixture containing a concentration of $\gamma$ g/L of salt enters the tank at a rate of 2 L/min, and the well-stirred mixture leaves the tank at the same rate. Find an expression for the amount of salt in the tank at any time $t$. Also, find the limiting amount of salt in the tank as $t \to \infty$.
  
  \item (1+) [2 points] Consider a tank of water with a hole at the bottom. Torricelli's Law states that the rate at which water flows out is proportional to the square root of the height of the water, i.e., $\frac{dV}{dt} = -k\sqrt{h}$.
Suppose the tank is a cylinder with cross-sectional area $A$. 
\begin{enumerate}
    \item Derive the differential equation for the height $h(t)$.
    \item If the tank is initially filled to height $h_0$, how long does it take to drain completely?
\end{enumerate}

\item (E) (1) [1 point] Consider a population modeled by the logistic equation $y' = r y (1 - y/K)$.
\begin{enumerate}
    \item Find the general solution using separation of variables.
    \item If $y(0) = K/2$, find the particular solution.
    \item Sketch the phase line and identify the stability of the equilibrium points.
\end{enumerate}

\item (2-) [3 points] For an object of mass $m$ falling under gravity with air resistance proportional to the \textbf{square} of velocity:
$$m \frac{dv}{dt} = mg - k v^2$$
\begin{enumerate}
    \item Determine the terminal velocity $v_\tau$.
    \item Solve the differential equation to find $v(t)$ assuming $v(0)=0$. (Hint: You may need partial fractions or hyperbolic functions).
    \item Compare the behavior of this solution to the linear drag model ($F = -kv$) as $t \to \infty$.
\end{enumerate}
  
  \item (2-) [3 points] Suppose a deer population \(P(t)\) satisfies the logistic equation
  \[
  \dv{P}{t}=0.225P-0.0003P^2.
  \]
  With a direction field and an appropriate solution curve answer: If there's 25 deer at time \(t=0\), how much time will it take to double the population of deer? What is the limiting population?

  \item (2) [3 points] A projectile is launched vertically from Earth's surface. Gravity is not constant but varies with distance $x$ from the center of the Earth: $F_g = -GMm/x^2$.
\begin{enumerate}
    \item Write the differential equation $m \frac{dv}{dt} = F_g$.
    \item Use the chain rule substitution $a = v \frac{dv}{dx}$ to convert this into a separable equation in $v$ and $x$.
    \item Determine the minimum initial velocity $v_0$ required for the projectile to never return (i.e., $v > 0$ as $x \to \infty$).
\end{enumerate}

  \item (2+) [4 points] Consider a cascade of two tanks. Tank 1 contains $V_1$ gallons of brine and Tank 2 contains $V_2$ gallons. Pure water flows into Tank 1 at rate $r$ gal/min. The mixture flows from Tank 1 to Tank 2 at rate $r$, and from Tank 2 out to a drain at rate $r$.
\begin{enumerate}
    \item Let $x(t)$ be the salt in Tank 1 and $y(t)$ be the salt in Tank 2. Write the system of differential equations modeling this system.
    \item Solve for $y(t)$ assuming initial conditions $x(0)=x_0$ and $y(0)=0$.
\end{enumerate}

\item (2+) [4 points] In fishery management, the Schaefer model modifies logistic growth to account for harvesting proportional to effort $E$. The equation is:
$$\frac{dy}{dt} = r y \left(1 - \frac{y}{K}\right) - E y$$
\begin{enumerate}
    \item Show that if $E < r$, there is a stable equilibrium $y_2 > 0$. Find this value in terms of $E, r, K$.
    \item The "sustainable yield" is $Y = E y_2$. Find the value of $E$ that maximizes this yield.
\end{enumerate}

\item (3) [9 points] If harvesting is done at a constant rate $h$ (independent of population size), the model becomes:
$$\frac{dy}{dt} = r y \left(1 - \frac{y}{K}\right) - h$$
Analyze the equilibrium points of this equation for different values of $h$. Show that a \textbf{saddle-node bifurcation} occurs at a critical value $h_c = rK/4$. What happens to the population if $h > h_c$?

\item (2) [3 points] Let $y$ be the proportion of infectious individuals in a population and $x$ be the susceptible proportion, with $x+y=1$. The spread of the disease is modeled by:
$$\frac{dy}{dt} = \alpha y (1 - y) - \beta y$$
where $\alpha$ is the transmission rate and $\beta$ is the recovery rate. Analyze the stability of the disease-free equilibrium ($y=0$). Under what condition ($\alpha$ vs $\beta$) does an epidemic occur?

\item (4) [10 points] Generalize this to a cascade of $n$ identical tanks each with volume $V$. Show that the salt in the $k$-th tank is given by a function resembling the Poisson distribution:
\[  
x_k(t) = \frac{x_0 (rt/V)^{k-1}}{(k-1)!} e^{-rt/V}
\]
  \item (2-) [3 points] Consider the population model \(y'=\frac{y}{10}\left(1-\frac{y}{K}\right)\) with capacity \(K>0\).
  \begin{enumerate}
    \item Find the general solution to this ODE.
    \item Assume that the initial population is a third of the capacity. Find the time at which population has doubled.
    \item Sketch the solution in the previous item on a \(ty\)-plane. Find the limit \(\lim_{t\to\infty}y(t)\).
  \end{enumerate}

  \item (5) The Collatz conjecture deals with the discrete map on integers:
  \[
  n\mapsto\left\lbrace
  \begin{aligned}
    &n/2,\quad\text{if }n\text{ is even},\\
    &3n+1,\quad\text{if }n\text{ is odd}.
  \end{aligned}
  \right.
  \]
Recent (1996) research attempts to embed this into a continuous differential equation. Consider a function $f(x)$ that interpolates these rules:
\[
f(x) = \frac{1}{4}(1 + 4x - (1 + 2x)\cos(\pi x))
\]
Analyze the differential equation $\frac{dx}{dt} = f(x) - x$. 
\begin{enumerate}
    \item Does this system possess limit cycles that correspond to the discrete cycle $4 \to 2 \to 1$?
    \item \textbf{Open Problem:} Can you construct a Lyapunov function or use numerical boundaries to prove that all trajectories for $x > 0$ eventually enter a bounded region? (Note: A rigorous proof would likely solve the Collatz conjecture).
\end{enumerate}

\end{enumerate}

\end{document}