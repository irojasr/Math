%----------------------------------------------------------------------------------------
%	PACKAGES AND OTHER DOCUMENT CONFIGURATIONS
%----------------------------------------------------------------------------------------
\DocumentMetadata{
pdfversion=2.0,  
lang=en-US,   
pdfstandard={ua-2, a-4f},
tagging = on, 
tagging-setup={math/setup={mathml-SE}} ,
}
\documentclass[12pt]{article}
\usepackage{amsmath}
\usepackage[extreme]{savetrees}
\usepackage[utf8]{inputenc}
\usepackage{newpxtext}
\usepackage{hyperref}
\usepackage{multicol}
\usepackage{physics}
\usepackage{siunitx}

%%%%%%%%% === Document Configuration === %%%%%%%%%%%%%%

\hypersetup{
pdftitle={MATH 340 -- Homework 3 Spring 2026},
pdfauthor={Ignacio Rojas},
pdfkeywords={third homework},%
}

%----------------------------------------------------------------------------------------
%	ARTICLE CONTENTS
%----------------------------------------------------------------------------------------
\begin{document}

\begin{center}
    {\Large MATH 340 -- Homework 3,\quad Spring 2026}
\end{center}

Recall that you must hand in a subset of the problems for which deleting any problem makes the total score less than 10. The maximum possible score on this homework is 10 points. See the syllabus for scoring details. NOTE: All problems from Boyce, Di Prima and Meade are from the \(11^{\text{th}}\)  edition, which is available online: \href{https://www.math.colostate.edu/~liu/MATH340_S26Crd/Textbook_BoyceDiPrimaMeade_11ed.pdf}{https://www.math.colostate.edu/\(\sim\)liu/MATH340\_S26Crd/Textbook\_BoyceDiPrimaMeade\_11ed.pdf}.

\begin{enumerate}

  \item (1) [1 point] Boyce, di Prima, and Meade chapter 2 section 3 problem 1.
  
  \item (1+) [2 points] Boyce, di Prima, and Meade chapter 2 section 3 problem 2.
  
  \item (1+) [2 points] Boyce, di Prima, and Meade chapter 2 section 3 problem 10.
  
  \item (1) [1 point] Boyce, di Prima, and Meade chapter 2 section 3 problem 12.
  
  \item (1+) [2 points] Boyce, di Prima, and Meade chapter 2 section 3 problem 13.
  
  \item (1-) [1 point] Boyce, di Prima, and Meade chapter 2 section 5 problems 3 and 4.
  
  \item (1+) [2 points] Boyce, di Prima, and Meade chapter 2 section 5 problem 15 part \(a\).
  
  \item (\textbf{E}) (2) [3 points] Consider a population modeled by the logistic equation \(y' = r y (1 - y/K)\).
\begin{enumerate}
    \item Solve the initial value problem assuming \(y(0) = K/2\).
    \item Sketch the phase line and identify the stability of the equilibrium points.
\end{enumerate}

\item (1+) [2 points] Suppose a deer population \(P(t)\) satisfies the logistic equation
  \[
  \dv{P}{t}=0.225P-0.0003P^2.
  \]
  With a direction field and an appropriate solution curve answer: If there's 25 deer at time \(t=0\), how much time will it take to double the population of deer? What is the limiting population?

  \item (2) [3 points] Boyce, di Prima, and Meade chapter 2 section 5 problem 16 parts \(a\) and \(b\). [Hint: Recall the \(2^{\text{nd}}\) derivative test.]

  \item (\textbf{E}) (2-) [3 points] A tank initially contains 100 \unit{\liter} of pure water. A mixture containing a concentration of \(\gamma \) \unit[per-mode = symbol]{\gram\per\liter} of salt enters the tank at a rate of \(2\) \unit{\litre\per\min}. The well-stirred mixture leaves the tank at the same rate.
\begin{enumerate}
    \item Establish an IVP for the amount of salt \(Q(t)\) in the tank.
    \item Find an expression in terms of \(\gamma\) for \(Q(t)\).
    \item Find the limiting amount of salt in the tank as \(t\to\infty\).
\end{enumerate}
[Hint: Sketching a tank with info for the entry, exit, and within-the-tank will be helpful.]

  \item (2+) [4 points] Consider a \emph{cascade} of two tanks with \(V_1=100\unit{\liter}\) and \(V_2=200\unit{\liter}\) the
volumes of brine in the two tanks. Each tank also initially
contains 50 \unit{\gram} of salt. The three flow rates, entering, tank1 to Tank 2 and exit are each 5 \unit{\liter\per\min}, with pure water flowing into Tank 1. 
\begin{enumerate}
    \item Find \(x_1(t)\), the salt in Tank 1 at time \(t\).
    \item Show that if \(x_2(t)\) is the amount of salt in Tank 2 then
    \[
    \dv{x_2}{t}=\frac{5x_1}{100}-\frac{5x_2}{200},
    \]
    and then solve for \(x_2\) using the function \(x_1\) found in the previous item.
    \item Conclude by finding the maximum amount of salt ever in Tank 2.
\end{enumerate}


\item (3-) [8 points] Consider a \emph{multiple cascade} of tanks. At time \(t=0\), Tank 0 contains \(1\) \unit{\liter} of brine and \(1\) \unit{\liter} of water. All remaining tanks contain \(2\) \unit{\liter} of water each.\par
Pure water is pumped into tank 0 at a rate of \(1\) \unit{\liter\per\min}. The varying mixture is pumped into the below at the same rate. Let \(x_n\) denote the amount of brine in Tank \(n\) at time \(t\).
\begin{enumerate}
  \item Show that \(x_0(t)=e^{-t/2}\).
  \item In general, show that 
  \[
  x_n(t)=\frac{t^ne^{-t/2}}{(n!)(2^n)},\quad n\geq 0.
  \]
  \item Show that the maximum value of \(x_n\) is \(M_n=x_n(2n)=n^ne^{-n}/n!\).
\end{enumerate}


\item (2+) [4 points] For an object of mass \(m\) falling under gravity with air resistance proportional to the \textbf{square} of velocity:
\[
m \dv{v}{t} = mg - k v^2
\]
\begin{enumerate}
    \item Determine the terminal velocity \(v_\tau\).
    \item Solve the differential equation to find \(v(t)\) assuming \(v(0)=0\).
    \item Compare the behavior of this solution to the linear drag model \(F = -kv\) as \(t \to \infty\).
\end{enumerate}
  

  \item (2+) [4 points] A projectile is launched vertically from Earth's surface with initial velocity \(v_0\). Gravity is not constant but varies with distance \(x\) from the surface of the Earth: 
  \[
  F_g = -\frac{mgR^2}{(R+x)^2}.
  \]
  Assuming no air resistance, Newton's \(2^{\text{nd}}\) law gives the differential equation \(m \dv{v}{t} = F_g\).
\begin{enumerate}
    \item Convert this into a separable equation in \(v\) and \(x\) by applying the chain rule to \(v\) as
    \[
    \dv{v}{t}=\dv{v}{x}\dv{x}{t}=v\dv{v}{x}.
    \]
    with this, find a solution to the IVP assuming \(v=v_0\) when \(x=0\).
    \item Find the initial velocity needed for the projectile to reach a maximum altitude \(A_{\max}\) above the surface of the Earth.
    \item Find the minimum initial velocity for which the projectile \emph{will not} return to the Earth. This is known as the \emph{escape velocity}.
\end{enumerate}


\end{enumerate}

\end{document}