%----------------------------------------------------------------------------------------
%	PACKAGES AND OTHER DOCUMENT CONFIGURATIONS
%----------------------------------------------------------------------------------------
\DocumentMetadata{
pdfversion=2.0,  
lang=en-US,   
pdfstandard={ua-2, a-4f},
tagging = on, 
tagging-setup={math/setup={mathml-SE}} ,
}
\documentclass[12pt]{article}
\usepackage{amsmath}
\usepackage[extreme]{savetrees}
\usepackage[utf8]{inputenc}
\usepackage{newpxtext}
\usepackage{hyperref}
\usepackage{multicol}
\usepackage{physics}
\usepackage{siunitx}

%%%%%%%%% === Document Configuration === %%%%%%%%%%%%%%

\hypersetup{
pdftitle={MATH 340 -- Homework 4 Spring 2026},
pdfauthor={Ignacio Rojas},
pdfkeywords={fourth homework},%
}

%----------------------------------------------------------------------------------------
%	ARTICLE CONTENTS
%----------------------------------------------------------------------------------------
\begin{document}

\begin{center}
    {\Large MATH 340 -- Homework 4,\quad Spring 2026}
\end{center}

Recall that you must hand in a subset of the problems for which deleting any problem makes the total score less than 10. The maximum possible score on this homework is 10 points. See the syllabus for scoring details. NOTE: All problems from Boyce, Di Prima and Meade are from the \(11^{\text{th}}\)  edition, which is available online: \href{https://www.math.colostate.edu/~liu/MATH340_S26Crd/Textbook_BoyceDiPrimaMeade_11ed.pdf}{https://www.math.colostate.edu/\(\sim\)liu/MATH340\_S26Crd/Textbook\_BoyceDiPrimaMeade\_11ed.pdf}.

\begin{enumerate}

  \item (1+) [2 points] Boyce, di Prima, and Meade chapter 3 section 1 problem 15.
  
  \item (2) [3 points] Boyce, di Prima, and Meade chapter 3 section 1 problem 19.
  
  \item (\textbf{E}) (1+) [2 points] Consider the ODE \(y^{(4)}+18y''+81y=0\). Show that the general solution can be written as \((A+Bt)\cos(3t)+(C+Dt)\sin(3t)\).
  
  \item (1+) [2 points] Boyce, di Prima, and Meade chapter 3 section 2 problems 19 and 20.

  \item (\textbf{E}) (1) [1 point] Verify that \(y_1(t)=e^{2t}\) and \(y_2(t)=e^{3t}\) are solutions to \(y''-5y'+6y=0\) and that their Wronskian \(W[y_1,y_2]\) is never zero.

  \item (\textbf{E}) (1) [1 point] Consider the ODE \(y''+4y'+13y=0\). Verify whether or not the functions \(y_1=e^{-3t}\cos(2t)\) and \(y_2=e^{-3t}\sin(2t)\) form a fundamental set of solutions. 
  
  \item (1) [1 point] Solve the IVP \(W[t^2+1,f(t)]=1\) with \(f(0)=1\).
  
  \item (1+) [2 points] If \(W[f,g] = t3 e^{2t}\) and \(f(t) = t^2\), find \(g(t)\). [Obs: Your answer will involve a non-elementary integral, you may leave your answer in terms of such integral.]
  
  \item (2-) [3 points] Boyce, di Prima, and Meade chapter 3 section 3 problem 13.
  
  \item (2) [3 points] Boyce, di Prima, and Meade chapter 3 section 3 problem 15.
  
  \item (1) [1 point] For which values of \(\lambda\) does the problem 
  \[y''+\lambda y=0,\quad y(0)=y(1)=0\]
  have non-zero solution? Find such solutions.

  \item (1) [1 point] Find a second order linear homogeneous equation whose corresponding characteristic equation has \(3-7i\) as one of its roots.

  \item (2+) [4 points] Boyce, di Prima, and Meade chapter 3 section 4 problem 12.

  \item (1+) [2 points] Boyce, di Prima, and Meade chapter 3 section 5 problem 10.

  \item (2-) [3 points] Boyce, di Prima, and Meade chapter 3 section 5 problem 12.

  
  \item (\textbf{E}) (2) [3 points] Find the general solution to the ODE \(\ddot{x}-4\dot{x}+3x=0\), then find a particular solution to \(\ddot{x}-4\dot{x}+3x=e^t\). Conclude by writing down the general solution of \(\ddot{x}-4\dot{x}+3x=e^t\).
  
  \item (\textbf{E}) (2) [3 points] Consider the ODE \(\ddot{x}-\dot{x}-2x=e^{-t}\). Find the general solution to the corresponding homogeneous equation. Then find a particular solution of the nonhomogenous equation. Conclude by finding the solution of the nonhomogenous equation satisfying \(x(0)=1\) and \(\dot{x}(0)=0\).
  
  \item (2) [3 points] Solve the following initial value problem where the characteristic equation is of degree 3 and higher. 
  \[y^{(3)}+y''-4y'-4y= 8x+ 8 + 6 e^{-x},\quad y(0)=2,\quad y'(0)=-4,\quad y''(0)=12.\]
  [Hint: At least one of its roots is an integer and can be found by inspection.]


\end{enumerate}

\end{document}