
\makeatletter
\ifx \nauthor\undefined
  \def\nauthor{Ignacio Rojas}
\else
\fi

\ifx \nextra \undefined
\ifx \nlang \undefined
\author{Basado en las clases impartidas por \nprofesor \\\small Notas tomadas por \nauthor}
\else
\author{Based on the lectures by \nprofesor \\\small Notes written by \nauthor}
\fi
\else
\author{\nauthor}
\fi
\date{\nterm\ \nyear}

%%%%%%%%%%%%%
%% 1. Pacotes
%%%%%%%%%%%%%

\usepackage{alltt}
\usepackage{amsfonts}
\usepackage{amsmath}
\usepackage{amssymb}
\usepackage{amsthm}
\usepackage{algorithm}
\usepackage[noend]{algpseudocode}
\usepackage{array}
\newcommand\hmmax{0} % default 3
\newcommand\bmmax{0} % default 4 %%tex.se/3676,219310
%\usepackage{bbold}
\usepackage{bm}
\usepackage{booktabs}
%\usepackage{caption}
\usepackage{cancel}
%\usepackage{dsfont}
\usepackage{esint}
\usepackage{fancyhdr}
\usepackage{graphicx}
\usepackage[utf8]{inputenc}
\usepackage{listings}
\usepackage{mathabx}
\usepackage[cal=euler]{mathalfa}
%\usepackage[cal=euler,frak=euler]{mathalfa} % mathcal (JIRR) precisabamos correr initexmf --mkmaps en cmd JCVDG
\usepackage{mathdots}
\usepackage{mathrsfs}
%\usepackage{mathtools}
\usepackage{microtype}
\usepackage{multicol}
\usepackage{multirow}
\usepackage[theoremfont,largesc,tighter,osf]{newpxtext} %JCV Diff
\let\widering\undefined
%\usepackage[bigdelims,vvarbb]{newpxmath} %JCVDG
%por alguna razón esto afectaba las tildes en \min, \lim y demás
%\usepackage{pdflscape}
\usepackage{pgfplots}
\usepackage{physics}
\usepackage{siunitx}
\usepackage{slashed}
%\usepackage{stmaryrd}
%\SetSymbolFont{stmry}{bold}{U}{stmry}{m}{n}
%\usepackage{subfigure}
\usepackage{subcaption}
\usepackage{tabularx}
\usepackage[breakable,skins]{tcolorbox}
\usepackage{textcomp} %%JCVDG
\usepackage{tikz}
\usepackage{tkz-euclide}
\usepackage[normalem]{ulem}
\usepackage[all]{xy}
\usepackage{imakeidx}
\ifx \nlang \undefined
\usepackage[spanish]{babel}
\else\fi 
\usepackage{wrapfig}

%%%%%%%%%%%%%%%%%%%%
%% 2. Document Setup
%%%%%%%%%%%%%%%%%%%%

\ifx \nextra \undefined
    \ifx \nlang \undefined
    \makeindex[intoc, title=Índice Analítico] %Título de índice analítico
    %El índice general es aquel en el que se indican los capítulos, títulos y subtítulos del libro.
    %Índice onomástico es donde aparece el nombre de personas mencionadas en el texto, por orden alfabético con el número de las páginas donde aparecen.
    %El índice analítico se refiere a los temas y conceptos que aparecen en el libro
    \indexsetup{othercode={\fancyhead[LE]{\emph{Índice Analítico}}}}
    \else
    \makeindex[intoc, title=Index] 
    \indexsetup{othercode={\fancyhead[LE]{\emph{Index}}}}
    \fi
  \usepackage[pdftex,
    hidelinks,
    pdfauthor={\nauthor},
    pdfsubject={Notas: \nsiglahead\ \nsemestre-\nyear},
    pdftitle={Semestre \nsemestre\ - \nsigla},
  pdfkeywords={UCR Costa Rica Matem\'aticas Mate \nsemestre\ \nterm\ \nyear\ \nsiglahead}]{hyperref}
  \title{\nsigla\ --- \nsiglahead}
\else
  \usepackage[pdftex,
     hidelinks,
    pdfauthor={\nauthor},
    pdfsubject={\nextra \nsiglahead\ \nsemestre-\nyear},
    pdftitle={Semestre \nsemestre\ - \nsigla},
  pdfkeywords={UCR Costa Rica Matem\'aticas Mate \nsemestre\ \nterm\ \nyear\ \nsiglahead\ \nextra}]{hyperref}

  \title{\nsigla\ --- \nsiglahead \\ {\Large \nextra}}
  \renewcommand\printindex{}
\fi

\pgfplotsset{compat=1.12}


\pagestyle{fancy}
\setlength{\headheight}{15.72pt} %preceding warning said make it at least this


\ifx \nsiglahead \undefined
\def\nsiglahead{\nsigla}
\fi

\lhead{} %%%empty lhead
\rfoot{\thepage}

\ifx \nextra \undefined
  \chead{
    \ifnum\thepage=1
    \else
      \ifx \nlang \undefined
      \textbf{Notas \nsiglahead\ \nsemestre-\nyear}
      \else
      \textbf{Notes \nsiglahead\ \nsemestre-\nyear}
      \fi
    \fi}
  \rhead{}%\firstxmark} % Top right header
\else
%    \chead{
%    \ifnum\thepage=1
%    \else
%      \textbf{Notas \nsiglahead\ \nsemestre-\nyear \ (\nextra)}
%    \fi}
     \chead{
       \textbf{\nextra\ \nsigla\ \nsemestre-\nyear}
     }
     \rhead{
       \textbf{\nauthor}
     }
\fi
\lfoot{}%\lastxmark} % Bottom left footer
\cfoot{} % Bottom center footer

\usetikzlibrary{arrows.meta}
\usetikzlibrary{decorations.markings}
\usetikzlibrary{decorations.pathmorphing}
\usetikzlibrary{positioning}
\usetikzlibrary{fadings}
\usetikzlibrary{intersections}
\usetikzlibrary{cd}

\ifx \nhtml \undefined
\else
  \renewcommand\printindex{}
  \DisableLigatures[f]{family = *}
  \let\Contentsline\contentsline
  \renewcommand\contentsline[3]{\Contentsline{#1}{#2}{}}
  \renewcommand{\@dotsep}{10000}
  \newlength\currentparindent
  \setlength\currentparindent\parindent

  \newcommand\@minipagerestore{\setlength{\parindent}{\currentparindent}}
  \usepackage[active,tightpage,pdftex]{preview}
  \renewcommand{\PreviewBorder}{0.1cm}

  \newenvironment{stretchpage}%
  {\begin{preview}\begin{minipage}{\hsize}}%
    {\end{minipage}\end{preview}}
  \AtBeginDocument{\begin{stretchpage}}
  \AtEndDocument{\end{stretchpage}}

  \newcommand{\@@newpage}{\end{stretchpage}\begin{stretchpage}}

  \let\@real@section\section
  \renewcommand{\section}{\@@newpage\@real@section}
  \let\@real@subsection\subsection
  \renewcommand{\subsection}{\@ifstar{\@real@subsection*}{\@@newpage\@real@subsection}}
\fi
\ifx \ntrim \undefined
\usepackage[shortlabels]{enumitem} %mfw package order matters por savetrees
\else
  \usepackage{geometry}
  \geometry{
    papersize={379pt, 699pt},
    textwidth=345pt,
    textheight=596pt,
    left=17pt,
    top=54pt,
    right=17pt
  }
  \headwidth=345pt
 \usepackage[extreme]{savetrees}
\fi

\ifx \darktheme\undefined
\else
\pagecolor[rgb]{0.2,0.231,0.302}%{0.23,0.258,0.321}
\color[rgb]{1,1,1}
\fi

\ifx \nextra \undefined
\let\@real@maketitle\maketitle
\renewcommand{\maketitle}{\@real@maketitle\begin{center}\begin{minipage}[c]{0.9\textwidth}\centering\footnotesize 
  \ifx \nlang \undefined
  Estas notas no están respaldadas por los profesores y han sido modificadas (a menudo de manera significativa) después de las clases. No están lejos de ser representaciones precisas de lo que realmente se dio en clase y en particular todos los errores son casi seguramente míos.
  \else 
  Please note that these notes were not provided or endorsed by the lecturer and have been significantly altered after the class. They may not accurately reflect the content covered in class and any errors are solely my responsibility.
  \fi
\end{minipage}\end{center}}
\else
\fi

\def\moverlay{\mathpalette\mov@rlay}
\def\mov@rlay#1#2{\leavevmode\vtop{%
   \baselineskip\z@skip \lineskiplimit-\maxdimen
   \ialign{\hfil$\m@th#1##$\hfil\cr#2\crcr}}}
\newcommand{\charfusion}[3][\mathord]{
    #1{\ifx#1\mathop\vphantom{#2}\fi
        \mathpalette\mov@rlay{#2\cr#3}
      }
    \ifx#1\mathop\expandafter\displaylimits\fi}

%%%%%%%%%%%%%%%%%%%%%%%%%%%%%%
%% 2.1 Some internal machinery
%%%%%%%%%%%%%%%%%%%%%%%%%%%%%%

\makeatletter
\renewcommand{\section}{\@startsection{section}{1}{\z@}%
							 {-3.25ex \@plus -1ex \@minus -.2ex}%
							 {1.5ex \@plus.2ex}%
							 {\normalfont\large\bfseries}}
\renewcommand{\subsection}{\@startsection{subsection}{2}{\z@}%
							 {-3.25ex \@plus -1ex \@minus -.2ex}%
							 {1.5ex \@plus .2ex}%
               {\normalfont\normalsize\bfseries}}
\newcommand*{\defeq}{\!\mathrel{\rlap{%
             \raisebox{0.3ex}{$\m@th\cdot$}}%
             \raisebox{-0.3ex}{$\m@th\cdot$}}%
                    =\!}
\makeatother
\ifx\ntrim\undefined
\newcommand{\coursetitle}{\nsigla: \nsiglahead}
\ifx\nextra\undefined
\pagestyle{ruled}
\makeoddhead{ruled}{\coursetitle}{}{\rightmark}
\else\fi
\settypeblocksize{49pc}{37pc}{*}
\setlrmargins{*}{*}{1.2}
\setulmargins{*}{*}{0.8}
\setheadfoot{16pt}{30pt}
\setheaderspaces{*}{1.5pc}{1}
\setmarginnotes{1pt}{1pt}{1pt}
\checkandfixthelayout

\setlength{\unitlength}{3pt}
\setlength{\hfuzz}{1pt}

\setlength{\fboxsep}{6pt}

\setlength{\footskip}{17pt}

\linespread{1.1}
\else\fi
\renewcommand{\cftdotsep}{\cftnodots} %%% no dots in ToC
\setpnumwidth{2em}  %%% width of page-number box in ToC


\newcommand{\stophere}{\relax} %% can be changed to `\endinput'
% \newcommand{\stophere}{\endinput} %% can be changed to `\relax'


\DeclareRobustCommand{\qned}{\ifmmode
  \else \leavevmode\unskip\penalty9999 \hbox{}\nobreak\hfill \fi
  \quad\hbox{\qnedsymbol}}
\newcommand{\qnedsymbol}{$\boxminus$} %% No-proofs end with `\qned'

\DeclareRobustCommand{\qef}{\ifmmode
  \else \leavevmode\unskip\penalty9999 \hbox{}\nobreak\hfill \fi
  \quad\hbox{\qefsymbol}}
\newcommand{\qefsymbol}{$\lozenge$} %% Examples end with `\qef'
\def\enddefn{\qef\endtrivlist}      %% `\qef' automático en defns
\def\endejem{\qef\endtrivlist}      %% `\qef' automático en ejemplos

\newcommand{\hideqed}{\renewcommand{\qed}{}} %% to suppress `\qed'
\newcommand{\hideqef}{\renewcommand{\qef}{}} %% to suppress `\qef'

% \newcommand{\ldbrack}{\ensuremath{[\mskip-2.5mu[}} %% corchetes [[
% \newcommand{\rdbrack}{\ensuremath{]\mskip-2.5mu]}} %% corchetes ]]

\newcommand{\stroke}{\mathbin|}     %% (for `\bbraket' and such)

\newcommand{\rtri}{\blacktriangleright} %% (for `\marker' and such)
\newcommand{\tribar}{|\mkern-2mu|\mkern-2mu|} %% norma triple: |||


%% Formatting changes:

\renewcommand{\labelitemi}{$\diamond$} %% instead of bullets

\renewcommand{\theenumi}{\alph{enumi}}  %% use lowercase letters
\renewcommand{\labelenumi}{\textup{(\theenumi)}} %% inside parentheses

%%%%%%%%%%%%%%
%% 2.2. Colors
%%%%%%%%%%%%%%

\definecolor{MATLABgreen}{RGB}{28,172,0} % color values Red, Green, Blue
\definecolor{MATLABlila}{RGB}{170,55,241}
\definecolor{dankBlue}{RGB}{51,60,77} % color values Red, Green, Blue
\definecolor{dankBlueLite}{RGB}{82,97,125} % color values Red, Green, Blue
\definecolor{celesUCR}{RGB}{0,192,243}
\definecolor{azulUCR}{RGB}{0,93,164}
\definecolor{verdeUCR}{RGB}{109,192,103}
\definecolor{yelloUCR}{RGB}{255,224,106}

%%%%%%%%%%%%%%%%%%%%%%%%%%%
%% 3. Theorems and suchlike
%%%%%%%%%%%%%%%%%%%%%%%%%%%

\ifx\nlang\undefined

\theoremstyle{plain}
\ifx \nextra \undefined
\newtheorem{Th}{Teorema}[section]      %%% Theorem 1.1.1
\newtheorem{Tmon}[Th]{Teoremón}
\newtheorem{Prop}[Th]{Proposición}     %%% Proposition 1.1.2
\newtheorem{Lem}[Th]{Lema}             %%% Lemma 1.1.3
\newtheorem{Cor}[Th]{Corolario}        %%% Corollary 1.1.4
\else
\newtheorem{Th}{Teorema}               %%% Theorem 1.1.1
\newtheorem{Tmon}{Teoremón}
\newtheorem{Prop}{Proposición}         %%% Proposition 1.1.2
\newtheorem{Lem}{Lema}                 %%% Lemma 3
\newtheorem{Cor}{Corolario}            %%% Corollary 4
\fi
\newtheorem*{nonum-Th}{Teorema}        %%% No-numbered Theorem
\newtheorem*{nonum-Cor}{Corolario}     %%% No-numbered Corollary

\theoremstyle{definition}
\ifx \nextra \undefined
\newtheorem{Def}[Th]{Definición}       %%% Definition 1.1.5
\newtheorem{Ex}[Th]{Ejemplo}           %%% Example 1.1.6
\newtheorem{Ej}[Th]{Ejercicio}         %%% Ejercicio 1.1.7
\else
\newtheorem{Def}{Definición}           %%% Definition 5
\newtheorem{Ex}{Ejemplo}               %%% Example 6
\newtheorem{Ej}{Ejercicio}             %%% Ejercicio 7
\fi
\newtheorem{Hec}[Th]{Hecho}            %%% Hecho 1.1.8
\newtheorem*{nonum-Def}{Definición}    %%% No number Definition
\newtheorem*{nonum-Ex}{Ejemplo}        %%% No number Example
\newtheorem*{nonum-Ej}{Ejercicio}      %%% No number Ejercicio
\newtheorem*{nonum-Hec}{Hecho}         %%% No number Fact


\theoremstyle{remark}
\newtheorem{Rmk}[Th]{Observación}      %%%Remark 1.1.9
\newtheorem*{nonum-Rmk}{Observación}         %%% No number Fact
\newtheorem*{Notn}{Notaci\'on}        %% Notaciones
\newtheorem*{Warn}{Advertencia}       %% Advertencias
\newtheorem*{Qn}{Pregunta}            %% Pregunta

\else

\theoremstyle{plain}
\ifx \nextra \undefined
\newtheorem{Th}{Theorem}[section]      %%% Theorem 1.1.1
\newtheorem{Tmon}[Th]{Teoremón}
\newtheorem{Prop}[Th]{Proposition}     %%% Proposition 1.1.2
\newtheorem{Lem}[Th]{Lemma}             %%% Lemma 1.1.3
\newtheorem{Cor}[Th]{Corollary}        %%% Corollary 1.1.4
\else
\newtheorem{Th}{Theorem}               %%% Theorem 1.1.1
\newtheorem{Tmon}{Teoremón}
\newtheorem{Prop}{Proposition}         %%% Proposition 1.1.2
\newtheorem{Lem}{Lemma}                 %%% Lemma 3
\newtheorem{Cor}{Corollary}            %%% Corollary 4
\fi
\newtheorem*{nonum-Th}{Theorem}        %%% No-numbered Theorem
\newtheorem*{nonum-Cor}{Corollary}     %%% No-numbered Corollary

\theoremstyle{definition}
\ifx \nextra \undefined
\newtheorem{Def}[Th]{Definition}       %%% Definition 1.1.5
\newtheorem{Ex}[Th]{Example}           %%% Example 1.1.6
\newtheorem{Ej}[Th]{Exercise}         %%% Exercise 1.1.7
\else
\newtheorem{Def}{Definition}           %%% Definition 5
\newtheorem{Ex}{Example}               %%% Example 6
\newtheorem{Ej}{Exercise}             %%% Exercise 7
\fi
\newtheorem{Hec}[Th]{Fact}            %%% Fact 1.1.8
\newtheorem*{nonum-Def}{Definition}    %%% No number Definition
\newtheorem*{nonum-Ex}{Example}        %%% No number Example
\newtheorem*{nonum-Ej}{Exercise}      %%% No number Exercise
\newtheorem*{nonum-Hec}{Fact}         %%% No number Fact


\theoremstyle{remark}
\newtheorem{Rmk}[Th]{Remark}      %%%Remark 1.1.9
\newtheorem*{nonum-Rmk}{Remark}         %%% No number Fact
\newtheorem*{Notn}{Notation}        %% Notaciones
\newtheorem*{Warn}{Warning}       %% Warnings
\newtheorem*{Qn}{Question}            %% Question

\fi 

\numberwithin{equation}{section}

\setlength{\parindent}{3ex}

% \renewcommand{\labelitemi}{--}
% \renewcommand{\labelitemii}{$\circ$}
% \renewcommand{\labelenumi}{(\roman{*})}

%\let\stdsection\section
%\renewcommand\section{\newpage\stdsection}

\newcommand\qedsym{\hfill\ensuremath{\square}}
% Strike through
\def\st{\bgroup \ULdepth=-.55ex \ULset}

%%%%%%%%% === My T Color Box === %%%%%%%%%%%%%%

\ifx\nlang\undefined
\ifx \darktheme\undefined
\newtcolorbox{ptcb}{
colframe = black,
colback = white,
breakable,
enhanced
}
\newtcolorbox{ptcbp}{
colframe = black,
colback = white,
coltitle = black,
colbacktitle = black!40,
title = Prueba,
breakable,
enhanced
}
\newtcolorbox{ptcbr}{
colframe = blue,
colback = white,
coltitle = blue,
colbacktitle = blue!40,
title = Respuesta,
breakable,
enhanced
}
\else
\newtcolorbox{ptcb}{
colframe = white,
colback = dankBlue,
colupper = white,
breakable,
enhanced
}
\newtcolorbox{ptcbp}{
colframe = white,
colback = dankBlue,
colupper = white,
coltitle = white,
colbacktitle = dankBlueLite,
title = Prueba,
breakable,
enhanced
}
\newtcolorbox{ptcbr}{
colframe = white,
colback = white,
coltitle = blue,
colbacktitle = blue!40,
title = Respuesta,
breakable,
enhanced
}
\fi

\else
\ifx \darktheme\undefined
\newtcolorbox{ptcb}{
colframe = black,
colback = white,
breakable,
enhanced
}
\newtcolorbox{ptcbp}{
colframe = black,
colback = white,
coltitle = black,
colbacktitle = black!40,
title = Proof,
breakable,
enhanced
}
\newtcolorbox{ptcbr}{
colframe = blue,
colback = white,
coltitle = blue,
colbacktitle = blue!40,
title = Answer,
breakable,
enhanced
}
\else
\newtcolorbox{ptcb}{
colframe = white,
colback = dankBlue,
colupper = white,
breakable,
enhanced
}
\newtcolorbox{ptcbp}{
colframe = white,
colback = dankBlue,
colupper = white,
coltitle = white,
colbacktitle = dankBlueLite,
title = Proof,
breakable,
enhanced
}
\newtcolorbox{ptcbr}{
colframe = white,
colback = white,
coltitle = blue,
colbacktitle = blue!40,
title = Answer,
breakable,
enhanced
}
\fi
\fi


%%%%%%%%% === Listings === %%%%%%%%%%%%%%
\lstset{basicstyle=\ttfamily,breaklines=true}

\lstset{language=Matlab,%
    %basicstyle=\color{red},
    breaklines=true,%
    morekeywords={matlab2tikz},
    keywordstyle=\color{blue},%
    morekeywords=[2]{1}, keywordstyle=[2]{\color{black}},
    identifierstyle=\color{black},%
    stringstyle=\color{MATLABlila},
    commentstyle=\color{MATLABgreen},%
    showstringspaces=false,%without this there will be a symbol in the places where there is a space
    numbers=left,%
    numberstyle={\tiny \color{black}},% size of the numbers
    numbersep=9pt, % this defines how far the numbers are from the text
   % emph=[1]{for,end,break,function,if,elseif,else},emphstyle=[1]\color{blue}, %some words to emphasise
    %emph=[2]{word1,word2}, emphstyle=[2]{style},
}

%%%%%%%%%%%%%%%%%%%%%%%%%%
%% 4. Simple abbreviations
%%%%%%%%%%%%%%%%%%%%%%%%%%

%%% Operator names:

\DeclareMathOperator{\area}{area}
\DeclareMathOperator{\card}{card}
\DeclareMathOperator{\ccl}{ccl}
\DeclareMathOperator{\ch}{ch}
\DeclareMathOperator{\cl}{cl}
\DeclareMathOperator{\coker}{coker}
\DeclareMathOperator{\Conv}{Conv}   %%Convex hull
\DeclareMathOperator{\cosec}{cosec}
\DeclareMathOperator{\cosech}{cosech}
\DeclareMathOperator{\covol}{covol}
\DeclareDocumentCommand\curl{}{\operatorname{curl}} 
\DeclareMathOperator{\diag}{diag}
\DeclareMathOperator{\diam}{diam}
\DeclareMathOperator{\Diff}{Diff}
\DeclareDocumentCommand\div{}{\operatorname{div}} 
\DeclareMathOperator{\energy}{energy}
\DeclareMathOperator{\erfc}{erfc}
\DeclareMathOperator{\Ext}{Ext}
\DeclareMathOperator{\fst}{fst}
\DeclareMathOperator{\Fit}{Fit}
\DeclareMathOperator{\gr}{gr}
\DeclareMathOperator{\hcf}{hcf}
\DeclareMathOperator{\id}{id}
\DeclareMathOperator{\Ind}{Ind}
\DeclareMathOperator{\Int}{Int}
\DeclareMathOperator{\Isom}{Isom}
\DeclareMathOperator{\lcm}{lcm}
\DeclareMathOperator{\length}{length}
\DeclareMathOperator{\Lie}{Lie}
\DeclareMathOperator{\like}{like}
\DeclareMathOperator{\Lk}{Lk}
\DeclareMathOperator{\Maps}{Maps}
\DeclareMathOperator{\mcd}{mcd}
\DeclareMathOperator{\mcm}{mcm}
\DeclareMathOperator{\Min}{Min}
\DeclareMathOperator{\orb}{orb}
\DeclareMathOperator{\ord}{ord}
\DeclareMathOperator{\otp}{otp}
\DeclareMathOperator{\pr}{pr}       %% proyector
\DeclareMathOperator{\poly}{poly}
\DeclareMathOperator{\rel}{rel}
\DeclareMathOperator{\Rad}{Rad}
\DeclareMathOperator*{\res}{res}
\DeclareMathOperator{\Ric}{Ric}
\DeclareMathOperator{\rk}{rk}
\DeclareMathOperator{\Rees}{Rees}
\DeclareMathOperator{\Root}{Root}
\DeclareMathOperator{\rot}{rot}         %% rotacional
\DeclareMathOperator{\spn}{span}
\DeclareMathOperator{\St}{St}
\DeclareMathOperator{\supp}{supp}
\DeclareMathOperator{\Syl}{Syl}
\DeclareMathOperator{\Sym}{Sym}
\DeclareMathOperator{\vol}{vol}

% not-math
\newcommand{\bolds}[1]{{\bfseries #1}}
\newcommand{\cat}[1]{\mathsf{#1}}
\newcommand{\ph}{\,\cdot\,}
\newcommand{\term}[1]{\un{#1}\index{#1}}
\newcommand{\phantomeq}{\hphantom{{}={}}}
\newcommand{\ttt}{\texttt}
\newcommand{\red}[1]{\textcolor{red}{#1}}
\newcommand{\prp}[1]{\textcolor{purple}{#1}}
\newcommand{\blu}[1]{\textcolor{azulUCR}{#1}}
\newcommand{\green}[1]{\textcolor{verdeUCR}{#1}}
\newcommand{\yelo}[1]{\textcolor{yelloUCR}{#1}}
\newcommand{\cele}[1]{\textcolor{celesUCR}{#1}}

%functions
\DeclareMathOperator{\sgn}{sgn}
\newcommand*{\Cdot}{{\raisebox{-0.25ex}{\scalebox{1.5}{$\cdot$}}}}      %% cdot más grande
\newcommand{\ind}{\mathbf{1}}       %%%indicator function
\newcommand{\mm}{\mathfrak{m}}      %%%metric


% Greek letters:

\newcommand{\al}{\alpha}                %% short for  \alpha
\newcommand{\bt}{\beta}                 %% short for  \beta
\newcommand{\Dl}{\Delta}                %% short for  \Delta
\newcommand{\dl}{\delta}                %% short for  \delta
\newcommand{\eps}{\varepsilon}          %% short for  \varepsilon
\newcommand{\Ga}{\Gamma}                %% short for  \Gamma
\newcommand{\ga}{\gamma}                %% short for  \gamma
\newcommand{\kp}{\kappa}                %% short for  \kappa
\newcommand{\La}{\Lambda}               %% short for  \Lambda
\newcommand{\la}{\lambda}               %% short for  \lambda
\newcommand{\Om}{\Omega}                %% short for  \Omega
\newcommand{\om}{\omega}                %% short for  \omega
\newcommand{\Sg}{\Sigma}                %% short for  \Sigma
\newcommand{\sg}{\sigma}                %% short for  \sigma
\newcommand{\Te}{\Theta}                %% short for  \Theta
\newcommand{\te}{\theta}                %% short for  \theta
\newcommand{\ups}{\upsilon}             %% short for  \upsilon
\newcommand{\vf}{\varphi}               %% short for  \varphi
\newcommand{\ze}{\zeta}                 %% short for  \zeta
\newcommand{\vsg}{\varsigma}            %% short for  \varsigma
\newcommand{\vte}{\vartheta}            %% short for  \vartheta

%Boldface letters

\newcommand{\bA}{\mathbb{A}}        %% antisimetrizador
\newcommand{\bB}{\mathbb{B}}        %% bola unitaria
\newcommand{\bC}{\mathbb{C}}    %%% números complejos
\newcommand{\bCP}{\mathbb{CP}}  %%% espacio proyectivo complejo
\newcommand{\bD}{\mathbb{D}}        %% Poincaré disk
\newcommand{\bE}{\mathbb{E}}
\newcommand{\bF}{\mathbb{F}}        %% un cuerpo
\newcommand{\bH}{\mathbb{H}}        %% cuaterniones
\newcommand{\bI}{\mathbb{I}}        %% ideal de zeros
\newcommand{\bK}{\mathbb{K}}            %% ein korper
\newcommand{\bN}{\mathbb{N}}    %%% números naturales
\newcommand{\bP}{\mathbb{P}}        %% números enteros positivos
\newcommand{\bQ}{\mathbb{Q}}    %%% números racionales
\newcommand{\bR}{\mathbb{R}}    %%% números reales
\newcommand{\bRP}{\mathbb{RP}}  %%% espacio proyectivo real
\newcommand{\bS}{\mathbb{S}}    %%% esfera
\newcommand{\bT}{\mathbb{T}}        %% círculo o toro
\newcommand{\bV}{\mathbb{V}}        %% lugar geométrico de ceros
\newcommand{\bZ}{\mathbb{Z}}    %%% números enteros

%Script letters:

\newcommand{\cA}{\mathcal{A}}           %% formas diferenciales
\newcommand{\cB}{\mathcal{B}}           %% una base vectorial
\newcommand{\cC}{\mathcal{C}}           %% otra base vectorial
\newcommand{\cD}{\mathcal{D}}           %% funciones de prueba
\newcommand{\cE}{\mathcal{E}}           %% un modulo proyectivo
\newcommand{\cF}{\mathcal{F}}           %% espacio de Fock
\newcommand{\cG}{\mathcal{G}}           %% funtor de Gelfand
\newcommand{\cH}{\mathcal{H}}           %% espacio de Hilbert
\newcommand{\cI}{\mathcal{I}}           %% un funtor de inclusion
\newcommand{\cJ}{\mathcal{J}}           %% otro funtor
\newcommand{\cK}{\mathcal{K}}           %% otro espacio de Hilbert
\newcommand{\cL}{\mathcal{L}}           %% operadores lineales
\newcommand{\cM}{\mathcal{M}}           %% multiplicadores
\newcommand{\cN}{\mathcal{N}}           %% funciones nulas
\newcommand{\cO}{\mathcal{O}}           %% funciones de crec-to lento
\newcommand{\cP}{\mathcal{P}}           %% una particion
\newcommand{\cR}{\mathcal{R}}           %% funciones representativas
\newcommand{\cQ}{\mathcal{Q}}           %% otra particion
\newcommand{\cS}{\mathcal{S}}           %% funciones de Schwartz
\newcommand{\cT}{\mathcal{T}}           %% una topologia
\newcommand{\cU}{\mathcal{U}}           %% cubrimiento abierto
\newcommand{\cV}{\mathcal{V}}           %% vecindarioas
\newcommand{\cW}{\mathcal{W}}           %% grupo de Weyl
\newcommand{\cZ}{\mathcal{Z}}           %% topología de Zariski

%%% Fraktur letters:

\newcommand{\gA}{\mathfrak{A}}      %% un atlas
\newcommand{\g}{\mathfrak{g}}       %% un álgebra de Lie
\newcommand{\gB}{\mathfrak{B}}      %% otro atlas
\newcommand{\ggl}{\mathfrak{gl}}    %% álg de Lie general lineal
\newcommand{\gsl}{\mathfrak{sl}}    %% álg de Lie especial lineal
\newcommand{\gso}{\mathfrak{so}}    %% álg de Lie especial ortogonal
\newcommand{\gsu}{\mathfrak{su}}    %% álg de Lie especial unitaria
\newcommand{\gX}{\mathfrak{X}}      %% campos vectoriales

%%% Roman letters:

\newcommand{\dR}{\mathrm{dR}}       %% cohomología de de Rham
\newcommand{\rGL}{\mathrm{GL}}      %% grupo general lineal
\newcommand{\rO}{\mathrm{O}}        %% grupo ortogonal
\newcommand{\rSL}{\mathrm{SL}}      %% grupo especial lineal
\newcommand{\rSO}{\mathrm{SO}}      %% grupo ortogonal especial
\newcommand{\rSp}{\mathrm{Sp}}      %% grupo simpléctico
\newcommand{\rSU}{\mathrm{SU}}      %% grupo unitario especial
\newcommand{\rU}{\mathrm{U}}        %% grupo unitario
\newcommand{\rUH}{\mathrm{UH}}      %% cuaterniones unitarias
\newcommand{\rT}{\mathrm{T}}        %% grupo triangular

% Sanserif letters:

\newcommand{\sA}{\mathsf{A}}            %% algebras de Lie A_n
\newcommand{\sB}{\mathsf{B}}            %% grupo como categoria
\newcommand{\sC}{\mathsf{C}}            %% una categoria
\newcommand{\sD}{\mathsf{D}}            %% otra categoria
\newcommand{\sE}{\mathsf{E}}            %% otra categoria mas
\newcommand{\sF}{\mathsf{F}}            %% algebra de Lie F_4
\newcommand{\sG}{\mathsf{G}}            %% algebra de Lie G_2
\newcommand{\sJ}{\mathsf{J}}            %% un poset
\newcommand{\sK}{\mathsf{K}}            %% un poset
\newcommand{\sL}{\mathcal{L}}           %% derivada de Lie
\newcommand{\sN}{\mathsf{N}}            %% categoría con objetos \bN
\newcommand{\sT}{\mathsf{T}}            %% transpuesta

%%% Boldface letters:

\bmdefine{\CC}{C}                       %% C negrilla
\bmdefine{\cc}{c}
%\bmdefine{\dd}{d}                       %% d negrilla
\bmdefine{\ee}{e}                       %% vector e
\bmdefine{\eeps}{\varepsilon}           %% basic form \eps
\bmdefine{\FF}{F}                       %% vector F
\bmdefine{\ff}{f}                       %% vector f
\bmdefine{\ii}{i}                       %% cuaternion i
\bmdefine{\jj}{j}                       %% cuaternion j
\bmdefine{\kk}{k}                       %% cuaternion k
\bmdefine{\lla}{\lambda}                %% sucesion \la
\bmdefine{\mmu}{\mu}                    %% sucesion \mu
\bmdefine{\pp}{p}                       %% vector p
\bmdefine{\qq}{q}                       %% vector q
\bmdefine{\rr}{r}                       %% vector r
\bmdefine{\ssg}{\sigma}                 %% vector \sg
%\bmdefine{\sss}{s}
%\bmdefine{\ttt}{t}
\bmdefine{\VV}{V}                       %% V negrilla
\bmdefine{\xx}{x}                       %% sucesion x
\bmdefine{\xxi}{\xi}                    %% vector \xi
\bmdefine{\yy}{y}                       %% sucesion y
\bmdefine{\zz}{z}                       %% sucesion z

% Matrix groups
\DeclareMathOperator{\GL}{GL}   %%% grupo general lineal
\DeclareMathOperator{\Or}{O}    %%% grupo ortogonal
\DeclareMathOperator{\PGL}{PGL} %%% grupo proyectivo lineal
\DeclareMathOperator{\PSL}{PSL} %%% grupo proyectivo lineal especial
\DeclareMathOperator{\PSO}{PSO} %%% grupo proyectivo ortogonal
\DeclareMathOperator{\PSU}{PSU} %%% grupo proyectivo unitario
\DeclareMathOperator{\SL}{SL}   %%% grupo especial lineal
\DeclareMathOperator{\SO}{SO}   %%% grupo especial ortogonal
\DeclareMathOperator{\SU}{SU}   %%% grupo especial unitario

% Numericc
\newcommand{\argmin}{\text{argm\'in}}
\DeclareMathOperator{\dof}{dof}

%% Brackets
\newcommand{\conj}[1]{\left\lbrace#1\right\rbrace}
\newcommand{\bonj}[1]{\left\lbrack#1\right\rbrack}
\newcommand{\obonj}[1]{\left\rbrack#1\right\lbrack}
\newcommand{\rbonj}[1]{\left\rbrack#1\right\rbrack}
\newcommand{\lbonj}[1]{\left\lbrack#1\right\lbrack}
\newcommand{\snm}[1]{\|#1\|}           %small norma
\newcommand{\nm}[1]{\left\|#1\right\|} %norma pegadita
\newcommand{\pnm}[1]{\biggl|\biggl|#1\biggr|\biggr|}
\let\oldvec=\vec
\renewcommand{\vec}[1]{\mathbf{#1}}
\newcommand\quot[2]{
        \mathchoice
            {% \displaystyle
                \text{\raise1ex\hbox{$#1$}\Big/\lower1ex\hbox{$#2$}}%
            }
            {% \textstyle
                {^{ #1}/_{ #2}}
            }
            {% \scriptstyle
                {^{ #1}/_{ #2}}
            }
            {% \scriptscriptstyle
                {^{ #1}/_{ #2}}
            }
    }
%\newcommand*\quot[2]{{^{\textstyle #1}\big/_{\textstyle #2}}}
\newcommand*\squot[2]{{^{ #1}/_{ #2}}}%%%small quotient
\newcommand{\multinom}[2]{\ensuremath{\left(\kern-.3em\left(\genfrac{}{}{0pt}{}{#1}{#2}\right)\kern-.3em\right)}}

% Probability
\DeclareMathOperator{\Bernoulli}{Bernoulli}
\DeclareMathOperator{\betaD}{beta}
\DeclareMathOperator{\bias}{bias}
\DeclareMathOperator{\binomial}{binomial}
\DeclareMathOperator{\corr}{corr}
\DeclareMathOperator{\cov}{cov}
\DeclareMathOperator{\gammaD}{gamma}
\DeclareMathOperator{\mse}{mse}
\DeclareMathOperator{\multinomial}{multinomial}
\DeclareMathOperator{\Poisson}{Poisson}
\DeclareMathOperator{\Var}{Var}     %%%variance
\DeclareMathOperator{\Cov}{Cov}     %%%Covariance
\renewcommand{\mid}{\;\ifnum\currentgrouptype=16 \middle\fi|\;}

% Combinatorics
\DeclareMathOperator{\ins}{ins}   % insertion tableaux
\DeclareMathOperator{\rw}{rw}     % reading word
\DeclareMathOperator{\rect}{rect} % rectification of young tableau
\DeclareMathOperator{\sh}{sh}     % shape of young tableau
\DeclareMathOperator{\std}{std}   % standarization
\DeclareMathOperator{\Frob}{Frob} % Frobenius map

% Algebra
\DeclareMathOperator{\Ad}{Ad}       %% acción adjunta
\DeclareMathOperator{\adj}{adj}
\DeclareMathOperator{\Ann}{Ann}     %% aniquilador o anulador de módulos
\DeclareMathOperator{\Ass}{Ass}     %% ideales asociados
\DeclareMathOperator{\Aut}{Aut}
\DeclareMathOperator{\Bl}{\mathcal{B}\!\ell}       %% blowup de un espacio
\DeclareMathOperator{\Char}{char}
\DeclareMathOperator{\codim}{codim}
\DeclareMathOperator{\disc}{disc}
\DeclareMathOperator{\dom}{dom}
\DeclareMathOperator{\End}{End}     %%%space of endomorphisms
\DeclareMathOperator{\Fix}{Fix}
\DeclareMathOperator{\Frac}{Frac}
\DeclareMathOperator{\Gal}{Gal}
\DeclareMathOperator{\gen}{gen}     %%%set generated by...
\DeclareMathOperator{\Hom}{Hom}
\DeclareMathOperator{\image}{image}
\DeclareMathOperator{\Mor}{Mor}
\DeclareMathOperator{\Nil}{Nil}
\DeclareMathOperator{\Orb}{Orb}
\DeclareMathOperator{\Pic}{Pic}     %%% grupo de Picard 
\DeclareMathOperator{\Quot}{Quot}
\DeclareMathOperator{\Spec}{Spec}
\DeclareMathOperator{\Stab}{Stab}
\DeclareMathOperator{\Taut}{Taut}

% Analysis
\DeclareMathOperator*{\esssup}{ess\hspace{0.5mm}sup}
\DeclareMathOperator*{\essinf}{ess\hspace{0.5mm}inf}
%\DeclareMathOperator{\Int}{Int}     %%%interior vacilon funcional

\newcommand{\loc}{\text{loc}}
\newcommand{\LB}{\cL_\cB}           %%%bounded linear operator

% Logic
\newcommand{\cleq}{\preccurlyeq}
\newcommand{\cgeq}{\succcurlyeq}

% Others
\DeclareMathOperator{\ebal}{ev}     %%%evalutation
\newcommand{\bigcupdot}{\charfusion[\mathop]{\bigcup}{\Cdot}} %%JCVDG
%\renewcommand{\bigcupdot}{\charfusion[\mathop]{\bigcup}{\Cdot}}
\newcommand{\cupdot}{\charfusion[\mathbin]{\cup}{\Cdot}}
\newcommand{\exterior}{\mathchoice{{\textstyle\bigwedge}}{{\bigwedge}}{{\textstyle\wedge}}{{\scriptstyle\wedge}}}
\newcommand{\hol}{\mathfrak{hol}}
\newcommand{\Id}{\mathrm{Id}}
\newcommand{\lie}[1]{\mathfrak{#1}}
\newcommand{\qeq}{\mathrel{``{=}"}}
\newcommand{\wsto}{\stackrel{\mathrm{w}^*}{\to}}
\newcommand{\wt}{\mathrm{wt}}

%\let\Im\relax
%\let\Re\relax

%%% Shorter symbol names:

\newcommand{\bull}{{\scriptstyle\bullet}}  %% vertice en figuras
\newcommand{\del}{\partial}             %% short for  \partial
\newcommand{\downto}{\downarrow}        %% limite a la derecha
\newcommand{\dsp}{\displaystyle}        %% despliegue en texto
\renewcommand{\geq}{\geqslant}          %% mayor o igual (variante)
\newcommand{\hookto}{\hookrightarrow}     %% inclusion arrow
\newcommand{\isom}{\simeq}              %% isomorfismo
\renewcommand{\l}{\ell}                   %% ele cursiva
\renewcommand{\leq}{\leqslant}          %% menor o igual (variante)
\newcommand{\less}{\setminus}           %% set difference
\newcommand{\otto}{\leftrightarrow}     %% bijection
\newcommand{\ox}{\otimes}               %% producto tensorial
\newcommand{\rt}{\triangleleft}         %% un orden parcial
\newcommand{\rteq}{\trianglelefteq}     %% normal subgroup
\newcommand{\up}{{\mathord{\uparrow}}}  %% espinor `up'
\newcommand{\upto}{\uparrow}            %% left hand limit
\newcommand{\w}{\wedge}                 %% producto exterior
\newcommand{\wto}{\rightharpoonup}      %% convergencia debil
\newcommand{\x}{\times}                 %% producto vectorial
\renewcommand{\.}{\Cdot}                %% producto escalar
\renewcommand{\:}{\mathbin{:}}          %% colon in  f: A -> B
\newcommand{\into}{\rightarrowtail}     %% injection arrow
\newcommand{\lr}{\dashv}                %% adjunction
\newcommand{\lt}{\triangleright}        %% a left action
\newcommand{\lteq}{\trianglerighteq}    %% normal supergroup
\newcommand{\nb}{\nabla}                %% homomorfismo de suma
\newcommand{\nisom}{\not\simeq}         %% negacion de isomorfismo
%\newcommand{\oast}{\circledast}         %% variante de * (ya existe en stmaryrd)
\newcommand{\onto}{\twoheadrightarrow}  %% surjection arrow
\newcommand{\opp}{\circ}                %% objeto opuesto
\newcommand{\ottto}{\longleftrightarrow} %% bijection in display
\newcommand{\pullb}{\lrcorner}          %% simbolo de pullback
\newcommand{\pushf}{\ulcorner}          %% simbolo de pushout
\newcommand{\rx}{\rtimes}               %% producto semidirecto
\newcommand{\To}{\Rightarrow}           %% entre funtores
\newcommand{\tofro}{\rightleftarrows}   %% pair of opposed maps
\newcommand{\toto}{\rightrightarrows}   %% pair of parallel maps

\renewcommand{\2}{\flat}                  %% marcador de sucesiones
\newcommand{\3}{\sharp}                 %% marcador de sucesiones
\newcommand{\4}{\natural}               %% marcador de morfismos
% \newcommand{\5}{\diamond}               %% for roots of trees
% \newcommand{\7}{\dagger}                %% adjunto de operador
\newcommand{\8}{\bullet}                %% anonymous degree

%%% Useful abbreviations:

\newcommand{\Coo}{\cC^\infty}         %% funciones suaves
\newcommand{\ctr}{\mathbin{\lrcorner\,}} %% contraction symbol
\newcommand{\nbf}{{\vec\nabla}}     %% short for  \vec\nabla

\newcommand{\as}{\quad\text{cuando}\enspace} %% `cuando' en límites
\newcommand{\bCoo}{{\bC_\infty}}    %% esfera de Riemann
% \newcommand{\bRoo}{{\bR_\infty}}    %% círculo real extendido

%%% Repeated relations:

\newcommand{\cupycup}{\cup\cdots\cup} %% unión repetida
\newcommand{\capycap}{\cap\cdots\cap} %% intersección repetida
\newcommand{\sys}{\subset\cdots\subset}%% subconjunto propio repetido
\newcommand{\subysub}{\subseteq\cdots\subseteq} %%subconjunto repetido
\newcommand{\oxyox}{\otimes\cdots\otimes} %% prod tensorial repetido
\newcommand{\wyw}{\wedge\cdots\wedge} %% producto exterior repetido
\newcommand{\opyop}{\oplus\cdots\oplus} %% suma directa repetida
\newcommand{\xyx}{\times\cdots\times} %% producto directo repetido

%%% Arrows with riders:

\newcommand{\longto}{\mathop{\longrightarrow}\limits}

%%% Small fractions in displays:

\newcommand{\half}{{\mathchoice{\nhalf}{\thalf}{\shalf}{\shalf}}} %%display text script script^2
\newcommand{\happi}{{\tfrac{\pi}{2}}} %% small fraction  \pi/2
\newcommand{\quarter}{\tfrac{1}{4}} %% small fraction  1/4
\newcommand{\nhalf}{\frac{1}{2}}
\newcommand{\shalf}{{\scriptstyle\frac{1}{2}}} %% tiny fraction 1/2
\newcommand{\thalf}{{\tfrac{1}{2}}} %% small fraction  1/2
\renewcommand{\third}{\tfrac{1}{3}}   %% small fraction  1/3 %Hay que renew porque mathabx toma second y third como x'' y x''' por ejemplo

\newcommand{\ihalf}{{\tfrac{i}{2}}} %% small fraction  i/2

%%%%%%%%%%%%%%%%%%%%%%%%%%%%%
%% 5. Commands with arguments
%%%%%%%%%%%%%%%%%%%%%%%%%%%%%

%%% Accent-like commands, abbreviated:

\newcommand{\ov}{\overline}        %% short for  \overline
\newcommand{\un}{\underline}       %% short for  \underline
\newcommand{\wh}{\widehat}          %% short for  \widehat

%%% Separate words in displays:

\newcommand{\word}[1]{\quad\text{#1}\quad} %% texto intercalado

%%% Webpage locator:

\newcommand{\zelda}[1]{$\langle${\footnotesize\texttt{#1}}$\rangle$}

%% Symbol placement:

\newcommand{\pre}[1]{{}^{#1\!}} %% upper left exponent

%%% Proof-part labels:

\newcommand{\Adiff}[2]{\ensuremath{\Ad\,(\mathrm{#1})\Longleftrightarrow
    (\mathrm{#2})}:\enspace}
\newcommand{\Adimp}[2]{\ensuremath{\Ad\,(\mathrm{#1})\Longrightarrow
    (\mathrm{#2})}:\enspace}
\newcommand{\Adit}[1]{\ensuremath{\Ad\,(\mathrm{#1})}:\enspace}

%%% Enclose one argument with delimiters:

\newcommand{\bool}[1]{\llbracket#1\rrbracket} %% condición booleana
\newcommand{\combo}[1]{\operatorname{co}(#1)} %% convex combo
\newcommand{\lin}[1]{\operatorname{lin}\langle#1\rangle} %% `span'
\newcommand{\set}[1]{\{\,#1\,\}}    %% set notation

\newcommand{\floor}[1]{\lfloor#1\rfloor} %% mayor entero <= x
\newcommand{\Set}[1]{\biggl\{\,#1\,\biggr\}} %% set notation (large)
\newcommand{\roof}[1]{\lceil#1\rceil} %% menor entero >= x
\newcommand{\genr}[1]{\left\langle #1\right\rangle}     %% grupo generado por #1

%%% Asides:

\newcommand{\aside}[1]{$\llbracket$\,#1\,$\rrbracket$} % nota lateral
\ifx \nlang \undefined
\newcommand{\hint}[1]{$\llbracket$\,In\-di\-ca\-ci\'on: #1\,$\rrbracket$}
\else
\newcommand{\hint}[1]{$\llbracket$\,Hint: #1\,$\rrbracket$}
\fi 


%%% Matrices:

\newcommand{\twobyone}[2]{\begin{pmatrix} %% 2 x 1 matrix
   #1 \\ #2 \end{pmatrix}}
\newcommand{\twobytwo}[4]{\begin{pmatrix} %% 2 x 2 matrix
   #1 & #2 \\ #3 & #4 \end{pmatrix}}
\newcommand{\twobythree}[6]{\begin{pmatrix} %% 2 x 3 matrix
    #1 & #2 & #3\\ #4 & #5 & #6 \end{pmatrix}}
\newcommand{\threebyone}[3]{\begin{pmatrix} %% 3 x 1 matrix
   #1 \\ #2 \\ #3 \end{pmatrix}}
\newcommand{\threebythree}[9]{\begin{pmatrix} %% 3 x 3 matrix
   #1 & #2 & #3 \\ #4 & #5 & #6 \\ #7 & #8 & #9 \end{pmatrix}}
\newcommand{\fourbyone}[4]{\begin{pmatrix} %% 2 x 1 matrix
   #1 \\ #2 \\ #3 \\ #4 \end{pmatrix}}
%\newcommand{\fourbyfour}[16]{\begin{pmatrix} %% 4 x 4 matrix
%  #1 & #2 & #3 & #4\\ #5 & #6 & #7 & #8 \\ #9 & #10 & #11 & #12 \\ #13 & #14 & #15 & #16 \end{pmatrix}}
\newcommand{\nbyn}[9]{\begin{pmatrix} %% 4 x 4 matrix with prefilled entries
  #1 & #2 & \cdots & #3\\ #4 & #5 & \cdots & #6 \\ \vdots & \vdots & \ddots & \vdots \\ #7 & #8 & \cdots & #9 \end{pmatrix}}

%%%%%%%%%%%%%%%%%%%%%%%%%%%%
%% 6. Hyphenation exceptions
%%%%%%%%%%%%%%%%%%%%%%%%%%%%

\hyphenation{auto-va-lor auto-va-lo-res auto-vec-tor auto-vec-to-res
car-di-na-li-dad ce-rra-da ce-rra-do ce-rra-das ce-rra-dos cons-tan-te
cons-tan-tes cons-truc-ci cons-truir con-ti-nua con-ti-nua-mente
con-ti-nuas con-ti-nui-dad con-ti-nuo con-ti-nuos co-rres-pon-den-cia
co-rres-pon-de co-rres-pon-den co-rres-pon-dien-te
co-rres-pon-dien-tes co-va-rian-te cual-quier cual-quiera
cu-bri-mien-to desa-rro-lla-do desa-rro-llar des-pu dia-go-nal
dia-go-na-les di-fe-ren-cia-ble di-fe-ren-cia-bles di-fe-ren-cial
di-fe-ren-cia-les di-fe-ren-te di-fe-ren-tes dis-cre-ta dis-cre-tas
dis-cre-to dis-cre-tos di-vi-si-bi-li-dad di-vi-si-ble ele-men-tal
ele-men-ta-les ele-men-to ele-men-tos equi-va-len-cia equi-va-lente
equi-va-lentes equi-va-rian-te equi-va-rian-tes eu-cli-dia-na
eu-cli-dia-nas eu-cli-dia-no eu-cli-dia-nos Fi-gu-ra Gal-ois
gal-oi-sia-na ge-ne-rada ge-ne-rado ge-ne-ra-dor ge-ne-ra-do-res
ge-ne-ral ge-ne-ra-les ge-ne-ra-li-dad ge-ne-ra-li-za ge-ne-ra-li-zan
ge-ne-ran ge-ne-rar geo-me-tr geo-me-try Ha-da-mard ho-meo-mor-fis-mo
ho-meo-mor-fo idea-les in-de-pen-dien-te in-de-pen-dien-tes
in-va-rian-cia in-va-rian-te in-va-rian-tes li-ne-a-les
li-ne-al-men-te ma-ne-ra me-dian-te mo-der-no nin-gu-no nues-tra
nues-tro nu-me-ra-ble ope-ra-ci ope-ra-cio-nes ope-ra-dor
ope-ra-do-res or-to-go-nal par-ti-cu-lar pro-ce-di-mien-to pro-duc-to
pro-duc-tos pro-pie-dad pro-pie-da-des pro-po-si-ci re-fe-ren-cia
re-fle-xi-va re-fle-xi-vas re-fle-xi-vo re-fle-xi-vos re-so-lu-ble
res-pec-ti-va-men-te res-pec-ti-vo res-pec-ti-vos res-pec-to
sa-tis-fa-ce sepa-ra-ble sepa-ra-bles si-guien-te si-guien-tes
subes-pa-cio subes-pa-cios te-dra-edro te-tra-edros tri-vial
tri-via-les uti-lidad va-lo-res va-ria-ble va-ria-bles va-rie-dad
va-rie-da-des ve-cin-da-rio ve-cin-da-rios vec-to-rial vec-to-ria-les
vice-versa}


%%% TikZ arrows and such

\pgfarrowsdeclarecombine{twolatex'}{twolatex'}{latex'}{latex'}{latex'}{latex'}
\tikzset{->/.style = {decoration={markings,
                                  mark=at position 1 with {\arrow[scale=2]{latex'}}},
                      postaction={decorate}}}
\tikzset{<-/.style = {decoration={markings,
                                  mark=at position 0 with {\arrowreversed[scale=2]{latex'}}},
                      postaction={decorate}}}
\tikzset{<->/.style = {decoration={markings,
                                   mark=at position 0 with {\arrowreversed[scale=2]{latex'}},
                                   mark=at position 1 with {\arrow[scale=2]{latex'}}},
                       postaction={decorate}}}
\tikzset{->-/.style = {decoration={markings,
                                   mark=at position #1 with {\arrow[scale=2]{latex'}}},
                       postaction={decorate}}}
\tikzset{-<-/.style = {decoration={markings,
                                   mark=at position #1 with {\arrowreversed[scale=2]{latex'}}},
                       postaction={decorate}}}
\tikzset{->>/.style = {decoration={markings,
                                  mark=at position 1 with {\arrow[scale=2]{latex'}}},
                      postaction={decorate}}}
\tikzset{<<-/.style = {decoration={markings,
                                  mark=at position 0 with {\arrowreversed[scale=2]{twolatex'}}},
                      postaction={decorate}}}
\tikzset{<<->>/.style = {decoration={markings,
                                   mark=at position 0 with {\arrowreversed[scale=2]{twolatex'}},
                                   mark=at position 1 with {\arrow[scale=2]{twolatex'}}},
                       postaction={decorate}}}
\tikzset{->>-/.style = {decoration={markings,
                                   mark=at position #1 with {\arrow[scale=2]{twolatex'}}},
                       postaction={decorate}}}
\tikzset{-<<-/.style = {decoration={markings,
                                   mark=at position #1 with {\arrowreversed[scale=2]{twolatex'}}},
                       postaction={decorate}}}

\tikzset{circ/.style = {fill, circle, inner sep = 0, minimum size = 3}}
\tikzset{scirc/.style = {fill, circle, inner sep = 0, minimum size = 1.5}}
\tikzset{mstate/.style={circle, draw, blue, text=black, minimum width=0.7cm}}

\tikzset{eqpic/.style={baseline={([yshift=-.5ex]current bounding box.center)}}}
\tikzset{commutative diagrams/.cd,cdmap/.style={/tikz/column 1/.append style={anchor=base east},/tikz/column 2/.append style={anchor=base west},row sep=tiny}}

\definecolor{mblue}{rgb}{0.2, 0.3, 0.8}
\definecolor{morange}{rgb}{1, 0.5, 0}
\definecolor{mgreen}{rgb}{0.1, 0.4, 0.2}
\definecolor{mred}{rgb}{0.5, 0, 0}

\def\drawcirculararc(#1,#2)(#3,#4)(#5,#6){%
    \pgfmathsetmacro\cA{(#1*#1+#2*#2-#3*#3-#4*#4)/2}%
    \pgfmathsetmacro\cB{(#1*#1+#2*#2-#5*#5-#6*#6)/2}%
    \pgfmathsetmacro\cy{(\cB*(#1-#3)-\cA*(#1-#5))/%
                        ((#2-#6)*(#1-#3)-(#2-#4)*(#1-#5))}%
    \pgfmathsetmacro\cx{(\cA-\cy*(#2-#4))/(#1-#3)}%
    \pgfmathsetmacro\cr{sqrt((#1-\cx)*(#1-\cx)+(#2-\cy)*(#2-\cy))}%
    \pgfmathsetmacro\cA{atan2(#2-\cy,#1-\cx)}%
    \pgfmathsetmacro\cB{atan2(#6-\cy,#5-\cx)}%
    \pgfmathparse{\cB<\cA}%
    \ifnum\pgfmathresult=1
        \pgfmathsetmacro\cB{\cB+360}%
    \fi
    \draw (#1,#2) arc (\cA:\cB:\cr);%
}
\newcommand\getCoord[3]{\newdimen{#1}\newdimen{#2}\pgfextractx{#1}{\pgfpointanchor{#3}{center}}\pgfextracty{#2}{\pgfpointanchor{#3}{center}}}

\newcommand\qedshift{\vspace{-17pt}}
\newcommand\fakeqed{\pushQED{\qed}\qedhere}

\def\Xint#1{\mathchoice
   {\XXint\displaystyle\textstyle{#1}}%
   {\XXint\textstyle\scriptstyle{#1}}%
   {\XXint\scriptstyle\scriptscriptstyle{#1}}%
   {\XXint\scriptscriptstyle\scriptscriptstyle{#1}}%
   \!\int}
\def\XXint#1#2#3{{\setbox0=\hbox{$#1{#2#3}{\int}$}
     \vcenter{\hbox{$#2#3$}}\kern-.5\wd0}}
\def\ddashint{\Xint=}
\def\dashint{\Xint-}

\newcommand\separator{{\centering\rule{2cm}{0.2pt}\vspace{2pt}\par}}

\newenvironment{own}{\color{gray!70!black}}{}

\newcommand\makecenter[1]{\raisebox{-0.5\height}{#1}}

\mathchardef\mdash="2D

\newenvironment{significant}{\begin{center}\begin{minipage}{0.9\textwidth}\centering\em}{\end{minipage}\end{center}}
\DeclareRobustCommand{\rvdots}{%
  \vbox{
    \baselineskip4\p@\lineskiplimit\z@
    \kern-\p@
    \hbox{.}\hbox{.}\hbox{.}
  }}
\DeclareRobustCommand\tph[3]{{\texorpdfstring{#1}{#2}}}
\def\BState{\State\hskip-\ALG@thistlm}

\makeatother 