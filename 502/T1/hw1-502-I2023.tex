\documentclass[12pt]{memoir}

\def\nsemestre {I}
\def\nterm {Spring}
\def\nyear {2023}
\def\nprofesor {Maria Gillespie}
\def\nsigla {MATH502}
\def\nsiglahead {Combinatorics 2}
\def\nextra {HW1}
\def\nlang {ENG}
\input{../../headerVarillyDiff}
\usepackage{youngtab}

\begin{document}

\begin{Ej}[Exercise 1]
    Show that every polynomial in two variables $x,y$ can be written uniquely as a sum of a
    (two variable) symmetric polynomial and a (two variable) antisymmetric polynomial.
\end{Ej}

\begin{ptcbr}
Suppose $p\in\bQ[x,y]$, then 
$$p(x,y)=\half\left(p(x,y)+p(y,x)\right)+\half\left(p(x,y)-p(y,x)\right).$$
Call the summands $s(x,y)$ and $a(x,y)$ respectively. We have that $s$ is a symmetric polynomial while $a$ is antisymmetric:
$$\begin{cases}
    s(y,x)=\half\left(p(y,x)+p(x,y)\right)=\half\left(p(x,y)+p(y,x)\right)=s(x,y),\\
    a(y,x)=\half\left(p(y,x)-p(x,y)\right)=-\half\left(p(x,y)-p(y,x)\right)=-a(x,y).
\end{cases}$$
Now suppose that there exist $s_1,s_2,a_1,a_2\in\bQ[x,y]$, symmetric and antisymmetric polynomials respectively such that 
\begin{align*}
    p(x,y)=s_1(x,y)+a_1(x,y)=s_2(x,y)+a_2(x,y),\\
    \To s_1(x,y)-s_2(x,y)=a_2(x,y)-a_1(x,y).
\end{align*}
From this last equation, after exchanging the variables we get
$$s_1(x,y)-s_2(x,y)=s_1(y,x)-s_2(y,x)=a_2(y,x)-a_1(y,x)=-a_2(x,y)+a_1(x,y)$$ 
which gives us the equation 
$$a_2(x,y)-a_1(x,y)=-a_2(x,y)+a_1(x,y)\To a_2(x,y)=a_1(x,y).$$
Now that we have that the antisymmetric parts are equal, we see from the original hypothesis that 
$$s_1(x,y)+a_1(x,y)=s_2(x,y)+a_2(x,y)\To s_1(x,y)=s_2(x,y).$$
We conclude that the representation is unique.
\end{ptcbr}

\begin{Ej}[Exercise 2]
    (Review.) Write the power sum symmetric function $p_3$ in terms of elementary symmetric
functions.
\end{Ej}

\begin{ptcbr}
This result is valid in any number of variables, so let verify it in three variables and then extrapolate the general formula. Recall that in three variables the elementary symmetric functions are 
$$e_1=x+y+z,\ e_2=xy+yz+zx,\ e_3=xyz$$
so naively we can cube $e_1$ first and see what we get:
$$e_1^3=(x+y+z)^3=p_3+3(x^2y+y^2z+z^2x+y^2x+z^2y+x^2z)+6e_3.$$
The way to obtain the middle term is to multiply $e_1$ with $e_2$:
$$e_1e_2=x^2y+y^2z+z^2x+y^2x+z^2y+x^2z+3e_3,$$
solving for the expression we want we obtain 
$$3e_1e_2-9e_3=3(x^2y+y^2z+z^2x+y^2x+z^2y+x^2z)$$
and finally 
$$e_1^3=p_3+3e_1e_2-9e_3+6e_3\To \un{p_3=e_1^3-3e_1e_2+3e_3}.$$
We conclude that this is indeed the representation of $p_3$ in terms of the $e_i's$. 
\end{ptcbr}


\begin{Ej}[Exercise 5]
    Compute the Schur polynomial $s_{(2,1)}(x,y,z)$ as a ratio of determinants and using semi-standard Young Tableaux. Show that both computations agree.
   \end{Ej}
%https://encyclopediaofmath.org/wiki/Schur_functions_in_algebraic_combinatorics
\begin{ptcbr}
    \footnote{I started doing this problem because \textbf{Kelsey} wanted to check the calculations with me.}The partition $(2,1)$ is associated to the following Ferrers Diagram\footnote{Is there a way to draw Ferrers diagrams in French notation? The package I'm using \ttt{youngtab} only admits English notation. I mean, \textbf{you} must surely know.}:
    $$\young(~~,~)$$
    and since we have 3 variables to work with, we must fill out the diagram with numbers from 1 to 3 with the condition that rows are weakly increasing and columns increase. Out of the possible 27 ways to fill out the diagram, only the following are possible given the condition:
    $$\young(11,2),\ \young(12,2),\ \young(13,2),\ \young(11,3),\ \young(12,3),\ \young(22,3),\ \young(13,3),\ \young(23,3).$$ 
    The associated monomials are 
    $$x^2y,\ xy^2,\ xyz,\ x^2z,\ xyz,\ y^2z,\ xz^2,\ yz^2.$$
    So it follows that $s_{(2,1)}(x,y,z)=x^2y+xy^2+2xyz+x^2z+y^2z+xz^2+yz^2$.\par 
    In this case it's important to notice that $(2,1)=(2,1,0)$ so $s_{(2,1)}=s_{(2,1,0)}$, now using the bi-alternant we have 
\begin{align*}
    a_{(2,1,0)}(x,y,z)=\det\threebythree{x^2}{x}{1}{y^2}{y}{1}{z^2}{z}{1}&=x^2y+xz^2+y^2z-z^2y-zx^2-y^2x,\\
    a_{(4,2,0)}(x,y,z)=\det\threebythree{x^4}{x^2}{1}{y^4}{y^2}{1}{z^4}{z^2}{1}&=x^4y^2+x^2z^4+y^4z^2-z^4y^2-z^2x^4-y^4x^2.
\end{align*}
To factor this polynomials we have to \emph{creatively sum zero}, in this case $0=xyz-xyz$ and $0=x^2y^2z^2-x^2y^2z^2$. For $a_{(4,2,0)}$ we have
\begin{align*}
    &x^4y^2+x^2z^4+y^4z^2-z^4y^2-z^2x^4-y^4x^2+x^2y^2z^2-x^2y^2z^2\\
    =&x^2(x^2y^2-z^2x^2-y^4+y^2z^2)-z^2(z^2y^2-x^2z^2-y^4+x^2y^2)\\
    =&(x^2-z^2)[x^2(y^2-z^2)-y^2(y^2-z^2)]\\
    =&(x^2-z^2)(x^2-y^2)(y^2-z^2)\\
    =&(x-z)(x+z)(x-y)(x+y)(y-z)(y+z)
\end{align*}
In the case of $a_{(2,1,0)}$ we get $(x-z)(x-y)(y-z)$. It thus follows that 
$$s_{(2,1)}=\frac{a_{(4,2,0)}}{a_{(2,1,0)}}=(x+z)(x+y)(y+z)=x^2 y + x^2 z + x y^2 + 2 x y z + x z^2 + y^2 z + y z^2$$
so both computations agree on the value of $s_{(2,1)}$.
\end{ptcbr}


\begin{Ej}[Exercise 7, Stanley 7.3]
    Expand the power series $\prod_{i\geq 1}(1+x_i+x_i^2)$ in terms of elementary symmetric functions.
   \end{Ej}
%https://math.stackexchange.com/questions/1218375/how-can-i-use-fundamental-theorem-of-symmetric-polynomials-to-factor-polynomials
\begin{ptcbr}
We will begin with a lower dimensional cases, call $f$ our infinite product and $f_n$ its partial products. Consider the case when $n=2$, we have the following decomposition for $f_2$:
\begin{align*}
    (1+x+x^2)(1+y+y^2)&=x^2 y^2 + x^2 y + x^2 + x y^2 + x y + x + y^2 + y + 1\\
    &=e_2^2+e_1^2+e_2e_1-e_2+e_1+1.
\end{align*}
Likewise for $f_3$ we have 
$$f_3=e_3^2+e_2^2+e_1^2+e_2e_1+e_3e_2-e_3e_1-2e_3-e_2+e_1+1.$$
Expanding\footnote{This work was done using the \textbf{Macaulay2} software.} $f_4$ we see a similar pattern with the second powers however:
$$f_4=p_2(e_1,\dots,e_4)+e_2e_1+e_3e_2+e_4e_3-2e_4e_1-e_3e_1-e_4e_2-e_4-2e_3-e_2+e_1+1.$$
A pattern that we can observe is the sum of powers, and also, all the previous terms are used in the coming expansion. Notice that 
\begin{align*}
&f_3-f_2=e_3^2+e_3e_2-e_3e_1-2e_3,\\
&f_4-f_3=e_4^2+e_4e_3-2e_4e_1-e_4e_2-e_4.    
\end{align*}
This strange coefficients might come from an unexpected place. Let us recall\footnote{This idea was brought to my attention by \textbf{Jae Hwang} while discussing the problem the problem with him.} that 
$$(x-1)^3=(x-1)(x^2+x+1)=(x-1)(x-e^{\frac{i\pi}{3}})(x-e^{i\frac{2\pi}{3}}).$$
Now, notice that the last two factors can be rearranged to be 
$$e^{i\frac{\pi}{3}}e^{i\frac{2\pi}{3}}(xe^{-i\frac{\pi}{3}}-1)(xe^{-i\frac{2\pi}{3}}-1)=(1-xe^{-i\frac{\pi}{3}})(1-xe^{-i\frac{2\pi}{3}}).$$
Thus we can rearrange the product in question as follows:
$$\prod_{i\geq 1}(1+x_i+x_i^2)=\left(\prod_{k\geq 1}(1-x_ke^{-i\frac{\pi}{3}})\right)\left(\prod_{k\geq 1}(1-x_ke^{-i\frac{2\pi}{3}})\right)$$
and the products on the right are the generating functions of the elementary symmetric functions evaluated at $e^{-i\frac{\pi}{3}}$ and $e^{-i\frac{2\pi}{3}}$ respectively. We can rewrite them as 
$$\left(\sum_{n\geq 1}e_ne^{i\frac{\pi n}{3}}\right)\left(\sum_{n\geq 1}e_ne^{i\frac{2\pi n}{3}}\right)=\sum_{n\geq 1}\left(\sum_{k=1}^n\left(e^{i\frac{\pi k}{3}}\right)\left(e^{i\frac{2\pi (n-k)}{3}}\right)\right)e_n$$
Simplifying the exponentials we get 
$$\left(e^{i\frac{\pi k}{3}}\right)\left(e^{i\frac{2\pi (n-k)}{3}}\right)=e^{\frac{i\pi}{3}(k+2n-2k)}=e^{\frac{i\pi(2n-k)}{3}}.$$
We conclude that 
$$\prod_{i\geq 1}(1+x_i+x_i^2)=\sum_{n\geq 1}\left(\sum_{k=1}^ne^{\frac{i\pi(2n-k)}{3}}\right)e_n.$$
\end{ptcbr}
\iffalse
\begin{Ej}[Exercise 8, Stanley 7.4]
    Show that 
    $$h_r(\un x)=\sum_{k=1}^nx_k^{n-1+r}\prod_{i\neq k}\frac{1}{x_k-x_i}.$$
\end{Ej}

\begin{ptcbr}
    The expression on the right looks like a Laplace expansion of a determinant so let us begin by considering matrices which might look like that. Such a matrix must have a row or column whose entries are $x_i^{n-1+r}$ and the cofactors through that array are the products in question.
\end{ptcbr}
\fi
\end{document} 
