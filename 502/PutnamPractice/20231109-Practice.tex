\documentclass[12pt]{memoir}

\def\nsemestre {II}
\def\nterm {Fall}
\def\nyear {2023}
\def\nprofesor {Ignacio Rojas}
\def\nsigla {Putnam}
\def\nsiglahead {Putnam Practice}
\def\nextra {PS2}
\def\nlang {ENG}
\def\nauthor {}
\input{../../headerVarillyDiff}

\begin{document}

\subsection*{Toolkit}
\begin{itemize}
    \item Quadratic forms, which are expressions of the form 
    $$ax^2+bxy+cy^2=\begin{pmatrix}x&y\end{pmatrix}\twobytwo{a}{b/2}{b/2}{c}\twobyone{x}{y}=\vec{x}^\sT A\vec{x}$$
    \item Relationship between quadratic forms and conics:\par  %https://en.wikipedia.org/wiki/Conic_section#Discriminant
    Quadratic forms determine conics, these are parabolas, ellipses, and hyperbolas. The type of conic can be determined diagonalizing the matrix $A$.
    \item Minkowski's theorem for the geometry of numbers: The idea in $\bR^2$ is that \emph{large enough} convex and symmetric sets contain points of $\bZ^2$ other than $(0,0)$.
\end{itemize}

\begin{nonum-Th}[Minkowski on $\bR^2$]
Suppose $X$ convex, bounded and symmetric set with 
$$\operatorname*{Area}(X)\geq 4$$
then $X$ contains a non-zero point $(x,y)\in\bZ^2$.
\end{nonum-Th}

\begin{Ej}[Polish Olympiad]
    Let $a,b,c$ be positive integers with 
    $$ac=b^2+b+1$$
    Prove that the equation 
    $$ax^2-(2b+1)xy+cy^2=1$$
    has integer solutions $(x,y)$.
\end{Ej}

\begin{Ej}[Hungarian Olympiad]
    Suppose $n$ is a positive integer such that 
    $$x^2+xy+y^2=n$$
    has rational solutions $(x,y)$. Show that the equation also has integer solutions.
\end{Ej}

\end{document} 
