\documentclass[12pt]{memoir}

\def\nsemestre {I}
\def\nterm {Semestre I}
\def\nyear {2025}
\def\nprofesor {Ignacio Rojas}
\def\nsigla {ECM}
\def\nsiglahead {Práctica Club de Lectura}
\def\nextra {CL}
%\def\nlang {ENG}
\def\nauthor {}
\input{../../headerVarillyDiff}
%\renewcommand\thesection{\arabic{section}}
\begin{document}
\numberwithin{equation}{chapter}
\setcounter{chapter}{1}

\begin{Ej}[Combinatoria]
    Verifique las siguientes identidades mediante doble conteo:
    \begin{align}
        &\binom{n}{k}k = n\binom{n-1}{k-1},\\
        &\binom{n}{k}(n-k) = n\binom{n-1}{k},\\
        &\binom{n}{k}\binom{k}{\l} = \binom{n}{\l}\binom{n-\l}{k-\l},\\
        &\binom{n}{k}\binom{n-k}{\l} = \binom{n}{\l}\binom{n-\l}{k}.
    \end{align}
    Note que las dos últimas generalizan las primeras.
\end{Ej}

\begin{Ej}[Putnam 2024 A6]%https://math.stackexchange.com/questions/2833589/about-a-sum-that-looks-like-a-determinant
    Sea
    \[
    f(x)=\frac{1-3x-\sqrt{1-14x+9x^2}}{4},\quad f(x)=\sum_{k=0}^{\infty}c_kx^k.
    \]
    Sea $C$ la matriz $n\times n$ con entradas $C_{ij}=c_{i+j-1}$. Calcule $\det C$.
\end{Ej}

\begin{Ej}[Putnam 2001 A2]
    Sea $G$ un grupo tal que $(xy)x = y$ para todo $x,y\in G$. Pruebe que también se cumple $x(yx)=y$.
\end{Ej}

\begin{Ej}[Putnam 1989 B2]%https://math.stackexchange.com/questions/203023/a-finite-cancellative-semigroup-is-a-group?noredirect=1&lq=1
%https://math.stackexchange.com/questions/2772940/is-this-structure-a-group
    Sea $(S,\circ)$ un semigrupo cancelable donde todo elemento tiene orden finito. Es decir:
    \begin{enumerate}
        \item $\circ$ es asociativa.
        \item Si $xz=yz$ o $zx=zy$, entonces $x=y$ para $x,y,z \in S$.
        \item El conjunto $\{x^k\mid k\in\bN\}$ es finito para todo $x\in S$.
    \end{enumerate}
    Demuestre que $(S,\circ)$ es un grupo.
\end{Ej}
\newpage
\begin{Ej}[Putnam 2005 B6]
    Sea $\lie S_n$ el grupo simétrico de $n$ elementos. Para $\sigma\in\lie S_n$, defina:
    \[
    \sgn \sigma =
    \begin{cases}
        1 & \text{si $\sigma$ es par,}\\
        -1 & \text{si $\sigma$ es impar.}
    \end{cases}
    \]
    Equivalentemente, $\sgn\sigma = (-1)^{\inv\sigma}$, donde $\inv\sigma$ es el número de inversiones. Esto es, la cantidad de números que aparecen ``en desorden''. Por ejemplo,
    $$\inv(3142)=3\word{por}(3,1),(3,2),(4,2).$$
    Sea $|\Fix \sigma|$ el número de puntos fijos. Pruebe que
    \[
    \sum_{\sigma\in\lie S_n}\frac{\sgn \sigma}{|\Fix\sigma|+1} = (-1)^n\frac{n}{n+1}.
    \]
\end{Ej}

\begin{Ej}[Extra]%https://artofproblemsolving.com/community/c7h64449
%https://math.stackexchange.com/questions/2833589/about-a-sum-that-looks-like-a-determinant
    Verifique las siguientes identidades:
    \begin{align}
        &\sum_{\sigma\in \lie S_n}\frac{\sgn(\sigma)}{(|\Fix(\sigma)|+1)^2} = (-1)^{n+1}\left(\frac{nH_n}{n+1} - \frac{1}{(n+1)^2}\right),\\
        &\sum_{\sigma\in\lie S_n}\frac{\sgn\sigma}{2^{|\Fix\sigma|}} = \frac{2n - 1}{2}\left(\frac{-1}{2}\right)^{n - 1},\\
        &\sum_{\sigma\in\lie S_n} \sgn\sigma\left(\frac{1}{3}\right)^{n - |\Fix\sigma|}\left(\frac{4}{3}\right)^{|\Fix\sigma|} = \frac{n+3}{3}.
    \end{align}
    Aquí, $H_n = \sum_{k=1}^{n}\frac{1}{k}$ es el $n$-ésimo número armónico.
\end{Ej}



\end{document} 
