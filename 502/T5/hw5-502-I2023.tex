\documentclass[12pt]{memoir}

\def\nsemestre {I}
\def\nterm {Spring}
\def\nyear {2023}
\def\nprofesor {Maria Gillespie}
\def\nsigla {MATH502}
\def\nsiglahead {Combinatorics 2}
\def\nextra {HW5}
\def\nlang {ENG}
\input{../../headerVarillyDiff}
\usepackage[enableskew]{youngtab}
\usepackage{ytableau}
\begin{document}

\begin{Ej}[Exercise 3]
    Prove that a word $w$ has highest weight (i.e., $E_i(w)=0$ for all $i$) if and only if $w$ is Yamanouchi
\end{Ej}
%https://arxiv.org/pdf/2007.11721.pdf
\begin{ptcbr}
    First suppose $w\in\bonj{n}^k$ is a word of length $k$ on the alphabet $\bonj{n}$. Now suppose additionally that $w$ is Yamanouchi. This means that for every $s\leq k$, the suffix 
    $$w_{k-s+1}\dots w_{k-1}w_k$$ 
    contains at least as many $i$'s after the $(i+1)$'s. In particular this holds when $s=k$. So when applying the raising $E_i$ operator we pair $(i+1)$ with an $i$ to its right as a parenthesis. There are as much $i$'s as $(i+1)$'s so every $(i+1)$ is paired and so the $E_i$ operator can't convert any $(i+1)$ to an $i$.\par 
    As $i$ is arbitrary, we can't apply any $E_i$ to $w$ which means that $w$ has highest weight.\par 
    On the other hand\footnote{Once again, \textbf{Kelsey}, \textbf{Trent}, and myself have talked about this problem in order to under the idea.} suppose $w\in\bonj{n}^k$ has highest weight. Then for all $i$, we can't apply $E_i$ to $w$. This means that in $w$, it is possible to match all the $(i+1)$'s with $i$'s that succeed them.\par 
    The previous fact lets us see that when reading $w$ from right-to-left we will find at least as many $i$'s as we find $(i+1)$'s. In other words, this means that $w$ is Yamanouchi.
    \end{ptcbr}


\begin{Ej}[Exercise 4]
    Formulate and prove a Yamanouchi-type condition for $w$ to be lowest weight, that is, $F_i(w) = 0$
for all $i\in\bonj{n}$. Such a word is called \un{anti-ballot}.
\end{Ej}
%FLUIT EFUTI
\begin{ptcbr}
We will define anti-ballot by remembering the definition of ballot. Recall a ballot word $w$ in $\bonj{n}^k$ has the property that for every $p\leq k$, the prefix 
$$w_1w_2\dots w_p$$
contains at least as many $i$'s \emph{before} the $(i+1)$'s. So an anti-ballot word will contain as many \un{$(i+1)$}'s before the $\un i$'s. The claim is as follows:
\begin{significant}
A word $w$ has lowest weight if and only if $w$ is anti-ballot.
\end{significant}
To prove this we first assume a word $w$ is anti-ballot. Before applying $F_i$ we notice that all the latter $i$'s must already be paired with a previous $(i+1)$ because the word is anti-ballot.\par 
This property holds for all $i$ so it happens that we can't apply $F_i$ to $w$, thus $w$ has lowest weight.\par
On the flip-side, if we assume $w$ has lowest weight, it means that all $i$'s in $w$ are matched with $(i+1)$ that precede them. So it must hold that there are at least as many $(i+1)$'s as $i$'s in $w$. This condition holds for every $i$ which means that $w$ is anti-ballot.
\end{ptcbr}

\begin{Rmk}
It is important to notice that an anti-ballot word is not necessarily Yamanouchi even if it contains all the possible letters of the alphabet in question. The word 
$$33\dots 321$$
is anti-ballot but not Yamanouchi.\par 
The examples from the crystal graphs contain Yamanouchi words on the top and anti-ballot words on the bottom.
\end{Rmk}

\begin{Ej}[Exercise 7]
    How many ballot words of length $n$ have only $1$'s and $2$'s?
\end{Ej}

\begin{ptcbr}
First let us begin by noticing that any Yamanouchi word can be associated to an SYT $T$ with $\text{sh}(T)=\la$, $|\la|=n$ and whose height is at most two.\par 
Reading the Yamanouchi word from right-to-left, the entry $w_{k-i+1}$ tells us the row in which we will insert $i$ into the table. This insertion process is different from RSK insertion in the sense that we only ``let the number fall along the row''.\par 
As an \textbf{example}, consider the the Yamanouchi word $212211121$. This word is associated to the tableau 
$$\young(2679,13458)$$
because reading from right-to-left, the first $1$ tells us to put the $1$ in the first row, the $2$ means that the $2$ goes on the second row. The coming string $111$ tells us that $3,4$ and $5$ all go in the first row after the one and so on.\par 
This algorithm produces a standard Young tableau of height 2 as the word is Yamanouchi. 
    \iffalse
    Let us begin by counting the small cases:
    \begin{itemize}
        \itemsep=-0.4em
        \item When $n=1$ we only have the word $1$. 
        \item When $n=2$ we have both $11$ and $21$. 
        \item When $n=3$ we have $111,211$ and $121$. We can't add $221$ because it stops being Yamanouchi.
        \item For $n=4$ the words are $1111,2111,1211,2211,1121$ and $2121$. 
        \item The next case is $n=5$ with 
        $$11111,21111,12111,22111,11211,21211,12211,11121,21121\word{and}12121.$$
        \item Notice now that we will get $20$ possibilities for $n=6$ because Yamanouchi words on $n=5$ all have three $1$'s and two $2$'s so it's possible to add a $1$ or a $2$ on each possibility which brings our total to $20$.
     \end{itemize}
We may count $n=7$ and $n=8$ to obtain $35$ and $70$.\par
Let us now call $a_n$, the number of Yamanouchi words of length $n$ with alphabet $\set{1,2}$. By using the duplication argument we have that 
$$a_{2n}=2a_{2n-1}$$
and now let us consider only the odd terms in $a_n$. We have  
$$1,3,10,35,126,\dots$$
\fi
\end{ptcbr}
\end{document} 
