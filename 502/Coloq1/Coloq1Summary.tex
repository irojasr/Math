\documentclass[12pt]{memoir}

\def\nsemestre {I}
\def\nterm {Spring}
\def\nyear {2023}
\def\nprofesor {Maria Gillespie}
\def\nsigla {MATH502}
\def\nsiglahead {Combinatorics 2}
\def\nextra {P}
\def\nlang {ENG}
\input{../../headerVarillyDiff}

\title{\vspace{-2.7cm}Permutation Statistics.\vspace{-0.7cm}}
\author{Ignacio Rojas}
\date{Summer, 2023}

\begin{document}
\bgroup
\renewcommand\thesection{\arabic{section}}
\renewcommand{\thefigure}{\arabic{figure}}
\maketitle
\vspace*{-2.5em}
\begin{abstract}
    \vspace*{-1.5em}
    When we are being taught permutations, it's common to introduce the cycle notation. There are other ways to write permutations, for example, in \emph{list notation}. For example in $S_4$:
    $$(123)=\begin{pmatrix}
        1&2&3&4\\
        2&3&1&4
    \end{pmatrix}=2314$$
    In list notation it's possible count certain characteristics of a permutation, for example, \emph{the number of inversions}. This particular permutation has $2$ inversions because $2$ is to the left of $1$ and $3$ is to the left of $1$.\par 
    This is an initial example of a permutation \emph{statistic}. We will talk about different statistics, see how they are related, find generating functions for them and express them in a different way as a $q$-analogue.\par 
    The idea for this colloquium is to introduce this topic with interactive examples among the participants, so come prepared with pen and paper and with an open mind!
    \end{abstract}
    
    \section{Introduction}

%%%%%%%%%%%% Contents end %%%%%%%%%%%%%%%%
\ifx\nextra\undefined
\printindex
\else\fi
\nocite{*}
\bibliographystyle{plain}
%\bibliography{bibiProyCombi2.bib}
\end{document}