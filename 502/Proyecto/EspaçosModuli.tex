\documentclass[12pt]{memoir}

\def\nsemestre {I}
\def\nterm {Spring}
\def\nyear {2023}
\def\nprofesor {Maria Gillespie}
\def\nsigla {MATH502}
\def\nsiglahead {Combinatorics 2}
\def\nextra {Project}
\def\nlang {ENG}
\input{../../headerVarillyDiff}

\begin{document}

Let us begin with a simple question:
\begin{significant}
What are all the quadratic curves which pass through $4$ points in general position in $\bR^2$?
\end{significant}
This question might be a bit tough to tackle right now, but let us consider a simplification. How about if the points are $(1,1),(1,-1),(-1,-1)$ and $(-1,1)$? At once the following idea should pop-in in our heads: \emph{a circle}!\par 
The circle which passes through these points is 
\begin{figure}
    fig 1
\end{figure}
Ideally we would like to stretch and shrink thes circle in order to make it an ellipse. We know ellipses have equations of the form $$\frac{x^2}{a^2}+\frac{y^2}{b^2}=1,$$ 
but to begin from our circle equation we will instead add coefficients to the equation $Ax^2+By^2=2$. These coefficients are bound by the points in the curve like $(x,y)=(1,1)$. We may derive the relation 
$$A(1)^2+B(1)^2=2\To B=2-A$$
so we take $t=A$ to get the parametrized equation $tx^2+(2-t)y^2=2$. If we let $t$ vary we start seeing the following behavior:
\begin{figure}
    fig 2
\end{figure}
If we let $t=1$ we recover the original circle equation. When $t>1$ we may choose for example $t=3$ to get the equation $3x^2-y^2=2$
\end{document}