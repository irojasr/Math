\documentclass[12pt]{memoir}

\def\nsemestre {I}
\def\nterm {Spring}
\def\nyear {2023}
\def\nprofesor {Maria Gillespie}
\def\nsigla {MATH502}
\def\nsiglahead {Combinatorics 2}
\def\nextra {P}
\def\nlang {ENG}
\input{../../headerVarillyDiff}

\begin{document}

Let us begin with a simple question:
\begin{significant}
Which are the quadratic curves which pass through $4$ points in general position in $\bR^2$?
\end{significant}
This question might be a bit tough to tackle right now, but let us consider a simplification. How about if the points are $(1,1),(1,-1),(-1,-1)$ and $(-1,1)$? At once the following idea should pop-in in our heads: \emph{a circle}!\par 
The circle which passes through these points is described by the equation $x^2+y^2=2$:\par
\begin{figure}[h]
    \centering
    \includegraphics[width=0.3\textwidth, trim= 0.8cm 22.9cm 16cm 0.6cm,clip]{fig1.pdf}
\end{figure}
Ideally we would like to stretch and shrink the circle in order to make it an ellipse. We know ellipses have equations of the form $x^2/a^2+y^2/b^2=1$, but to begin from our circle equation we will instead add coefficients to the equation 
$$Ax^2+By^2=2.$$
These coefficients are bound by the points in the curve like $(x,y)=(1,1)$. We may derive the relation 
$$A(1)^2+B(1)^2=2\To B=2-A\To tx^2+(2-t)y^2=2$$
where we take $t=A$ to get the last equation.
The parameter $t$ can vary to exhibit the following behavior:\par
\begin{figure}[h!]
    \centering
    \includegraphics[width=0.8\textwidth, trim= 0.8cm 21cm 0.8cm 0.6cm,clip]{fig1.pdf}
\end{figure}

We annotate the curves we obtain given the values of $t$:
\vspace{-0.5em}
\begin{itemize}
    \itemsep=-0.4em
    \item $(t=1)$: A circle.
    \item $(1<t<2)$: An ellipse.
    \item $(t=2)$: The pair of lines $x^2=1$.
    \item $(t>2)$: A hyperbola asymptotic to the $y$-axis.
\end{itemize}
The same phenomenon occurs (but in the $x$ axis) for $t<1$. We may summarize the behavior as follows:\par
\begin{figure}[h!]
    \centering
    \includegraphics[width=0.8\textwidth, trim= 0.8cm 15cm 0.8cm 6.6cm,clip]{fig1.pdf}
\end{figure}
However we are left with one curve which passes through the points in question. To find it we will assume $t$ is non-zero. From our parametric equation we obtain 
$$tx^2+(2-t)y^2=2\To x^2+o(t^2)+y^2=\frac{2}{t^2}\xrightarrow[t\to\infty]{}x^2=y^2$$
which is the pair of lines $y=\pm x$.
\begin{figure}[h!]
    \centering
    \includegraphics[width=0.3\textwidth, trim= 0.8cm 7cm 16cm 14.6cm,clip]{fig1.pdf}
\end{figure}
Observe that this behavior is independent of the sign of the infinity we are going to.\par 
In essence what we have seen is that all the quadratic curves passing through our set of points can be parametrized by $\bR\cup\set{\infty}$. Formally:
\begin{Prop}
The moduli space $\ov{\cM}_{0,4}$ can be identified with $\bP^1_\bR$.
\end{Prop}
Intuitively the \term{moduli space} is a set where we may vary points \emph{continuously} and the objects which they parametrize will deform \emph{continuously} as well.
\end{document}