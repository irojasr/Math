\documentclass[12pt]{memoir}

\def\nsemestre {I}
\def\nterm {Spring}
\def\nyear {2023}
\def\nprofesor {Maria Gillespie}
\def\nsigla {MATH502}
\def\nsiglahead {Combinatorics 2}
\def\nextra {HW9}
\def\nlang {ENG}
\input{../../headerVarillyDiff}
\usepackage[enableskew]{youngtab}
\usepackage{ytableau}
\DeclareMathOperator{\SYT}{SYT}
\DeclareMathOperator{\inv}{inv}
\DeclareMathOperator{\maj}{maj}
\begin{document}

\begin{Ej}[Exercise 1]
    Prove that the tensor product of two Hadamard matrices is a Hadamard matrix.
\end{Ej}

\begin{ptcbr}
Suppose $H,K$ are two Hadamard matrices. We have $HH^\sT=mI$ and $KK^\sT=nI$ and we must show that $(H\ox K)(H\ox K)^\sT=mnI$ where the last identity matrix has size $(mn)\x(mn)$.\par 
The product enjoys two properties which are essential for our purpose:
\begin{itemize}
    \item Transposition distributes over the product: $(A\ox B)^\sT=A^\sT\ox B^\sT$.
    \item The \emph{mixed-product property}: If $A,B,C,D$ are matrices, then 
    $$(A\ox B)(C\ox D)=AC\ox BD.$$
\end{itemize}
With this in hand we see that
$$(H\ox K)(H\ox K)^\sT=HH^\sT\ox KK^\sT=mn I_m\ox I_n=mn I_{mn}$$
and thus $H\ox K$ is Hadamard as desired.
\end{ptcbr}
%https://www.statlect.com/matrix-algebra/Kronecker-product-properties
\begin{Lem}
Transposition is distributive with respect to the product.
\end{Lem}

\begin{ptcbp}
Assume $A$ has size $k\x\l$, then
$$(A\ox B)^\sT=\threebythree{a_{11}B}{\dots}{a_{1\l}B}{\vdots}{\ddots}{\vdots}{a_{k1}B}{\dots}{a_{k\l}B}^\sT=\threebythree{a_{11}B^\sT}{\dots}{a_{k1}B^\sT}{\vdots}{\ddots}{\vdots}{a_{1\l}B^\sT}{\dots}{a_{k\l}B^\sT}=\threebythree{a_{11}}{\dots}{a_{k1}}{\vdots}{\ddots}{\vdots}{a_{1\l}}{\dots}{a_{k\l}}\ox B^\sT.$$
And we may recognize $A^\sT$ as the last matrix. So the transposition property holds.
\end{ptcbp}

\begin{Lem}
The mixed product property holds.
\end{Lem}

\begin{ptcbp}
Assume $A$ has size $k\x\l$ and $C$ has size $\l\x m$, then
$$(A\ox B)(C\ox D)=\threebythree{a_{11}B}{\dots}{a_{1\l}B}{\vdots}{\ddots}{\vdots}{a_{k1}B}{\dots}{a_{k\l}B}\threebythree{c_{11}D}{\dots}{c_{1m}D}{\vdots}{\ddots}{\vdots}{c_{\l1}D}{\dots}{c_{\l m}D}$$
and multiplying this two matrices we obtain entries of the form 
$$\left(\sum_{r=1}^{\l}a_{ir}c_{rj}\right)BD=(AC)_{ij}BD.$$
Thus we have 
$$(A\ox B)(C\ox D)=\threebythree{(AC)_{11}BD}{\dots}{(AC)_{1m}BD}{\vdots}{\ddots}{\vdots}{(AC)_{k1}BD}{\dots}{(AC)_{km}BD}=(AC)\ox(BD).$$
\end{ptcbp}

\begin{Ej}[Exercise 2]
    Prove that there's only one $2-(7,3,1)$ design up to isomorphism.
\end{Ej}

\begin{ptcbr}
    We know that the Fano plane is an example of a $2-(7,3,1)$ design. Given another $2-(7,3,1)$ design, $(X,\cB)$, we will find an isomorphism between the Fano plane and our design.\par 
    First note that this is a square design. Take 2 blocks $B,B'$, then it must happen that $|B\cap B'|=1$.
    \begin{itemize}
        \itemsep=-0.4em
        \item The intersection can't be larger than $1$ because every pair of points is contained in precisely $1$ block, not more than $1$.
        \item If two blocks are disjoint then we name their elements
        $$B=\set{1,2,3},\word{and}B'=\set{4,5,6}$$
        which leaves $7$ out of the mix. 
        \begin{center}
            \includegraphics[width=0.3\textwidth, trim= 2.1cm 21.8cm 14cm 1.9cm,clip]{fig1.pdf}
        \end{center}
        Take two elements from $B$ and $B'$, without losing generality there's one block which contains them $\set{1,4,x}$. The element $x$ can't come from $B$ as $1$ is already paired with $2$ and $3$ there. It can't also come from $B'$ because there will be two blocks which contain two common elements. It must happen that the new block is $\set{1,4,7}$. In the same fashion we can construct blocks ${2,5,7}$ and ${3,6,7}$. 
        \begin{center}
            \includegraphics[width=0.3\textwidth, trim= 8cm 21.8cm 8cm 1.9cm,clip]{fig1.pdf}
        \end{center}
        However we've reached an impasse, because we must somehow pair $1$ and $5$ in a block. We can't add an element from $B$ as ${1,5,x}$ will intersect $B$ in two elements. In the same fashion, if $y\in B'$, then $|\set{1,5,y}\cap B'|=2$. Finally we can't have $\set{1,5,7}$ because $1,7$ and $5,7$ will be repeated in two blocks. 
        \begin{center}
            \includegraphics[width=0.3\textwidth, trim= 14.1cm 21.8cm 2cm 1.8cm,clip]{fig1.pdf}
        \end{center}
        This means that we can't form more than $5$ blocks given our constraint. But this contradicts Fisher's inequality as we must have at least as many blocks as vertices and $b=5<v=7$ in this $2$-design.
    \end{itemize}
    Our assumption that there are two disjoint blocks is false, so it must occur that any two blocks intersect in exactly $1$ vertex. Immediately this tells us that $b=v$ and $r=k$ which means that:
    \begin{itemize}
        \itemsep=-0.4em
        \item There are $7$ blocks.
        \item Every point is contained in exactly $3$ blocks.
    \end{itemize}
    Consider the block $B=\set{1,2,3}$ and then the following graph:
    \begin{center}
        \includegraphics[width=0.3\textwidth, trim= 1.5cm 23.5cm 15cm 0.2cm,clip]{fig2.pdf}
    \end{center}
    We will build the remaining blocks as follows, take the remaining elements $\set{4,5,6,7}$ and consider the pairs of elements in the set $\set{45,46,47,56,57,67}$. By matching disjoint pairs we form the graph on top.\par 
    The edges may be named by any arbitrary choice\footnote{This means that we could also consider the blocks $\set{1,4,5}$ and $\set{1,6,7}$ for example.} of our elements in $B$. The blocks will be an edge and one of the vertex endpoints. We get the blocks 
    $$B,\set{1,4,7},\set{1,5,6},\set{2,4,6},\set{2,5,7},\set{3,4,5},\word{and}\set{3,6,7}.$$
    This is a square $2-(7,3,1)$ design and we may associate the lines of the Fano plane in the obvious way to the blocks of our design while vertices are also mapped to vertices. 
\end{ptcbr}

\begin{Ej}[Exercise 4]
    The \textbf{complementary design} to a design $\cD = (X, \cB)$ is the pair $\cD^c = (X, \cB^c)$ where $\cB^c = \set{X \less B \: B \in B}$. Show that if $\cD$ is a $1 - (v, k, \la)$ design then $\cD^c$ is a $1 - (v, v - k, v\la/k-\la)$ design.
\end{Ej}
%https://math.stackexchange.com/questions/370188/empty-intersection-and-empty-union
\begin{ptcbr}
We know that in $\cD^c$ we have $v$ vertices. Any block is of the form $X\less B$ with $B$ having size $k$, so all the blocks in $\cD^c$ have size $v-k$ as desired.\par 
It remains to show that every point is in exactly $\frac{v\la}{k}-\la$ blocks. Now, let us manipulate this quantity:\par 
Recall $r$ is the number of blocks containing a point, in this case as we have a $1$-design, we have that $r=\la$, so 
$$\frac{v\la}{k}=\frac{vr}{k}=\frac{bk}{k}=b,\word{the number of blocks.}$$
So we must show that every point is in $b-r$ blocks, but now this is immediate because any point already on $r$ blocks, is not inside the remaining $b-r$ blocks. But that is what it means to be inside a block in the complementary design. We conclude that $\cD^c$ is indeed a $1-(v,v-k,b-r)$ design.
\end{ptcbr}

\begin{Ej}[Exercise 6]
    Prove that the edge-complement of a strongly regular graph is strongly regular, and find the new parameters in terms of the previous.
\end{Ej}

\begin{ptcbr}
    Suppose $G$ is strongly regular with parameters $(n,k,\la,\mu)$. Pick a vertex $v$ and look at its neighbors, there are $k$ of them. Out of the remaining $n-1$ vertices, $v$ is not connected to $n-1-k$ of them.\par 
    Now pick another vertex $w$, if they are not connected then they share $\mu$ common neighbors. From the remaining $n-2$ vertices, $u,v$ are only connected to their neighborhoods. Removing all the vertices in the neighborhoods doubly counts the intersection, and we know that intersection has size $\mu$. So in total, $v,w$ are not connected to  
    $$(n-2)-2k+\mu\word{vertices together.}$$
    In a same fashion if they were connected, the number of shared neighbors is $\la$ so together they wouldn't be connected to $(n-2)-2k+\la$ vertices.\par 
    We conclude that if $G$ is strongly regular with parameters $(n,k,\la,\mu)$, then $G^c$ has parameters 
    $$(n,(n-1)-k,(n-2)-2k+\mu,(n-2)-2k+\la).$$
\end{ptcbr}
\end{document}