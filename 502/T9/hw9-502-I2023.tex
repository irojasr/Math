\documentclass[12pt]{memoir}

\def\nsemestre {I}
\def\nterm {Spring}
\def\nyear {2023}
\def\nprofesor {Maria Gillespie}
\def\nsigla {MATH502}
\def\nsiglahead {Combinatorics 2}
\def\nextra {HW9}
\def\nlang {ENG}
\input{../../headerVarillyDiff}
\usepackage[enableskew]{youngtab}
\usepackage{ytableau}
\DeclareMathOperator{\SYT}{SYT}
\DeclareMathOperator{\inv}{inv}
\DeclareMathOperator{\maj}{maj}
\begin{document}

\begin{Ej}[Exercise 1]
    Prove that the tensor product of two Hadamard matrices is a Hadamard matrix.
\end{Ej}

\begin{ptcbr}
Suppose $H,K$ are two Hadamard matrices. We have $HH^\sT=mI$ and $KK^\sT=nI$ and we must show that $(H\ox K)(H\ox K)^\sT=mnI$ where the last identity matrix has size $(mn)\x(mn)$.\par 
The product enjoys two properties which are essential for our purpose:
\begin{itemize}
    \item Transposition distributes over the product: $(A\ox B)^\sT=A^\sT\ox B^\sT$.
    \item The \emph{mixed-product property}: If $A,B,C,D$ are matrices, then 
    $$(A\ox B)(C\ox D)=AC\ox BD.$$
\end{itemize}
With this in hand we see that
$$(H\ox K)(H\ox K)^\sT=HH^\sT\ox KK^\sT=mn I_m\ox I_n=mn I_{mn}$$
and thus $H\ox K$ is Hadamard as desired.
\end{ptcbr}

\begin{Ej}[Exercise 4]
    The \textbf{complementary design} to a design $\cD = (X, \cB)$ is the pair $\cD^c = (X, \cB^c)$ where $\cB^c = \set{X \less B \: B \in B}$. Show that if $\cD$ is a $1 - (v, k, \la)$ design then $\cD^c$ is a $1 - (v, v - k, v\la/k-\la)$ design.
\end{Ej}
%https://math.stackexchange.com/questions/370188/empty-intersection-and-empty-union
\begin{ptcbr}
We know that in $\cD^c$ we have $v$ vertices. Any block is of the form $X\less B$ with $B$ having size $k$, so all the blocks in $\cD^c$ have size $v-k$ as desired.\par 
It remains to show that every point is in exactly $\frac{v\la}{k}-\la$ blocks. Now, let us manipulate this quantity:\par 
Recall $r$ is the number of blocks containing a point, in this case as we have a $1$-design, we have that $r=\la$, so 
$$\frac{v\la}{k}=\frac{vr}{k}=\frac{bk}{k}=b,\word{the number of blocks.}$$
So we must show that every point is in $b-r$ blocks, but now this is immediate because any point already on $r$ blocks, is not inside the remaining $b-r$ blocks. But that is what it means to be inside a block in the complementary design. We conclude that $\cD^c$ is indeed a $1-(v,v-k,b-r)$ design.
\end{ptcbr}

\begin{Ej}[Exercise 6]
    Prove that the edge-complement of a strongly regular graph is strongly regular, and find the new parameters in terms of the previous.
\end{Ej}

\begin{ptcbr}
    Suppose $G$ is strongly regular with parameters $(n,k,\la,\mu)$. Pick a vertex $v$ and look at its neighbors, there are $k$ of them. Out of the remaining $n-1$ vertices, $v$ is not connected to $n-1-k$ of them.\par 
    Now pick another vertex $w$, if they are not connected then they share $\mu$ common neighbors. From the remaining $n-2$ vertices, $u,v$ are only connected to their neighborhoods. Removing all the vertices in the neighborhoods doubly counts the intersection, and we know that intersection has size $\mu$. So in total, $v,w$ are not connected to  
    $$(n-2)-2k+\mu\word{vertices}$$
\end{ptcbr}
\end{document}