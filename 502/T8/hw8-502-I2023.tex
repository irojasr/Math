\documentclass[12pt]{memoir}

\def\nsemestre {I}
\def\nterm {Spring}
\def\nyear {2023}
\def\nprofesor {Maria Gillespie}
\def\nsigla {MATH502}
\def\nsiglahead {Combinatorics 2}
\def\nextra {HW8}
\def\nlang {ENG}
\input{../../headerVarillyDiff}
\usepackage[enableskew]{youngtab}
\usepackage{ytableau}
\DeclareMathOperator{\SYT}{SYT}
\DeclareMathOperator{\inv}{inv}
\DeclareMathOperator{\maj}{maj}
\begin{document}

\begin{Ej}[6.8 Sagan]
    Do the following:
    \begin{enumerate}[i)]
        \itemsep=-0.4em
        \item  The group of symmetries of a regular $n$-gon is called a dihedral group and
        consists of the $n$ rotations and $n$ reflections which map the $n$-gon to itself.\par 
        Find the number of different 4-bead, $r$-color necklaces if necklaces are considered the same when one is a rotation or reflection of the other.
        \item Find an expression for the number of distinct $n$-bead, $r$-color necklaces if two
        are the same when one is a rotation or a reflection of the other.        
    \end{enumerate}
\end{Ej}

\begin{ptcbr}
 Observe that the dihedral group which acts on a $4$-bead necklace is $D_8=\genr{(1234),(13)}\leq S_4$. We can succinctly view the elements of the group as follows:
        \begin{align*}
            &\set{(1)(2)(3)(4),(1234),(13)(24),(1432),(13),(1234)(13),(13)(24)(13),(1432)(13)}\\
            =&\set{(1)(2)(3)(4),(1234),(13)(24),(1432),(13),(14)(23),(24),(12)(24)}
        \end{align*}
        With this, we may use Burnside's lemma to find the number of orbits:
        \begin{align*}
        \#\text{orbits}&=\frac{1}{|D_8|}\sum_{g\in D_{8}}|\Fix(g)|=\frac{1}{8}\sum_{g\in D_{8}}r^{c(g)}\\
        &=\frac{1}{8}(r^4+r^1+r^2+r^1+r^3+r^2+r^3+r^2)
        =\frac{1}{8}(r^4+2r^3+3r^2+2r).
        \end{align*}
Therefore when $r$-coloring we have $\frac{1}{8}(r^4+2r^3+3r^2+2r)$ different necklaces.\par 
The game changes a bit when considering $n$ beads as descr
\end{ptcbr}

\end{document} 
