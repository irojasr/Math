\documentclass[12pt]{memoir}

\def\nsemestre {I}
\def\nterm {Spring}
\def\nyear {2023}
\def\nprofesor {Maria Gillespie}
\def\nsigla {MATH502}
\def\nsiglahead {Combinatorics 2}
\def\nextra {HW2}
\def\nlang {ENG}
\input{../../headerVarillyDiff}
\usepackage{youngtab}

\begin{document}

\begin{Ej}[Exercise 1]
    Evaluate the following expressions using the properties of the Hall inner product discussed in class:\vspace{-0.4em}
    \begin{enumerate}[i)]
        \itemsep=-0.4em
        \item $\braket{s_{(2,1)}}{h_{(1,1,1)}}$.
        \item $\braket{s_{(3,1,1)}}{s_{(3,2)}}$.
        \item $\braket{e_{(2,1)}}{h_{(2,1)}}$.
        \item $\braket{p_{(3,2,2,1)}}{p_{(3,2,2,1)}}$.
    \end{enumerate}
\end{Ej}

\begin{ptcbr}
    \begin{enumerate}
        \item From the previous homework we have calculated 
        $$s_{(2,1)}(x,y,z)=x^2y+xy^2+2xyz+x^2z+y^2z+xz^2+yz^2$$
        which tells us in a general that $s_{(2,1)}=m_{(2,1)}+2m_{(1,1,1)}$. Applying the inner product we get
        $$\braket{s_{(2,1)}}{h_{(1,1,1)}}=\braket{m_{(2,1)}}{h_{(1,1,1)}}+2\braket{m_{(1,1,1)}} {h_{(1,1,1)}}=2.$$
        \item As $s_\la$ form an orthonormal basis, it holds that $\braket{s_{(3,1,1)}}{s_{(3,2)}}=0$.
        \item Turning $e_{(2,1)}$ to the monomial basis we obtain 
        \begin{align*}
            e_{(2,1)}(x,y,z)&=(xy+yz+zx)(x+y+z)\\
            &=x^2y+y^2z+z^2x+y^2x+z^2y+x^2z+3xyz\\
        &=m_{(2,1)}(x,y,z)+3m_{(1,1,1)}(x,y,z).
        \end{align*}
        Inputting into the inner product we get 
        $$\braket{e_{(2,1)}}{h_{(2,1)}}=\braket{m_{(2,1)}}{h_{(2,1)}}+3\braket{m_{(1,1,1)}}{h_{(2,1)}}=1$$
        and this is because $m$ and $h$ form a dual pair.
        \item As $(p_\la)$ forms an orthogonal basis, the following holds $\braket{p_\la}{p_\la}=z_\la$ where $z_\la=\prod k^{m_k}m_k!$ and $m_k$ is the number of parts of $\la$ equal to $k$. In this case 
        $$\braket{p_{(3,2,2,1)}}{p_{(3,2,2,1)}}=z_{(3,2,2,1)}=(3^1)(1!)(2^2)(2!)=24.$$
    \end{enumerate}
\end{ptcbr}
\end{document} 
