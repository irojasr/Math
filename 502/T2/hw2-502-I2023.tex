\documentclass[12pt]{memoir}

\def\nsemestre {I}
\def\nterm {Spring}
\def\nyear {2023}
\def\nprofesor {Maria Gillespie}
\def\nsigla {MATH502}
\def\nsiglahead {Combinatorics 2}
\def\nextra {HW2}
\def\nlang {ENG}
\input{../../headerVarillyDiff}
\usepackage{youngtab}

\begin{document}

\begin{Ej}[Exercise 1]
    Evaluate the following expressions using the properties of the Hall inner product discussed in class:\vspace{-0.4em}
    \begin{enumerate}[i)]
        \itemsep=-0.4em
        \item $\braket{s_{(2,1)}}{h_{(1,1,1)}}$.
        \item $\braket{s_{(3,1,1)}}{s_{(3,2)}}$.
        \item $\braket{e_{(2,1)}}{h_{(2,1)}}$.
        \item $\braket{p_{(3,2,2,1)}}{p_{(3,2,2,1)}}$.
    \end{enumerate}
\end{Ej}

\begin{ptcbr}
    \begin{enumerate}
        \item From the previous homework we have calculated 
        $$s_{(2,1)}(x,y,z)=x^2y+xy^2+2xyz+x^2z+y^2z+xz^2+yz^2$$
        which tells us in a general that $s_{(2,1)}=m_{(2,1)}+2m_{(1,1,1)}$. Applying the inner product we get
        $$\braket{s_{(2,1)}}{h_{(1,1,1)}}=\braket{m_{(2,1)}}{h_{(1,1,1)}}+2\braket{m_{(1,1,1)}} {h_{(1,1,1)}}=2.$$
        \item As $s_\la$ form an orthonormal basis, it holds that $\braket{s_{(3,1,1)}}{s_{(3,2)}}=0$.
        \item Turning $e_{(2,1)}$ to the monomial basis we obtain 
        \begin{align*}
            e_{(2,1)}(x,y,z)&=(xy+yz+zx)(x+y+z)\\
            &=x^2y+y^2z+z^2x+y^2x+z^2y+x^2z+3xyz\\
        &=m_{(2,1)}(x,y,z)+3m_{(1,1,1)}(x,y,z).
        \end{align*}
        Inputting into the inner product we get 
        $$\braket{e_{(2,1)}}{h_{(2,1)}}=\braket{m_{(2,1)}}{h_{(2,1)}}+3\braket{m_{(1,1,1)}}{h_{(2,1)}}=1$$
        and this is because $m$ and $h$ form a dual pair.
        \item As $(p_\la)$ forms an orthogonal basis, the following holds $\braket{p_\la}{p_\la}=z_\la$ where $z_\la=\prod k^{m_k}m_k!$ and $m_k$ is the number of parts of $\la$ equal to $k$. In this case 
        $$\braket{p_{(3,2,2,1)}}{p_{(3,2,2,1)}}=z_{(3,2,2,1)}=(3^1)(1!)(2^2)(2!)=24.$$
    \end{enumerate}
\end{ptcbr}

\begin{Ej}[Exercise 2]
    Apply the $\om$ involution to both sides of the Jacobi-Trudi formula to derive a formula
for Schur functions in terms of elementary symmetric functions.
\end{Ej}

\begin{ptcbr}
    Recall that the omega involution is an homomorphism. This means that it respects sums and products which in particular means that it respects polynomials.\par 
    Now, the Jacobi-Trudi formula states that 
    $$s_\la=\det(h_{\la_i-i+j})_{i,j\in[n]}.$$
    As the determinant is a polynomial on its entries we can apply the omega involution on both sides of the equality to obtain 
    $$s_{\la^\sT}=\om\left(\det(h_{\la_i-i+j})\right)=\det(\om(h_{\la_i-i+j}))=\det(e_{\la_i-i+j})_{i,j\in[n]}.$$
    Applying a transposition we get the dual Jacobi-Trudi identity:
    $$s_{\la}=\det(e_{\la^\sT_i-i+j})_{i,j\in[n]}.$$
\end{ptcbr}

\begin{Ej}[Exercise 3]
    Write out the six permutations of $1, 2, 3$ and the pairs of standard Young tableaux corresponding to each under the RSK bijection.
\end{Ej}

\begin{ptcbr}
    As recording tableau are customarily read from the bottom row, we will write the list notation permutations from bottom to top. The following are the elements of $S_3$:
    \begin{gather*}
        \twobythree{1}{2}{3}{1}{2}{3},\ \twobythree{2}{1}{3}{1}{2}{3},\ \twobythree{3}{2}{1}{1}{2}{3}.\ \twobythree{1}{3}{2}{1}{2}{3},\ \twobythree{2}{3}{1}{1}{2}{3},\ \twobythree{3}{1}{2}{1}{2}{3}. 
    \end{gather*}
    Applying the algorithm from $\cM_{2\x n}$ to $(S,T)$ we get the following pairs of tableaux:
    \begin{gather*}
        \left[\young(123),\young(123)\right],\ \left[\young(2,13),\young(2,13)\right],\ \left[\young(3,2,1),\young(3,2,1)\right],\\
        \left[\young(3,12),\young(3,12)\right],\ \left[\young(2,13),\young(3,12)\right],\ \left[\young(3,12),\young(2,13)\right].
    \end{gather*}
    These pairs of tableaux are ordered respectively with the matrices on top. 
\end{ptcbr}

\begin{Ej}[Exercise 4]
    Write out all rearrangments of the letters $1, 1, 2, 3$, and the pairs $(S, T )$ of a semi-standard
and standard Young tableau, respectively, corresponding to each word under the RSK bijection.
\end{Ej}

\begin{Ej}[Exercise 5]
    What two-line array corresponds to the following pair of semistandard Young tableaux
    under the RSK bijection?
    $$\young(4,223,11134),\quad\young(3,223,11112).$$
\end{Ej}

\begin{ptcbr}
    The reading word of our insertion tableau is $322311112$
\end{ptcbr}

Quick questions on problems 6, and 7.
\begin{itemize}
    \item For problem 6 I've seen the result that if $\twobyone{a}{b}$ is a two-line array which under RSK maps to $(S,T)$, then $\twobyone{b}{a}$ maps to $(T,S)$. Assuming that result, proving question 6 is as follows:\par 
    \emph{As permutations correspond to SYT's we have that $\text{RSK}(\pi)=(S,T)$ and $\text{RSK}(\pi^{-1})=(T,S)$. When $S=T$ it holds that}
    $$\text{RSK}(\pi)=(T,T)=\text{rsk}(\pi^{-1})\To \pi=\pi^{-1}$$
    \emph{which means $\pi$ is an involution.} However I don't understand how to prove the result I used because I'm unable to adapt the proof we saw in class to the case of $\pi$ and $\pi^{-1}$.
    \item I can't quite put my head around the last problem. What I'm interpreting is that $s=Km$ then $h=K^{\sT}m$ where $K$ is the respective matrix of Kostka coefficients. But I'm unsure on how to proceed or how to use the inner product.\par 
    Also, going a bit ahead, I've seen the statements of the Pieri rules and that they are used to prove that $h$ and $e$ are Schur positive. So the last sum looked strikingly similar to the statement of $h_{\mu}=\sum_\la K_{\la\mu}s_\mu$. However it might be my mind playing me some tricks.
\end{itemize}
\end{document} 
