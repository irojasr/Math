\documentclass[12pt]{memoir}

\def\nsemestre {I}
\def\nterm {Spring}
\def\nyear {2023}
\def\nprofesor {Maria Gillespie}
\def\nsigla {MATH502}
\def\nsiglahead {Combinatorics 2}
\def\nextra {HW7}
\def\nlang {ENG}
\input{../../headerVarillyDiff}
\usepackage[enableskew]{youngtab}
\usepackage{ytableau}
\DeclareMathOperator{\SYT}{SYT}
\DeclareMathOperator{\inv}{inv}
\DeclareMathOperator{\maj}{maj}
\begin{document}

\begin{Ej}[Exercise 2]
    Compute the following:\vspace{-0.4em}
    \begin{enumerate}[i)]
        \itemsep=-0.4em
        \item $s_{(5,3,3,1)}s_{(4)}$.
        \item $s_{(2,1)}s_{(2,1)}$.
        \item The decomposition of $V_{(2,1)}\ox_oV_{(2,1)}$ into irreducible $S_6$ representations.
    \end{enumerate}
\end{Ej}

\begin{ptcbr}
    Recall that the product of Schur functions can be calculated using Littlewood-Richardson coefficients as follows:
    $$s_\mu s_\nu=\sum_{\la}c^\la_{\mu\nu}s_\la$$
    where $c^{\la}_{\mu\nu}$ is the number of Littlewood-Richardson tableaux of shape $\la/\mu$ with content $\nu$.\par 
    In order to fill out $\la/\mu$ with content $\nu$ it must hold that $|\la|-|\mu|=|\nu|$.
    \begin{enumerate}[i)]
        \itemsep=-0.4em
        \item In this first case we have $\mu=(5,3,3,1)$ and $\nu=(4)$ which means that $\la$ should partition $16$.\par 
        This means that we must append 4 new blocks to our partition $(5,3,3,1)$ to obtain a partition of $16$ which we will fill with only ones.\par 
        As the tableaux must be semi-standard, it can't happen that we append more than one block in the same column. In other words, we must append a horizontal strip of ones. The following tableaux correspond to partitions $\la$ such that $\la/\mu$ is a horizontal strip:
        \begin{itemize}
            \begin{multicols*}{3}
                \item $\young(1,~11,~~~,~~~1,~~~~~)$
                \item $\young(1,~11,~~~,~~~,~~~~~1)$
                \item $\young(1,~1,~~~,~~~11,~~~~~)$
                \item $\young(1,~1,~~~,~~~1,~~~~~1)$
                \item $\young(1,~1,~~~,~~~,~~~~~11)$
                \item $\young(1,~,~~~,~~~11,~~~~~1)$
                \item $\young(1,~,~~~,~~~1,~~~~~11)$
                \item $\young(1,~,~~~,~~~,~~~~~111)$
                \item $\young(~11,~~~,~~~11,~~~~~)$
                \item $\young(~11,~~~,~~~1,~~~~~1)$
                \item $\young(~11,~~~,~~~1,~~~~~1)$
                \item $\young(~1,~~~,~~~11,~~~~~1)$
                \item $\young(~1,~~~,~~~1,~~~~~11)$
                \item $\young(~1,~~~,~~~,~~~~~111)$
                \item $\young(~,~~~,~~~11,~~~~~11)$
                \item $\young(~,~~~,~~~1,~~~~~111)$
                \item $\young(~,~~~,~~~,~~~~~1111)$
            \end{multicols*}
        \end{itemize}
        As there is only one possible tableau for each shape we have that each coefficient is one. Therefore we have 
        \begin{align*}
            s_{(5,3,3,1)}s_{(4)}=&s_{(5,4,3,3,1)}+s_{(6,3,3,3,1)}+s_{(5,5,3,2,1)}+s_{(6,4,3,2,1)}+s_{(7,3,3,2,1)}\\
            +&s_{(6,5,3,1,1)}+s_{(7,4,3,1,1)}+s_{(8,3,3,1,1)}+s_{(5,5,3,3)}+s_{(6,4,3,3)}+s_{(7,3,3,3)}\\
            +&s_{(6,5,3,2)}+s_{(7,4,3,2)}+s_{(8,3,3,2)}+s_{(7,5,3,1)}+s_{(8,4,3,1)}+s_{(9,3,3,1)}.    
        \end{align*}
        \item For this next case we have $\mu = \nu =(2,1)$. This means that $\la\vdash 6$ and we must append two $1$'s and one $2$. As our tableau must be Littlewood-Richardson, the reading word must be one of either 
        $$211,\word{or}121.$$
        The following tableaux correspond to $\la\vdash 6$ such that $\la/\mu$ is a Littlewood-Richardson tableau with content $(2,1)$:
        \begin{itemize}
            \begin{multicols*}{3}
                \item $\young(2,1,~1,~~)$
                \item $\young(2,1,~,~~1)$
                \item $\young(2,~1,~~1)$
                \item $\young(2,~,~~11)$
                \item $\young(1,~2,~~1)$
                \item $\young(~2,~~11)$
                \item $\young(12,~1,~~)$
                \item $\young(~12,~~1)$
            \end{multicols*}
        \end{itemize}
        We see that $\text{sh}(\la)=(3,2,1)$ is repeated once so we account for a coefficient of $2$ with that tableau. The decomposition is thus:
        $$s_{(2,1)}^2=s_{(2,2,1,1)}+s_{(3,1,1,1)}+2s_{(3,2,1)}+s_{(4,1,1)}+s_{(4,2)}+s_{(2,2,2)}+s_{(3,3)}.$$
        \item By the correspondence of the outer tensor product with the product of Schur functions we have 
        $$V_{(2,1)}\ox_o V_{(2,1)}=V_{(2,2,1,1)}\oplus V_{(3,1,1,1)}\oplus2V_{(3,2,1)}\oplus V_{(4,1,1)}\oplus V_{(4,2)}\oplus V_{(2,2,2)}\oplus V_{(3,3)}.$$
        
\end{enumerate}    
\end{ptcbr}

\begin{Ej}[Exercise 3]
    Use the Straightening Algorithm (Garnir Relations) to express the Garnir polynomial $F_T$ where $T$ is the filling 
    $$\young(24,31)$$
    in terms of $F_S$'s where each $S$ is a standard Young tableau.
    Write out each polynomial in your formula and check that your answer works.
\end{Ej}

\begin{ptcbr}
    We first column straighten $T$ by ordering its columns, we obtain
    $$\young(34,21).$$
    To row-straighten we find the topmost row with a descent and consider rightmost decrease. The block in question is 
    $$\young(3,21)$$
    and for this block we will find permutations of $S_3$ which preserve the columns-increasing condition withing the blocks. We have the permutations 
    $$\id,\quad (12),\quad (123)$$
    which correspond to 
    $$\young(3,21),\quad \young(3,12),\word{and}\young(2,13).$$
    This means that 
    $$(-1)^0(x_3-x_2)(x_4-x_1)+(-1)^1(x_3-x_1)(x_4-x_2)+(-1)^2(x_2-x_1)(x_4-x_3)=0$$
    and from this relation we obtain 
    $$(x_3-x_2)(x_4-x_1)=(x_3-x_1)(x_4-x_2)-(x_2-x_1)(x_4-x_3).$$
    Similarly just by calculating the Garnir polynomial of $T$ we obtain $(x_2-x_3)(x_4-x_1)$ which corresponds to our formula, just with a sign change.
\end{ptcbr}

\begin{Ej}[Exercise 5]
    Decompose the inner tensor product $V(2,1)\ox_i V(2,1)$ into irreducible representations of $S_3$.
\end{Ej}

\begin{ptcbr}
Each of the factors in the product is generated by the pair of polynomials 
$$(x_2-x_1),(x_3-x_1),\word{and}(y_2-y_1),(y_3-y_1)$$
and therefore the tensor product is generated by the pairwise products of these polynomials. Let us call 
$$
\left\lbrace
\begin{aligned}
    &v_1=(x_2-x_1)(y_2-y_1)=x_2y_2-x_2y_1-x_1y_2+x_1y_1\\
    &v_2=(x_2-x_1)(y_3-y_1)=x_2y_3-x_2y_1-x_1y_3+x_1y_1\\
    &v_3=(x_3-x_1)(y_2-y_1)=x_3y_2-x_3y_1-x_1y_2+x_1y_1\\
    &v_4=(x_3-x_1)(y_3-y_1)=x_3y_3-x_3y_1-x_1y_3+x_1y_1
\end{aligned}
\right.
$$
We now consider the action of $S_3$ on each of our basic elements. It suffices to consider the action per representative of cycle type, so we only consider $\id,(12)$ and $(123)$.\par 
Applying the identity to each of our elements returns it as it was. So we have 
$$M_{\id}=\begin{pmatrix}
    1&0&0&0\\0&1&0&0\\0&0&1&0\\0&0&0&1
\end{pmatrix}$$
Now, we apply the transposition $(12)$ to our elements to obtain 
\begin{itemize}
    \item $(12)v_1=(x_1-x_2)(y_1-y_2)=v_1$
    \item $(12)v_2=(x_1-x_2)(y_3-y_2)=x_1y_3-x_1y_2-x_2y_3+x_2y_2$
    \item $(12)v_3=(x_3-x_2)(y_1-y_2)=x_3y_1-x_3y_2-x_2y_1+x_2y_2$
    \item $(12)v_4=(x_3-x_1)(y_3-y_1)=x_3y_3-x_3y_2-x_2y_3+x_2y_2$
\end{itemize}
On the other hand notice that 
\begin{align*}
    v_1-v_2&=x_2y_2-x_2y_1-x_1y_2+x_1y_1-(x_2y_3-x_2y_1-x_1x_3+x_1y_1)\\
    &=x_2y_2-x_2y_3-x_1y_2+x_1y_3\\
    v_1-v_3&=x_2y_2-x_2y_1-x_1y_2+x_1y_1-(x_3y_2-x_3y_1-x_1y_2+x_1y_1)\\
    &=x_2y_2-x_2y_1-x_3y_2+x_3y_1\\
    v_1-v_2-v_3+v_4&=x_2y_2-x_2y_3-x_1y_2+x_1y_3-(x_3y_2-x_3y_1-x_1y_2+x_1y_1)\\
    &\hphantom{=}+x_3y_3-x_3y_1-x_1y_3+x_1y_1\\
    &=x_2y_2-x_2y_3-x_3y_2+x_3y_3
    \end{align*} 
We can identity the image of the transposition with this elements to build the matrix 
$$M_{(12)}=\begin{pmatrix}
    1&0&0&0\\1&-1&0&0\\1&0&-1&0\\1&-1&-1&1
\end{pmatrix}$$
And in a similar fashion we may calculate the image of our basic elements through the $3$-cycle:
\begin{itemize}
    \item $(123)v_1=(x_3-x_2)(y_3-y_2)=v_1-v_2-v_3+v_4$
    \item $(123)v_2=(x_3-x_2)(y_1-y_2)=v_1-v_3$
    \item $(123)v_3=(x_1-x_2)(y_3-y_2)=v_1-v_2$
    \item $(123)v_4=(x_1-x_2)(y_1-y_2)=v_1$
\end{itemize}
With this information we may build
$$M_{(123)}=\begin{pmatrix}
    1&-1&-1&1\\1&0&-1&0\\1&-1&0&0\\1&0&0&0
\end{pmatrix}$$
The traces of these matrices allow us to build character table for $V(2,1)\ox_i V(2,1)$:
    $$\chi_{\text{inner}}=\begin{pmatrix}
        4&0&1
    \end{pmatrix}$$
To find the decomposition we solve a system of linear equations by row reducing the following matrix:
$$\begin{pmatrix}
    1&1&2&4\\1&-1&0&0\\1&1&-1&1
\end{pmatrix}\xrightarrow[]{\text{RREF}}\begin{pmatrix}
    1&0&0&1\\0&1&0&1\\0&0&1&1
\end{pmatrix}$$
which means that
$$V(2,1)\ox_i V(2,1)=V_{(3)}\oplus V_{(1,1,1)}\oplus V_{(2,1)}.$$
\end{ptcbr}

\begin{Ej}[Exercise 6]
    Use the Murnaghan-Nakayama rule to compute $\braket{s_{(3,3,1)}}{p_{(3,3,1)}}$.
\end{Ej}

\begin{ptcbr}
    Recall that the Murnaghan-Nakayama rule has the following statement:
    $$\chi_\mu(\la)=\braket{s_\mu}{p_\la}=\sum_{(\ast)}\prod_{i=1}^{\ell(\la)}(-1)^{\operatorname{ht}(\text{strip}(i))}$$
    where the sum runs through border strip tableaux $T$ with shape $\mu$ and content $\la$. In this case we have $\mu=\la=(3,3,1)$ which means we must fill the diagram
    $$\young(~,~~~,~~~)$$
    with the word $1112223$ only forming border strips. The only possible fillings are 
    $$\young(3,222,111),\quad\young(2,223,111),\quad\young(3,122,112),\quad\young(1,123,122)$$
    Also, recall that the height of the strip is one less than the \emph{actual height}. With this we have the corresponding height vectors:
    $$(0,0,0),\quad(0,1,0),\quad(1,1,0),\quad(2,1,0)$$
    and we may calculate $(-1)^\vec{v}$ for each of these to obtain 
    $$\braket{s_{(3,3,1)}}{p_{(3,3,1)}}=1-1+1-1=0.$$
    Then $s_{(3,3,1)}$ and $p_{(3,3,1)}$ form a dual pair. \emph{Is this true in general for any $\la$?}

\end{ptcbr}
\end{document} 
