\documentclass[12pt]{memoir}

\def\nsemestre {I}
\def\nterm {Spring}
\def\nyear {2023}
\def\nprofesor {Maria Gillespie}
\def\nsigla {MATH502}
\def\nsiglahead {Combinatorics 2}
\def\nlang {ENG}
\input{../../headerVarillyDiff}

\begin{document}
%\clearpage
\maketitle
%\thispagestyle{empty}
{\small
\setlength{\parindent}{0em}
\setlength{\parskip}{1em}



\subsubsection*{Requirements}

}
\newpage
\tableofcontents
%\begin{multicols}{2}
\chapter{Symmetric functions}

\section{Recall}

\begin{Def}
$f(x_1,x_2,\dots)$ is \term{symmetric} if it's fixed under permutations of variables.
\end{Def}

\begin{Ex}
    $f(x_1,\dots,x_4)=x_1^5+\dots+x_4^5$. This is known as $p_5$ or $m_{(5)}$, where $p$ is the power-sum symmetric function and $m$, the monomial symmetric function.  
\end{Ex}

\begin{Ex}
    Consider $g=x_1^4x_2+x_1^4x_3+\dots+x_i^4x_j+\dots+3x_1+\dots=m_{(4,1)}+3m_{(1)}$.
\end{Ex}

Let us recall some \textbf{notation}:
\begin{enumerate}[i)]
    \itemsep=-0.4em
    \item $\La_R(x_1,\dots,x_n)$ is the ring of symmetric polynomials over $R$. In \emph{infinitely} many variables we have $\La_R(\un{x})$.
\end{enumerate}

In the case $R=\bQ$, then $\dim \La_Q(\un x)_{(d)}$, where every monomial has degree $d$, is $p(d)$. This is the number of partitions of $d$. Because for every partition we can form monomials and monomials form a basis.

\subsection*{Bases of $\La_Q$}

Suppose $\la=(\la_1,\dots,\la_k)$ with $\la_1\geq\dots\geq\la_k$. 
\begin{itemize}
    \item Monomial: $m_\la=\sum_{i_1\neq\dots i_k}x_{i_1}^{\la_1}\dots x_{i_k}^{\la_k}$.
    \item Elementary: $e_\la=\prod e_{\la_i}$ where $e_d=m_{(1,1,\dots,1)}$ ($d$ ones).
    \item Homogenous: $h_\la=\prod h_{\la_i}$ and $h_d=x_1^d+\dots+x_1^{d-1}x_2+\dots+x_1^{d-2}x_2^2+x_1^{d-2}x_2x_3+\dots$. In general $h_d=\sum_{\la\vdash d}m_\la$.
    \item Power sum: $p_\la=\prod p_{\la_i}$ and $p_d=\sum x_i^d$.
\end{itemize}

For Schur basis recall SSYT 

\begin{Ex}
    Consider $\la=(5,4,1)$, rows $\leq\to$ and columns $<$, we associate the monomial $x_1^2x_2^3x_3^3x_4^2:=x^T$.
\end{Ex}

\begin{itemize}
    \itemsep=-0.4em
    \item Schur: $s_\la=\sum_{T\in SSYT(\la)}x^T$ but also $\sum K_{\la\mu}m_\mu$ where the sum is over SSYT of shape $\la$, content $\mu$.
\end{itemize}

\subsubsection{Schur function motivation (preview)}

The first place they showed up is in the representation theory of Lie group.  The function $s_\la(x_1,\dots,x_n)$ is a character of irreducible polynomial representations of $GL_n$. In theoretical physics we have matrix groups acting on particles, representations are smaller matrix groups of things that they are mapping to. We want to take tensor product and direct sums of representations, the tensor product is related to multiplication of Schur function while direct sum into sum of Schur functions.\par 
There's also the Schur-Weyl duality which takes representations into the Weyl group. Under the \emph{Frobenius map}, $s_\la$ corresponds to irreducible representations of $S_n$.\par 
A more modern application of Schur function goes into geometry, $s_\la$ correspond to Schubert varieties in Grassmannians. Multiplication corresponds to interesections and sum to unions.\par 
There's also context in Probability Theory. But in the end, Schur positivity is important because of this connections. 

\begin{Def}
    $f\in\La$ is \term{Schur-positive} if $f=\sum c_\la s_\la$, $c_\la\geq 0$.
\end{Def}

\begin{Ex}
    $3s_{(2,1)}+2s_{(3)}$ schur pos but change $2$ to $-\half$ then not.
\end{Ex}

\section{day 2}

\subsection*{Alg defn Schur fncs}

\begin{Def}
    A function is \term{antisymmetric} if for $\pi\in S_n$,
    $$f(x_{\pi(1)},\dots,x_{\pi(n)})=\sgn(\pi)f(x_1,\dots,x_n).$$
\end{Def}

\begin{Ex}
    The following functions are antisymmetric:
    \begin{enumerate}
        \itemsep=-0.4em
        \item $f(x,y)=x-y$ then $f(y,x)=-f(x,y)$.
        \item $g(x,y)=(x-y)(x+y)$.
        \item $h(x,y)=x^2y-y^2x$.
    \end{enumerate}
\end{Ex}

Notice that the last function can factor as $h=-xy(x-y)$. We claim that this is always the case.

\begin{Lem} 
    Every antisymmetric polynomial $f$ in two variables $x,y$ can factor as $f(x,y)=(x-y)g(x,y)$ where $g$ is symmetric.
\end{Lem}

\begin{ptcbp}
Suppose $f$ is antisymmetric, then $f(x,x)=0$ by taking $y=x$. This means that $(x-y)\mid f$. Thus $f(x,y)=(x-y)g(x,y)$ and we now need to show that $g$ is symmetric. 
$$g(y,x)=\frac{f(y,x)}{y-x}=\frac{-f(x,y)}{-(x-y)}=\frac{f(x,y)}{x-y}=g(x,y).$$
\end{ptcbp}

\subsubsection*{Monomial Antisymmetric Functions}

\begin{Def}
Given a strict partition $\la=(\la_1,\dots,\la_k)$, $\la_1>\dots>\la_k$, we define 
$$a_\la(x_1,\dots,x_n)=x_1^{\la_1}\cdots x_k^{\la_k}\pm\text{similar terms}=\sum_{\pi\in S_n}\sgn(\pi)\prod_{k}x_{\pi(k)}^{\la_k}.$$ 
This $a_\la$ can be zero. 
\end{Def}

\begin{Ex}
    For two variables we've seen some antisymmetric polynomials. Let us calculate 
    $$a_{(3,1)}(x,y)=x^3y-y^3x.$$
    The smallest possible example in 3 variables is 
    $$a_{(2,1,0)}(x,y,z)=x^2y+y^2z+z^2x-y^2x-z^2y-x^2z.$$
    This can be factored as $(x-y)(y-z)(x-z)$. A similar construction gives us
    $$a_{(4,2,0)}(x,y,z)=x^4y^2+y^4z^2+z^4x^2-y^4x^2-z^4y^2-x^4z^2,$$
    but how does this factor? We get 
    $$a_{(4,2,0)}(x,y,z)=(x^2-y^2)(y^2-z^2)(x^2-z^2)=a_{(2,1,0)}(x,y,z)(x+y)(y+z)(x+z).$$
\end{Ex}

\begin{Lem}
The set $\set{a_\la}_{\la\ \text{strict}}$ is a basis of the antisymmetric polynomials over $\bQ$, $A_\bQ$. Even more any $a_\la$ is divisible by $a_\rho$ where $\rho=(n-1,n-2,\dots,2,1,0)$. 
\end{Lem}

As an algebra generator, $a_\rho$ is a generator.
\begin{ptcbp}
    \red{WRITE}
\end{ptcbp}

\begin{Prop}
The $a_\rho$ antisymmetric function is also the \term{Vandermonde determinant}: 
$$a_\rho=\det\begin{pmatrix}
    x_1^{n-1}&x_1^{n-2}&\dots&x_1^2&x_1&1\\
    x_2^{n-1}&x_2^{n-2}&\dots&x_2^2&x_2&1\\
    \vdots&\vdots&\ddots&\vdots&\vdots&\vdots\\
    x_n^{n-1}&x_n^{n-2}&\dots&x_n^2&x_n&1\\
\end{pmatrix}$$
\end{Prop}

\subsubsection{Schur Polynomials}

\begin{Def}
    The \term{Schur polynomial} of $\la\in\text{Par}$ is 
    $$s_\la(x_1,\dots,x_n)=\frac{a_{\la+\rho}(\un{x})}{a_\rho(\un x)}.$$
    Here $\la+\rho$ is the pointwise sum as arrays.
\end{Def}

\begin{Rmk}
This is the Weyl character proof. 
\end{Rmk}

The following proof is due to Proctor(1987) \red{find ref}

\begin{Lem}
    Any $a_\la$ can be seen as a determinant in the following way:
    $$a_\la(\un x)=\det\begin{pmatrix}
        x_1^{\la_1}&x_1^{\la_2}&\dots&x_1^{\la_n}\\
        x_2^{\la_1}&x_2^{\la_2}&\dots&x_2^{\la_n}\\
        \vdots&\vdots&\ddots&\vdots\\
        x_n^{\la_1}&x_n^{\la_2}&\dots&x_n
    \end{pmatrix}$$
\end{Lem}
\begin{ptcbp}
    We want to see that 
    $$\frac{a_{\la+\rho}(\un{x})}{a_\rho(\un x)}=\sum x^T$$
    where the sum ranges through $T$'s which are SSYT(la) with max entry $n$. 
    \begin{enumerate}
        \item We will show a recursion for the combinatorial definition that the character formula will also satisfy. It holds that 
        $$s_\la(\un x)=\sum s_\mu(\un x)x_n^{|\la|-|\mu|}$$
        where $\mu$ has $n-1$ parts with $\la_1\geq\mu_1\geq\la_2\geq\mu_2\dots$. 
        \item We also show that the ratio of determinants satisfies the same recursion. 
    \end{enumerate}
\end{ptcbp}

\begin{Ex}
    Consider $\la=(8,8,4,1,1)$ and $\mu=(8,5,2,1)$, then $\la\less\mu$ is a skew-table in which we can fill in $n$'s
\end{Ex}

\begin{Cor}
The Schur polynomials are a basis of $\La_\bQ$. 
\end{Cor}

\section{Day 3|20230125}

Recall $\La=\bQ[e_1,e_2,\dots]$ where the $e_j$'s are the elementary symmetric functions. So the $e_j$'s are algebraic generators of $\La$ and they're algebraically independent. Equivalently, as a vector space, $\set{e_\la\:\ \la\in\text{Par}}$ is a basis.

\begin{Prop}
    A homomorphism $f\:\La\to\La$ ($f(a+b)=f(a)+f(b),\ f(ab)f(a)f(b)$ for $a,b\in\La$) is fully determined by where it sends the $e_i's$. 
\end{Prop}

\begin{Def}
    The map $\om\in\End(\La)$ will send $e_j$ to $h_j$. 
\end{Def}

\begin{Ex}
    Consider $f=3e_{(2,1)}+2e_3$, then applying $\om$ we get 
    $$\om(f)=\om(3e_{(2,1)}+2e_3)=3h_{(2,1)}+2h_3.$$
    For $p_2$, we can decompose to $e_1^2-2e_2$. So 
    $$\om(p_2)=\om(e_1^2-2e_2)=h_1^2-2h_2$$
    and we can expand this last expression into 
    $$(x_1+x_2+\dots)^2-2(x_1^2+x_2^2+\dots+x_1x_2+x_1x_3+\dots)=-x_1^2-x_2^2-\dots$$
    and we recognize this last term as $-p_2$. \emph{This is not a coincidence.}
\end{Ex}

\begin{Th}
The map $\om$ is involutive.
\end{Th}

\begin{ptcbp}
    It suffices to prove that $\om(h_j)=e_j$. We will use power expansions and generating functions. We have 
    $$H(t)=\frac{1}{1-x_1t}\frac{1}{1-x_2t}\dots=\sum h_n(\un x)t^n,$$
    and this comes from expanding the $1/(1-y)$'s as geometric series. When collecting the coefficients of $t^n$ we get exactly $h_n(\un x)$. Similarly, for the elementary symmetric functions, 
    $$E(t)=(1+x_1t)(1+x_2t)\dots=\sum e_nt^n.$$
    When multiplying to obtain the coefficient of $t^n$ we get a plethora of different $x_j$'s which form the $e_j$'s. Now from this expressions we have $H(t)E(-t)=1$ which means that
    $$\left(\sum h_n(\un x)t^n\right)\left(\sum e_n(\un x)(-t)^n\right)\To \sum_{k=0}^{n}(-1)^ke_kh_{n-k}=0,\ n\geq 1.$$
    Now applying the map to the equation we get 
    $$\om\left(\sum_{k=0}^{n}(-1)^ke_kh_{n-k}\right)=\sum_{k=0}^{n}(-1)^kh_k\om(h_{n-k})=0.$$
    After reindexing, we get that both $e_j$'s and $\om(h_j)$'s are determined recursively by the $h_j$'s in the same way. Thus we conclude that $\om(h_j)=e_j$. 
\end{ptcbp}

\begin{Lem}
    The following equation holds for the power-sum symmetric functions:
    $$\exp\left(\sum\frac{1}{n}p_n(\un x)p_n(\un y)\right)=\prod_{i,j=1}^\infty\frac{1}{1-x_iy_j}=\:\Om(\un x,\un y).$$
    It also holds that 
    $$\Om(\un x,\un y)=\sum_la\frac{1}{z^\la}p_\la(\un x)p_\la(\un y)$$
    where $z_\la=\prod k^{m_k}k!$ where $m_k$ is the number of parts of $\la$ equal to $k$. 
\end{Lem}

\begin{ptcbp}
    We will prove both parts separately. For the first equation we will take the logarithm on both sides: 
    $$\sum\frac{1}{n}p_n(\un x)p_n(\un y)=\log\left(\prod_{i,j=1}^\infty\frac{1}{1-x_iy_j}\right)$$
    and after manipulating the logarithm we get 
    $$\sum_{i,j=1}^\infty(\log(1)-\log(1-x_iy_j))=\sum_{i,j=1}^\infty\sum_{n=1}^\infty \frac{1}{n}x_i^ny_j^n.$$
    We can separate\footnote{Are we using Fubini-Tonelli here?} into 
    $$\sum_{n=1}^\infty\frac{1}{n}\left(\sum_i x_i^n\right)\left(\sum_j y_j^n\right).$$
    Now taking $\exp$ on both sides we get equality.\par 
    By not removing the exponential we get the following expression
    $$\exp\left(\sum\frac{1}{n}p_n(\un x)p_n(\un y)\right)=\sum_{k=0}^\infty\frac{1}{k!}\left(\sum\frac{1}{n}p_n(\un x)p_n(\un y)\right)^k.$$
    To get a term of the form $p_\la(\un x)p_\la(\un y)$ we have to choose which parts of the $\la$ come from each of the factors in $\sum\frac{1}{n}p_n(\un x)p_n(\un y)$. If $\l(\la)=k$ then it comes from the $k^\textsuperscript{th}$ term in the exponential sum. If $\la=(\la_1,dots,\la_1,\dots,2,\dots,2,1,\dots,1)$ with $m_{\la_1}$ $\la_1$'s, $m_1$ $1$'s, then out of $k$ elements we have to choose $m_1$ $1$'s and so on. Thus there are $\binom{k}{m_{\la_1},\dots,m_1}$ choices and each $i$ in $\la$ comes with a $\frac{1}{i}$. Therefore the coefficient of $p_\la(\un x)p_\la(\un y)$ is 
    $$\frac{1}{k!}\frac{k!}{m_1!m_2!\dots}\frac{1}{1^{m_1}}\frac{1}{2^{m_2}}\dots=\frac{1}{z_\la}.$$
\end{ptcbp}

\begin{Lem}
    We have the following identities 
    $$\exp\left(\sum\frac{(-1)^{n-1}}{n}p_n(\un x)p_n(\un y)\right)=\prod_{i,j=1}^\infty\frac{1}{1+x_iy_j}=\sum_\la\frac{(-1)^{n-\l(\la)}}{z_\la}p_\la(\un x)p_\la(\un y).$$
\end{Lem}

\begin{Lem}
    Another equality for $\Om(\un x,\un y)$ is 
    $$\Om(\un x,\un y)=\sum_\la m_\la(\un x)h_\la(\un y)$$
\end{Lem}

\begin{Th}
    It holds that $\om(p_\la)=(-1)^{n-k}p_\la$ where $k$ is the number of parts of $\la$.
\end{Th}

\begin{ptcbp}
    Applying $\om$ to $\Om$, but \emph{only working with $\un y$ variables} we get 
    $$\om(\Om)=\om\left(\sum_\la m_\la(\un x)h_\la(\un y)\right)=\sum_\la m_\la(\un x)e_\la(\un y)=\prod_{i,j=1}^\infty (1+x_iy_j)=\sum_{\la}\frac{1}{z_\la}(-1)^{n-k_\la}p_\la(\un x)p_\la(\un y).$$
    Comparing coefficients with 
    $$\om\left(\sum_la\frac{1}{z^\la}p_\la(\un x)p_\la(\un y)\right)$$
    we get the result.
\end{ptcbp}

\end{document}
