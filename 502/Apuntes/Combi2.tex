\documentclass[12pt]{memoir}

\def\nsemestre {I}
\def\nterm {Spring}
\def\nyear {2023}
\def\nprofesor {Maria Gillespie}
\def\nsigla {MATH502}
\def\nsiglahead {Combinatorics 2}
\def\nlang {ENG}
%\def\darktheme{}

\makeatletter
\ifx \nauthor\undefined
  \def\nauthor{Ignacio Rojas}
\else
\fi

\ifx \nextra \undefined
\ifx \nlang \undefined
\author{Basado en las clases impartidas por \nprofesor \\\small Notas tomadas por \nauthor}
\else
\author{Based on the lectures by \nprofesor \\\small Notes written by \nauthor}
\fi
\else
\author{\nauthor}
\fi
\date{\nterm\ \nyear}

%%%%%%%%%%%%%
%% 1. Pacotes
%%%%%%%%%%%%%

\usepackage{alltt}
\usepackage{amsfonts}
\usepackage{amsmath}
\usepackage{amssymb}
\usepackage{amsthm}
\usepackage{algorithm}
\usepackage[noend]{algpseudocode}
\usepackage{array}
\newcommand\hmmax{0} % default 3
\newcommand\bmmax{0} % default 4 %%tex.se/3676,219310
%\usepackage{bbold}
\usepackage{bm}
\usepackage{booktabs}
%\usepackage{caption}
%\usepackage{cancel}
%\usepackage{dsfont}
\usepackage{esint}
\usepackage{fancyhdr}
\usepackage{graphicx}
\usepackage[utf8]{inputenc}
\usepackage{listings}
\usepackage{mathabx}
\usepackage[cal=euler]{mathalfa}
%\usepackage[cal=euler,frak=euler]{mathalfa} % mathcal (JIRR) precisabamos correr initexmf --mkmaps en cmd JCVDG
\usepackage{mathdots}
\usepackage{mathrsfs}
%\usepackage{mathtools}
\usepackage{microtype}
\usepackage{multicol}
\usepackage{multirow}
\usepackage[theoremfont,largesc,tighter,osf]{newpxtext} %JCV Diff
\let\widering\undefined
%\usepackage[bigdelims,vvarbb]{newpxmath} %JCVDG
%por alguna razón esto afectaba las tildes en \min, \lim y demás
%\usepackage{pdflscape}
\usepackage{pgfplots}
\usepackage{physics}
\usepackage{siunitx}
\usepackage{slashed}
%\usepackage{stmaryrd}
%\SetSymbolFont{stmry}{bold}{U}{stmry}{m}{n}
%\usepackage{subfigure}
\usepackage{subcaption}
\usepackage{tabularx}
\usepackage[breakable,skins]{tcolorbox}
\usepackage{textcomp} %%JCVDG
\usepackage{tikz}
\usepackage{tkz-euclide}
\usepackage[normalem]{ulem}
\usepackage[all]{xy}
\usepackage{imakeidx}
\ifx \nlang \undefined
\usepackage[spanish]{babel}
\else\fi 
\usepackage{wrapfig}

%%%%%%%%%%%%%%%%%%%%
%% 2. Document Setup
%%%%%%%%%%%%%%%%%%%%

\ifx \nextra \undefined
    \ifx \nlang \undefined
    \makeindex[intoc, title=Índice Analítico] %Título de índice analítico
    %El índice general es aquel en el que se indican los capítulos, títulos y subtítulos del libro.
    %Índice onomástico es donde aparece el nombre de personas mencionadas en el texto, por orden alfabético con el número de las páginas donde aparecen.
    %El índice analítico se refiere a los temas y conceptos que aparecen en el libro
    \indexsetup{othercode={\fancyhead[LE]{\emph{Índice Analítico}}}}
    \else
    \makeindex[intoc, title=Index] 
    \indexsetup{othercode={\fancyhead[LE]{\emph{Index}}}}
    \fi
  \usepackage[pdftex,
    hidelinks,
    pdfauthor={\nauthor},
    pdfsubject={Notas: \nsiglahead\ \nsemestre-\nyear},
    pdftitle={Semestre \nsemestre\ - \nsigla},
  pdfkeywords={UCR Costa Rica Matem\'aticas Mate \nsemestre\ \nterm\ \nyear\ \nsiglahead}]{hyperref}
  \title{\nsigla\ --- \nsiglahead}
\else
  \usepackage[pdftex,
     hidelinks,
    pdfauthor={\nauthor},
    pdfsubject={\nextra \nsiglahead\ \nsemestre-\nyear},
    pdftitle={Semestre \nsemestre\ - \nsigla},
  pdfkeywords={UCR Costa Rica Matem\'aticas Mate \nsemestre\ \nterm\ \nyear\ \nsiglahead\ \nextra}]{hyperref}

  \title{\nsigla\ --- \nsiglahead \\ {\Large \nextra}}
  \renewcommand\printindex{}
\fi

\pgfplotsset{compat=1.12}


\pagestyle{fancy}
\setlength{\headheight}{15.72pt} %preceding warning said make it at least this


\ifx \nsiglahead \undefined
\def\nsiglahead{\nsigla}
\fi

\lhead{} %%%empty lhead
\rfoot{\thepage}

\ifx \nextra \undefined
  \chead{
    \ifnum\thepage=1
    \else
      \ifx \nlang \undefined
      \textbf{Notas \nsiglahead\ \nsemestre-\nyear}
      \else
      \textbf{Notes \nsiglahead\ \nsemestre-\nyear}
      \fi
    \fi}
  \rhead{}%\firstxmark} % Top right header
\else
%    \chead{
%    \ifnum\thepage=1
%    \else
%      \textbf{Notas \nsiglahead\ \nsemestre-\nyear \ (\nextra)}
%    \fi}
     \chead{
       \textbf{\nextra\ \nsigla\ \nsemestre-\nyear}
     }
     \rhead{
       \textbf{\nauthor}
     }
\fi
\lfoot{}%\lastxmark} % Bottom left footer
\cfoot{} % Bottom center footer

\usetikzlibrary{arrows.meta}
\usetikzlibrary{decorations.markings}
\usetikzlibrary{decorations.pathmorphing}
\usetikzlibrary{positioning}
\usetikzlibrary{fadings}
\usetikzlibrary{intersections}
\usetikzlibrary{cd}

\ifx \nhtml \undefined
\else
  \renewcommand\printindex{}
  \DisableLigatures[f]{family = *}
  \let\Contentsline\contentsline
  \renewcommand\contentsline[3]{\Contentsline{#1}{#2}{}}
  \renewcommand{\@dotsep}{10000}
  \newlength\currentparindent
  \setlength\currentparindent\parindent

  \newcommand\@minipagerestore{\setlength{\parindent}{\currentparindent}}
  \usepackage[active,tightpage,pdftex]{preview}
  \renewcommand{\PreviewBorder}{0.1cm}

  \newenvironment{stretchpage}%
  {\begin{preview}\begin{minipage}{\hsize}}%
    {\end{minipage}\end{preview}}
  \AtBeginDocument{\begin{stretchpage}}
  \AtEndDocument{\end{stretchpage}}

  \newcommand{\@@newpage}{\end{stretchpage}\begin{stretchpage}}

  \let\@real@section\section
  \renewcommand{\section}{\@@newpage\@real@section}
  \let\@real@subsection\subsection
  \renewcommand{\subsection}{\@ifstar{\@real@subsection*}{\@@newpage\@real@subsection}}
\fi
\ifx \ntrim \undefined
\usepackage[shortlabels]{enumitem} %mfw package order matters por savetrees
\else
  \usepackage{geometry}
  \geometry{
    papersize={379pt, 699pt},
    textwidth=345pt,
    textheight=596pt,
    left=17pt,
    top=54pt,
    right=17pt
  }
  \headwidth=345pt
 \usepackage[extreme]{savetrees}
\fi

\ifx \darktheme\undefined
\else
\pagecolor[rgb]{0.2,0.231,0.302}%{0.23,0.258,0.321}
\color[rgb]{1,1,1}
\fi

\ifx \nextra \undefined
\let\@real@maketitle\maketitle
\renewcommand{\maketitle}{\@real@maketitle\begin{center}\begin{minipage}[c]{0.9\textwidth}\centering\footnotesize 
  \ifx \nlang \undefined
  Estas notas no están respaldadas por los profesores y han sido modificadas (a menudo de manera significativa) después de las clases. No están lejos de ser representaciones precisas de lo que realmente se dio en clase y en particular todos los errores son casi seguramente míos.
  \else 
  Please note that these notes were not provided or endorsed by the lecturer and have been significantly altered after the class. They may not accurately reflect the content covered in class and any errors are solely my responsibility.
  \fi
\end{minipage}\end{center}}
\else
\fi

\def\moverlay{\mathpalette\mov@rlay}
\def\mov@rlay#1#2{\leavevmode\vtop{%
   \baselineskip\z@skip \lineskiplimit-\maxdimen
   \ialign{\hfil$\m@th#1##$\hfil\cr#2\crcr}}}
\newcommand{\charfusion}[3][\mathord]{
    #1{\ifx#1\mathop\vphantom{#2}\fi
        \mathpalette\mov@rlay{#2\cr#3}
      }
    \ifx#1\mathop\expandafter\displaylimits\fi}

%%%%%%%%%%%%%%%%%%%%%%%%%%%%%%
%% 2.1 Some internal machinery
%%%%%%%%%%%%%%%%%%%%%%%%%%%%%%

\makeatletter
\renewcommand{\section}{\@startsection{section}{1}{\z@}%
							 {-3.25ex \@plus -1ex \@minus -.2ex}%
							 {1.5ex \@plus.2ex}%
							 {\normalfont\large\bfseries}}
\renewcommand{\subsection}{\@startsection{subsection}{2}{\z@}%
							 {-3.25ex \@plus -1ex \@minus -.2ex}%
							 {1.5ex \@plus .2ex}%
               {\normalfont\normalsize\bfseries}}
\newcommand*{\defeq}{\!\mathrel{\rlap{%
             \raisebox{0.3ex}{$\m@th\cdot$}}%
             \raisebox{-0.3ex}{$\m@th\cdot$}}%
                    =\!}
\makeatother
\ifx\ntrim\undefined
\newcommand{\coursetitle}{\nsigla: \nsiglahead}
\ifx\nextra\undefined
\pagestyle{ruled}
\makeoddhead{ruled}{\coursetitle}{}{\rightmark}
\else\fi
\settypeblocksize{49pc}{37pc}{*}
\setlrmargins{*}{*}{1.2}
\setulmargins{*}{*}{0.8}
\setheadfoot{16pt}{30pt}
\setheaderspaces{*}{1.5pc}{1}
\setmarginnotes{1pt}{1pt}{1pt}
\checkandfixthelayout

\setlength{\unitlength}{3pt}
\setlength{\hfuzz}{1pt}

\setlength{\fboxsep}{6pt}

\setlength{\footskip}{17pt}

\linespread{1.1}
\else\fi
\renewcommand{\cftdotsep}{\cftnodots} %%% no dots in ToC
\setpnumwidth{2em}  %%% width of page-number box in ToC


\newcommand{\stophere}{\relax} %% can be changed to `\endinput'
% \newcommand{\stophere}{\endinput} %% can be changed to `\relax'


\DeclareRobustCommand{\qned}{\ifmmode
  \else \leavevmode\unskip\penalty9999 \hbox{}\nobreak\hfill \fi
  \quad\hbox{\qnedsymbol}}
\newcommand{\qnedsymbol}{$\boxminus$} %% No-proofs end with `\qned'

\DeclareRobustCommand{\qef}{\ifmmode
  \else \leavevmode\unskip\penalty9999 \hbox{}\nobreak\hfill \fi
  \quad\hbox{\qefsymbol}}
\newcommand{\qefsymbol}{$\lozenge$} %% Examples end with `\qef'
\def\enddefn{\qef\endtrivlist}      %% `\qef' automático en defns
\def\endejem{\qef\endtrivlist}      %% `\qef' automático en ejemplos

\newcommand{\hideqed}{\renewcommand{\qed}{}} %% to suppress `\qed'
\newcommand{\hideqef}{\renewcommand{\qef}{}} %% to suppress `\qef'

% \newcommand{\ldbrack}{\ensuremath{[\mskip-2.5mu[}} %% corchetes [[
% \newcommand{\rdbrack}{\ensuremath{]\mskip-2.5mu]}} %% corchetes ]]

\newcommand{\stroke}{\mathbin|}     %% (for `\bbraket' and such)

\newcommand{\rtri}{\blacktriangleright} %% (for `\marker' and such)
\newcommand{\tribar}{|\mkern-2mu|\mkern-2mu|} %% norma triple: |||


%% Formatting changes:

\renewcommand{\labelitemi}{$\diamond$} %% instead of bullets

\renewcommand{\theenumi}{\alph{enumi}}  %% use lowercase letters
\renewcommand{\labelenumi}{\textup{(\theenumi)}} %% inside parentheses

%%%%%%%%%%%%%%
%% 2.2. Colors
%%%%%%%%%%%%%%

\definecolor{MATLABgreen}{RGB}{28,172,0} % color values Red, Green, Blue
\definecolor{MATLABlila}{RGB}{170,55,241}
\definecolor{dankBlue}{RGB}{51,60,77} % color values Red, Green, Blue
\definecolor{dankBlueLite}{RGB}{82,97,125} % color values Red, Green, Blue
\definecolor{celesUCR}{RGB}{0,192,243}
\definecolor{azulUCR}{RGB}{0,93,164}
\definecolor{verdeUCR}{RGB}{109,192,103}
\definecolor{yelloUCR}{RGB}{255,224,106}

%%%%%%%%%%%%%%%%%%%%%%%%%%%
%% 3. Theorems and suchlike
%%%%%%%%%%%%%%%%%%%%%%%%%%%

\ifx\nlang\undefined

\theoremstyle{plain}
\ifx \nextra \undefined
\newtheorem{Th}{Teorema}[section]      %%% Theorem 1.1.1
\newtheorem{Tmon}[Th]{Teoremón}
\newtheorem{Prop}[Th]{Proposición}     %%% Proposition 1.1.2
\newtheorem{Lem}[Th]{Lema}             %%% Lemma 1.1.3
\newtheorem{Cor}[Th]{Corolario}        %%% Corollary 1.1.4
\else
\newtheorem{Th}{Teorema}               %%% Theorem 1.1.1
\newtheorem{Tmon}{Teoremón}
\newtheorem{Prop}{Proposición}         %%% Proposition 1.1.2
\newtheorem{Lem}{Lema}                 %%% Lemma 3
\newtheorem{Cor}{Corolario}            %%% Corollary 4
\fi
\newtheorem*{nonum-Th}{Teorema}        %%% No-numbered Theorem
\newtheorem*{nonum-Cor}{Corolario}     %%% No-numbered Corollary

\theoremstyle{definition}
\ifx \nextra \undefined
\newtheorem{Def}[Th]{Definición}       %%% Definition 1.1.5
\newtheorem{Ex}[Th]{Ejemplo}           %%% Example 1.1.6
\newtheorem{Ej}[Th]{Ejercicio}         %%% Ejercicio 1.1.7
\else
\newtheorem{Def}{Definición}           %%% Definition 5
\newtheorem{Ex}{Ejemplo}               %%% Example 6
\newtheorem{Ej}{Ejercicio}             %%% Ejercicio 7
\fi
\newtheorem{Hec}[Th]{Hecho}            %%% Hecho 1.1.8
\newtheorem*{nonum-Def}{Definición}    %%% No number Definition
\newtheorem*{nonum-Ex}{Ejemplo}        %%% No number Example
\newtheorem*{nonum-Ej}{Ejercicio}      %%% No number Ejercicio
\newtheorem*{nonum-Hec}{Hecho}         %%% No number Fact


\theoremstyle{remark}
\newtheorem{Rmk}[Th]{Observación}      %%%Remark 1.1.9
\newtheorem*{nonum-Rmk}{Observación}         %%% No number Fact
\newtheorem*{Notn}{Notaci\'on}        %% Notaciones
\newtheorem*{Warn}{Advertencia}       %% Advertencias
\newtheorem*{Qn}{Pregunta}            %% Pregunta

\else

\theoremstyle{plain}
\ifx \nextra \undefined
\newtheorem{Th}{Theorem}[section]      %%% Theorem 1.1.1
\newtheorem{Tmon}[Th]{Teoremón}
\newtheorem{Prop}[Th]{Proposition}     %%% Proposition 1.1.2
\newtheorem{Lem}[Th]{Lemma}             %%% Lemma 1.1.3
\newtheorem{Cor}[Th]{Corollary}        %%% Corollary 1.1.4
\else
\newtheorem{Th}{Theorem}               %%% Theorem 1.1.1
\newtheorem{Tmon}{Teoremón}
\newtheorem{Prop}{Proposition}         %%% Proposition 1.1.2
\newtheorem{Lem}{Lemma}                 %%% Lemma 3
\newtheorem{Cor}{Corollary}            %%% Corollary 4
\fi
\newtheorem*{nonum-Th}{Theorem}        %%% No-numbered Theorem
\newtheorem*{nonum-Cor}{Corollary}     %%% No-numbered Corollary

\theoremstyle{definition}
\ifx \nextra \undefined
\newtheorem{Def}[Th]{Definition}       %%% Definition 1.1.5
\newtheorem{Ex}[Th]{Example}           %%% Example 1.1.6
\newtheorem{Ej}[Th]{Exercise}         %%% Exercise 1.1.7
\else
\newtheorem{Def}{Definition}           %%% Definition 5
\newtheorem{Ex}{Example}               %%% Example 6
\newtheorem{Ej}{Exercise}             %%% Exercise 7
\fi
\newtheorem{Hec}[Th]{Fact}            %%% Fact 1.1.8
\newtheorem*{nonum-Def}{Definition}    %%% No number Definition
\newtheorem*{nonum-Ex}{Example}        %%% No number Example
\newtheorem*{nonum-Ej}{Exercise}      %%% No number Exercise
\newtheorem*{nonum-Hec}{Fact}         %%% No number Fact


\theoremstyle{remark}
\newtheorem{Rmk}[Th]{Remark}      %%%Remark 1.1.9
\newtheorem*{nonum-Rmk}{Remark}         %%% No number Fact
\newtheorem*{Notn}{Notation}        %% Notaciones
\newtheorem*{Warn}{Warning}       %% Warnings
\newtheorem*{Qn}{Question}            %% Question

\fi 

\numberwithin{equation}{section}

\setlength{\parindent}{3ex}

% \renewcommand{\labelitemi}{--}
% \renewcommand{\labelitemii}{$\circ$}
% \renewcommand{\labelenumi}{(\roman{*})}

%\let\stdsection\section
%\renewcommand\section{\newpage\stdsection}

\newcommand\qedsym{\hfill\ensuremath{\square}}
% Strike through
\def\st{\bgroup \ULdepth=-.55ex \ULset}

%%%%%%%%% === My T Color Box === %%%%%%%%%%%%%%

\ifx\nlang\undefined
\ifx \darktheme\undefined
\newtcolorbox{ptcb}{
colframe = black,
colback = white,
breakable,
enhanced
}
\newtcolorbox{ptcbp}{
colframe = black,
colback = white,
coltitle = black,
colbacktitle = black!40,
title = Prueba,
breakable,
enhanced
}
\newtcolorbox{ptcbr}{
colframe = blue,
colback = white,
coltitle = blue,
colbacktitle = blue!40,
title = Respuesta,
breakable,
enhanced
}
\else
\newtcolorbox{ptcb}{
colframe = white,
colback = dankBlue,
colupper = white,
breakable,
enhanced
}
\newtcolorbox{ptcbp}{
colframe = white,
colback = dankBlue,
colupper = white,
coltitle = white,
colbacktitle = dankBlueLite,
title = Prueba,
breakable,
enhanced
}
\newtcolorbox{ptcbr}{
colframe = white,
colback = white,
coltitle = blue,
colbacktitle = blue!40,
title = Respuesta,
breakable,
enhanced
}
\fi

\else
\ifx \darktheme\undefined
\newtcolorbox{ptcb}{
colframe = black,
colback = white,
breakable,
enhanced
}
\newtcolorbox{ptcbp}{
colframe = black,
colback = white,
coltitle = black,
colbacktitle = black!40,
title = Proof,
breakable,
enhanced
}
\newtcolorbox{ptcbr}{
colframe = blue,
colback = white,
coltitle = blue,
colbacktitle = blue!40,
title = Answer,
breakable,
enhanced
}
\else
\newtcolorbox{ptcb}{
colframe = white,
colback = dankBlue,
colupper = white,
breakable,
enhanced
}
\newtcolorbox{ptcbp}{
colframe = white,
colback = dankBlue,
colupper = white,
coltitle = white,
colbacktitle = dankBlueLite,
title = Proof,
breakable,
enhanced
}
\newtcolorbox{ptcbr}{
colframe = white,
colback = white,
coltitle = blue,
colbacktitle = blue!40,
title = Answer,
breakable,
enhanced
}
\fi
\fi


%%%%%%%%% === Listings === %%%%%%%%%%%%%%
\lstset{basicstyle=\ttfamily,breaklines=true}

\lstset{language=Matlab,%
    %basicstyle=\color{red},
    breaklines=true,%
    morekeywords={matlab2tikz},
    keywordstyle=\color{blue},%
    morekeywords=[2]{1}, keywordstyle=[2]{\color{black}},
    identifierstyle=\color{black},%
    stringstyle=\color{MATLABlila},
    commentstyle=\color{MATLABgreen},%
    showstringspaces=false,%without this there will be a symbol in the places where there is a space
    numbers=left,%
    numberstyle={\tiny \color{black}},% size of the numbers
    numbersep=9pt, % this defines how far the numbers are from the text
   % emph=[1]{for,end,break,function,if,elseif,else},emphstyle=[1]\color{blue}, %some words to emphasise
    %emph=[2]{word1,word2}, emphstyle=[2]{style},
}

%%%%%%%%%%%%%%%%%%%%%%%%%%
%% 4. Simple abbreviations
%%%%%%%%%%%%%%%%%%%%%%%%%%

%%% Operator names:

\DeclareMathOperator{\area}{area}
\DeclareMathOperator{\card}{card}
\DeclareMathOperator{\ccl}{ccl}
\DeclareMathOperator{\ch}{ch}
\DeclareMathOperator{\cl}{cl}
\DeclareMathOperator{\coker}{coker}
\DeclareMathOperator{\Conv}{Conv}   %%Convex hull
\DeclareMathOperator{\cosec}{cosec}
\DeclareMathOperator{\cosech}{cosech}
\DeclareMathOperator{\covol}{covol}
\DeclareDocumentCommand\curl{}{\operatorname{curl}} 
\DeclareMathOperator{\diag}{diag}
\DeclareMathOperator{\diam}{diam}
\DeclareMathOperator{\Diff}{Diff}
\DeclareDocumentCommand\div{}{\operatorname{div}} 
\DeclareMathOperator{\energy}{energy}
\DeclareMathOperator{\erfc}{erfc}
\DeclareMathOperator{\Ext}{Ext}
\DeclareMathOperator{\fst}{fst}
\DeclareMathOperator{\Fit}{Fit}
\DeclareMathOperator{\gr}{gr}
\DeclareMathOperator{\hcf}{hcf}
\DeclareMathOperator{\Hilb}{Hilb} %Hilbert scheme
\DeclareMathOperator{\id}{id}
\DeclareMathOperator{\Ind}{Ind}
\DeclareMathOperator{\Int}{Int}
\DeclareMathOperator{\Isom}{Isom}
\DeclareMathOperator{\lcm}{lcm}
\DeclareMathOperator{\length}{length}
\DeclareMathOperator{\Lie}{Lie}
\DeclareMathOperator{\like}{like}
\DeclareMathOperator{\Lk}{Lk}
\DeclareMathOperator{\Maps}{Maps}
\DeclareMathOperator{\mcd}{mcd}
\DeclareMathOperator{\mcm}{mcm}
\DeclareMathOperator{\Min}{Min}
\DeclareMathOperator{\orb}{orb}
\DeclareMathOperator{\ord}{ord}
\DeclareMathOperator{\otp}{otp}
\DeclareMathOperator{\pr}{pr}       %% proyector
\DeclareMathOperator{\poly}{poly}
\DeclareMathOperator{\rel}{rel}
\DeclareMathOperator{\Rad}{Rad}
\DeclareMathOperator*{\res}{res}
\DeclareMathOperator{\Ric}{Ric}
\DeclareMathOperator{\rk}{rk}
\DeclareMathOperator{\Rees}{Rees}
\DeclareMathOperator{\Root}{Root}
\DeclareMathOperator{\rot}{rot}         %% rotacional
\DeclareMathOperator{\spn}{span}
\DeclareMathOperator{\St}{St}
\DeclareMathOperator{\supp}{supp}
\DeclareMathOperator{\Syl}{Syl}
\DeclareMathOperator{\Sym}{Sym}
\DeclareMathOperator{\vol}{vol}

% not-math
\newcommand{\bolds}[1]{{\bfseries #1}}
\newcommand{\cat}[1]{\mathsf{#1}}
\newcommand{\ph}{\,\cdot\,}
\newcommand{\term}[1]{\un{#1}\index{#1}}
\newcommand{\phantomeq}{\hphantom{{}={}}}
\newcommand{\ttt}{\texttt}
\newcommand{\red}[1]{\textcolor{red}{#1}}
\newcommand{\prp}[1]{\textcolor{purple}{#1}}
\newcommand{\blu}[1]{\textcolor{azulUCR}{#1}}
\newcommand{\green}[1]{\textcolor{verdeUCR}{#1}}
\newcommand{\yelo}[1]{\textcolor{yelloUCR}{#1}}
\newcommand{\cele}[1]{\textcolor{celesUCR}{#1}}

%functions
\DeclareMathOperator{\sgn}{sgn}
\newcommand*{\Cdot}{{\raisebox{-0.25ex}{\scalebox{1.5}{$\cdot$}}}}      %% cdot más grande
\newcommand{\ind}{\mathbf{1}}       %%%indicator function
\newcommand{\mm}{\mathfrak{m}}      %%%metric


% Greek letters:

\newcommand{\al}{\alpha}                %% short for  \alpha
\newcommand{\bt}{\beta}                 %% short for  \beta
\newcommand{\Dl}{\Delta}                %% short for  \Delta
\newcommand{\dl}{\delta}                %% short for  \delta
\newcommand{\eps}{\varepsilon}          %% short for  \varepsilon
\newcommand{\Ga}{\Gamma}                %% short for  \Gamma
\newcommand{\ga}{\gamma}                %% short for  \gamma
\newcommand{\kp}{\kappa}                %% short for  \kappa
\newcommand{\La}{\Lambda}               %% short for  \Lambda
\newcommand{\la}{\lambda}               %% short for  \lambda
\newcommand{\Om}{\Omega}                %% short for  \Omega
\newcommand{\om}{\omega}                %% short for  \omega
\newcommand{\Sg}{\Sigma}                %% short for  \Sigma
\newcommand{\sg}{\sigma}                %% short for  \sigma
\newcommand{\Te}{\Theta}                %% short for  \Theta
\newcommand{\te}{\theta}                %% short for  \theta
\newcommand{\ups}{\upsilon}             %% short for  \upsilon
\newcommand{\vf}{\varphi}               %% short for  \varphi
\newcommand{\ze}{\zeta}                 %% short for  \zeta
\newcommand{\vsg}{\varsigma}            %% short for  \varsigma
\newcommand{\vte}{\vartheta}            %% short for  \vartheta

%Boldface letters

\newcommand{\bA}{\mathbb{A}}        %% antisimetrizador
\newcommand{\bB}{\mathbb{B}}        %% bola unitaria
\newcommand{\bC}{\mathbb{C}}    %%% números complejos
\newcommand{\bCP}{\mathbb{CP}}  %%% espacio proyectivo complejo
\newcommand{\bD}{\mathbb{D}}        %% Poincaré disk
\newcommand{\bE}{\mathbb{E}}
\newcommand{\bF}{\mathbb{F}}        %% un cuerpo
\newcommand{\bH}{\mathbb{H}}        %% cuaterniones
\newcommand{\bI}{\mathbb{I}}        %% ideal de zeros
\newcommand{\bK}{\mathbb{K}}            %% ein korper
\newcommand{\bN}{\mathbb{N}}    %%% números naturales
\newcommand{\bP}{\mathbb{P}}        %% números enteros positivos
\newcommand{\bQ}{\mathbb{Q}}    %%% números racionales
\newcommand{\bR}{\mathbb{R}}    %%% números reales
\newcommand{\bRP}{\mathbb{RP}}  %%% espacio proyectivo real
\newcommand{\bS}{\mathbb{S}}    %%% esfera
\newcommand{\bT}{\mathbb{T}}        %% círculo o toro
\newcommand{\bV}{\mathbb{V}}        %% lugar geométrico de ceros
\newcommand{\bZ}{\mathbb{Z}}    %%% números enteros

%Script letters:

\newcommand{\cA}{\mathcal{A}}           %% formas diferenciales
\newcommand{\cB}{\mathcal{B}}           %% una base vectorial
\newcommand{\cC}{\mathcal{C}}           %% otra base vectorial
\newcommand{\cD}{\mathcal{D}}           %% funciones de prueba
\newcommand{\cE}{\mathcal{E}}           %% un modulo proyectivo
\newcommand{\cF}{\mathcal{F}}           %% espacio de Fock
\newcommand{\cG}{\mathcal{G}}           %% funtor de Gelfand
\newcommand{\cH}{\mathcal{H}}           %% espacio de Hilbert
\newcommand{\cI}{\mathcal{I}}           %% un funtor de inclusion
\newcommand{\cJ}{\mathcal{J}}           %% otro funtor
\newcommand{\cK}{\mathcal{K}}           %% otro espacio de Hilbert
\newcommand{\cL}{\mathcal{L}}           %% operadores lineales
\newcommand{\cM}{\mathcal{M}}           %% multiplicadores
\newcommand{\cN}{\mathcal{N}}           %% funciones nulas
\newcommand{\cO}{\mathcal{O}}           %% funciones de crec-to lento
\newcommand{\cP}{\mathcal{P}}           %% una particion
\newcommand{\cR}{\mathcal{R}}           %% funciones representativas
\newcommand{\cQ}{\mathcal{Q}}           %% otra particion
\newcommand{\cS}{\mathcal{S}}           %% funciones de Schwartz
\newcommand{\cT}{\mathcal{T}}           %% una topologia
\newcommand{\cU}{\mathcal{U}}           %% cubrimiento abierto
\newcommand{\cV}{\mathcal{V}}           %% vecindarioas
\newcommand{\cW}{\mathcal{W}}           %% grupo de Weyl
\newcommand{\cZ}{\mathcal{Z}}           %% topología de Zariski

%%% Fraktur letters:

\newcommand{\gA}{\mathfrak{A}}      %% un atlas
\newcommand{\g}{\mathfrak{g}}       %% un álgebra de Lie
\newcommand{\gB}{\mathfrak{B}}      %% otro atlas
\newcommand{\ggl}{\mathfrak{gl}}    %% álg de Lie general lineal
\newcommand{\gsl}{\mathfrak{sl}}    %% álg de Lie especial lineal
\newcommand{\gso}{\mathfrak{so}}    %% álg de Lie especial ortogonal
\newcommand{\gsu}{\mathfrak{su}}    %% álg de Lie especial unitaria
\newcommand{\gX}{\mathfrak{X}}      %% campos vectoriales

%%% Roman letters:

\newcommand{\dR}{\mathrm{dR}}       %% cohomología de de Rham
\newcommand{\rGL}{\mathrm{GL}}      %% grupo general lineal
\newcommand{\rO}{\mathrm{O}}        %% grupo ortogonal
\newcommand{\rSL}{\mathrm{SL}}      %% grupo especial lineal
\newcommand{\rSO}{\mathrm{SO}}      %% grupo ortogonal especial
\newcommand{\rSp}{\mathrm{Sp}}      %% grupo simpléctico
\newcommand{\rSU}{\mathrm{SU}}      %% grupo unitario especial
\newcommand{\rU}{\mathrm{U}}        %% grupo unitario
\newcommand{\rUH}{\mathrm{UH}}      %% cuaterniones unitarias
\newcommand{\rT}{\mathrm{T}}        %% grupo triangular

% Sanserif letters:

\newcommand{\sA}{\mathsf{A}}            %% algebras de Lie A_n
\newcommand{\sB}{\mathsf{B}}            %% grupo como categoria
\newcommand{\sC}{\mathsf{C}}            %% una categoria
\newcommand{\sD}{\mathsf{D}}            %% otra categoria
\newcommand{\sE}{\mathsf{E}}            %% otra categoria mas
\newcommand{\sF}{\mathsf{F}}            %% algebra de Lie F_4
\newcommand{\sG}{\mathsf{G}}            %% algebra de Lie G_2
\newcommand{\sJ}{\mathsf{J}}            %% un poset
\newcommand{\sK}{\mathsf{K}}            %% un poset
\newcommand{\sL}{\mathcal{L}}           %% derivada de Lie
\newcommand{\sN}{\mathsf{N}}            %% categoría con objetos \bN
\newcommand{\sT}{\mathsf{T}}            %% transpuesta

%%% Boldface letters:

\bmdefine{\CC}{C}                       %% C negrilla
\bmdefine{\cc}{c}
%\bmdefine{\dd}{d}                       %% d negrilla
\bmdefine{\ee}{e}                       %% vector e
\bmdefine{\eeps}{\varepsilon}           %% basic form \eps
\bmdefine{\FF}{F}                       %% vector F
\bmdefine{\ff}{f}                       %% vector f
\bmdefine{\ii}{i}                       %% cuaternion i
\bmdefine{\jj}{j}                       %% cuaternion j
\bmdefine{\kk}{k}                       %% cuaternion k
\bmdefine{\lla}{\lambda}                %% sucesion \la
\bmdefine{\mmu}{\mu}                    %% sucesion \mu
\bmdefine{\pp}{p}                       %% vector p
\bmdefine{\qq}{q}                       %% vector q
\bmdefine{\rr}{r}                       %% vector r
\bmdefine{\ssg}{\sigma}                 %% vector \sg
%\bmdefine{\sss}{s}
%\bmdefine{\ttt}{t}
\bmdefine{\VV}{V}                       %% V negrilla
\bmdefine{\xx}{x}                       %% sucesion x
\bmdefine{\xxi}{\xi}                    %% vector \xi
\bmdefine{\yy}{y}                       %% sucesion y
\bmdefine{\zz}{z}                       %% sucesion z

% Matrix groups
\DeclareMathOperator{\GL}{GL}   %%% grupo general lineal
\DeclareMathOperator{\Or}{O}    %%% grupo ortogonal
\DeclareMathOperator{\PGL}{PGL} %%% grupo proyectivo lineal
\DeclareMathOperator{\PSL}{PSL} %%% grupo proyectivo lineal especial
\DeclareMathOperator{\PSO}{PSO} %%% grupo proyectivo ortogonal
\DeclareMathOperator{\PSU}{PSU} %%% grupo proyectivo unitario
\DeclareMathOperator{\SL}{SL}   %%% grupo especial lineal
\DeclareMathOperator{\SO}{SO}   %%% grupo especial ortogonal
\DeclareMathOperator{\SU}{SU}   %%% grupo especial unitario

% Numericc
\newcommand{\argmin}{\text{argm\'in}}
\DeclareMathOperator{\dof}{dof}

%% Brackets
\newcommand{\conj}[1]{\left\lbrace#1\right\rbrace}
\newcommand{\bonj}[1]{\left\lbrack#1\right\rbrack}
\newcommand{\obonj}[1]{\left\rbrack#1\right\lbrack}
\newcommand{\rbonj}[1]{\left\rbrack#1\right\rbrack}
\newcommand{\lbonj}[1]{\left\lbrack#1\right\lbrack}
\newcommand{\snm}[1]{\|#1\|}           %small norma
\newcommand{\nm}[1]{\left\|#1\right\|} %norma pegadita
\newcommand{\pnm}[1]{\biggl|\biggl|#1\biggr|\biggr|}
\let\oldvec=\vec
\renewcommand{\vec}[1]{\mathbf{#1}}
\newcommand\quot[2]{
        \mathchoice
            {% \displaystyle
                \text{\raise1ex\hbox{$#1$}\Big/\lower1ex\hbox{$#2$}}%
            }
            {% \textstyle
                {^{ #1}/_{ #2}}
            }
            {% \scriptstyle
                {^{ #1}/_{ #2}}
            }
            {% \scriptscriptstyle
                {^{ #1}/_{ #2}}
            }
    }
%\newcommand*\quot[2]{{^{\textstyle #1}\big/_{\textstyle #2}}}
\newcommand*\squot[2]{{^{ #1}/_{ #2}}}%%%small quotient
\newcommand{\multinom}[2]{\ensuremath{\left(\kern-.3em\left(\genfrac{}{}{0pt}{}{#1}{#2}\right)\kern-.3em\right)}}

% Probability
\DeclareMathOperator{\Bernoulli}{Bernoulli}
\DeclareMathOperator{\betaD}{beta}
\DeclareMathOperator{\bias}{bias}
\DeclareMathOperator{\binomial}{binomial}
\DeclareMathOperator{\corr}{corr}
\DeclareMathOperator{\cov}{cov}
\DeclareMathOperator{\gammaD}{gamma}
\DeclareMathOperator{\mse}{mse}
\DeclareMathOperator{\multinomial}{multinomial}
\DeclareMathOperator{\Poisson}{Poisson}
\DeclareMathOperator{\Var}{Var}     %%%variance
\DeclareMathOperator{\Cov}{Cov}     %%%Covariance
\renewcommand{\mid}{\;\ifnum\currentgrouptype=16 \middle\fi|\;}

% Combinatorics
\DeclareMathOperator{\ins}{ins}   % insertion tableaux
\DeclareMathOperator{\asc}{asc}   % ascents
\DeclareMathOperator{\rw}{rw}     % reading word
\DeclareMathOperator{\rev}{rev}     % reading word
\DeclareMathOperator{\rect}{rect} % rectification of young tableau
\DeclareMathOperator{\sh}{sh}     % shape of young tableau
\DeclareMathOperator{\std}{std}   % standarization
\DeclareMathOperator{\Fl}{\mathcal{F}\ell}       %% conjunto de Flags
\DeclareMathOperator{\Frob}{Frob} % Frobenius map

% Algebra
\DeclareMathOperator{\Ad}{Ad}       %% acción adjunta
\DeclareMathOperator{\adj}{adj}
\DeclareMathOperator{\Ann}{Ann}     %% aniquilador o anulador de módulos
\DeclareMathOperator{\Ass}{Ass}     %% ideales asociados
\DeclareMathOperator{\Aut}{Aut}
\DeclareMathOperator{\Bl}{\mathcal{B}\!\ell}       %% blowup de un espacio
\DeclareMathOperator{\Char}{char}
\DeclareMathOperator{\codim}{codim}
\DeclareMathOperator{\disc}{disc}
\DeclareMathOperator{\dom}{dom}
\DeclareMathOperator{\End}{End}     %%%space of endomorphisms
\DeclareMathOperator{\Fix}{Fix}
\DeclareMathOperator{\Frac}{Frac}
\DeclareMathOperator{\Gal}{Gal}
\DeclareMathOperator{\gen}{gen}     %%%set generated by...
\DeclareMathOperator{\Gr}{Gr}       %%%Grassmannian
\DeclareMathOperator{\Hom}{Hom}
\DeclareMathOperator{\Hurw}{Hurw}
\DeclareMathOperator{\image}{image}
\DeclareMathOperator{\Mor}{Mor}
\DeclareMathOperator{\Nil}{Nil}
\DeclareMathOperator{\Orb}{Orb}
\DeclareMathOperator{\Pic}{Pic}     %%% grupo de Picard 
\DeclareMathOperator{\Quot}{Quot}
\DeclareMathOperator{\Spec}{Spec}
\DeclareMathOperator{\Stab}{Stab}
\DeclareMathOperator{\Taut}{Taut}

% Analysis
\DeclareMathOperator*{\esssup}{ess\hspace{0.5mm}sup}
\DeclareMathOperator*{\essinf}{ess\hspace{0.5mm}inf}
%\DeclareMathOperator{\Int}{Int}     %%%interior vacilon funcional

\newcommand{\loc}{\text{loc}}
\newcommand{\LB}{\cL_\cB}           %%%bounded linear operator

% Logic
\newcommand{\cleq}{\preccurlyeq}
\newcommand{\cgeq}{\succcurlyeq}

% Others
\renewcommand{\ev}{\operatorname{ev}}     %%%evalutation previously expectation value physics package
\newcommand{\bigcupdot}{\charfusion[\mathop]{\bigcup}{\Cdot}} %%JCVDG
%\renewcommand{\bigcupdot}{\charfusion[\mathop]{\bigcup}{\Cdot}}
\newcommand{\cupdot}{\charfusion[\mathbin]{\cup}{\Cdot}}
\newcommand{\exterior}{\mathchoice{{\textstyle\bigwedge}}{{\bigwedge}}{{\textstyle\wedge}}{{\scriptstyle\wedge}}}
\newcommand{\hol}{\mathfrak{hol}}
\newcommand{\Id}{\mathrm{Id}}
\newcommand{\lie}[1]{\mathfrak{#1}}
\newcommand{\qeq}{\mathrel{``{=}"}}
\newcommand{\wsto}{\stackrel{\mathrm{w}^*}{\to}}
\newcommand{\wt}{\mathrm{wt}}

%\let\Im\relax
%\let\Re\relax

%%% Shorter symbol names:

\newcommand{\bull}{{\scriptstyle\bullet}}  %% vertice en figuras
\newcommand{\del}{\partial}             %% short for  \partial
\newcommand{\downto}{\downarrow}        %% limite a la derecha
\newcommand{\dsp}{\displaystyle}        %% despliegue en texto
\renewcommand{\geq}{\geqslant}          %% mayor o igual (variante)
\newcommand{\hookto}{\hookrightarrow}     %% inclusion arrow
\newcommand{\isom}{\simeq}              %% isomorfismo
\renewcommand{\l}{\ell}                   %% ele cursiva
\renewcommand{\leq}{\leqslant}          %% menor o igual (variante)
\newcommand{\less}{\setminus}           %% set difference
\newcommand{\otto}{\leftrightarrow}     %% bijection
\newcommand{\ox}{\otimes}               %% producto tensorial
\newcommand{\rt}{\triangleleft}         %% un orden parcial
\newcommand{\rteq}{\trianglelefteq}     %% normal subgroup
\newcommand{\up}{{\mathord{\uparrow}}}  %% espinor `up'
\newcommand{\upto}{\uparrow}            %% left hand limit
\newcommand{\w}{\wedge}                 %% producto exterior
\newcommand{\wto}{\rightharpoonup}      %% convergencia debil
\newcommand{\x}{\times}                 %% producto vectorial
\renewcommand{\.}{\Cdot}                %% producto escalar
\renewcommand{\:}{\mathbin{:}}          %% colon in  f: A -> B
\newcommand{\into}{\rightarrowtail}     %% injection arrow
\newcommand{\lr}{\dashv}                %% adjunction
\newcommand{\lt}{\triangleright}        %% a left action
\newcommand{\lteq}{\trianglerighteq}    %% normal supergroup
\newcommand{\nb}{\nabla}                %% homomorfismo de suma
\newcommand{\nisom}{\not\simeq}         %% negacion de isomorfismo
%\newcommand{\oast}{\circledast}         %% variante de * (ya existe en stmaryrd)
\newcommand{\onto}{\twoheadrightarrow}  %% surjection arrow
\newcommand{\opp}{\circ}                %% objeto opuesto
\newcommand{\ottto}{\longleftrightarrow} %% bijection in display
\newcommand{\pullb}{\lrcorner}          %% simbolo de pullback
\newcommand{\pushf}{\ulcorner}          %% simbolo de pushout
\newcommand{\rx}{\rtimes}               %% producto semidirecto
\newcommand{\To}{\Rightarrow}           %% entre funtores
\newcommand{\tofro}{\rightleftarrows}   %% pair of opposed maps
\newcommand{\toto}{\rightrightarrows}   %% pair of parallel maps

\renewcommand{\2}{\flat}                  %% marcador de sucesiones
\newcommand{\3}{\sharp}                 %% marcador de sucesiones
\newcommand{\4}{\natural}               %% marcador de morfismos
% \newcommand{\5}{\diamond}               %% for roots of trees
% \newcommand{\7}{\dagger}                %% adjunto de operador
\newcommand{\8}{\bullet}                %% anonymous degree

%%% Useful abbreviations:

\newcommand{\Coo}{\cC^\infty}         %% funciones suaves
\newcommand{\ctr}{\mathbin{\lrcorner\,}} %% contraction symbol
\newcommand{\nbf}{{\vec\nabla}}     %% short for  \vec\nabla

\newcommand{\as}{\quad\text{cuando}\enspace} %% `cuando' en límites
\newcommand{\bCoo}{{\bC_\infty}}    %% esfera de Riemann
% \newcommand{\bRoo}{{\bR_\infty}}    %% círculo real extendido

%%% Repeated relations:

\newcommand{\cupycup}{\cup\cdots\cup} %% unión repetida
\newcommand{\capycap}{\cap\cdots\cap} %% intersección repetida
\newcommand{\sys}{\subset\cdots\subset}%% subconjunto propio repetido
\newcommand{\subysub}{\subseteq\cdots\subseteq} %%subconjunto repetido
\newcommand{\oxyox}{\otimes\cdots\otimes} %% prod tensorial repetido
\newcommand{\wyw}{\wedge\cdots\wedge} %% producto exterior repetido
\newcommand{\opyop}{\oplus\cdots\oplus} %% suma directa repetida
\newcommand{\xyx}{\times\cdots\times} %% producto directo repetido

%%% Arrows with riders:

\newcommand{\longto}{\mathop{\longrightarrow}\limits}

%%% Small fractions in displays:

\newcommand{\half}{{\mathchoice{\nhalf}{\thalf}{\shalf}{\shalf}}} %%display text script script^2
\newcommand{\happi}{{\tfrac{\pi}{2}}} %% small fraction  \pi/2
\newcommand{\quarter}{\tfrac{1}{4}} %% small fraction  1/4
\newcommand{\nhalf}{\frac{1}{2}}
\newcommand{\shalf}{{\scriptstyle\frac{1}{2}}} %% tiny fraction 1/2
\newcommand{\thalf}{{\tfrac{1}{2}}} %% small fraction  1/2
\renewcommand{\third}{\tfrac{1}{3}}   %% small fraction  1/3 %Hay que renew porque mathabx toma second y third como x'' y x''' por ejemplo

\newcommand{\ihalf}{{\tfrac{i}{2}}} %% small fraction  i/2

%%%%%%%%%%%%%%%%%%%%%%%%%%%%%
%% 5. Commands with arguments
%%%%%%%%%%%%%%%%%%%%%%%%%%%%%

%%% Accent-like commands, abbreviated:

\newcommand{\ov}{\overline}        %% short for  \overline
\newcommand{\un}{\underline}       %% short for  \underline
\newcommand{\wh}{\widehat}          %% short for  \widehat

%%% Separate words in displays:

\newcommand{\word}[1]{\quad\text{#1}\quad} %% texto intercalado

%%% Webpage locator:

\newcommand{\zelda}[1]{$\langle${\footnotesize\texttt{#1}}$\rangle$}

%% Symbol placement:

\newcommand{\pre}[1]{{}^{#1\!}} %% upper left exponent

%%% Proof-part labels:

\newcommand{\Adiff}[2]{\ensuremath{\Ad\,(\mathrm{#1})\Longleftrightarrow
    (\mathrm{#2})}:\enspace}
\newcommand{\Adimp}[2]{\ensuremath{\Ad\,(\mathrm{#1})\Longrightarrow
    (\mathrm{#2})}:\enspace}
\newcommand{\Adit}[1]{\ensuremath{\Ad\,(\mathrm{#1})}:\enspace}

%%% Enclose one argument with delimiters:

\newcommand{\bool}[1]{\llbracket#1\rrbracket} %% condición booleana
\newcommand{\combo}[1]{\operatorname{co}(#1)} %% convex combo
\newcommand{\lin}[1]{\operatorname{lin}\langle#1\rangle} %% `span'
\newcommand{\set}[1]{\{\,#1\,\}}    %% set notation

\newcommand{\floor}[1]{\lfloor#1\rfloor} %% mayor entero <= x
\newcommand{\Set}[1]{\biggl\{\,#1\,\biggr\}} %% set notation (large)
\newcommand{\roof}[1]{\lceil#1\rceil} %% menor entero >= x
\newcommand{\genr}[1]{\left\langle #1\right\rangle}     %% grupo generado por #1

%%% Asides:

\newcommand{\aside}[1]{$\llbracket$\,#1\,$\rrbracket$} % nota lateral
\ifx \nlang \undefined
\newcommand{\hint}[1]{$\llbracket$\,In\-di\-ca\-ci\'on: #1\,$\rrbracket$}
\else
\newcommand{\hint}[1]{$\llbracket$\,Hint: #1\,$\rrbracket$}
\fi 


%%% Matrices:

\newcommand{\onebytwo}[2]{\begin{pmatrix} %% 1 x 2 matrix
  #1 & #2 \end{pmatrix}}
\newcommand{\onebythree}[3]{\begin{pmatrix} %% 1 x 3 matrix
  #1 & #2 & #3 \end{pmatrix}}
\newcommand{\onebyfour}[4]{\begin{pmatrix} %% 1 x 4 matrix
  #1 & #2 & #3 & #4 \end{pmatrix}}
\newcommand{\twobyone}[2]{\begin{pmatrix} %% 2 x 1 matrix
   #1 \\ #2 \end{pmatrix}}
\newcommand{\twobytwo}[4]{\begin{pmatrix} %% 2 x 2 matrix
   #1 & #2 \\ #3 & #4 \end{pmatrix}}
\newcommand{\twobythree}[6]{\begin{pmatrix} %% 2 x 3 matrix
    #1 & #2 & #3\\ #4 & #5 & #6 \end{pmatrix}}
\newcommand{\threebyone}[3]{\begin{pmatrix} %% 3 x 1 matrix
   #1 \\ #2 \\ #3 \end{pmatrix}}
\newcommand{\threebythree}[9]{\begin{pmatrix} %% 3 x 3 matrix
   #1 & #2 & #3 \\ #4 & #5 & #6 \\ #7 & #8 & #9 \end{pmatrix}}
\newcommand{\fourbyone}[4]{\begin{pmatrix} %% 2 x 1 matrix
   #1 \\ #2 \\ #3 \\ #4 \end{pmatrix}}
%\newcommand{\fourbyfour}[16]{\begin{pmatrix} %% 4 x 4 matrix
%  #1 & #2 & #3 & #4\\ #5 & #6 & #7 & #8 \\ #9 & #10 & #11 & #12 \\ #13 & #14 & #15 & #16 \end{pmatrix}}
\newcommand{\nbyn}[9]{\begin{pmatrix} %% 4 x 4 matrix with prefilled entries
  #1 & #2 & \cdots & #3\\ #4 & #5 & \cdots & #6 \\ \vdots & \vdots & \ddots & \vdots \\ #7 & #8 & \cdots & #9 \end{pmatrix}}

%%%%%%%%%%%%%%%%%%%%%%%%%%%%
%% 6. Hyphenation exceptions
%%%%%%%%%%%%%%%%%%%%%%%%%%%%

\hyphenation{auto-va-lor auto-va-lo-res auto-vec-tor auto-vec-to-res
car-di-na-li-dad ce-rra-da ce-rra-do ce-rra-das ce-rra-dos cons-tan-te
cons-tan-tes cons-truc-ci cons-truir con-ti-nua con-ti-nua-mente
con-ti-nuas con-ti-nui-dad con-ti-nuo con-ti-nuos co-rres-pon-den-cia
co-rres-pon-de co-rres-pon-den co-rres-pon-dien-te
co-rres-pon-dien-tes co-va-rian-te cual-quier cual-quiera
cu-bri-mien-to desa-rro-lla-do desa-rro-llar des-pu dia-go-nal
dia-go-na-les di-fe-ren-cia-ble di-fe-ren-cia-bles di-fe-ren-cial
di-fe-ren-cia-les di-fe-ren-te di-fe-ren-tes dis-cre-ta dis-cre-tas
dis-cre-to dis-cre-tos di-vi-si-bi-li-dad di-vi-si-ble ele-men-tal
ele-men-ta-les ele-men-to ele-men-tos equi-va-len-cia equi-va-lente
equi-va-lentes equi-va-rian-te equi-va-rian-tes eu-cli-dia-na
eu-cli-dia-nas eu-cli-dia-no eu-cli-dia-nos Fi-gu-ra Gal-ois
gal-oi-sia-na ge-ne-rada ge-ne-rado ge-ne-ra-dor ge-ne-ra-do-res
ge-ne-ral ge-ne-ra-les ge-ne-ra-li-dad ge-ne-ra-li-za ge-ne-ra-li-zan
ge-ne-ran ge-ne-rar geo-me-tr geo-me-try Ha-da-mard ho-meo-mor-fis-mo
ho-meo-mor-fo idea-les in-de-pen-dien-te in-de-pen-dien-tes
in-va-rian-cia in-va-rian-te in-va-rian-tes li-ne-a-les
li-ne-al-men-te ma-ne-ra me-dian-te mo-der-no nin-gu-no nues-tra
nues-tro nu-me-ra-ble ope-ra-ci ope-ra-cio-nes ope-ra-dor
ope-ra-do-res or-to-go-nal par-ti-cu-lar pro-ce-di-mien-to pro-duc-to
pro-duc-tos pro-pie-dad pro-pie-da-des pro-po-si-ci re-fe-ren-cia
re-fle-xi-va re-fle-xi-vas re-fle-xi-vo re-fle-xi-vos re-so-lu-ble
res-pec-ti-va-men-te res-pec-ti-vo res-pec-ti-vos res-pec-to
sa-tis-fa-ce sepa-ra-ble sepa-ra-bles si-guien-te si-guien-tes
subes-pa-cio subes-pa-cios te-dra-edro te-tra-edros tri-vial
tri-via-les uti-lidad va-lo-res va-ria-ble va-ria-bles va-rie-dad
va-rie-da-des ve-cin-da-rio ve-cin-da-rios vec-to-rial vec-to-ria-les
vice-versa}


%%% TikZ arrows and such

\pgfarrowsdeclarecombine{twolatex'}{twolatex'}{latex'}{latex'}{latex'}{latex'}
\tikzset{->/.style = {decoration={markings,
                                  mark=at position 1 with {\arrow[scale=2]{latex'}}},
                      postaction={decorate}}}
\tikzset{<-/.style = {decoration={markings,
                                  mark=at position 0 with {\arrowreversed[scale=2]{latex'}}},
                      postaction={decorate}}}
\tikzset{<->/.style = {decoration={markings,
                                   mark=at position 0 with {\arrowreversed[scale=2]{latex'}},
                                   mark=at position 1 with {\arrow[scale=2]{latex'}}},
                       postaction={decorate}}}
\tikzset{->-/.style = {decoration={markings,
                                   mark=at position #1 with {\arrow[scale=2]{latex'}}},
                       postaction={decorate}}}
\tikzset{-<-/.style = {decoration={markings,
                                   mark=at position #1 with {\arrowreversed[scale=2]{latex'}}},
                       postaction={decorate}}}
\tikzset{->>/.style = {decoration={markings,
                                  mark=at position 1 with {\arrow[scale=2]{latex'}}},
                      postaction={decorate}}}
\tikzset{<<-/.style = {decoration={markings,
                                  mark=at position 0 with {\arrowreversed[scale=2]{twolatex'}}},
                      postaction={decorate}}}
\tikzset{<<->>/.style = {decoration={markings,
                                   mark=at position 0 with {\arrowreversed[scale=2]{twolatex'}},
                                   mark=at position 1 with {\arrow[scale=2]{twolatex'}}},
                       postaction={decorate}}}
\tikzset{->>-/.style = {decoration={markings,
                                   mark=at position #1 with {\arrow[scale=2]{twolatex'}}},
                       postaction={decorate}}}
\tikzset{-<<-/.style = {decoration={markings,
                                   mark=at position #1 with {\arrowreversed[scale=2]{twolatex'}}},
                       postaction={decorate}}}

\tikzset{circ/.style = {fill, circle, inner sep = 0, minimum size = 3}}
\tikzset{scirc/.style = {fill, circle, inner sep = 0, minimum size = 1.5}}
\tikzset{mstate/.style={circle, draw, blue, text=black, minimum width=0.7cm}}

\tikzset{eqpic/.style={baseline={([yshift=-.5ex]current bounding box.center)}}}
\tikzset{commutative diagrams/.cd,cdmap/.style={/tikz/column 1/.append style={anchor=base east},/tikz/column 2/.append style={anchor=base west},row sep=tiny}}

\definecolor{mblue}{rgb}{0.2, 0.3, 0.8}
\definecolor{morange}{rgb}{1, 0.5, 0}
\definecolor{mgreen}{rgb}{0.1, 0.4, 0.2}
\definecolor{mred}{rgb}{0.5, 0, 0}

\def\drawcirculararc(#1,#2)(#3,#4)(#5,#6){%
    \pgfmathsetmacro\cA{(#1*#1+#2*#2-#3*#3-#4*#4)/2}%
    \pgfmathsetmacro\cB{(#1*#1+#2*#2-#5*#5-#6*#6)/2}%
    \pgfmathsetmacro\cy{(\cB*(#1-#3)-\cA*(#1-#5))/%
                        ((#2-#6)*(#1-#3)-(#2-#4)*(#1-#5))}%
    \pgfmathsetmacro\cx{(\cA-\cy*(#2-#4))/(#1-#3)}%
    \pgfmathsetmacro\cr{sqrt((#1-\cx)*(#1-\cx)+(#2-\cy)*(#2-\cy))}%
    \pgfmathsetmacro\cA{atan2(#2-\cy,#1-\cx)}%
    \pgfmathsetmacro\cB{atan2(#6-\cy,#5-\cx)}%
    \pgfmathparse{\cB<\cA}%
    \ifnum\pgfmathresult=1
        \pgfmathsetmacro\cB{\cB+360}%
    \fi
    \draw (#1,#2) arc (\cA:\cB:\cr);%
}
\newcommand\getCoord[3]{\newdimen{#1}\newdimen{#2}\pgfextractx{#1}{\pgfpointanchor{#3}{center}}\pgfextracty{#2}{\pgfpointanchor{#3}{center}}}

\newcommand\qedshift{\vspace{-17pt}}
\newcommand\fakeqed{\pushQED{\qed}\qedhere}

\def\Xint#1{\mathchoice
   {\XXint\displaystyle\textstyle{#1}}%
   {\XXint\textstyle\scriptstyle{#1}}%
   {\XXint\scriptstyle\scriptscriptstyle{#1}}%
   {\XXint\scriptscriptstyle\scriptscriptstyle{#1}}%
   \!\int}
\def\XXint#1#2#3{{\setbox0=\hbox{$#1{#2#3}{\int}$}
     \vcenter{\hbox{$#2#3$}}\kern-.5\wd0}}
\def\ddashint{\Xint=}
\def\dashint{\Xint-}

\newcommand\separator{{\centering\rule{2cm}{0.2pt}\vspace{2pt}\par}}

\newenvironment{own}{\color{gray!70!black}}{}

\newcommand\makecenter[1]{\raisebox{-0.5\height}{#1}}

\mathchardef\mdash="2D

\newenvironment{significant}{\begin{center}\begin{minipage}{0.9\textwidth}\centering\em}{\end{minipage}\end{center}}
\DeclareRobustCommand{\rvdots}{%
  \vbox{
    \baselineskip4\p@\lineskiplimit\z@
    \kern-\p@
    \hbox{.}\hbox{.}\hbox{.}
  }}
\DeclareRobustCommand\tph[3]{{\texorpdfstring{#1}{#2}}}
\def\BState{\State\hskip-\ALG@thistlm}

\makeatother 
\usepackage{youngtab}
\begin{document}
%\clearpage
\maketitle
%\thispagestyle{empty}
{\small 
\setlength{\parindent}{0em}
\setlength{\parskip}{1em}

This is the second semester of an introductory graduate-level course on combinatorics. We will be covering symmetric function theory, Young tableaux, counting with group actions, designs, matroids, finite geometries, and not-so-finite geometries.\par 
The goal of this class is to give an overview of the wide variety of topics and techniques in both classical and modern combinatorial theory.

\subsubsection*{Requirements}
Knowledge on theory of enumeration, generating functions, combinatorial species, the basics of graph theory, posets, partitions and tableaux, and basic symmetric function theory is required.
}
\newpage
\tableofcontents
%\begin{multicols}{2}
\chapter{Symmetric functions}

\section{Day 1|20230120}
\begin{Def}
$f(x_1,x_2,\dots)$ is \term{symmetric} if it's fixed under permutations of variables. For a permutation $\sg$ this is, 
$$f(x_{\sg(1),x_{\sg(2)}},\dots)=f(x_1,x_2,\dots).$$
\end{Def}

\begin{Ex}
    The function 
    $$f(x_1,\dots,x_4)=x_1^5+\dots+x_4^5$$ 
    is known as $p_5$ or $m_{(5)}$, where $p$ is the power-sum symmetric function and $m$, the monomial symmetric function.\par 
    We can have the function defined on infinitely many variables. Consider the function $g$ defined as 
    $$g=x_1^4x_2+x_1^4x_3+\dots+x_i^4x_j+\dots+3x_1+\dots+3x_i+\dots=m_{(4,1)}+3m_{(1)}.$$ 
\end{Ex}

Let us recall some \textbf{notation}, 
$$
\begin{cases}
    \La_R(x_1,\dots,x_n)\to\text{symmetric functions on }n\text{ variables over }R,\\
    \La_R(\un{x})\to\text{symmetric functions on \emph{infinitely} many variables over }R.
\end{cases}
$$
In our case $R=\bQ$, so the object of study is $\La_\bQ$.
\begin{Prop}\label{prop-dim-LambdaQ}
    The space $\La_\bQ^n$ is the space of symmetric functions of degree $n$. Its dimension is $p(n)$, the number of partitions of $n$.
\end{Prop}

This is because, for every such function we can decompose it into monomials and the monomial symmetric functions form a basis.

\subsection*{Bases of $\La_Q$}

Suppose $\la=(\la_1,\dots,\la_k)\vdash n$ with $\la_1\geq\dots\geq\la_k$. 

\subsubsection*{Monomial Symmetric Functions}

The function $m_\la(\un x)$ is the smallest symmetric function which contains the monomial $x_1^{\la_1}x_2^{\la_2}\dots x_k^{\la_k}$ as a term. In general 
$$m_\la=\sum_{i_1\neq\dots\neq i_k}x_{i_1}^{\la_1}\dots x_{i_k}^{\la_k}.$$

\begin{Ex}
    Consider the partition $(5,3)\vdash 8$. The function $m_{(5,3)}$ will be different depending on the number of variables:
    \begin{itemize}
        \itemsep=-0.4em
        \item In one variable we can't have monomials of the form $x_ix_j$, so $m_{(5,3)}=0$.
        \item In two variables we have $m_{(5,3)}(x,y)=x^5y^3+y^5x^3$.
        \item In three variables the function is 
        $$m_{(5,3)}(x,y,z)=x^5y^3+y^5z^3+z^5x^3+y^5x^3+z^5y^3+x^5z^3.$$
    \end{itemize}
    Considering some special cases, take the partition $(1,1,1,1)\vdash 4$, then 
\begin{align*}
    m_{(1,1,1,1)}(u,v,x,y,z)&=uvxy+vxyz+xyzu+yzuv+zuvx\\
    &=uvxy+uxyz+uvyz+uvxz+vxyz.
\end{align*}
For cases with less than $4$ variables the function is zero and in exactly four, it has $1$ term. The partition $(4)\vdash 4$ returns the function 
$$m_{(4)}(x)=x^4,\ m_{(4)}(x,y)=x^4+y^4,\ m_{(4)}(x,y,z)=x^4+y^4+z^4,$$
and so on with any number of variables.
\end{Ex}

\begin{Rmk}
    The number of terms in $m_\la(x_1,\dots,x_d)$ is \red{I actually don't know}, while the degree of $m_\la$ is $|\la|=n$.
\end{Rmk}

\subsubsection{Elementary Symmetric Functions}

\begin{Def}
For any $r\in\bN$, the elementary symmetric function $e_r$ is $m_{(1,1,\dots,1)}$ ($r$ ones). For $\la$, a partition, $e_\la=\prod e_{\la_i}$. As an alternative for $m_{(1,1,\dots,1)}$ we can also write 
$$e_r(x_1,\dots,x_d)=\sum_{1\leq i_1<\dots<i_r\leq n}x_{i_1}\dots x_{i_r}.$$ 
\end{Def}

\begin{Ex}
    Let us calculate $e_{(2,1)}$ for $1$ through $3$ variables. When we have $e_{(2,1)}(x)=e_2(x)e_1(x)$, we can't compute $e_2(x)$ because there are no two-term monomials with only one variable. On two variables we have the following
    \begin{align*}
        e_{(2,1)}(x,y)&=e_2(x,y)e_1(x,y)=(xy)(x+y)=x^2y+y^2x
    \end{align*}
    and when talking about $3$ variables the following happens:
    \begin{align*}
        e_{(2,1)}(x,y,z)&=e_2(x,y,z)e_1(x,y,z)\\
        &=(xy+yz+zx)(x+y+z)\\
        &=x^2y+y^2z+z^2x+y^2x+z^2y+x^2z+2xyz.
    \end{align*}
    Consider now the partitions $(2,2,2,2)$ and $(5)$. Then 
    $$e_{(2,2,2,2)}=e_2^4\To e_{(r,r,\dots,r)}=e_r^{m_r(\la)}$$
    where $m_i(\la)$ is number of parts of $\la$ equal to $i$. For the partition $(5)$ we have that $e_{(5)}=e_5$ and in general $e_{(n)}=e_n$.
\end{Ex}

\begin{Rmk}
    As before \red{we don't know how many terms per function}, but knowing $m$ implies knowing $e$. As for the degree, it holds that $\deg(e_\la)=|\la|$.
\end{Rmk}

\subsubsection{Homogenous Symmetric Functions}
%https://garsia.math.yorku.ca/ghana03/chapters/mainfile3.pdf
%https://www.symmetricfunctions.com/index.htm
%http://www.mathematicalgemstones.com/gemstones/diamond/summary-symmetric-functions-transition-table/

\begin{itemize}
    
    \item Homogenous: $h_\la=\prod h_{\la_i}$ and $h_d=x_1^d+\dots+x_1^{d-1}x_2+\dots+x_1^{d-2}x_2^2+x_1^{d-2}x_2x_3+\dots$. In general $h_d=\sum_{\la\vdash d}m_\la$.
    \item Power sum: $p_\la=\prod p_{\la_i}$ and $p_d=\sum x_i^d$.
\end{itemize}

For Schur basis recall SSYT 

\begin{Ex}
    Consider $\la=(5,4,1)$, rows $\leq\to$ and columns $<$, we associate the monomial $x_1^2x_2^3x_3^3x_4^2:=x^T$.
\end{Ex}

\begin{itemize}
    \itemsep=-0.4em
    \item Schur: $s_\la=\sum_{T\in SSYT(\la)}x^T$ but also $\sum K_{\la\mu}m_\mu$ where the sum is over SSYT of shape $\la$, content $\mu$.
\end{itemize}

\subsubsection{Schur function motivation (preview)}

The first place they showed up is in the representation theory of Lie group.  The function $s_\la(x_1,\dots,x_n)$ is a character of irreducible polynomial representations of $GL_n$. In theoretical physics we have matrix groups acting on particles, representations are smaller matrix groups of things that they are mapping to. We want to take tensor product and direct sums of representations, the tensor product is related to multiplication of Schur function while direct sum into sum of Schur functions.\par 
There's also the Schur-Weyl duality which takes representations into the Weyl group. Under the \emph{Frobenius map}, $s_\la$ corresponds to irreducible representations of $S_n$.\par 
A more modern application of Schur function goes into geometry, $s_\la$ correspond to Schubert varieties in Grassmannians. Multiplication corresponds to interesections and sum to unions.\par 
There's also context in Probability Theory. But in the end, Schur positivity is important because of this connections. 

\begin{Def}
    $f\in\La$ is \term{Schur-positive} if $f=\sum c_\la s_\la$, $c_\la\geq 0$.
\end{Def}

\begin{Ex}
    $3s_{(2,1)}+2s_{(3)}$ schur pos but change $2$ to $-\half$ then not.
\end{Ex}

\section{day 2}

\subsection*{Alg defn Schur fncs}

\begin{Def}
    A function is \term{antisymmetric} if for $\pi\in S_n$,
    $$f(x_{\pi(1)},\dots,x_{\pi(n)})=\sgn(\pi)f(x_1,\dots,x_n).$$
\end{Def}

\begin{Ex}
    The following functions are antisymmetric:
    \begin{enumerate}
        \itemsep=-0.4em
        \item $f(x,y)=x-y$ then $f(y,x)=-f(x,y)$.
        \item $g(x,y)=(x-y)(x+y)$.
        \item $h(x,y)=x^2y-y^2x$.
    \end{enumerate}
\end{Ex}

Notice that the last function can factor as $h=-xy(x-y)$. We claim that this is always the case.

\begin{Lem} 
    Every antisymmetric polynomial $f$ in two variables $x,y$ can factor as $f(x,y)=(x-y)g(x,y)$ where $g$ is symmetric.
\end{Lem}

\begin{ptcbp}
Suppose $f$ is antisymmetric, then $f(x,x)=0$ by taking $y=x$. This means that $(x-y)\mid f$. Thus $f(x,y)=(x-y)g(x,y)$ and we now need to show that $g$ is symmetric. 
$$g(y,x)=\frac{f(y,x)}{y-x}=\frac{-f(x,y)}{-(x-y)}=\frac{f(x,y)}{x-y}=g(x,y).$$
\end{ptcbp}

\subsubsection*{Monomial Antisymmetric Functions}

\begin{Def}
Given a strict partition $\la=(\la_1,\dots,\la_k)$, $\la_1>\dots>\la_k$, we define 
$$a_\la(x_1,\dots,x_n)=x_1^{\la_1}\cdots x_k^{\la_k}\pm\text{similar terms}=\sum_{\pi\in S_n}\sgn(\pi)\prod_{k}x_{\pi(k)}^{\la_k}.$$ 
This $a_\la$ can be zero. 
\end{Def}

\begin{Ex}
    For two variables we've seen some antisymmetric polynomials. Let us calculate 
    $$a_{(3,1)}(x,y)=x^3y-y^3x.$$
    The smallest possible example in 3 variables is 
    $$a_{(2,1,0)}(x,y,z)=x^2y+y^2z+z^2x-y^2x-z^2y-x^2z.$$
    This can be factored as $(x-y)(y-z)(x-z)$. A similar construction gives us
    $$a_{(4,2,0)}(x,y,z)=x^4y^2+y^4z^2+z^4x^2-y^4x^2-z^4y^2-x^4z^2,$$
    but how does this factor? We get 
    $$a_{(4,2,0)}(x,y,z)=(x^2-y^2)(y^2-z^2)(x^2-z^2)=a_{(2,1,0)}(x,y,z)(x+y)(y+z)(x+z).$$
\end{Ex}

\begin{Lem}
The set $\set{a_\la}_{\la\ \text{strict}}$ is a basis of the antisymmetric polynomials over $\bQ$, $A_\bQ$. Even more any $a_\la$ is divisible by $a_\rho$ where $\rho=(n-1,n-2,\dots,2,1,0)$. 
\end{Lem}

As an algebra generator, $a_\rho$ is a generator.
\begin{ptcbp}
    \red{WRITE}
\end{ptcbp}

\begin{Prop}
The $a_\rho$ antisymmetric function is also the \term{Vandermonde determinant}: 
$$a_\rho=\det\begin{pmatrix}
    x_1^{n-1}&x_1^{n-2}&\dots&x_1^2&x_1&1\\
    x_2^{n-1}&x_2^{n-2}&\dots&x_2^2&x_2&1\\
    \vdots&\vdots&\ddots&\vdots&\vdots&\vdots\\
    x_n^{n-1}&x_n^{n-2}&\dots&x_n^2&x_n&1\\
\end{pmatrix}$$
\end{Prop}

\subsubsection{Schur Polynomials}

\begin{Def}
    The \term{Schur polynomial} of $\la\in\text{Par}$ is 
    $$s_\la(x_1,\dots,x_n)=\frac{a_{\la+\rho}(\un{x})}{a_\rho(\un x)}.$$
    Here $\la+\rho$ is the pointwise sum as arrays.
\end{Def}

\begin{Rmk}
This is the Weyl character proof. 
\end{Rmk}

The following proof is due to Proctor(1987) \red{find ref}

\begin{Lem}
    Any $a_\la$ can be seen as a determinant in the following way:
    $$a_\la(\un x)=\det\begin{pmatrix}
        x_1^{\la_1}&x_1^{\la_2}&\dots&x_1^{\la_n}\\
        x_2^{\la_1}&x_2^{\la_2}&\dots&x_2^{\la_n}\\
        \vdots&\vdots&\ddots&\vdots\\
        x_n^{\la_1}&x_n^{\la_2}&\dots&x_n
    \end{pmatrix}$$
\end{Lem}
\begin{ptcbp}
    We want to see that 
    $$\frac{a_{\la+\rho}(\un{x})}{a_\rho(\un x)}=\sum x^T$$
    where the sum ranges through $T$'s which are SSYT(la) with max entry $n$. 
    \begin{enumerate}
        \item We will show a recursion for the combinatorial definition that the character formula will also satisfy. It holds that 
        $$s_\la(\un x)=\sum s_\mu(\un x)x_n^{|\la|-|\mu|}$$
        where $\mu$ has $n-1$ parts with $\la_1\geq\mu_1\geq\la_2\geq\mu_2\dots$. 
        \item We also show that the ratio of determinants satisfies the same recursion. 
    \end{enumerate}
\end{ptcbp}

\begin{Ex}
    Consider $\la=(8,8,4,1,1)$ and $\mu=(8,5,2,1)$, then $\la\less\mu$ is a skew-table in which we can fill in $n$'s
\end{Ex}

\begin{Cor}
The Schur polynomials are a basis of $\La_\bQ$. 
\end{Cor}

\section{Day 3|20230125}

Recall $\La=\bQ[e_1,e_2,\dots]$ where the $e_j$'s are the elementary symmetric functions. So the $e_j$'s are algebraic generators of $\La$ and they're algebraically independent. Equivalently, as a vector space, $\set{e_\la\:\ \la\in\text{Par}}$ is a basis.

\begin{Prop}
    A homomorphism $f\:\La\to\La$ ($f(a+b)=f(a)+f(b),\ f(ab)f(a)f(b)$ for $a,b\in\La$) is fully determined by where it sends the $e_i's$. 
\end{Prop}

\begin{Def}
    The map $\om\in\End(\La)$ will send $e_j$ to $h_j$. 
\end{Def}

\begin{Ex}
    Consider $f=3e_{(2,1)}+2e_3$, then applying $\om$ we get 
    $$\om(f)=\om(3e_{(2,1)}+2e_3)=3h_{(2,1)}+2h_3.$$
    For $p_2$, we can decompose to $e_1^2-2e_2$. So 
    $$\om(p_2)=\om(e_1^2-2e_2)=h_1^2-2h_2$$
    and we can expand this last expression into 
    $$(x_1+x_2+\dots)^2-2(x_1^2+x_2^2+\dots+x_1x_2+x_1x_3+\dots)=-x_1^2-x_2^2-\dots$$
    and we recognize this last term as $-p_2$. \emph{This is not a coincidence.}
\end{Ex}

\begin{Th}
The map $\om$ is involutive.
\end{Th}

\begin{ptcbp}
    It suffices to prove that $\om(h_j)=e_j$. We will use power expansions and generating functions. We have 
    $$H(t)=\frac{1}{1-x_1t}\frac{1}{1-x_2t}\dots=\sum h_n(\un x)t^n,$$
    and this comes from expanding the $1/(1-y)$'s as geometric series. When collecting the coefficients of $t^n$ we get exactly $h_n(\un x)$. Similarly, for the elementary symmetric functions, 
    $$E(t)=(1+x_1t)(1+x_2t)\dots=\sum e_nt^n.$$
    When multiplying to obtain the coefficient of $t^n$ we get a plethora of different $x_j$'s which form the $e_j$'s. Now from this expressions we have $H(t)E(-t)=1$ which means that
    $$\left(\sum h_n(\un x)t^n\right)\left(\sum e_n(\un x)(-t)^n\right)\To \sum_{k=0}^{n}(-1)^ke_kh_{n-k}=0,\ n\geq 1.$$
    Now applying the map to the equation we get 
    $$\om\left(\sum_{k=0}^{n}(-1)^ke_kh_{n-k}\right)=\sum_{k=0}^{n}(-1)^kh_k\om(h_{n-k})=0.$$
    After reindexing, we get that both $e_j$'s and $\om(h_j)$'s are determined recursively by the $h_j$'s in the same way. Thus we conclude that $\om(h_j)=e_j$. 
\end{ptcbp}

\begin{Lem}
    The following equation holds for the power-sum symmetric functions:
    $$\exp\left(\sum\frac{1}{n}p_n(\un x)p_n(\un y)\right)=\prod_{i,j=1}^\infty\frac{1}{1-x_iy_j}=\:\Om(\un x,\un y).$$
    It also holds that 
    $$\Om(\un x,\un y)=\sum_la\frac{1}{z^\la}p_\la(\un x)p_\la(\un y)$$
    where $z_\la=\prod k^{m_k}m_k!$ where $m_k$ is the number of parts of $\la$ equal to $k$. 
\end{Lem}

\begin{ptcbp}
    We will prove both parts separately. For the first equation we will take the logarithm on both sides: 
    $$\sum\frac{1}{n}p_n(\un x)p_n(\un y)=\log\left(\prod_{i,j=1}^\infty\frac{1}{1-x_iy_j}\right)$$
    and after manipulating the logarithm we get 
    $$\sum_{i,j=1}^\infty(\log(1)-\log(1-x_iy_j))=\sum_{i,j=1}^\infty\sum_{n=1}^\infty \frac{1}{n}x_i^ny_j^n.$$
    We can separate\footnote{Are we using Fubini-Tonelli here?} into 
    $$\sum_{n=1}^\infty\frac{1}{n}\left(\sum_i x_i^n\right)\left(\sum_j y_j^n\right).$$
    Now taking $\exp$ on both sides we get equality.\par 
    By not removing the exponential we get the following expression
    $$\exp\left(\sum\frac{1}{n}p_n(\un x)p_n(\un y)\right)=\sum_{k=0}^\infty\frac{1}{k!}\left(\sum\frac{1}{n}p_n(\un x)p_n(\un y)\right)^k.$$
    To get a term of the form $p_\la(\un x)p_\la(\un y)$ we have to choose which parts of the $\la$ come from each of the factors in $\sum\frac{1}{n}p_n(\un x)p_n(\un y)$. If $\l(\la)=k$ then it comes from the $k^\textsuperscript{th}$ term in the exponential sum. If $\la=(\la_1,dots,\la_1,\dots,2,\dots,2,1,\dots,1)$ with $m_{\la_1}$ $\la_1$'s, $m_1$ $1$'s, then out of $k$ elements we have to choose $m_1$ $1$'s and so on. Thus there are $\binom{k}{m_{\la_1},\dots,m_1}$ choices and each $i$ in $\la$ comes with a $\frac{1}{i}$. Therefore the coefficient of $p_\la(\un x)p_\la(\un y)$ is 
    $$\frac{1}{k!}\frac{k!}{m_1!m_2!\dots}\frac{1}{1^{m_1}}\frac{1}{2^{m_2}}\dots=\frac{1}{z_\la}.$$
\end{ptcbp}

\begin{Lem}
    We have the following identities 
    $$\exp\left(\sum\frac{(-1)^{n-1}}{n}p_n(\un x)p_n(\un y)\right)=\prod_{i,j=1}^\infty\frac{1}{1+x_iy_j}=\sum_\la\frac{(-1)^{n-\l(\la)}}{z_\la}p_\la(\un x)p_\la(\un y).$$
\end{Lem}

\begin{Lem}\label{lem:james1}
    Another equality for $\Om(\un x,\un y)$ is 
    $$\Om(\un x,\un y)=\sum_\la m_\la(\un x)h_\la(\un y)$$
\end{Lem}

\begin{Th}
    It holds that $\om(p_\la)=(-1)^{n-k}p_\la$ where $k$ is the number of parts of $\la$.
\end{Th}

\begin{ptcbp}
    Applying $\om$ to $\Om$, but \emph{only working with $\un y$ variables} we get 
    $$\om(\Om)=\om\left(\sum_\la m_\la(\un x)h_\la(\un y)\right)=\sum_\la m_\la(\un x)e_\la(\un y)=\prod_{i,j=1}^\infty (1+x_iy_j)=\sum_{\la}\frac{1}{z_\la}(-1)^{n-k_\la}p_\la(\un x)p_\la(\un y).$$
    Comparing coefficients with 
    $$\om\left(\sum_la\frac{1}{z^\la}p_\la(\un x)p_\la(\un y)\right)$$
    we get the result.
\end{ptcbp}

\section{Day 4|20230127}

To continue exploring the ring of symmetric functions we need a couple of tools. One of them is the involution which we have already seen. But the other one is a scalar product which is compatible with the multiplication.

\subsection{Hall Inner Product}

Recall an inner product is a function 
$$\braket*{-}{-}\:\ V\x V\to \bQ$$
which is bilinear $\braket{u+v}{w}=\braket{u}{w}+\braket{v}{w}$ and the same on the other entry. For scalars the following behavior is expected $\braket{\la u}{v}=\braket{u}{\la v}=\la\braket{u}{v}$. Recall that if the base field is the complex numbers, then the inner product is Hermitian.

\begin{Def}
    We say that two vectors are \term{orthogonal} when $\braket{u}{v}=0$.
\end{Def}

This gives us a possible decomposition of space into several components. Suppose that $\set{u_\la}_{\la\in\text{Par}(n)},\set{v_\la}_{\la\in\text{Par}(n)}$ are basis of $\La^n$. So we would like a condition such as 
$$\braket{u_\la}{v_\mu}=\begin{cases}
    0\ \la\neq\mu,\\
    1\ \la=\mu.
\end{cases}$$

If we cap the dimension this says that $\braket{u}{v}$ is the usual dot product. But in infinite dimensions we don't have matrices. We'll call this basis \term{dual} to one another. If miraculously we have the same basis, then this basis is \term{orthonormal}.

\begin{Def}[Phillip Hall]
    The \term{Hall inner product} is defined so that $\braket{m_\la}{h_\mu}=\dl_{\la\mu}$.
\end{Def}

By defining the product on two basis, we have defined it for all other elements by bilinearity. 

\begin{Lem}
    The Hall inner product is symmetric.
\end{Lem}

\begin{Th}
    The Hall inner product is positive definite, this is $\braket{f}{f}\geq 0$ and equality is achieved when $f=0$.
\end{Th}

It's important to note that this statement is symmetric. However we are talking about an asymmetric definition. Last, before proving the statement we need a criteria for dual bases. But importantly, recall the result from last lecture: \ref{lem:james1}

\begin{Th}
    If ${u_\la},\set{v_\mu}$ are dual, then $\sum_\la u_\la v_\la=\Om$. 
\end{Th}

\begin{ptcbp}
    Fix a partition of $n$, then 
    $$\dl_{\la\mu}=\braket{m_\la}{h_\mu}=\braket{\sum_{\rho\vdash n}\al_{\la_\rho}u_\rho}{\sum_{\tau\vdash n}\bt_{\mu_\tau}v_\tau}=\sum_{\rho,\tau}\al_{\la_\rho}\bt_{\mu_\tau}\braket{u_\rho}{v_\tau}.$$
    We want $\braket{u_\rho}{v_\tau}=\dl_{\rho\tau}$, to that effect name $A_{\rho\tau}$ the matrix whose entries are $\braket{u_\rho}{v_\tau}$.\par 
    As $u$ and $v$ are dual bases, we have that $A=\id$. Thus $I=\al\bt^\sT$ and now $\dl_{\rho\tau}=\sum \al_{\la_\rho}\bt_{\la_\tau}$. We are now going to use the hypothesis and the interpretation of $m,h$ in the $u,v$ basis. We have 
    $$\Om=\sum\left(\sum\al u\right)\left(\sum\bt v\right)=\sum\left(\sum \al\bt\right)uv=\sum uv$$
    so the inner sum must be one and thus we are done. 
\end{ptcbp}

\begin{Cor}
    For the Hall inner product it holds that $\braket{p_\la}{p_\mu}=z_\la\dl_{\la\mu}$.
\end{Cor}

The key is to recall that $p_\la$ is an eigenfunction of $\om$. Also 1.3.5. By using a power-sum decomposition it is possible to prove that the Hall inner product is positive definite.

\begin{Cor}
    The $\om$ involution is orthogonal with respect to $\braket{-}{-}$. This is $\braket{\om f}{\om g}=\braket{f}{g}$. 
\end{Cor}
 
Once again, the idea is to transfer to power-sum and use the fact that it's an eigenfunction.

\section{Interim 1}

\begin{Th}[Fundamental Theorem of Sym. Fnc. Thry.]
    Every symmetric function can be written uniquely in the form $\sum_{\la}c_\la e_\la$ with $c_\la\in\bQ$. 
\end{Th}

There are at least two proofs if not more of this fact. The first comes from Maria Gillespie's blog which Mark Haiman presented to her. 

\begin{ptcbp}%http://www.mathematicalgemstones.com/gemstones/opal/the-fundamental-theorem-of-symmetric-function-theory/3/
    It suffices to prove the transition matrix between $m$ and $e$ is invertible.
\end{ptcbp}

For proof 2 read \cite{StanleyEnum2} pg. 290. Proof 3 in another Maria post %http://www.mathematicalgemstones.com/gemstones/opal/addendum-an-alternate-proof-of-the-ftsft/

\section{Day 5|20230130}

\begin{Ej}
Compute $\om(s_{(3,1)})$.
\end{Ej}

\begin{ptcbr}
We have that 
%$$s_{(3,1)}=m_{(3,1)}+\young{112,2}+\young{112,3}+\young{113,2}+\young{123,4}+$$
By Jacobi-Trudi 
$$s_{(3,1)}=\det\twobytwo{h_3}{h_4}{1}{h_1}=h_{(3,1)}-h_4.$$
Using the omega involution, we get 
%$$\om(s_{(3,1)})=e_{(3,1)}-e_4=s(\young{~,~,~~}).$$
\end{ptcbr}

Recall that $\om: h_n\otto e_n$, $\om p_k =(-1)^{k-1}p_k$. We have the following questions, where do $m$ and $s$ map to? Also 
$$\braket{m}{h}=\dl,\ \braket{p}{p/z}=\dl,$$
but what are $e$ and $s$ dual to?

\begin{Def}
    We call $\om m_\la = f_\la$ the \term{forgotten basis}.
\end{Def}

There's not much we could say about them, they are not Schur positive and there's no patterns. 

\subsubsection{Dual to $e$}

Recall $\om$ is an isometry, so $\braket{\om f}{\om g}=\braket{f}{g}$, so
$$\braket{e_\la}{?}=\braket{h_\la}{\om ?}=\dl_{\la\mu}.$$
Since $\braket{h}{m}=\dl$, then applying $\om$ again we get that $\braket{e_\la}{f_\mu}=\dl_{\la\mu}$.

\subsubsection{RSK algorithm}

We want to show two things:
$$\om s_\la=s_{\la^\sT},\ \braket{s_\la}{s_\mu}=\dl_{\la\mu}.$$

\begin{Prop}
    It holds that 
    $$\sum_{\la}s_\la(\un x)s_\la(\un y)=\Om=\sum_{\la}m_\la(\un x)h_\la(\un y)$$
\end{Prop}

\begin{ptcbp}
    The sum on the left is 
    $$\sum_{(S,T)SSYT}x^Sy^T$$
    so we will study pairs $(S,T)$ of SSYT of the same shape to show that they're equal to the sum on the right. 
\end{ptcbp}

algorithm: process of doing the bijection.\par 
The RSK bijection takes a pair $(S,T)$ of SSYT of the same shape and it maps it to ``two-line arrays'' of length $n$. 

\begin{Def}
    A \term{two-line array} is a matrix in $\cM_{2\x n}(\bZ_{\geq 0})$ such that 
    \begin{enumerate}[i)]
        \itemsep=-0.4em
        \item The bottom row is weakly increasing.
        \item If $b_i=b_{i+1}$, then $a_{i}\leq a_{i+1}$, where $a$'s are the top row and $b$'s the bottom row.
    \end{enumerate}
\end{Def}

\begin{Ex}
    Consider the matrix 
    $$\begin{pmatrix}
        1&1&2&1&4&2&3&1&2\\
        1&1&1&2&2&3&3&4&4
    \end{pmatrix}$$
    Within ``blocks'', there is a weak increment. From right-to-left we will find a pair of SSYT. We will ``insert'' top row letters from left-to-right.
    \begin{enumerate}
        \item Place 1st letter $\young(1)$
        \item For each letter, if it can go at the end of last row, put it there 
        $$\young(11)\leftarrow 2,\ \young(112)\leftarrow 1$$
        but one can't go after 2.
        \item Otherwise if inserting $b_1$, let $c$ be the leftmost $>b$, ``bump $c$'', then insert $c$ into the next row. 
        $$\young(111,2)$$
    \end{enumerate}
    For the bottom row, place in a new square at each step to form a ``recording tableau''. The recording tableau always matches the shape of the insertion one. The first three steps lead to $\young(111)$ in the recording one. But in the fourth step we get $\young(111,2)$. The next step leads us to 
    $$\young(1114,2),\quad \young(1112,2)$$
    then in insertion, 2 bumps 4 and 4 doesn't bump 2 on next row, so we get 
    $$\young(1112,24),\quad \young(1112,23)$$
    The three is no problem so 
    $$\young(11123,24),\quad \young(11123,23)$$
    then the next one bumps out the 2, the 2 bumps the 4 on the second row to get 
    $$\young(11113,22,4),\quad \young(11123,23,4)$$
    Finally 
    $$\young(11112,223,4),\quad \young (11123,234,4).$$
\end{Ex}

Why do we get SSYT. The insertion tableau gives us the question, can we make a column non-increasing? No, we are always bumping something bigger. Imagine we bump $c>b$ with $b$, then $c$ replaces something that goes to the left.
$$\young(\leq bc,~~,~)\To\young(~b~,~d,~)$$
and $d>c$ so it bumps something else. The recording tableau is also a SSYT. Let us prove it. 

\begin{Lem}[Key Lemma 1]\label{lemma-key-lemma-SSYT}
    The insertion path (sequence of squares that are bumped) moves up and weakly left. 
\end{Lem}

\begin{Lem}[Key Lemma 2]\label{lemma-consec-inserts}
If $a\leq b$ and $T$ is a SSYT, computing 
$$T\leftarrow\young(a)\leftarrow\young(b),$$
the intersection path of $a$ in $T$ lies strictly left of the intersection path of $b$ in $T\leftarrow\young(a)$.
\end{Lem}

\begin{ptcbp}
    We will do induction on the rows with an example.
\end{ptcbp}

\begin{Ex}
    Consider 
    $$\young(111223,22334,3355,44)$$
    Inserting $1$ we bump the 2, then the 3 and finally the 5. We get 
    $$\young(111o23,22t34,33t55,44f)$$
    so inserting the 2 we bump 3,4,5. And they will be to the side of the last sequence. 
\end{Ex}

\section{Day 6| 20230201}
\begin{Ej}
    Apply RSK to $\begin{pmatrix}
        3&2&4&1&5\\1&2&3&4&5
    \end{pmatrix}$
\end{Ej}

\begin{ptcbr}
    We get 
    $\young(145,2,3),\quad \young(135,2,4)$.
\end{ptcbr}

Notice that we got STANDARD Young tableau. So to prove it's a bijection we will begin with all different numbers.

\begin{Lem}
    The RSK bijection is a bijection between pairs of standard Young tableaux of the same shape and ``permutations'' ($2\x n$ matrices whose rows are permutations.)
\end{Lem}

To prove it's a bijection we will find an inverse by reversing the process. Look at the recording tableau, we will bump out the largest number. We will take $S$ as the recording tableau. Then we start with the spot on $S,T$ which corresponds to largest label in S.
\begin{itemize}
    \item If $b$ is the item in such a square we ``un-bump'' it.
    \begin{itemize}
        \item If in bottom row, just remove.
        \item Else, let $c$ be the rightmost entry in row below $b$ that is less than $b$. Then replace $b$ with $c$ and repeat the process with $c$ until the letter that is removed is done by the just removing it. 
    \end{itemize}
    Then we add the two letters to the matrix from right-t-left.
\end{itemize}

With the original tableau we remove the $5$ and the $5$ to get 
$$\young(14,2,3),\quad \young(13,2,4)$$
then the 4 indicates that in $T$ we must ``un-bump'' the 3. The three un-bumps the 2, the 2 to the 1 so that we get 
$$\young(24,3),\quad \young(13,2).$$
Now we get the matrix $\begin{pmatrix}
    x&x&x&1&5\\x&x&x&4&5
\end{pmatrix}$ and removing the 3 from $S$ just removes the 4 from $T$ as it is in the bottom row.\par 
Now as this two sets are in bijection, this means that they have the same size.

\begin{Cor}
    Let $f^\la$ be the number of standard Young tableau of shape $\la$. Then 
    $$\sum_{\la\vdash n}(f^\la)^2=n!.$$
\end{Cor}

We will generalize one step at a time. Let us now assume that $T$ is semi-standard. On the matrix, we will have that the top row is now random, but the bottom row is still from $1$ to $n$. 

\begin{Lem}[Schensted]
    There is a bijection between $(S,T)$, $S$ is standard, $T$ is SSYT, and words of length $n$.
\end{Lem}

\begin{Ex}
    Consider the matrix $\begin{pmatrix}
        2&1&3&1&3\\1&2&3&4&5
    \end{pmatrix}$ which returns the two Young tableau 
    $$\young(113,23),\quad \young(135,24).$$
\end{Ex}

The proof of the inverse is similar but when un-bumping, we must bump the rightmost entry \emph{strictly} smaller than $b$. But we don't need this, we will do it more creatively.

\begin{Def}
    Suppose $T$ is a Young tableau. Then 
    \begin{enumerate}[i)]
        \itemsep=-0.4em
        \item The \term{reading word} of $T$ $\text{rw}(T)$ is the concatenation of rows from top to bottom.
        \item The \term{standarization} of an SSYT $T$, $\text{std}(T)$, is the unique    SYT with same relative order of entries, ties broken with ``reading order''.
        \item The standarization of a word is similar
       \end{enumerate}
\end{Def}

In the previous example, the reading word is 
$$\young(113,23)\to 23113.$$
The standarization are as follows:
$$\young(113,23)\to \young(125,34),\quad 23113\to34125.$$
We can standarize the matrix 
$$\begin{pmatrix}
    2&1&3&1&3\\1&2&3&4&5
\end{pmatrix}\to\begin{pmatrix}
    3&4&1&2&5\\1&2&3&4&5
\end{pmatrix}$$
and this matrix corresponds to the pair $(S,T)$ where $T$ is $\young(125,34)$. In essence, the following diagram commutes
\begin{figure}[h]
    \centering
    % https://tikzcd.yichuanshen.de/#N4Igdg9gJgpgziAXAbVABwnAlgFyxMJZABgBpiBdUkANwEMAbAVxiRAAoBlUgFQEoQAX1LpMufIRQBGclVqMWbAExSAzGqEiQGbHgJEyUufWatEHbjwDkA4aN0SiMo9ROLz6gCxKArELkwUADm8ESgAGYAThAAtkhkIDgQSErUABYwdFBIYEwMDK4KZiBwONnUDHQARjAMAApiepIgkVhBaTiaEdFxiDKJyYiqdiBRsUiq1EkphaZsAEqcANJdoz1I-dOICW7FiysV1bUNDvrmre2dghSCQA
\begin{tikzcd}
    {(S,T)} \arrow[d, "std"'] & 21313 \arrow[d] \arrow[l, "RSK"'] \\
    {(S,T')}                  & 31425 \arrow[l, "RSK"]           
    \end{tikzcd}
\end{figure}

\begin{Def}
    Given a content $\mu=(\mu_1,\dots,\mu_k)$ with $\sum\mu_k=n$ (not nec. partition). Then the de-standarization with respect to $\mu$ of a SYT $T$ is a $SSYT$ $T'$ such that $\text{std}(T')=T$.
\end{Def}

In this case 
$$\young(125,34)\xrightarrow[\text{std}^{-1}(2,1,2)]{}\young(113,23).$$

Recall now lemma \ref{lemma-consec-inserts} about consecutive insertions.

\subsubsection{The Full RSK}

We are now going to prove that there is an inverse to the original RSK function. Consider the following example 
$$\begin{pmatrix}
    1&1&2&1&4&2&3&1&2\\
    1&2&3&4&5&6&7&8&9
\end{pmatrix}\to \young(11112,223,4),\quad \young(12357,469,8)$$
The matrix $\begin{pmatrix}
    1&1&2&1&4&2&3&1&2\\
    1&1&1&2&2&3&3&4&4
\end{pmatrix}$
can be standarized to our word matrix. Then the table $\young(11123,234,4)$ also standarizes to the word table. 

\section{Day 7| 20230203}

\begin{Ej}
    Expand $h_{(3,2)}$ in Schur basis.
\end{Ej}

\begin{ptcbr}
    This is $s_{(3,2)}+s_{(4,1)}+s_{(5)}$.
\end{ptcbr}

Recall that $(s_\la)$ form an orthonormal basis and $m$ and $h$ are dual basis. This means that if $f$ is a symmetric function then 
$$f=\sum_\la c_\la s_\la\To c_\la=\braket{f}{s_\la},\ f=\sum_\la a_\la m_\la\To a_\la=\braket{f}{h_\la}.$$
Lets suppose now that $f$ is any homogenous symmetric function. We will calculate the coefficient of $s_\la$ in an $h_\mu$ expansion:
$$\braket{h_\mu}{s_\la}=\braket{s_\la}{h_\mu}$$
and we can interpret this as the coefficient of $m_\mu$ in $s_\la$. This amount is precisely the Kostka coefficient $K_{\la\mu}$. Thus we have the formula $h_\mu=\sum_\la K_{\la\mu}s_\la$.

\subsection{Properties of the Schur functions}

We wish to show that $\braket{s_\la}{s_\mu}=\dl_{\la\mu}$ and $\om s_\la=s_{\la^\sT}$.

\begin{Prop}
    $\sum_\la s_\la(\un{x})s_\la(\un{y})=\sum_\la m_\la(\un{x})h_\la(\un{y})$
\end{Prop}

\begin{ptcbp}
    Expanding the sum on the left we obtain 
    $$\sum_\la s_\la(\un{x})s_\la(\un{y})=\sum\la\left(\sum_{T\in SSYT(\la)}x^T\right)\left(\sum_{S\in SSYT(\la)}y^S\right)=\sum_{(T,S),\ SSYT\text{same shape}}x^Ty^S$$

    This is basically an RSK pair and this correspond to two-line arrays, so this sum could be the same as summing over them. Thus this is  
    $$\sum_{2\text{line arrays}}x_{a_1}\dots x_{a_n}y_{b_1}\dots y_{b_n}.$$
We will now find the coefficient of $m_\la(\un y)$ in this expansion and show that it is $h_\la(\un x)$\par 
What are all the ways to obtain $y_1^{\la_1}\dots y_k^{\la_k}$?
$$\begin{pmatrix}
    a_1^{(1)}&\dots&a_k^{(1)}&a_1^{(2)}&\dots&a_k^{(2)}&\dots\\
    1&1&1&2&2&2&\dots
\end{pmatrix}$$
And note that $a_1^{(i)}\leq\dots\leq a_{\la_i}^{(i)}$ for all $i$, so the coefficient is 
$$\sum_{(a^{(i)}) valid tuples}x_{a_1^{(1)}}\dots x_{a_{\la_k}^{(k)}}$$
but this factors as 
$$\prod_{i=1}^k\sum_{a_1^(i)\leq\dots\leq a_{\la_k}^{(i)}}x_{a_1^{(i)}}\dots x_{a_{\la_k}^{(i)}}.$$
We can split this because the choices are independent of the blocks and then multiply the functions together. The last term is $h_{\la_i}$ and the product is $h_\la.$
\end{ptcbp}

If $(T,S)$ RSKs inverse to $\twobythree{1}{3}{2}{1}{1}{2}$ then $x^Ty^S$ is $x_1x_3x_2y_1y_1y_2$.

\begin{Cor}
    $\om s_\la=s_{\la^\sT}$. 
\end{Cor}

\begin{ptcbp}
    It suffices to show $\braket{s_{\la^\sT}}{e_\mu}=K_{\la\mu}$ because $\braket{s_\la}{h_\mu}=K_{\la\mu}$ which implies that $\braket{\om s_\la}{e_\mu}=K_{\la\mu}$.\par 
    In other words, we wish to show that the coefficient of $s_\la$ in $e_\mu$ is $K_{\la^\sT\mu}$, the number of $SSYT$ shape $\la^\sT$, content $\mu$.\par 
\red{CONT}
\end{ptcbp}

\subsection*{Pieri Rule}

\begin{Def}
    A \term{skew shape} is a diagram formed by subtracting a smaller Young diagram from a larger one.\par 
    A \term{horizontal strip} is a skew shape where no two boxes are in the same column. Similar a \term{vertical strip} doesn't have boxes in the same row.
\end{Def}

\begin{Ex}
    Suppose $\la=(5,4,4,1)$ and $\mu=(4,2,2)$. Then 
    $$\la=\young(~,~~~~,~~~~,~~~~~),\quad \mu=\young(~~,~~,~~~~)$$
    so $la/\mu$ is INSERT DIAG.
    Not horizontal nor vertical.
\end{Ex}

In a Young tableau, the biggest number forms a horizontal strip, so in general Young tableaux are made up of horizontal strips.
$$\young(4,344,2233,111234).$$

\begin{Th}[Pieri]
    Let $r\in\bN$, then 
    $$e_rs_\la=\sum_{\rho/\la\text{vert. strip size }r}s_\rho$$ 
    $$h_rs_\la=\sum_{\rho/\la\text{horiz. strip size }r}s_\rho$$
\end{Th}

This is basically all the ways to fill up the shapes.

\begin{ptcbp}
    $$h_rs_la=s_{(r)}s_\la=\left(\sum_{T\in SSYT((r))}x^T\right)\left(\sum_{S\in SSYT(\la)}x^S\right)$$
\end{ptcbp}

\begin{Ex}
    $h_3s_{(3,1)}$ is $x^Tx^S$ is inserting the boxes of $T$ one at a time in $S$. 
    $$\young(3,123)\leftarrow \young(112)=\young(3,23,1112)$$
    so by \ref{lemma-key-lemma-SSYT} about insertion path, the new squares are a horizontal strip which is the $s_\rho$ in the Pieri rule. Unbumping we recover \red{something}.
\end{Ex}

\section{Day 8| 20230206}

\begin{Ej}
    Apply RSK to $82357146$ and $62235124$.
\end{Ej}

\begin{ptcbr}
    $$\young(8,2357),\young(2,1345)$$
    then $1$ bumps the 2, 2 bumps 8
    $$\young(8,257,1346),\young(6,278,1345)$$
    The next one standarizes to the last string. The same recording table but we get for insertion 
    $$\young(6,235,1224).$$
\end{ptcbr}

\subsection{Consequences of RSK}

We will talk about increasing and decreasing subsequences.

\begin{Def}
    A longest increasing subsequence of a word $w\in\bN^{n}$ is a subsequence $w_{i_1}\leq\dots\leq w_{i_\l}$ with $i_1<\dots<i_\l$ such that $\l$ is as large as possible. We will write $\l(w)$ to be the length of the longest increasing subsequence.\par 
    A longest decreasing subsequence of a word is 
    $w_{i_1}>\dots>w_{i_d}$ with $i_1<\dots<i_d$. In this case $d(w)$ is the longest decreasing.
\end{Def}

\begin{Ex}
    In the case of $82357146$, we have $2357,2356,146,2346$. Notice that this is the length of ?? of the Young tableau. For decreasing we have $821,831,\dots$, the height of the Young tableau is the longest decreasing subsequence.
\end{Ex}

\begin{Th}\label{th-length-long-dec-inc-word}
    Suppose $w$ is a word, $S=\text{ins}(w)$ is the insertion tableau through RSK and $\la=\text{sh}(S)$ is the shape of the table. Then $\l(w)=\la_1$ and $d(w)=\la_1^{\sT}$. 
\end{Th}

To prove this we will develop some tools.

\begin{Lem}\label{lem-reading-word}
    For a tableau $T$, $\text{ins}(\text{rw}(T))=T$.
\end{Lem}

The reading word of $\young(8,257,1346)$ is $82357146$ which inserts to the same table precisely.

\begin{Rmk}
    The column reading word also works! For this table it's $82153746$. We get a bunch of decreasing subsequences. $821$ creates the first column by bumping, then $53$ creates the second column and so on. 
\end{Rmk}

Let's analyze the longest increasing subsequence of the reading word. Clearly we can get the bottom row as a longest subsequence, but looking in the reading order we need to go to the right. Going down decreases!

\begin{Lem}
    If $\la=\text{sh}(T)$ then $\l(\text{rw}(T))=\la_1$ and $d(\text{rw}(T))=\la_1^\sT$.
\end{Lem}

\begin{ptcbp}
    Given an entry $a\in T$, let $b\in T$ such that $a<_{\text{ro}}b$. Then $b$ is in a column to the right of $a$, this means that 
    $$\l(\text{rw}(T))\leq \#\text{columns}=\la_1.$$
    The bottom row is an example of a subsequence where the length is achieved. So equality holds.
\end{ptcbp}

For decreasing it's equivalent. Now, how do we tell when two words have the same insertion tableau?

\begin{Ex}
    In the case of all permutations in $S_3$ we have that some are equivalent \red{FILL}
\end{Ex}

\subsubsection{Knuth equivalence}

\begin{Def}
    A \term{Knuth move} on a permutation swaps two letters $a,c$ if $a<b<c$ (reading order) and one of consecutive subsequences $acb,cab,bac,bca$ appears in the word.\par 
    Two words are \term{Knuth-equivalent} if they differ by a sequence of Knuth-moves.
\end{Def}

In the first case, $b$ is between $a,c$ and those are always together.

\begin{Prop}
    Knuth equivalence defines an equivalence relation on $S_n$. 
\end{Prop}

\begin{Th}
    Two words $\pi,w$ are Knuth-equivalent iff $\text{ins}(w)=\text{ins}(\pi)$.
\end{Th}

\begin{Ex}
    In size $4$, $1234$ is in its own class because we don't have any Knuth moves available. Same thing happens with $4321$.\par 
    Consider $1243$, if we apply Knuth moves we can get 
    \begin{itemize}
        \begin{multicols}{2}
            \itemsep=-0.4em
            \item $1423$
            \item $4123$
        \end{multicols}
    \end{itemize}
    All of these have the insertion tableau $\young(4,123)$ whose reading word is $4123$.\par 
    For the tableau $\young(24,13)$, its reading word is $2413$. Applying Knuth moves we get only $2143$, which is the column reading word.\par 
    The tableau $\young(34,21)$'s equivalence class also has size 2. 
\end{Ex}

\begin{Prop}
    If two tableau have the same shape, their equivalence classes have the same size. 
\end{Prop}

We are seeking to prove $\l(w)$ is invariant under Knuth moves. This will imply the theorem \ref{th-length-long-dec-inc-word} because once we know that things have the same insertion tableau and the reading word has the same longest increasing subsequence length.

\section{Day 9| 20230208}

\begin{Ej}
    Insert $f,g$ and then $c$ into 
    $$\young(k,eij,abdhl)$$
    and then $f,c$ and then $g$.
\end{Ej}

\begin{Ex}
    The Knuth equivalence class of words whose insertion tableau is 
    $$\young(34,125).$$
    The reading word is $34125$ and we can Knuth-move it. The $341$ can switch into $314$ (this has the form $bac$). From that one we can switch $2$ and $5$ to get $31452$. Once again with $314$ we get $34152$ and $34512$.\par 
    In total we have $5$ elements.
\end{Ex}

\begin{Prop}
    The size of the Knuth equivalence class whose insertion tableau is $T$ with shape $\la$ is $\#SYT(\la)$.
\end{Prop}

\begin{ptcbp}
    We have one permutation in the Knuth equivalence class for every recording tableau $S$ that can be paired with $T$.
\end{ptcbp}

The Knuth equivalence class of $\young(34,125)$ can be identified by RSK with the pairs $(T,S)$ and $S$ varies through all SYT of corresponding shape.\par
Also, recall that by that hook-length formula we have that 
$$\#SYT(\la)=\frac{|\la|!}{\prod_{\text{hooks}\subseteq T} \text{size hooks}}.$$

\begin{Th}
    Two permutations $\pi, w$ have the same insertion tableau if and only if $\pi$ is Knuth-equivalent to $w$.
\end{Th}
%BUMP: QUITAR EL PRIMERO MAS GRANDE Y PARRIBA
\begin{ptcbp}
By induction on the length, we can assume $\pi, w$ differ by a single Knuth-move on the last $3$ letters. We separate into cases:
\begin{enumerate}[i)]
    \itemsep=-0.4em
    \item Want 
    $$T'\leftarrow b\leftarrow c\leftarrow a=T'\leftarrow b\leftarrow a\leftarrow c$$
    Note that $\ttt{IP}(b)<\ttt{IP}(c)$ by lemma \ref{lemma-consec-inserts} of consecutive insertions and $\ttt{IP}(a)$ is \emph{weakly left} of $\ttt{IP}(b)$ from which holds $\ttt{IP}(a)$ is strictly left of $c$'s. So we can switch order.
    \item In the other case  we want
    $$T'\leftarrow c\leftarrow a\leftarrow b=T'\leftarrow a\leftarrow c\leftarrow b.$$
    $\ttt{IP}(a)$ is \emph{weakly left} of $c$'s. If it's \emph{strictly}, then we can switch, but otherwise the insertion paths of $a$ and $c$ collide. \red{CHECK NOTES}
\end{enumerate}
Now on the other direction, we wish to show that two permutations with the same insertion tableau are Knuth-equivalent.\par 
It suffices to show that they are Knuth-equivalent to the reading word. By induction of the size of the word, suppose $\text{ins}(w')=T'$. Then $w'\sim\text{rw}(T')$ for $w'$ of length $n-1$.\par 
Let $w\in S_n$ with $b=w_n$. If $T'=\text{ins}(w_1,\dots,w_{n-1})$, by induction $w_1\dots w_{n-1}\sim\text{rw}(T')=(first\ row)\dots(last\ row)$. 
\end{ptcbp}

\begin{Ex}
    For the second case consider the table 
$$\young(389,147)$$
and we insert $6,2$ then $5$ but then $2,6$ and then $5$. In the first case, \red{DUNNO}\par 
In the second case consider 
$$T'=\young(6,47,1258)$$
\end{Ex}

%%%%%%%%%%%% Contents end %%%%%%%%%%%%%%%%
\ifx\nextra\undefined
\printindex
\else\fi
\nocite{*}
\bibliographystyle{plain}
\bibliography{bibiCombi2.bib}
\end{document} 

