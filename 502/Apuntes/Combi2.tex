\documentclass[12pt]{memoir}

\def\nsemestre {I}
\def\nterm {Spring}
\def\nyear {2023}
\def\nprofesor {Jeff Achter}
\def\nsigla {MATH519}
\def\nsiglahead {Complex Analysis}
\def\nlang {ENG}
\input{../../headerVarillyDiff}

\begin{document}
%\clearpage
\maketitle
%\thispagestyle{empty}
{\small
\setlength{\parindent}{0em}
\setlength{\parskip}{1em}

This course is an introduction to analytic functions of a single complex variable.  The subject is beautiful.-- it turns out that a function with a complex derivative is highly structured -- and enjoys a give and take with many other areas of mathematics.

\subsubsection*{Requirements}
Knowledge of convergence of sequences, series: limits, continuity, differentiation, integration of one-variable functions is required.
}
\newpage
\tableofcontents
%\begin{multicols}{2}
\chapter{Symmetric functions}

\section{Recall}

\begin{Def}
$f(x_1,x_2,\dots)$ is \term{symmetric} if it's fixed under permutations of variables.
\end{Def}

\begin{Ex}
    $f(x_1,\dots,x_4)=x_1^5+\dots+x_4^5$. This is known as $p_5$ or $m_{(5)}$, where $p$ is the power-sum symmetric function and $m$, the monomial symmetric function.  
\end{Ex}

\begin{Ex}
    Consider $g=x_1^4x_2+x_1^4x_3+\dots+x_i^4x_j+\dots+3x_1+\dots=m_{(4,1)}+3m_{(1)}$.
\end{Ex}

Let us recall some \textbf{notation}:
\begin{enumerate}[i)]
    \itemsep=-0.4em
    \item $\La_R(x_1,\dots,x_n)$ is the ring of symmetric polynomials over $R$. In \emph{infinitely} many variables we have $\La_R(\un{x})$.
\end{enumerate}

In the case $R=\bQ$, then $\dim \La_Q(\un x)_{(d)}$, where every monomial has degree $d$, is $p(d)$. This is the number of partitions of $d$. Because for every partition we can form monomials and monomials form a basis.

\subsection*{Bases of $\La_Q$}

Suppose $\la=(\la_1,\dots,\la_k)$ with $\la_1\geq\dots\geq\la_k$. 
\begin{itemize}
    \item Monomial: $m_\la=\sum_{i_1\neq\dots i_k}x_{i_1}^{\la_1}\dots x_{i_k}^{\la_k}$.
    \item Elementary: $e_\la=\prod e_{\la_i}$ where $e_d=m_{(1,1,\dots,1)}$ ($d$ ones).
    \item Homogenous: $h_\la=\prod h_{\la_i}$ and $h_d=x_1^d+\dots+x_1^{d-1}x_2+\dots+x_1^{d-2}x_2^2+x_1^{d-2}x_2x_3+\dots$. In general $h_d=\sum_{\la\vdash d}m_\la$.
    \item Power sum: $p_\la=\prod p_{\la_i}$ and $p_d=\sum x_i^d$.
\end{itemize}

For Schur basis recall SSYT 

\begin{Ex}
    Consider $\la=(5,4,1)$, rows $\leq\to$ and columns $<$, we associate the monomial $x_1^2x_2^3x_3^3x_4^2:=x^T$.
\end{Ex}

\begin{itemize}
    \itemsep=-0.4em
    \item Schur: $s_\la=\sum_{T\in SSYT(\la)}x^T$ but also $\sum K_{\la\mu}m_\mu$ where the sum is over SSYT of shape $\la$, content $\mu$.
\end{itemize}

\subsubsection{Schur function motivation (preview)}

The first place they showed up is in the representation theory of Lie group.  The function $s_\la(x_1,\dots,x_n)$ is a character of irreducible polynomial representations of $GL_n$. In theoretical physics we have matrix groups acting on particles, representations are smaller matrix groups of things that they are mapping to. We want to take tensor product and direct sums of representations, the tensor product is related to multiplication of Schur function while direct sum into sum of Schur functions.\par 
There's also the Schur-Weyl duality which takes representations into the Weyl group. Under the \emph{Frobenius map}, $s_\la$ corresponds to irreducible representations of $S_n$.\par 
A more modern application of Schur function goes into geometry, $s_\la$ correspond to Schubert varieties in Grassmannians. Multiplication corresponds to interesections and sum to unions.\par 
There's also context in Probability Theory. But in the end, Schur positivity is important because of this connections. 

\begin{Def}
    $f\in\La$ is \term{Schur-positive} if $f=\sum c_\la s_\la$, $c_\la\geq 0$.
\end{Def}

\begin{Ex}
    $3s_{(2,1)}+2s_{(3)}$ schur pos but change $2$ to $-\half$ then not.
\end{Ex}

\begin{Ej}[1.1 Stein \& Shakarchi]
Describe geometrically the sets of points $z$ in the complex plane defined by the following relations:
\begin{enumerate}
    \itemsep=-0.4em
    \item $|z-z_1|=|z-z_2|$ where $z_1,z_2\in\bC$.
    \item $1/z=\ov z$.
    \item $\Re(z)=3$
    \item $\Re(z)>c$, (resp.,$\geq c$) where $c\in\bR$.
    \item $\Re(az+b)>0$ where $a,b\in\bC$.
    \item $|z|=\Re(z)+1$.
    \item $\Im(z)=c$ with $c\in\bR$.
\end{enumerate}
\end{Ej}

\begin{ptcbr}
    \begin{enumerate}[i)]
        \itemsep=-0.4em
        \item The first set is the set of points at the same distance from $z_1$ and $z_2$. If we consider the line segment $z_1z_2$, then the set in question is the bisector of that line segment.
        \item Note that
        $$1/z=\ov z\iff 1=\ov zz\iff 1=|z|^2\iff 1=|z|,$$
        thus the set is the unit circle.
        \item The set is a perpendicular line to the real axis at $z=3$.
        \item This infinite set is an infinite half plane to the right (but not including) of the line $z=c$. In the other case, we do include the line in question.
        \item \red{DO}
        \item The equation in question is equivalent to 
        $$\Re(z)^2+\Im(z)^2=(\Re(z)+1)^2.$$
        To ease the notation, assume $z=x+iy$. Then the equation reads 
        $$x^2+y^2=x^2+2x+1\iff y^2=2x+1\iff x=(y^2-1)/2.$$
        It holds the the parabola in question contains the points which satisfy the equation.
        \item This set is a line parallel to the real axis at $z=c$
    \end{enumerate}
\end{ptcbr}

\begin{Ej}
    Do the following:
    \begin{enumerate}[i)]
        \itemsep=-0.4em
        \item Show that the complex conjugation map $\kp:\bC\to\bC,\ z\mapsto\ov z$ is an involution, i.e., a ring homomorphism such that $\kp\circ\kp=\id$.
        \item Suppose $a\in\bR,\ z\in\bC$. Show that 
        $$\Re(az)=a\Re(z),\word{and}\Im(az)=a\Im(z).$$
    \end{enumerate}
\end{Ej}

\begin{ptcbr}
    Let us take $z=x+iy$ with $x,y\in\bR$.
    \begin{enumerate}[i)]
        \itemsep=-0.4em
        \item We have $\ov z=x+i(-y)=x-iy$. Once more we get $\ov{\ov z}=x-i(-y)=x+iy=z$. Thus $\ov{\ov z}=z$ for any $z\in\bC$. In conclusion $\ov{\ov \.}=\id$.
        \item It holds that 
        \begin{align*}
            &\Re(az)=\Re(ax+aiy)=ax=a\Re(z),\\
            &\Im(az)=\Im(ax+aiy)=ay=a\Im(z).
        \end{align*}
    \end{enumerate}
\end{ptcbr}

\begin{Ej}
    Do the following:
    \begin{enumerate}[i)]
        \itemsep=-0.4em
        \item Prove that $|z+w|^2=|z|^2+|w|^2+2\Re(z\ov w)$.
        \item Use this to prove the parallelogram rule: $|z+w|^2+|z-w|^2=2(|z|^2+|w|^2)$.
    \end{enumerate}
\end{Ej}

\begin{ptcbr}
    \begin{enumerate}[i)]
        \itemsep=-0.4em
        \item Note that 
        $$|z+w|^2=(z+w)\ov{(z+w)}=(z+w)(\ov z+\ov w)=z\ov{z}+w\ov{z}+z\ov{w}+w\ov w.$$
        The number $w\ov z$ is the conjugate of $z\ov w$, and summing a number and its conjugate returns twice its real part. Thus we get the desired identity. 
        \item As the past identity holds for all complex numbers, it holds when $w=-w$. This means that 
        $|z-w|^2=|z|^2+|-w|^2+2\Re(z(\ov{-w}))=|z|^2+|w|^2-2\Re(z\ov w)$
        and summing this together with the first identity gives us the parallelogram law.
    \end{enumerate}
\end{ptcbr}

\begin{Ej}[1.5 Stein \& Shakarchi]
    A set $\Om$ is said to be pathwise connected if any two points in $\Om$ can be joined by a (piecewise-smooth) curve entirely contained in $\Om$. The purpose of this exercise is to prove that an open set $\Om$ is pathwise connected if and only if $\Om$ is connected.
    \begin{enumerate}[i)]
        \itemsep=-0.4em
        \item Suppose first that $\Om$ is open and pathwise connected, and that it can be written as $\Om$ = $\Om_1\cup\Om_2$ where $\Om_1$ and $\Om_2$ are disjoint non-empty open sets. Choose two points $w_1\in\Om_1$ and $w_2\in\Om_2$ and let $\ga$ denote a curve in $\Om$ joining $w_1$ to $w_2$. Consider a parametrization $z:\bonj{0,1}\to\Om$ of this curve with $z(0) = w_1$ and $z(1) = w_2$, and let
        $$t_\ast = \sup_{0\leq t\leq 1}\set{t\:\forall s [(0\leq s<t)\To (z(s)\in\Om_1)]}.$$
        Arrive at a contradiction by considering the point $z(t_\ast)$.
        \item Conversely, suppose that $\Om$ is open and connected. Fix a point $w\in\Om$ and let $\Om_1\subseteq\Om$ denote the set of all points that can be joined to $w$ by a curve contained in $\Om$. Also, let $\Om_2\subseteq\Om$ denote the set of all points that cannot be joined to $w$ by a curve in $\Om$. Prove that both $\Om_1$ and $\Om_2$ are open, disjoint and their union is $\Om$. Finally, since $\Om_1$ is non-empty (why?) conclude that $\Om$ = $\Om$1 as desired.
    \end{enumerate}
    \end{Ej}

\begin{ptcbr}
    \begin{enumerate}[i)]
        \itemsep=-0.4em
        \item Recall first, that by definition of supremum we have that if $S$ is our set, then 
        $$\exists s\in S(s>t_\ast-\eps)$$
        for $\eps>0$. Following the idea, we consider the point $z(t_\ast)$. We have two options to place $z(t_\ast)$, either in $\Om_1$ or $\Om_2$.\par 
        Let's start by definition of supremum
    \end{enumerate}
\end{ptcbr}

\begin{Ej}[1.7 Stein \& Shakarchi]
    The family of mappings introduced here plays an important role in complex analysis. These mappings, sometimes called \textbf{Blaschke factors}, will reappear in various applications in later chapters.
    \begin{enumerate}[i)]
        \itemsep=-0.4em
        \item Let $z,w\in\bC$ such that $\ov{z}w\neq 1$. Prove that 
        $$\left|\frac{w-z}{1-\ov w z}\right|<1$$
        if $|z|<1$ and $|w|<1$, and also that 
        $$\left|\frac{w-z}{1-\ov w z}\right|=1$$
        if $|z|=1$ or $|w|=1$. \hint{Why can one assume that $z$ is real? I then suffices to prove that $(r-w)(r-\ov w)\leq (1-rw)(1-r\ov w)$ with equality for appropriate $r$ and $|w|$.}\aside{Here is an alternate approach, which you may use if you like. Fix $w\in\bC$ with $w<1$, and consider the function $z\mapsto \frac{w-z}{1-\ov w z}$. What is $\ov{f(z)}$? By computing $f(z)\ov{f(z)}$, show that $|z|=1$ implies $|f(z)|=1$. Find a point $z$ with $|z|<1$ such that $|f(z)|<1$. Since $f$ is continuous, this shows that $f$ takes the unit disc to itself. (Why?)}
        \item Prove that for a fixed $w\in\bD$, the mapping $F\:z\mapsto\frac{w-z}{1-\ov w z}$ satisfies the following:
        \begin{enumerate}[a)]
            \itemsep=-0.4em
            \item $F$ maps the unit disc to itself (that is, $F:\bD\to\bD$), and is holomorphic.
            \item $F$ interchanges $0$ and $w$. 
            \item $|F(z)|=1$ if $|z|=1$.
            \item $F$ is bijective. \hint{Calculate $F\circ F$.}
        \end{enumerate}
    \end{enumerate}
    \end{Ej}

    \begin{ptcbr}
        \begin{enumerate}[i)]
            \itemsep=-0.4em
            \item The inequality in question is equivalent to 
            $$0\leq|w-z|<|1-\ov wz|.$$
            Since the quantities are positive, we can square them and preserve the order. It holds that 
            $$0\leq|w-z|^2<|1-\ov wz|^2\iff 0\leq (w-z)\ov{(w-z)}<(1-\ov wz)\ov{(1-\ov wz)},$$
            Simplifying this expression we get 
            \begin{align*}
                &(w-z)(\ov w-\ov z)<(1-\ov wz)(1-w\ov z)\\
                \iff&
            \end{align*}
        \end{enumerate}
    \end{ptcbr}
\section{Day 1| 20230120}

\subsection{The Complex Numbers}

To construct the complex numbers we take the real numbers, adjoin a variable and mod out by $\genr{x^2+1}$. We can also define $\bC$ as $\set{a+bi:\ a,b\in\bR}$ with the property $i^2=-1$. This means that we can multiply complex numbers in the following way:
$$(a+bi)(c+di)=ac+(bc+ad)i+bdi^2=(ac-bd)+(ad+bc)i.$$
Also as $x^2+1$ is irreducible in $\bR[x]$, $\bC$ is a finite field extension of $\bR$ of degree 2. As a 2-dimensional vector space $\set{1,i}$ is a basis for $\bC$.\par 
The map $a+bi\mapsto\twobyone{a}{b}$ is not a ring homomorphism, it's a bijection with a bit of structure. The map $z\mapsto \al z$, when $\al=a+bi$, is a linear map with the following action over the basis 
\begin{align*}
    &\al\. 1=\al\To[\al]\twobyone{1}{0}=\twobyone{a}{b}\\
    &\al\. i=-b+ai\To[\al]\twobyone{0}{1}=\twobyone{-b}{a}
\end{align*}
which means that $[\al]=\twobytwo{a}{-b}{b}{a}$. The converse, if we have a $\bR$-linear transformation, then it's $\bC$-linear if and only if it looks like $\twobytwo{a}{-b}{b}{a}$.

\begin{Def}
The \term{complex conjugation} map is $a+bi\mapsto a-bi$, or $z\mapsto\ov z$.
\end{Def}

This map is $\bR$-linear but not $\bC$-linear. 

\begin{Ex}
For $\al=a+bi$, we have 
$$\ov{2\al}=\ov{2(a+bi)}=\ov{2a+2bi}=2a-2bi=2\ov{al}.$$
Whereas if instead 
$$\ov{i\al}=\ov{ai-b}=-b-ai\neq i\ov{\al}=b+ai.$$
\end{Ex}

As a $\bR$-linear map, we can identify with the matrix $\twobytwo{1}{0}{0}{-1}$. By looking at the shape of this matrix we can see that it is not $\bC$-linear.

\begin{Lem}
The map $z\mapsto\ov z$ is a ring homomorphism
\end{Lem}

\begin{ptcbp}
$\ov{z+w}=\ov z+\ov w$ and $\ov{zw}=\ov z\ov w$.
\end{ptcbp}

With the complex conjugation we can pick out the real and imaginary parts of $\al=a+bi$. 
$$\al+\ov\al=2\Re(\al),\quad \al-\ov\al=2i\Im(\al)$$
\subsubsection{A Notion of Size}
Can't do geometry without one. Notice that for $z=a+bi$
$$z\ov z=a^2+b^2>0.$$
From a complex number we have extracted a positive quantity.

\begin{Def}
    The \term{complex modulus} of $z$ is $|z|=\sqrt{z\ov z}$.
\end{Def}

The fact that every number has $n$ roots is very important in complex analysis.\par 
As a vector in the plane, the norm of $z$ is $|z|$
\begin{center}
    INC FIG
\end{center}
This means that $a+bi\mapsto\twobyone{a}{b}$ is an isometry. In this sense the distance between two complex numbers is $d(z,w)=|z-w|$.

\subsubsection{Polar Coordinates (\emph{ad hoc})}

For $\te\in\bR$, define 
$$\exp(i\te)=e^{i\te}=\cos(\te)+i\sin(\te)\To |\exp(i\te)|=\sqrt{\cos^2(\te)+\sin^2(\te)}=1.$$
Every point in the unit circle is of the form $e^{i\te}$ and vice-versa.
\begin{center}
    INC FIG
\end{center}
For non-zero complex numbers, $z=|z|e^{i\te}$ for some $\te$.

\begin{Def}
    For a complex number $z=re^{i\te}$, an \term{argument} of $z$ is $\te$. 
\end{Def}
To have a well defined function, we mod out by multiples of $2\pi$: $$\arg:\bC\less\set{0}\to\quot{\bR}{2\pi\bZ},$$
and we obtain a group isomorphism. In general, ``lengths multiply, angles add.''\par 
For inverses if $z=re^{i\te}$, then $\frac{1}{z}=\frac1re^{-i\te}$.

\begin{Def}
    The \term{upper-half plane} is $\bH=\set{\Im(z)>0}$.
\end{Def}

\begin{Lem}
    If $H$ is a half plane $\Im(z-\bt/\ga)>0$
\end{Lem}
\end{document}
