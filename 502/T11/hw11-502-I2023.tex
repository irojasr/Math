\documentclass[12pt]{memoir}

\def\nsemestre {I}
\def\nterm {Spring}
\def\nyear {2023}
\def\nprofesor {Maria Gillespie}
\def\nsigla {MATH502}
\def\nsiglahead {Combinatorics 2}
\def\nextra {HW11}
\def\nlang {ENG}
\input{../../headerVarillyDiff}
\usepackage[enableskew]{youngtab}
\usepackage{ytableau}
\DeclareMathOperator{\SYT}{SYT}
\DeclareMathOperator{\inv}{inv}
\DeclareMathOperator{\maj}{maj}
\begin{document}

\begin{Ej}[Exercise 3]
Prove that in the lattice of flats of a matroid, if $X,Y$ are flats, then $X\lor Y=\cl(X\cup Y)$ and $X\land Y=X\cap Y$. You may use any of the matroid axioms.
\end{Ej}

We will use the following lemma.
\begin{Lem}
For a set $X$, 
$$\cl(X)=\bigcap_{\substack{F\ \text{is flat}\\ F\supseteq X}} F.$$
\end{Lem}

\begin{ptcbp}
If $F$ is flat and $F\supseteq X$ then $\cl(X)\subseteq\cl(F)=F$ because $F$ is already closed. This is true for all flats which contain $X$ so $\cl(X)\subseteq\bigcap F$.
On the other hand $\cl(X)$ is a flat which contains $X$. Now $\bigcap F$ is the intersection of \emph{all} the flat sets which contain $X$, in particular, $\cl(X)$. This means that $\bigcap F\subseteq \cl(X)$ and so they are equal.
\end{ptcbp}

The importance of this lemma is that it gives meaning to the phrase, \emph{the closure of a set is the smallest flat which contains it.}
%1.7.3
%https://math.stackexchange.com/questions/53523/why-is-the-intersection-of-two-flats-a-flat
\begin{ptcbr}
    Observe that 
    $$X,Y\subseteq X\cup Y\subseteq\cl(X\cup Y)$$
    which means $\cl(X\cup Y)$ is an upper bound for $X,Y$. Thus $X\lor Y\subseteq \cl(X\cup Y)$.\par 
    \footnote{\textbf{Ian} just entered the house and I wanted to discuss the problem with him. After a brief moment of thinking we got it.}On the other hand, 
    $$X,Y\subseteq X\lor Y\To X\cup Y\subseteq X\lor Y.$$
    As $X\lor Y$ is a flat and $\cl(X\cup Y)$ is the smallest flat which contains $X\cup Y$, it must occur that 
    $$\cl(X\cup Y)\subseteq X\lor Y\word{and so}\cl(X\cup Y)=X\lor Y.$$
    We will now prove that $X\cap Y$ is flat, observe that 
    $$X\cap Y\subseteq \cl(X\cap Y)\subseteq\cl(X),\cl(Y)=X,Y$$
    where the middle containment follows from the subset property of closure. Now $\cl(X\cap Y)$ being a subset of both $X,Y$ is also a subset of $X\cap Y$. So we conclude that $X\cap Y=\cl(X\cap Y)$ and so $X\cap Y$ is flat.\par 
    Finally we have 
    $$X\land Y\subseteq X,Y\To X\land Y\subseteq X\cap Y,$$
    and the other direction follows from greatest lower bound property. As $X\cap Y$ is a flat contained in $X,Y$ then it is smaller than the greatest lower bound of $X,Y$ which is $X\land Y$. In this context, \emph{smaller} means \emph{contained in}. Therefore $X\cap Y=X\land Y$.
\end{ptcbr}

\begin{Ej}[Exercise 5]
   Suppose $\set{x,y},\set{y,z}$ are circuits of a matroid and none of $x,y,z$ are loops. Show that $\set{x,z}$ is a circuit.
\end{Ej}

\begin{ptcbr}
    Applying the circuit exchange axiom to $\set{x,y}$ and $\set{y,z}$ we get a circuit $C$ such that 
    $$C\subseteq(\set{x,y}\cup\set{y,z})\less\set{y}=\set{x,z}.$$
    Out of the subsets of $\set{x,z}$, none of $\emptyset,\set{x},\set{z}$ are circuits, so $C=\set{x,z}$ and therefore $\set{x,z}$ is a circuit.
\end{ptcbr}

\begin{Ej}[Exercise 6]
    Given a matroid $M=(E,\cB)$ in terms of basis axioms, define the associated simple matroid $\ov M$ by:
    \begin{itemize}
        \itemsep=-0.4em
        \item Removing loops from $E$.
        \item Removing one element of each $2$-circuit from $E$ arbitrarily (i.e. choose one element from each
        \emph{parallel class} - equivalence class of elements in $2$-circuits with each other - to keep).
        \item Defining $\ov \cB$ to be the set of basis formed by replacing any element of a parallel class that was in a basis $B$ by the corresponding element in that parallel class.
    \end{itemize}
    Take the remaining edges to be $\ov E$. Define $\ov M=(\ov E,\ov \cB)$. Show that $\ov M$ is a simple matroid. \hint{Use the previous problem.}
\end{Ej}

\begin{ptcbr}
    We\footnote{In this problem we have collaborated with \textbf{Clare} and \textbf{Andrew}.} wish to show that the collection $\ov\cB$ satisfies the basis axioms. As the original $\cB$ was non-empty, there is a $B\in\cB$ which was a basis. It couldn't consist entirely of loops as it's a basis so when taking representatives we don't make $B$ empty. This basis makes our collection of bases non-empty.\par
    Regarding the other base axiom, suppose have $\ov B_1\neq \ov B_2$ in our collection $\ov\cB$ with two particular elements $x\in \ov B_1\less \ov B_2$ and $y\in \ov B_2\less \ov B_1$. We wish to show that 
    $$\ov B_1\less\set{x}\cup\set{y}\in\ov{\cB}.$$
    Observe that $\ov B_1$ and $\ov B_2$ were created from bases $B_1,B_2\in\cB$ and $x,y$ are the chosen representatives of their parallel class. In particular $x,y$ still satisfy the desired condition in the original matroid. As the original matroid \emph{is a matroid} we can say that 
    $$B_1\less\set{x}\cup\set{y}\in\cB.$$
    Taking the quotient by the parallel relation we can find that it corresponds to $\ov B_1\less\set{x}\cup\set{y}$ in $\cB$. 
\end{ptcbr}

\begin{Ej}[Exercise 7]
    Let $(E, \cI)$ be a matroid and let $w\: E\to\bR$ be any weight function.\par
    Show that applying the greedy algorithm to find a basis of minimal weight does indeed find a basis of minimal weight.
\end{Ej}

\begin{ptcbr}
    \footnote{As in the last homework, I received invaluable assistance from \textbf{Oxley}'s book.}We begin by recalling Kruskal's algorithm for graphs. From this we deduce what the algorithm for matroids must be.\par 
    Given a graph $G$, Kruskal's algorithm produces a minimum weight spanning tree $T$ by doing the following:
    \begin{enumerate}[i)]
        \itemsep=-0.4em
        \item Begin by choosing an edge of minimum weight.
        \item The next edge to be chosen must fulfill these condition:
        \begin{itemize}
            \itemsep=-0.4em 
            \item It must not have been chosen previously.
            \item It doesn't form a cycle with the previously chosen edges.
        \end{itemize}
        Among the edges that fulfill this condition, we pick one of minimal weight and repeat the procedure.
    \end{enumerate}
    With this we may elaborate a similar algorithm, but instead of guaranteeing that edges don't form a cycle, we want edges to not form a circuit instead. In the same fashion we are still adding edges of minimal weight to our potential basis. The algorithm in this case is as follows:
    \begin{enumerate}[i)]
        \itemsep=-0.4em
        \item We begin by choosing an edge of minimal weight. If the set is empty we are done.
        \item The next edge must satisfy the following:
        \begin{itemize}
            \itemsep=-0.4em 
            \item It must not have been chosen previously.
            \item It doesn't form a \emph{circuit} with the previously chosen edges.
        \end{itemize}
        Among the edges that fulfill this condition, we pick one of minimal weight and repeat the procedure until we arrive to a maximal independent set.
    \end{enumerate}
    Suppose that the algorithm outputs a basis $B=\set{e_1,\dots,e_r}$ with $w(e_1)\leq\dots\leq w(e_r)$. If $B'=\set{f_1,\dots,f_r}$ is another basis, we may reorder the elements so that $w(f_1)\leq\dots\leq w(f_r)$. To show $B$ is of minimum weight it suffices to show that $w(e_k)\leq w(f_k)$ for all $k$.\par 
    Let us assume the contrary, not all of the $e_k$ weigh less than the $f_k$. Assume $k_0$ is the first index at which this occurs. This means that 
    $$w(e_{k_0})>w(f_{k_0})$$
    so let us retrace our steps with the algorithm. At step $k_0$ we have our partially built basis $B$:
    $$B_{\text{inc}}=\set{e_1,e_2,\dots,e_{k_0-1}}.$$
    Consider also the partial base of $B'$, $B'_{\text{inc}}=\set{f_1,\dots,f_{k_0}}$.\par 
    At this moment we have two independent sets $B_{\text{inc}}$ and $B_{\text{inc}}'$ and $|B_{\text{inc}}|\leq|B_{\text{inc}}'|$. Invoking the augmentation axiom we may find $f_\l\in B_{\text{inc}}'\less B_{\text{inc}}$ such that $B_{\text{inc}}\cup\set{f_\l}$ is independent. However, it must occur that 
    $$w(f_\l)\leq w(f_{k_0})<w(e_{k_0})\To w(f_\l)<w(e_{k_0})$$
    first because all the elements in $B_{\text{inc}}'$ have lower weight than $f_{k_0}$ and second because of what we assumed. But this contradicts the algorithm's process.\par 
    At this step, the algorithm should've chosen $f_\l$ instead of $e_{k_0}$. But it did not, it picked $e_{k_0}$. This means our assumption was wrong, and thus no $f_k$ weighs less than an $e_k$. In conclusion the algorithm produces a basis of minimum weight. 
\end{ptcbr}
\end{document}