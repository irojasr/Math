\documentclass[12pt]{memoir}

\def\nsemestre {I}
\def\nterm {Primer Semestre}
\def\nyear {2022}
\def\nprofesor {Roberto Ulloa}
\def\nsigla {OPT742}
\def\nsiglahead {Geometr\'ia Algebraica}
\def\nextra {C1-GW}
\input{../headerVarillyDiff}

\begin{document}
\begin{multicols}{2}

\begin{Ej}
  Sea $\emptyset\neq E\subsetneq\cC(\bR)$, muestre que
  $$I_E=\set{f\in\cC(\bR)\mid\forall x\in E(f(x)=0)}$$
  no es un ideal principal.
\end{Ej}

\begin{ptcbp}
Suponga a manera de contradicción que $I_E=\gen(g)$ para $g\in\cC(\bR)$. Luego toda función en $I_E$ se puede escribir como $f=gh$ y $g(x)=0$ para $x\in E$. Suponga adicionalmente que $E$ no es denso. De esta manera hay un abierto $U$ tal que $U\cap E =\emptyset$. Dadas las condiciones $g$ se anula en la frontera de $U$.\par
Considere así la función
$$h=\sqrt{|g|}\ind(U)+g\ind(U^c)$$
que es positiva en $U$ y además $h\in I_E$. Así $h=fg$ para alguna $f$ continua y de inmediato vemos que
$$f\upharpoonright_{U}=\frac{\sqrt{|g|}}{g}\sim\frac{1}{\sqrt{g}}\xrightarrow[x\to\del U]{}\infty.$$
Esto contradice la continuidad de $f$ por lo que nuestra suposición está errada. Así $I_E$ no es principal.
\end{ptcbp}

\end{multicols}
\end{document} 