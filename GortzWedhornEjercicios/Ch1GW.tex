\documentclass[12pt]{memoir}

\def\nsemestre {I}
\def\nterm {Primer Semestre}
\def\nyear {2022}
\def\nprofesor {Roberto Ulloa}
\def\nsigla {OPT742}
\def\nsiglahead {Geometr\'ia Algebraica}
\def\nextra {C1-GW}
\input{../headerVarillyDiff}

\begin{document}
%\begin{multicols}{2}

\begin{Ej}
  Sea $\emptyset\neq E\subsetneq\cC(\bR)$, muestre que
  $$I_E=\set{f\in\cC(\bR)\mid\forall x\in E(f(x)=0)}$$
  no es un ideal principal.
\end{Ej}

\begin{ptcbp}
Suponga a manera de contradicción que $I_E=\gen(g)$ para $g\in\cC(\bR)$. Luego toda función en $I_E$ se puede escribir como $f=gh$ y $g(x)=0$ para $x\in E$. Suponga adicionalmente que $E$ no es denso. De esta manera hay un abierto $U$ tal que $U\cap E =\emptyset$. Dadas las condiciones $g$ se anula en la frontera de $U$.\par
Considere así la función
$$h=\sqrt{|g|}\ind(U)+g\ind(U^c)$$
que es positiva en $U$ y además $h\in I_E$. Así $h=fg$ para alguna $f$ continua y de inmediato vemos que
$$f\upharpoonright_{U}=\frac{\sqrt{|g|}}{g}\sim\frac{1}{\sqrt{g}}\xrightarrow[x\to\del U]{}\infty.$$
Esto contradice la continuidad de $f$ por lo que nuestra suposición está errada. Así $I_E$ no es principal.
\end{ptcbp}

\begin{ptcb}
To determine whether the series $\sum_{n=2}^{\infty} \log(1+(-1)^n/\sqrt{n})$ is convergent or divergent, we can use the alternating series test.

First, note that the terms of the series alternate in sign, since each term is multiplied by $(-1)^n$.

Next, consider the absolute value of the terms:

$|\log(1+(-1)^n/\sqrt{n})|$

$= \log(1+(-1)^n/\sqrt{n})$

Since $\log(x)$ is an increasing function for $x > 1$, we can say that:

$\log(1+(-1)^n/\sqrt{n}) \leq \log(1+1/\sqrt{n})$

Now we can apply the limit comparison test by comparing our series with the series $1/\sqrt{n}$, which is a $p$-series with $p=1/2$.

By taking the limit of the ratio of the two series, we get:

$\lim_{n \to \infty} \frac{\log(1+(-1)^n/\sqrt{n})}{1/\sqrt{n}}$

$= \lim_{n \to \infty} \sqrt{n} \log(1+(-1)^n/\sqrt{n})$

Using L'Hopital's rule on this limit, we can get:

$\lim_{n \to \infty} \frac{\log(1+(-1)^n/\sqrt{n})}{1/(2\sqrt{n})}$

$= \lim_{n \to \infty} 2\sqrt{n} \log(1+(-1)^n/\sqrt{n})$

Since the limit of $2\sqrt{n}$ as $n$ approaches infinity is infinity, the limit of our ratio is 0, meaning that our series converges by the limit comparison test.

Therefore, the series $\sum_{n=2}^{\infty} \log(1+(-1)^n/\sqrt{n})$ is convergent.

And for the second part of your question:

To compute the limit:

$\lim_{n \to \infty} \left| \frac{\log(1+(-1)^n/\sqrt{n})}{(-1)^n/\sqrt{n}} \right|$

we can simplify the expression first:

$\left|\frac{\log(1+(-1)^n/\sqrt{n})}{(-1)^n/\sqrt{n}}\right|$

$= \left|\log(1+(-1)^n/\sqrt{n}) \cdot (-1)^n \sqrt{n}\right|$

$= \log(1+(-1)^n/\sqrt{n}) \cdot \sqrt{n}$

Now, since the expression inside the absolute value is positive for all $n \geq 2$, we can drop the absolute value sign.

Taking the limit as $n$ approaches infinity, we can apply L'Hopital's rule:

$\lim_{n \to \infty} \log(1+(-1)^n/\sqrt{n}) \cdot \sqrt{n}$

$= \lim_{n \to \infty} \frac{1}{2n(1+(-1)^n/\sqrt{n})}$

Since the denominator goes to infinity as $n$ approaches infinity, and the numerator is bounded, the limit is 0.

Therefore, the limit of the expression is 0 as $n$ approaches infinity.

$\left|\frac{\log(1+(-1)^n/\sqrt{n})}{(-1)^n/\sqrt{n}}\right|$

$= \left|\log(1+(-1)^n/\sqrt{n}) \cdot (-1)^n \sqrt{n}\right|$

$= \log(1+(-1)^n/\sqrt{n}) \cdot \left|\sqrt{n}\right|$.

Then we can proceed with taking the limit as $n$ approaches infinity using L'Hopital's rule, and the final limit of the expression is still 0.
\end{ptcb}
%\end{multicols}
\end{document} 