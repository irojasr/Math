%----------------------------------------------------------------------------------------
%	PACKAGES AND OTHER DOCUMENT CONFIGURATIONS
%----------------------------------------------------------------------------------------

\documentclass[12pt]{article}
\usepackage[spanish]{babel} %Tildes
\usepackage[extreme]{savetrees} %Espaciado e interlineado. Comentar si no gusta el interlineado.
\usepackage[utf8]{inputenc} %Encoding para tildes
\usepackage[breakable,skins]{tcolorbox} %Cajitas
\usepackage{fancyhdr} % Se necesita para el título arriba
\usepackage{lastpage} % Se necesita para poner el número de página
\usepackage{amsmath,amsfonts,amssymb,amsthm} %simbolos y demás
\usepackage{mathabx} %más símbolos
\usepackage{physics} %simbolos de derivadas, bra-ket.
\usepackage{multicol}
\usepackage[customcolors]{hf-tikz}
\usepackage[shortlabels]{enumitem}
\usepackage{tikz}

%\def\darktheme
%%%%%%%%% === Document Configuration === %%%%%%%%%%%%%%

\pagestyle{fancy}
\setlength{\headheight}{14.49998pt} %NO MODIFICAR
\setlength{\footskip}{14.49998pt} %NO MODIFICAR

\ifx \darktheme\undefined

\lhead{MA1004G8} % Nombre de autor
\chead{\textbf{Lección 0623}} % Titulo
\rhead{}%\firstxmark} 
\lfoot{}%\lastxmark}
\cfoot{}
\rfoot{P\'ag.\ \thepage\ de\ \pageref{LastPage}} %A la derecha saldrá pág. 6 de 9. 
\else
\pagenumbering{gobble}
\pagecolor[rgb]{0,0,0}%{0.23,0.258,0.321}
\color[rgb]{1,1,1}
\fi

%%%%%%%%% === My T Color Box === %%%%%%%%%%%%%%

\ifx \darktheme\undefined
\newtcolorbox{ptcb}{
colframe = black,
colback = white,
breakable,
enhanced
}
\newtcolorbox{ptcbP}{
colframe = black,
colback = white,
coltitle = black,
colbacktitle = black!40,
title = Práctica,
breakable,
enhanced
}

\else
\newtcolorbox{ptcb}{
colframe = white,
colback = black,
colupper = white,
breakable,
enhanced
}
\newtcolorbox{ptcbP}{
colframe = white,
colback = black,
colupper = white,
coltitle = white,
colbacktitle = black,
title = Práctica,
breakable,
enhanced
}
\fi

%%%%%%%%% === Tikz para matrices === %%%%%%%%%%%%%%

\tikzset{
  style green/.style={
    set fill color=green!50!lime!60,
    set border color=white,
  },
  style cyan/.style={
    set fill color=cyan!90!blue!60,
    set border color=white,
  },
  style orange/.style={
    set fill color=orange!80!red!60,
    set border color=white,
  },
  row/.style={
    above left offset={-0.15,0.31},
    below right offset={0.15,-0.125},
    #1
  },
  col/.style={
    above left offset={-0.1,0.3},
    below right offset={0.15,-0.15},
    #1
  }
}

%%%%%%%%% === Theorems and suchlike === %%%%%%%%%%%%%%

\theoremstyle{plain}
\newtheorem{Th}{Teorema}  %%% Theorem 1.1
\newtheorem*{nTh}{Teorema}             %%% No-numbered Theorem
\newtheorem{Prop}[Th]{Proposición}     %%% Proposition 1.2
\newtheorem{Lem}[Th]{Lema}             %%% Lemma 1.3
\newtheorem*{nLem}{Lema}               %%% No-numbered Lemma
\newtheorem{Cor}[Th]{Corolario}        %%% Corollary 1.4
\newtheorem*{nCor}{Corolario}          %%% No-numbered Corollary

\theoremstyle{definition}
\newtheorem*{Def}{Definición}       %%% Definition 1.5
\newtheorem*{nonum-Def}{Definición}    %%% No number Definition
\newtheorem*{nEx}{Ejemplo}             %%% No number Example
\newtheorem{Ex}[Th]{Ejemplo}           %%% Example
\newtheorem{Ej}[Th]{Ejercicio}         %%% Exercise
\newtheorem*{nEj}{Ejercicio}           %%% No number Excercise
\newtheorem*{Not}{Notación}       %%% Definition 1.5

\theoremstyle{remark}
\newtheorem*{Rmk}{Observación}      %%%Remark 1.6

%\numberwithin{equation}{section}

\setlength{\parindent}{3ex}

%%====== Useful macros: =======%%%

\DeclareMathOperator{\gen}{gen}     %%%set generated by...
\DeclareMathOperator{\Rng}{Rng}     %%%rangomat
\DeclareMathOperator{\Nul}{Nul}     %%%rangomat
\DeclareMathOperator{\Proy}{Proy}   %%%proyección
\DeclareMathOperator{\id}{id}       %%%identity operator

\newcommand{\la}{\lambda}           %%%short for \lambda
\newcommand{\sg}{\sigma}            %%%short for \sigma
\newcommand{\te}{\theta}                %% short for  \theta
\renewcommand{\l}{\ell}

\newcommand{\thickhat}[1]{\mathbf{\hat{\text{$#1$}}}}
\newcommand{\ii}{\vu{\imath}}
\newcommand{\jj}{\vu{\jmath}}
\newcommand{\kk}{\thickhat{k}}

\newcommand{\bC}{\mathbb{C}}        %%%complex numbers
\newcommand{\bN}{\mathbb{N}}        %%%natural numbers
\newcommand{\bP}{\mathbb{P}}        %%%polynomials
\newcommand{\bR}{\mathbb{R}}        %%%real numbers
\newcommand{\bZ}{\mathbb{Z}}        %%%integer numbers
\newcommand{\cB}{\mathcal{B}}       %%%basis
\newcommand{\cC}{\mathcal{C}}       %%%basis
\newcommand{\cM}{\mathcal{M}}       %%%matrix family

\newcommand{\sT}{\mathsf{T}}        %%%traspuesta

\renewcommand{\geq}{\geqslant}      %%%(to save typing)
\renewcommand{\leq}{\leqslant}      %%%(to save typing)
\newcommand{\x}{\times}             %%%product
\renewcommand{\:}{\colon}           %%%colon in  f: A -> B
\newcommand{\isom}{\simeq}              %% isomorfismo

\newcommand{\un}[1]{\underline{#1}}
\newcommand{\half}{\frac12}

\newcommand*{\Cdot}{{\raisebox{-0.25ex}{\scalebox{1.5}{$\cdot$}}}}      %% cdot más grande
\renewcommand{\.}{\Cdot}                %% producto escalar

\newcommand{\twobyone}[2]{\begin{pmatrix} %% 2 x 1 matrix
  #1 \\ #2 \end{pmatrix}}
  \newcommand{\twobytwo}[4]{\begin{pmatrix} %% 2 x 2 matrix
    #1 & #2 \\ #3 & #4 \end{pmatrix}}
    \newcommand{\twobythree}[6]{\begin{pmatrix} %% 2 x 3 matrix
        #1 & #2 & #3\\ #4 & #5 & #6 \end{pmatrix}}
\newcommand{\threebyone}[3]{\begin{pmatrix} %% 3 x 1 matrix
  #1 \\ #2 \\ #3 \end{pmatrix}}
  \newcommand{\threebytwo}[6]{\begin{pmatrix} %% 3 x 1 matrix
    #1 & #2\\ #3 & #4\\ #5&#6 \end{pmatrix}}
\newcommand{\threebythree}[9]{\begin{pmatrix} %% 3 x 3 matrix
  #1 & #2 & #3 \\ #4 & #5 & #6 \\ #7 & #8 & #9 \end{pmatrix}}

\newcommand{\To}{\Rightarrow}

\newcommand{\vaf}{\overrightarrow}

\newcommand{\set}[1]{\{\,#1\,\}}    %% set notation
\newcommand{\Set}[1]{\biggl\{\,#1\,\biggr\}} %% set notation (large)
\newcommand{\red}[1]{\textcolor{red}{#1}}
\newcommand{\blu}[1]{\textcolor{blue}{#1}}

%----------------------------------------------------------------------------------------
%	ARTICLE CONTENTS
%----------------------------------------------------------------------------------------

\begin{document}

\begin{multicols}{2}

\subsubsection*{Diagonalización}

Al principio del curso mencionamos que trabajar con matrices diagonales era de lo más sencillo. Con el proceso de diagonalización vamos a tomar una matriz y la despedazaremos para ver qué es lo que hace que funcione.\par 
Supongamos que $T$ es una T.L. con $n$ autovectores $\cB=\set{\vec v_1,\dots,\vec v_n}$. Si forman una base podemos representar $T$ en $\cB$ como una matriz:
$$[T]_\cB=\begin{pmatrix}
  [T\vec v_1]_\cB&[T\vec v_2]_\cB&\cdots&[T\vec v_n]_\cB
\end{pmatrix}.$$
Esto que tenemos aquí en palabras es: \emph{la representación en $T$ en $\cB$ es la matriz cuyas columnas son los $T\vec{v}_i$ escritos en base $\cB$}. Adicionalmente recordemos que en $\cB$ cada $\vec{v}_i$ se escribe como
\begin{gather*}
  \vec{v}_i=0\.\vec{v}_1+0\.\vec{v}_2+\dots+1\.\vec{v}_1+\dots+0\.\vec{v}_n\\
  \To[\vec{v}_i]_\cB=(0,0,\dots,1,\dots,0).
\end{gather*}
Ahora como $\vec{v}_i$ es un autovector de $T$, entonces $T\vec{v_i}=\la_i\vec{v}_i$. Pero escribiendo eso en coordenadas de $\cB$ obtenemos
$$[T\vec{v}_i]_\cB=\la_i(0,0,\dots,1,\dots,0)=(0,0,\dots,\la_i,\dots,0).$$
Este vector es la columna $i$ de $[T]_\cB$, entonces repitiendo el proceso para todos los demás autovectores obtenemos la representación en base $\cB$ de $T$ como
$$[T]_\cB=\threebythree{\la_1}{\cdots}{0}{\vdots}{\ddots}{\vdots}{0}{\cdots}{\la_n}=\text{diag}(\la_1,\la_2,\dots,\la_n).$$
En resumen podemos ver que diagonalizar es un proceso de cambio de base a final de cuentas. Sin embargo, hay una condición que asumimos \textbf{que los autovectores de $T$ forman una base}.

\begin{Th}
Una T.L. $T$ se puede representar en \un{forma diagonal} cuando alguna de las siguientes condiciones vale:
\vspace*{-0.5em}
\begin{enumerate}[i)]
  \itemsep=-0.4em
  \item Todo autovalor de $T$ cumple \un{$m.a.=m.g.$}
  \item $\dim(E_{\la_1})+\dots+\dim(E_{\la_n})=\dim\bR^n=n$. (La suma de las dimensiones de los espacios invariantes es la dimensión de todo el espacio.)
  \item Los autovectores de $T$ \un{forman una base} de $\bR^n$.
\end{enumerate}
\end{Th}

\begin{Ex} 
Tenemos $C=\twobytwo{1}{3}{0}{1}$, un cizallamiento. 
\begin{itemize}
  \itemsep=-0.5em
  \item $C$ triangular $\To$ $\la=1$ con m.a.$=2$.
  \item Tiene un único autovector $\ii$, entonces m.g.$=1$.
  \item Como m.a. $\neq$ m.g. entonces $C$ no es diagonalizable.
\end{itemize}
\end{Ex}
%https://ece.uwaterloo.ca/~dwharder/Integer_eigenvalues/

\begin{Ex}
  Consideremos $A=\twobytwo{2}{1}{1}{2}$. Buscamos sus autovectores con el fin de diagonalizar.
  \begin{enumerate}
    \itemsep=-0.5em
    \item Utilizando la fórmula de autovalores $[2\x 2]$ tenemos que 
    $$\blu{m}=\frac{1}{2}(\red{2}+\red{2}),\ \blu{p}=\det A=3\To \la=\blu{2}\pm\sqrt{\blu{2}^2-\blu{3}},$$
    por lo que $\la_1=3$ y $\la_2=1$.
    \item El primer espacio invariante es 
    $$E_{\la_1}=\ker(A-3I)=\ker\twobytwo{-1}{1}{1}{-1}=\ker\twobytwo{1}{-1}{0}{0}.$$
    La ecuación asociada a esa matriz es $x-y=0$ y su solución es 
    $$(x,y)=(x,x)=x(1,1)\To E_{\la_1}=\gen(1,1).$$
    Concluimos que $\vec{v}_1=(1,1)$ es un autovector de $A$ asociado a $\la_1=3$.
    \item El segundo espacio invariante es 
    $$E_{\la_2}=\ker(A-I)=\ker\twobytwo{1}{1}{1}{1}=\ker\twobytwo{1}{1}{0}{0}.$$
    La ecuación asociada es $x+y=0$. Su solución es 
    $$(x,y)=(x,-x)=x(1,-1)\To E_{\la_2}=\gen(1,-1)$$
    y así el otro autovector es $\vec{v}_2=(1,-1)$ asociado a $\la_2=1$.
  \end{enumerate}
  Tenemos que los autovalores cumplen que m.a.$=$m.g. y así $A$ es diagonalizable.\par 
  Según lo mencionado antes, la forma diagonal de $A$ tiene a sus autovalores en la diagonal. La base en la cual es diagonal es $\cB=\set{\vec{v}_1,\vec{v}_2}$, con los autovectores anteriores.\par 
  Así $[\id]^\cB_\cC=P=\twobytwo{1}{1}{1}{-1}$ entonces la forma diagonal de $A$ se obtiene con 
  $$D=PAP^{-1}\To\twobytwo{3}{0}{0}{1}=\twobytwo{1}{1}{1}{-1}\twobytwo{2}{1}{1}{2}\twobytwo{1/2}{1/2}{1/2}{-1/2}.$$
\end{Ex}

Podemos intuir que toda matriz simétrica es diagonalizable. Pero de hecho el resultado es más poderoso todavía. Observemos que 
$$\braket{\vec{v}_1}{\vec{v}_2}=(1)(1)+(1)(-1)=0,$$
es decir la base $\cB$ es \emph{ortogonal}.

\begin{Th}
  Toda matriz simétrica es diagonalizable y sus autovectores son ortogonales. En concreto diremos que es \un{ortogonalmente diagonalizable}.
\end{Th}
\begin{ptcbP}
Realice el mismo proceso para diagonalizar la matriz $A=\twobytwo{2}{-2}{-2}{5}$.
\end{ptcbP}
\end{multicols}
\end{document}