%----------------------------------------------------------------------------------------
%	PACKAGES AND OTHER DOCUMENT CONFIGURATIONS
%----------------------------------------------------------------------------------------

\documentclass[12pt]{article}
\usepackage[spanish]{babel} %Tildes
\usepackage[extreme]{savetrees} %Espaciado e interlineado. Comentar si no gusta el interlineado.
\usepackage[utf8]{inputenc} %Encoding para tildes
\usepackage[breakable,skins]{tcolorbox} %Cajitas
\usepackage{fancyhdr} % Se necesita para el título arriba
\usepackage{lastpage} % Se necesita para poner el número de página
\usepackage{amsmath,amsfonts,amssymb,amsthm} %simbolos y demás
\usepackage{mathabx} %más símbolos
\usepackage{physics} %simbolos de derivadas, bra-ket.
\usepackage{multicol}
\usepackage[customcolors]{hf-tikz}
\usepackage[shortlabels]{enumitem}
\usepackage{tikz}

\def\darktheme
%%%%%%%%% === Document Configuration === %%%%%%%%%%%%%%

\pagestyle{fancy}
\setlength{\headheight}{14.49998pt} %NO MODIFICAR
\setlength{\footskip}{14.49998pt} %NO MODIFICAR

\ifx \darktheme\undefined

\lhead{MA1004G8} % Nombre de autor
\chead{\textbf{Lección 0630}} % Titulo
\rhead{}%\firstxmark} 
\lfoot{}%\lastxmark}
\cfoot{}
\rfoot{P\'ag.\ \thepage\ de\ \pageref{LastPage}} %A la derecha saldrá pág. 6 de 9. 
\else
\pagenumbering{gobble}
\pagecolor[rgb]{0,0,0}%{0.23,0.258,0.321}
\color[rgb]{1,1,1}
\fi

%%%%%%%%% === My T Color Box === %%%%%%%%%%%%%%

\ifx \darktheme\undefined
\newtcolorbox{ptcb}{
colframe = black,
colback = white,
breakable,
enhanced
}
\newtcolorbox{ptcbP}{
colframe = black,
colback = white,
coltitle = black,
colbacktitle = black!40,
title = Práctica,
breakable,
enhanced
}

\else
\newtcolorbox{ptcb}{
colframe = white,
colback = black,
colupper = white,
breakable,
enhanced
}
\newtcolorbox{ptcbP}{
colframe = white,
colback = black,
colupper = white,
coltitle = white,
colbacktitle = black,
title = Práctica,
breakable,
enhanced
}
\fi

%%%%%%%%% === Tikz para matrices === %%%%%%%%%%%%%%

\tikzset{
  style green/.style={
    set fill color=green!50!lime!60,
    set border color=white,
  },
  style cyan/.style={
    set fill color=cyan!90!blue!60,
    set border color=white,
  },
  style orange/.style={
    set fill color=orange!80!red!60,
    set border color=white,
  },
  row/.style={
    above left offset={-0.15,0.31},
    below right offset={0.15,-0.125},
    #1
  },
  col/.style={
    above left offset={-0.1,0.3},
    below right offset={0.15,-0.15},
    #1
  }
}

%%%%%%%%% === Theorems and suchlike === %%%%%%%%%%%%%%

\theoremstyle{plain}
\newtheorem{Th}{Teorema}  %%% Theorem 1.1
\newtheorem*{nTh}{Teorema}             %%% No-numbered Theorem
\newtheorem{Prop}[Th]{Proposición}     %%% Proposition 1.2
\newtheorem{Lem}[Th]{Lema}             %%% Lemma 1.3
\newtheorem*{nLem}{Lema}               %%% No-numbered Lemma
\newtheorem{Cor}[Th]{Corolario}        %%% Corollary 1.4
\newtheorem*{nCor}{Corolario}          %%% No-numbered Corollary

\theoremstyle{definition}
\newtheorem*{Def}{Definición}       %%% Definition 1.5
\newtheorem*{nonum-Def}{Definición}    %%% No number Definition
\newtheorem*{nEx}{Ejemplo}             %%% No number Example
\newtheorem{Ex}[Th]{Ejemplo}           %%% Example
\newtheorem{Ej}[Th]{Ejercicio}         %%% Exercise
\newtheorem*{nEj}{Ejercicio}           %%% No number Excercise
\newtheorem*{Not}{Notación}       %%% Definition 1.5

\theoremstyle{remark}
\newtheorem*{Rmk}{Observación}      %%%Remark 1.6

%\numberwithin{equation}{section}

\setlength{\parindent}{3ex}

%%====== Useful macros: =======%%%

\DeclareMathOperator{\gen}{gen}     %%%set generated by...
\DeclareMathOperator{\Rng}{Rng}     %%%rangomat
\DeclareMathOperator{\Nul}{Nul}     %%%rangomat
\DeclareMathOperator{\Proy}{Proy}   %%%proyección
\DeclareMathOperator{\id}{id}       %%%identity operator

\newcommand{\la}{\lambda}           %%%short for \lambda
\newcommand{\sg}{\sigma}            %%%short for \sigma
\newcommand{\te}{\theta}                %% short for  \theta
\renewcommand{\l}{\ell}

\newcommand{\thickhat}[1]{\mathbf{\hat{\text{$#1$}}}}
\newcommand{\ii}{\vu{\imath}}
\newcommand{\jj}{\vu{\jmath}}
\newcommand{\kk}{\thickhat{k}}

\newcommand{\bC}{\mathbb{C}}        %%%complex numbers
\newcommand{\bN}{\mathbb{N}}        %%%natural numbers
\newcommand{\bP}{\mathbb{P}}        %%%polynomials
\newcommand{\bR}{\mathbb{R}}        %%%real numbers
\newcommand{\bZ}{\mathbb{Z}}        %%%integer numbers
\newcommand{\cB}{\mathcal{B}}       %%%basis
\newcommand{\cC}{\mathcal{C}}       %%%basis
\newcommand{\cM}{\mathcal{M}}       %%%matrix family

\newcommand{\sT}{\mathsf{T}}        %%%traspuesta

\renewcommand{\geq}{\geqslant}      %%%(to save typing)
\renewcommand{\leq}{\leqslant}      %%%(to save typing)
\newcommand{\x}{\times}             %%%product
\renewcommand{\:}{\colon}           %%%colon in  f: A -> B
\newcommand{\isom}{\simeq}              %% isomorfismo

\newcommand{\un}[1]{\underline{#1}}
\newcommand{\half}{\frac12}

\newcommand*{\Cdot}{{\raisebox{-0.25ex}{\scalebox{1.5}{$\cdot$}}}}      %% cdot más grande
\renewcommand{\.}{\Cdot}                %% producto escalar

\newcommand{\twobyone}[2]{\begin{pmatrix} %% 2 x 1 matrix
  #1 \\ #2 \end{pmatrix}}
\newcommand{\twobytwo}[4]{\begin{pmatrix} %% 2 x 2 matrix
    #1 & #2 \\ #3 & #4 \end{pmatrix}}
\newcommand{\twobythree}[6]{\begin{pmatrix} %% 2 x 3 matrix
        #1 & #2 & #3\\ #4 & #5 & #6 \end{pmatrix}}
\newcommand{\threebyone}[3]{\begin{pmatrix} %% 3 x 1 matrix
  #1 \\ #2 \\ #3 \end{pmatrix}}
\newcommand{\threebytwo}[6]{\begin{pmatrix} %% 3 x 1 matrix
    #1 & #2\\ #3 & #4\\ #5&#6 \end{pmatrix}}
\newcommand{\threebythree}[9]{\begin{pmatrix} %% 3 x 3 matrix
  #1 & #2 & #3 \\ #4 & #5 & #6 \\ #7 & #8 & #9 \end{pmatrix}}

\newcommand{\To}{\Rightarrow}

\newcommand{\vaf}{\overrightarrow}

\newcommand{\set}[1]{\{\,#1\,\}}    %% set notation
\newcommand{\Set}[1]{\biggl\{\,#1\,\biggr\}} %% set notation (large)
\newcommand{\red}[1]{\textcolor{red}{#1}}
\newcommand{\blu}[1]{\textcolor{blue}{#1}}

%----------------------------------------------------------------------------------------
%	ARTICLE CONTENTS
%----------------------------------------------------------------------------------------

\begin{document}

\begin{multicols}{2}

\subsection*{Repaso Formas Cuadráticas}

Recordemos que toda matriz simétrica está asociada a una forma cuadrática y vice-versa por medio de la relación 
$$\vec{x}^\sT A\vec{x}=\braket{A\vec{x}}{\vec{x}}.$$
En una forma cuadrática, el término \un{mixto} ó \un{cruzado} es el término $xy$.
\begin{ptcbP}
  Consideremos la siguientes formas cuadráticas junto con las matrices a su lado. Asocie las formas cuadráticas con su matriz correspondiente.
  \begin{multicols}{2}
  \begin{itemize}
    %\itemsep=0.5em
    \item $x^2+2xy-3y^2$.
    \item $4y^2+6xy$.
    \item $9x^2+16y^2$.
    \item $7xy$.
  \end{itemize}
  \columnbreak
  \begin{itemize}
    \itemsep=-0.5em
    \item $\twobytwo{9}{0}{0}{16}$. %3
    \item $\twobytwo{1}{1}{1}{-3}$. %1
    \item $\twobytwo{0}{3}{3}{4}$. %2
    \item $\twobytwo{0}{7/2}{7/2}{0}$. %4
  \end{itemize}
\end{multicols}
\end{ptcbP}

\subsubsection*{Eliminación del Término Mixto}

\begin{Rmk}
De las formas cuadráticas anteriores sólo una no tenía término mixto. ¿Qué caracteriza a la matriz asociada a esa forma?
\end{Rmk}

Eliminemos el término mixto con un ejemplo.

\begin{Ex} 
Si $Q(x,y)=x^2+6xy+y^2$, entonces $Q=\vec{x}^\sT A\vec{x}$ con $A=\twobytwo{1}{3}{3}{1}$. Vamos a diagonalizar $A$:
\begin{enumerate}
  \itemsep=-0.5em 
  \item Los autovalores de $A$ se obtienen con 
  $$\la=m\pm\sqrt{m^2-p},\ m=(a+d)/2,\ p=\det A.$$
  En este caso $\la=1\pm\sqrt{1^2-(-8)}$. Obtenemos $\la_1=4$ y $\la_2=-2$.
  \item El espacio invariante asociado a $\la_1$ es 
  $$E_{\la_1}=\ker(A-4I)=\ker\twobytwo{-3}{3}{3}{-3}=\ker\twobytwo{1}{-1}{0}{0}.$$
  La ecuación asociada es $x-y=0$ por lo que el autovector $\vec{v}_1=(1,1)$ está asociado a $\la_1=4$.
  \item El espacio invariante asociado a $\la_2$ es 
  $$E_{\la_2}=\ker(A-(-2)I)=\ker\twobytwo{3}{3}{3}{3}=\ker\twobytwo{1}{1}{0}{0}.$$
  La ecuación asociada es $x+y=0$ por lo que el autovector $\vec{v}_2=(1,-1)$ está asociado a $\la_2=-2$.
  \item Así $P=[\id]^\cB_\cC=\twobytwo{1}{1}{1}{-1}$ y $D=\twobytwo{4}{0}{0}{-2}$. De forma que $A=PDP^{-1}$.
\end{enumerate}
Sin embargo, buscamos una característica que es que $P$ \emph{respete orientación y medidas}. Esto significa dos cosas:
\begin{itemize}
  \itemsep=-0.5em
  \item Respeta orientación cuando $\det P>0$.
  \item Respeta medida cuando $\det P=1$.
\end{itemize}
En este caso $\det(P)=-2$. Para obtener lo pedido hacemos lo siguiente:
\begin{itemize}
  \itemsep=-0.5em
  \item Intercambiamos las columnas de $P$ para cambiar el signo.
  \item Normalizamos las columnas de $P$.
\end{itemize}
Obtenemos así $P=\twobytwo{1/\sqrt{2}}{1/\sqrt{2}}{-1/\sqrt{2}}{1/\sqrt{2}}$. Lo que hemos hecho fue intercambiar los autovectores de orden y los reescalamos para que tuvieran norma 1. De la misma forma que antes $A=PDP^{-1}$ sólo que ahora $D=\text{diag}(-2,4)$.\par 
Con esto, ¿podemos decir que la forma sin término mixto es $-2x^2+4y^2$? Sí y no, ¡debemos cambiar variables! El cambio de variables es según la MCB
$$\vec{x}=[\id]^\cB_\cC\vec{u}\To\twobyone{x}{y}=\twobytwo{1/\sqrt{2}}{1/\sqrt{2}}{-1/\sqrt{2}}{1/\sqrt{2}}\twobyone{u}{v}.$$
De esta forma obtenemos 
$$\left(\frac{u + v}{\sqrt2}\right)^2 + 6\left(\frac{u + v}{\sqrt2}\right)\left(\frac{v-u}{\sqrt2}\right)+\left(\frac{v-u}{\sqrt2}\right)^2$$
y simplificando obtenemos $-2u^2+4v^2$.
\end{Ex}

En resumen para eliminar el término mixto se debe diagonalizar y cerciorarse que la MCB respete medidas. Esto pues cuando $\det=1$ se respeta la orientación.

\subsection*{Curvas Cuadráticas}

\begin{Def} 
Una \un{curva cuadrática} es una ecuación de la forma 
$$ax^2+bxy+cy^2+dx+ey+f=0$$
que se puede representar de la forma 
$$\vec{x}^\sT A\vec{x}+\langle\vec{b}\mid\vec{x}\rangle+f=0.$$
La \un{forma normal} de una curva cuadrática se obtiene cambiando variables con $\vec{x}=P\vec{u}$ donde $P=[\id]^\cB_\cC$ y $\cB$ es la base de autovectores de $A$. 
\end{Def}

\begin{Ex}
  Consideremos la curva dada por la ecuación
  $$Q(x,y)=3x^2+y^2-2\sqrt{3}-12x-12\sqrt{3}y=0$$
  y encontremos su forma normal. Extraemos su forma vectorial con 
  $$A=\twobytwo{3}{-\sqrt{3}}{-\sqrt{3}}{1},\ \vec{b}=\twobyone{-12}{-12\sqrt{3}},$$
  y así $Q(\vec{x})=\vec{x}^\sT A\vec{x}+\langle\vec{b}\mid \vec{x}\rangle$. Para encontrar su forma normal diagonalizamos $A$:
  \begin{enumerate}
    \itemsep=-0.5em 
    \item Los autovalores de $A$ se obtienen con 
    $$\la=m\pm\sqrt{m^2-p},\ m=(a+d)/2,\ p=\det A.$$
    En este caso $\la=2\pm\sqrt{2^2-(0)}$. Obtenemos $\la_1=0$ y $\la_2=4$.
    \item El espacio invariante asociado a $\la_1$ es $E_{\la_1}$
    $$\ker(A)=\ker\twobytwo{3}{-\sqrt{3}}{-\sqrt{3}}{1}=\ker\twobytwo{-\sqrt{3}}{1}{0}{0}.$$
    La ecuación asociada es $-\sqrt{3}x+y=0$ por lo que el autovector $\vec{v}_1=(1,\sqrt{3})$ está asociado a $\la_1=0$.
    \item El espacio invariante asociado a $\la_2$ es $E_{\la_2}$
    $$\ker(A-4I)=\ker\twobytwo{-1}{-\sqrt{3}}{-\sqrt{3}}{-3}=\ker\twobytwo{1}{\sqrt{3}}{0}{0}.$$
    La ecuación asociada es $x+\sqrt{3}y=0$ por lo que el autovector $\vec{v}_2=(-\sqrt{3},1)$ está asociado a $\la_2=4$.
    \item La matriz $P$ con columnas $\vec{v}_1$, $\vec{v}_2$ es
    $$P=\twobytwo{1}{-\sqrt{3}}{\sqrt{3}}{1}\To\det P=4.$$
    Como es positivo, nada más normalizamos las columnas de $P$ para obtener
    $$P=\twobytwo{1/2}{-\sqrt{3}/2}{\sqrt{3}/2}{1/2}\To\det P=1.$$
    \item El cambio de variables entonces es $\vec{x}=P\vec{u}$
    $$\twobyone{x}{y}=\twobytwo{1/2}{-\sqrt{3}/2}{\sqrt{3}/2}{1/2}\twobyone{u}{v}=\twobyone{(u-\sqrt{3}v)/2}{(\sqrt{3}u+v)/2}.$$
    Sustituyendo en la ecuación obtenemos 
    $$\vec{u}^\sT P^\sT AP\vec{u}+\langle\vec{b}\mid P\vec{u}\rangle=0\To \vec{u}^\sT D\vec{u}+\langle P^\sT \vec{b}\mid \vec{u}\rangle=0$$
    donde $P^\sT\vec{b}=(-24,0)$.
    Por lo tanto la forma normal de $Q$ es
    $$Q(u,v)=4v^2+24u=0\To v^2=6u.$$
  \end{enumerate}
\end{Ex}
\begin{Rmk}
  Aquí hemos utilizado una propiedad del producto punto que no conocíamos:
  $$\langle A\vec{x}\mid\vec{y}\rangle=\langle \vec{x}\mid A^\sT\vec{y}\rangle$$
\end{Rmk}

Geometricamente la ecuación que representa esta curva es una parábola horizontal centrada en el origen. Sin embargo eso es en base $\cB$ de autovectores, es decir, en coordenadas $(u,v)$. Vale que\vspace{-0.5em}
\begin{itemize}
  \itemsep=-0.5em 
  \item El eje $u$ es $\gen(\vec{v}_1)$ y el $v$ es $\gen(\vec{v}_2)$.
  \item El ángulo de rotación entre los ejes es el ángulo entre el eje $x$ y el eje $u$ ó el eje $y$ y el $v$. Esto se obtiene midiendo el ángulo entre el vector $\ii$ y $\vec{v}_1$ por ejemplo.
\end{itemize} 
En este caso el ángulo es $\arccos(1/2)=\pi/3$. 
\begin{center}
  
\tikzset{every picture/.style={line width=0.75pt}} %set default line width to 0.75pt        

\begin{tikzpicture}[x=0.75pt,y=0.75pt,yscale=-1,xscale=1]
  %uncomment if require: \path (0,300); %set diagram left start at 0, and has height of 300
  
  %Shape: Axis 2D [id:dp008647302586463512] 
  \draw [color={rgb, 255:red, 0; green, 93; blue, 164 }  ,draw opacity=1 ] (200,170.17) -- (300,170.17)(220,100) -- (220,200) (293,165.17) -- (300,170.17) -- (293,175.17) (215,107) -- (220,100) -- (225,107)  ;
  %Shape: Parabola [id:dp7879792848124503] 
  \draw  [color={rgb, 255:red, 185; green, 217; blue, 137 }  ,draw opacity=1 ][dash pattern={on 4.5pt off 4.5pt}] (260,135.17) .. controls (206.67,158.5) and (206.67,181.84) .. (260,205.17) ;
  %Shape: Parabola [id:dp8893474009896134] 
  \draw  [color={rgb, 255:red, 0; green, 134; blue, 65 }  ,draw opacity=1 ] (209.69,118.03) .. controls (203.23,175.88) and (223.44,187.55) .. (270.31,153.03) ;
  %Straight Lines [id:da9799488869450543] 
  \draw [color={rgb, 255:red, 243; green, 112; blue, 33 }  ,draw opacity=1 ]   (219.88,170.1) -- (233.86,145.84) ;
  \draw [shift={(234.86,144.1)}, rotate = 119.95] [color={rgb, 255:red, 243; green, 112; blue, 33 }  ,draw opacity=1 ][line width=0.75]    (10.93,-3.29) .. controls (6.95,-1.4) and (3.31,-0.3) .. (0,0) .. controls (3.31,0.3) and (6.95,1.4) .. (10.93,3.29)   ;
  %Straight Lines [id:da41881091313619967] 
  \draw [color={rgb, 255:red, 0; green, 192; blue, 243 }  ,draw opacity=1 ]   (219.93,170.29) -- (195.67,156.31) ;
  \draw [shift={(193.94,155.31)}, rotate = 29.95] [color={rgb, 255:red, 0; green, 192; blue, 243 }  ,draw opacity=1 ][line width=0.75]    (10.93,-3.29) .. controls (6.95,-1.4) and (3.31,-0.3) .. (0,0) .. controls (3.31,0.3) and (6.95,1.4) .. (10.93,3.29)   ;
  
  %Shape: Arc [id:dp16720961077472074] 
  \draw  [draw opacity=0] (223.67,163.81) .. controls (225.86,165.08) and (227.33,167.45) .. (227.33,170.17) -- (220,170.17) -- cycle ; \draw  [color={rgb, 255:red, 243; green, 112; blue, 33 }  ,draw opacity=1 ] (223.67,163.81) .. controls (225.86,165.08) and (227.33,167.45) .. (227.33,170.17) ;  
  
  % Text Node
  \draw (234.86,142.7) node [anchor=south] [inner sep=0.75pt]  [font=\footnotesize,color={rgb, 255:red, 243; green, 112; blue, 33 }  ,opacity=1 ]  {$\vec{u}$};
  % Text Node
  \draw (188.94,151.91) node [anchor=south west] [inner sep=0.75pt]  [font=\footnotesize,color={rgb, 255:red, 0; green, 192; blue, 243 }  ,opacity=1 ]  {$\vec{v}$};
  % Text Node
  \draw (254,196.77) node [anchor=south west] [inner sep=0.75pt]  [font=\scriptsize,color={rgb, 255:red, 185; green, 217; blue, 137 }  ,opacity=1 ]  {$y^{2} =6x$};
  % Text Node
  \draw (270.31,149.63) node [anchor=south] [inner sep=0.75pt]  [font=\scriptsize,color={rgb, 255:red, 0; green, 134; blue, 65 }  ,opacity=1 ]  {$v^{2} =6u$};
  % Text Node
  \draw (227,169.45) node [anchor=south west] [inner sep=0.75pt]  [font=\tiny,color={rgb, 255:red, 243; green, 112; blue, 33 }  ,opacity=1 ]  {$\pi /3$};
  
  
  \end{tikzpicture}
  
  

\end{center}

\begin{ptcbP}
  Encuentre la forma normal de la curva dada por
  $$3\sqrt{2}x^2+3\sqrt{2}y^2+2\sqrt{2}xy-4x-12+2\sqrt{2}=0.$$
  Seguidamente indique \vspace{-1em}
  \begin{enumerate}
    \itemsep=-0.5em
    \item Los ejes del nuevo sistema de coordenadas.
    \item El ángulo de rotación entre los sistemas.
  \end{enumerate}
\end{ptcbP}

\subsection*{Formas Normales de Curvas Cuadráticas}

Enumeramos los tipos de curvas cuadráticas en dos dimensiones:
\vspace{-1em}
\begin{enumerate}
  \itemsep=-0.5em
  \item $(y-s)^2=c(x-r)$ es una \un{parábola horizontal} centrada en $(r,s)$. Su eje de simetría es el eje $y=s$.\par 
  Si cambiamos el cuadrado obtenemos
  $$(x-r)^2=c(y-s)$$
  y esto es una \un{parábola vertical} centrada en $(r,s)$ con eje de simetría $x=r$.
  \item La \un{elipse} centrada en $(r,s)$ se describe con 
  $$\frac{(x-r)^2}{c^2}+\frac{(y-s)^2}{d^2}=1.$$
  Cuando $d=c$ se obtiene un círculo de radio $c$.
  \item La ecuación 
  $$\frac{(x-r)^2}{c^2}-\frac{(y-s)^2}{d^2}=1$$
  describe una \un{hipérbola horizontal} con vértices $(r\pm c,s)$ y ejes $x=r$, $y=s$. En cambio si cambiamos el signo obtenemos una \un{hipérbola vertical} dada por 
  $$\frac{(y-s)^2}{d^2}-\frac{(x-r)^2}{c^2}=1.$$
  Los vértices aquí son $(r,s\pm d)$.
\end{enumerate}
\begin{ptcbP}
  Basado en lo anterior, clasifique la curva del ejercicio anterior.
\end{ptcbP}

A manera de práctica, para esta parte se recomiendan los ejercicios de C. Fonseca o los del último capítulo del libro de J. Sánchez.
\end{multicols}
\end{document}