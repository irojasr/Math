%----------------------------------------------------------------------------------------
%	PACKAGES AND OTHER DOCUMENT CONFIGURATIONS
%----------------------------------------------------------------------------------------

\documentclass[12pt]{article}
\usepackage[spanish]{babel} %Tildes
\usepackage[extreme]{savetrees} %Espaciado e interlineado. Comentar si no gusta el interlineado.
\usepackage[utf8]{inputenc} %Encoding para tildes
\usepackage[breakable,skins]{tcolorbox} %Cajitas
\usepackage{fancyhdr} % Se necesita para el título arriba
\usepackage{lastpage} % Se necesita para poner el número de página
\usepackage{amsmath,amsfonts,amssymb,amsthm} %simbolos y demás
\usepackage{mathabx} %más símbolos
\usepackage{physics} %simbolos de derivadas, bra-ket.
\usepackage{multicol}
\usepackage[customcolors]{hf-tikz}
\usepackage[shortlabels]{enumitem}
\usepackage{tikz}

\def\darktheme
%%%%%%%%% === Document Configuration === %%%%%%%%%%%%%%

\pagestyle{fancy}
\setlength{\headheight}{14.49998pt} %NO MODIFICAR
\setlength{\footskip}{14.49998pt} %NO MODIFICAR

\ifx \darktheme\undefined

\lhead{MA1004G8} % Nombre de autor
\chead{\textbf{Lección 0519}} % Titulo
\rhead{}%\firstxmark} 
\lfoot{}%\lastxmark}
\cfoot{}
\rfoot{P\'ag.\ \thepage\ de\ \pageref{LastPage}} %A la derecha saldrá pág. 6 de 9. 
\else
\pagenumbering{gobble}
\pagecolor[rgb]{0,0,0}%{0.23,0.258,0.321}
\color[rgb]{1,1,1}
\fi

%%%%%%%%% === My T Color Box === %%%%%%%%%%%%%%

\ifx \darktheme\undefined
\newtcolorbox{ptcb}{
colframe = black,
colback = white,
breakable,
enhanced
}
\newtcolorbox{ptcbP}{
colframe = black,
colback = white,
coltitle = black,
colbacktitle = black!40,
title = Práctica,
breakable,
enhanced
}

\else
\newtcolorbox{ptcb}{
colframe = white,
colback = black,
colupper = white,
breakable,
enhanced
}
\newtcolorbox{ptcbP}{
colframe = white,
colback = black,
colupper = white,
coltitle = white,
colbacktitle = black,
title = Práctica,
breakable,
enhanced
}
\fi

%%%%%%%%% === Tikz para matrices === %%%%%%%%%%%%%%

\tikzset{
  style green/.style={
    set fill color=green!50!lime!60,
    set border color=white,
  },
  style cyan/.style={
    set fill color=cyan!90!blue!60,
    set border color=white,
  },
  style orange/.style={
    set fill color=orange!80!red!60,
    set border color=white,
  },
  row/.style={
    above left offset={-0.15,0.31},
    below right offset={0.15,-0.125},
    #1
  },
  col/.style={
    above left offset={-0.1,0.3},
    below right offset={0.15,-0.15},
    #1
  }
}

%%%%%%%%% === Theorems and suchlike === %%%%%%%%%%%%%%

\theoremstyle{plain}
\newtheorem{Th}{Teorema}  %%% Theorem 1.1
\newtheorem*{nTh}{Teorema}             %%% No-numbered Theorem
\newtheorem{Prop}[Th]{Proposición}     %%% Proposition 1.2
\newtheorem{Lem}[Th]{Lema}             %%% Lemma 1.3
\newtheorem*{nLem}{Lema}               %%% No-numbered Lemma
\newtheorem{Cor}[Th]{Corolario}        %%% Corollary 1.4
\newtheorem*{nCor}{Corolario}          %%% No-numbered Corollary

\theoremstyle{definition}
\newtheorem*{Def}{Definición}       %%% Definition 1.5
\newtheorem*{nonum-Def}{Definición}    %%% No number Definition
\newtheorem*{nEx}{Ejemplo}             %%% No number Example
\newtheorem{Ex}[Th]{Ejemplo}           %%% Example
\newtheorem{Ej}[Th]{Ejercicio}         %%% Exercise
\newtheorem*{nEj}{Ejercicio}           %%% No number Excercise
\newtheorem*{Not}{Notación}       %%% Definition 1.5

\theoremstyle{remark}
\newtheorem*{Rmk}{Observación}      %%%Remark 1.6

%\numberwithin{equation}{section}

\setlength{\parindent}{3ex}

%%====== Useful macros: =======%%%

\DeclareMathOperator{\gen}{gen}     %%%set generated by...
\DeclareMathOperator{\Rng}{Rng}     %%%rangomat
\DeclareMathOperator{\Nul}{Nul}     %%%rangomat
\DeclareMathOperator{\Proy}{Proy}   %%%proyección

\newcommand{\la}{\lambda}           %%%short for \lambda
\newcommand{\sg}{\sigma}            %%%short for \sigma
\newcommand{\te}{\theta}                %% short for  \theta
\renewcommand{\l}{\ell}

\newcommand{\thickhat}[1]{\mathbf{\hat{\text{$#1$}}}}
\newcommand{\ii}{\vu{\imath}}
\newcommand{\jj}{\vu{\jmath}}
\newcommand{\kk}{\thickhat{k}}

\newcommand{\bC}{\mathbb{C}}        %%%complex numbers
\newcommand{\bN}{\mathbb{N}}        %%%natural numbers
\newcommand{\bP}{\mathbb{P}}        %%%polynomials
\newcommand{\bR}{\mathbb{R}}        %%%real numbers
\newcommand{\bZ}{\mathbb{Z}}        %%%integer numbers
\newcommand{\cB}{\mathcal{B}}       %%%basis
\newcommand{\cC}{\mathcal{C}}       %%%basis
\newcommand{\cM}{\mathcal{M}}       %%%matrix family

\newcommand{\sT}{\mathsf{T}}        %%%traspuesta

\renewcommand{\geq}{\geqslant}      %%%(to save typing)
\renewcommand{\leq}{\leqslant}      %%%(to save typing)
\newcommand{\x}{\times}             %%%product
\renewcommand{\:}{\colon}           %%%colon in  f: A -> B
\newcommand{\isom}{\simeq}              %% isomorfismo

\newcommand{\un}[1]{\underline{#1}}
\newcommand{\half}{\frac12}

\newcommand*{\Cdot}{{\raisebox{-0.25ex}{\scalebox{1.5}{$\cdot$}}}}      %% cdot más grande
\renewcommand{\.}{\Cdot}                %% producto escalar

\newcommand{\twobyone}[2]{\begin{pmatrix} %% 2 x 1 matrix
  #1 \\ #2 \end{pmatrix}}
\newcommand{\twobytwo}[4]{\begin{pmatrix} %% 2 x 2 matrix
  #1 & #2 \\ #3 & #4 \end{pmatrix}}
\newcommand{\threebyone}[3]{\begin{pmatrix} %% 3 x 1 matrix
  #1 \\ #2 \\ #3 \end{pmatrix}}
\newcommand{\threebythree}[9]{\begin{pmatrix} %% 3 x 3 matrix
  #1 & #2 & #3 \\ #4 & #5 & #6 \\ #7 & #8 & #9 \end{pmatrix}}

\newcommand{\To}{\Rightarrow}

\newcommand{\vaf}{\overrightarrow}

\newcommand{\set}[1]{\{\,#1\,\}}    %% set notation
\newcommand{\Set}[1]{\biggl\{\,#1\,\biggr\}} %% set notation (large)

%----------------------------------------------------------------------------------------
%	ARTICLE CONTENTS
%----------------------------------------------------------------------------------------

\begin{document}

\begin{multicols}{2}
\subsection*{Espacios Vectoriales}

Recordemos las propiedades de la suma y producto escalar:

\begin{multicols}{2}
  \textbf{Props. Suma}
  \begin{enumerate}[i)]
    \itemsep=-0.4em
    \item $\vec{x}+\vec{y}=\vec{y}+\vec x$.
    \item {\footnotesize{$(\vec{x}+\vec y)+\vec z=\vec x+(\vec y+\vec z)$.}}
    \item $\vec x+0=\vec x$.
    \item $\vec x+(-\vec x)=0$.
  \end{enumerate}
  \textbf{Props. Mult.}
  \begin{enumerate}[i)]
    \itemsep=-0.4em
    \item $1\vec{x}=\vec x$.
    \item $(cd)\vec x=c(d\vec x)$.
    \item $c(\vec x+\vec y)=c\vec x+c\vec y$.
    \item $(c+d)\vec x=c\vec x+d\vec x$.
  \end{enumerate}
\end{multicols}

Al considerar $\bR^n$ con ``$+$'' y ``$\.$'', entonces valen estas propiedades para la suma y producto. Como $\bR^n$ satisface estas propiedades entonces podemos decir que es un espacio vectorial.

\begin{Def}
Un \un{espacio vectorial} (\textbf{e.v.}) es un conjunto $V$ con 2 operaciones ``$+$'' y ``$\.$'' (\textbf{suma vectorial y producto escalar}) que obedecen las propiedades anteriores.\par 
En ese caso un \un{vector} es un elemento de un espacio vectorial.
\end{Def}

\begin{Ex}
  $\bR$ es un espacio vectorial.
  \begin{itemize}
    \itemsep=-0.4em
    \item Suma vectorial: Suma usual de números reales.
    \item Producto escalar: Multiplicación usual.
  \end{itemize}
  Claramente estas operaciones satisfacen las propiedades.
\end{Ex}

\begin{Ex} 
El conjunto $\set{2n+1:\ n\in\bN}$ de los enteros impares \textbf{no forma un espacio vectorial} con la suma y producto usual. $3+5=8$ y $8$ no es impar. Además $0$ no es impar.
\end{Ex}

\begin{Ex}
  $\cM_{m\x n}$ es un espacio vectorial. 
  \begin{itemize}
    \itemsep=-0.4em
    \item Suma vectorial: Suma de matrices.
    \item Producto escalar: Multiplicación de una matriz por un escalar.
  \end{itemize}
\end{Ex}

\begin{Ex}
  Las funciones continuas sobre $\bR$, $\cC(\bR)$, forman un espacio vectorial. 
  \begin{itemize}
    \itemsep=-0.4em
    \item Suma vectorial: $(f+g)(x)=f(x)+g(x)$.
    \item Producto escalar: $(cf)(x)=c\.f(x)$.
  \end{itemize}
\end{Ex}

\begin{Rmk}
Nuestros vectores originales eran \emph{arreglos de datos}, como $(a,b,c)\in\bR^3$. Veamos los arreglos en estos ejemplos.
\begin{itemize}
  \itemsep=-0.4em
  \item En $\bR$ el tamaño del arreglo es \textbf{1}, cada número es un vector. 
  \item En $\bR^n$ los arreglos son $(x_1,x_2,\dots,x_n)$, es decir, tienen \textbf{tamaño $\mathbf{n}$}.
  \item Las $m$ filas de longitud $n$ de una matriz $[m\x n]$ se pueden acomodar en un arreglo enorme de \textbf{tamaño} $\mathbf{mn}$.
  \item En el caso de una función continua\dots\\¿Cómo armamos un arreglo? ¿Cómo vemos $e^x$ o $x^3+2\sin(5x)$ como un arreglo?
\end{itemize}
\end{Rmk}

A este \emph{tamaño} le llamaremos \un{dimensión}. Precisaremos la definición más adelante. Denotamos $\dim(V)$.

\begin{Rmk}
  Esto quiere decir que $\bR,\bR^n$ y $\cM_{m\x n}$ tienen dimensión finita. Respectivamente $1,n$ y $mn$. Nos preguntamos ¿cuál es la dimensión de $\cC(\bR)$?
\end{Rmk}

\subsection*{Subespacios}

Denotamos $\bP_n$ al conjunto de polinomios de grado $n$. Observemos lo siguiente:

\begin{itemize}
  \itemsep=-0.4em
  \item Todos lo polinomios son funciones continuas, entonces $\bP_n\subseteq \cC(\bR)$.
  \item Ocurre que $\dim(\bP_n)\leq \dim(\cC(\bR))$.
  \item \textbf{Suave}, ¿acaso $\bP_n$ es un espacio vectorial? No podemos hablar de dimensión sin que lo sea.
\end{itemize}

\begin{Th}[$\ast\ast$]
Supongamos que $W\subseteq V$ ya siendo $V$ un espacio vectorial. Entonces $W$ es espacio vectorial si valen las condiciones:
\begin{enumerate}[i)]
  \itemsep=-0.4em 
  \item $x,y\in W\To x+y\in W$. (\textbf{suma permanece})
  \item $x\in W\To cx\in W$. (\textbf{múltiplos permanecen})
\end{enumerate}
En este caso diremos que $W$ es \un{subespacio} de $V$. Denotamos $W\leq V$.
\end{Th}

Verifiquemos que $\bP_n\subseteq\cC(\bR)$ es un espacio vectorial aplicando el teorema. 
\begin{enumerate}
  \itemsep=-0.4em
  \item ¿Primero, $\cC(\bR)$ es un espacio vectorial? Eso lo sabemos de supuesto.
  \item Ahora, si $p,q\in\bP_n$, entonces ¿$p+q\in\bP_n$? ($\star$)
  \item Y si $p\in\bP_n$, ¿vale que $cp\in\bP_n$? ($\star$)
\end{enumerate}

Si $p\in\bP_n$ entonces 
\begin{align*}
  p(x)&=a_nx^n+a_{n-1}x^{n-1}+\dots+a_1x+a_0\\
  &=a_0+a_1x+\dots+a_{n-1}x^{n-1}+a_nx^n.
\end{align*}
Asociamos a $p$ el arreglo $(a_0,a_1,\dots,a_{n-1},a_n)$. Entonces $\dim(\bP_n)=n+1$. 

\begin{Ex}
  En $\bP_2$ consideramos $p(x)=x^2+5x+6$ asociado al vector $(6,5,1)$, mientras que $q(x)=2x^2-6x+8$ sería $(8,-6,2)$.\par 
  El vector $\jj=(0,1,0)$ corresponde con el polinomio $r(x)=x$ y $x^2$ corresponde con $\kk$.
\end{Ex}

\begin{Rmk}
  Lo que ocurre detrás del telón es que $\bP_n\isom\bR^{n+1}$.
\end{Rmk}

\subsection*{Bases y Dimensión}
\begin{itemize}
  \itemsep=-0.4em
  \item Nuestro concepto de dimensión es\\
  \emph{Dimensión  = Long. del arreglo vectorial.}
  \item ¿Podemos definir dimensión sin coordenadas?
\end{itemize} %https://math.stackexchange.com/questions/2508165/how-dimensions-for-vector-space-of-functions-are-computed

\begin{Ex}[$\star$]
  Para familiarizarnos con algunos conceptos hacemos una analogía. Imaginemos un mapa:
  \begin{itemize}
    \itemsep=-0.4em
    \item Hay distintos marcos de referencia (\textbf{bases}) para guiarse distintos al marco \emph{canónico} $(\ii,\jj)$.
    \item Los distintos puntos a los que llegamos con esas direcciones son las \textbf{combinaciones lineales}.
    \item Y en un mapa plano no precisamos más de dos direcciones para guiarnos. Una tercera es superflua. (\textbf{dependencia lineal}) 
  \end{itemize}
\end{Ex}

\begin{Def}
  Si $C=\set{\vec{v}_1,\dots,\vec{v}_n}$ es un conjunto de vectores, una \un{combinación lineal} de vectores de $C$ es cualquier vector $\vec{v}$ de la forma 
  $$\vec{v}=c_1\vec{v}_1+\dots+c_n\vec v_n.$$
\end{Def}

\begin{Ex}
  $(2,-3,3)$ es combinación lineal de $(1,0,0)$ y de $(0,1,-1)$ pues 
 $$(2,-3,3)=2\.(1,0,0)+(-3)\.(0,1,-1).$$
\end{Ex}

\begin{Def}
  El \un{conjunto generado} por una colección de vectores $C=\set{\vec{v}_1,\dots,\vec{v}_n}$ es el conjunto de \textbf{todas} las combinaciones lineales de $C$. Denotamos $\gen(C)$.
\end{Def}

\begin{Prop}
Si $C\subseteq\bR^n$, entonces $\gen(C)\leq \bR^n$.
\end{Prop}

\begin{Ex}
  Dentro de $\gen((1,0,0),(0,-1,1))$ está $(2,-3,3)$. También $(19,31,-31)$, $(4,0,0)$, $(0,-5,5)$ y $(-8,7,-7)$. ¡El conjunto es infinito!
\end{Ex}

\begin{Def}
  Un vector $\vec v$ es \un{linealmente dependendiente} (\textbf{l.d.}) de conjunto de vectores $C=\set{\vec{v}_1,\dots,\vec{v}_n}$ si $\vec{v}$ es combinación lineal de $C$.\\
  Un conjunto es linealmente dependiente si posee al menos un vector linealmente dependendiente.\\ 
  Por el contrario, un vector es \un{linealmente independiente} (\textbf{l.i.}) de un conjunto si no es l.d. Análogamente para conjuntos.
\end{Def}

\begin{Ex}
  $(2,-3,3)$ es l.d. de $(1,0,0)$ y $(0,-1,1)$, pues
  $$(2,-3,3)=2(1,0,0)+3(0,-1,1).$$
  En consecuencia $\set{(2,-3,3),(1,0,0),(0,-1,1)}$ es un conjunto l.d.\\
  Pero a su vez ocurre que 
  $$(1,0,0)=\half(2,-3,3)-\frac32(0,-1,1).$$
  Es decir los demás vectores son c.l. de los otros.
\end{Ex}

\begin{Rmk}
  Notemos que si $\vec u,\vec v$ son l.d. entonces $\vec u=c\vec v$. Tenemos entonces las siguientes equivalencias, dos vectores son:
 $$ \textbf{l.d.}\iff \textbf{paralelos}\iff\textbf{ múltiplos uno del otro}.$$
\end{Rmk}

\subsubsection*{El teorema resumen de independencia lineal}

\begin{Th}
  Considere $\set{\vec v_1,\dots,\vec{v}_m}\subseteq \bR^n$ con $m\leq n$. Si $A$ es la matriz cuyas filas son los vectores, entonces la \textbf{cantidad de vectores l.i.} es $\Rng(A)$.\\
  Cuando $m>n$, el conjunto es l.d.
\end{Th}

\begin{Ex}
Si $C=\Set{\threebyone{3}{5}{-8},\threebyone{1}{1}{-2},\threebyone{0}{-1}{1},\threebyone{1}{-3}{2}}$, y queremos ver cuantos vectores l.i. hay en $C$, consideramos la matriz
  $$A=\begin{pmatrix}
    3&5&-8\\1&1&-2\\ 0&-1&1\\ 1&-3&2
  \end{pmatrix}$$
  Al reducir vemos que $\Rng(A)=2$. Es decir, sólo dos vectores son l.i. \textbf{¿Cuáles dos?} \begin{itemize}
    \itemsep=-0.4em
    \item Tomamos $(0,-1,1)$ primero.
    \item Luego uno que no sea múltiplo, como $(1,1,-2)$.
  \end{itemize}
  Si queremos escribir $(3,5,-8)$ como c.l. de $(1,1,-2)$ y $(0,-1,1)$ debemos encontrar $a,b$ reales tales que 
  $$(3,5,-8)=a(1,1,-2)+b(0,-1,1)\To\left\lbrace\begin{aligned}
    &3=a\\
    &5=a-b\\
    &-8=-2a+b
  \end{aligned}\right.$$
  Con esto llegamos a $a=3$ y $b=-2$.
\end{Ex}

\begin{Def}
  Una \un{base} de un espacio vectorial es un conjunto generador que es l.i.\\
  La \un{dimensión} de un espacio vectorial es la cantidad de elementos en una base.
\end{Def}

\begin{Ex}
  La base canónica de $\bR^2$ es $\cC=\set{\ii,\jj}$. Pero también $\set{(1,1),\jj}$ y $\set{\ii,(-1,-2)}$ son bases de $\bR^2$. Como vimos anteriormente, $\dim(\bR^n)=n$, entonces cualquier base de $\bR^n$ tiene $n$ vectores l.i.
\end{Ex}
\begin{ptcbP}
Encuentre $\dim(\gen(C))$ donde $C\subseteq\bR^3$ es
$$\Set{\threebyone{-2}{2}{0},\threebyone{-1}{0}{1},\threebyone{0}{-1}{1},\threebyone{0}{-3}{3},\threebyone{-1}{-3}{4},\threebyone{1}{0}{-1}}.$$
\end{ptcbP}
Para finalizar consideramos $\cC(\bR)$, aquí los monomios $\set{1,x,x^2,\dots,x^n,x^{n+1},\dots}$ forman un conjunto l.i. infinito. Hay más funciones l.i. que esas pero ya con esto vemos que $\dim(\cC(\bR))=\infty$.
\end{multicols}
\end{document}