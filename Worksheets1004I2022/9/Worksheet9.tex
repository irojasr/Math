%----------------------------------------------------------------------------------------
%	PACKAGES AND OTHER DOCUMENT CONFIGURATIONS
%----------------------------------------------------------------------------------------

\documentclass[12pt]{article}
\usepackage[spanish]{babel} %Tildes
\usepackage[extreme]{savetrees} %Espaciado e interlineado. Comentar si no gusta el interlineado.
\usepackage[utf8]{inputenc} %Encoding para tildes
\usepackage[breakable,skins]{tcolorbox} %Cajitas
\usepackage{fancyhdr} % Se necesita para el título arriba
\usepackage{lastpage} % Se necesita para poner el número de página
\usepackage{amsmath,amsfonts,amssymb,amsthm} %simbolos y demás
\usepackage{mathabx} %más símbolos
\usepackage{physics} %simbolos de derivadas, bra-ket.
\usepackage{multicol}
\usepackage[customcolors]{hf-tikz}
\usepackage[shortlabels]{enumitem}
\usepackage{tikz}

\def\darktheme
%%%%%%%%% === Document Configuration === %%%%%%%%%%%%%%

\pagestyle{fancy}
\setlength{\headheight}{14.49998pt} %NO MODIFICAR
\setlength{\footskip}{14.49998pt} %NO MODIFICAR

\ifx \darktheme\undefined

\lhead{MA1004G8} % Nombre de autor
\chead{\textbf{Lección 0609}} % Titulo
\rhead{}%\firstxmark} 
\lfoot{}%\lastxmark}
\cfoot{}
\rfoot{P\'ag.\ \thepage\ de\ \pageref{LastPage}} %A la derecha saldrá pág. 6 de 9. 
\else
\pagenumbering{gobble}
\pagecolor[rgb]{0,0,0}%{0.23,0.258,0.321}
\color[rgb]{1,1,1}
\fi

%%%%%%%%% === My T Color Box === %%%%%%%%%%%%%%

\ifx \darktheme\undefined
\newtcolorbox{ptcb}{
colframe = black,
colback = white,
breakable,
enhanced
}
\newtcolorbox{ptcbP}{
colframe = black,
colback = white,
coltitle = black,
colbacktitle = black!40,
title = Práctica,
breakable,
enhanced
}

\else
\newtcolorbox{ptcb}{
colframe = white,
colback = black,
colupper = white,
breakable,
enhanced
}
\newtcolorbox{ptcbP}{
colframe = white,
colback = black,
colupper = white,
coltitle = white,
colbacktitle = black,
title = Práctica,
breakable,
enhanced
}
\fi

%%%%%%%%% === Tikz para matrices === %%%%%%%%%%%%%%

\tikzset{
  style green/.style={
    set fill color=green!50!lime!60,
    set border color=white,
  },
  style cyan/.style={
    set fill color=cyan!90!blue!60,
    set border color=white,
  },
  style orange/.style={
    set fill color=orange!80!red!60,
    set border color=white,
  },
  row/.style={
    above left offset={-0.15,0.31},
    below right offset={0.15,-0.125},
    #1
  },
  col/.style={
    above left offset={-0.1,0.3},
    below right offset={0.15,-0.15},
    #1
  }
}

%%%%%%%%% === Theorems and suchlike === %%%%%%%%%%%%%%

\theoremstyle{plain}
\newtheorem{Th}{Teorema}  %%% Theorem 1.1
\newtheorem*{nTh}{Teorema}             %%% No-numbered Theorem
\newtheorem{Prop}[Th]{Proposición}     %%% Proposition 1.2
\newtheorem{Lem}[Th]{Lema}             %%% Lemma 1.3
\newtheorem*{nLem}{Lema}               %%% No-numbered Lemma
\newtheorem{Cor}[Th]{Corolario}        %%% Corollary 1.4
\newtheorem*{nCor}{Corolario}          %%% No-numbered Corollary

\theoremstyle{definition}
\newtheorem*{Def}{Definición}       %%% Definition 1.5
\newtheorem*{nonum-Def}{Definición}    %%% No number Definition
\newtheorem*{nEx}{Ejemplo}             %%% No number Example
\newtheorem{Ex}[Th]{Ejemplo}           %%% Example
\newtheorem{Ej}[Th]{Ejercicio}         %%% Exercise
\newtheorem*{nEj}{Ejercicio}           %%% No number Excercise
\newtheorem*{Not}{Notación}       %%% Definition 1.5

\theoremstyle{remark}
\newtheorem*{Rmk}{Observación}      %%%Remark 1.6

%\numberwithin{equation}{section}

\setlength{\parindent}{3ex}

%%====== Useful macros: =======%%%

\DeclareMathOperator{\gen}{gen}     %%%set generated by...
\DeclareMathOperator{\Rng}{Rng}     %%%rangomat
\DeclareMathOperator{\Nul}{Nul}     %%%rangomat
\DeclareMathOperator{\Proy}{Proy}   %%%proyección
\DeclareMathOperator{\id}{id}       %%%identity operator

\newcommand{\la}{\lambda}           %%%short for \lambda
\newcommand{\sg}{\sigma}            %%%short for \sigma
\newcommand{\te}{\theta}                %% short for  \theta
\renewcommand{\l}{\ell}

\newcommand{\thickhat}[1]{\mathbf{\hat{\text{$#1$}}}}
\newcommand{\ii}{\vu{\imath}}
\newcommand{\jj}{\vu{\jmath}}
\newcommand{\kk}{\thickhat{k}}

\newcommand{\bC}{\mathbb{C}}        %%%complex numbers
\newcommand{\bN}{\mathbb{N}}        %%%natural numbers
\newcommand{\bP}{\mathbb{P}}        %%%polynomials
\newcommand{\bR}{\mathbb{R}}        %%%real numbers
\newcommand{\bZ}{\mathbb{Z}}        %%%integer numbers
\newcommand{\cB}{\mathcal{B}}       %%%basis
\newcommand{\cC}{\mathcal{C}}       %%%basis
\newcommand{\cM}{\mathcal{M}}       %%%matrix family

\newcommand{\sT}{\mathsf{T}}        %%%traspuesta

\renewcommand{\geq}{\geqslant}      %%%(to save typing)
\renewcommand{\leq}{\leqslant}      %%%(to save typing)
\newcommand{\x}{\times}             %%%product
\renewcommand{\:}{\colon}           %%%colon in  f: A -> B
\newcommand{\isom}{\simeq}              %% isomorfismo

\newcommand{\un}[1]{\underline{#1}}
\newcommand{\half}{\frac12}

\newcommand*{\Cdot}{{\raisebox{-0.25ex}{\scalebox{1.5}{$\cdot$}}}}      %% cdot más grande
\renewcommand{\.}{\Cdot}                %% producto escalar

\newcommand{\twobyone}[2]{\begin{pmatrix} %% 2 x 1 matrix
  #1 \\ #2 \end{pmatrix}}
  \newcommand{\twobytwo}[4]{\begin{pmatrix} %% 2 x 2 matrix
    #1 & #2 \\ #3 & #4 \end{pmatrix}}
    \newcommand{\twobythree}[6]{\begin{pmatrix} %% 2 x 3 matrix
        #1 & #2 & #3\\ #4 & #5 & #6 \end{pmatrix}}
\newcommand{\threebyone}[3]{\begin{pmatrix} %% 3 x 1 matrix
  #1 \\ #2 \\ #3 \end{pmatrix}}
  \newcommand{\threebytwo}[6]{\begin{pmatrix} %% 3 x 1 matrix
    #1 & #2\\ #3 & #4\\ #5&#6 \end{pmatrix}}
\newcommand{\threebythree}[9]{\begin{pmatrix} %% 3 x 3 matrix
  #1 & #2 & #3 \\ #4 & #5 & #6 \\ #7 & #8 & #9 \end{pmatrix}}

\newcommand{\To}{\Rightarrow}

\newcommand{\vaf}{\overrightarrow}

\newcommand{\set}[1]{\{\,#1\,\}}    %% set notation
\newcommand{\Set}[1]{\biggl\{\,#1\,\biggr\}} %% set notation (large)
\newcommand{\red}[1]{\textcolor{red}{#1}}

%----------------------------------------------------------------------------------------
%	ARTICLE CONTENTS
%----------------------------------------------------------------------------------------

\begin{document}

\begin{multicols}{2}

\subsubsection*{Síntesis y Resumen sobre T.L.'s}

\begin{enumerate}[i)]
    \itemsep=-0.5em
    \item Toda T.L. corresponde con una matriz y toda matriz corresponde con una T.L. Para pasar entre una y otra:\vspace{-1em}\begin{itemize}
        \itemsep=-0.42em 
        \item Si tenemos el criterio, $[T]_\cC$ tiene como columnas a $T\ii,\ T\jj$ y $T\kk$. En general, las cols. son las imágenes de la base.
        \item Si tenemos la matriz $[T]_\cC$, entonces multiplicamos por el vector de variables: $(x,y),\ (x,y,z)$ ó en general $(x_1,x_2,\dots,x_n)$.
    \end{itemize}
    \item La composición de T.L.'s corresponde con el producto de matrices. Si $S$ y $T$ son T.L.'s entonces $(S\circ T)(\vec{x})=S(T(\vec{x}))$ y $[S\circ T]_\cC=[S]_\cC[T]_\cC$. Al igual que el producto de matrices, la composición no necesariamente conmuta.
    \item Hay rectas (o más generalmente, subespacios) que permanecen \emph{invariantes} al aplicar una T.L., esos vectores directores son los \un{autovectores de T}.\par 
    \emph{Por ejemplo para los reescalamientos} $\ii,\ \jj$ \emph{son los autovectores.}\vspace{0.5em}
    \item Hay rectas (o subespacios) que \emph{colapsan} cuando se les aplica una T.L. tales vectores forman el \emph{núcleo o kernel} de la matriz. Las T.L.'s que no colapsan algo se llaman \un{no-singulares}.\par 
    \emph{En el caso de las proyecciones, el núcleo es el eje ortogonal al eje sobre el que se proyecta.}\vspace{0.5em}
    \item El determinante de una T.L. es el factor por el que reescala un área. Las T.L.'s con determinante 1 preservan áreas.
\end{enumerate}


\subsubsection*{Cambios de Base y Dimensiones Distintas}

Supongamos que tenemos una base $\cB=\set{\vec{u},\vec v}$ y aplicamos un cizallamiento en ese marco de referencia.
\begin{center}   
\tikzset{every picture/.style={line width=0.6pt}} 
\begin{tikzpicture}[x=1pt,y=1pt,yscale=-0.55,xscale=0.55]

%Straight Lines [id:da980472923582731] 
\draw [color={rgb, 255:red, 243; green, 112; blue, 33 }  ,draw opacity=1 ]   (199.6,100) -- (161.54,90.49) ;
\draw [shift={(159.6,90)}, rotate = 14.04] [color={rgb, 255:red, 243; green, 112; blue, 33 }  ,draw opacity=1 ][line width=0.75]    (10.93,-3.29) .. controls (6.95,-1.4) and (3.31,-0.3) .. (0,0) .. controls (3.31,0.3) and (6.95,1.4) .. (10.93,3.29)   ;
%Straight Lines [id:da6262535889615066] 
\draw [color={rgb, 255:red, 0; green, 93; blue, 164 }  ,draw opacity=1 ]   (199.6,100) -- (190.09,138.06) ;
\draw [shift={(189.6,140)}, rotate = 284.04] [color={rgb, 255:red, 0; green, 93; blue, 164 }  ,draw opacity=1 ][line width=0.75]    (10.93,-3.29) .. controls (6.95,-1.4) and (3.31,-0.3) .. (0,0) .. controls (3.31,0.3) and (6.95,1.4) .. (10.93,3.29)   ;
%Straight Lines [id:da9793804108402699] 
\draw [color={rgb, 255:red, 255; green, 173; blue, 148 }  ,draw opacity=1 ] [dash pattern={on 4.5pt off 4.5pt}]  (159.6,90) -- (79.6,70) ;
%Straight Lines [id:da8506561141387743] 
\draw [color={rgb, 255:red, 0; green, 192; blue, 243 }  ,draw opacity=1 ] [dash pattern={on 4.5pt off 4.5pt}]  (189.6,140) -- (169.6,220) ;
%Straight Lines [id:da5712803674192566] 
\draw [color={rgb, 255:red, 0; green, 192; blue, 243 }  ,draw opacity=1 ] [dash pattern={on 4.5pt off 4.5pt}]  (159.6,90) -- (129.6,210) ;
%Straight Lines [id:da44251538783707933] 
\draw [color={rgb, 255:red, 255; green, 173; blue, 148 }  ,draw opacity=1 ] [dash pattern={on 4.5pt off 4.5pt}]  (189.6,140) -- (69.6,110) ;
%Straight Lines [id:da678921018941278] 
\draw [color={rgb, 255:red, 255; green, 173; blue, 148 }  ,draw opacity=1 ] [dash pattern={on 4.5pt off 4.5pt}]  (179.6,180) -- (60,150) ;
%Straight Lines [id:da5212402356085759] 
\draw [color={rgb, 255:red, 0; green, 192; blue, 243 }  ,draw opacity=1 ] [dash pattern={on 4.5pt off 4.5pt}]  (119.6,80) -- (89.6,200) ;
%Straight Lines [id:da014028219065511882] 
\draw [color={rgb, 255:red, 243; green, 112; blue, 33 }  ,draw opacity=1 ]   (490,110) -- (451.94,100.49) ;
\draw [shift={(450,100)}, rotate = 14.04] [color={rgb, 255:red, 243; green, 112; blue, 33 }  ,draw opacity=1 ][line width=0.75]    (10.93,-3.29) .. controls (6.95,-1.4) and (3.31,-0.3) .. (0,0) .. controls (3.31,0.3) and (6.95,1.4) .. (10.93,3.29)   ;
%Straight Lines [id:da42399562863930607] 
\draw [color={rgb, 255:red, 0; green, 93; blue, 164 }  ,draw opacity=1 ]   (490,110) -- (441.71,138.97) ;
\draw [shift={(440,140)}, rotate = 329.04] [color={rgb, 255:red, 0; green, 93; blue, 164 }  ,draw opacity=1 ][line width=0.75]    (10.93,-3.29) .. controls (6.95,-1.4) and (3.31,-0.3) .. (0,0) .. controls (3.31,0.3) and (6.95,1.4) .. (10.93,3.29)   ;
%Straight Lines [id:da6118576469582722] 
\draw [color={rgb, 255:red, 255; green, 173; blue, 148 }  ,draw opacity=1 ] [dash pattern={on 4.5pt off 4.5pt}]  (450,100) -- (370,80) ;
%Straight Lines [id:da31315179169637486] 
\draw [color={rgb, 255:red, 0; green, 192; blue, 243 }  ,draw opacity=1 ] [dash pattern={on 4.5pt off 4.5pt}]  (440,140) -- (340,200) ;
%Straight Lines [id:da010507129954688166] 
\draw [color={rgb, 255:red, 255; green, 173; blue, 148 }  ,draw opacity=1 ] [dash pattern={on 4.5pt off 4.5pt}]  (440,140) -- (320,110) ;
%Straight Lines [id:da23516408999231997] 
\draw [color={rgb, 255:red, 255; green, 173; blue, 148 }  ,draw opacity=1 ] [dash pattern={on 4.5pt off 4.5pt}]  (390,170) -- (270,140) ;
%Straight Lines [id:da19803010265747734] 
\draw [color={rgb, 255:red, 0; green, 192; blue, 243 }  ,draw opacity=1 ] [dash pattern={on 4.5pt off 4.5pt}]  (450,100) -- (300,190) ;
%Straight Lines [id:da18145968188666606] 
\draw [color={rgb, 255:red, 0; green, 192; blue, 243 }  ,draw opacity=1 ] [dash pattern={on 4.5pt off 4.5pt}]  (410,90) -- (260,180) ;
%Curve Lines [id:da889784139018897] 
\draw [color={rgb, 255:red, 0; green, 192; blue, 243 }  ,draw opacity=1 ]   (217,140.6) .. controls (265.95,193.93) and (249.76,54.41) .. (294.72,108.57) ;
\draw [shift={(295.4,109.4)}, rotate = 230.92] [color={rgb, 255:red, 0; green, 192; blue, 243 }  ,draw opacity=1 ][line width=0.75]    (10.93,-3.29) .. controls (6.95,-1.4) and (3.31,-0.3) .. (0,0) .. controls (3.31,0.3) and (6.95,1.4) .. (10.93,3.29)   ;
%Straight Lines [id:da7235074852272385] 
\draw [color={rgb, 255:red, 142; green, 216; blue, 248 }  ,draw opacity=1 ] [dash pattern={on 0.84pt off 2.51pt}]  (490,110) -- (460,230) ;
%Straight Lines [id:da7764226579287561] 
\draw [color={rgb, 255:red, 142; green, 216; blue, 248 }  ,draw opacity=1 ] [dash pattern={on 0.84pt off 2.51pt}]  (450,100) -- (420,220) ;
%Straight Lines [id:da14699081812685577] 
\draw [color={rgb, 255:red, 142; green, 216; blue, 248 }  ,draw opacity=1 ] [dash pattern={on 0.84pt off 2.51pt}]  (410,90) -- (380,210) ;

% Text Node
\draw (181.6,91.6) node [anchor=south west] [inner sep=0.75pt]  [font=\footnotesize,color={rgb, 255:red, 243; green, 112; blue, 33 }  ,opacity=1 ]  {$\vec{u}$};
% Text Node
\draw (196.6,123.4) node [anchor=north west][inner sep=0.75pt]  [font=\footnotesize,color={rgb, 255:red, 0; green, 93; blue, 164 }  ,opacity=1 ]  {$\vec{v}$};
% Text Node
\draw (471.6,101.6) node [anchor=south west] [inner sep=0.75pt]  [font=\footnotesize,color={rgb, 255:red, 243; green, 112; blue, 33 }  ,opacity=1 ]  {$\vec{u}$};
% Text Node
\draw (452.89,135.58) node [anchor=north west][inner sep=0.75pt]  [font=\scriptsize,color={rgb, 255:red, 0; green, 93; blue, 164 }  ,opacity=1 ,rotate=-328]  {$T_h(\vec{v})$};
% Text Node
\draw (301.56,91.79) node [anchor=south east] [inner sep=0.75pt]  [color={rgb, 255:red, 0; green, 192; blue, 243 }  ,opacity=1 ]  {$[ T_h]_{\mathcal{B}}$};


\end{tikzpicture}
\end{center}
Consideremos el cizallamiento horizontal dado por:
$$[T_h]_\cB=\twobytwo{1}{1}{0}{1}.$$
Es importante notar que el subíndice $\cB$ nos indica sobre qué base está actuando $T_h$. Es decir: 
\vspace{-1em}
\begin{itemize}
    \itemsep=-0.5em
    \item La primera columna es $T_h(\vec{u})=1\.\vec{u}+0\.\vec{v}=\vec{u}$.
    \item La segunda es $T_h(\vec{v})=1\.\vec{u}+1\.\vec{v}=\vec{u}+\vec{v}$.
\end{itemize}
$T_h$ es un cizallamiento sobre $\cB$ pues deja invariante al eje $\vec{u}$ (y su representación es una \emph{matriz triangular con diagonal de 1's}). Sin embargo no sabemos cómo actúa esta transformación sobre $\cC=\set{\ii,\jj}$. ¿Qué son $T_h(\ii)$ y $T_h(\jj)$?
\begin{center}
  \tikzset{every picture/.style={line width=0.75pt}} %set default line width to 0.75pt        

  \begin{tikzpicture}[x=0.75pt,y=0.75pt,yscale=-1.1,xscale=1.1]
  %uncomment if require: \path (0,300); %set diagram left start at 0, and has height of 300
  
  %Straight Lines [id:da1726449388739626] 
  \draw [color={rgb, 255:red, 243; green, 112; blue, 33 }  ,draw opacity=1 ]   (169.2,104.4) -- (131.14,94.89) ;
  \draw [shift={(129.2,94.4)}, rotate = 14.04] [color={rgb, 255:red, 243; green, 112; blue, 33 }  ,draw opacity=1 ][line width=0.75]    (10.93,-3.29) .. controls (6.95,-1.4) and (3.31,-0.3) .. (0,0) .. controls (3.31,0.3) and (6.95,1.4) .. (10.93,3.29)   ;
  %Straight Lines [id:da9122311536389964] 
  \draw [color={rgb, 255:red, 0; green, 93; blue, 164 }  ,draw opacity=1 ]   (169.2,104.4) -- (159.69,142.46) ;
  \draw [shift={(159.2,144.4)}, rotate = 284.04] [color={rgb, 255:red, 0; green, 93; blue, 164 }  ,draw opacity=1 ][line width=0.75]    (10.93,-3.29) .. controls (6.95,-1.4) and (3.31,-0.3) .. (0,0) .. controls (3.31,0.3) and (6.95,1.4) .. (10.93,3.29)   ;
  %Straight Lines [id:da2568425749351968] 
  \draw [color={rgb, 255:red, 243; green, 112; blue, 33 }  ,draw opacity=1 ]   (342,103) -- (303.94,93.49) ;
  \draw [shift={(302,93)}, rotate = 14.04] [color={rgb, 255:red, 243; green, 112; blue, 33 }  ,draw opacity=1 ][line width=0.75]    (10.93,-3.29) .. controls (6.95,-1.4) and (3.31,-0.3) .. (0,0) .. controls (3.31,0.3) and (6.95,1.4) .. (10.93,3.29)   ;
  %Straight Lines [id:da947129521705526] 
  \draw [color={rgb, 255:red, 0; green, 93; blue, 164 }  ,draw opacity=1 ]   (342,103) -- (293.71,131.97) ;
  \draw [shift={(292,133)}, rotate = 329.04] [color={rgb, 255:red, 0; green, 93; blue, 164 }  ,draw opacity=1 ][line width=0.75]    (10.93,-3.29) .. controls (6.95,-1.4) and (3.31,-0.3) .. (0,0) .. controls (3.31,0.3) and (6.95,1.4) .. (10.93,3.29)   ;
  %Straight Lines [id:da33793734256356256] 
  \draw [color={rgb, 255:red, 255; green, 173; blue, 148 }  ,draw opacity=1 ] [dash pattern={on 4.5pt off 4.5pt}]  (382,73) -- (342,63) ;
  %Curve Lines [id:da4632235651634029] 
  \draw [color={rgb, 255:red, 0; green, 192; blue, 243 }  ,draw opacity=1 ]   (209.4,118.6) .. controls (258.35,171.93) and (242.16,32.41) .. (287.12,86.57) ;
  \draw [shift={(287.8,87.4)}, rotate = 230.92] [color={rgb, 255:red, 0; green, 192; blue, 243 }  ,draw opacity=1 ][line width=0.75]    (10.93,-3.29) .. controls (6.95,-1.4) and (3.31,-0.3) .. (0,0) .. controls (3.31,0.3) and (6.95,1.4) .. (10.93,3.29)   ;
  %Straight Lines [id:da7896137172079905] 
  \draw [color={rgb, 255:red, 0; green, 134; blue, 65 }  ,draw opacity=1 ]   (169.2,104.4) -- (169.2,66.4) ;
  \draw [shift={(169.2,64.4)}, rotate = 90] [color={rgb, 255:red, 0; green, 134; blue, 65 }  ,draw opacity=1 ][line width=0.75]    (10.93,-3.29) .. controls (6.95,-1.4) and (3.31,-0.3) .. (0,0) .. controls (3.31,0.3) and (6.95,1.4) .. (10.93,3.29)   ;
  %Straight Lines [id:da29494251217733325] 
  \draw [color={rgb, 255:red, 253; green, 185; blue, 18 }  ,draw opacity=1 ]   (169.2,104.4) -- (207.2,104.4) ;
  \draw [shift={(209.2,104.4)}, rotate = 180] [color={rgb, 255:red, 253; green, 185; blue, 18 }  ,draw opacity=1 ][line width=0.75]    (10.93,-3.29) .. controls (6.95,-1.4) and (3.31,-0.3) .. (0,0) .. controls (3.31,0.3) and (6.95,1.4) .. (10.93,3.29)   ;
  %Straight Lines [id:da8919394724771395] 
  \draw [color={rgb, 255:red, 0; green, 134; blue, 65 }  ,draw opacity=1 ]   (342,103) -- (380.4,74.2) ;
  \draw [shift={(382,73)}, rotate = 143.13] [color={rgb, 255:red, 0; green, 134; blue, 65 }  ,draw opacity=1 ][line width=0.75]    (10.93,-3.29) .. controls (6.95,-1.4) and (3.31,-0.3) .. (0,0) .. controls (3.31,0.3) and (6.95,1.4) .. (10.93,3.29)   ;
  %Straight Lines [id:da03544134550017941] 
  \draw [color={rgb, 255:red, 253; green, 185; blue, 18 }  ,draw opacity=1 ]   (342,103) -- (420.02,112.75) ;
  \draw [shift={(422,113)}, rotate = 187.13] [color={rgb, 255:red, 253; green, 185; blue, 18 }  ,draw opacity=1 ][line width=0.75]    (10.93,-3.29) .. controls (6.95,-1.4) and (3.31,-0.3) .. (0,0) .. controls (3.31,0.3) and (6.95,1.4) .. (10.93,3.29)   ;
  %Straight Lines [id:da864823174966127] 
  \draw [color={rgb, 255:red, 255; green, 173; blue, 148 }  ,draw opacity=1 ] [dash pattern={on 4.5pt off 4.5pt}]  (422,113) -- (382,103) ;
  %Straight Lines [id:da08881362099560519] 
  \draw [color={rgb, 255:red, 185; green, 217; blue, 137 }  ,draw opacity=0.5 ] [dash pattern={on 4.5pt off 4.5pt}]  (342,103) -- (342,65) ;
  \draw [shift={(342,63)}, rotate = 90] [color={rgb, 255:red, 185; green, 217; blue, 137 }  ,draw opacity=0.5 ][line width=0.75]    (10.93,-3.29) .. controls (6.95,-1.4) and (3.31,-0.3) .. (0,0) .. controls (3.31,0.3) and (6.95,1.4) .. (10.93,3.29)   ;
  %Straight Lines [id:da37703616732499823] 
  \draw [color={rgb, 255:red, 255; green, 224; blue, 106 }  ,draw opacity=1 ] [dash pattern={on 4.5pt off 4.5pt}]  (342,103) -- (380,103) ;
  \draw [shift={(382,103)}, rotate = 180] [color={rgb, 255:red, 255; green, 224; blue, 106 }  ,draw opacity=1 ][line width=0.75]    (10.93,-3.29) .. controls (6.95,-1.4) and (3.31,-0.3) .. (0,0) .. controls (3.31,0.3) and (6.95,1.4) .. (10.93,3.29)   ;
  
  % Text Node
  \draw (131.2,91) node [anchor=south west] [inner sep=0.75pt]  [font=\footnotesize,color={rgb, 255:red, 243; green, 112; blue, 33 }  ,opacity=1 ]  {$\vec{u}$};
  % Text Node
  \draw (166.2,127.8) node [anchor=north west][inner sep=0.75pt]  [font=\footnotesize,color={rgb, 255:red, 0; green, 93; blue, 164 }  ,opacity=1 ]  {$\vec{v}$};
  % Text Node
  \draw (302,89.4) node [anchor=south west] [inner sep=0.75pt]  [font=\footnotesize,color={rgb, 255:red, 243; green, 112; blue, 33 }  ,opacity=1 ]  {$\vec{u}$};
  % Text Node
  \draw (304.89,128.58) node [anchor=north west][inner sep=0.75pt]  [font=\scriptsize,color={rgb, 255:red, 0; green, 93; blue, 164 }  ,opacity=1 ,rotate=-328]  {$T_{h}(\vec{v})$};
  % Text Node
  \draw (293.96,69.79) node [anchor=south east] [inner sep=0.75pt]  [color={rgb, 255:red, 0; green, 192; blue, 243 }  ,opacity=1 ]  {$[ T_{h}]_{\mathcal{B}}$};
  % Text Node
  \draw (171.2,61) node [anchor=south west] [inner sep=0.75pt]  [font=\footnotesize,color={rgb, 255:red, 0; green, 134; blue, 65 }  ,opacity=1 ]  {$\jj$};
  % Text Node
  \draw (211.2,101) node [anchor=south west] [inner sep=0.75pt]  [font=\footnotesize,color={rgb, 255:red, 253; green, 185; blue, 18 }  ,opacity=1 ]  {$\ii$};
  % Text Node
  \draw (384,69.6) node [anchor=south west] [inner sep=0.75pt]  [font=\footnotesize,color={rgb, 255:red, 0; green, 134; blue, 65 }  ,opacity=1 ]  {¿$T_{h}(\jj)$?};
  % Text Node
  \draw (422,116.4) node [anchor=north] [inner sep=0.75pt]  [font=\footnotesize,color={rgb, 255:red, 253; green, 185; blue, 18 }  ,opacity=1 ]  {¿$T_{h}(\ii)$?};
  
  
  \end{tikzpicture}
  
\end{center}
¡Podemos resolver por medio de cambios de base! 

\begin{Ex} 
Si $\cB$ es $\set{(-1,1/4),(-1/4,-1)}$ y $T_h$ es una T.L. dada por $[T_h]_\cB=\twobytwo{1}{1}{0}{1}$, entonces $T_h$ actúa sobre vectores \emph{escritos en coordenadas de} $\cB$.\par 
Para encontrar $T_h(\ii),T_h(\jj)$ traducimos $\ii,\jj$ a $\cB$, aplicamos $T_h$ en $\cB$ y traducimos de vuelta:
\begin{itemize}
  \itemsep=-0.48em
  \item Primero encontramos la matriz de cambio de base tomando los vectores de $\cB$ como columnas:
  $$MCB=[\id]^\cB_\cC=\twobytwo{-1}{-1/4}{1/4}{-1}.$$
  \item Esta anterior es la que va de $\cB$ en $\cC$, entonces invertimos para obtener la que va de $\cC$ en $\cB$:
  $$MCB^{-1}=[\id]^\cC_\cB=\twobytwo{-16/17}{4/17}{-4/17}{-16/17}.$$
  \item Traducimos $\ii,\jj$ a $[\ii]_\cB,[\jj]_\cB$:
  $$\left\lbrace
  \begin{aligned}
    &[\ii]_\cB=[\id]^\cC_\cB(1,0)=(-16/17,-4/17),\\
    &[\jj]_\cB=[\id]^\cC_\cB(0,1)=(4/17,-16/17).
  \end{aligned}
  \right.$$
  \item Ya podemos aplicarle a estos vectores $[T_h]_\cB$:
  $$\left\lbrace
  \begin{aligned}
    &[T_h\ii]_\cB=[T_h]_\cB[\ii]_\cB=(-20/17,-4/17),\\
    &[T_h\jj]_\cB=[T_h]_\cB[\jj]_\cB=(-12/17,-16/17).
  \end{aligned}
  \right.$$
  \item Los vectores que obtuvimos están escritos en términos de $\cB$, entonces traducimos de vuelta:
  $$\left\lbrace
  \begin{aligned}
    &[T_h\ii]_\cC=[\id]^{\cB}_\cC[T_h\ii]_\cB=(21/17,-1/17),\\
    &[T_h\jj]_\cC=[\id]^{\cB}_\cC[T_h\jj]_\cB=(16/17,13/17),\\
  \end{aligned}
  \right.$$
\end{itemize}
Estos últimos vectores son las imágenes de $\ii$ y $\jj$ bajo $T_h$, entonces podemos formar la matriz $T_h$ en base canónica:
$$[T_h]_\cC=\frac{1}{17}\twobytwo{21}{16}{-1}{13}.$$
\end{Ex}

\begin{Rmk}
La matriz que hemos obtenido no es un cizallamiento, ni ninguno de los otros 4 tipos de T.L.'s que vimos. Pero es un cizallamiento en la base $\cB$.\par 
A propósito aprovechamos para consolidar la notación formal para matrices de cambio de base. $[\id]^{\cB_1}_{\cB_2}$ significa la matriz de cambio de base desde $\cB_1$ hacia $\cB_2$.
\end{Rmk} 
\subsubsection*{Proceso General}
El proceso no es más que una multiplicación de matrices en ambos lados. Resumamos lo que hicimos:
\begin{enumerate}[i)]
  \itemsep=-0.5em
  \item Pasamos los vectores canónicos de $\cC$ a $\cB$ con $[\id]^\cC_\cB$.
  \item Multiplicamos $[T_h]_\cB$ a los vectores traducidos.
  \item Devolvemos a $\cC$ el resultado con $[\id]^\cB_\cC$.
\end{enumerate}
Esto nos dice entonces que la operación completa es:
$$[T_h]_\cC=[\id]^\cB_\cC[T_h]_\cB[\id]^\cC_\cB.$$
Entonces en general si tenemos una T.L. $S$ dada en términos de una base $\cB_1$, $[S]_{\cB_1}$ y queremos traducirla a una base $\cB_2$, entonces multiplicamos de la siguiente forma:
$$[S]_{\cB_2}=[\id]_{\cB_2}^{\cB_1}[S]_{\cB_1}[\id]_{\cB_1}^{\cB_2}.$$
Pero, no siempre vamos a tener T.L.'s de una misma base en si misma, o de la misma dimensión tanto para la entrada como para la salida.

\begin{Ex}
  Consideremos $T:\bR^3\to\bR^2$ dada por 
  $$T(x,y,z)=(x+y,3x-z)$$ 
  y las bases 
  $$\left\lbrace
  \begin{aligned}
    &\cB_1=\set{(1,1,2),(-3,0,1),(2,4,3)},\\
    &\cB_2=\set{(4,1),(3,1)}.
  \end{aligned}
  \right.$$
  Queremos encontrar $[T]_{\cB_2}^{\cB_1}$.\par 
  Primero llamamos $\cC_1$ y $\cC_2$ a las bases canónicas de $\bR^3$ y $\bR^2$ respectivamente. Entonces encontramos las imágenes de $\cC_1$ bajo $T$:
  $$\left\lbrace
  \begin{aligned}
    &T\ii=(\red{1}+\red{0},3\.\red{1}-\red{0})=(1,3),\\
    &T\jj=(\red{0}+\red{1},3\.\red{0}-\red{0})=(1,0),\\
    &T\jj=(\red{0}+\red{0},3\.\red{0}-\red{1})=(0,-1).
  \end{aligned}
  \right.$$
  Esto nos dice que $[T]_{\cC_2}^{\cC_1}=\twobythree{1}{1}{0}{3}{0}{-1}$. Seguidamente, queremos que vaya desde $\cB_1$ a $\cB_2$, para eso seguimos el siguiente camino:
  $$\bR^3(\cB_1)\xrightarrow[]{[\id]_{\cC_1}^{\cB_1}}\bR^3(\cC_1)\xrightarrow[]{[T]_{\cC_2}^{\cC_1}}\bR^2(\cC_2)\xrightarrow[]{[\id]_{\cB_2}^{\cC_2}}\bR^2(\cB_2).$$
  Podemos encontrar las matrices de cambio de base tomando los vectores de las bases como columnas:
  $$[\id]_{\cC_1}^{\cB_1}=\threebythree{1}{-3}{2}{1}{0}{4}{2}{1}{3},\ [\id]_{\cC_2}^{\cB_2}=\twobytwo{4}{3}{1}{1}.$$
  Invertimos $[\id]_{\cC_2}^{\cB_2}$ para obtener $[\id]_{\cB_2}^{\cC_2}=\twobytwo{1}{-3}{-1}{4}$. Así multiplicamos para obtener:
  \begin{align*}
    [T]^{\cB_1}_{\cB_2}=[\id]_{\cB_2}^{\cC_2}[T]_{\cC_2}^{\cC_1}[\id]_{\cC_1}^{\cB_1}=\twobythree{-1}{27}{-3}{2}{-37}{6}.
  \end{align*}
\end{Ex}

\begin{Ex}
Supongamos que $T:\bR^3\to\bR^2$ cumple que 
$$\left\lbrace
\begin{aligned}
  &T(6,3,-2)=(a,1),\\
  &T(3,-2,6)=(2a,b),\\
  &T(-2,6,3)=(3a,2b).
\end{aligned}
\right.$$
Donde $a,b\in\bZ$. Si $\vec{x}=(49,49,49)$, buscamos $T(\vec{x})$. \begin{itemize}
  \itemsep=-0.5em
  \item Para este efecto notamos que el conjunto $\cB=\set{\vec{u},\vec{v},\vec{w}}$, donde estos son los vectores anteriores, es una base de $\bR^3$. 
  \item Como no tenemos indicación expresa sobre los vectores en $\bR^2$, asumimos que están escritos en coordenadas canónicas.
  \item Esto significa que 
  $$[T]^{\cB}_{\cC_2}=\twobythree{a}{2a}{3a}{1}{b}{2b}.$$
  \item Por lo tanto debemos traducir esta matriz para que tome vectores desde $\cC_1$, la base canónica de $\bR^3$. La matriz $[\id]^{\cC_1}_\cB$ es la inversa de 
  $$[\id]^{\cB}_{\cC_1}=\threebythree{6}{3}{-2}{3}{-2}{6}{-2}{6}{3}.$$
  Es este caso resulta que 
  $$[\id]^{\cC_1}_{\cB}=([\id]^{\cB}_{\cC_1})^{-1}=\frac{1}{49}\threebythree{6}{3}{-2}{3}{-2}{6}{-2}{6}{3}.$$
  \item Traducimos $\vec{x}$ a $\cB$ por medio de esta matriz:
  $$(49,49,49)\xrightarrow[]{[\id]^{\cC_1}_{\cB}}(7,7,7).$$
  \item Como ya lo tenemos escrito en coordenadas de $\cB$, podemos aplicar $[T]^{\cB}_{\cC_2}$ para obtener 
  $$\twobythree{a}{2a}{3a}{1}{b}{2b}\threebyone{7}{7}{7}=\twobyone{42a}{7(3b+1)}.$$
\end{itemize}
\end{Ex}

\begin{ptcbP}
Considere la T.L. $T:\bR^2\to\bR^2$ dada por
$$T(a,b)=(a+2b,3a-b)$$
y las bases $\cB_1=\set{(1,1),(1,0)}$ y $\cB_2=\set{(4,7),(4,8)}$.
\vspace{-0.5em}
\begin{enumerate}[i)]
  \itemsep=-0.5em
  \item Encuentre $[T]_{\cB_1}$ y $[T]_{\cB_2}$.
  \item Encuentre una matriz invertible $P$ tal que $[T]_{\cB_2}=P^{-1}[T]_{\cB_1}P$.
\end{enumerate}
\end{ptcbP}
\end{multicols}
\end{document}