%----------------------------------------------------------------------------------------
%	PACKAGES AND OTHER DOCUMENT CONFIGURATIONS
%----------------------------------------------------------------------------------------

\documentclass[12pt]{article}
\usepackage[spanish]{babel} %Tildes
\usepackage[extreme]{savetrees} %Espaciado e interlineado. Comentar si no gusta el interlineado.
\usepackage[utf8]{inputenc} %Encoding para tildes
\usepackage[breakable,skins]{tcolorbox} %Cajitas
\usepackage{fancyhdr} % Se necesita para el título arriba
\usepackage{lastpage} % Se necesita para poner el número de página
\usepackage{amsmath,amsfonts,amssymb,amsthm} %simbolos y demás
\usepackage{mathabx} %más símbolos
\usepackage{physics} %simbolos de derivadas, bra-ket.
\usepackage{multicol}
\usepackage[customcolors]{hf-tikz}
\usepackage[shortlabels]{enumitem}
\usepackage{tikz}

%\def\darktheme
%%%%%%%%% === Document Configuration === %%%%%%%%%%%%%%

\pagestyle{fancy}
\setlength{\headheight}{14.49998pt} %NO MODIFICAR
\setlength{\footskip}{14.49998pt} %NO MODIFICAR

\ifx \darktheme\undefined

\lhead{MA1004G8} % Nombre de autor
\chead{\textbf{Lección 0512}} % Titulo
\rhead{}%\firstxmark} 
\lfoot{}%\lastxmark}
\cfoot{}
\rfoot{P\'ag.\ \thepage\ de\ \pageref{LastPage}} %A la derecha saldrá pág. 6 de 9. 
\else
\pagenumbering{gobble}
\pagecolor[rgb]{0,0,0}%{0.23,0.258,0.321}
\color[rgb]{1,1,1}
\fi

%%%%%%%%% === My T Color Box === %%%%%%%%%%%%%%

\ifx \darktheme\undefined
\newtcolorbox{ptcb}{
colframe = black,
colback = white,
breakable,
enhanced
}
\newtcolorbox{ptcbP}{
colframe = black,
colback = white,
coltitle = black,
colbacktitle = black!40,
title = Práctica,
breakable,
enhanced
}

\else
\newtcolorbox{ptcb}{
colframe = white,
colback = black,
colupper = white,
breakable,
enhanced
}
\newtcolorbox{ptcbP}{
colframe = white,
colback = black,
colupper = white,
coltitle = white,
colbacktitle = black,
title = Práctica,
breakable,
enhanced
}
\fi

%%%%%%%%% === Tikz para matrices === %%%%%%%%%%%%%%

\tikzset{
  style green/.style={
    set fill color=green!50!lime!60,
    set border color=white,
  },
  style cyan/.style={
    set fill color=cyan!90!blue!60,
    set border color=white,
  },
  style orange/.style={
    set fill color=orange!80!red!60,
    set border color=white,
  },
  row/.style={
    above left offset={-0.15,0.31},
    below right offset={0.15,-0.125},
    #1
  },
  col/.style={
    above left offset={-0.1,0.3},
    below right offset={0.15,-0.15},
    #1
  }
}

%%%%%%%%% === Theorems and suchlike === %%%%%%%%%%%%%%

\theoremstyle{plain}
\newtheorem{Th}{Teorema}  %%% Theorem 1.1
\newtheorem*{nTh}{Teorema}             %%% No-numbered Theorem
\newtheorem{Prop}[Th]{Proposición}     %%% Proposition 1.2
\newtheorem{Lem}[Th]{Lema}             %%% Lemma 1.3
\newtheorem*{nLem}{Lema}               %%% No-numbered Lemma
\newtheorem{Cor}[Th]{Corolario}        %%% Corollary 1.4
\newtheorem*{nCor}{Corolario}          %%% No-numbered Corollary

\theoremstyle{definition}
\newtheorem*{Def}{Definición}       %%% Definition 1.5
\newtheorem*{nonum-Def}{Definición}    %%% No number Definition
\newtheorem*{nEx}{Ejemplo}             %%% No number Example
\newtheorem{Ex}[Th]{Ejemplo}           %%% Example
\newtheorem{Ej}[Th]{Ejercicio}         %%% Exercise
\newtheorem*{nEj}{Ejercicio}           %%% No number Excercise
\newtheorem*{Not}{Notación}       %%% Definition 1.5

\theoremstyle{remark}
\newtheorem*{Rmk}{Observación}      %%%Remark 1.6

%\numberwithin{equation}{section}

\setlength{\parindent}{3ex}

%%====== Useful macros: =======%%%

\DeclareMathOperator{\gen}{gen}     %%%set generated by...
\DeclareMathOperator{\Rng}{Rng}     %%%rangomat
\DeclareMathOperator{\Nul}{Nul}     %%%rangomat
\DeclareMathOperator{\Proy}{Proy}   %%%proyección

\newcommand{\la}{\lambda}           %%%short for \lambda

\newcommand{\sg}{\sigma}            %%%short for \sigma

\newcommand{\bC}{\mathbb{C}}        %%%complex numbers
\newcommand{\bN}{\mathbb{N}}        %%%natural numbers
\newcommand{\bR}{\mathbb{R}}        %%%real numbers
\newcommand{\bZ}{\mathbb{Z}}        %%%integer numbers
\newcommand{\cB}{\mathcal{B}}       %%%basis
\newcommand{\cC}{\mathcal{C}}       %%%basis
\newcommand{\cM}{\mathcal{M}}       %%%matrix family

\newcommand{\sT}{\mathsf{T}}        %%%traspuesta

\renewcommand{\geq}{\geqslant}      %%%(to save typing)
\renewcommand{\leq}{\leqslant}      %%%(to save typing)
\newcommand{\x}{\times}             %%%product
\renewcommand{\:}{\colon}           %%%colon in  f: A -> B

\newcommand{\un}[1]{\underline{#1}}
\newcommand{\half}{\frac12}

\newcommand*{\Cdot}{{\raisebox{-0.25ex}{\scalebox{1.5}{$\cdot$}}}}      %% cdot más grande
\renewcommand{\.}{\Cdot}                %% producto escalar

\newcommand{\twobyone}[2]{\begin{pmatrix} %% 2 x 1 matrix
  #1 \\ #2 \end{pmatrix}}
\newcommand{\twobytwo}[4]{\begin{pmatrix} %% 2 x 2 matrix
  #1 & #2 \\ #3 & #4 \end{pmatrix}}
\newcommand{\threebyone}[3]{\begin{pmatrix} %% 3 x 1 matrix
  #1 \\ #2 \\ #3 \end{pmatrix}}
\newcommand{\threebythree}[9]{\begin{pmatrix} %% 3 x 3 matrix
  #1 & #2 & #3 \\ #4 & #5 & #6 \\ #7 & #8 & #9 \end{pmatrix}}

\newcommand{\To}{\Rightarrow}

\newcommand{\thickhat}[1]{\mathbf{\hat{\text{$#1$}}}}
\newcommand{\ii}{\vu{\imath}}
\newcommand{\jj}{\vu{\jmath}}
\newcommand{\kk}{\thickhat{k}}

\newcommand{\vaf}{\overrightarrow}

\newcommand{\set}[1]{\{\,#1\,\}}    %% set notation
\newcommand{\Set}[1]{\biggl\{\,#1\,\biggr\}} %% set notation (large)

%----------------------------------------------------------------------------------------
%	ARTICLE CONTENTS
%----------------------------------------------------------------------------------------

\begin{document}
\begin{multicols}{2}
\subsection*{Planos}

\subsubsection*{Ecuación Normal}

Vimos que una ecuación normal 
$$\braket{\vec{n}}{\vec{x}-P}=0$$
representa una recta en 2D. ¿Pero qué pasa en 3D?

\begin{Ex}
  Consideremos la ecuación \begin{align*}
    &\braket{(1,1,1)}{(x,y,z)-(1,2,3)}=0\\
    \iff&\braket{(1,1,1)}{(x-1,y-2,z-3)}=0\\
    \iff&x+y+z=6.
  \end{align*}
  Esta ecuación describe un plano que interseca los ejes cartesianos en $(6,0,0),(0,6,0)$ y $(0,0,6)$.
\end{Ex}

\begin{Def}
Un \un{plano} en $\bR^3$ se describe con la ecuación normal 
$$ax+by+cz=d.$$
El \un{vector normal} es ortogonal a todos los vectores afines que están en el plano y está dado por $\vec{n}=(a,b,c)$
\end{Def}

\subsubsection*{Vectores Directores}

Al igual que una recta está definida por \emph{dos puntos}, un plano se define con \emph{tres puntos}. Pero, ¿cómo encontramos el plano que contiene tres puntos distintos $P,Q,R$?\par 
Los vectores afines $\overrightarrow{PQ}$ y $\overrightarrow{PR}$ son dos vectores afines dentro del plano. Podemos describir entonces el plano de \emph{forma paramétrica}:

$$\vec{x}=s\vaf{PQ}+t\vaf{PR}+P.$$

\begin{Prop}
El \un{vector normal} al plano $\pi=\set{\vec{x}=s\vaf{PQ}+t\vaf{PR}+P}$ es 
$$\vec{n}=\vaf{PQ}\x\vaf{PR}.$$
\end{Prop}

\begin{ptcbP}
Considere los puntos $A=(1,2,3),\ B=(4,5,6)$ y $C=(7,8,9)$. Encuentre el plano que contiene a estos puntos.
\end{ptcbP}

\begin{Ex}
Consideremos la ecuación $2x-3y=6$, en 2D tiene la forma:
\begin{align*}
  &\braket{(2,-3)}{(x,y)}=6\\
  \iff&\braket{(2,-3)}{(x-3,y)}=0\\
  \iff&\braket{(2,-3)}{(x,y+2)}=0.
\end{align*}
Es decir, tiene vector normal $(2,-3)$ y vector director $(3,2)$.\par 
Para encontrar la \un{forma vectorial} despejamos una de las variables:
\begin{align*}
  &2x-3y=6\iff y=2x/3-2\\
  \iff&(x,y)=\left(x,2x/3-2\right)=\left(1,2/3\right)x+(0,-2).
\end{align*}
Parametrizamos tomando $x=t$ y así obtenemos las \un{ecuaciones paramétricas}:
$$
\left\lbrace
\begin{aligned}
  &x=t\\
  &y=\frac23t-2
\end{aligned}
\right.
$$
En 3D la ecuación tiene $z$ como variable libre, $(3,0,0)$ resuelve la ecuación, pero también lo hace $(3,0,2022)$. Vemos que la misma ecuación define un \emph{plano} en 3D.\par 
Esto ve con la inclusión de un parámetro más en la \un{forma paramétrica} del plano:
\begin{align*}
  (x,y,z)&=\left(x,\frac23x-2,z\right)\\
  &=\left(1,\frac23,0\right)x+(0,0,1)z+(0,-2,0).
\end{align*}
Los vectores directores del plano son los que acompañan a la $x$ y a la $z$. El vector normal de este plano será:
$$\vec{n}=\left(1,\frac23,0\right)\x(0,0,1)=\det\threebythree{\ii}{\jj}{\kk}{1}{2/3}{0}{0}{0}{1}=\threebyone{2/3}{-1}{0}.$$
Finalmente observemos que este vector y el normal del plano son paralelos pues 
$$3\vec{n}=(2,-3,0)\parallel(2,-3).$$
\end{Ex}
\subsection*{Intersecciones y Direcciones}

\begin{Rmk}
  Si tenemos dos conjuntos algebraicos (rectas o planos), su intersección está dada por los puntos que satisfagan las ecuaciones de ambos conjuntos.
\end{Rmk}

Intuitivamente la intersección de dos planos \emph{no paralelos} es una recta. Verifiquémoslo.

\begin{Ex}
  Consideremos los planos 
  $$
\left\lbrace
\begin{aligned}
  &\pi_1=\set{17x+2y+7z=34},\\
  &\pi_2=\set{23x-6y+16z=46}.
\end{aligned}
\right.
$$
Su intersección está dada por los puntos que satisfacen ambas ecuaciones al mismo tiempo. Es decir, por el conjunto solución del sistema:
$$
\left\lbrace
\begin{aligned}
  &17x+2y+7z=34\\
  &23x-6y+16z=46
\end{aligned}
\right.
$$
Extrayendo la matriz del sistema y reduciendo obtenemos:
$$\left(\begin{array}{ccc|c}
  17&2&7&34\\
  23&-6&16&46
\end{array}\right)\to\left(\begin{array}{ccc|c}
  1&0&1/2&2\\
  0&1&-3/4&0
\end{array}\right)$$
De aquí extraemos las ecuaciones 
$$
\left\lbrace
\begin{aligned}
  &x+z/2=2\\
  &y-3z/4=0
\end{aligned}
\right.
$$
Y por lo tanto la solución es
$$(x,y,z)=\left(-z/2+2,3z/4,z\right)=\left(-1/2,3/4,1\right)z+(2,0,0).$$
\end{Ex}

\subsection*{Distancias}
\end{multicols}
\end{document}