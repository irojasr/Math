%----------------------------------------------------------------------------------------
%	PACKAGES AND OTHER DOCUMENT CONFIGURATIONS
%----------------------------------------------------------------------------------------

\documentclass[12pt]{article}
\usepackage[spanish]{babel} %Tildes
\usepackage[extreme]{savetrees} %Espaciado e interlineado. Comentar si no gusta el interlineado.
\usepackage[utf8]{inputenc} %Encoding para tildes
\usepackage[breakable,skins]{tcolorbox} %Cajitas
\usepackage{fancyhdr} % Se necesita para el título arriba
\usepackage{lastpage} % Se necesita para poner el número de página
\usepackage{amsmath,amsfonts,amssymb,amsthm} %simbolos y demás
\usepackage{mathabx} %más símbolos
\usepackage{physics} %simbolos de derivadas, bra-ket.
\usepackage{multicol}
\usepackage[customcolors]{hf-tikz}
\usepackage[shortlabels]{enumitem}
\usepackage{tikz}

\def\darktheme
%%%%%%%%% === Document Configuration === %%%%%%%%%%%%%%

\pagestyle{fancy}
\setlength{\headheight}{14.49998pt} %NO MODIFICAR
\setlength{\footskip}{14.49998pt} %NO MODIFICAR

\ifx \darktheme\undefined

\lhead{MA1004G8} % Nombre de autor
\chead{\textbf{Lección 0512}} % Titulo
\rhead{}%\firstxmark} 
\lfoot{}%\lastxmark}
\cfoot{}
\rfoot{P\'ag.\ \thepage\ de\ \pageref{LastPage}} %A la derecha saldrá pág. 6 de 9. 
\else
\pagenumbering{gobble}
\pagecolor[rgb]{0,0,0}%{0.23,0.258,0.321}
\color[rgb]{1,1,1}
\fi

%%%%%%%%% === My T Color Box === %%%%%%%%%%%%%%

\ifx \darktheme\undefined
\newtcolorbox{ptcb}{
colframe = black,
colback = white,
breakable,
enhanced
}
\newtcolorbox{ptcbP}{
colframe = black,
colback = white,
coltitle = black,
colbacktitle = black!40,
title = Práctica,
breakable,
enhanced
}

\else
\newtcolorbox{ptcb}{
colframe = white,
colback = black,
colupper = white,
breakable,
enhanced
}
\newtcolorbox{ptcbP}{
colframe = white,
colback = black,
colupper = white,
coltitle = white,
colbacktitle = black,
title = Práctica,
breakable,
enhanced
}
\fi

%%%%%%%%% === Tikz para matrices === %%%%%%%%%%%%%%

\tikzset{
  style green/.style={
    set fill color=green!50!lime!60,
    set border color=white,
  },
  style cyan/.style={
    set fill color=cyan!90!blue!60,
    set border color=white,
  },
  style orange/.style={
    set fill color=orange!80!red!60,
    set border color=white,
  },
  row/.style={
    above left offset={-0.15,0.31},
    below right offset={0.15,-0.125},
    #1
  },
  col/.style={
    above left offset={-0.1,0.3},
    below right offset={0.15,-0.15},
    #1
  }
}

%%%%%%%%% === Theorems and suchlike === %%%%%%%%%%%%%%

\theoremstyle{plain}
\newtheorem{Th}{Teorema}  %%% Theorem 1.1
\newtheorem*{nTh}{Teorema}             %%% No-numbered Theorem
\newtheorem{Prop}[Th]{Proposición}     %%% Proposition 1.2
\newtheorem{Lem}[Th]{Lema}             %%% Lemma 1.3
\newtheorem*{nLem}{Lema}               %%% No-numbered Lemma
\newtheorem{Cor}[Th]{Corolario}        %%% Corollary 1.4
\newtheorem*{nCor}{Corolario}          %%% No-numbered Corollary

\theoremstyle{definition}
\newtheorem*{Def}{Definición}       %%% Definition 1.5
\newtheorem*{nonum-Def}{Definición}    %%% No number Definition
\newtheorem*{nEx}{Ejemplo}             %%% No number Example
\newtheorem{Ex}[Th]{Ejemplo}           %%% Example
\newtheorem{Ej}[Th]{Ejercicio}         %%% Exercise
\newtheorem*{nEj}{Ejercicio}           %%% No number Excercise
\newtheorem*{Not}{Notación}       %%% Definition 1.5

\theoremstyle{remark}
\newtheorem*{Rmk}{Observación}      %%%Remark 1.6

%\numberwithin{equation}{section}

\setlength{\parindent}{3ex}

%%====== Useful macros: =======%%%

\DeclareMathOperator{\gen}{gen}     %%%set generated by...
\DeclareMathOperator{\Rng}{Rng}     %%%rangomat
\DeclareMathOperator{\Nul}{Nul}     %%%rangomat
\DeclareMathOperator{\Proy}{Proy}   %%%proyección

\newcommand{\la}{\lambda}           %%%short for \lambda
\newcommand{\sg}{\sigma}            %%%short for \sigma
\newcommand{\te}{\theta}                %% short for  \theta
\renewcommand{\l}{\ell}

\newcommand{\thickhat}[1]{\mathbf{\hat{\text{$#1$}}}}
\newcommand{\ii}{\vu{\imath}}
\newcommand{\jj}{\vu{\jmath}}
\newcommand{\kk}{\thickhat{k}}

\newcommand{\bC}{\mathbb{C}}        %%%complex numbers
\newcommand{\bN}{\mathbb{N}}        %%%natural numbers
\newcommand{\bR}{\mathbb{R}}        %%%real numbers
\newcommand{\bZ}{\mathbb{Z}}        %%%integer numbers
\newcommand{\cB}{\mathcal{B}}       %%%basis
\newcommand{\cC}{\mathcal{C}}       %%%basis
\newcommand{\cM}{\mathcal{M}}       %%%matrix family

\newcommand{\sT}{\mathsf{T}}        %%%traspuesta

\renewcommand{\geq}{\geqslant}      %%%(to save typing)
\renewcommand{\leq}{\leqslant}      %%%(to save typing)
\newcommand{\x}{\times}             %%%product
\renewcommand{\:}{\colon}           %%%colon in  f: A -> B

\newcommand{\un}[1]{\underline{#1}}
\newcommand{\half}{\frac12}

\newcommand*{\Cdot}{{\raisebox{-0.25ex}{\scalebox{1.5}{$\cdot$}}}}      %% cdot más grande
\renewcommand{\.}{\Cdot}                %% producto escalar

\newcommand{\twobyone}[2]{\begin{pmatrix} %% 2 x 1 matrix
  #1 \\ #2 \end{pmatrix}}
\newcommand{\twobytwo}[4]{\begin{pmatrix} %% 2 x 2 matrix
  #1 & #2 \\ #3 & #4 \end{pmatrix}}
\newcommand{\threebyone}[3]{\begin{pmatrix} %% 3 x 1 matrix
  #1 \\ #2 \\ #3 \end{pmatrix}}
\newcommand{\threebythree}[9]{\begin{pmatrix} %% 3 x 3 matrix
  #1 & #2 & #3 \\ #4 & #5 & #6 \\ #7 & #8 & #9 \end{pmatrix}}

\newcommand{\To}{\Rightarrow}

\newcommand{\vaf}{\overrightarrow}

\newcommand{\set}[1]{\{\,#1\,\}}    %% set notation
\newcommand{\Set}[1]{\biggl\{\,#1\,\biggr\}} %% set notation (large)

%----------------------------------------------------------------------------------------
%	ARTICLE CONTENTS
%----------------------------------------------------------------------------------------

\begin{document}
\begin{multicols}{2}
\subsection*{Planos}

\subsubsection*{Ecuación Normal}

Vimos que una ecuación normal 
$$\braket{\vec{n}}{\vec{x}-P}=0$$
representa una recta en 2D. ¿Pero qué pasa en 3D?

\begin{Ex}
  Consideremos la ecuación \begin{align*}
    &\braket{(1,1,1)}{(x,y,z)-(1,2,3)}=0\\
    \iff&\braket{(1,1,1)}{(x-1,y-2,z-3)}=0\\
    \iff&x+y+z=6.
  \end{align*}
  Esta ecuación describe un plano que interseca los ejes cartesianos en $(6,0,0),(0,6,0)$ y $(0,0,6)$.
\end{Ex}

\begin{Def}
Un \un{plano} en $\bR^3$ se describe con la ecuación normal 
$$ax+by+cz=d.$$
El \un{vector normal} es ortogonal a todos los vectores afines que están en el plano y está dado por $\vec{n}=(a,b,c)$
\end{Def}

\subsubsection*{Vectores Directores}

Al igual que una recta está definida por \emph{dos puntos}, un plano se define con \emph{tres puntos}. Pero, ¿cómo encontramos el plano que contiene tres puntos distintos $P,Q,R$?\par 
Los vectores afines $\overrightarrow{PQ}$ y $\overrightarrow{PR}$ son dos vectores afines dentro del plano. Podemos describir entonces el plano de \emph{forma paramétrica}:

$$\vec{x}=s\vaf{PQ}+t\vaf{PR}+P.$$

\begin{Prop}
El \un{vector normal} al plano $\pi=\set{\vec{x}=s\vaf{PQ}+t\vaf{PR}+P}$ es 
$$\vec{n}=\vaf{PQ}\x\vaf{PR}.$$
\end{Prop}

\begin{ptcbP}
Considere los puntos $A=(1,2,3),\ B=(4,5,6)$ y $C=(7,8,9)$. Encuentre el plano que contiene a estos puntos.
\end{ptcbP}

\begin{Ex}
Consideremos la ecuación $2x-3y=6$, en 2D tiene la forma:
\begin{align*}
  &\braket{(2,-3)}{(x,y)}=6\\
  \iff&\braket{(2,-3)}{(x-3,y)}=0\\
  \iff&\braket{(2,-3)}{(x,y+2)}=0.
\end{align*}
Es decir, tiene vector normal $(2,-3)$ y vector director $(3,2)$.\par 
Para encontrar la \un{forma vectorial} despejamos una de las variables:
\begin{align*}
  &2x-3y=6\iff y=2x/3-2\\
  \iff&(x,y)=\left(x,2x/3-2\right)=\left(1,2/3\right)x+(0,-2).
\end{align*}
Parametrizamos tomando $x=t$ y así obtenemos las \un{ecuaciones paramétricas}:
$$
\left\lbrace
\begin{aligned}
  &x=t\\
  &y=2/3t-2
\end{aligned}
\right.
$$
En 3D la ecuación tiene $z$ como variable libre, $(3,0,0)$ resuelve la ecuación, pero también lo hace $(3,0,2022)$. Vemos que la misma ecuación define un \emph{plano} en 3D.\par 
Esto ve con la inclusión de un parámetro más en la \un{forma paramétrica} del plano:
\begin{align*}
  (x,y,z)&=\left(x,2x/3-2,z\right)\\
  &=\left(1,2/3,0\right)x+(0,0,1)z+(0,-2,0).
\end{align*}
Los vectores directores del plano son los que acompañan a la $x$ y a la $z$. El vector normal de este plano será:
$$\vec{n}=\left(1,2/3,0\right)\x(0,0,1)=\det\threebythree{\ii}{\jj}{\kk}{1}{2/3}{0}{0}{0}{1}=\threebyone{2/3}{-1}{0}.$$
Finalmente observemos que este vector y el normal del plano son paralelos pues 
$$3\vec{n}=(2,-3,0)\parallel(2,-3).$$
\end{Ex}
\subsection*{Intersecciones y Direcciones}

\begin{Rmk}
  Si tenemos dos conjuntos algebraicos (rectas o planos), su intersección está dada por los puntos que satisfagan las ecuaciones de ambos conjuntos.
\end{Rmk}

Intuitivamente la intersección de dos planos \emph{no paralelos} es una recta. Verifiquémoslo.

\begin{Ex}
  Consideremos los planos 
  $$
\left\lbrace
\begin{aligned}
  &\pi_1=\set{17x+2y+7z=34},\\
  &\pi_2=\set{23x-6y+16z=46}.
\end{aligned}
\right.
$$
Su intersección está dada por los puntos que satisfacen ambas ecuaciones al mismo tiempo. Es decir, por el conjunto solución del sistema:
$$
\left\lbrace
\begin{aligned}
  &17x+2y+7z=34\\
  &23x-6y+16z=46
\end{aligned}
\right.
$$
Extrayendo la matriz del sistema y reduciendo obtenemos:
$$\left(\begin{array}{ccc|c}
  17&2&7&34\\
  23&-6&16&46
\end{array}\right)\to\left(\begin{array}{ccc|c}
  1&0&1/2&2\\
  0&1&-3/4&0
\end{array}\right)$$
De aquí extraemos las ecuaciones 
$$
\left\lbrace
\begin{aligned}
  &x+z/2=2\\
  &y-3z/4=0
\end{aligned}
\right.
$$
Y por lo tanto la solución es
$$(x,y,z)=\left(-z/2+2,3z/4,z\right)=\left(-1/2,3/4,1\right)z+(2,0,0).$$
Esto es una recta con vector director $\vec{v}_\l=\left(-1/2,3/4,1\right)$ que atraviesa el punto $P=(2,0,0)$.
\end{Ex}

\begin{Rmk}
  Notemos que la dirección de la recta es ortogonal a ambas direcciones normales de los planos en cuestión. Entonces otra forma de encontrar el vector director $\vec v_\l$ es con el producto cruz de los vectores normales:
  $$\vec v_\l\parallel\vec n_1\x\vec n_2=\det\threebythree{\ii}{\jj}{\kk}{17}{2}{7}{23}{-6}{16}=(74, -111, -148).$$
  Observemos que $-148\vec v_\l=\vec n_1\x\vec n_2$.
\end{Rmk}

Es importante recordar que dos vectores son paralelos si y sólo si uno es múltiplo del otro.

\begin{Ex}
  Consideremos los planos 
  $$
\left\lbrace
\begin{aligned}
  &\pi_1=\set{6x+8y-3z=12},\\
  &\pi_2=\set{9x+2y+3z=-12}.
\end{aligned}
\right.
$$
Si queremos encontrar el ángulo formado entre ellos, aprovechamos los vectores normales. Observe que el ángulo entre $\vec n_1$ y $\vec n_2$ es el mismo ángulo que hay entre $\pi_1$ y $\pi_2$:
\begin{center}


  \tikzset{every picture/.style={line width=0.75pt}} %set default line width to 0.75pt        

  \begin{tikzpicture}[x=0.75pt,y=0.75pt,yscale=-1,xscale=1]
  %uncomment if require: \path (0,231); %set diagram left start at 0, and has height of 231
  
  %Straight Lines [id:da0994319358352318] 
  \draw    (300,130) -- (350,130) ;
  %Straight Lines [id:da5452438015129433] 
  \draw    (350,130) -- (380,100) ;
  %Straight Lines [id:da33334264214999165] 
  \draw    (330,100) -- (380,100) ;
  %Straight Lines [id:da563271022118635] 
  \draw    (250,130) -- (300,130) ;
  %Straight Lines [id:da867268690506326] 
  \draw    (250,130) -- (280,100) ;
  %Straight Lines [id:da9945567476910919] 
  \draw  [dash pattern={on 4.5pt off 4.5pt}]  (280,100) -- (330,100) ;
  %Straight Lines [id:da38351081882114957] 
  \draw    (300,130) -- (270,80) ;
  %Straight Lines [id:da6645403990161602] 
  \draw    (270,80) -- (300,50) ;
  %Straight Lines [id:da458783257797331] 
  \draw  [dash pattern={on 4.5pt off 4.5pt}]  (300,130) -- (330,100) ;
  %Straight Lines [id:da07292245962307775] 
  \draw    (330,100) -- (300,50) ;
  %Straight Lines [id:da5698543420005264] 
  \draw    (300,130) -- (315,155) ;
  %Straight Lines [id:da2723004524957362] 
  \draw  [dash pattern={on 4.5pt off 4.5pt}]  (330,100) -- (345,125) ;
  %Straight Lines [id:da6784333644618068] 
  \draw    (315,155) -- (345,125) ;
  %Straight Lines [id:da8158667125784482] 
  \draw    (315,115) -- (315,57) ;
  \draw [shift={(315,55)}, rotate = 90] [color={rgb, 255:red, 0; green, 0; blue, 0 }  ][line width=0.75]    (10.93,-3.29) .. controls (6.95,-1.4) and (3.31,-0.3) .. (0,0) .. controls (3.31,0.3) and (6.95,1.4) .. (10.93,3.29)   ;
  %Straight Lines [id:da7133545166441528] 
  \draw    (315,115) -- (378.3,76.05) ;
  \draw [shift={(380,75)}, rotate = 148.39] [color={rgb, 255:red, 0; green, 0; blue, 0 }  ][line width=0.75]    (10.93,-3.29) .. controls (6.95,-1.4) and (3.31,-0.3) .. (0,0) .. controls (3.31,0.3) and (6.95,1.4) .. (10.93,3.29)   ;
  %Shape: Arc [id:dp37285144311992235] 
  \draw  [draw opacity=0][line width=0.75]  (315.58,85.01) .. controls (326.29,85.21) and (335.65,91.04) .. (340.8,99.68) -- (315,115) -- cycle ; \draw  [line width=0.75]  (315.58,85.01) .. controls (326.29,85.21) and (335.65,91.04) .. (340.8,99.68) ;  
  %Shape: Arc [id:dp9897935732618544] 
  \draw  [draw opacity=0][line width=0.75]  (270.01,129.39) .. controls (270.22,118.67) and (276.06,109.32) .. (284.71,104.18) -- (300,130) -- cycle ; \draw  [line width=0.75]  (270.01,129.39) .. controls (270.22,118.67) and (276.06,109.32) .. (284.71,104.18) ;  
  
  % Text Node
  \draw (288,126.27) node [anchor=south east] [inner sep=0.75pt]  [font=\normalsize]  {$\theta $};
  % Text Node
  \draw (316.67,47) node [anchor=west] [inner sep=0.75pt]  [font=\normalsize]  {$\vec{n}_{1}$};
  % Text Node
  \draw (327.57,87.01) node [anchor=south west] [inner sep=0.75pt]  [font=\normalsize]  {$\theta $};
  % Text Node
  \draw (383,71.6) node [anchor=south east] [inner sep=0.75pt]  [font=\normalsize]  {$\vec{n}_{2}$};
  % Text Node
  \draw (352,133.4) node [anchor=north west][inner sep=0.75pt]    {$\pi _{1}$};
  % Text Node
  \draw (268,76.6) node [anchor=south east] [inner sep=0.75pt]    {$\pi _{2}$};
  
  
  \end{tikzpicture}
  \end{center}
  De esta forma tenemos que el ángulo $\te$ se obtiene de la ecuación
  $$\braket{\vec{n}_1}{\vec{n}_2}=\norm{\vec{n}_1}\norm{\vec{n}_2}\cos(\te)\To\te=\arccos\left(\frac{61}{\sqrt{109}\sqrt{94}}\right).$$
  Este valor es irracional pero una aproximación es $\te\approx53^\circ$.
\end{Ex}

\begin{Ex}
  Considere las rectas 
  $$
  \left\lbrace
  \begin{aligned}
    &\l_1=\set{\vec{x}=t(-2,-2,-3)+(-2,2,3)},\\
    &\l_2=\set{\vec{x}=t(-2,4,4)+(-4,0,0)}.
  \end{aligned}
  \right.
  $$
  El ángulo entre $\l_1$ y $\l_2$ es el mismo que entre sus vectores directores. Usando la identidad del producto punto tenemos que
  $$\braket{\vec{v}_1}{\vec{v}_2}=\norm{\vec{v}_1}\norm{\vec{v}_2}\cos(\te)\To\te=\arccos\left(\frac{16}{\sqrt{17}\.6}\right).$$
  Es valor es aproximadamente $\te\approx 50^\circ$.
\end{Ex}

\begin{Rmk}
  El ángulo entre una recta y un plano es importante para determinar si son paralelos o perpendiculares entre si. En general vale que:
  \begin{align*}
    &\l\parallel\pi\iff\vec v_\l\perp\vec n_\pi,&\l\perp\pi\iff\vec v_\l\parallel\vec n_\pi.
  \end{align*}
  En cualquier otro caso diremos que las rectas y los planos son \un{oblicuos}. 
\end{Rmk}
\subsection*{Distancias}
Para encontrar distancias entre conjuntos algebraicos consideramos los casos más básicos:
\begin{enumerate}[i)]
  \item Si $P$ es un punto fuera de una \un{recta} $\l$ que contiene a $Q$, entonces la distancia entre $P$ y $\l$ es 
  $$d(P,\l)=\frac{\norm{\vaf{QP}\x\vec{v}_l}}{\norm{\vec{v}_l}}.$$
  \item Si $P$ es un punto fuera de un \un{plano} $\pi$ que contiene a $Q$, entonces la distancia entre $P$ y $\pi$ es 
  $$d(P,\pi)=\frac{\left|\braket{\vaf{QP}}{\vec{n}_\pi}\right|}{\norm{\vec{n}_\pi}}.$$
  \item Si $\l_1:\ t\vec{v}_1+P$ y $\l_2:\ t\vec{v}_2+Q$, con $\vec{v}_1\not\parallel\vec{v}_2$, entonces
  $$d(\l_1,\l_2)=\frac{\left|\braket{\vaf{QP}}{\vec{v}_1\x\vec{v}_2}\right|}{\norm{\vec{v}_1\x\vec{v}_2}}.$$
  %%https://math.stackexchange.com/questions/2213165/find-shortest-distance-between-lines-in-3d
\end{enumerate}
La distancia entre una recta y un plano paralelos, dos planos paralelos o dos rectas paralelas se encuentra utilizando la \emph{primera fórmula}.

\begin{ptcbP}
  Considere los puntos $P=(1,2,0), Q=(1,0,1)$ y $R=(2,1,0)$. Si $\l$ es la recta que contiene a $P$ y $Q$, encuentre $d(R,\l)$. También encuentre $\angle(R,\l)$.
\end{ptcbP}

\begin{ptcbP}
  Encuentre la distancia entre los planos 
  $$
  \left\lbrace
  \begin{aligned}
    &\pi_1=\set{x+3y+5z=-4},\\
    &\pi_2=\set{x+3y+5z=16}.
  \end{aligned}
  \right.
  $$
\end{ptcbP}
\subsubsection*{Resumen}
\begin{itemize}
  \itemsep=-0.4em
  \item El vector normal de $ax+by+cz=d$ es $\vec{n}=(a,b,c)$.
  \item Encontrar intersecciones es resolver sistemas lineales.
  \item La dirección de la recta de intersección de dos planos es $\vec n_1\x\vec n_2$.
  \item El ángulo formado por dos planos es el mismo que hay entre $\vec n_1,\ \vec n_2$.
  \item El ángulo formado por dos rectas es el mismo que hay entre $\vec v_1,\ \vec v_2$.
  \item Las distancias entre conjuntos algebraicos dependen de un punto fuera del conjunto y del vector director o normal de ese conjunto.
\end{itemize}
\end{multicols}
\end{document}