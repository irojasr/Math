%----------------------------------------------------------------------------------------
%	PACKAGES AND OTHER DOCUMENT CONFIGURATIONS
%----------------------------------------------------------------------------------------

\documentclass[12pt]{article}
\usepackage[spanish]{babel} %Tildes
\usepackage[extreme]{savetrees} %Espaciado e interlineado. Comentar si no gusta el interlineado.
\usepackage[utf8]{inputenc} %Encoding para tildes
\usepackage[breakable,skins]{tcolorbox} %Cajitas
\usepackage{fancyhdr} % Se necesita para el título arriba
\usepackage{lastpage} % Se necesita para poner el número de página
\usepackage{amsmath,amsfonts,amssymb,amsthm} %simbolos y demás
\usepackage{mathabx} %más símbolos
\usepackage{physics} %simbolos de derivadas, bra-ket.
\usepackage{multicol}
\usepackage[customcolors]{hf-tikz}
\usepackage[shortlabels]{enumitem}
\usepackage{tikz}

%\def\darktheme
%%%%%%%%% === Document Configuration === %%%%%%%%%%%%%%

\pagestyle{fancy}
\setlength{\headheight}{14.49998pt} %NO MODIFICAR
\setlength{\footskip}{14.49998pt} %NO MODIFICAR

\ifx \darktheme\undefined

\lhead{MA1004} % Nombre de autor
\chead{\textbf{Propuestas Repo 2}} % Titulo
\rhead{}%\firstxmark} 
\lfoot{}%\lastxmark}
\cfoot{}
\rfoot{P\'ag.\ \thepage\ de\ \pageref{LastPage}} %A la derecha saldrá pág. 6 de 9. 
\else
\pagenumbering{gobble}
\pagecolor[rgb]{0,0,0}%{0.23,0.258,0.321}
\color[rgb]{1,1,1}
\fi

%%%%%%%%% === My T Color Box === %%%%%%%%%%%%%%

\ifx \darktheme\undefined
\newtcolorbox{ptcb}{
colframe = black,
colback = white,
breakable,
enhanced
}
\newtcolorbox{ptcbP}{
colframe = black,
colback = white,
coltitle = black,
colbacktitle = black!40,
title = Práctica,
breakable,
enhanced
}

\else
\newtcolorbox{ptcb}{
colframe = white,
colback = black,
colupper = white,
breakable,
enhanced
}
\newtcolorbox{ptcbP}{
colframe = white,
colback = black,
colupper = white,
coltitle = white,
colbacktitle = black,
title = Práctica,
breakable,
enhanced
}
\fi

%%%%%%%%% === Tikz para matrices === %%%%%%%%%%%%%%

\tikzset{
  style green/.style={
    set fill color=green!50!lime!60,
    set border color=white,
  },
  style cyan/.style={
    set fill color=cyan!90!blue!60,
    set border color=white,
  },
  style orange/.style={
    set fill color=orange!80!red!60,
    set border color=white,
  },
  row/.style={
    above left offset={-0.15,0.31},
    below right offset={0.15,-0.125},
    #1
  },
  col/.style={
    above left offset={-0.1,0.3},
    below right offset={0.15,-0.15},
    #1
  }
}

%%%%%%%%% === Theorems and suchlike === %%%%%%%%%%%%%%

\theoremstyle{plain}
\newtheorem{Th}{Teorema}  %%% Theorem 1.1
\newtheorem*{nTh}{Teorema}             %%% No-numbered Theorem
\newtheorem{Prop}[Th]{Proposición}     %%% Proposition 1.2
\newtheorem{Lem}[Th]{Lema}             %%% Lemma 1.3
\newtheorem*{nLem}{Lema}               %%% No-numbered Lemma
\newtheorem{Cor}[Th]{Corolario}        %%% Corollary 1.4
\newtheorem*{nCor}{Corolario}          %%% No-numbered Corollary

\theoremstyle{definition}
\newtheorem*{Def}{Definición}       %%% Definition 1.5
\newtheorem*{nonum-Def}{Definición}    %%% No number Definition
\newtheorem*{nEx}{Ejemplo}             %%% No number Example
\newtheorem{Ex}[Th]{Ejemplo}           %%% Example
\newtheorem{Ej}[Th]{Ejercicio}         %%% Exercise
\newtheorem*{nEj}{Ejercicio}           %%% No number Excercise
\newtheorem*{Not}{Notación}       %%% Definition 1.5

\theoremstyle{remark}
\newtheorem*{Rmk}{Observación}      %%%Remark 1.6

%\numberwithin{equation}{section}

\setlength{\parindent}{3ex}

%%====== Useful macros: =======%%%

\DeclareMathOperator{\gen}{gen}     %%%set generated by...
\DeclareMathOperator{\Rng}{Rng}     %%%rangomat
\DeclareMathOperator{\Nul}{Nul}     %%%rangomat
\DeclareMathOperator{\Proy}{Proy}   %%%proyección
\DeclareMathOperator{\id}{id}       %%%identity operator

\newcommand{\la}{\lambda}           %%%short for \lambda
\newcommand{\sg}{\sigma}            %%%short for \sigma
\newcommand{\te}{\theta}                %% short for  \theta
\renewcommand{\l}{\ell}

\newcommand{\thickhat}[1]{\mathbf{\hat{\text{$#1$}}}}
\newcommand{\ii}{\vu{\imath}}
\newcommand{\jj}{\vu{\jmath}}
\newcommand{\kk}{\thickhat{k}}

\newcommand{\bC}{\mathbb{C}}        %%%complex numbers
\newcommand{\bN}{\mathbb{N}}        %%%natural numbers
\newcommand{\bP}{\mathbb{P}}        %%%polynomials
\newcommand{\bR}{\mathbb{R}}        %%%real numbers
\newcommand{\bZ}{\mathbb{Z}}        %%%integer numbers
\newcommand{\cB}{\mathcal{B}}       %%%basis
\newcommand{\cC}{\mathcal{C}}       %%%basis
\newcommand{\cM}{\mathcal{M}}       %%%matrix family

\newcommand{\sT}{\mathsf{T}}        %%%traspuesta

\renewcommand{\geq}{\geqslant}      %%%(to save typing)
\renewcommand{\leq}{\leqslant}      %%%(to save typing)
\newcommand{\x}{\times}             %%%product
\renewcommand{\:}{\colon}           %%%colon in  f: A -> B
\newcommand{\isom}{\simeq}              %% isomorfismo

\newcommand{\un}[1]{\underline{#1}}
\newcommand{\half}{\frac12}

\newcommand*{\Cdot}{{\raisebox{-0.25ex}{\scalebox{1.5}{$\cdot$}}}}      %% cdot más grande
\renewcommand{\.}{\Cdot}                %% producto escalar

\newcommand{\twobyone}[2]{\begin{pmatrix} %% 2 x 1 matrix
  #1 \\ #2 \end{pmatrix}}
\newcommand{\twobytwo}[4]{\begin{pmatrix} %% 2 x 2 matrix
    #1 & #2 \\ #3 & #4 \end{pmatrix}}
\newcommand{\twobythree}[6]{\begin{pmatrix} %% 2 x 3 matrix
        #1 & #2 & #3\\ #4 & #5 & #6 \end{pmatrix}}
\newcommand{\threebyone}[3]{\begin{pmatrix} %% 3 x 1 matrix
  #1 \\ #2 \\ #3 \end{pmatrix}}
\newcommand{\threebytwo}[6]{\begin{pmatrix} %% 3 x 1 matrix
    #1 & #2\\ #3 & #4\\ #5&#6 \end{pmatrix}}
\newcommand{\threebythree}[9]{\begin{pmatrix} %% 3 x 3 matrix
  #1 & #2 & #3 \\ #4 & #5 & #6 \\ #7 & #8 & #9 \end{pmatrix}}

\newcommand{\To}{\Rightarrow}

\newcommand{\vaf}{\overrightarrow}

\newcommand{\set}[1]{\{\,#1\,\}}    %% set notation
\newcommand{\Set}[1]{\biggl\{\,#1\,\biggr\}} %% set notation (large)
\newcommand{\red}[1]{\textcolor{red}{#1}}
\newcommand{\blu}[1]{\textcolor{blue}{#1}}

%----------------------------------------------------------------------------------------
%	ARTICLE CONTENTS
%----------------------------------------------------------------------------------------

\begin{document}

\subsection*{Respuesta Breve}

\textbf{Primera idea:} Tomar una matriz con autovalores enteros, reemplazar una entrada de la matriz con un parámetro y preguntar por el valor del parámetro para el cual la matriz tiene tales autovalores. 

\begin{Ej}
Considere la matriz $A=\twobytwo{0}{3}{r}{-4}$. ¿Para cuál valor de $r$ se cumple que los autovalores de $A$ son $2$ y $-6$?
\end{Ej}

Podemos responder de varias formas:

\begin{enumerate}
  \item Usando la fórmula de autovalores $[2\x 2]$ tenemos que 
  $$\blu{\la=m\pm\sqrt{m^2-p},\ m=1/2(a+d),\ p=\det(A).}$$
  En este caso $m=-2$ y $p=-3r$. Despejando la fórmula obtenemos 
  $$(\la+2)^2=4+3r\To \la^2+4\la=3r.$$
  Tomando $\la=-2$ la ecuación es $4+8=3r$ y así $\un{r=4}$, ó bien si $\la=-6$ obtenemos $36-24=3r$ y de la misma forma $\un{r=4}$.
  \item Encontrando el polinomio característico llegamos a que 
  $$\blu{f_A(\la)=\la^2+4\la-3r.}$$
  Como los autovalores anulan a $f_A$, al sustituir cualquiera de los dos autovalores en la ecuación igualada a cero llegamos a que $\un{r=4}$. 
  \item O bien, al desarrollar el polinomio característico que sabemos que es $(\la-2)(\la+6)$ y \un{comparar coeficientes} para obtener
  $$\la^2+4\la-12\To -3r=-12\To \un{r=4}.$$
\end{enumerate}

\begin{Ej}
Considere la matriz $A=\threebythree{-4}{-4}{4}{2}{-4}{s}{4}{4}{-4}$. ¿Para cuál valor de $s$ se cumple que los autovalores de $A$ son $0$, $-6$ y $-6$?
\end{Ej}
%-6, -6, 0	[-4 -4  4;  2 -4  1;  4  4 -4]

En el caso $[3\x 3]$ no tenemos una fórmula para encontrar autovalores rápidamente, entonces no podemos usar el primer método.

\begin{enumerate}
  \item El polinomio característico de $A$ se puede encontrar con la regla de Sarrus de forma que obtenemos 
  $$\blu{f_A(\la)=-\la^3-12\la^2-40\la+4s\la.}$$
  Como los autovalores anulan $f_A$, sustituimos $\la=-6$ (sustituir $\la=0$ no da información) y obtenemos 
  $$24-24s=0\To \un{s=1}.$$
  \item \un{Comparando coeficientes} tenemos que 
  $$f_A(\la)=-\la^3-12\la^2-40\la+4s\la=\la(\la+6)^2=\la^3+12\la^2+36\la.$$
  (Aquí cambiamos el signo porque lo calculé con $A-\la I$ y no $\la I-A$). Comparando el coeficiente lineal vemos que 
  $$40-4s=36\To s=1.$$
\end{enumerate}
\newpage
\textbf{Segunda idea:} Preguntar por cuál es el autovalor asociado a un autovector de manera creativa.\par 
Recordar que si $\vec{v}$ es autovector de $T$ entonces $T\vec v=\la\vec{v}$. Es decir $T$ reescala $\vec v$ por un factor de $\la$.

\begin{Ej}
  Considere la T.L. $T(x,y)=(3x+2y,-2x-2y)$. Si $\vec{v}=(-2,1)$ es un autovector de $T$, ¿bajo cuál factor reescala $T$ a $\vec v$?
\end{Ej}
%-1, 2	[3  2; -2 -2] [-1 2] [-2 1]

La dificultad de esta pregunta está en entender qué significa reescalamiento. Incluso sin decir que $(-2,1)$ es un autovector, basta con evaluar 
$$T(-2,1)=(3\.-2+2\.1,-2\.-2-2\.1)=(-4,2)=\un{2}(-2,1).$$
El factor de reescalamiento es $2$ y por tanto ese valor es el autovalor asociado a $(-2,1)$.\par 
Otra forma de resolver sería diagonalizando $T$ y comparando posiciones de los autovectores en la matriz de cambio de base con la forma diagonal de $T$ pero eso podría ser excesivo.

\begin{Rmk}
Otra forma de preguntar sería: Si $\vec{v}=(-2,1)$ es un autovector de $T$, ¿a cuál autovalor de $T$ está asociado?
\end{Rmk}

\begin{Ej}
  Considere la T.L. $T(x,y,z)=(x-y+z,-x+y+z,x)$. Si $\vec{v}=(2,-1,1)$ es un autovector de $T$, ¿bajo cuál factor reescala $T$ a $\vec v$?
\end{Ej}
%-1, 1, 2	[1 -1  1; -1  1  1;  1  0  0]

Evaluando $T$ en tal vector vemos que vale la ecuación $T\vec v=2\vec v$.\vspace*{1em}

\textbf{Tercera idea:} Usar propiedades de linealidad para averiguar imágenes de una T.L.

\begin{Ej}
  Considere una T.L. que cumple $T(2,1)=(6,3)$. ¿Cuál es el valor de $T(4044,2022)$?
\end{Ej}

Como $(2,1)$ es un autovector de $T$ asociado al autovalor $\la=3$. En consecuencia 
$$T(4044,2022)=2022\.T(2,1)=2022\.3(2,1)=(12132,6066).$$

\begin{Rmk}
  Ni siquiera es necesario saber que $(2,1)$ es autovector para resolver esta pregunta. Basta con recordar que si $T$ es lineal entonces $T(c\vec{v})=cT\vec{v}$.
\end{Rmk}

\textbf{Cuarta idea:} Usar propiedades de autovalores para encontrar imágenes de una T.L. 

Por ejemplo si $A\vec{v}=\la \vec{v}$ entonces $A^{-1}\vec{v}=\frac{1}{\la}\vec{v}$.

\begin{Ej}
  Considere una T.L. $T$ que cumple que $T(\vec{v})=3\vec{v}$. Si $T$ es invertible, para cuál valor de $k$ vale que $T^{-1}(\vec{v})=k\vec{v}$? (\emph{Sugerencia: $k$ no necesariamente es un valor entero.})
\end{Ej}

Alternativamente el enunciado podría ser $T(\vec{v})=\frac{1}{3}\vec{v}$ para que así el valor de $k$ sí sea entero.

\begin{Ej}
  Considere una T.L. $T$ que cumple que $T(\vec{v})=4\vec{v}$. Entonces, ¿para cuál valor de $k$ se cumple que $T^2(\vec{v})=k\vec{v}$?
\end{Ej}

Aquí como $T$ tiene autovalor $\la=4$ entonces $T^2$ tiene como autovalor a $4^2=16$. Entonces $k=16$ satisface lo pedido.
\newpage
\subsection*{Desarrollo}

\begin{Ej}
  Considere la matriz 
  $$A=\begin{pmatrix}
    2&4&6&8\\1&3&0&5\\1&1&6&3
  \end{pmatrix}.$$
\begin{enumerate}
  \item Encuentre una base para $\ker(A)$.
  \item Encuentre una base para el espacio de filas de $A$.
  \item Encuentre una base para $\Im(A)$ que sólo contenga columnas de $A$.
  \item Para cada vector columna que no pertenece a la base de la imagen, expréselo como combinación lineal de la base.
\end{enumerate}
\end{Ej}

\begin{enumerate}
  \item Reducimos $A$ a $R=\begin{pmatrix}
    1&0&9&2\\0&1&-3&1\\0&0&0&0
  \end{pmatrix}$.
  Las ecuaciones homogéneas asociadas a esta matriz son 
  $$
  \left\lbrace\begin{aligned}
    &x_1+9x_3+2x_4=0\To x_1=-9x_3-2x_4\\
    &x_2-3x_3+x_4=0\To x_2=3x_3-x_4
  \end{aligned}\right.
  $$
  por lo que la solución es 
  $$(x_1,x_2,x_3,x_4)=(-9x_3-2x_4,3x_3-x_4,x_3,x_4)=(-9,3,1,0)x_3+(-2,-1,0,1)x_4.$$
  Los vectores $\blu{\vec{v}_1=(-9,3,1,0)}$ y $\blu{\vec{v}_2=(-2,-1,0,1)}$ forman una base de $\ker(A)$.
  \item El espacio de filas de $A$ tiene como base a las filas de $R$. Tales vectores son $\vec{u}_1=(1,0,9,2)$ y $\vec{u}_2=(0,1,-3,1)$.
  \item Primero recordemos que $\un{\Rng(A)=\dim(\Im(A))=\dim(\text{esp.cols.}(A))}$. Por el trabajo anterior $\Rng(A)$ es 2 y así sólo dos vectores l.i. generan el espacio de columnas.\par 
  Sin perdida de generalidad tomamos $\vec{u}=(2,1,1)$ y $\vec{v}=(4,3,1)$, que \blu{son l.i. pues no son múltiplos uno del otro}, como base del espacio de columnas.
  \item Las columnas que sobran las escribimos como combinación lineal de $\vec{u}$ y $\vec{v}$ de forma que debemos encontrar constantes $a_1,b_1,a_2,b_2$ tales que 
  $$(6,0,6)=a_1\vec{u}+b_1\vec{v},\ (8,5,3)=a_2\vec{u}+b_2\vec{v}.$$
  Una forma rápida de resolver es ver que las \un{últimas dos columnas de $R$} contienen los coeficientes pedidos: $a_1=9$, $b_1=-3$ y $a_2=2$, $b_2=1$. Alternativamente se pueden resolver los sistemas a mano. 
\end{enumerate}

\begin{Rmk}
  Los items 3 y 4 no tienen solución única. 
\end{Rmk}

\begin{Rmk}
  Este tipo de ejercicio se puede crear elaborando una matriz reducida de rango incompleto en $\bZ$ y luego aplicarle distintas operaciones de fila con enteros para así garantizar que su forma reducida tenga enteros.
\end{Rmk}
\end{document}