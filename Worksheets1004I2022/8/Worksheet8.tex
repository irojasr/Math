%----------------------------------------------------------------------------------------
%	PACKAGES AND OTHER DOCUMENT CONFIGURATIONS
%----------------------------------------------------------------------------------------

\documentclass[12pt]{article}
\usepackage[spanish]{babel} %Tildes
\usepackage[extreme]{savetrees} %Espaciado e interlineado. Comentar si no gusta el interlineado.
\usepackage[utf8]{inputenc} %Encoding para tildes
\usepackage[breakable,skins]{tcolorbox} %Cajitas
\usepackage{fancyhdr} % Se necesita para el título arriba
\usepackage{lastpage} % Se necesita para poner el número de página
\usepackage{amsmath,amsfonts,amssymb,amsthm} %simbolos y demás
\usepackage{mathabx} %más símbolos
\usepackage{physics} %simbolos de derivadas, bra-ket.
\usepackage{multicol}
\usepackage[customcolors]{hf-tikz}
\usepackage[shortlabels]{enumitem}
\usepackage{tikz}

\def\darktheme
%%%%%%%%% === Document Configuration === %%%%%%%%%%%%%%

\pagestyle{fancy}
\setlength{\headheight}{14.49998pt} %NO MODIFICAR
\setlength{\footskip}{14.49998pt} %NO MODIFICAR

\ifx \darktheme\undefined

\lhead{MA1004G8} % Nombre de autor
\chead{\textbf{Lección 0609}} % Titulo
\rhead{}%\firstxmark} 
\lfoot{}%\lastxmark}
\cfoot{}
\rfoot{P\'ag.\ \thepage\ de\ \pageref{LastPage}} %A la derecha saldrá pág. 6 de 9. 
\else
\pagenumbering{gobble}
\pagecolor[rgb]{0,0,0}%{0.23,0.258,0.321}
\color[rgb]{1,1,1}
\fi

%%%%%%%%% === My T Color Box === %%%%%%%%%%%%%%

\ifx \darktheme\undefined
\newtcolorbox{ptcb}{
colframe = black,
colback = white,
breakable,
enhanced
}
\newtcolorbox{ptcbP}{
colframe = black,
colback = white,
coltitle = black,
colbacktitle = black!40,
title = Práctica,
breakable,
enhanced
}

\else
\newtcolorbox{ptcb}{
colframe = white,
colback = black,
colupper = white,
breakable,
enhanced
}
\newtcolorbox{ptcbP}{
colframe = white,
colback = black,
colupper = white,
coltitle = white,
colbacktitle = black,
title = Práctica,
breakable,
enhanced
}
\fi

%%%%%%%%% === Tikz para matrices === %%%%%%%%%%%%%%

\tikzset{
  style green/.style={
    set fill color=green!50!lime!60,
    set border color=white,
  },
  style cyan/.style={
    set fill color=cyan!90!blue!60,
    set border color=white,
  },
  style orange/.style={
    set fill color=orange!80!red!60,
    set border color=white,
  },
  row/.style={
    above left offset={-0.15,0.31},
    below right offset={0.15,-0.125},
    #1
  },
  col/.style={
    above left offset={-0.1,0.3},
    below right offset={0.15,-0.15},
    #1
  }
}

%%%%%%%%% === Theorems and suchlike === %%%%%%%%%%%%%%

\theoremstyle{plain}
\newtheorem{Th}{Teorema}  %%% Theorem 1.1
\newtheorem*{nTh}{Teorema}             %%% No-numbered Theorem
\newtheorem{Prop}[Th]{Proposición}     %%% Proposition 1.2
\newtheorem{Lem}[Th]{Lema}             %%% Lemma 1.3
\newtheorem*{nLem}{Lema}               %%% No-numbered Lemma
\newtheorem{Cor}[Th]{Corolario}        %%% Corollary 1.4
\newtheorem*{nCor}{Corolario}          %%% No-numbered Corollary

\theoremstyle{definition}
\newtheorem*{Def}{Definición}       %%% Definition 1.5
\newtheorem*{nonum-Def}{Definición}    %%% No number Definition
\newtheorem*{nEx}{Ejemplo}             %%% No number Example
\newtheorem{Ex}[Th]{Ejemplo}           %%% Example
\newtheorem{Ej}[Th]{Ejercicio}         %%% Exercise
\newtheorem*{nEj}{Ejercicio}           %%% No number Excercise
\newtheorem*{Not}{Notación}       %%% Definition 1.5

\theoremstyle{remark}
\newtheorem*{Rmk}{Observación}      %%%Remark 1.6

%\numberwithin{equation}{section}

\setlength{\parindent}{3ex}

%%====== Useful macros: =======%%%

\DeclareMathOperator{\gen}{gen}     %%%set generated by...
\DeclareMathOperator{\Rng}{Rng}     %%%rangomat
\DeclareMathOperator{\Nul}{Nul}     %%%rangomat
\DeclareMathOperator{\Proy}{Proy}   %%%proyección

\newcommand{\la}{\lambda}           %%%short for \lambda
\newcommand{\sg}{\sigma}            %%%short for \sigma
\newcommand{\te}{\theta}                %% short for  \theta
\renewcommand{\l}{\ell}

\newcommand{\thickhat}[1]{\mathbf{\hat{\text{$#1$}}}}
\newcommand{\ii}{\vu{\imath}}
\newcommand{\jj}{\vu{\jmath}}
\newcommand{\kk}{\thickhat{k}}

\newcommand{\bC}{\mathbb{C}}        %%%complex numbers
\newcommand{\bN}{\mathbb{N}}        %%%natural numbers
\newcommand{\bP}{\mathbb{P}}        %%%polynomials
\newcommand{\bR}{\mathbb{R}}        %%%real numbers
\newcommand{\bZ}{\mathbb{Z}}        %%%integer numbers
\newcommand{\cB}{\mathcal{B}}       %%%basis
\newcommand{\cC}{\mathcal{C}}       %%%basis
\newcommand{\cM}{\mathcal{M}}       %%%matrix family

\newcommand{\sT}{\mathsf{T}}        %%%traspuesta

\renewcommand{\geq}{\geqslant}      %%%(to save typing)
\renewcommand{\leq}{\leqslant}      %%%(to save typing)
\newcommand{\x}{\times}             %%%product
\renewcommand{\:}{\colon}           %%%colon in  f: A -> B
\newcommand{\isom}{\simeq}              %% isomorfismo

\newcommand{\un}[1]{\underline{#1}}
\newcommand{\half}{\frac12}

\newcommand*{\Cdot}{{\raisebox{-0.25ex}{\scalebox{1.5}{$\cdot$}}}}      %% cdot más grande
\renewcommand{\.}{\Cdot}                %% producto escalar

\newcommand{\twobyone}[2]{\begin{pmatrix} %% 2 x 1 matrix
  #1 \\ #2 \end{pmatrix}}
  \newcommand{\twobytwo}[4]{\begin{pmatrix} %% 2 x 2 matrix
    #1 & #2 \\ #3 & #4 \end{pmatrix}}
    \newcommand{\twobythree}[6]{\begin{pmatrix} %% 2 x 3 matrix
        #1 & #2 & #3\\ #4 & #5 & #6 \end{pmatrix}}
\newcommand{\threebyone}[3]{\begin{pmatrix} %% 3 x 1 matrix
  #1 \\ #2 \\ #3 \end{pmatrix}}
  \newcommand{\threebytwo}[6]{\begin{pmatrix} %% 3 x 1 matrix
    #1 & #2\\ #3 & #4\\ #5&#6 \end{pmatrix}}
\newcommand{\threebythree}[9]{\begin{pmatrix} %% 3 x 3 matrix
  #1 & #2 & #3 \\ #4 & #5 & #6 \\ #7 & #8 & #9 \end{pmatrix}}

\newcommand{\To}{\Rightarrow}

\newcommand{\vaf}{\overrightarrow}

\newcommand{\set}[1]{\{\,#1\,\}}    %% set notation
\newcommand{\Set}[1]{\biggl\{\,#1\,\biggr\}} %% set notation (large)

%----------------------------------------------------------------------------------------
%	ARTICLE CONTENTS
%----------------------------------------------------------------------------------------

\begin{document}

\begin{multicols}{2}
\subsection*{Transformaciones Lineales}

\subsubsection*{Acción sobre la Base}

Primero consideremos una transformación lineal $T$ sobre $\bR^3$. 
\begin{itemize}
    \itemsep=-0.42em
    \item $T$ actúa sobre vectores $\vec v\in\bR^3$.
    \item Cualquier $\vec v=(a,b,c)$ se puede puede expresar como c.l. de la base canónica. Es decir:
    $$\vec{v}=a\ii+b\jj+c\kk.$$
    \item Como $T$ es lineal, entonces 
    $$T(\vec{v})=T(a\ii+b\jj+c\kk)=aT(\ii)+bT(\jj)+cT(\kk).$$
\end{itemize}
Esto quiere decir que $T(\vec{v})$ es c.l. de $\set{T(\ii),T(\jj),T(\kk)}$. Es decir, c.l. de las \emph{imágenes} de la base canónica.

\begin{Ex}
Consideremos la función $T(x,y)=(2x-y,x+2y)$. Esta T.L. va de $\bR^2$ en $\bR^2$. Es un \emph{ejercicio} verificar que $T$ es lineal $(\ast\ast)$. Calculemos la imagen del vector $(4,5)$ de dos formas:
\begin{itemize}
    \itemsep=-0.42em
    \item $T(4,5)=(2(4)-(5),(4)+2(5))=(3,14)$.
    \item $T(\ii)=T(1,0)=(2(1)-(0),(1)+2(0))=(2,1)$ y análogamente $T(\jj)=(-1,2)$. Entonces 
    \begin{align*}
        T(4,5)&=4T(\ii)+5T(\jj)\\
        &=4(2,1)+5(-1,2)\\
        &=(8,4)+(-5,10)=(3,14).
    \end{align*}
\end{itemize}
\end{Ex}

\begin{Rmk}
¡Lo importante de la T.L. es que está totalmente determinada por su acción sobre la base! Es decir, con sólo saber qué hace $T$ con la \emph{base canónica} sabemos qué hace con \emph{cualquier} vector.
\end{Rmk}

\begin{Rmk}
    También notemos que la operación $4(2,1)+5(-1,2)$ es lo mismo que 
        $$(4,5)\twobyone{(2,1)}{(-1,2)}=(4,5)\twobytwo{2}{1}{-1}{2}=\left(\twobytwo{2}{-1}{1}{2}\twobyone{4}{5}\right)^\sT.$$
    Es decir, podemos determinar $T$ como la multiplicación de la matriz $A=\twobytwo{2}{-1}{1}{2}$. Por lo tanto 
    $$T(4,5)=\twobytwo{2}{-1}{1}{2}\twobyone{4}{5}=\twobyone{3}{14}.$$
\end{Rmk}

En general esto vale para cualquier transformación lineal y en cualquier dimensión. Tenemos el siguiente resultado:

\begin{Th}
Si $T$ es una T.L. entonces existe una matriz $A$ de forma que $T(\vec{x})=A\vec{x}$.
\end{Th}

Surge naturalmente la pregunta, ¿cómo representamos $T$ con una matriz? \emph{¡Usamos la base canónica!}

\begin{Ex}
    Consideremos la T.L. $T(x,y,z)=(x-2y+z,2x-y-4z)$, $T$ va de $\bR^3$ en $\bR^2$. Para encontrar la matriz de $T$ primero buscamos la imagen de la base canónica:
    $$
    \left\lbrace
    \begin{aligned}
        &T(\ii)=T(1,0,0)=(1-0+0,2-0-0)=(1,2),\\
        &T(\jj)=(0-2+0,0-1-0)=(-2,-1),\\
        &T(\kk)=(1,-4).
    \end{aligned}
    \right.
    $$
    Así 
    $$T(x,y,z)=x(1,2)+y(-2,-1)+z(1,-4)$$
    y si suponemos que la imagen de $(x,y,z)$ es algún vector $(a,b)$ entonces traducimos esto en un sistema lineal:
    \begin{gather*}
        (a,b)=x(1,2)+y(-2,-1)+z(1,-4)\\
        =(x-2y+z,2x-y-4z)\\
        \To \left\lbrace
        \begin{aligned}
            a=x-2y+z\\
            b=2x-y-4z
        \end{aligned}
        \right.
    \end{gather*}
    Convirtiendo a forma matricial, lo podemos ver como 
    $$\twobyone{a}{b}=\twobythree{1}{-2}{1}{2}{-1}{-4}\threebyone{x}{y}{z}.$$
    Por lo tanto $T(x,y,z)=\twobythree{1}{-2}{1}{2}{-1}{-4}\threebyone{x}{y}{z}$, y entonces concluimos que esa es la matriz de $T$.
\end{Ex}

\begin{Rmk}
    La matriz de $T$ está compuesta por las imágenes de la base canónica tomadas como columnas. Por ejemplo en el caso anterior $T(\ii)$ era $(1,2)$, entonces ese vector es la primera columna de $A$.
\end{Rmk}

Denotaremos a la matriz de una transformación lineal $A$, $A_T$ ó sólo $T$. Otra notación más \emph{sugestiva} de lo que haremos más adelante es $[T]_\cC$.
\begin{Ex}
    Supongamos que $T:\bR^2\to\bR^3$ es una T.L. que cumple 
    $$
    \left\lbrace
    \begin{aligned}
        &T(\ii)=(-1,3,8),\\
        &T(\jj)=(0,-2,7).\\
    \end{aligned}
    \right.
    $$
    Si queremos encontrar la imagen de $(6,-2)$ entonces formamos la matriz y multiplicamos. En este caso:
    \begin{gather*}
        [T]_\cC=\threebytwo{-1}{0}{3}{-2}{8}{7}\\
        \To T(6,-2)=\threebytwo{-1}{0}{3}{-2}{8}{7}\twobyone{6}{-2}=\threebyone{-6}{22}{34}.
    \end{gather*}
    Por lo tanto $T(6,-2)=(-6,22,34)$.
\end{Ex}
Como las T.L.'s están determinadas por su acción sobre la base, entonces podemos ver cómo actúa $T$ sobre los vectores por medio de su acción sobre $\ii,\jj$.
\subsection*{T.L.'s en el plano (en $\bR^2$)}
Para guiarnos, tomaremos vectores cuyas puntas estén en la imagen a continuación:
\begin{multicols*}{2}
    \begin{minipage}{0.2\textwidth}
            
\tikzset{every picture/.style={line width=0.75pt}} %set default line width to 0.75pt        

\begin{tikzpicture}[x=0.75pt,y=0.75pt,yscale=-1,xscale=1]
%uncomment if require: \path (0,231); %set diagram left start at 0, and has height of 231

%Image [id:dp13455761886309237] 
\draw (170,110) node  {\includegraphics[width=30pt,height=30pt]{identity.png}};
%Shape: Axis 2D [id:dp6896642177197074] 
\draw [color={rgb, 255:red, 0; green, 93; blue, 164 }  ,draw opacity=1 ] (120,150) -- (220,150)(130,60) -- (130,160) (213,145) -- (220,150) -- (213,155) (125,67) -- (130,60) -- (135,67)  ;
%Straight Lines [id:da5422861688648966] 
\draw [color={rgb, 255:red, 0; green, 255; blue, 0 }  ,draw opacity=1 ]   (130,150) -- (158.97,101.71) ;
\draw [shift={(160,100)}, rotate = 120.96] [color={rgb, 255:red, 0; green, 255; blue, 0 }  ,draw opacity=1 ][line width=0.75]    (10.93,-3.29) .. controls (6.95,-1.4) and (3.31,-0.3) .. (0,0) .. controls (3.31,0.3) and (6.95,1.4) .. (10.93,3.29)   ;
%Straight Lines [id:da6669485051601183] 
\draw [color={rgb, 255:red, 255; green, 0; blue, 255 }  ,draw opacity=1 ]   (130,150) -- (178.29,121.03) ;
\draw [shift={(180,120)}, rotate = 149.04] [color={rgb, 255:red, 255; green, 0; blue, 255 }  ,draw opacity=1 ][line width=0.75]    (10.93,-3.29) .. controls (6.95,-1.4) and (3.31,-0.3) .. (0,0) .. controls (3.31,0.3) and (6.95,1.4) .. (10.93,3.29)   ;

% Text Node
\draw (146,121.6) node [anchor=south east] [inner sep=0.75pt]  [font=\scriptsize,color={rgb, 255:red, 0; green, 255; blue, 0 }  ,opacity=1 ]  {$\vec{w}_{1}$};
% Text Node
\draw (157,134.4) node [anchor=north west][inner sep=0.75pt]  [font=\scriptsize,color={rgb, 255:red, 255; green, 0; blue, 255 }  ,opacity=1 ]  {$\vec{w}_{2}$};


\end{tikzpicture}

        \end{minipage}
        \vfill\null
        \columnbreak
        \begin{minipage}{0.2\textwidth}
                Los vectores \emph{del cuadro} son aquellos cuyas puntas caen en el cuadro y salen desde el origen.
        \end{minipage}
\end{multicols*}
\subsubsection*{Reescalamiento (Elongar ó acortar)}
El reescalamiento alonga o acorta. ¿Cómo se ve una matriz de un reescalamiento? ¿Adónde van $\ii$ y $\jj$?
\begin{Ex}
    Llamemos $\vec{u}=T\ii$, $\vec{v}=T\jj$, entonces en este caso $\vec{u}=(2,0),\ \vec{v}=(0,3)$. 

    \begin{multicols*}{2}
        \begin{minipage}{0.2\textwidth}   

            \tikzset{every picture/.style={line width=0.75pt}} %set default line width to 0.75pt        

            \begin{tikzpicture}[x=0.75pt,y=0.75pt,yscale=-1,xscale=1]
            %uncomment if require: \path (0,455); %set diagram left start at 0, and has height of 455
            
            %Image [id:dp671723579415217] 
            \draw (157.52,272.72) node  {\includegraphics[width=22.51pt,height=22.51pt]{identity.png}};
            %Shape: Axis 2D [id:dp21317848394516292] 
            \draw [color={rgb, 255:red, 0; green, 93; blue, 164 }  ,draw opacity=1 ] (120,302.77) -- (270.08,302.77)(127.5,100.5) -- (127.5,325.25) (263.08,297.77) -- (270.08,302.77) -- (263.08,307.77) (122.5,107.5) -- (127.5,100.5) -- (132.5,107.5)  ;
            %Straight Lines [id:da6481908103652667] 
            \draw [color={rgb, 255:red, 0; green, 255; blue, 0 }  ,draw opacity=1 ]   (127.5,302.74) -- (148.99,266.93) ;
            \draw [shift={(150.02,265.22)}, rotate = 120.96] [color={rgb, 255:red, 0; green, 255; blue, 0 }  ,draw opacity=1 ][line width=0.75]    (10.93,-3.29) .. controls (6.95,-1.4) and (3.31,-0.3) .. (0,0) .. controls (3.31,0.3) and (6.95,1.4) .. (10.93,3.29)   ;
            %Straight Lines [id:da617275403773422] 
            \draw [color={rgb, 255:red, 255; green, 0; blue, 255 }  ,draw opacity=1 ]   (127.5,302.74) -- (163.31,281.25) ;
            \draw [shift={(165.03,280.22)}, rotate = 149.04] [color={rgb, 255:red, 255; green, 0; blue, 255 }  ,draw opacity=1 ][line width=0.75]    (10.93,-3.29) .. controls (6.95,-1.4) and (3.31,-0.3) .. (0,0) .. controls (3.31,0.3) and (6.95,1.4) .. (10.93,3.29)   ;
            %Image [id:dp3928464358438706] 
            \draw (206.3,182.67) node  {\includegraphics[width=50.65pt,height=67.54pt]{identity.png}};
            %Straight Lines [id:da6070471418080821] 
            \draw [color={rgb, 255:red, 0; green, 255; blue, 0 }  ,draw opacity=0.5 ]   (127.5,302.74) -- (186.76,162) ;
            \draw [shift={(187.54,160.16)}, rotate = 112.83] [color={rgb, 255:red, 0; green, 255; blue, 0 }  ,draw opacity=0.5 ][line width=0.75]    (10.93,-3.29) .. controls (6.95,-1.4) and (3.31,-0.3) .. (0,0) .. controls (3.31,0.3) and (6.95,1.4) .. (10.93,3.29)   ;
            %Straight Lines [id:da5207304117390015] 
            \draw [color={rgb, 255:red, 255; green, 0; blue, 255 }  ,draw opacity=0.5 ]   (127.5,302.77) -- (223.64,206.6) ;
            \draw [shift={(225.06,205.18)}, rotate = 134.99] [color={rgb, 255:red, 255; green, 0; blue, 255 }  ,draw opacity=0.5 ][line width=0.75]    (10.93,-3.29) .. controls (6.95,-1.4) and (3.31,-0.3) .. (0,0) .. controls (3.31,0.3) and (6.95,1.4) .. (10.93,3.29)   ;
            %Straight Lines [id:da1267706466466263] 
            \draw [color={rgb, 255:red, 128; green, 128; blue, 128 }  ,draw opacity=1 ] [dash pattern={on 4.5pt off 4.5pt}]  (187.54,160.16) -- (150.02,265.22) ;
            %Straight Lines [id:da5623826022352811] 
            \draw [color={rgb, 255:red, 128; green, 128; blue, 128 }  ,draw opacity=1 ] [dash pattern={on 4.5pt off 4.5pt}]  (225.06,205.18) -- (165.03,280.22) ;
            
            % Text Node
            \draw (138.76,282.7) node [anchor=south east] [inner sep=0.75pt]  [font=\scriptsize,color={rgb, 255:red, 0; green, 255; blue, 0 }  ,opacity=1 ]  {$\vec{w}_1$};
            % Text Node
            \draw (146.27,292.76) node [anchor=north west][inner sep=0.75pt]  [font=\scriptsize,color={rgb, 255:red, 255; green, 0; blue, 255 }  ,opacity=1 ]  {$\vec{w}_2$};
            % Text Node
            \draw (157.89,230.17) node [anchor=south east] [inner sep=0.75pt]  [font=\scriptsize,color={rgb, 255:red, 0; green, 255; blue, 0 }  ,opacity=1 ]  {$T\vec{w}_1$};
            % Text Node
            \draw (175.54,255.26) node [anchor=north west][inner sep=0.75pt]  [font=\scriptsize,color={rgb, 255:red, 255; green, 0; blue, 255 }  ,opacity=1 ]  {$T\vec{w}_2$};
            
            
            \end{tikzpicture}
            
\end{minipage}
\columnbreak
\begin{minipage}{0.2\textwidth}
    Armando la matriz de $T$ obtenemos 
    $$[T]_\cC=\twobytwo{2}{0}{0}{3}.$$ 
    Y encontramos el \emph{criterio} de $T$ por medio de 
    \begin{align*}
        T(x,y)&=\twobytwo{2}{0}{0}{3}\twobyone{x}{y}\\
        &=(2x,3y).
    \end{align*}
     \end{minipage}
\end{multicols*}
\end{Ex}

\begin{ptcbP}
Si las medidas iniciales del cuadro son $1\x 1$: 
\vspace*{-1em}
\begin{enumerate}
    \itemsep=-0.42em
    \item ¿Cuales son las medidas de la figura nueva? 
    \item ¿Cuál es el área la figura nueva?
    \item ¿Cuánto vale $\det([T]_\cC)$? 
\end{enumerate}
\end{ptcbP}

En general un reescalamiento se representa con una matriz diagonal $\twobytwo{m}{0}{0}{n}$. Su acción sobre $\ii$ y $\jj$ es mandarlos a múltiplos de ellos respectivamente.

\begin{Rmk}
    El vector $(x,y)$ es enviado a $(mx,ny)$. Si un vector es de la forma $(x,0)$ entonces su imagen es $(mx,0)=m(x,0)$. Esto quiere decir que son \emph{paralelos}.
\end{Rmk}

\begin{Def}
Un vector $\vec{v}$ es \un{invariante} bajo una T.L. $T$ si $T(\vec{v})\in\gen(\vec{v})$.
\end{Def}

Por lo anterior los vectores del eje $x$ y los del eje $y$ son invariantes bajo reescalamientos.

\subsubsection*{Proyecciones (Justo como antes)}

\begin{Ex}
    Proyectamos sobre el eje $x$ con la transformación $T(\vec{x})=\Proy_{\ii}(\vec{x})$. ¿Cuál es la matriz de esta T.L.?
    \begin{multicols*}{2}
        %\vspace*{-1.9cm}
        \tikzset{every picture/.style={line width=0.75pt}} %set default line width to 0.75pt        
        \begin{tikzpicture}[x=0.75pt,y=0.75pt,yscale=-1,xscale=1]
            %uncomment if require: \path (0,231); %set diagram left start at 0, and has height of 231
            
            %Image [id:dp748790092233157] 
            \draw (210,110) node  {\includegraphics[width=30pt,height=30pt]{identity.png}};
            %Shape: Axis 2D [id:dp1272083877710768] 
            \draw [color={rgb, 255:red, 0; green, 93; blue, 164 }  ,draw opacity=1 ] (120,150) -- (270,150)(130,60) -- (130,160) (263,145) -- (270,150) -- (263,155) (125,67) -- (130,60) -- (135,67)  ;
            %Straight Lines [id:da18414779753231492] 
            \draw [color={rgb, 255:red, 0; green, 255; blue, 0 }  ,draw opacity=1 ]   (130,150) -- (198.37,101.16) ;
            \draw [shift={(200,100)}, rotate = 144.46] [color={rgb, 255:red, 0; green, 255; blue, 0 }  ,draw opacity=1 ][line width=0.75]    (10.93,-3.29) .. controls (6.95,-1.4) and (3.31,-0.3) .. (0,0) .. controls (3.31,0.3) and (6.95,1.4) .. (10.93,3.29)   ;
            %Straight Lines [id:da6980787203678696] 
            \draw [color={rgb, 255:red, 255; green, 0; blue, 255 }  ,draw opacity=1 ]   (130,150) -- (218.1,120.63) ;
            \draw [shift={(220,120)}, rotate = 161.57] [color={rgb, 255:red, 255; green, 0; blue, 255 }  ,draw opacity=1 ][line width=0.75]    (10.93,-3.29) .. controls (6.95,-1.4) and (3.31,-0.3) .. (0,0) .. controls (3.31,0.3) and (6.95,1.4) .. (10.93,3.29)   ;
            %Straight Lines [id:da38659333197924717] 
            \draw [color={rgb, 255:red, 0; green, 255; blue, 0 }  ,draw opacity=0.5 ]   (130,150) -- (198,150) ;
            \draw [shift={(200,150)}, rotate = 180] [color={rgb, 255:red, 0; green, 255; blue, 0 }  ,draw opacity=0.5 ][line width=0.75]    (10.93,-3.29) .. controls (6.95,-1.4) and (3.31,-0.3) .. (0,0) .. controls (3.31,0.3) and (6.95,1.4) .. (10.93,3.29)   ;
            %Straight Lines [id:da41148795468866584] 
            \draw [color={rgb, 255:red, 255; green, 0; blue, 255 }  ,draw opacity=0.5 ]   (130,150) -- (218,150) ;
            \draw [shift={(220,150)}, rotate = 180] [color={rgb, 255:red, 255; green, 0; blue, 255 }  ,draw opacity=0.5 ][line width=0.75]    (10.93,-3.29) .. controls (6.95,-1.4) and (3.31,-0.3) .. (0,0) .. controls (3.31,0.3) and (6.95,1.4) .. (10.93,3.29)   ;
            %Straight Lines [id:da6113456571645162] 
            \draw [color={rgb, 255:red, 128; green, 128; blue, 128 }  ,draw opacity=0.5 ] [dash pattern={on 4.5pt off 4.5pt}]  (200,100) -- (200,150) ;
            %Straight Lines [id:da11507208885303988] 
            \draw [color={rgb, 255:red, 128; green, 128; blue, 128 }  ,draw opacity=0.5 ] [dash pattern={on 4.5pt off 4.5pt}]  (220,120) -- (220,150) ;
            %Straight Lines [id:da8880270984466043] 
            \draw [color={rgb, 255:red, 243; green, 112; blue, 33 }  ,draw opacity=1 ]   (130,150) -- (130,102) ;
            \draw [shift={(130,100)}, rotate = 90] [color={rgb, 255:red, 243; green, 112; blue, 33 }  ,draw opacity=1 ][line width=0.75]    (10.93,-3.29) .. controls (6.95,-1.4) and (3.31,-0.3) .. (0,0) .. controls (3.31,0.3) and (6.95,1.4) .. (10.93,3.29)   ;
            %Curve Lines [id:da28927712963210794] 
            \draw [color={rgb, 255:red, 255; green, 173; blue, 148 }  ,draw opacity=0.5 ] [dash pattern={on 4.5pt off 4.5pt}]  (130,100) .. controls (115.37,120.64) and (115.33,129.39) .. (128.93,148.51) ;
            \draw [shift={(130,150)}, rotate = 236.93] [color={rgb, 255:red, 255; green, 173; blue, 148 }  ,draw opacity=0.5 ][line width=0.75]    (8.74,-2.63) .. controls (5.56,-1.12) and (2.65,-0.24) .. (0,0) .. controls (2.65,0.24) and (5.56,1.12) .. (8.74,2.63)   ;
            %Shape: Circle [id:dp2834359107479949] 
            \draw  [color={rgb, 255:red, 255; green, 173; blue, 148 }  ,draw opacity=1 ] (126.75,150) .. controls (126.75,148.21) and (128.21,146.75) .. (130,146.75) .. controls (131.79,146.75) and (133.25,148.21) .. (133.25,150) .. controls (133.25,151.79) and (131.79,153.25) .. (130,153.25) .. controls (128.21,153.25) and (126.75,151.79) .. (126.75,150) -- cycle ;
            
            % Text Node
            \draw (163,121.6) node [anchor=south east] [inner sep=0.75pt]  [font=\scriptsize,color={rgb, 255:red, 0; green, 255; blue, 0 }  ,opacity=1 ]  {$\vec{w}_{1}$};
            % Text Node
            \draw (175,131.6) node [anchor=south] [inner sep=0.75pt]  [font=\scriptsize,color={rgb, 255:red, 255; green, 0; blue, 255 }  ,opacity=1 ]  {$\vec{w}_{2}$};
            % Text Node
            \draw (165,153.4) node [anchor=north] [inner sep=0.75pt]  [font=\tiny,color={rgb, 255:red, 0; green, 255; blue, 0 }  ,opacity=1 ]  {$T(\vec{w}_{1})$};
            % Text Node
            \draw (220,153.4) node [anchor=north] [inner sep=0.75pt]  [font=\tiny,color={rgb, 255:red, 255; green, 0; blue, 255 }  ,opacity=1 ]  {$T(\vec{w}_{2})$};
            % Text Node
            \draw (128,100) node [anchor=east] [inner sep=0.75pt]  [color={rgb, 255:red, 243; green, 112; blue, 33 }  ,opacity=1 ]  {$\hat{\jmath }$};
            
            
            \end{tikzpicture}
    \vfill\null
    \columnbreak  
    \vspace*{0.08cm}
        $$
        \left\lbrace
        \begin{aligned}
            &\vec u=\Proy_{\ii}(\ii)=\braket{\ii}{\ii}\ii=\ii,\\
            &\vec v=\Proy_{\ii}(\jj)=\braket{\ii}{\jj}\ii=0.\\
        \end{aligned}
        \right.
        $$
    \end{multicols*}
    \vspace*{-0.9cm}
Así $[T]_\cC=\twobytwo{1}{0}{0}{0}\To T(x,y)=(x,0)$.
\end{Ex}

\begin{ptcbP}
    ¿Qué pasó con el cuadro? ¿Cuál es su área bajo la proyección? ¿Cuánto vale el determinante de $T$ aquí?
\end{ptcbP}
\begin{Rmk}
    Si $b\neq 0$, $(0,b)$ va para cero bajo $\Proy_{\ii}$. También colapsa la cuadrícula en un solo eje. Esta es la idea de una T.L. \un{singular}.
\end{Rmk}

\begin{Def}
    Si $T$ es una $T.L.$, su núcleo es el conjunto $\ker T=\set{\vec{x}:\ T(\vec{x})=0}$.
\end{Def}

\begin{Th}
    Si $A=[T]_\cC$ es la representación matricial de $T$, entonces $\ker T= \ker A$.
\end{Th}

\begin{Ex}
    En este caso $\ker (\Proy_{\ii})=\ker\twobytwo{1}{0}{0}{0}$. El espacio de soluciones está definido por la ecuación $x=0$ de forma que las soluciones son de la forma $(0,y)$. Entonces $\ker (\Proy_{\ii})=\gen(\jj)$.
\end{Ex}

\begin{Def}
    Una T.L. se llama \un{singular} si existe algún vector $\vec{v}$ no nulo que es enviado a cero bajo la T.L.
\end{Def}

\begin{Rmk}
    En el caso de la proyección, como los $(0,y)$ van para cero, entonces es una T.L. singular.
\end{Rmk}

\begin{Def}
    Una T.L. se dice \un{no-singular} si el único vector que manda a cero es el cero. Es decir, si $\ker T=\set{0}$.
\end{Def}
Vale que 
\begin{align*}  
    &T\ \text{es no-singular}\iff\ker[T]_\cC=\set{0}\\
    \iff&\Nul([T]_\cC)=0\iff\Rng[T]_\cC\ \text{es completo}\\
    \iff&\det[T]_\cC\neq 0\iff[T]_\cC\ \text{es invertible}.
\end{align*}

\begin{Def}
    Una T.L. es \un{invertible} si su matriz es invertible. En este caso si $T\vec{x}=A\vec{x}$ entonces $T^{-1}\vec{x}=A^{-1}\vec{x}$.
\end{Def}

\begin{Ex}
    Si $T(x,y)=(2x,3y)$, entonces $\twobytwo{2}{0}{0}{3}^{-1}=\twobytwo{\frac12}{0}{0}{\frac13}$ y entonces $T^{-1}(x,y)=(x/2,y/3)$.
\end{Ex}
\end{multicols}
\end{document}