%----------------------------------------------------------------------------------------
%	PACKAGES AND OTHER DOCUMENT CONFIGURATIONS
%----------------------------------------------------------------------------------------

\documentclass[12pt]{article}
\usepackage[spanish]{babel} %Tildes
\usepackage[extreme]{savetrees} %Espaciado e interlineado. Comentar si no gusta el interlineado.
\usepackage[utf8]{inputenc} %Encoding para tildes
\usepackage[breakable,skins]{tcolorbox} %Cajitas
\usepackage{fancyhdr} % Se necesita para el título arriba
\usepackage{lastpage} % Se necesita para poner el número de página
\usepackage{amsmath,amsfonts,amssymb,amsthm} %simbolos y demás
\usepackage{mathabx} %más símbolos
\usepackage{physics} %simbolos de derivadas, bra-ket.
\usepackage{multicol}
\usepackage[customcolors]{hf-tikz}
\def\darktheme
%%%%%%%%% === Document Configuration === %%%%%%%%%%%%%%

\pagestyle{fancy}
\setlength{\headheight}{14.49998pt} %NO MODIFICAR
\setlength{\footskip}{14.49998pt} %NO MODIFICAR

\ifx \darktheme\undefined

\lhead{MA1004G8} % Nombre de autor
\chead{\textbf{Lección 0421}} % Titulo
\rhead{}%\firstxmark} 
\lfoot{}%\lastxmark}
\cfoot{}
\rfoot{P\'ag.\ \thepage\ de\ \pageref{LastPage}} %A la derecha saldrá pág. 6 de 9. 
\else
\pagenumbering{gobble}
\pagecolor[rgb]{0,0,0}%{0.23,0.258,0.321}
\color[rgb]{1,1,1}
\fi

%%%%%%%%% === My T Color Box === %%%%%%%%%%%%%%

\ifx \darktheme\undefined
\newtcolorbox{ptcb}{
colframe = black,
colback = white,
breakable,
enhanced
}
\newtcolorbox{ptcbP}{
colframe = black,
colback = white,
coltitle = black,
colbacktitle = black!40,
title = Práctica,
breakable,
enhanced
}

\else
\newtcolorbox{ptcb}{
colframe = white,
colback = black,
colupper = white,
breakable,
enhanced
}
\newtcolorbox{ptcbP}{
colframe = white,
colback = black,
colupper = white,
coltitle = white,
colbacktitle = black,
title = Práctica,
breakable,
enhanced
}
\fi

%%%%%%%%% === Tikz para matrices === %%%%%%%%%%%%%%

\tikzset{
  style green/.style={
    set fill color=green!50!lime!60,
    set border color=white,
  },
  style cyan/.style={
    set fill color=cyan!90!blue!60,
    set border color=white,
  },
  style orange/.style={
    set fill color=orange!80!red!60,
    set border color=white,
  },
  row/.style={
    above left offset={-0.15,0.31},
    below right offset={0.15,-0.125},
    #1
  },
  col/.style={
    above left offset={-0.1,0.3},
    below right offset={0.15,-0.15},
    #1
  }
}

%%%%%%%%% === Theorems and suchlike === %%%%%%%%%%%%%%

\theoremstyle{plain}
\newtheorem{Th}{Teorema}  %%% Theorem 1.1
\newtheorem*{nTh}{Teorema}             %%% No-numbered Theorem
\newtheorem{Prop}[Th]{Proposición}     %%% Proposition 1.2
\newtheorem{Lem}[Th]{Lema}             %%% Lemma 1.3
\newtheorem*{nLem}{Lema}               %%% No-numbered Lemma
\newtheorem{Cor}[Th]{Corolario}        %%% Corollary 1.4
\newtheorem*{nCor}{Corolario}          %%% No-numbered Corollary

\theoremstyle{definition}
\newtheorem*{Def}{Definición}       %%% Definition 1.5
\newtheorem*{nonum-Def}{Definición}    %%% No number Definition
\newtheorem*{nEx}{Ejemplo}             %%% No number Example
\newtheorem{Ex}[Th]{Ejemplo}           %%% Example
\newtheorem{Ej}[Th]{Ejercicio}         %%% Exercise
\newtheorem*{nEj}{Ejercicio}           %%% No number Excercise

\theoremstyle{remark}
\newtheorem*{Rmk}{Observación}      %%%Remark 1.6

%\numberwithin{equation}{section}

\setlength{\parindent}{3ex}

%%====== Useful macros: =======%%%

\DeclareMathOperator{\End}{End}     %%%space of endomorphisms
\DeclareMathOperator{\Hom}{Hom}     %%%space of homomorphisms
\DeclareMathOperator{\id}{id}       %%%identity map
\DeclareMathOperator{\gen}{gen}     %%%set generated by...
\DeclareMathOperator{\Rng}{Rng}     %%%rangomat
\DeclareMathOperator{\Nul}{Nul}     %%%rangomat

\newcommand{\la}{\lambda}           %%%short for \lambda
\newcommand{\Om}{\varOmega}         %%%short for \varOmega
\newcommand{\sg}{\sigma}            %%%short for \sigma

\newcommand{\bC}{\mathbb{C}}        %%%complex numbers
\newcommand{\bN}{\mathbb{N}}        %%%natural numbers
\newcommand{\bQ}{\mathbb{Q}}        %%%rational numbers
\newcommand{\bR}{\mathbb{R}}        %%%real numbers
\newcommand{\bS}{\mathbb{S}}        %%%sphere
\newcommand{\bZ}{\mathbb{Z}}        %%%integer numbers
\newcommand{\cA}{\mathcal{A}}       %%%no me puse creativo con esta letra
\newcommand{\cB}{\mathcal{B}}       %%%basis
\newcommand{\cC}{\mathcal{C}}       %%%basis
\newcommand{\cF}{\mathcal{F}}       %%%set family
\newcommand{\cM}{\mathcal{M}}       %%%matrix family
\newcommand{\cP}{\mathcal{P}}       %%%power set
\newcommand{\cR}{\mathcal{R}}       %%%relations
\newcommand{\cU}{\mathcal{U}}       %%%open set family
\newcommand{\mm}{\mathfrak{m}}      %%%measure

\newcommand{\sT}{\mathsf{T}}            %% transpuesta

\renewcommand{\geq}{\geqslant}      %%%(to save typing)
\renewcommand{\leq}{\leqslant}      %%%(to save typing)
\newcommand{\x}{\times}             %%%product
\newcommand{\ox}{\otimes}           %%%tensor product
\renewcommand{\:}{\colon}           %%%colon in  f: A -> B


\newcommand*\quot[2]{{^{\textstyle #1}\big/_{\textstyle #2}}}

\newcommand{\conj}[1]{\left\lbrace#1\right\rbrace} %{...}%
\newcommand{\bonj}[1]{\left\lbrack#1\right\rbrack} %[...]
\newcommand{\obonj}[1]{\left\rbrack#1\right\lbrack} %]...[
\newcommand{\rbonj}[1]{\left\rbrack#1\right\rbrack} %]...]
\newcommand{\lbonj}[1]{\left\lbrack#1\right\lbrack} %[...[

\newcommand{\un}[1]{\underline{#1}}
\newcommand{\half}{\frac12}

\newcommand*{\Cdot}{{\raisebox{-0.25ex}{\scalebox{1.5}{$\cdot$}}}}      %% cdot más grande
\renewcommand{\.}{\Cdot}                %% producto escalar


%----------------------------------------------------------------------------------------
%	ARTICLE CONTENTS
%----------------------------------------------------------------------------------------

\begin{document}
\begin{multicols}{2}

\subsection*{El Producto Matricial}

La condición que garantiza que dos matrices $A,B$ se puedan multiplicar es:
$$\#\text{columnas } A = \#\text{filas }B.$$ 
Si $A$ es $[m\x p]$ y $B$ es $[p\x n]$, entonces las entradas del producto se definen con la fórmula:
$$(AB)_{ij}=\sum_{k=1}^p A_{ik}B_{kj}.$$
Más fácilmente:
$$(AB)_{ij}=\braket{\text{fila }i\ A}{\text{col. }j\ B}.$$
La matriz $AB$ será de tamaño $[m\x n]$.

\begin{Ex}
Consideremos las matrices
$$A=\begin{pmatrix}
    1&-1\\
    3&4\\
    -2&0\\
    0&4
\end{pmatrix},\ B=\begin{pmatrix}
    2&0&-2&8&-5\\
    1&-2&3&9&6
\end{pmatrix}.$$ 
Notemos que:
\begin{itemize}
    \item $A$ es $[4\x 2]$.
    \item $B$ es $[2\x 5]$.
    \begin{itemize}
        \item Vale que $\#\text{cols. } A = \#\text{filas }B$.
        \item ¡Podemos multiplicar $A$ con $B$!
    \end{itemize}
    \item La matriz $AB$ será de tamaño $[4\x 5]$.
\end{itemize}
Si queremos encontrar la entrada $(2,4)$ de $AB$ utilizamos la fila 2 de $A$ y la columna $4$ de $B$:

$$
\left(\begin{array}{cc}
    1&-1\\
    \tikzmarkin[row=style cyan]{r2}3&4\tikzmarkend{r2}\\
    -2&0\\
    0&4
  \end{array}\right)\quad\text{y}\quad
  \left(\begin{array}{ccccc}
    2&0&-2&\tikzmarkin[col=style green]{c4}8&-5\\
    1&-2&3&9\tikzmarkend{c4}&6
  \end{array}\right).
$$
Tomamos el producto punto de estos vectores:
$$\braket{(3,4)}{(8,9)}=3\.8+4\.9=60$$
y por lo tanto $(AB)_{24}=60.$
\end{Ex}

En general podemos acomodar las matrices de la siguiente forma:
$$
\begin{array}{cc|c|c|c|c|c}
    &&2&0&-2&\tikzmarkin[col = style green]{c4d2}8&-5\\
    &&1&-2&3&9&6\\
    \hline1&-1&&&&&\\
    \hline\tikzmarkin[row = style cyan]{r2d2}3&4&&&&(\ast)\tikzmarkend{r2d2}\tikzmarkend{c4d2}&\\
\hline-2&0&&&&&\\
   \hline 0&4&&&&&\\
\end{array}
$$

\begin{ptcbP}
  \begin{itemize}
    \item Encuentre la entrada $(1,3)$ de $AB$ resaltando las filas y columnas correspondientes.
    \item Realice el mismo procedimiento para encontrar la entrada $(3,2)$ de $AB$.
  \end{itemize}
  \end{ptcbP}
%https://tex.stackexchange.com/questions/232516/cannot-compile-matrix-multiplication-example-from-texample-net
%https://tex.stackexchange.com/questions/448843/matrix-multiplication-with-tikz
Para encontrar cualquier entrada, ubicamos la fila y columna correspondiente de las matrices y calculamos su producto punto.



\begin{Ej}
Ya multiplicamos $A$ a la izquierda de $B$. ¿Podemos multiplicar $BA$? Es decir, invirtiendo el orden de multiplicación.
\end{Ej}

\begin{Rmk}
En general si $A\in\cM_{m\x n}$ y $B\in\cM_{n\x m}$ entonces tanto $AB$ como $BA$ existen.
\end{Rmk}

\begin{Ex}
  Consideremos las matrices
$$A=\begin{pmatrix}
    2&0&-2&7\\
    4&-3&0&-1
\end{pmatrix},\ B=\begin{pmatrix}
    0&8\\
    2&-1\\
    -2&2\\
    7&0
\end{pmatrix}.$$ 

Si queremos encontrar $AB$ entonces verificamos primero si podemos multiplicarlas. $A$ es $[2\x 4]$ y $B$ es $[4\x 2]$, en este caso 
$$\#\text{cols. } A = \#\text{filas }B$$ 
entonces podemos calcular $AB$.\par 
De la misma forma si buscamos $BA$, también podemos calcularla porque 
$$\#\text{cols. } B = \#\text{filas }A.$$ 
\end{Ex}

\begin{ptcbP}
  \begin{enumerate}
    \item ¿Qué tamaño tendrán las matrices $AB$ y $BA$?
    \item Trabaje en conjunto con alguien para encontrar las matrices que resultan al multiplicar.
  \end{enumerate}
\end{ptcbP}
\vspace{5cm}
\subsection*{Propiedades}
Supongamos que $A,B$ y $C$ son matrices de tamaños adecuados y $c$ un escalar. El producto matricial goza de las siguientes propiedades:
\begin{enumerate}
  \item Asocia: $ABC=A(BC)=(AB)C$.
  \item Identidad: Hay una matriz $I$ tal que $AI=A=IA$. 
  \item Anulador: Hay una matriz $0$ tal que $A0=0=0A$.
  \item Distribución: $A(B+C)=AB+AC$.
  \item Conmuta con prod. escalar: $cAB=(cA)B=A(cB)$.
  \item Traspuesta de prod.: $(AB)^\sT=B^\sT A^\sT$.
  \item \textbf{NO necesariamente} conmuta.
\end{enumerate}
\newpage
\subsection*{Matrices Invertibles}

\begin{Rmk}
  Todo número real posee un inverso multiplicativo. Considere por ejemplo el número 20220421, existe el número $\frac{1}{20220421}$ que cumple que 
  $$20220421\.\frac{1}{20220421}=1.$$
\end{Rmk}

En el caso de las matrices el 1 es la matriz identidad que mencionamos en las propiedades anteriores.

\begin{Def}
La matriz identidad es la matriz cuadrada cuya diagonal contiene sólo 1's.
\end{Def}

\begin{Ex}
  $I=\begin{pmatrix}
    1&0\\0&1
  \end{pmatrix}$ es la matriz identidad de orden 2 mientras que $I=\begin{pmatrix}
    1&0&0\\0&1&0\\0&0&1
  \end{pmatrix}$ es la identidad de orden 3.
\end{Ex}

\begin{Def}
  Diremos que una matriz $A$ es \un{invertible} si existe una matriz $B$ que cumple 
  $$AB=I=BA.$$
  A esta matriz la denotamos como $A^{-1}$.
\end{Def}

\subsection*{Propiedades}
Si $A,B$ son matrices de tamaño adecuado invertibles y $c$ es un escalar no nulo entonces:
\begin{enumerate}
  \item Escalares: $(cA)^{-1}=\frac{1}{c}A^{-1}$.
  \item Producto: $(AB)^{-1}=B^{-1}A^{-1}$.
  \item Traspuestas: $(A^\sT)^{-1}=(A^{-1})^\sT$.
\end{enumerate}

\begin{Ej}
  Si $A$ y $B$ son invertibles, ¿$A+B$ es invertible?
\end{Ej}

\begin{Th}[Adendo al Tma. Resumen]
Una matriz $A$ es invertible si y sólo si es equivalente por filas a la identidad.
\end{Th}

\begin{Ex}
  Supongamos que $A=\begin{pmatrix}
    a&b\\c&d
  \end{pmatrix}$ entonces la inversa de $A$ es 
  $$A^{-1}=\frac{1}{ad-bc}\begin{pmatrix}
    d&-b\\-c&a
  \end{pmatrix}.$$
\end{Ex}

\begin{ptcbP}
  Verifiquemos en conjunto que en efecto esas matrices son inversas. Luego encuentre las inversas de 
  $$A=\begin{pmatrix}
    1&a\\0&1
  \end{pmatrix},\quad B=\begin{pmatrix}
    a&-b\\b&a
  \end{pmatrix}.$$
\end{ptcbP}
\vfill\null\columnbreak
\subsection*{Encontrar Inversas: Reducción en Paralelo}

Para el caso $[2\x 2]$ es sencillo encontrar la inversa con la fórmula. Para matrices más grandes utilizamos el proceso de reducción por filas.

\begin{Ex}
  Consideremos la matriz
  $$A=\begin{pmatrix}
    3&0&-1\\ 0&6&0\\
    9&0&-2
  \end{pmatrix}.$$
  Encontramos su inversa \emph{reduciendo en paralelo}. Aumentamos la matriz, pero con una identidad a la izquierda y luego reducimos:
$$\left(\begin{array}{ccc|ccc}
  3&0&-1&1&0&0\\ 0&6&0&0&1&0\\
    9&0&-2&0&0&1
\end{array}\right).$$
Veremos que la matriz en cuestión es 
$$A^{-1}=\frac{1}{6}\begin{pmatrix}
  -4&0&2\\0&1&0\\18&0&6
\end{pmatrix}.$$
\end{Ex}
\end{multicols}
\end{document} 