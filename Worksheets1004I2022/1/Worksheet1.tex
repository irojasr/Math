%----------------------------------------------------------------------------------------
%	PACKAGES AND OTHER DOCUMENT CONFIGURATIONS
%----------------------------------------------------------------------------------------

\documentclass[12pt]{article}
\usepackage[spanish]{babel} %Tildes
\usepackage[extreme]{savetrees} %Espaciado e interlineado. Comentar si no gusta el interlineado.
\usepackage[utf8]{inputenc} %Encoding para tildes
\usepackage[breakable,skins]{tcolorbox} %Cajitas
\usepackage{fancyhdr} % Se necesita para el título arriba
\usepackage{lastpage} % Se necesita para poner el número de página
\usepackage{amsmath,amsfonts,amssymb,amsthm} %simbolos y demás
\usepackage{mathabx} %más símbolos
\usepackage{physics} %simbolos de derivadas, bra-ket.
\usepackage{multicol}
\def\darktheme
%%%%%%%%% === Document Configuration === %%%%%%%%%%%%%%

\pagestyle{fancy}
\setlength{\headheight}{14.49998pt} %NO MODIFICAR
\setlength{\footskip}{14.49998pt} %NO MODIFICAR

\ifx \darktheme\undefined

\lhead{MA1004G8} % Nombre de autor
\chead{\textbf{Lección 0407}} % Titulo
\rhead{}%\firstxmark} 
\lfoot{}%\lastxmark}
\cfoot{}
\rfoot{P\'ag.\ \thepage\ de\ \pageref{LastPage}} %A la derecha saldrá pág. 6 de 9. 
\else
\pagenumbering{gobble}
\pagecolor[rgb]{0,0,0}%{0.23,0.258,0.321}
\color[rgb]{1,1,1}
\fi

%%%%%%%%% === My T Color Box === %%%%%%%%%%%%%%

\ifx \darktheme\undefined
\newtcolorbox{ptcb}{
colframe = black,
colback = white,
breakable,
enhanced
}
\newtcolorbox{ptcbP}{
colframe = black,
colback = white,
coltitle = black,
colbacktitle = black!40,
title = Práctica,
breakable,
enhanced
}

\else
\newtcolorbox{ptcb}{
colframe = white,
colback = black,
colupper = white,
breakable,
enhanced
}
\newtcolorbox{ptcbP}{
colframe = white,
colback = black,
colupper = white,
coltitle = white,
colbacktitle = black,
title = Práctica,
breakable,
enhanced
}
\fi
%%%%%%%%% === Theorems and suchlike === %%%%%%%%%%%%%%

\theoremstyle{plain}
\newtheorem{Th}{Teorema}  %%% Theorem 1.1
\newtheorem*{nTh}{Teorema}             %%% No-numbered Theorem
\newtheorem{Prop}[Th]{Proposición}     %%% Proposition 1.2
\newtheorem{Lem}[Th]{Lema}             %%% Lemma 1.3
\newtheorem*{nLem}{Lema}               %%% No-numbered Lemma
\newtheorem{Cor}[Th]{Corolario}        %%% Corollary 1.4
\newtheorem*{nCor}{Corolario}          %%% No-numbered Corollary

\theoremstyle{definition}
\newtheorem*{Def}{Definición}       %%% Definition 1.5
\newtheorem*{nonum-Def}{Definición}    %%% No number Definition
\newtheorem*{nEx}{Ejemplo}             %%% No number Example
\newtheorem{Ex}[Th]{Ejemplo}           %%% Example
\newtheorem{Ej}[Th]{Ejercicio}         %%% Exercise
\newtheorem*{nEj}{Ejercicio}           %%% No number Excercise

\theoremstyle{remark}
\newtheorem*{Rmk}{Observación}      %%%Remark 1.6

%\numberwithin{equation}{section}

\setlength{\parindent}{3ex}

%%====== Useful macros: =======%%%

\DeclareMathOperator{\End}{End}     %%%space of endomorphisms
\DeclareMathOperator{\Hom}{Hom}     %%%space of homomorphisms
\DeclareMathOperator{\id}{id}       %%%identity map
\DeclareMathOperator{\gen}{gen}     %%%set generated by...
\DeclareMathOperator{\Rng}{Rng}     %%%rangomat
\DeclareMathOperator{\Nul}{Nul}     %%%rangomat

\newcommand{\la}{\lambda}           %%%short for \lambda
\newcommand{\Om}{\varOmega}         %%%short for \varOmega
\newcommand{\sg}{\sigma}            %%%short for \sigma

\newcommand{\bC}{\mathbb{C}}        %%%complex numbers
\newcommand{\bN}{\mathbb{N}}        %%%natural numbers
\newcommand{\bQ}{\mathbb{Q}}        %%%rational numbers
\newcommand{\bR}{\mathbb{R}}        %%%real numbers
\newcommand{\bS}{\mathbb{S}}        %%%sphere
\newcommand{\bZ}{\mathbb{Z}}        %%%integer numbers
\newcommand{\cA}{\mathcal{A}}       %%%no me puse creativo con esta letra
\newcommand{\cB}{\mathcal{B}}       %%%basis
\newcommand{\cC}{\mathcal{C}}       %%%basis
\newcommand{\cF}{\mathcal{F}}       %%%set family
\newcommand{\cP}{\mathcal{P}}       %%%power set
\newcommand{\cR}{\mathcal{R}}       %%%relations
\newcommand{\cU}{\mathcal{U}}       %%%open set family
\newcommand{\mm}{\mathfrak{m}}      %%%measure

\renewcommand{\geq}{\geqslant}      %%%(to save typing)
\renewcommand{\leq}{\leqslant}      %%%(to save typing)
\newcommand{\ox}{\otimes}           %%%tensor product
\renewcommand{\:}{\colon}           %%%colon in  f: A -> B


\newcommand*\quot[2]{{^{\textstyle #1}\big/_{\textstyle #2}}}

\newcommand{\conj}[1]{\left\lbrace#1\right\rbrace} %{...}%
\newcommand{\bonj}[1]{\left\lbrack#1\right\rbrack} %[...]
\newcommand{\obonj}[1]{\left\rbrack#1\right\lbrack} %]...[
\newcommand{\rbonj}[1]{\left\rbrack#1\right\rbrack} %]...]
\newcommand{\lbonj}[1]{\left\lbrack#1\right\lbrack} %[...[

\newcommand{\un}[1]{\underline{#1}}
\newcommand{\half}{\frac12}




%----------------------------------------------------------------------------------------
%	ARTICLE CONTENTS
%----------------------------------------------------------------------------------------

\begin{document}
\begin{multicols}{2}
\textbf{¡Atención a la notación!}
\begin{itemize}
  \item Intercambio: Intercambia filas.
  \begin{itemize}
    \item $F_i=F_j,\ F_i\leftrightarrow F_j,\ f_i\leftrightarrow f_j$.
  \end{itemize}
  \item Reescalamiento: Multiplica filas por constantes.
  \begin{itemize}
    \item $cF_i,\ F_i=cF_i, cf_i$.
  \end{itemize}
  \item Combinación: Sumar múltiplos de filas.
  \begin{itemize}
    \item $F_i+cF_j, F_i=F_i+cF_j, F_i\leftarrow F_i+cF_j$.
  \end{itemize}
\end{itemize}
\begin{Rmk}
  En el libro de Grossmann se utiliza $R_i$ para denotar la fila $i$-ésima.
\end{Rmk}

\begin{Def}
Una matriz está en \un{forma escalonada} si:
\begin{enumerate}
  \item Las filas nulas están lo \textbf{más abajo posible}.
  \item La $1^{\text{era}}$ entrada no cero de cada fila \textbf{está a la derecha} de la $1^{\text{era}}$ entrada no cero de la fila anterior.
\end{enumerate}
\end{Def}

\begin{Ex}
Consideremos la matriz 

$$A=\begin{pmatrix}
  1&0&1&-2&0\\ 0&0&0&-1&3\\ 0&0&2&4&-5
\end{pmatrix}.$$

 Veamos si A está en forma escalonada.\par 
 
\begin{itemize}
  \item[C1] La condición 1 de la definición se cumple porque no hay filas nulas.
  \item[C2] Las $1^{\text{eras}}$ entradas no cero de cada fila son $1,-1$ y $2$. El $-1$ de la segunda fila sí está a la \textbf{derecha} del 1. Pero el $2$ está a la \textbf{izquierda} del -1
\end{itemize}

La condición 2 no se cumple entonces $A$ no está en forma escalonada. Pero, ¿podemos convertirla? ¡Sí! Con una operación de fila. Observemos que al aplicar un \textbf{intercambio} obtenemos:

$$\begin{pmatrix}
  1&0&1&-2&0\\ 0&0&0&-1&3\\ 0&0&2&4&-5
\end{pmatrix}\xrightarrow[]{F_2\leftrightarrow F_3}\begin{pmatrix}
  1&0&1&-2&0\\ 0&0&2&4&-5\\ 0&0&0&-1&3
\end{pmatrix}.$$
La matriz $A$ y la nueva matriz \textbf{no son la misma}. Sin embargo está nueva matriz \textbf{sí} está en forma escalonada.
\end{Ex}

\begin{ptcbP}
Considere la matriz $B=\begin{pmatrix}
  1&2\\1&3\\0&0\\0&1
\end{pmatrix}$. ¿Está en forma escalonada? Si no, ¡conviértala con operaciones de fila!\par Como \textbf{sugerencia} debe usar la operación de \textbf{combinación}. 
\end{ptcbP}
\vspace{5cm}

\begin{Def}
  Una matriz está en \un{forma escalonada} \un{reducida} (\textbf{FER}) si:
  \begin{enumerate}
    \item Está en \textbf{forma escalonada}.
    \item La $1^{\text{era}}$ entrada de cada fila es un 1.
    \item En cada \textbf{columna}, a los unos sólo \textbf{los acompañan ceros}.
  \end{enumerate}
\end{Def}

\begin{Ex}
  Considere la última matriz que obtuvimos del ejemplo anterior.
  $$\begin{pmatrix}
    1&0&1&-2&0\\ 0&0&2&4&-5\\ 0&0&0&-1&3
  \end{pmatrix}$$
  ¿está en \textbf{forma escalonada reducida}?
  \begin{enumerate}
    \item[C1] Sí está en forma escalonada.
    \item[C2] Las $1^{\text{eras}}$ entradas no cero de cada fila son $1,2$ y $-1$. Entonces sólo la primera fila lo cumple.
  \end{enumerate}
  Remediamos esto aplicando dos \textbf{reescalamientos}:
  \begin{align*}
    \begin{pmatrix}
      1&0&1&-2&0\\ 0&0&2&4&-5\\ 0&0&0&-1&3
    \end{pmatrix}&\xrightarrow[]{\half F_2}\begin{pmatrix}
      1&0&1&-2&0\\ 0&0&1&2&\frac{-5}{2}\\ 0&0&0&-1&3\\
    \end{pmatrix}\\
      &\xrightarrow[]{-1 F_3}\begin{pmatrix}
        1&0&1&-2&0\\ 0&0&1&2&\frac{-5}{2}\\ 0&0&0&1&-3\\
    \end{pmatrix}
  \end{align*}
  Con esto ya vale la condición 2.\par Sin embargo la condición 3 todavía no se cumple. El 1 de la segunda fila tiene un 1 en cima y el de la tercera un $-2$ y un $2$. Para deshacernos de ellos aplicamos una \textbf{combinación}.
  \begin{align*}
    \begin{pmatrix}
        1&0&1&-2&0\\ 0&0&1&2&\frac{-5}{2}\\ 0&0&0&1&-3\\
    \end{pmatrix}&\xrightarrow[]{F_1-F_2} \begin{pmatrix}
      1&0&0&-4&\frac{5}{2}\\ 0&0&1&2&\frac{-5}{2}\\ 0&0&0&1&-3\\
  \end{pmatrix}\\
  &\xrightarrow[]{F_2-2F_3} \begin{pmatrix}
      1&0&0&-4&\frac{5}{2}\\ 0&0&1&0&\frac{7}{2}\\ 0&0&0&1&-3\\
\end{pmatrix}\\
&\xrightarrow[]{F_1+4F_3} \begin{pmatrix}
  1&0&0&0&\frac{-19}{2}\\ 0&0&1&0&\frac{7}{2}\\ 0&0&0&1&-3\\\end{pmatrix}
  \end{align*}
  Se cumple así la condición 3 y está es la forma escalonada reducida de $A$.
\end{Ex}

\begin{Rmk}
  Al decir \emph{la} forma escalonada reducida hacemos alusión a unicidad. Esto es porque \textbf{toda matriz} es equivalente por filas a una \textbf{única} matriz en \textbf{FER}.
\end{Rmk}

\begin{Def}
  El \un{rango} de una matriz es el \textbf{número de filas no nulas} (no cero) de su \textbf{FER}. Denotamos el rango de $A$ como $\Rng(A)$.
\end{Def}

\begin{Rmk}
  En el ejemplo anterior, la matriz no tenía filas nulas por lo que $\Rng(A)=3$.\par 
  Además igual que la cantidad de soluciones, el rango se preserva por medio de operaciones de fila.
\end{Rmk}

\subsection*{El Tma. Resumen de Soluciones de un Sist. Lin.}

\begin{nTh}
Consideremos el sistema $A\vec{x}=\vec{b}$:
\begin{enumerate}
  \item Tiene \textbf{solución única} cuando
  $$ \Rng(A)=\Rng(A\mid\vec b)=\#\text{cols.} A.$$
  \item Tiene \textbf{infinitas soluciones} cuando
  $$ \Rng(A)=\Rng(A\mid\vec b)<\#\text{cols.} A.$$
  \item Es \textbf{inconsistente} cuando 
 $$\Rng(A)<\Rng(A\mid\vec b).$$
\end{enumerate}
\end{nTh}

\begin{Rmk}
  Observemos lo siguiente:
  \begin{itemize}
    \item Según la tercera condición, si $\Rng(A)<\Rng(A\mid\vec b)$, el sistema no tiene solución. En un sistema homogéneo $\vec b=0$ (vector de ceros), y eso no afecta el rango. Vale $\Rng(A)=\Rng(A\mid 0)$ entonces \textbf{no es posible que no tenga solución}. Es decir, \textbf{un sistema homogéneo siempre tiene solución}.
  \end{itemize}
\end{Rmk}

\begin{Def}
  Si $A$ cumple  $\Rng(A)=\#\text{cols.} A$, diremos que $A$ tiene \un{rango completo}.\par 
  Si no, vale $\Rng(A)<\#\text{cols.} A$. En ese caso llamamos \un{nulidad} a la cantidad $\Nul(A)=\#\text{cols.} A-\Rng(A)$.
\end{Def}

\subsection*{Diferencia entre $\Rng(A)$ y $\Rng(A\mid\vec b)$}

\begin{ptcbP}
  Consideremos el sistema de ecuaciones:
  $$
\left\lbrace\begin{aligned}
  &x+4y-7z=4\\
  &2x+7y-17z=-1\\
  &y+3z=2
            \end{aligned}\right.
$$
Encontremos el \textbf{rango} de la matriz de coeficientes y de la matriz aumentada y veamos qué nos dice el teorema resumen.
\end{ptcbP}
\vspace{3cm}



\begin{ptcbP}
  Resolvamos el sistema de ecuaciones del problema estequiométrico de la primera clase:
  $$
  \left\lbrace\begin{aligned}
    &3a-c=0\\
    &8a-2d=0\\
    &2b-2c-d=0
              \end{aligned}\right.
  $$
\end{ptcbP}
\vspace{5cm}

\begin{ptcbP}
  Consideremos el sistema de ecuaciones a continuación:
  $$
  \left\lbrace\begin{aligned}
    &u+v-y=1\\
    &v+2w+x+3y=1\\
    &u-w+x+y=0
              \end{aligned}\right.
  $$
  \begin{enumerate}
    \item ¿Cuál es la matriz asociada al sistema? ¿Y la aumentada?
    \item Reduzca la matriz aumentada para encontrar la solución. ¿Qué nos dice el teorema resumen?
    \item ¿Cuáles variables son libres en la solución? ¿De cuantos parámetros depende la solución? ¿Cuál es la nulidad de la matriz?
  \end{enumerate}
\end{ptcbP}
\end{multicols}
\end{document} 