%----------------------------------------------------------------------------------------
%	PACKAGES AND OTHER DOCUMENT CONFIGURATIONS
%----------------------------------------------------------------------------------------

\documentclass[12pt]{article}
\usepackage[spanish]{babel} %Tildes
\usepackage[extreme]{savetrees} %Espaciado e interlineado. Comentar si no gusta el interlineado.
\usepackage[utf8]{inputenc} %Encoding para tildes
\usepackage[breakable,skins]{tcolorbox} %Cajitas
\usepackage{fancyhdr} % Se necesita para el título arriba
\usepackage{lastpage} % Se necesita para poner el número de página
\usepackage{amsmath,amsfonts,amssymb,amsthm} %simbolos y demás
\usepackage{mathabx} %más símbolos
\usepackage{physics} %simbolos de derivadas, bra-ket.
\usepackage{multicol}
\usepackage[customcolors]{hf-tikz}
\usepackage[shortlabels]{enumitem}
\usepackage{tikz}

\def\darktheme
%%%%%%%%% === Document Configuration === %%%%%%%%%%%%%%

\pagestyle{fancy}
\setlength{\headheight}{14.49998pt} %NO MODIFICAR
\setlength{\footskip}{14.49998pt} %NO MODIFICAR

\ifx \darktheme\undefined

\lhead{MA1004G8} % Nombre de autor
\chead{\textbf{Lección 0602}} % Titulo
\rhead{}%\firstxmark} 
\lfoot{}%\lastxmark}
\cfoot{}
\rfoot{P\'ag.\ \thepage\ de\ \pageref{LastPage}} %A la derecha saldrá pág. 6 de 9. 
\else
\pagenumbering{gobble}
\pagecolor[rgb]{0,0,0}%{0.23,0.258,0.321}
\color[rgb]{1,1,1}
\fi

%%%%%%%%% === My T Color Box === %%%%%%%%%%%%%%

\ifx \darktheme\undefined
\newtcolorbox{ptcb}{
colframe = black,
colback = white,
breakable,
enhanced
}
\newtcolorbox{ptcbP}{
colframe = black,
colback = white,
coltitle = black,
colbacktitle = black!40,
title = Práctica,
breakable,
enhanced
}

\else
\newtcolorbox{ptcb}{
colframe = white,
colback = black,
colupper = white,
breakable,
enhanced
}
\newtcolorbox{ptcbP}{
colframe = white,
colback = black,
colupper = white,
coltitle = white,
colbacktitle = black,
title = Práctica,
breakable,
enhanced
}
\fi

%%%%%%%%% === Tikz para matrices === %%%%%%%%%%%%%%

\tikzset{
  style green/.style={
    set fill color=green!50!lime!60,
    set border color=white,
  },
  style cyan/.style={
    set fill color=cyan!90!blue!60,
    set border color=white,
  },
  style orange/.style={
    set fill color=orange!80!red!60,
    set border color=white,
  },
  row/.style={
    above left offset={-0.15,0.31},
    below right offset={0.15,-0.125},
    #1
  },
  col/.style={
    above left offset={-0.1,0.3},
    below right offset={0.15,-0.15},
    #1
  }
}

%%%%%%%%% === Theorems and suchlike === %%%%%%%%%%%%%%

\theoremstyle{plain}
\newtheorem{Th}{Teorema}  %%% Theorem 1.1
\newtheorem*{nTh}{Teorema}             %%% No-numbered Theorem
\newtheorem{Prop}[Th]{Proposición}     %%% Proposition 1.2
\newtheorem{Lem}[Th]{Lema}             %%% Lemma 1.3
\newtheorem*{nLem}{Lema}               %%% No-numbered Lemma
\newtheorem{Cor}[Th]{Corolario}        %%% Corollary 1.4
\newtheorem*{nCor}{Corolario}          %%% No-numbered Corollary

\theoremstyle{definition}
\newtheorem*{Def}{Definición}       %%% Definition 1.5
\newtheorem*{nonum-Def}{Definición}    %%% No number Definition
\newtheorem*{nEx}{Ejemplo}             %%% No number Example
\newtheorem{Ex}[Th]{Ejemplo}           %%% Example
\newtheorem{Ej}[Th]{Ejercicio}         %%% Exercise
\newtheorem*{nEj}{Ejercicio}           %%% No number Excercise
\newtheorem*{Not}{Notación}       %%% Definition 1.5

\theoremstyle{remark}
\newtheorem*{Rmk}{Observación}      %%%Remark 1.6

%\numberwithin{equation}{section}

\setlength{\parindent}{3ex}

%%====== Useful macros: =======%%%

\DeclareMathOperator{\gen}{gen}     %%%set generated by...
\DeclareMathOperator{\Rng}{Rng}     %%%rangomat
\DeclareMathOperator{\Nul}{Nul}     %%%rangomat
\DeclareMathOperator{\Proy}{Proy}   %%%proyección

\newcommand{\la}{\lambda}           %%%short for \lambda
\newcommand{\sg}{\sigma}            %%%short for \sigma
\newcommand{\te}{\theta}                %% short for  \theta
\renewcommand{\l}{\ell}

\newcommand{\thickhat}[1]{\mathbf{\hat{\text{$#1$}}}}
\newcommand{\ii}{\vu{\imath}}
\newcommand{\jj}{\vu{\jmath}}
\newcommand{\kk}{\thickhat{k}}

\newcommand{\bC}{\mathbb{C}}        %%%complex numbers
\newcommand{\bN}{\mathbb{N}}        %%%natural numbers
\newcommand{\bP}{\mathbb{P}}        %%%polynomials
\newcommand{\bR}{\mathbb{R}}        %%%real numbers
\newcommand{\bZ}{\mathbb{Z}}        %%%integer numbers
\newcommand{\cB}{\mathcal{B}}       %%%basis
\newcommand{\cC}{\mathcal{C}}       %%%basis
\newcommand{\cM}{\mathcal{M}}       %%%matrix family

\newcommand{\sT}{\mathsf{T}}        %%%traspuesta

\renewcommand{\geq}{\geqslant}      %%%(to save typing)
\renewcommand{\leq}{\leqslant}      %%%(to save typing)
\newcommand{\x}{\times}             %%%product
\renewcommand{\:}{\colon}           %%%colon in  f: A -> B
\newcommand{\isom}{\simeq}              %% isomorfismo

\newcommand{\un}[1]{\underline{#1}}
\newcommand{\half}{\frac12}

\newcommand*{\Cdot}{{\raisebox{-0.25ex}{\scalebox{1.5}{$\cdot$}}}}      %% cdot más grande
\renewcommand{\.}{\Cdot}                %% producto escalar

\newcommand{\twobyone}[2]{\begin{pmatrix} %% 2 x 1 matrix
  #1 \\ #2 \end{pmatrix}}
\newcommand{\twobytwo}[4]{\begin{pmatrix} %% 2 x 2 matrix
  #1 & #2 \\ #3 & #4 \end{pmatrix}}
\newcommand{\threebyone}[3]{\begin{pmatrix} %% 3 x 1 matrix
  #1 \\ #2 \\ #3 \end{pmatrix}}
\newcommand{\threebythree}[9]{\begin{pmatrix} %% 3 x 3 matrix
  #1 & #2 & #3 \\ #4 & #5 & #6 \\ #7 & #8 & #9 \end{pmatrix}}

\newcommand{\To}{\Rightarrow}

\newcommand{\vaf}{\overrightarrow}

\newcommand{\set}[1]{\{\,#1\,\}}    %% set notation
\newcommand{\Set}[1]{\biggl\{\,#1\,\biggr\}} %% set notation (large)

%----------------------------------------------------------------------------------------
%	ARTICLE CONTENTS
%----------------------------------------------------------------------------------------

\begin{document}

\begin{multicols}{2}
\subsection*{Ortogonalidad y Proyecciones}

\begin{Def}
  Diremos que dos conjuntos $C,D$ de vectores son \un{ortogonales} si sus vectores lo son pareja por pareja.\par 
  Análogamente dos subespacios son ortogonales si lo son como conjuntos. Denotaremos igual que como con vectores, $C\perp D$ si $C$ y $D$ son ortogonales.
\end{Def}

\begin{Ex}
  Consideremos $V=\gen((0,7,0))$ y $W=\gen((1,0,-1),(0,0,2))$. Entonces $V\perp W$. No tenemos que probar que \emph{todos} los elementos son ortogonales entre si, sino sólo los elementos prototípicos de cada espacio.
  \begin{itemize}
    \itemsep=-0.4em
    \item En $V$ los vectores son de la forma $c(0,7,0)$.
    \item En $W$ se ven como $a(1,0,-1)+b(0,0,2)$.
  \end{itemize}
  Basta verificar entonces que para cualesquiera $a,b,c$ reales:
  $$(0,7c,0)\perp(a,0,2b-a).$$
Pero en efecto
$$\braket{(0,7c,0)}{(a,0,2b-a)}=0\.a+0\.(7c)+0\.(2b-a)=0.$$
Como independiente de $a,b,c$, los vectores son ortogonales, entonces los espacios son ortogonales.
\end{Ex}

\begin{Prop}
  Si $V=\gen(C)$ y $W=\gen(D)$, entonces $V\perp W\iff C\perp D$. 
\end{Prop}

Es decir, con sólo que los generadores sean ortogonales, ya los espacios son ortogonales.

\begin{Def}
  Un conjunto de vectores se dice \un{ortogonal} si todos sus vectores son ortogonales entre si.
\end{Def}

\begin{Ex}
  El conjunto $\cC=\set{\ii,\jj,\kk}$, la base canónica de $\bR^3$ forma un conjunto ortogonal de $\bR^3$.
\end{Ex}

\begin{Ex}
  Si $\cB=\set{(2,2,-1),(2,-1,2),(-1,2,2)}$ es ortogonal. Llamemos $\vec{u},\vec{v}$ y $\vec{w}$ a estos vectores. Observemos que 
  \begin{align*}
    &\braket{\vec{u}}{\vec{v}}=4-2-2=0,\\
    &\braket{\vec{w}}{\vec{v}}=-2-2+4=0.
  \end{align*}
  Claramente $\vec{w}\perp\vec{u}$ también.
\end{Ex}

Pero, hay una diferencia entre este conjunto $\cB$ y la base canónica. ¿Será que $\cB$ también es base? Verificamos montando una matriz cuyas filas son los vectores de $\cB$:

$$A=\threebythree{2}{2}{-1}{2}{-1}{2}{-1}{2}{2}\To\det(A)=-27\neq 0.$$

Para facilitarnos un poco la búsqueda de vectores l.i. usamos el siguiente resultado.

\begin{Prop}
  Dos vectores ortogonales son l.i.
\end{Prop}

No necesariamente vale al contrario, $\vec{u}=(1,1,1)$ y $\vec{v}=(-1,1,-1)$ son l.i. pero no son ortogonales pues $\braket{\vec{u}}{\vec v}=-1\neq 0$.\par 
Como $A$ tiene determinante no cero, tiene rango completo y por tanto sus filas son l.i. y así $\cB$ forma una base de $\bR^3$. Es decir, en términos de que son base y en términos de que son ortogonales no hay diferencia.

\begin{Def}
  Una base es \un{ortogonal} si lo es como conjunto. Pero una base es \un{ortonormal} si ya es ortogonal y todos sus vectores tienen norma 1.
\end{Def}

\begin{Ex}
  En el ejemplo anterior tenemos que $\cB$ forma una base ortogonal, sin embargo
  $$\norm{(2,2,-1)}^2=4+4+1=9\To\norm{(2,2,-1)}=3\neq 1.$$
  Lo mismo vale para $\vec{v}$ y $\vec{w}$. Como los vectores no tienen norma 1, entonces no forman una base ortonormal.
\end{Ex}

Podemos normalizar la base para obtener vectores de norma 1 ahora si. Recuerde que para normalizar, dividimos entre su norma. 

\begin{Ex}
  El conjunto $\hat{\cB}=\set{\hat u,\hat v, \hat w}$ sí forma una base ortonormal. Aquí por ejemplo 
  $$\hat u=\frac{\vec u}{\norm{\vec u}}=\frac{1}{3}(2,2,-1)=\left(\frac23,\frac23,-\frac13\right).$$
  Estos vectores sí tienen norma 1 y por tanto forman una base ortonormal.
\end{Ex}

\begin{Ex}
  La base canónica también forma una base ortonormal. Cada vector canónico tiene norma 1.
\end{Ex}

\begin{ptcbP}
¿Cuándo vale que $\cB=\set{\vec{u},\vec v,\vec u\x\vec v}$ es una base ortogonal de $\bR^3$?
\end{ptcbP}

\begin{ptcbP}
Considere los vectores $(1,-2,0),(0,1,1)$. Encuentre un tercer vector que junto con estos forme una base ortogonal de $\bR^3$.
\end{ptcbP}

\subsection*{Matrices}

Analicemos la matriz que corresponde a una base ortogonal. 

\begin{Ex}
  Veamos la matriz 
  $$A=\threebythree{2}{2}{-1}{2}{-1}{2}{-1}{2}{2}.$$
  Podemos notar que $A$ es simétrica primero que nada y además podemos ver que su inversa es 
  $$A^{-1}=\frac19\threebythree{2}{2}{-1}{2}{-1}{2}{-1}{2}{2}.$$
  Es decir vale que $A^{-1}=\frac{1}{9}A$.\par 
  Usando la propiedad $(cM)^{-1}=\frac1cM^{-1}$ podemos considerar $B=\frac13A$ y de esta manera $(B)^{-1}=B$.
\end{Ex}

Sin embargo este es un caso muy especial. No todas las matrices que corresponden a una base ortogonal cumplen que sus inversas son un múltiplo de ellas mismas.

\begin{ptcbP}
  Si $\cB=\set{(18,9,-6),(9,-6,19),(14,-42,-21)}$, entonces realice lo siguiente:
  \begin{enumerate}
    \itemsep=-0.41em
    \item Verifique que $\cB$ forma una base ortogonal de $\bR^3$.
    \item Encuentre la matriz inversa de $A$, donde $A$ es tomar como filas los vectores de $\cB$.
  \end{enumerate}
\end{ptcbP}

En general con bases ortogonales no podemos esperar mucho de la matriz. Pero podemos hablar de un tipo especial de matrices que sí nos daran bases ortogonales con propiedades especiales.

\begin{Def}
  Una matriz $A$ es \un{ortogonal} si $AA^\sT=I$. Es decir, si $A^{-1}=A^T$.
\end{Def}

\begin{Rmk}
  La matriz del ejemplo anterior es ortogonal y además es simétrica (eso la convierte en un tipo aún más especial de matriz que se llama \emph{involutiva}). La de la práctica no es ninguna de las dos. 
\end{Rmk}

\begin{Ex}
  Las matrices 
  $$R_\te=\twobytwo{\cos(\te)}{\sin(\te)}{-\sin(\te)}{\cos(\te)},\ S_\te=\twobytwo{\cos(\te)}{\sin(\te)}{\sin(\te)}{-\cos(\te)}$$
  son ortogonales para cualquier $\te\in\bR$.\par 
  De hecho \emph{todas} las matrices $[2\x 2]$ ortogonales tienen esa forma.
\end{Ex}

\begin{Prop}
  $$A\ \text{ortogonal}\ \To\ \det(A)=\pm 1.$$
\end{Prop}

Sin embargo el converso no es cierto. La matriz $\twobytwo{3}{0}{0}{\frac13}$ tiene determinante 1 pero no es ortogonal.

\subsection*{Proceso de Gram-Schmidt}

Resta la pregunta, ¿cómo generamos matrices ortogonales a partir de vectores? Por ejemplo con los de la práctica, ¿qué hacemos para que la matriz sea ortogonal? Aplicamos el llamado Proceso de Gram-Schmidt para obtener una base ortonormal.\par 
Para 3 vectores el proceso es el siguiente:
\begin{enumerate}
  \itemsep=-0.42em
  \item Comenzamos con tres vectores $\cB_1=\set{\vec{v}_1,\vec{v}_2,\vec{v}_3}$.
  \item El primer vector nuevo será $\vec{w}_1=\hat{v}_1=\frac{\vec{v}_1}{\norm{\vec{v}_1}}$.
  \item El segundo vector será $\vec{w}_2=\vec{v}_2-\Proy_{\vec{w}_1}\vec{v}_2$. Que es lo mismo que 
  $$\vec{w}_2=\vec{v}_2-\braket{\vec{w}_1}{\vec{v}_2}\vec{w}_1.$$
  Y al final debemos normalizar $\vec{w}_2$.
  \item Finalmente el tercer vector es 
  $$\vec{w}_3=\vec{v}_3-\Proy_{\vec{w}_1}\vec{v}_3-\Proy_{\vec{w}_2}\vec{v}_3.$$
  Y al igual que antes se debe de normalizar al final.
\end{enumerate}
Este proceso determina una base nueva $\cB_2=\set{\vec{w}_1,\vec{w}_2,\vec{w}_3}$ que sí es ortogonal. 

\begin{Rmk}
  No es necesario que los vectores de $\cB_1$ sean ortogonales desde el principio, nada más es necesario que sean l.i.
\end{Rmk}

\begin{Ex}
  Si $\cB_1=\set{(1,2,3),(2,3,1),(3,1,2)}$ es una base y aplicamos Gram-Schmidt entonces:
  \begin{enumerate}
    \itemsep=-0.42em
    \item $\vec{w}_1=\frac{1}{\sqrt{14}}(1,2,3)$.
    \item $\vec{w}_2=\vec{v}_2-\Proy_{\vec{w}_1}\vec{v}_2$ es decir
    \begin{align*}
      \vec{w}_2&=(2,3,1)-\braket{\frac{1}{\sqrt{14}}(1,2,3)}{(2,3,1)}\frac{1}{\sqrt{14}}(1,2,3)\\
      &=(2,3,1)-\frac{11}{14}(1,2,3)=\left(\frac{17}{14},\frac{10}{7},\frac{-19}{14}\right)
    \end{align*}
    Normalizando obtenemos $\vec{w}_2=\left(\frac{17}{5\sqrt{42}},\frac{4}{\sqrt{42}},\frac{-19}{5\sqrt{42}}\right)$.
    \item Finalmente 
    \begin{align*}
      \vec{w}_3&=\vec{v}_3-\Proy_{\vec{w}_1}\vec{v}_3-\Proy_{\vec{w}_2}\vec{v}_3\\
      &=(3,1,2)-\braket{\frac{1}{\sqrt{14}}(1,2,3)}{(3,1,2)}\frac{1}{\sqrt{14}}(1,2,3)\\
      &-\braket{\frac{1}{\sqrt{42}}\left(\frac{17}{5},4,\frac{-19}{5}\right)}{(3,1,2)}\frac{1}{\sqrt{42}}\left(\frac{17}{5},4,\frac{-19}{5}\right).
    \end{align*}
    Normalizando obtenemos $\vec{w_3}=\left(\frac{7}{5\sqrt{3}},\frac{-1}{\sqrt{3}},\frac{1}{5\sqrt{3}}\right)$.
  \end{enumerate}
\end{Ex}
\end{multicols}
\end{document}