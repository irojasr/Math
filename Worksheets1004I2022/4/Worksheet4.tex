%----------------------------------------------------------------------------------------
%	PACKAGES AND OTHER DOCUMENT CONFIGURATIONS
%----------------------------------------------------------------------------------------

\documentclass[12pt]{article}
\usepackage[spanish]{babel} %Tildes
\usepackage[extreme]{savetrees} %Espaciado e interlineado. Comentar si no gusta el interlineado.
\usepackage[utf8]{inputenc} %Encoding para tildes
\usepackage[breakable,skins]{tcolorbox} %Cajitas
\usepackage{fancyhdr} % Se necesita para el título arriba
\usepackage{lastpage} % Se necesita para poner el número de página
\usepackage{amsmath,amsfonts,amssymb,amsthm} %simbolos y demás
\usepackage{mathabx} %más símbolos
\usepackage{physics} %simbolos de derivadas, bra-ket.
\usepackage{multicol}
\usepackage[customcolors]{hf-tikz}
\usepackage[shortlabels]{enumitem}
\usepackage{tikz}

\def\darktheme
%%%%%%%%% === Document Configuration === %%%%%%%%%%%%%%

\pagestyle{fancy}
\setlength{\headheight}{14.49998pt} %NO MODIFICAR
\setlength{\footskip}{14.49998pt} %NO MODIFICAR

\ifx \darktheme\undefined

\lhead{MA1004G8} % Nombre de autor
\chead{\textbf{Lección 0505}} % Titulo
\rhead{}%\firstxmark} 
\lfoot{}%\lastxmark}
\cfoot{}
\rfoot{P\'ag.\ \thepage\ de\ \pageref{LastPage}} %A la derecha saldrá pág. 6 de 9. 
\else
\pagenumbering{gobble}
\pagecolor[rgb]{0,0,0}%{0.23,0.258,0.321}
\color[rgb]{1,1,1}
\fi

%%%%%%%%% === My T Color Box === %%%%%%%%%%%%%%

\ifx \darktheme\undefined
\newtcolorbox{ptcb}{
colframe = black,
colback = white,
breakable,
enhanced
}
\newtcolorbox{ptcbP}{
colframe = black,
colback = white,
coltitle = black,
colbacktitle = black!40,
title = Práctica,
breakable,
enhanced
}

\else
\newtcolorbox{ptcb}{
colframe = white,
colback = black,
colupper = white,
breakable,
enhanced
}
\newtcolorbox{ptcbP}{
colframe = white,
colback = black,
colupper = white,
coltitle = white,
colbacktitle = black,
title = Práctica,
breakable,
enhanced
}
\fi

%%%%%%%%% === Tikz para matrices === %%%%%%%%%%%%%%

\tikzset{
  style green/.style={
    set fill color=green!50!lime!60,
    set border color=white,
  },
  style cyan/.style={
    set fill color=cyan!90!blue!60,
    set border color=white,
  },
  style orange/.style={
    set fill color=orange!80!red!60,
    set border color=white,
  },
  row/.style={
    above left offset={-0.15,0.31},
    below right offset={0.15,-0.125},
    #1
  },
  col/.style={
    above left offset={-0.1,0.3},
    below right offset={0.15,-0.15},
    #1
  }
}

%%%%%%%%% === Theorems and suchlike === %%%%%%%%%%%%%%

\theoremstyle{plain}
\newtheorem{Th}{Teorema}  %%% Theorem 1.1
\newtheorem*{nTh}{Teorema}             %%% No-numbered Theorem
\newtheorem{Prop}[Th]{Proposición}     %%% Proposition 1.2
\newtheorem{Lem}[Th]{Lema}             %%% Lemma 1.3
\newtheorem*{nLem}{Lema}               %%% No-numbered Lemma
\newtheorem{Cor}[Th]{Corolario}        %%% Corollary 1.4
\newtheorem*{nCor}{Corolario}          %%% No-numbered Corollary

\theoremstyle{definition}
\newtheorem*{Def}{Definición}       %%% Definition 1.5
\newtheorem*{nonum-Def}{Definición}    %%% No number Definition
\newtheorem*{nEx}{Ejemplo}             %%% No number Example
\newtheorem{Ex}[Th]{Ejemplo}           %%% Example
\newtheorem{Ej}[Th]{Ejercicio}         %%% Exercise
\newtheorem*{nEj}{Ejercicio}           %%% No number Excercise
\newtheorem*{Not}{Notación}       %%% Definition 1.5

\theoremstyle{remark}
\newtheorem*{Rmk}{Observación}      %%%Remark 1.6

%\numberwithin{equation}{section}

\setlength{\parindent}{3ex}

%%====== Useful macros: =======%%%

\DeclareMathOperator{\gen}{gen}     %%%set generated by...
\DeclareMathOperator{\Rng}{Rng}     %%%rangomat
\DeclareMathOperator{\Nul}{Nul}     %%%rangomat
\DeclareMathOperator{\Proy}{Proy}   %%%proyección

\newcommand{\la}{\lambda}           %%%short for \lambda

\newcommand{\sg}{\sigma}            %%%short for \sigma

\newcommand{\bC}{\mathbb{C}}        %%%complex numbers
\newcommand{\bN}{\mathbb{N}}        %%%natural numbers
\newcommand{\bR}{\mathbb{R}}        %%%real numbers
\newcommand{\bZ}{\mathbb{Z}}        %%%integer numbers
\newcommand{\cB}{\mathcal{B}}       %%%basis
\newcommand{\cC}{\mathcal{C}}       %%%basis
\newcommand{\cM}{\mathcal{M}}       %%%matrix family

\newcommand{\sT}{\mathsf{T}}        %%%traspuesta

\renewcommand{\geq}{\geqslant}      %%%(to save typing)
\renewcommand{\leq}{\leqslant}      %%%(to save typing)
\newcommand{\x}{\times}             %%%product
\renewcommand{\:}{\colon}           %%%colon in  f: A -> B

\newcommand{\un}[1]{\underline{#1}}
\newcommand{\half}{\frac12}

\newcommand*{\Cdot}{{\raisebox{-0.25ex}{\scalebox{1.5}{$\cdot$}}}}      %% cdot más grande
\renewcommand{\.}{\Cdot}                %% producto escalar

\newcommand{\twobyone}[2]{\begin{pmatrix} %% 2 x 1 matrix
  #1 \\ #2 \end{pmatrix}}
\newcommand{\twobytwo}[4]{\begin{pmatrix} %% 2 x 2 matrix
  #1 & #2 \\ #3 & #4 \end{pmatrix}}
\newcommand{\threebyone}[3]{\begin{pmatrix} %% 3 x 1 matrix
  #1 \\ #2 \\ #3 \end{pmatrix}}
\newcommand{\threebythree}[9]{\begin{pmatrix} %% 3 x 3 matrix
  #1 & #2 & #3 \\ #4 & #5 & #6 \\ #7 & #8 & #9 \end{pmatrix}}

\newcommand{\To}{\Rightarrow}

\newcommand{\thickhat}[1]{\mathbf{\hat{\text{$#1$}}}}
\newcommand{\ii}{\vu{\imath}}
\newcommand{\jj}{\vu{\jmath}}
\newcommand{\kk}{\thickhat{k}}

%----------------------------------------------------------------------------------------
%	ARTICLE CONTENTS
%----------------------------------------------------------------------------------------

\begin{document}
\begin{multicols}{2}
\subsection*{Vectores Afines}
\vspace{-0.35em}
\begin{itemize}
  \itemsep=-0.35em
  \item Los vectores afines \un{no salen del origen}.
  \item Si van de $A$ a $B$ denotamos $\overrightarrow{AB}$.
  \item El vector asociado a $\overrightarrow{AB}$ tiene coordenadas $B-A$.
\end{itemize}
\vspace{-1em}
\subsection*{Geometría Vectorial: Álgebra}

La idea clave será interpretar los vectores como \emph{movimiento} desde el origen. En los siguientes ejemplos suponemos que $\vec u =(1,3)$ y $\vec v = (2,1)$.
\vspace{-0.7em}
\begin{Ex}
  Sumamos $\vec u$ con $\vec v$.\vspace{-0.35em}
  \begin{itemize}
    \itemsep=-0.35em
    \item Sumamos ordinariamente $\vec u+\vec v=(3,4)$.
    \item Gráficamente, nos movemos $\vec u$ y luego $\vec v$. El movimiento total es $\vec u + \vec v$.
  
  \end{itemize} 
\begin{center}
  \tikzset{every picture/.style={line width=0.75pt}} %set default line width to 0.75pt        

  \begin{tikzpicture}[x=0.75pt,y=0.75pt,yscale=-1,xscale=1]
    %uncomment if require: \path (0,141); %set diagram left start at 0, and has height of 141
    
    %Shape: Axis 2D [id:dp9513515516892241] 
    \draw [color={rgb, 255:red, 0; green, 93; blue, 164 }  ,draw opacity=1 ] (14,115.32) -- (130,115.32)(37,5) -- (37,130.75) (123,110.32) -- (130,115.32) -- (123,120.32) (32,12) -- (37,5) -- (42,12) (57,110.32) -- (57,120.32)(77,110.32) -- (77,120.32)(97,110.32) -- (97,120.32)(32,95.32) -- (42,95.32)(32,75.32) -- (42,75.32)(32,55.32) -- (42,55.32)(32,35.32) -- (42,35.32) ;
    \draw [color={rgb, 255:red, 0; green, 93; blue, 164 }  ,opacity=1 ]  (64,127.32) node[anchor=east, scale=0.75]{1} (84,127.32) node[anchor=east, scale=0.75]{2} (104,127.32) node[anchor=east, scale=0.75]{3} (34,95.32) node[anchor=east, scale=0.75]{1} (34,75.32) node[anchor=east, scale=0.75]{2} (34,55.32) node[anchor=east, scale=0.75]{3} (34,35.32) node[anchor=east, scale=0.75]{4} ;
    %Straight Lines [id:da6820344485664449] 
    \draw [color={rgb, 255:red, 251; green, 217; blue, 18 }  ,draw opacity=1 ]   (37,115.32) -- (56.37,56.9) ;
    \draw [shift={(57,55)}, rotate = 108.34] [color={rgb, 255:red, 251; green, 217; blue, 18 }  ,draw opacity=1 ][line width=0.75]    (10.93,-3.29) .. controls (6.95,-1.4) and (3.31,-0.3) .. (0,0) .. controls (3.31,0.3) and (6.95,1.4) .. (10.93,3.29)   ;
    %Straight Lines [id:da5656424527402777] 
    \draw [color={rgb, 255:red, 255; green, 224; blue, 106 }  ,draw opacity=1 ] [dash pattern={on 0.84pt off 2.51pt}]  (57,55) -- (57,115) ;
    %Straight Lines [id:da34378724095280044] 
    \draw [color={rgb, 255:red, 185; green, 217; blue, 137 }  ,draw opacity=1 ] [dash pattern={on 0.84pt off 2.51pt}]  (77,95) -- (77,105.6) -- (77,115) ;
    %Straight Lines [id:da02942500229892686] 
    \draw [color={rgb, 255:red, 0; green, 150; blue, 65 }  ,draw opacity=1 ]   (37,115.32) -- (75.22,95.91) ;
    \draw [shift={(77,95)}, rotate = 153.07] [color={rgb, 255:red, 0; green, 150; blue, 65 }  ,draw opacity=1 ][line width=0.75]    (10.93,-3.29) .. controls (6.95,-1.4) and (3.31,-0.3) .. (0,0) .. controls (3.31,0.3) and (6.95,1.4) .. (10.93,3.29)   ;
    %Straight Lines [id:da27964032819131024] 
    \draw [color={rgb, 255:red, 185; green, 217; blue, 137 }  ,draw opacity=1 ] [dash pattern={on 4.5pt off 4.5pt}]  (57,55) -- (95.22,35.59) ;
    \draw [shift={(97,34.68)}, rotate = 153.07] [color={rgb, 255:red, 185; green, 217; blue, 137 }  ,draw opacity=1 ][line width=0.75]    (10.93,-3.29) .. controls (6.95,-1.4) and (3.31,-0.3) .. (0,0) .. controls (3.31,0.3) and (6.95,1.4) .. (10.93,3.29)   ;
    %Straight Lines [id:da6963121827562042] 
    \draw [color={rgb, 255:red, 243; green, 112; blue, 33 }  ,draw opacity=1 ]   (37,115.32) -- (95.81,36.28) ;
    \draw [shift={(97,34.68)}, rotate = 126.65] [color={rgb, 255:red, 243; green, 112; blue, 33 }  ,draw opacity=1 ][line width=0.75]    (10.93,-3.29) .. controls (6.95,-1.4) and (3.31,-0.3) .. (0,0) .. controls (3.31,0.3) and (6.95,1.4) .. (10.93,3.29)   ;
    
    % Text Node
    \draw (57,51.6) node [anchor=south] [inner sep=0.75pt]  [font=\scriptsize,color={rgb, 255:red, 253; green, 185; blue, 18 }  ,opacity=1 ]  {$\vec{u}$};
    % Text Node
    \draw (79,95) node [anchor=west] [inner sep=0.75pt]  [font=\scriptsize,color={rgb, 255:red, 0; green, 134; blue, 65 }  ,opacity=1 ]  {$\vec{v}$};
    % Text Node
    \draw (99,38.08) node [anchor=north west][inner sep=0.75pt]  [font=\scriptsize,color={rgb, 255:red, 243; green, 112; blue, 33 }  ,opacity=1 ]  {$\vec{u} +\vec{v}$};
    % Text Node
    \draw (63.8,33) node [anchor=west] [inner sep=0.75pt]  [font=\scriptsize,color={rgb, 255:red, 185; green, 217; blue, 137 }  ,opacity=1 ]  {$\vec{v}_{\text{af}}$};
    
    
    \end{tikzpicture} 
\end{center}
\end{Ex}
\vspace{-1.5em}
\begin{Ex}
  ¡La resta en realidad es una suma!\vspace{-0.35em}
  \begin{itemize}
    \itemsep=-0.35em
    \item Vale que $\vec u-\vec v=\vec u+(-\vec v)=(-1,2)$.
    \item $-\vec v$ es $\vec v$ reflejado a través del origen. 
  \end{itemize}
  \vspace{-1.8em}
  \begin{center}
    
\tikzset{every picture/.style={line width=0.75pt}} %set default line width to 0.75pt        

\begin{tikzpicture}[x=0.75pt,y=0.75pt,yscale=-1,xscale=1]
%uncomment if require: \path (0,129); %set diagram left start at 0, and has height of 129

%Shape: Axis 2D [id:dp2638385335564226] 
\draw [color={rgb, 255:red, 0; green, 93; blue, 164 }  ,draw opacity=1 ] (10.8,85.02) -- (126.8,85.02)(66,2.5) -- (66,111.05) (119.8,80.02) -- (126.8,85.02) -- (119.8,90.02) (61,9.5) -- (66,2.5) -- (71,9.5) (86,80.02) -- (86,90.02)(106,80.02) -- (106,90.02)(46,80.02) -- (46,90.02)(26,80.02) -- (26,90.02)(61,65.02) -- (71,65.02)(61,45.02) -- (71,45.02)(61,25.02) -- (71,25.02)(61,105.02) -- (71,105.02) ;
\draw [color={rgb, 255:red, 0; green, 93; blue, 164 }  ,opacity=1 ]  (93,97.02) node[anchor=east, scale=0.75]{1} (113,97.02) node[anchor=east, scale=0.75]{2} (53,97.02) node[anchor=east, scale=0.75]{-1} (33,97.02) node[anchor=east, scale=0.75]{-2} (63,65.02) node[anchor=east, scale=0.75]{1} (63,45.02) node[anchor=east, scale=0.75]{2} (63,25.02) node[anchor=east, scale=0.75]{3} (63,105.02) node[anchor=east, scale=0.75]{-1} ;
%Straight Lines [id:da6051203823855107] 
\draw [color={rgb, 255:red, 251; green, 217; blue, 18 }  ,draw opacity=1 ]   (66,85.02) -- (85.37,26.6) ;
\draw [shift={(86,24.7)}, rotate = 108.34] [color={rgb, 255:red, 251; green, 217; blue, 18 }  ,draw opacity=1 ][line width=0.75]    (10.93,-3.29) .. controls (6.95,-1.4) and (3.31,-0.3) .. (0,0) .. controls (3.31,0.3) and (6.95,1.4) .. (10.93,3.29)   ;
%Straight Lines [id:da09085056843889738] 
\draw [color={rgb, 255:red, 255; green, 224; blue, 106 }  ,draw opacity=1 ] [dash pattern={on 0.84pt off 2.51pt}]  (86,24.7) -- (86,85) ;
%Straight Lines [id:da0738895681042433] 
\draw [color={rgb, 255:red, 185; green, 217; blue, 137 }  ,draw opacity=1 ] [dash pattern={on 0.84pt off 2.51pt}]  (106,64.7) -- (106,75.3) -- (106,85) ;
%Straight Lines [id:da17619981276398367] 
\draw [color={rgb, 255:red, 0; green, 134; blue, 65 }  ,draw opacity=1 ]   (66,85.02) -- (104.22,65.61) ;
\draw [shift={(106,64.7)}, rotate = 153.07] [color={rgb, 255:red, 0; green, 134; blue, 65 }  ,draw opacity=1 ][line width=0.75]    (10.93,-3.29) .. controls (6.95,-1.4) and (3.31,-0.3) .. (0,0) .. controls (3.31,0.3) and (6.95,1.4) .. (10.93,3.29)   ;
%Straight Lines [id:da5921644756618589] 
\draw [color={rgb, 255:red, 243; green, 112; blue, 33 }  ,draw opacity=1 ]   (66,85.02) -- (46.89,46.47) ;
\draw [shift={(46,44.68)}, rotate = 63.63] [color={rgb, 255:red, 243; green, 112; blue, 33 }  ,draw opacity=1 ][line width=0.75]    (10.93,-3.29) .. controls (6.95,-1.4) and (3.31,-0.3) .. (0,0) .. controls (3.31,0.3) and (6.95,1.4) .. (10.93,3.29)   ;
%Straight Lines [id:da6972664631960757] 
\draw [color={rgb, 255:red, 248; green, 142; blue, 222 }  ,draw opacity=1 ]   (66,85.02) -- (27.79,104.11) ;
\draw [shift={(26,105)}, rotate = 333.46] [color={rgb, 255:red, 248; green, 142; blue, 222 }  ,draw opacity=1 ][line width=0.75]    (10.93,-3.29) .. controls (6.95,-1.4) and (3.31,-0.3) .. (0,0) .. controls (3.31,0.3) and (6.95,1.4) .. (10.93,3.29)   ;
%Straight Lines [id:da9005291180530461] 
\draw [color={rgb, 255:red, 248; green, 142; blue, 222 }  ,draw opacity=1 ] [dash pattern={on 4.5pt off 4.5pt}]  (86,24.7) -- (47.79,43.79) ;
\draw [shift={(46,44.68)}, rotate = 333.46] [color={rgb, 255:red, 248; green, 142; blue, 222 }  ,draw opacity=1 ][line width=0.75]    (10.93,-3.29) .. controls (6.95,-1.4) and (3.31,-0.3) .. (0,0) .. controls (3.31,0.3) and (6.95,1.4) .. (10.93,3.29)   ;

% Text Node
\draw (86,21.3) node [anchor=south] [inner sep=0.75pt]  [font=\scriptsize,color={rgb, 255:red, 253; green, 185; blue, 18 }  ,opacity=1 ]  {$\vec{u}$};
% Text Node
\draw (108,64.7) node [anchor=west] [inner sep=0.75pt]  [font=\scriptsize,color={rgb, 255:red, 0; green, 134; blue, 65 }  ,opacity=1 ]  {$\vec{v}$};
% Text Node
\draw (44,48.08) node [anchor=north east] [inner sep=0.75pt]  [font=\scriptsize,color={rgb, 255:red, 243; green, 112; blue, 33 }  ,opacity=1 ]  {$\vec{u} -\vec{v}$};
% Text Node
\draw (24,105) node [anchor=east] [inner sep=0.75pt]  [font=\scriptsize,color={rgb, 255:red, 248; green, 142; blue, 222 }  ,opacity=1 ]  {$-\vec{v}$};
% Text Node
\draw (56,34.88) node [anchor=south east] [inner sep=0.75pt]  [font=\scriptsize,color={rgb, 255:red, 248; green, 142; blue, 222 }  ,opacity=1 ]  {$-\vec{v}_{\text{af}}$};


\end{tikzpicture}
\hspace{1cm}
\begin{tikzpicture}[x=0.75pt,y=0.75pt,yscale=-1,xscale=1]
  %uncomment if require: \path (0,170); %set diagram left start at 0, and has height of 170
  
  %Shape: Axis 2D [id:dp4836968436706903] 
  \draw [color={rgb, 255:red, 0; green, 93; blue, 164 }  ,draw opacity=1 ] (9.6,89) -- (105.2,89)(42,13.5) -- (42,154.2) (98.2,84) -- (105.2,89) -- (98.2,94) (37,20.5) -- (42,13.5) -- (47,20.5) (62,84) -- (62,94)(82,84) -- (82,94)(22,84) -- (22,94)(37,69) -- (47,69)(37,49) -- (47,49)(37,29) -- (47,29)(37,109) -- (47,109)(37,129) -- (47,129)(37,149) -- (47,149) ;
  \draw [color={rgb, 255:red, 0; green, 93; blue, 164 }  ,opacity=1 ]  (69,101) node[anchor=east, scale=0.75]{1} (89,101) node[anchor=east, scale=0.75]{2} (29,101) node[anchor=east, scale=0.75]{-1} (39,69) node[anchor=east, scale=0.75]{1} (39,49) node[anchor=east, scale=0.75]{2} (39,29) node[anchor=east, scale=0.75]{3} (39,109) node[anchor=east, scale=0.75]{-1} (39,129) node[anchor=east, scale=0.75]{-2} (39,149) node[anchor=east, scale=0.75]{-3} ;
  %Straight Lines [id:da982558668739484] 
  \draw [color={rgb, 255:red, 251; green, 217; blue, 18 }  ,draw opacity=1 ]   (42,89) -- (61.37,30.9) ;
  \draw [shift={(62,29)}, rotate = 108.43] [color={rgb, 255:red, 251; green, 217; blue, 18 }  ,draw opacity=1 ][line width=0.75]    (10.93,-3.29) .. controls (6.95,-1.4) and (3.31,-0.3) .. (0,0) .. controls (3.31,0.3) and (6.95,1.4) .. (10.93,3.29)   ;
  %Straight Lines [id:da6441278628747766] 
  \draw [color={rgb, 255:red, 255; green, 224; blue, 106 }  ,draw opacity=1 ] [dash pattern={on 0.84pt off 2.51pt}]  (62,29) -- (62,89) ;
  %Straight Lines [id:da249945804740904] 
  \draw [color={rgb, 255:red, 185; green, 217; blue, 137 }  ,draw opacity=1 ] [dash pattern={on 0.84pt off 2.51pt}]  (82,69) -- (82.6,80.9) -- (82,89) ;
  %Straight Lines [id:da8006658963311084] 
  \draw [color={rgb, 255:red, 0; green, 134; blue, 65 }  ,draw opacity=1 ]   (42,89) -- (80.21,69.89) ;
  \draw [shift={(82,69)}, rotate = 153.43] [color={rgb, 255:red, 0; green, 134; blue, 65 }  ,draw opacity=1 ][line width=0.75]    (10.93,-3.29) .. controls (6.95,-1.4) and (3.31,-0.3) .. (0,0) .. controls (3.31,0.3) and (6.95,1.4) .. (10.93,3.29)   ;
  %Straight Lines [id:da19392249110498727] 
  \draw [color={rgb, 255:red, 142; green, 216; blue, 248 }  ,draw opacity=1 ]   (42,89) -- (22.63,147.1) ;
  \draw [shift={(22,149)}, rotate = 288.43] [color={rgb, 255:red, 142; green, 216; blue, 248 }  ,draw opacity=1 ][line width=0.75]    (10.93,-3.29) .. controls (6.95,-1.4) and (3.31,-0.3) .. (0,0) .. controls (3.31,0.3) and (6.95,1.4) .. (10.93,3.29)   ;
  %Straight Lines [id:da6782772918222488] 
  \draw [color={rgb, 255:red, 142; green, 216; blue, 248 }  ,draw opacity=1 ] [dash pattern={on 4.5pt off 4.5pt}]  (82,69) -- (62.63,127.1) ;
  \draw [shift={(62,129)}, rotate = 288.43] [color={rgb, 255:red, 142; green, 216; blue, 248 }  ,draw opacity=1 ][line width=0.75]    (10.93,-3.29) .. controls (6.95,-1.4) and (3.31,-0.3) .. (0,0) .. controls (3.31,0.3) and (6.95,1.4) .. (10.93,3.29)   ;
  %Straight Lines [id:da3785618529333372] 
  \draw [color={rgb, 255:red, 243; green, 112; blue, 33 }  ,draw opacity=1 ]   (42,89) -- (61.11,127.21) ;
  \draw [shift={(62,129)}, rotate = 243.43] [color={rgb, 255:red, 243; green, 112; blue, 33 }  ,draw opacity=1 ][line width=0.75]    (10.93,-3.29) .. controls (6.95,-1.4) and (3.31,-0.3) .. (0,0) .. controls (3.31,0.3) and (6.95,1.4) .. (10.93,3.29)   ;
  
  % Text Node
  \draw (62.6,26.9) node [anchor=south] [inner sep=0.75pt]  [font=\scriptsize,color={rgb, 255:red, 253; green, 185; blue, 18 }  ,opacity=1 ]  {$\vec{u}$};
  % Text Node
  \draw (84,69) node [anchor=west] [inner sep=0.75pt]  [font=\scriptsize,color={rgb, 255:red, 0; green, 134; blue, 65 }  ,opacity=1 ]  {$\vec{v}$};
  % Text Node
  \draw (17.2,130) node [anchor=south] [inner sep=0.75pt]  [font=\scriptsize,color={rgb, 255:red, 142; green, 216; blue, 248 }  ,opacity=1 ]  {$-\vec{u}$};
  % Text Node
  \draw (82.4,132.8) node [anchor=south] [inner sep=0.75pt]  [font=\scriptsize,color={rgb, 255:red, 142; green, 216; blue, 248 }  ,opacity=1 ]  {$-\vec{u}_{\text{af}}$};
  % Text Node
  \draw (75,132.08) node [anchor=north east] [inner sep=0.75pt]  [font=\scriptsize,color={rgb, 255:red, 243; green, 112; blue, 33 }  ,opacity=1 ]  {$\vec{v} -\vec{u}$};
  
  
  \end{tikzpicture}
  
  \end{center}
  \vspace{-2.3em}
  \begin{itemize}
    \itemsep=-0.35em
    \item Análogamente $\vec v-\vec u=\vec v+(-\vec u)=(1,-2)$.
    \item Y aquí reflejamos $\vec u$.
  \end{itemize}
\end{Ex}
\vspace{-0.8em}
En general dos vectores afines generan un paralelogramo, si $\vec u=\vec{AB},\ \vec v=\vec{AC}$ entonces obtenemos un paralelogramo $ACDB$ donde $\vec{AD}=\vec u+\vec v$.
\vspace{-0.8em}
\begin{center}
  
\tikzset{every picture/.style={line width=0.75pt}} %set default line width to 0.75pt           
\begin{tikzpicture}[x=0.75pt,y=0.75pt,yscale=-1,xscale=1]
%uncomment if require: \path (0,92); %set diagram left start at 0, and has height of 92

%Straight Lines [id:da8945069537993959] 
\draw [color={rgb, 255:red, 251; green, 217; blue, 18 }  ,draw opacity=1 ]   (7,86.9) -- (26.37,28.79) ;
\draw [shift={(27,26.9)}, rotate = 108.43] [color={rgb, 255:red, 251; green, 217; blue, 18 }  ,draw opacity=1 ][line width=0.75]    (10.93,-3.29) .. controls (6.95,-1.4) and (3.31,-0.3) .. (0,0) .. controls (3.31,0.3) and (6.95,1.4) .. (10.93,3.29)   ;
%Straight Lines [id:da12014101276865796] 
\draw [color={rgb, 255:red, 0; green, 134; blue, 65 }  ,draw opacity=1 ]   (7,86.9) -- (45.21,67.79) ;
\draw [shift={(47,66.9)}, rotate = 153.43] [color={rgb, 255:red, 0; green, 134; blue, 65 }  ,draw opacity=1 ][line width=0.75]    (10.93,-3.29) .. controls (6.95,-1.4) and (3.31,-0.3) .. (0,0) .. controls (3.31,0.3) and (6.95,1.4) .. (10.93,3.29)   ;
%Straight Lines [id:da46779412682873445] 
\draw [color={rgb, 255:red, 255; green, 224; blue, 106 }  ,draw opacity=1 ] [dash pattern={on 4.5pt off 4.5pt}]  (47,66.9) -- (66.37,8.79) ;
\draw [shift={(67,6.9)}, rotate = 108.43] [color={rgb, 255:red, 255; green, 224; blue, 106 }  ,draw opacity=1 ][line width=0.75]    (10.93,-3.29) .. controls (6.95,-1.4) and (3.31,-0.3) .. (0,0) .. controls (3.31,0.3) and (6.95,1.4) .. (10.93,3.29)   ;
%Straight Lines [id:da8587066767883254] 
\draw [color={rgb, 255:red, 185; green, 217; blue, 137 }  ,draw opacity=1 ] [dash pattern={on 4.5pt off 4.5pt}]  (27,26.9) -- (65.21,7.79) ;
\draw [shift={(67,6.9)}, rotate = 153.43] [color={rgb, 255:red, 185; green, 217; blue, 137 }  ,draw opacity=1 ][line width=0.75]    (10.93,-3.29) .. controls (6.95,-1.4) and (3.31,-0.3) .. (0,0) .. controls (3.31,0.3) and (6.95,1.4) .. (10.93,3.29)   ;
%Straight Lines [id:da7381244976432155] 
\draw [color={rgb, 255:red, 243; green, 112; blue, 33 }  ,draw opacity=1 ] [dash pattern={on 4.5pt off 4.5pt}]  (27,26.9) -- (46.11,65.11) ;
\draw [shift={(47,66.9)}, rotate = 243.43] [color={rgb, 255:red, 243; green, 112; blue, 33 }  ,draw opacity=1 ][line width=0.75]    (10.93,-3.29) .. controls (6.95,-1.4) and (3.31,-0.3) .. (0,0) .. controls (3.31,0.3) and (6.95,1.4) .. (10.93,3.29)   ;
%Straight Lines [id:da33453586069957875] 
\draw [color={rgb, 255:red, 243; green, 112; blue, 33 }  ,draw opacity=1 ]   (7,86.9) -- (65.81,7.86) ;
\draw [shift={(67,6.26)}, rotate = 126.65] [color={rgb, 255:red, 243; green, 112; blue, 33 }  ,draw opacity=1 ][line width=0.75]    (10.93,-3.29) .. controls (6.95,-1.4) and (3.31,-0.3) .. (0,0) .. controls (3.31,0.3) and (6.95,1.4) .. (10.93,3.29)   ;

% Text Node
\draw (14,54) node [anchor=south] [inner sep=0.75pt]  [font=\scriptsize,color={rgb, 255:red, 253; green, 185; blue, 18 }  ,opacity=1 ]  {$\vec{u}$};
% Text Node
\draw (30.2,75.9) node [anchor=north west][inner sep=0.75pt]  [font=\scriptsize,color={rgb, 255:red, 0; green, 134; blue, 65 }  ,opacity=1 ]  {$\vec{v}$};
% Text Node
\draw (51,60.3) node [anchor=south west] [inner sep=0.75pt]  [font=\tiny,color={rgb, 255:red, 243; green, 112; blue, 33 }  ,opacity=1 ]  {$(\vec{v} -\vec{u})_{\text{af}}$};
% Text Node
\draw (68.6,15.5) node [anchor=north west][inner sep=0.75pt]  [font=\scriptsize,color={rgb, 255:red, 243; green, 112; blue, 33 }  ,opacity=1 ]  {$\vec{u} +\vec{v}$};
% Text Node
\draw (5,86.9) node [anchor=east] [inner sep=0.75pt]  [font=\tiny]  {$A$};
% Text Node
\draw (25,23.5) node [anchor=south east] [inner sep=0.75pt]  [font=\tiny]  {$B$};
% Text Node
\draw (49,70.3) node [anchor=north west][inner sep=0.75pt]  [font=\tiny]  {$C$};
% Text Node
\draw (69,6.26) node [anchor=west] [inner sep=0.75pt]  [font=\tiny]  {$D$};


\end{tikzpicture}
\end{center}
\vspace{-0.8em}
\begin{ptcbP}
  Los puntos $(2,1),(4,2)$ y $(3,5)$ forman un paralelogramo con otro punto. ¿Cuál es?\par \emph{Hay más de una respuesta.}
\end{ptcbP}

\begin{Ex}
  Si ahora $\vec{u}=(-2,2)$, entonces:
  \vspace{-0.7em}
  \begin{itemize}
    \itemsep=-0.35em
    \item $2\vec{u}=(-4,4)$ y $-\half\vec{u}=(1,-1)$.
    \item Geometricamente al multiplicar $c\in\bR$ tenemos:\vspace*{-1em}
    \begin{itemize}
      \itemsep=-0.35em
      \item $c>1$: elongamos.
      \item $c<1$: acortamos.
      \item $c$ negativo es cambiar dirección.
    \end{itemize}
  \end{itemize}
  \vspace{-1.5em}
  \begin{center}
    \tikzset{every picture/.style={line width=0.75pt}} %set default line width to 0.75pt        
\begin{tikzpicture}[x=0.75pt,y=0.75pt,yscale=-1,xscale=1]
%uncomment if require: \path (0,148); %set diagram left start at 0, and has height of 148

%Shape: Axis 2D [id:dp8428186049340991] 
\draw [color={rgb, 255:red, 0; green, 93; blue, 164 }  ,draw opacity=1 ] (24.8,108) -- (153.4,108)(117,12.6) -- (117,136) (146.4,103) -- (153.4,108) -- (146.4,113) (112,19.6) -- (117,12.6) -- (122,19.6) (137,103) -- (137,113)(97,103) -- (97,113)(77,103) -- (77,113)(57,103) -- (57,113)(37,103) -- (37,113)(112,88) -- (122,88)(112,68) -- (122,68)(112,48) -- (122,48)(112,28) -- (122,28)(112,128) -- (122,128) ;
\draw [color={rgb, 255:red, 0; green, 93; blue, 164 }  ,opacity=1 ]  (144,120) node[anchor=east, scale=0.75]{1} (104,120) node[anchor=east, scale=0.75]{-1} (84,120) node[anchor=east, scale=0.75]{-2} (64,120) node[anchor=east, scale=0.75]{-3} (44,120) node[anchor=east, scale=0.75]{-4} (114,88) node[anchor=east, scale=0.75]{1} (114,68) node[anchor=east, scale=0.75]{2} (114,48) node[anchor=east, scale=0.75]{3} (114,28) node[anchor=east, scale=0.75]{4} (114,128) node[anchor=east, scale=0.75]{-1} ;
%Straight Lines [id:da9124830915084599] 
\draw [color={rgb, 255:red, 251; green, 217; blue, 18 }  ,draw opacity=1 ]   (117,108) -- (78.41,69.41) ;
\draw [shift={(77,68)}, rotate = 45] [color={rgb, 255:red, 251; green, 217; blue, 18 }  ,draw opacity=1 ][line width=0.75]    (10.93,-3.29) .. controls (6.95,-1.4) and (3.31,-0.3) .. (0,0) .. controls (3.31,0.3) and (6.95,1.4) .. (10.93,3.29)   ;
%Straight Lines [id:da557543717735316] 
\draw [color={rgb, 255:red, 255; green, 224; blue, 106 }  ,draw opacity=1 ] [dash pattern={on 0.84pt off 2.51pt}]  (77,68) -- (77,108) ;
%Straight Lines [id:da07340949304051358] 
\draw [color={rgb, 255:red, 142; green, 216; blue, 248 }  ,draw opacity=1 ]   (77,68) -- (38.41,29.41) ;
\draw [shift={(37,28)}, rotate = 45] [color={rgb, 255:red, 142; green, 216; blue, 248 }  ,draw opacity=1 ][line width=0.75]    (10.93,-3.29) .. controls (6.95,-1.4) and (3.31,-0.3) .. (0,0) .. controls (3.31,0.3) and (6.95,1.4) .. (10.93,3.29)   ;
%Straight Lines [id:da6059091373446317] 
\draw [color={rgb, 255:red, 142; green, 216; blue, 248 }  ,draw opacity=1 ] [dash pattern={on 0.84pt off 2.51pt}]  (77,68) -- (117,108) ;
%Straight Lines [id:da9782028104831733] 
\draw [color={rgb, 255:red, 142; green, 216; blue, 248 }  ,draw opacity=1 ]   (117,108) -- (135.59,126.59) ;
\draw [shift={(137,128)}, rotate = 225] [color={rgb, 255:red, 142; green, 216; blue, 248 }  ,draw opacity=1 ][line width=0.75]    (10.93,-3.29) .. controls (6.95,-1.4) and (3.31,-0.3) .. (0,0) .. controls (3.31,0.3) and (6.95,1.4) .. (10.93,3.29)   ;

% Text Node
\draw (79,64.6) node [anchor=south west] [inner sep=0.75pt]  [font=\scriptsize,color={rgb, 255:red, 253; green, 185; blue, 18 }  ,opacity=1 ]  {$\vec{u}$};
% Text Node
\draw (37,39.4) node [anchor=north] [inner sep=0.75pt]  [font=\scriptsize,color={rgb, 255:red, 142; green, 216; blue, 248 }  ,opacity=1 ]  {$2\vec{u}$};
% Text Node
\draw (139,128) node [anchor=west] [inner sep=0.75pt]  [font=\scriptsize,color={rgb, 255:red, 142; green, 216; blue, 248 }  ,opacity=1 ]  {$-\vec{u}$};


\end{tikzpicture}
  \end{center}
  \vspace{-1.5em}
\end{Ex}

\begin{multicols}{2}
  \textbf{Props. Suma}
  \begin{enumerate}[i)]
    \itemsep=-0.4em
    \item $\vec{x}+\vec{y}=\vec{y}+\vec x$.
    \item {\footnotesize{$(\vec{x}+\vec y)+\vec z=\vec x+(\vec y+\vec z)$.}}
    \item $\vec x+0=\vec x$.
    \item $\vec x+(-\vec x)=0$.
  \end{enumerate}
  \textbf{Props. Mult.}
  \begin{enumerate}[i)]
    \itemsep=-0.4em
    \item $1\vec{x}=\vec x$.
    \item $(cd)\vec x=c(d\vec x)$.
    \item $c(\vec x+\vec y)=c\vec x+c\vec y$.
    \item $(c+d)\vec x=c\vec x+d\vec x$.
  \end{enumerate}
\end{multicols}
\subsection*{Distancia y Ángulos}

Recordemos que $\norm{\vec{u}}$ es la \un{longitud del vector} $\vec u$. Es decir, la distancia entre su origen y su punta.\par Si tenemos $\vec{u}=(1,4)$ y $\vec v=(4,2)$, ¿cuál es la distancia entre sus puntas?
\vspace{-0.6em}
\begin{center}
\tikzset{every picture/.style={line width=0.75pt}} %set default line width to 0.75pt        
\begin{tikzpicture}[x=0.75pt,y=0.75pt,yscale=-1,xscale=1]
%uncomment if require: \path (0,142); %set diagram left start at 0, and has height of 142

%Shape: Axis 2D [id:dp22337725810657827] 
\draw [color={rgb, 255:red, 0; green, 93; blue, 164 }  ,draw opacity=1 ] (8.7,122.42) -- (141.7,122.42)(25.03,6.25) -- (25.03,132) (134.7,117.42) -- (141.7,122.42) -- (134.7,127.42) (20.03,13.25) -- (25.03,6.25) -- (30.03,13.25) (45.03,117.42) -- (45.03,127.42)(65.03,117.42) -- (65.03,127.42)(85.03,117.42) -- (85.03,127.42)(105.03,117.42) -- (105.03,127.42)(125.03,117.42) -- (125.03,127.42)(20.03,102.42) -- (30.03,102.42)(20.03,82.42) -- (30.03,82.42)(20.03,62.42) -- (30.03,62.42)(20.03,42.42) -- (30.03,42.42)(20.03,22.42) -- (30.03,22.42) ;
\draw [color={rgb, 255:red, 0; green, 93; blue, 164 }  ,opacity=1 ]  (52.03,134.42) node[anchor=east, scale=0.75]{1} (72.03,134.42) node[anchor=east, scale=0.75]{2} (92.03,134.42) node[anchor=east, scale=0.75]{3} (112.03,134.42) node[anchor=east, scale=0.75]{4} (132.03,134.42) node[anchor=east, scale=0.75]{5} (22.03,102.42) node[anchor=east, scale=0.75]{1} (22.03,82.42) node[anchor=east, scale=0.75]{2} (22.03,62.42) node[anchor=east, scale=0.75]{3} (22.03,42.42) node[anchor=east, scale=0.75]{4} (22.03,22.42) node[anchor=east, scale=0.75]{5} ;
%Straight Lines [id:da6945787109755741] 
\draw [color={rgb, 255:red, 251; green, 217; blue, 18 }  ,draw opacity=1 ]   (25.03,122.42) -- (44.52,44.39) ;
\draw [shift={(45,42.45)}, rotate = 104.02] [color={rgb, 255:red, 251; green, 217; blue, 18 }  ,draw opacity=1 ][line width=0.75]    (10.93,-3.29) .. controls (6.95,-1.4) and (3.31,-0.3) .. (0,0) .. controls (3.31,0.3) and (6.95,1.4) .. (10.93,3.29)   ;
%Straight Lines [id:da1439928618426234] 
\draw [color={rgb, 255:red, 255; green, 224; blue, 106 }  ,draw opacity=1 ] [dash pattern={on 0.84pt off 2.51pt}]  (45,42.45) -- (45,122.42) ;
%Straight Lines [id:da11637193376664245] 
\draw [color={rgb, 255:red, 185; green, 217; blue, 137 }  ,draw opacity=1 ] [dash pattern={on 0.84pt off 2.51pt}]  (105,82.45) -- (105,93.05) -- (105,122.45) ;
%Straight Lines [id:da5939181601227368] 
\draw [color={rgb, 255:red, 0; green, 150; blue, 65 }  ,draw opacity=1 ]   (25.03,122.42) -- (103.21,83.34) ;
\draw [shift={(105,82.45)}, rotate = 153.44] [color={rgb, 255:red, 0; green, 150; blue, 65 }  ,draw opacity=1 ][line width=0.75]    (10.93,-3.29) .. controls (6.95,-1.4) and (3.31,-0.3) .. (0,0) .. controls (3.31,0.3) and (6.95,1.4) .. (10.93,3.29)   ;
%Straight Lines [id:da10119946122358425] 
\draw [color={rgb, 255:red, 248; green, 142; blue, 222 }  ,draw opacity=1 ] [dash pattern={on 4.5pt off 4.5pt}]  (45,42.45) -- (105,82.45) ;
%Straight Lines [id:da5728432594210906] 
\draw [color={rgb, 255:red, 251; green, 217; blue, 18 }  ,draw opacity=1 ]   (163.03,123.47) -- (183,43.5) ;
%Straight Lines [id:da3334499488577267] 
\draw [color={rgb, 255:red, 0; green, 150; blue, 65 }  ,draw opacity=1 ]   (163.03,123.47) -- (243,83.5) ;
%Straight Lines [id:da2528499531624988] 
\draw [color={rgb, 255:red, 248; green, 142; blue, 222 }  ,draw opacity=1 ] [dash pattern={on 3.75pt off 3pt on 3.75pt off 3pt}]  (183,43.5) -- (243,83.5) ;
%Shape: Arc [id:dp9335803647036491] 
\draw  [draw opacity=0] (170.9,94.51) .. controls (176.82,96.12) and (182.31,99.55) .. (186.44,104.7) .. controls (187.73,106.31) and (188.82,108) .. (189.73,109.76) -- (163.03,123.47) -- cycle ; \draw  [color={rgb, 255:red, 243; green, 112; blue, 33 }  ,draw opacity=1 ] (170.9,94.51) .. controls (176.82,96.12) and (182.31,99.55) .. (186.44,104.7) .. controls (187.73,106.31) and (188.82,108) .. (189.73,109.76) ;  

% Text Node
\draw (45,39.05) node [anchor=south] [inner sep=0.75pt]  [font=\scriptsize,color={rgb, 255:red, 253; green, 185; blue, 18 }  ,opacity=1 ]  {$\vec{u}$};
% Text Node
\draw (107,82.45) node [anchor=west] [inner sep=0.75pt]  [font=\scriptsize,color={rgb, 255:red, 0; green, 134; blue, 65 }  ,opacity=1 ]  {$\vec{v}$};
% Text Node
\draw (77,59.05) node [anchor=south west] [inner sep=0.75pt]    {$\textcolor[rgb]{0.97,0.56,0.87}{\ell ?}$};
% Text Node
\draw (168.51,112.91) node [anchor=south west] [inner sep=0.75pt]  [font=\footnotesize,color={rgb, 255:red, 243; green, 112; blue, 33 }  ,opacity=1 ]  {$\theta ?$};
% Text Node
\draw (171.02,80.08) node [anchor=south east] [inner sep=0.75pt]    {$\textcolor[rgb]{0.99,0.73,0.07}{x}$};
% Text Node
\draw (203.02,106.88) node [anchor=north] [inner sep=0.75pt]  [color={rgb, 255:red, 0; green, 134; blue, 65 }  ,opacity=1 ]  {$y$};
% Text Node
\draw (215,60.1) node [anchor=south west] [inner sep=0.75pt]  [color={rgb, 255:red, 248; green, 142; blue, 222 }  ,opacity=1 ]  {$z?$};


\end{tikzpicture}
\end{center}
\vspace{-0.8em}
Con ley de cosenos \un{$z^2=x^2+y^2-2xy\cos(\theta)$}
podemos buscar $z$, pero encontrar $\theta$ es complicado.\par 
Sin embargo, entre $\vec{u}$ y $\vec{v}$ hay una copia afín de $\vec{v}-\vec{u}$. Entonces:
\begin{align*}
  \ell=\text{dist.}(\vec{u},\vec{v})&=\text{long.}\ \vec{v}-\vec{u}\ \text{(afín)}\\
  &=\text{long.}\ \vec{v}-\vec{u}\\
  &=\norm{\vec{v}-\vec{u}}
\end{align*}
\begin{Def}
La \un{distancia} entre $\vec{u},\vec{v}$ es $d(\vec{u},\vec{v})=\norm{\vec v-\vec u}$.
\end{Def}
\begin{multicols}{2}
  \textbf{Propiedades:}
\vspace{-0.5em}
\begin{enumerate}[i)]
  \itemsep=-0.35em
  \item $d(\vec{u},\vec v)=0$ si $\vec{u}=\vec v$.
  \item $d(\vec{u},\vec v)=d(\vec{v},\vec u)$.
\end{enumerate}
\columnbreak
\textbf{Props. Norma:}
\vspace{-0.8em}
\begin{enumerate}[i)]
  \itemsep=-0.35em
  \item $\norm{\vec u}=0\iff\vec{u}=0$.
  \item $\norm{\vec v-\vec u}=\norm{\vec u-\vec v}$.
  \item $\norm{c\vec u}=|c|\norm{\vec u}$.
\end{enumerate}
\end{multicols}
En el ejemplo anterior:
$$\vec{v}-\vec{u}=(3,-2)\To\norm{\vec{v}-\vec{u}}^2=3^2+(-2)^2=13$$
y así $d(\vec{u},\vec{v})=\norm{\vec{v}-\vec{u}}=\sqrt{13}$.
\newpage
Queda la espina de resolver utilizando ley de cosenos. Tenemos la siguiente fórmula:

\begin{Prop}
Para $\vec{u},\vec v$, vale: \un{$\braket{\vec{u}}{\vec{v}}=\norm{\vec{u}}\norm{\vec v}\cos(\theta)$}.
\end{Prop}
\vspace*{-0.3em}
Podemos despejar para obtener \un{$\theta=\arccos\left(\frac{\braket{\vec{u}}{\vec{v}}}{\norm{\vec{u}}\norm{\vec v}}\right)$}.

\begin{Def}
  Si el ángulo entre $\vec{u}$, $\vec{v}$ es $90^\circ=\frac{\pi}{2}$ rad, entonces diremos que son \un{ortogonales}. Denotamos $\vec{u}\perp\vec{v}$.
\end{Def}

\begin{Th}
  $\vec u\perp\vec v\iff \braket{\vec u}{\vec v}=0$.
\end{Th}
Para un vector $\vec{u}$ podemos caracterizar 3 regiones del espacio asociadas a este.

\begin{center}


  \tikzset{every picture/.style={line width=0.75pt}} %set default line width to 0.75pt        

  \begin{tikzpicture}[x=0.75pt,y=0.75pt,yscale=-1,xscale=1]
  %uncomment if require: \path (0,300); %set diagram left start at 0, and has height of 300
  
  %Shape: Axis 2D [id:dp6688832326630063] 
  \draw [color={rgb, 255:red, 0; green, 93; blue, 164 }  ,draw opacity=1 ] (120.3,100) -- (240.3,100)(179.97,40.5) -- (179.97,160.5) (233.3,95) -- (240.3,100) -- (233.3,105) (174.97,47.5) -- (179.97,40.5) -- (184.97,47.5)  ;
  %Straight Lines [id:da9744891950606467] 
  \draw [color={rgb, 255:red, 253; green, 185; blue, 18 }  ,draw opacity=1 ]   (179.97,100) -- (218.59,61.41) ;
  \draw [shift={(220,60)}, rotate = 135.02] [color={rgb, 255:red, 253; green, 185; blue, 18 }  ,draw opacity=1 ][line width=0.75]    (10.93,-3.29) .. controls (6.95,-1.4) and (3.31,-0.3) .. (0,0) .. controls (3.31,0.3) and (6.95,1.4) .. (10.93,3.29)   ;
  %Straight Lines [id:da11494680149590053] 
  \draw [color={rgb, 255:red, 184; green, 117; blue, 175 }  ,draw opacity=1 ] [dash pattern={on 4.5pt off 4.5pt}]  (140,60) -- (191,110.5) ;
  %Shape: Triangle [id:dp05425208575314078] 
  \draw  [color={rgb, 255:red, 0; green, 134; blue, 65 }  ,draw opacity=1 ] (225.5,54.67) -- (224.92,135.42) -- (144.92,55.42) -- cycle ;
  %Shape: Triangle [id:dp8357277790196767] 
  \draw  [color={rgb, 255:red, 243; green, 112; blue, 33 }  ,draw opacity=1 ] (134.97,145.25) -- (135.56,64.5) -- (215.56,144.5) -- cycle ;
  
  % Text Node
  \draw (223.5,52.27) node [anchor=south east] [inner sep=0.75pt]  [font=\tiny,color={rgb, 255:red, 0; green, 134; blue, 65 }  ,opacity=1 ]  {$\bra{\vec{u}}\ket{\vec{v}}  >0$};
  % Text Node
  \draw (136.97,146.65) node [anchor=north west][inner sep=0.75pt]  [font=\tiny,color={rgb, 255:red, 243; green, 112; blue, 33 }  ,opacity=1 ]  {$\bra{\vec{u}}\ket{\vec{v}} < 0$};
  % Text Node
  \draw (185.33,69.07) node [anchor=north west][inner sep=0.75pt]  [font=\scriptsize,color={rgb, 255:red, 253; green, 185; blue, 18 }  ,opacity=1 ]  {$\vec{u}$};
  % Text Node
  \draw (210.93,138.81) node [anchor=south east] [inner sep=0.75pt]  [font=\tiny,color={rgb, 255:red, 184; green, 117; blue, 175 }  ,opacity=1 ,rotate=-45]  {$\bra{\vec{u}}\ket{\vec{v}} =0$};
  
  
  \end{tikzpicture}
  
\end{center}

\begin{ptcbP}
  Si $\vec{x}=2\vu{\imath}+\thickhat{k}$ y $\vec{y}=3\vu{\jmath}-\thickhat{k}$ entonces:
  \vspace{-0.35em}
  \begin{enumerate}[i)]
    \itemsep=-0.35em
    \item Encuentre la longitud de $3\vec{x}+\vec{y}$.
    \item Encuentre $\vec{u}$ con componente $\jj$ igual a $-4$ y $\vec{u}\perp \vec{x},\vec{y}$.
    \item Encuentre $\vec{v}$ con $\norm{\vec{v}}=21$ y $\vec{v}\perp \vec{x},\vec{y}$.
  \end{enumerate}
\end{ptcbP}

\begin{Rmk}
Al igual que $\ii=(1,0)$, $\jj=(0,1)$ en $\bR^2$, $\kk$ es el vector canónico que apunta en dirección $z^+$ en $\bR^3$, aquí $\kk=(0,0,1)$. En general, el vector $(a,b,c)\in\bR^3$ es $a\ii+b\jj+c\kk$. 
\end{Rmk}

\subsection*{Producto Cruz (Sólo en $\bR^3$)}

\begin{Def}
  Si $\vec u,\vec v\in\bR^3$, entonces su \un{producto cruz} es 
  $$\vec u\x\vec v=\threebyone{u_2v_3-u_3v_2}{u_3v_1-u_1v_3}{u_1v_2-u_2v_1}=\text{det}^\ast\threebythree{\ii}{\jj}{\kk}{u_1}{u_2}{u_3}{v_1}{v_2}{v_3}.$$
\end{Def}

\textbf{Propiedades:}
\begin{enumerate}[i)]
  \itemsep=-0.4em
  \item  $\vec u\x\vec v=-\vec v\x\vec u$.
  \item  $\vec u\x 0=\vec u\x\vec u=0$.
  \item  $(\vec u+\vec v)\x\vec w=(\vec u\x \vec w)+(\vec v\x \vec w)$.
  \item $(c\vec u)\x\vec v=c(\vec{u}\x\vec v)$.
\end{enumerate}

\begin{Prop}
  Para $\vec{u},\vec v$, vale: \un{$\norm{\vec{u}\x\vec{v}}=\norm{\vec{u}}\norm{\vec v}\sin(\theta)$}.
\end{Prop}

\begin{Prop}
  Vale que $\vec{u}\x\vec{v}\perp\vec{u},\vec{v}$.
\end{Prop}

\begin{Def}
  Dos vectores son \un{paralelos} si el ángulo entre ellos es $0^\circ$ ó $180^\circ=\pi$ rad. Denotamos como $\vec{u}\parallel\vec v$.
\end{Def}

\begin{Th}
  $\vec{u}\parallel\vec v\iff \vec{u}\x \vec v=0$.
\end{Th}

\begin{Ex}
  El tercer inciso de la práctica pide encontrar $\vec v\perp\vec x,\vec y$. ¡Aprovechamos el producto cruz!
  \begin{align*}
    \vec x\x\vec y&=\det\threebythree{\ii}{\jj}{\kk}{2}{0}{1}{0}{3}{-1}\\
    &=0\ii+0\jj+6\kk-0\kk-3\ii-(-2)\jj=(-3,2,6).
  \end{align*}
  En este caso $\norm{\vec x\x\vec y}^2=9+4+36=49$, entonces $\norm{\vec x\x\vec y}=7$. Si multiplicamos ese vector por 3, obtendremos uno de norma 21. Concluimos que el vector $\vec v$ buscado es $(-9,6,18)$.
\end{Ex}

\subsection*{Aplicaciones: Área, Volumen y Proyección}

En el plano, $\vec{u}$ y $\vec{v}$ generan un paralelogramo. ¿Cuál es su área?\par
\vspace{-1em}
\begin{multicols}{2}
 \begin{center}

    \tikzset{every picture/.style={line width=0.75pt}} %set default line width to 0.75pt        
    \begin{tikzpicture}[x=0.75pt,y=0.75pt,yscale=-1,xscale=1]
    %uncomment if require: \path (0,92); %set diagram left start at 0, and has height of 92
    
    %Shape: Parallelogram [id:dp06145446522127873] 
    \draw  [color={rgb, 255:red, 255; green, 255; blue, 255 }  ,draw opacity=1 ][fill={rgb, 255:red, 248; green, 142; blue, 222 }  ,fill opacity=0.5 ] (132.37,17.38) -- (164.33,4.83) -- (153.96,69.47) -- (121.99,82.02) -- cycle ;
    %Straight Lines [id:da6975086524290921] 
    \draw [color={rgb, 255:red, 0; green, 134; blue, 65 }  ,draw opacity=1 ]   (121.99,82.02) -- (152.1,70.2) ;
    \draw [shift={(153.96,69.47)}, rotate = 158.57] [color={rgb, 255:red, 0; green, 134; blue, 65 }  ,draw opacity=1 ][line width=0.75]    (10.93,-3.29) .. controls (6.95,-1.4) and (3.31,-0.3) .. (0,0) .. controls (3.31,0.3) and (6.95,1.4) .. (10.93,3.29)   ;
    %Straight Lines [id:da8284088686225004] 
    \draw [color={rgb, 255:red, 253; green, 185; blue, 18 }  ,draw opacity=1 ]   (121.99,82.02) -- (132.05,19.36) ;
    \draw [shift={(132.37,17.38)}, rotate = 99.12] [color={rgb, 255:red, 253; green, 185; blue, 18 }  ,draw opacity=1 ][line width=0.75]    (10.93,-3.29) .. controls (6.95,-1.4) and (3.31,-0.3) .. (0,0) .. controls (3.31,0.3) and (6.95,1.4) .. (10.93,3.29)   ;
    %Straight Lines [id:da2015147023154964] 
    \draw [color={rgb, 255:red, 255; green, 224; blue, 106 }  ,draw opacity=1 ] [dash pattern={on 4.5pt off 4.5pt}]  (153.96,69.47) -- (164.33,4.83) ;
    %Straight Lines [id:da6113039908903415] 
    \draw [color={rgb, 255:red, 185; green, 217; blue, 137 }  ,draw opacity=1 ] [dash pattern={on 4.5pt off 4.5pt}]  (132.37,17.38) -- (164.33,4.83) ;
    
    % Text Node
    \draw (125.18,49.7) node [anchor=east] [inner sep=0.75pt]  [font=\scriptsize,color={rgb, 255:red, 253; green, 185; blue, 18 }  ,opacity=1 ]  {$\vec{u}$};
    % Text Node
    \draw (139.98,79.14) node [anchor=north west][inner sep=0.75pt]  [font=\scriptsize,color={rgb, 255:red, 0; green, 134; blue, 65 }  ,opacity=1 ]  {$\vec{v}$};
    % Text Node
    \draw (119.99,82.02) node [anchor=east] [inner sep=0.75pt]  [font=\tiny]  {$A$};
    % Text Node
    \draw (130.37,13.98) node [anchor=south east] [inner sep=0.75pt]  [font=\tiny]  {$B$};
    % Text Node
    \draw (155.96,72.87) node [anchor=north west][inner sep=0.75pt]  [font=\tiny]  {$C$};
    % Text Node
    \draw (166.33,4.83) node [anchor=west] [inner sep=0.75pt]  [font=\tiny]  {$D$};
    % Text Node
    \draw (143.16,43.42) node  [color={rgb, 255:red, 184; green, 117; blue, 175 }  ,opacity=1 ]  {$A?$};
    
    
    \end{tikzpicture}   
  \end{center}
\vfill\null\columnbreak
    La fórmula del área será 
    $$A=\norm{\vec u\x \vec v}.$$
\end{multicols}
\vspace{-1cm}
Análogamente en tres dimensiones, tres vectores $\vec{u},\vec v$ y $\vec{w}$ generan un paralelepípedo. 

\begin{Prop}
  El volumen del paralelepípedo formado por $\vec{u},\vec v,\vec{w}$ es 
  $V=|\braket{\vec{u}}{\vec{v}\x\vec w}|$.
\end{Prop}

La proyección ortogonal responde otro problema de encontrar distancias. Consideremos dos vectores $\vec u,\vec v$. Queremos encontrar la \un{menor distancia} entre $\vec{u}$ y $\vec v$. Es decir, desde la punta de $\vec{v}$ y un punto de $\vec u$.
\begin{center}
  
\tikzset{every picture/.style={line width=0.75pt}} %set default line width to 0.75pt        

\begin{tikzpicture}[x=0.75pt,y=0.75pt,yscale=-1,xscale=1]
%uncomment if require: \path (0,142); %set diagram left start at 0, and has height of 142

%Shape: Axis 2D [id:dp4413235062428935] 
\draw [color={rgb, 255:red, 0; green, 93; blue, 164 }  ,draw opacity=1 ] (8.7,122.42) -- (141.7,122.42)(25.03,6.25) -- (25.03,132) (134.7,117.42) -- (141.7,122.42) -- (134.7,127.42) (20.03,13.25) -- (25.03,6.25) -- (30.03,13.25) (45.03,117.42) -- (45.03,127.42)(65.03,117.42) -- (65.03,127.42)(85.03,117.42) -- (85.03,127.42)(105.03,117.42) -- (105.03,127.42)(125.03,117.42) -- (125.03,127.42)(20.03,102.42) -- (30.03,102.42)(20.03,82.42) -- (30.03,82.42)(20.03,62.42) -- (30.03,62.42)(20.03,42.42) -- (30.03,42.42)(20.03,22.42) -- (30.03,22.42) ;
\draw [color={rgb, 255:red, 0; green, 93; blue, 164 }  ,opacity=1 ]  ;
%Straight Lines [id:da9706363596957956] 
\draw [color={rgb, 255:red, 251; green, 217; blue, 18 }  ,draw opacity=1 ]   (25.03,122.42) -- (44.52,44.39) ;
\draw [shift={(45,42.45)}, rotate = 104.02] [color={rgb, 255:red, 251; green, 217; blue, 18 }  ,draw opacity=1 ][line width=0.75]    (10.93,-3.29) .. controls (6.95,-1.4) and (3.31,-0.3) .. (0,0) .. controls (3.31,0.3) and (6.95,1.4) .. (10.93,3.29)   ;
%Straight Lines [id:da5755928426571768] 
\draw [color={rgb, 255:red, 0; green, 150; blue, 65 }  ,draw opacity=1 ]   (75.02,102.43) -- (123.14,83.19) ;
\draw [shift={(125,82.45)}, rotate = 158.21] [color={rgb, 255:red, 0; green, 150; blue, 65 }  ,draw opacity=1 ][line width=0.75]    (10.93,-3.29) .. controls (6.95,-1.4) and (3.31,-0.3) .. (0,0) .. controls (3.31,0.3) and (6.95,1.4) .. (10.93,3.29)   ;
%Straight Lines [id:da40302285922837955] 
\draw [color={rgb, 255:red, 248; green, 142; blue, 222 }  ,draw opacity=1 ] [dash pattern={on 4.5pt off 4.5pt}]  (45,42.45) -- (75.02,102.43) ;
%Straight Lines [id:da6911517283104183] 
\draw [color={rgb, 255:red, 0; green, 134; blue, 65 }  ,draw opacity=1 ] [dash pattern={on 4.5pt off 4.5pt}]  (25.03,122.42) -- (75.02,102.43) ;
%Straight Lines [id:da964413363427931] 
\draw [color={rgb, 255:red, 243; green, 112; blue, 33 }  ,draw opacity=1 ] [dash pattern={on 0.84pt off 2.51pt}]  (45,42.45) -- (125,82.45) ;

% Text Node
\draw (127,82.45) node [anchor=west] [inner sep=0.75pt]  [font=\scriptsize,color={rgb, 255:red, 0; green, 134; blue, 65 }  ,opacity=1 ]  {$\vec{u}$};
% Text Node
\draw (45,39.05) node [anchor=south] [inner sep=0.75pt]  [font=\scriptsize,color={rgb, 255:red, 253; green, 185; blue, 18 }  ,opacity=1 ]  {$\vec{v}$};
% Text Node
\draw (64.72,78.03) node [anchor=south west] [inner sep=0.75pt]  [rotate=-335.74]  {$\textcolor[rgb]{0.97,0.56,0.87}{\ell ?}$};
% Text Node
\draw (39.75,96.23) node [anchor=north west][inner sep=0.75pt]  [color={rgb, 255:red, 0; green, 134; blue, 65 }  ,opacity=1 ,rotate=-335.51]  {$r?$};
% Text Node
\draw (86.57,59.44) node [anchor=south] [inner sep=0.75pt]  [font=\footnotesize,color={rgb, 255:red, 243; green, 112; blue, 33 }  ,opacity=1 ,rotate=-27.58]  {$\| \vec{v} -\vec{u} \| $};


\end{tikzpicture}

\end{center}
\begin{Prop}
  Valen las fórmulas $r=\braket{\vec v}{\thickhat{u}}=\frac{\braket{\vec v}{\vec u}}{\norm{\vec u}}$ y $\ell^2+r^2=\norm{\vec v}^2$.
\end{Prop}

\begin{Def}
  La \un{proyección ortogonal} de $\vec v$ sobre $\vec{u}$ es $\Proy_{\vec{u}}(\vec v)=r\thickhat{u}=\braket{\vec v}{\thickhat{u}}\thickhat{u}=\frac{\braket{\vec v}{\vec{u}}}{\norm{\vec u}^2}\vec{u}$.
\end{Def}
\end{multicols}
\end{document}