%----------------------------------------------------------------------------------------
%	PACKAGES AND OTHER DOCUMENT CONFIGURATIONS
%----------------------------------------------------------------------------------------

\documentclass[12pt]{article}
\usepackage[spanish]{babel} %Tildes
\usepackage[extreme]{savetrees} %Espaciado e interlineado. Comentar si no gusta el interlineado.
\usepackage[utf8]{inputenc} %Encoding para tildes
\usepackage[breakable,skins]{tcolorbox} %Cajitas
\usepackage{fancyhdr} % Se necesita para el título arriba
\usepackage{lastpage} % Se necesita para poner el número de página
\usepackage{amsmath,amsfonts,amssymb,amsthm} %simbolos y demás
\usepackage{mathabx} %más símbolos
\usepackage{physics} %simbolos de derivadas, bra-ket.
\usepackage{multicol}
\usepackage[customcolors]{hf-tikz}
\usepackage[shortlabels]{enumitem}
\def\darktheme
%%%%%%%%% === Document Configuration === %%%%%%%%%%%%%%

\pagestyle{fancy}
\setlength{\headheight}{14.49998pt} %NO MODIFICAR
\setlength{\footskip}{14.49998pt} %NO MODIFICAR

\ifx \darktheme\undefined

\lhead{MA1004G8} % Nombre de autor
\chead{\textbf{Lección 0428}} % Titulo
\rhead{}%\firstxmark} 
\lfoot{}%\lastxmark}
\cfoot{}
\rfoot{P\'ag.\ \thepage\ de\ \pageref{LastPage}} %A la derecha saldrá pág. 6 de 9. 
\else
\pagenumbering{gobble}
\pagecolor[rgb]{0,0,0}%{0.23,0.258,0.321}
\color[rgb]{1,1,1}
\fi

%%%%%%%%% === My T Color Box === %%%%%%%%%%%%%%

\ifx \darktheme\undefined
\newtcolorbox{ptcb}{
colframe = black,
colback = white,
breakable,
enhanced
}
\newtcolorbox{ptcbP}{
colframe = black,
colback = white,
coltitle = black,
colbacktitle = black!40,
title = Práctica,
breakable,
enhanced
}

\else
\newtcolorbox{ptcb}{
colframe = white,
colback = black,
colupper = white,
breakable,
enhanced
}
\newtcolorbox{ptcbP}{
colframe = white,
colback = black,
colupper = white,
coltitle = white,
colbacktitle = black,
title = Práctica,
breakable,
enhanced
}
\fi

%%%%%%%%% === Tikz para matrices === %%%%%%%%%%%%%%

\tikzset{
  style green/.style={
    set fill color=green!50!lime!60,
    set border color=white,
  },
  style cyan/.style={
    set fill color=cyan!90!blue!60,
    set border color=white,
  },
  style orange/.style={
    set fill color=orange!80!red!60,
    set border color=white,
  },
  row/.style={
    above left offset={-0.15,0.31},
    below right offset={0.15,-0.125},
    #1
  },
  col/.style={
    above left offset={-0.1,0.3},
    below right offset={0.15,-0.15},
    #1
  }
}

%%%%%%%%% === Theorems and suchlike === %%%%%%%%%%%%%%

\theoremstyle{plain}
\newtheorem{Th}{Teorema}  %%% Theorem 1.1
\newtheorem*{nTh}{Teorema}             %%% No-numbered Theorem
\newtheorem{Prop}[Th]{Proposición}     %%% Proposition 1.2
\newtheorem{Lem}[Th]{Lema}             %%% Lemma 1.3
\newtheorem*{nLem}{Lema}               %%% No-numbered Lemma
\newtheorem{Cor}[Th]{Corolario}        %%% Corollary 1.4
\newtheorem*{nCor}{Corolario}          %%% No-numbered Corollary

\theoremstyle{definition}
\newtheorem*{Def}{Definición}       %%% Definition 1.5
\newtheorem*{nonum-Def}{Definición}    %%% No number Definition
\newtheorem*{nEx}{Ejemplo}             %%% No number Example
\newtheorem{Ex}[Th]{Ejemplo}           %%% Example
\newtheorem{Ej}[Th]{Ejercicio}         %%% Exercise
\newtheorem*{nEj}{Ejercicio}           %%% No number Excercise
\newtheorem*{Not}{Notación}       %%% Definition 1.5

\theoremstyle{remark}
\newtheorem*{Rmk}{Observación}      %%%Remark 1.6

%\numberwithin{equation}{section}

\setlength{\parindent}{3ex}

%%====== Useful macros: =======%%%

\DeclareMathOperator{\gen}{gen}     %%%set generated by...
\DeclareMathOperator{\Rng}{Rng}     %%%rangomat
\DeclareMathOperator{\Nul}{Nul}     %%%rangomat

\newcommand{\la}{\lambda}           %%%short for \lambda

\newcommand{\sg}{\sigma}            %%%short for \sigma

\newcommand{\bC}{\mathbb{C}}        %%%complex numbers
\newcommand{\bN}{\mathbb{N}}        %%%natural numbers
\newcommand{\bR}{\mathbb{R}}        %%%real numbers
\newcommand{\bZ}{\mathbb{Z}}        %%%integer numbers
\newcommand{\cB}{\mathcal{B}}       %%%basis
\newcommand{\cC}{\mathcal{C}}       %%%basis
\newcommand{\cM}{\mathcal{M}}       %%%matrix family

\newcommand{\sT}{\mathsf{T}}        %%%traspuesta

\renewcommand{\geq}{\geqslant}      %%%(to save typing)
\renewcommand{\leq}{\leqslant}      %%%(to save typing)
\newcommand{\x}{\times}             %%%product
\renewcommand{\:}{\colon}           %%%colon in  f: A -> B

\newcommand{\un}[1]{\underline{#1}}
\newcommand{\half}{\frac12}

\newcommand*{\Cdot}{{\raisebox{-0.25ex}{\scalebox{1.5}{$\cdot$}}}}      %% cdot más grande
\renewcommand{\.}{\Cdot}                %% producto escalar

\newcommand{\twobyone}[2]{\begin{pmatrix} %% 2 x 1 matrix
  #1 \\ #2 \end{pmatrix}}
\newcommand{\twobytwo}[4]{\begin{pmatrix} %% 2 x 2 matrix
  #1 & #2 \\ #3 & #4 \end{pmatrix}}
\newcommand{\threebyone}[3]{\begin{pmatrix} %% 3 x 1 matrix
  #1 \\ #2 \\ #3 \end{pmatrix}}
\newcommand{\threebythree}[9]{\begin{pmatrix} %% 3 x 3 matrix
  #1 & #2 & #3 \\ #4 & #5 & #6 \\ #7 & #8 & #9 \end{pmatrix}}

%----------------------------------------------------------------------------------------
%	ARTICLE CONTENTS
%----------------------------------------------------------------------------------------

\begin{document}
\begin{multicols}{2}

\subsection*{Determinantes}

El \un{determinante} es una función cuya entrada es una matriz cuadrada y devuelve un escalar. 

\begin{Not}
El determinante de una matriz $A$ se denota $\det(A)$ ó bien $|A|$.
\end{Not}

\begin{itemize}
  \item En el caso $[2\x 2]$, si $A=\begin{pmatrix}
    a&b\\c&d
  \end{pmatrix}$ entonces
  $$\det(A)=ad-bc.$$
  
  \item El caso $[3\x 3]$, la \textbf{Regla de Sarrus}:\par 
  Si $A=\begin{pmatrix}
    a&b&c\\d&e&f\\g&h&i
  \end{pmatrix}$ entonces calculamos su determinante tomando las dos primeras columnas y aumentando $A$ con ellas de forma que obtenemos
  $$\left(\begin{array}{ccc|cc}
    a&b&c&a&b\\d&e&f&d&e\\g&h&i&g&h
  \end{array}\right)$$
  de aquí sumamos las diagonales de arriba hacia abajo y restamos las diagonales de abajo hacia arriba de forma que 
  $$\det(A)=aei+bfg+cdh-gec-hfa-idb.$$
\end{itemize}

\subsection*{Propiedades}

\begin{enumerate}[i)]
  \item Escalares: $\det(cA)=c^{\#\text{filas} A}\det(A)$.
  \item Traspuestas: $\det(A)=\det(A^\sT)$.
  \item Multiplicatividad: $\det(AB)=\det(A)\det(B)$.
  \item Inversas: $\det(A^{-1})=\frac{1}{\det(A)}$.
\end{enumerate}

El siguiente resultado es parte del teorema resumen que hemos visto la lección pasada.

\begin{Th}[Adendo al Tma. Resumen]
Una matriz $A$ es invertible si y sólo si $\det(A)\neq 0$.
\end{Th}

\begin{ptcbP} 
Considere las matrices $$I=\begin{pmatrix}
  1&0\\0&1
\end{pmatrix},\ -I=\begin{pmatrix}
  -1&0\\0&-1
\end{pmatrix}.$$ 
Calcule sus determinantes. Ahora sume ambas matrices y calcule el determinante del resultado.
\end{ptcbP}
\begin{Ej} 
  En general, ¿se cumple que $\det(A+B)=\det(A)+\det(B)$? ¿Puede encontrar un ejemplo donde sí se cumpla y uno donde no?
\end{Ej}

%https://math.stackexchange.com/questions/17776/inverse-of-the-sum-of-matrices

\subsection*{Determinantes de orden superior}

Para matrices en $\cM_4$ o de tamaño más grande existen dos maneras de calcular sus determinantes.

\subsubsection*{Expansión por menores (Laplace)}

\begin{Ex}
Calculamos el determinante de $A=\begin{pmatrix}
  a&b&c\\d&e&f\\g&h&i
\end{pmatrix}$ expandiendo por la primera fila. 

$$
  \left|\left(\begin{array}{ccc}
    \tikzmarkin[row = style cyan]{m1r1}a&b&c\tikzmarkend{m1r1}\\d&e&f\\g&h&i
  \end{array}\right)\right|=a\left|\begin{pmatrix}e&f\\h&i\end{pmatrix}\right|-b\left|\begin{pmatrix}d&f\\g&i\end{pmatrix}\right|+c\left|\begin{pmatrix}d&e\\g&h\end{pmatrix}\right|.$$
\end{Ex}
  Estas \emph{submatrices} llevan el siguiente nombre:
  \begin{Def}
El \un{menor $(i,j)$} de una matriz $A$ se obtiene al eliminar de $A$ la fila $i$ y la columna $j$.
  \end{Def}
El proceso en concreto es:
\begin{enumerate}[i)]
  \item Tomamos una entrada de la fila escogida.
  \item Eliminamos el resto de la fila y columna que la contienen para obtener un menor.
  \item Tomamos el determinante de ese menor.
  \item Lo multiplicamos por la entrada en cuestión.
\end{enumerate}
  $$
  \approx \left(\begin{array}{ccc}
    \tikzmarkin[row = style green]{m2c1}a\tikzmarkend{m2c1}&\x&\x\\ \x&\tikzmarkin[col = style green]{m2b1}e&f\\\x&h&i\tikzmarkend{m2b1}
  \end{array}\right)+\left(\begin{array}{ccc}
    \x&\tikzmarkin[row = style orange]{m3e2}b\tikzmarkend{m3e2}&\x\\ \tikzmarkin[col = style orange]{m3c1}d&\x&\tikzmarkin[col = style orange]{m3c2}f\\g\tikzmarkend{m3c1}&\x&i\tikzmarkend{m3c2}
  \end{array}\right)+\left(\begin{array}{ccc}
    \x&\x&\tikzmarkin[row = style green]{m4e3}a\tikzmarkend{m4e3}\\ \tikzmarkin[col = style green]{m4b1}d&e&\x\\g&h\tikzmarkend{m4b1}&\x
  \end{array}\right)
$$

Para los signos, seguimos la \emph{convención} de que la entrada $(1,1)$ es $+$, y los vecinos de $+$ son $-$ y vice-versa. Obtenemos una matriz de signos:
$$\begin{pmatrix}
  \tikzmarkin[row = style green]{s1}+\tikzmarkend{s1}&\tikzmarkin[row = style orange]{s2}-\tikzmarkend{s2}&\tikzmarkin[row = style green]{s3}+\tikzmarkend{s3}\\\tikzmarkin[row = style orange]{s4}-\tikzmarkend{s4}&\tikzmarkin[row = style green]{s5}+\tikzmarkend{s5}&\tikzmarkin[row = style orange]{s6}-\tikzmarkend{s6}\\\tikzmarkin[row = style green]{s7}+\tikzmarkend{s7}&\tikzmarkin[row = style orange]{s8}-\tikzmarkend{s8}&\tikzmarkin[row = style green]{s9}+\tikzmarkend{s9}
\end{pmatrix}$$
Aquí $+$ es \textbf{no} cambiar de signo, pero $-$ \textbf{sí} es cambiarlo.

\begin{ptcbP}
  Calcule con alguien $\det(A)$ si
  $$A=\begin{pmatrix}
    -10&7&3\\
    0&8&0\\
    6&-4&0
  \end{pmatrix}.$$
  \emph{Sugerencia: Puede expandir por cualquier fila o columna.}
\end{ptcbP}
\vfill\null
\newpage
\subsection*{Reducción y Determinantes}
Primero veamos lo siguiente:
\begin{Th}
  Si $A$ es triangular o diagonal, su determinante es el producto de su diagonal.
\end{Th}

\begin{Ex}
  Consideremos las matrices 
  $$A=\begin{pmatrix}
    2&3&5\\0&7&11\\0&0&13
  \end{pmatrix},\ B=\begin{pmatrix}
    k&0&0\\-k&3&0\\2k&0&1
  \end{pmatrix},\ D=\begin{pmatrix}
    1&0&0\\0&-6&0\\0&0&9
  \end{pmatrix}.$$
  Si calculamos sus determinantes expandiendo vemos que 
  $$\det(A)=2\left|\begin{pmatrix}7&11\\0&13\end{pmatrix}\right|+0=2\.7\.13.$$
  Análogamente
  $$\det(B)=k\left|\begin{pmatrix}
    3&0\\0&1
  \end{pmatrix}\right|+0=3k.$$
  Lo mismo ocurre con $D$ ya sea expandiendo por cualquier fila o columna.
\end{Ex}

\begin{itemize}
  \item Ya sabemos reducir matrices.
  \item Basta con reducir a forma triangular.
  \item Pero, ¿cambiará el determinante cuando se hacen operaciones de fila? \textbf{¡Sí!}
\end{itemize}
Ocurre lo siguiente:
\begin{Ex}
  Si $A=\begin{pmatrix}
    1&2\\3&5
  \end{pmatrix}$ y aplicamos $F_1\leftrightarrow F_2$ para obtener $B=\begin{pmatrix}
    3&5\\1&2
  \end{pmatrix}$, entonces $\det(B)=-\det(A)$.\par 
  \textbf{Intercambiar} filas \textbf{cambia el signo} del determinante.
\end{Ex}

\begin{Ex}
  Si $A=\begin{pmatrix}
    1&2\\3&5
  \end{pmatrix}$ y aplicamos $6F_1$ para obtener $B=\begin{pmatrix}
    6&12\\3&5
  \end{pmatrix}$, entonces $\det(B)=6\.\det(A)$.\par 
  \textbf{Reescalar} filas \textbf{multiplica} el determinante \textbf{por el mismo escalar}.
\end{Ex}

Finalmente, a la hora de \textbf{combinar} filas, no le hacemos ningún cambio al determinante.

\begin{Ej}
  Supongamos que $A\in\cM_4$ tiene determinante $12$. Aplicamos las siguientes operaciones:
  \begin{enumerate}\begin{multicols*}{2}
    \item $A\mapsto A^\sT$.
    \item $6F_2$, $\frac{1}{2}F_3$.
    \item $F_3-F_2$, $F_1+F_2$.
    \item $B\mapsto B^\sT$, $F_1\leftrightarrow F_4$.
  \end{multicols*}
  \end{enumerate}
  ¿Cuál es el determinante de la matriz resultante?
\end{Ej}
\vfill\null
\columnbreak
\subsection*{La Regla de Cramer}
Este es otro método para resolver sistemas lineales.\par 
\begin{itemize}
  \item En el caso $[2\x2]$ los sistemas son de la forma
  $$A\vec{x}=\vec{b}\:\ \twobytwo{a}{b}{c}{d}\twobyone{x}{y}=\twobyone{\textcolor{red}{e}}{\textcolor{red}{f}}.$$
  La solución del sistema será:
  $$x=\frac{\left|\twobytwo{\textcolor{red}{e}}{b}{\textcolor{red}{f}}{d}\right|}{\left|\twobytwo{a}{b}{c}{d}\right|},\ y=\frac{\left|\twobytwo{a}{\textcolor{red}{e}}{c}{\textcolor{red}{f}}\right|}{\left|\twobytwo{a}{b}{c}{d}\right|}$$
  \item En el caso $[3\x3]$ ocurre que:
  $$A\vec{x}=\vec{b}\:\ \threebythree{a}{b}{c}{d}{e}{f}{g}{h}{i}\threebyone{x}{y}{z}=\threebyone{\textcolor{red}{j}}{\textcolor{red}{k}}{\textcolor{red}{l}}.$$
  Aquí la solución será:
  \footnotesize
  $$x=\frac{\left|\threebythree{\textcolor{red}{j}}{b}{c}{\textcolor{red}{k}}{e}{f}{\textcolor{red}{l}}{h}{i}\right|}{\left|\threebythree{a}{b}{c}{d}{e}{f}{g}{h}{i}\right|},\ y=\frac{\left|\threebythree{a}{\textcolor{red}{j}}{c}{d}{\textcolor{red}{k}}{f}{g}{\textcolor{red}{l}}{i}\right|}{\left|\threebythree{a}{b}{c}{d}{e}{f}{g}{h}{i}\right|},\ z=\frac{\left|\threebythree{a}{b}{\textcolor{red}{j}}{d}{e}{\textcolor{red}{k}}{g}{h}{\textcolor{red}{l}}\right|}{\left|\threebythree{a}{b}{c}{d}{e}{f}{g}{h}{i}\right|}$$
\end{itemize}
El proceso es 
\begin{enumerate}[i)]
  \item Tomamos una variable.
  \item Eliminamos de $A$ la columna que le corresponde.
  \item En tal columna incluimos el vector de constantes $\vec b$.
  \item Calculamos el determinante de la nueva matriz y lo dividimos por $\det(A)$.
\end{enumerate}
\begin{ptcbP}
  Suponga que en un sistema conocemos la siguiente información:
  $$A\vec x=\vec b\:\ \threebythree{3}{0}{\x}{0}{-2}{\x}{-15}{0}{\x}\threebyone{x}{y}{z}=\threebyone{-1}{0}{1}.$$
  Si $\vec{x}$ es solución del sistema y $\det(A)=12$, ¿cuánto vale $z$?
\end{ptcbP}
\end{multicols}
\end{document}