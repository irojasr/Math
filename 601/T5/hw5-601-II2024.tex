\documentclass[12pt]{memoir}

\def\nsemestre {II}
\def\nterm {Fall}
\def\nyear {2024}
\def\nprofesor {Maria Gillespie}
\def\nsigla {MATH601}
\def\nsiglahead {Advanced Combinatorics}
\def\nextra {HW5}
\def\nlang {ENG}
\input{../../headerVarillyDiff}
\usepackage{youngtab}

\begin{document}

\iffalse
\begin{Ej}[1]
    Consider a word of 1's and 2's. Show that if we remove a 2 from this word, then at most
one 1 that was bracketed becomes unbracketed, and all other 1's retain their status (of being bracketed
with some 2 or not). This was shown with mountain graphs on the board; write out a proof carefully.
\end{Ej}

\begin{ptcbr}
    We analyze the following cases:
    \begin{itemize}
        \item If there's no $1$'s after our removed $2$, then no $1$ becomes unbracketed.
        \item If there was at least one $1$ after our $2$
    \end{itemize}
\end{ptcbr}
\fi

\begin{Ej}[7(a)]
    Show that if two words $w$ and $v$ are Knuth equivalent, then removing the smallest $k$
    letters from each of $w$ and $v$ (with ties broken in order from left to right) results in two Knuth
    equivalent words $w'$ and $v'$. (Hint: It suffices to show it for just one letter removed, and it also
    suffices to assume that $w$ and $v$ differ by an elementary Knuth move.)
\end{Ej}

Let us recall the Knuth move definition from a year ago:

\begin{Def}
    A \term{Knuth move} on a permutation swaps two letters $a,c$ if $a<b<c$ (reading order) and one of consecutive subsequences $acb,cab,bac,bca$ appears in the word.\par 
    Two words are \term{Knuth-equivalent} if they differ by a sequence of Knuth-moves.
\end{Def}

\begin{ptcbr}
   We\footnote{\bfseries{Ross}, \bfseries{Joel} and myself discussed this exercise together.} follow the hint as indicated. This is possible because after doing a sequence of Knuth moves for a particular letter and then doing it for all the letters, we get to the complete result.\par
   So for our case let us take a word $w$ whose smallest letter is $x_0$. We have two cases:
   \begin{itemize}
    \item If $x_0$ is not in the affix $acb$ we are Knuth moving then 
    $$w=\dots x_0\dots acb\dots.$$
    It is equivalent if $x_0$ is after the affix. Let us Knuth move and remove $x_0$ and vice versa.
    \begin{center}
        % https://tikzcd.yichuanshen.de/#N4Igdg9gJgpgziAXAbVABwnAlgFyxMJZABgBpiBdUkANwEMAbAVxiRAB12oIc4ACAB4B9Yp268+dAMYAjMTwQBfUuky58hFAEZyVWoxZt5E4aK4K+UunPO8Qy1djwEiZLXvrNWiDrf7SbcSUVEAwnDSIdd2pPQx9jfitAhXs9GCgAc3giUAAzACcIAFskMhAcCCQAJhiDb18cGAEcYAAlGCLFIWBTRRBqBmsYBgAFNWdNEHysDIALHHsQguKkHXLKxABmWq8jdkbmto6unpE+hxBlksQa9aRtkEGZYbHwlx9puYWduIamloA0mAmDhZucloVrmUKqsfvVOAdAcDQX0BkNRuMIj4GDBcgtFBRFEA
\begin{tikzcd}
    \dots x_0\dots acb\dots \arrow[d, "\text{Rem}_{x_0}"'] \arrow[r, "\text{Knuth}"] & \dots x_0\dots cab\dots \arrow[d, "\text{Rem}_{x_0}"] \\
    \dots acb\dots \arrow[r, "\text{Knuth}"']                                        & \dots cab\dots                                       
    \end{tikzcd}
    \end{center}
    The case where our affix is $bac$ is analogous.
    \item However if $x_0$ was in the affix we have that $x_0$ takes the place of $a$ as the smallest letter. Then we have $w=\dots x_0cb\dots$ and so
    \begin{center}
      % https://tikzcd.yichuanshen.de/#N4Igdg9gJgpgziAXAbVABwnAlgFyxMJZABgBpiBdUkANwEMAbAVxiRAB12oIc4ACAB4B9YgGMARp268QAX1LpMufIRQBGUmqq1GLNlJ78JBmfMXY8BIgCZy2+s1aIOXQ31HDik16e0woAObwRKAAZgBOEAC2SGQgOBBItjqO+uw4MAI4wADSYEw4ABaycgogEdGx1AlIGil6zpwZWcAASjBRskLAniXUDHTiMAwACkqWqiDhWAGFOKVhkTGIyTWIdQ4NLs3Z7Z3dvXIUskA
\begin{tikzcd}
    \dots x_0cb\dots \arrow[rr, "\text{Knuth}"] \arrow[rd, "\text{Rem}_{x_0}"'] &               & \dots cx_0b\dots \arrow[ld, "\text{Rem}_{x_0}"] \\
                                                                                & \dots cb\dots &                                                
    \end{tikzcd}
    \end{center}
    which leaves us with the same word in both cases. In a similar fashion to the previous part, working with the affix $bx_0c$ is analogous.
   \end{itemize}
   In both cases we can see that removing the smallest letter and Knuth moving commutes on words. So this tells us that originally Knuth equivalent words will still be Knuth equivalent after removing the smallest letter.
\end{ptcbr}

\begin{Ej}[7(b)]
    Given a pair $(S, T )$ in $P^\nu_{\la\mu}$, show that $T$ must be the unique highest weight tableau of shape $\mu$. Conclude that the pair $(T, T )$ corresponds under RSK to a two line array
    $$
    \begin{pmatrix}
        a_1&a_2&\dots&a_{|\mu|}\\
        b_1&b_2&\dots&b_{|\mu|}
    \end{pmatrix}
    $$
where $a_1,a_2,\dots,a_{|\mu|}$ is the unique weakly increasing word of content $\mu$ and $b_1,b_2,\dots,b_{|\mu|}$ is a ballot word.
\end{Ej}

\begin{ptcbr}
    As $(S,T)$ is in $P^{\nu}_{\la\mu}$, then $\rw(S)\rw(T)$ is a ballot word. We\footnote{\bfseries Ross, Joel and myself worked on this, but then Parker helped me out with the final details.} have that every suffix of a ballot word is itself ballot. So in particular $\rw(T)$ is ballot.\par
    On every row, $T$ has the same entries and so its structure is determined. This means that it is of highest weight. Having shape $\mu$ uniquely determines it.\par
    Now, instead of applying reverse RSK to $(T,T)$, we consider the two-line array comprised of the reverse column reading word of $T$, $b$ and the reverse reading word of $T$, $a=\rev\rw(T)$.\par
    The reverse column reading word is obtained by reading left-to-right, and down-up through $T$, so when verifying it's ballot we must read right-to-left and down-up. This produces a ballot word $b$ because of the structure of $T$, always before adding a higher number, there will come a smaller number. And the reverse reading word of $T$ is weakly increasng due to its structure.
\end{ptcbr}

\begin{Ej}[7(c)(i)]
    Insert $b_1,\dots,b_n$ into $S$ and write $a_1,\dots,a_{|\mu|}$ into the boxes that appear in order. Let $R$ be the
resulting skew tableau with letters $a_1,\dots,a_\mu$.\par
Show that $R$ has shape $\nu/\la$ with content $\mu$.
\end{Ej}

\begin{ptcbr}
    Observe the $|\la|+|\mu|=|\nu|$ by the fact that $(S,T)\in P^\nu_{\la\mu}$. So $|\nu|-|\la|=|\mu|$. Now, our assumption of $\rw(T)\rw(S)$ being ballot guarantees that when inserting $a$, we will get shape $\nu$, thus when skewing $S$ we get shape $\nu/\la$. The content of $b$ is precisely $\mu$.
\end{ptcbr}

\begin{Ej}[7(c)(ii)]
    To show it (the reading word of $R$) is ballot, consider the tableau $\tilde R$ formed by filling the shape $\la$ underneath
$R$ with a highest weight word of negative entries $0,-1,-2,\dots$, in other words, fill the top
row of $\la$ with $0$'s, the next with $-1$, the next with $-2$, and so on. Let $V$ be the highest
weight tableau of shape $\nu$. Apply RSK to the pair $( \tilde R, V )$, and use part $(a)$ of this problem
to show that the letters corresponding to the entries of $\tilde R$ in the resulting two-line array form
a highest weight word. Conclude that the reading word of $R$ is ballot
\end{Ej}

\begin{ptcbr}
As $V$ is of highest weight, its rows are all the same elements and it has content $\nu$. In order to apply RSK we once again do it in reverse, following the same strategy as before.\par
We look at the reverse column reading word of $\tilde R$ which we've seen, inserts into the tableau. Following the previous algorithm we will get that $\tilde{b}$ is a ballot word and using the first part of this problem we have that when removing the negative letters we still obtain a ballot word. This is because Knuth moves commute with crystal operators and thus preserve weight. In particular, our word is of highest weight and therefore is ballot.
\end{ptcbr}

\begin{Ej}[7(c)(iii)]
    Conclude that the map $(S,T)\mapsto R$ using the above construction is a well-defined function from $P^\nu_{\la\mu}$ to $Q^\nu_{\la\mu}$.
\end{Ej}

\begin{ptcbr}
    The previous two problems show that $(S,T)$ indeed land in $Q^\nu_{\la\mu}$ when applying this map as $\sh(R)=\nu/\la$ with content $\mu$ and its reading word is ballot.\par
    Showing that the map is well defined amounts to showing that if we have $(S_1,T_1)=(S_2,T_2)\in P^\nu_{\la\mu}$ then their image coincides in $Q^\nu_{\la\mu}$. As we have specified the process above, applying it to the same pair of tableau should give us the same result guaranteeing that their image is the same.
\end{ptcbr}
\end{document} 