\documentclass[12pt]{memoir}

\def\nsemestre {II}
\def\nterm {Fall}
\def\nyear {2024}
\def\nprofesor {Maria Gillespie}
\def\nsigla {MATH601}
\def\nsiglahead {Advanced Combinatorics}
\def\nextra {CHWB}
\def\nlang {ENG}
\input{../../headerVarillyDiff}
\usepackage{youngtab}

\begin{document}
\begin{Ej}
Draw the $\gsl_3$ crystal for weight $(3,3,0)$.
\end{Ej}

\begin{Ej}
    Prove that the elements of the hyperoctahedral group, written in cycle notation as a permutation on $\set{\pm1,\dots , \pm n}$, has all of its cycles coming in either pairs of the form $(a_1\dots a_k)(-a_1\dots -a_k)$, or of the form $(a_1\dots a_k -a_1\ -a_2\dots-a_k)$.
\end{Ej}

\begin{Ej}
    Define the Lie algebra $\gso_{2n+1}$ as $\set{X\:\ X^\sT S+SX=0}$ where 
    $$S=\threebythree{1}{0}{0}{0}{0_n}{I_n}{0}{I_n}{0_n}$$
    and $I_n$ is the $n\x n$ identity matrix and $1$ is in the upper left corner. Write down what an arbitrary element $X$ looks like, and using the fact that with respect to this setup the torus is simply the set of diagonal matrices $X$ satisfying these conditions, explain how one obtains the type $B$ root system.
\end{Ej}

\begin{Ej}
    What is the dimension of the adjoint representation of $\gso_7$?
\end{Ej}

\begin{Ej}
    Explain why the set of $5^{\text{th}}$ roots of unity in the plane don't form a root system. Which axioms of root systems does it satisfy?
\end{Ej}

\begin{Ej}
    Compute the evacuation of the Young tableau below, and then evacuate again, and show you have returned to the starting tableau.
    $$\young(5,278,1346)$$
\end{Ej}

\begin{Ej}
    Compute the Hall-Littlewood polynomial $\tilde{H}_{(2,1,1)}(x;q)$.
\end{Ej}

\begin{Ej}
    Let $w=w_1\dots w_n$ be a word of partition content, and suppose $w_1\neq 1$. Let $w'=w_2\dots w_nw_1$ be formed by cycling $w_1$ around to the end of the word. Show that $\cc(w')=\cc(w)=1$ where $\cc$ is cocharge. This operation is called \emph{cyclage}.
\end{Ej}

\begin{Ej}
    Give a counterexample showing that the formula in the above problem does not hold in general when $w_1=1$.
\end{Ej}
\end{document} 
