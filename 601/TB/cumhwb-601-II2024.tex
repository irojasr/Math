\documentclass[12pt]{memoir}

\def\nsemestre {II}
\def\nterm {Fall}
\def\nyear {2024}
\def\nprofesor {Maria Gillespie}
\def\nsigla {MATH601}
\def\nsiglahead {Advanced Combinatorics}
\def\nextra {CHWB}
\def\nlang {ENG}
\input{../../headerVarillyDiff}
\usepackage[vcentermath,enableskew]{youngtab}

\begin{document}
\begin{Ej}
Draw the $\gsl_3$ crystal for weight $(3,3,0)$.
\end{Ej}

\begin{Ej}
    Prove that the elements of the hyperoctahedral group, written in cycle notation as a permutation on $\set{\pm1,\dots , \pm n}$, has all of its cycles coming in either pairs of the form $(a_1\dots a_k)(-a_1\dots -a_k)$, or of the form $(a_1\dots a_k -a_1\ -a_2\dots-a_k)$.
\end{Ej}

\begin{Ej}
    Define the Lie algebra $\gso_{2n+1}$ as $\set{X\:\ X^\sT S+SX=0}$ where 
    $$S=\threebythree{1}{0}{0}{0}{0_n}{I_n}{0}{I_n}{0_n}$$
    and $I_n$ is the $n\x n$ identity matrix and $1$ is in the upper left corner. Write down what an arbitrary element $X$ looks like, and using the fact that with respect to this setup the torus is simply the set of diagonal matrices $X$ satisfying these conditions, explain how one obtains the type $B$ root system.
\end{Ej}

\begin{ptcbr}
    First observe that the matrix $S$ is a permutation matrix which acts by row permutation when applied as left multiplication. The permutation it applies is a product of disjoint transpositions of the form $\onebytwo{i}{n+i}$ for $i\in[n+1]\less\set{1}$.
\end{ptcbr}

\begin{Ej}
    What is the dimension of the adjoint representation of $\gso_7$?
\end{Ej}


\begin{ptcbr}
    Via the isomorphism $X\mapsto\bonj{X,-}$ we have that the dimension of the adjoint representation is the same as $\dim\gso_7$ which is $\binom{7}{2}=21$.
\end{ptcbr}

\begin{Ej}
    Explain why the set of $5^{\text{th}}$ roots of unity in the plane don't form a root system. Which axioms of root systems does it satisfy?
\end{Ej}

\begin{ptcbr}
    The axioms we should check are:
    \begin{enumerate}
        \item The roots span our vector space.
        \item The reflections across hyperplanes are still roots.
        \item Projections onto the span of a single root are an integer multiple or a half-integer multiple of the root.
        \item If $\al,\bt$ are roots such that $\bt=\la\al$ then $\la=\pm1$. 
    \end{enumerate}
    The first axiom is satisfied as any non-zero complex number spans $\bC$. The last axiom is satisfied vacuously.\par
    The second axiom isn't satisfied as reflections across hyperplanes send the $5^{\text{th}}$ roots to $10^{\text{th}}$ (primitive) roots of unity. Projections are also not integer multiples nor half-integer multiples of other roots.
\end{ptcbr}
\begin{Ej}
    Compute the evacuation of the Young tableau below, and then evacuate again, and show you have returned to the starting tableau.
    $$\young(5,278,1346)$$
\end{Ej}

\begin{ptcbr}
We switch entries following the rule $k\mapsto n+1-k$ and then rotating $180^\circ$:
    $$\young(5,278,1346)\longto\young(4,721,8653)\longto\young(3568,\cross127,:::4).$$
Here we have already marked the first inner corner we will move. This leads us to
\newcommand{\gon}{\green{1}}
\newcommand{\gtw}{\green{2}}
\newcommand{\gth}{\green{3}}
\newcommand{\gfo}{\green{4}}
\newcommand{\gfiv}{\green{5}}
\newcommand{\gsi}{\green{6}}
\newcommand{\gse}{\green{7}}
\newcommand{\gei}{\green{8}}
$$\young(35\gei,\gon\gtw\gsi7,::\cross4)\longto\young(358,126,:\cross\gfo\gse)\longto\young(3\gei,1\gfiv6,\cross\gtw47)\longto\young(\gei,\gth56,\gon247)$$
where every \green{green} character moved when clearing out the inner corner in the previous step. Redoing the process we obtain the skew tableau
$$\young(8,356,1247)\longto\young(\gon,\gsi\gfo\gth,\gei\gse\gfiv\gtw)\longto\young(\gtw\gfiv\gse\gei,\cross\gth\gfo\gsi,:::\gon).$$
With the first inner corner marked, we move it out and continue the process:
$$\young(\gfiv\gse\gei,\gtw346,::\cross1)\longto\young(578,234,:\cross\gon\gsi)\longto\young(57,23\gei,\cross\gon\gfo6)\longto\young(5,2\gse8,\gon\gth46).$$
As we have returned to our original tableau we conclude that the process is correct.
\end{ptcbr}

\begin{Ej}
    Compute the Hall-Littlewood polynomial $\tilde{H}_{(2,1,1)}(x;q)$.
\end{Ej}

\begin{Ej}
    Let $w=w_1\dots w_n$ be a word of partition content, and suppose $w_1\neq 1$. Let $w'=w_2\dots w_nw_1$ be formed by cycling $w_1$ around to the end of the word. Show that $\cc(w')=\cc(w)-1$ where $\cc$ is cocharge. This operation is called \emph{cyclage}.
\end{Ej}

\begin{ptcbr}
Observe that it suffices to view this on standard words. This is because we may separate a word into standard subwords and calculate cocharge\footnote{Ah! Inadvertently \textbf{you} helped me with this problem as the decomposition idea was written on your thesis!}. Consider the subword $\tilde{w}$ of $w$ which contains $w_1$ in the previous decomposition sense, as $w$ has partition content so does $\tilde{w}$.\par
When cycling $w_1$ to the end of $\tilde{w}$, cocharge is reduced by $1$ as there is a element in $\tilde{w}$ smaller than $w_1$ which was to the right of $w_1$. After cycling, it's to the \emph{left} and so the cocharge labeling drops by one. 
\end{ptcbr}

\begin{Ej}
    Give a counterexample showing that the formula in the above problem does not hold in general when $w_1=1$.
\end{Ej}

\begin{ptcbr}
The word $121$ has cocharge labeling $000$ giving it a cocharge of $0$ whereas $211$ has cocharge labeling $100$ with cocharge $1$.
\end{ptcbr}
\end{document} 
