\documentclass[12pt]{memoir}

\def\nsemestre {II}
\def\nterm {Fall}
\def\nyear {2024}
\def\nprofesor {Maria Gillespie}
\def\nsigla {MATH601}
\def\nsiglahead {Advanced Combinatorics}
\def\nextra {HW1}
\def\nlang {ENG}
\input{../../headerVarillyDiff}
\usepackage{youngtab}

\begin{document}

\begin{Ej}[Exercise 1]
    Prove that all three definitions of representations of finite groups given in the lecture notes are equivalent. Then, for the examples of the groups $G$ and $H$ from Examples 2.1 and 2.2 in the lecture notes, express these representations as a vector space with an action, and as a module.
\end{Ej}

The definitions in question are:

\begin{Def}
A representation of a group $G$ over a field $\bF$ is a homomorphism 
$$\rho\: G\to\rGL_n(\bF)$$
where $\rGL_n(\bF)$ is the group of invertible $n\x n$ matrices over $\bF$.
\end{Def}

\begin{Def}
A representation of a group $G$ over a field $\bF$ is an $\bF$-vector space $V$ along with an action $G\lt V$ by linear transformations, i.e. a homomorphism $\rho\: G\to\rGL(V)$.
\end{Def}

\begin{Def}
A representation of a group $G$ over a field $\bF$ is an $\bF G$-module $V$. (Here $\bF G$ is the group ring consisting of formal linear combinations of elements of $G$ over $\bF$. A module is essentially a ``vector space over a ring''.)
\end{Def}

\begin{ptcbr}
Observe that the first definition gives rise to a map 
$$G\to\Aut(V),g\mapsto\rho(g)$$
where $V=\bF^n$ and $\rho(g)$ acts as a linear transformation on $v\in V$. This is an action because $\rho$ is a homomorphism, namely:
\begin{itemize}
    \item $\rho(e)=I_{n\x n}$.
    \item And $\rho(gh)=\rho(g)\rho(h)$ once again because $\rho$ is a homomorphism.
\end{itemize}
So naturally we get the definition of representation as a vector space.\par 
On the flipside any representation map can extend its domain to the group ring via linearity:
$$\wh{\rho}\left(\sum_{g\in G}c_gg\right)\defeq \sum_{g\in G}c_g\rho(g)$$
\end{ptcbr}


\end{document} 
