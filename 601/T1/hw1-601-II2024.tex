\documentclass[12pt]{memoir}

\def\nsemestre {II}
\def\nterm {Fall}
\def\nyear {2024}
\def\nprofesor {Maria Gillespie}
\def\nsigla {MATH601}
\def\nsiglahead {Advanced Combinatorics}
\def\nextra {HW1}
\def\nlang {ENG}
\input{../../headerVarillyDiff}
\usepackage{youngtab}

\begin{document}

\begin{Ej}[Exercise 1]
    Prove that all three definitions of representations of finite groups given in the lecture notes are equivalent. Then, for the examples of the groups $G$ and $H$ from Examples 2.1 and 2.2 in the lecture notes, express these representations as a vector space with an action, and as a module.
\end{Ej}

The definitions in question are:

\begin{Def}
A representation of a group $G$ over a field $\bF$ is a homomorphism 
$$\rho\: G\to\rGL_n(\bF)$$
where $\rGL_n(\bF)$ is the group of invertible $n\x n$ matrices over $\bF$.
\end{Def}

\begin{Def}
A representation of a group $G$ over a field $\bF$ is an $\bF$-vector space $V$ along with an action $G\lt V$ by linear transformations, i.e. a homomorphism $\rho\: G\to\rGL(V)$.
\end{Def}

\begin{Def}
A representation of a group $G$ over a field $\bF$ is an $\bF G$-module $V$. (Here $\bF G$ is the group ring consisting of formal linear combinations of elements of $G$ over $\bF$. A module is essentially a ``vector space over a ring''.)
\end{Def}
%https://math.stackexchange.com/questions/1961082/modules-over-group-algebra-representations-example
\begin{ptcbr}
    The first definition introduces a map 
    $$G \to \Aut(V), \quad g \mapsto \rho(g),$$
    where $V = \bF^n$, and $\rho(g)$ acts as a linear transformation on $v \in V$. This map defines an action since $\rho$ is a homomorphism, specifically:
    \begin{itemize}
        \item $\rho(e) = I_{n \times n}$.
        \item $\rho(gh) = \rho(g)\rho(h)$, by virtue that $\rho$ is a homomorphism.
    \end{itemize}
    This leads to the definition of a representation of $G$ on the vector space $V$.\par 
    Conversely, any $\bF G$-module is, by definition, also a vector space over $\bF$, given that $\bF \subseteq \bF G$. To equip this vector space with the structure of a $\bF G$-module, we define the action of elements of $\bF G$ using the representation. Specifically, for any $v \in V$ and $\sum_{g \in G} c_g g \in \bF G$, we define:
    $$\left(\sum_{g \in G} c_g g\right) \. v \defeq \sum_{g \in G} c_g [\rho(g)v].$$
    This shows how the representation can be viewed as an $\bF G$-module.
    \tcblower %https://tex.stackexchange.com/questions/291617/how-can-i-create-a-tcolorbox-with-line-numbering
    Now the groups in question are 
    \begin{gather*}
        G=\set{e,a,b,c\: ab=c,\ a^2=b^2=e}\isom(\bZ/2\bZ)^2\\
        H=\genr{a\:\ a^4=e}\isom\bZ/4\bZ
    \end{gather*}
  To realize the group representations as vector spaces with particular transformations let us follow the example in the notes. First, both groups can be represented by the space $\bF^2$.For the first group we have the transformations
  $$e\mapsto\id,\quad a\mapsto \twobytwo{-1}{0}{0}{1},\quad b\mapsto \twobytwo{1}{0}{0}{-1},\quad c\mapsto \twobytwo{-1}{0}{0}{-1}$$ 
  while for the cyclic group we have 
  $$e\mapsto \id,\word{and}a\mapsto \twobytwo{0}{-1}{1}{0}.$$
  In order to see the group rings, we will use the following facts:
  $$\bF(\bZ/n\bZ)\isom \quot{\bF[x]}{\genr{x^n-1}},\word{and}\bF(G\x H)\isom \bF G\ox_\bF\bF H.$$
    The group rings are thus 
    $$\quot{\bF[x]}{\genr{x^4-1}},\word{and}\left(\quot{\bF[x]}{\genr{x^2-1}}\right)^{\ox 2}$$
    and observe that in the case that $\bF$ is algebraically closed both group rings are just $\bF$.\par 
    In both cases, these are $\bF$-vector spaces to which we extend the matrix action.
    \end{ptcbr}

\begin{Ej}
    Consider the representation of $S_3$ in which each permutation $\pi\in S_3$ is sent to its corresponding permutation matrix $P$ , in which $P_{i,\pi(i)}=1$ for all $i$, and all other entries are $0$.
    \begin{enumerate}
        \item Find a common eigenvector of all of the permutation matrices.
        \item Write the representation as a direct sum of irreducible representations.
    \end{enumerate}
\end{Ej}
%https://math.stackexchange.com/questions/129227/cross-product-in-complex-vector-spaces
%https://www.wolframalpha.com/input?i=%7B%7B1%2C1%2C1%7D%2C%7B1%2C%28-1%2F2%2Bisqrt%283%29%2F2%29%2C%28-1%2F2%2Bisqrt%283%29%2F2%29%5E2%7D%2C%7B1%2C%28-1%2F2%2Bisqrt%283%29%2F2%29%5E2%2C%28-1%2F2%2Bisqrt%283%29%2F2%29%7D%7D
\begin{ptcbr}
    The matrices in question are 
    \begin{gather*}
        \id\mapsto I_3,\quad (12)\mapsto\threebythree{0}{1}{0}{1}{0}{0}{0}{0}{1},\quad (123)\mapsto\threebythree{0}{0}{1}{1}{0}{0}{0}{1}{0}\\
        (13)\mapsto\threebythree{0}{0}{1}{0}{1}{0}{1}{0}{0},\quad (23)\mapsto\threebythree{1}{0}{0}{0}{0}{1}{0}{1}{0},\quad(132)\mapsto\threebythree{0}{1}{0}{0}{0}{1}{1}{0}{0}.
    \end{gather*}
    We can see that all matrices share the eigenvector $(1,1,1)$. This can be seen in two ways:
    \begin{itemize}
        \item If a matrix's rows sum all to the same value, then the diagonal vector is an eigenvector with eigenvalue equal to the value of the row sum. In our case, the matrices all have row sum equal to $1$, so not only $(1,1,1)$ is an eigenvector, but it's a fixed point.
        \item The action of a matrix corresponding to $\sg\in S^3$ on a vector $(v_1,v_2,v_3)$ is to permute the entries to get $(v_{\sg(1)},v_{\sg(2)},v_{\sg(3)})$. The vector $(1,1,1)$ is invariant under that action so it must be an eigenvector.
    \end{itemize}
    \tcblower
    In order to decompose the representation into irreducibles we look at the orthogonal complement of $(1,1,1)$. In $\bC^3$ we can take $\vec a=(1,\om,\om^2)$ as one of the orthogonal vectors because 
    $$(1,1,1)\.(1,\om,\om^2)=\braket{(1,1,1)}{\ov{(1,\om,\om^2)}}=1+\om^2+\om=0,$$
    where the dot represents the Hermitian inner product in $\bC$ and the brackets denote the usual inner product. From this, our third vector comes easily as the cross product of these two:
    \begin{align*}
    (1,1,1)\x(1,\om,\om^2)&=(\ov{\om^2-\om},\ov{1-\om^2},\ov{\om-1})\\
    &=(\om-\om^2,1-\om,\om^2-1)\\
    &=(1-\om)(\om,1,-\om-1)
    \end{align*}
    We must conjugate the entries in the cross product to obtain consistency with the inner product. But also, we can rescale the result as desired because it will still be orthogonal. Observe that
    $$(\om,1,-\om-1)=(\om,1,\om^2)\xrightarrow[]{\.\om^2}(1,\om^2,\om^4)=(1,\om^2,\om).$$
    Let $\vec b=(1,\om^2,\om)$ and with this call $\cB=\set{\vec 1,\vec a,\vec b}$ our basis. The change of basis matrix is 
    $$M\defeq [\id]_{\cB}^{\cC}=\threebythree{1}{1}{1}{1}{\om}{\om^2}{1}{\om^2}{\om}$$
    and an inverse can be found easily as 
    $$\threebythree{1}{1}{1}{1}{\om}{\om^2}{1}{\om^2}{\om}\threebythree{1}{1}{1}{1}{\om^2}{\om}{1}{\om}{\om^2}=\threebythree{3}{0}{0}{0}{3}{0}{0}{0}{3}$$
    so that the inverse is a third of the second matrix. Conjugating each of our original matrices $A$ via $M^{-1}AM$ we get:
    \begin{gather*}
        \id\mapsto I_3,\quad (12)\mapsto\threebythree{1}{0}{0}{0}{0}{\om^2}{0}{\om}{0},\quad (123)\mapsto\threebythree{1}{0}{0}{0}{\om^2}{0}{0}{0}{\om}\\
        (13)\mapsto\threebythree{1}{0}{0}{0}{0}{\om}{0}{\om^2}{0},\quad (23)\mapsto\threebythree{1}{0}{0}{0}{0}{1}{0}{1}{0},\quad(132)\mapsto\threebythree{1}{0}{0}{0}{\om}{0}{0}{0}{\om^2}.
    \end{gather*}
    These matrices decompose into block diagonal form giving us the decomposition of our representation into the trivial representation (1-dimensional) and the standard representation (2-dimensional).\footnote{I must admit I was fifty-fifty between saying regular and standard. I don't remember the names very well. I was also going to do the $V$ with a tableau subindex, but it went horribly wrong because the tableau appeared like HUGE and it was terrible.}
\end{ptcbr}
\begin{Ej}
    Let $N$ be any open neighborhood of the identity element $e$ in a connected Lie group $G$.
Show that $N$ generates $G$ as a group.
\end{Ej}


%https://math.stackexchange.com/questions/1339399/a-bijective-continuous-map-is-a-homeomorphism-iff-it-is-open-or-equivalently-i
%https://math.stackexchange.com/questions/3161917/left-multiplication-is-homeomorphism-of-topological-groups
%https://math.stackexchange.com/questions/359693/overview-of-basic-results-about-images-and-preimages
%https://math.stackexchange.com/questions/4281287/a-set-u-is-neighborhood-of-g-iff-g-1u-is-neighborhood-of-identity-eleme?rq=1
%https://proofwiki.org/wiki/Definition:Neighborhood_(Topology)/Set
%https://proofwiki.org/wiki/Intersection_with_Complement_is_Empty_iff_Subset

Before proceeding with the proof, let's briefly review the multiplication map in a topological group and recall some essential properties of images and inverse images.

\begin{Rmk}
In any topological group, the multiplication map is a homeomorphism. Specifically, the maps 
$$x \mapsto gx \word{and} x \mapsto g^{-1}x$$
are inverses of each other and are continuous by definition. Additionally, if $f$ is a bijective function, then 
$$f^{-1}[f[A]] = f[f^{-1}[A]] = A.$$
With these concepts in mind, we can now state and prove the following lemma.
\end{Rmk}

\begin{Lem}
For a topological group $G$, a subset $U \subseteq G$ is a neighborhood of an element $g \in G$ if and only if $g^{-1}U$ is a neighborhood of the identity element $e$.
\end{Lem}

\begin{ptcbp}
Define the map $m_g$ by $x\mapsto gx$. Notice that 
$$g^{-1}U = m_{g^{-1}}[U].$$ 
Since $m_g$ is a homeomorphism, $U$ is open if and only if $g^{-1}U$ is open.
\end{ptcbp}

\begin{ptcbr}
Now, consider the subgroup $\genr{N}$ generated by $N$. This subgroup is non-empty, as it contains at least the identity element $e$.

\begin{itemize}
    \item First, we show that $\genr{N}$ is open by proving that it's a neighborhood of each of its elements. Let $x \in \genr{N}$. Since $N$ is a neighborhood of $e$, the set $xN$ is a neighborhood of $x$ by our key lemma. Observe that $xN \subseteq \genr{N}$, as products of elements of $N$ with $x$ definitely are products of elements of $N$ to begin with. This means that $\genr{N}$ is open.
    
    \item Next, we prove that $\genr{N}$ is closed by showing that its complement is open. Suppose $x \notin \genr{N}$. The set $xN$ is a neighborhood of $x$, and we aim to demonstrate that it's entirely contained outside of $\genr{N}$. Assume, for the sake of contradiction, that 
    $$\genr{N} \cap xN \neq \emptyset.$$
    Then there exists an element 
    $$z \in \genr{N} \cap xN.$$
    Being in $xN$ means that $z = xy$ for some $y \in N$. Consequently, $x = zy^{-1}$, and since both $z$ and $y^{-1}$ belong to $\genr{N}$, it follows that $x \in \genr{N}$. This contradicts the assumption that $x$ was outside $\genr{N}$. Therefore, $xN$ does not intersect $\genr{N}$, meaning the complement of $\genr{N}$ is open, so $\genr{N}$ is closed.
\end{itemize}

In conclusion, the subgroup $\genr{N}$ is both open and closed, and it's non-empty. In a connected topological space, the only set with these properties is the entire space. Hence, we conclude that $\genr{N} = G$, as required.
\end{ptcbr}

\end{document} 
