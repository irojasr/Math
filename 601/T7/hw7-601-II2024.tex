\documentclass[12pt]{memoir}

\def\nsemestre {II}
\def\nterm {Fall}
\def\nyear {2024}
\def\nprofesor {Maria Gillespie}
\def\nsigla {MATH601}
\def\nsiglahead {Advanced Combinatorics}
\def\nextra {HW7}
\def\nlang {ENG}
\input{../../headerVarillyDiff}
\usepackage{youngtab}

\begin{document}

\begin{Ej}[Exercise 1.a]
Compute $s_2(T)$ where $T$ is the tableau below:
$$\young(34,22233,1111222233).$$    
\end{Ej}

\begin{ptcbr}
    We have 
    $$\rw(T)=34222331111222233.$$
    With this, we can pair $3$ and $2$'s as follows:
    $$\un{342}22\ov{3\un{311112}2}2233$$
    and we can see we have $4$ unpaired $2$'s and $2$ unpaired $3$'s. By applying $s_2$ we wish to reflect this about the corresponding $\gsl_2$ chain to get $2$ unpaired $2$'s and $4$ $3$'s. We have 
    $$34222331111222233\xrightarrow{F_2}34222331111222\red{3}33\xrightarrow{F_2}3422233111122\red{33}33$$
    and this is the desired word. The corresponding tableau is 
    $$\young(34,22233,1111223333)$$
\end{ptcbr}

\begin{Ej}[Exercise 1.b]
    Explain why it suffices to show that $s_is_{i+1}s_i=s_{i+1}s_is_{i+1}$ when acting on tableaux, for all $i$.
\end{Ej}

\begin{ptcbr}
Checking a braid relation of the form $s_is_js_i$ where $|j-i|\neq 1$ is unnecessary because in those cases $s_i$ commutes with $s_j$.    
\end{ptcbr}

\begin{Ej}[Exercise 1.c]
    Show that, using the compatibility of JDT slides with crystal operators (which you may use as a fact), it suffices to show part 1 for $i = 1$, and therefore it suffices to work with $\gsl_3$ crystals, that is, tableaux whose letters are all $1, 2, 3$.
\end{Ej}

\begin{ptcbr}
Consider the following diagram:
\begin{center}
    % https://tikzcd.yichuanshen.de/#N4Igdg9gJgpgziAXAbVABwnAlgFyxMJZABgBpiBdUkANwEMAbAVxiRABUQBfU9TXfIRQBGclVqMWbADrScMAB45gcANYwA7lwAU7AJTdeIDNjwEio4ePrNWiELPlKV6rdoDKBnn1OCiZK2obKXt3Qx8BcxQAJlJAiVs2dwBycON+MyFkWMogyTsOVK5xGCgAc3giUAAzACcIAFskMhAcCCQAZmoGOgAjGAYABQy-e1qsMoALHBA8xPtZCDQYWro22rA6BphgXtWsKC40usbO6jakWISQh2kllbWIDa2d2pgGiBoYI+6+geHfFEQOMpjNvCATk1EFcLogACxzG6yN4AYxmv36QxGQJB02O9Shola7UQAFZEQVkTA0fjTogWrCicFKXdlqt1pttsA3h8vkdwZCkOTiUgEdcWfd2U9OTs9nQDj8QD1MQDIkIlTBqmCKFwgA
\begin{tikzcd}
    T \arrow[d, "\operatorname{braid}"'] \arrow[r, "\operatorname{rem}"] & \text{skew}(T) \arrow[r, "\rect"]  & T' \arrow[d, "\operatorname{braid}"] \\
    S \arrow[r, "\operatorname{rem}"']                                   & \text{skew}(S) \arrow[r, "\rect"'] & S'                                  
    \end{tikzcd}
\end{center}
where $\operatorname{rem}$ is the operation which removes all letters but $i,i+1,i+2$. This diagram commutes because the braid operation is a combination of raising and lowering operators and the removal plus rectification is a JDT.\par
So from this, we may relabel $i,i+1,i+2$ to $1,2,3$ and then we can work on the tableau because the diagram commutes.
\end{ptcbr}

\begin{Ej}[Exercise 2]
    Compute the chromatic symmetric function of the triangle graph, that is, the complete graph $K_3$, and express it in terms of elementary symmetric functions and in terms of Schur functions.
\end{Ej}

\begin{ptcbr}
Observe that $\chi(K_3)=3$ which means that there's no proper colorings with $1$ or $2$ colors. Thus we must color vertices $1,2,3$ with colors $i,j,k\in\bN$. However there's $3!$ ways of doing this, so that each monomial $x_ix_jx_k$ is accounted $3!$ times. We thus have that 
$$X_{K_3}=3!m_{(1,1,1)}=3!s_{(1,1,1)}=3!e_3.$$
\end{ptcbr}
\end{document} 