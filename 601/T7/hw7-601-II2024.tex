\documentclass[12pt]{memoir}

\def\nsemestre {II}
\def\nterm {Fall}
\def\nyear {2024}
\def\nprofesor {Maria Gillespie}
\def\nsigla {MATH601}
\def\nsiglahead {Advanced Combinatorics}
\def\nextra {HW7}
\def\nlang {ENG}
\input{../../headerVarillyDiff}
\usepackage[enableskew]{youngtab}

\begin{document}

\begin{Ej}[Exercise 1.a]
Compute $s_2(T)$ where $T$ is the tableau below:
$$\young(34,22233,1111222233).$$    
\end{Ej}

\begin{ptcbr}
    We have 
    $$\rw(T)=34222331111222233.$$
    With this, we can pair $3$ and $2$'s as follows:
    $$\un{342}22\ov{3\un{311112}2}2233$$
    and we can see we have $4$ unpaired $2$'s and $2$ unpaired $3$'s. By applying $s_2$ we wish to reflect this about the corresponding $\gsl_2$ chain to get $2$ unpaired $2$'s and $4$ $3$'s. We have 
    $$34222331111222233\xrightarrow{F_2}34222331111222\red{3}33\xrightarrow{F_2}3422233111122\red{33}33$$
    and this is the desired word. The corresponding tableau is 
    $$\young(34,22233,1111223333)$$
\end{ptcbr}

\begin{Ej}[Exercise 1.b]
    Explain why it suffices to show that $s_is_{i+1}s_i=s_{i+1}s_is_{i+1}$ when acting on tableaux, for all $i$.
\end{Ej}

\begin{ptcbr}
Checking a braid relation of the form $s_is_js_i$ where $|j-i|\neq 1$ is unnecessary because in those cases $s_i$ commutes with $s_j$.    
\end{ptcbr}

\begin{Ej}[Exercise 1.c]
    Show that, using the compatibility of JDT slides with crystal operators (which you may use as a fact), it suffices to show part 1 for $i = 1$, and therefore it suffices to work with $\gsl_3$ crystals, that is, tableaux whose letters are all $1, 2, 3$.
\end{Ej}

\begin{ptcbr}
Consider the following diagram:
\begin{center}
    % https://tikzcd.yichuanshen.de/#N4Igdg9gJgpgziAXAbVABwnAlgFyxMJZABgBpiBdUkANwEMAbAVxiRABUQBfU9TXfIRQBGclVqMWbADrScMAB45gcANYwA7lwAU7AJTdeIDNjwEio4ePrNWiELPlKV6rdoDKBnn1OCiZK2obKXt3Qx8BcxQAJlJAiVs2dwBycON+MyFkWMogyTsOVK5xGCgAc3giUAAzACcIAFskMhAcCCQAZmoGOgAjGAYABQy-e1qsMoALHBA8xPtZCDQYWro22rA6BphgXtWsKC40usbO6jakWISQh2kllbWIDa2d2pgGiBoYI+6+geHfFEQOMpjNvCATk1EFcLogACxzG6yN4AYxmv36QxGQJB02O9Shola7UQAFZEQVkTA0fjTogWrCicFKXdlqt1pttsA3h8vkdwZCkOTiUgEdcWfd2U9OTs9nQDj8QD1MQDIkIlTBqmCKFwgA
\begin{tikzcd}
    T \arrow[d, "\operatorname{braid}"'] \arrow[r, "\operatorname{rem}"] & \text{skew}(T) \arrow[r, "\rect"]  & T' \arrow[d, "\operatorname{braid}"] \\
    S \arrow[r, "\operatorname{rem}"']                                   & \text{skew}(S) \arrow[r, "\rect"'] & S'                                  
    \end{tikzcd}
\end{center}
where $\operatorname{rem}$ is the operation which removes all letters but $i,i+1,i+2$. This diagram commutes because the braid operation is a combination of raising and lowering operators and the removal plus rectification is a JDT.\par
So from this, we may relabel $i,i+1,i+2$ to $1,2,3$ and then we can work on the tableau because the diagram commutes.
\end{ptcbr}

\begin{Ej}[Exercise 1.d]
    Show that one can further reduce to the case that the tableau shape has two rows.
\end{Ej}

\begin{ptcbr}
    Given the previous facts, we can reduce our case to a $3$-row tableau on the alphabet $\set{1,2,3}$. Observe that it can't have more than $3$ rows due to semistandardness. Now the third row must be comprised of $3$'s who should be paired with $2$'s below them and further those with $1$'s.\par
    As $s_1,s_2$ act by raising/lowering, there's no way that elements on columns with more than two rows could get affected by the $s_i$'s. So it suffices to work on tableau with only two rows as it will be the only part affected by $s_i$'s.
\end{ptcbr}

\begin{Ej}[Exercise 1.e]
    Show that, using the symmetry in the relation $(s_1s_2)^3=1$, it suffices to consider the
case when the tableau has partition weight, that is, its content is $(a, b, c)$ for some $a\geq b\geq c$.
\end{Ej}

\begin{ptcbr}
    Observe that the possible contents for our words are the $6$ permutations of the word $abc$ arranged into a content vector, given the condition that $a\geq b\geq c$. Applying $s_1$ will switch the number of 1's and 2's whereas $s_2$ does that 2's and 3's.\par
    From this we have 
    \begin{align*}
    \to(a,b,c)\xrightarrow{s_1}(b,a,c)\xrightarrow{s_2}(b,c,a)\xrightarrow{s_1}(c,b,a)\xrightarrow{s_2}(c,a,b)\xrightarrow{s_1}(a,c,b)\xrightarrow{s_2}(a,b,c)\to
    \end{align*}
    which means that all possible contents are obtained after applying $s_i$'s. Since starting from any point gets us any content, it suffices to only consider those tableau with content $(a,b,c)$.
\end{ptcbr}

\begin{Ej}[Exercise 1.f]
    Show that the reading word of such tableau must be of the form $2^d3^e1^a2^f3^g$ where $d + f = b$ and $e + g = c$.
\end{Ej}

\begin{ptcbr}
   Given that our tableau has two rows and content $(a,b,c)$ it must occur that it looks like
   $$\young(?,11\dots1?)$$
   where there are $a$ ones in a row. Twos can be placed on top of our row of ones or to the side:
   $$\young(2\dots2?,11\dots12\dots2?).$$
   There can be no third row on top with threes which means that the reading word doesn't start with $3$. Our three's can only be placed in following the twos as
   $$\young(2\dots23\dots3,11\dots12\dots23\dots 3)$$
   which means that the reading word should be a string of twos, one of threes, the a string of one and a couple more of twos and threes.\par
   The given conditions, $d + f = b$ and $e + g = c$ guarantee that there's no overlap between the rows or that the tableau looks like 
   $$\young(3333,1122)$$
   as that will not have content $(a,b,c)$ with $a\geq b\geq c$.

\end{ptcbr}
\begin{Ej}[Exercise 2]
    Compute the chromatic symmetric function of the triangle graph, that is, the complete graph $K_3$, and express it in terms of elementary symmetric functions and in terms of Schur functions.
\end{Ej}

\begin{ptcbr}
Observe that $\chi(K_3)=3$ which means that there's no proper colorings with $1$ or $2$ colors. Thus we must color vertices $1,2,3$ with colors $i,j,k\in\bN$. However there's $3!$ ways of doing this, so that each monomial $x_ix_jx_k$ is accounted $3!$ times. We thus have that 
$$X_{K_3}=3!m_{(1,1,1)}=3!s_{(1,1,1)}=3!e_3.$$
\end{ptcbr}
\end{document} 