\documentclass[12pt]{memoir}

\def\nsemestre {II}
\def\nterm {Fall}
\def\nyear {2024}
\def\nprofesor {Maria Gillespie}
\def\nsigla {MATH601}
\def\nsiglahead {Advanced Combinatorics}
\def\nextra {HW6}
\def\nlang {ENG}
\input{../../headerVarillyDiff}
\usepackage{youngtab}

\begin{document}
\iffalse
\begin{Ej}[Exercise 1]
    Determine the dimension of the adjoint representation of $\ggl_n$.
\end{Ej}

\begin{ptcbr}
    The dimension of the adjoint representation can be obtained via the isomorphism of vector spaces $X\mapsto\bonj{X,-}$. This means that the dimension of the adjoint representation of $\ggl_n$ is $n^2$.
\end{ptcbr}
\fi

In this homework I was able to collaborate with \textbf{Kylie, Nate} and \textbf{Parker}. It was quite the endeavor, but I really like keeping track of my indices and bracketing elementary matrices.
\begin{Ej}[Exercise 3]
    Determine the dimension of the adjoint representation of $\gso_{2n+1}$.
\end{Ej}

\begin{ptcbr}
    The dimension of the adjoint representation can be seen to be the same as the dimension of the original space by considering the isomorphism $X\mapsto\bonj{X,-}$. Then the dimension of the adjoint representation is the same as $\gso_{2n+1}$'s so it's
    $$\binom{2n+1}{2}=(2n+1)n.$$
\end{ptcbr}

\begin{Ej}[Exercise 4]
    Write out a basis for the adjoint representation of $\gso_5$ and show how it corresponds to the root system in type $B$.
\end{Ej}

\begin{ptcbr}
    We may take advantage of the relationship $X^\sT S+SX=0$ where $$S=\threebythree{1}{0}{0}{0}{0_2}{I_2}{0}{I_2}{0_2}$$
    to determine the basic elements. Assume $X=(x_{ij})\in\gso_5$, then $SX$ permutes rows $2$ with $4$ and $3$ with $5$. Observe that transposing $SX$ returns $X^\sT S$. So the relation 
    $X^{\sT}S+SX=0$ allows us to rewrite the matrix $X$ as
    $$
    \begin{pmatrix}
        0&x_{12}&x_{13}&x_{14}&x_{15}\\
        -x_{14}&x_{22}&x_{23}&0&x_{25}\\
        -x_{15}&x_{32}&x_{33}&-x_{25}&0\\
        -x_{12}&0&x_{43}&-x_{22}&-x_{32}\\
        -x_{13}&-x_{43}&0&-x_{23}&-x_{33}
    \end{pmatrix}
    $$
    Now, let us use the notation $E_{ij}$ to denote the matrix whose $(i,j)^{\text{th}}$ entry is $1$ and 0 otherwise. The basis for $\gso_{5}$ may be written as:
    \begin{multicols}{2}
        \begin{itemize}
            \item $X_1=E_{12}-E_{41}$
            \item $X_2=E_{13}-E_{51}$
            \item $X_3=E_{14}-E_{21}$
            \item $X_4=E_{15}-E_{31}$
            \item $X_5=E_{22}-E_{44}$
            \item $X_6=E_{23}-E_{54}$
            \item $X_7=E_{25}-E_{34}$
            \item $X_8=E_{32}-E_{45}$
            \item $X_9=E_{33}-E_{55}$
            \item $X_{10}=E_{43}-E_{52}$
        \end{itemize}
    \end{multicols}
    The Cartan subalgebra is generated by the matrices $X_5$ and $X_9$ and in order to identify our elements with type $B$ root system we must take the Lie bracket of them with $H=aX_5+bX_9$. We will apply the formula 
    $$\bonj{E_{ij},E_{k\l}}=\dl_{jk}E_{i\l}-\dl_{i\l}E_{kj}$$
    and with this we obtain
    \begin{align*}
        \bonj{H,X_1}&=\bonj{aX_5+bX_9,X_1}\\
        &=a\bonj{X_5,X_1}+b\bonj{X_9,X_1}\\
        &=a\bonj{E_{22}-E_{44},E_{12}-E_{41}}+b\bonj{E_{33}-E_{55},E_{12}-E_{41}}\\
        &=a(-E_{12}-0-0+E_{41})+0\\
        &=-a(E_{12}-E_{41})=-aX_1
    \end{align*}
    from which we deduce that $X_1$ matches with $-L_1$. As another example we compute
    \begin{align*}
        \bonj{H,X_6}&=\bonj{aX_5+bX_9,X_6}\\
        &=a\bonj{X_5,X_6}+b\bonj{X_9,X_6}\\
        &=a\bonj{E_{22}-E_{44},E_{23}-E_{54}}+b\bonj{E_{33}-E_{55},E_{23}-E_{54}}\\
        &=a(E_{23}-0-0+(-E_{54}))+b(-E_{23}-0-0+E_{54})\\
        &=a(E_{23}-E_{54})-b(E_{23}-E_{54})=(a-b)X_6
    \end{align*}
    and this tells us that $X_6$ corresponds to $L_1-L_2$. Similarly
    \begin{itemize}
        \item $\bonj{H,X_2}=-bX_2\To X_2\leftrightarrow-L_2$
        \item $\bonj{H,X_3}=aX_3\To X_3\leftrightarrow L_1$
        \item $\bonj{H,X_4}=bX_4\To X_4\leftrightarrow L_2$
        \item $\bonj{H,X_7}=(a+b)X_7\To X_7\leftrightarrow L_1+L_2$
        \item $\bonj{H,X_8}=(-a+b)X_8\To X_8\leftrightarrow -L_1+L_2$
        \item $\bonj{H,X_8}=(-a-b)X_{10}\To X_{10}\leftrightarrow -L_1-L_2$
    \end{itemize}
    With this, we have our desired correspondence.
\end{ptcbr}

\begin{Ej}[Exercise 6]
    Show that the Killing form on $\gsl_n$ satisfies $\braket{X}{Y}=2n\tr(XY)$ in general for any elements of $\gsl_n$.
\end{Ej}
%https://math.stackexchange.com/questions/135567/what-is-the-killing-form-of-mathfrakgl-m?noredirect=1&lq=1
\begin{ptcbr}
    For this problem we will find the Killing form on $\ggl_n$ and then specialize to the case of $\gsl_n$. This will be done to only deal with one type of basic element $E_{ij}$ instead of $E_{ij}$ and $H_k$.\par
    Take $Y=\sum_{k,\l=1}^n y_{k\l}E_{k\l}$ and $X=\sum_{r,s=1}^n x_{rs}E_{rs}$, then it suffices to view the action of $\bonj{X,\bonj{Y,-}}$ on a basic element $E_{ij}$. So let us fix $(i,j)$ and observe that 
    \begin{align*}
        \bonj{Y,E_{ij}}&=\bonj{\sum_{k,\l=1}^n y_{k\l}E_{k\l},E_{ij}}\\
        &=\sum_{k,\l=1}^n y_{k\l}\bonj{E_{k\l},E_{ij}}\\
        &=\sum_{k,\l=1}^n y_{k\l}(\dl_{i\l}E_{kj}-\dl_{kj}E_{i\l})\\
        &=\sum_{k,\l=1}^n y_{k\l}\dl_{i\l}E_{kj}-\sum_{k,\l=1}^ny_{k\l}\dl_{kj}E_{i\l}\\
        &=\sum_{k=1}^n y_{ki}E_{kj}-\sum_{\l=1}^ny_{j\l}E_{i\l}
        %&=\sum_{k=1}^n y_{ki}\sum_{\l=1}^n\dl_{j\l}E_{k\l}-\sum_{\l=1}^ny_{j\l}\sum_{k=1}^n\dl_{ik}E_{k\l}
    \end{align*}
    Reindexing the second sum we get 
    $$\bonj{Y,E_{ij}}=\sum_{k=1}^n y_{ki}E_{kj}-y_{jk}E_{ik}$$
    In a similar fashion we apply $\Ad(X)$ to this matrix in order to obtain
    $$\bonj{X,\bonj{Y,E_{ij}}}=\sum_{k=1}^n y_{ki}\bonj{X,E_{kj}}-y_{jk}\bonj{X,E_{ik}}$$
    and to not lose track of indices we will do the brackets separately
    $$
    \left\lbrace
    \begin{aligned}
        &\bonj{X,E_{kj}}=\sum_{r,s=1}^{n}x_{rs}\bonj{E_{rs},E_{kj}}=\sum_{r=1}^{n}x_{rk}E_{rj}-\sum_{s=1}^nx_{js}E_{ks}=\sum_{r=1}^{n}x_{rk}E_{rj}-x_{jr}E_{kr},\\
        &\bonj{X,E_{ik}}=\sum_{r,s=1}^{n}x_{rs}\bonj{E_{rs},E_{ik}}=\sum_{r=1}^{n}x_{ri}E_{rk}-\sum_{s=1}^nx_{ks}E_{is}=\sum_{r=1}^{n}x_{ri}E_{rk}-x_{kr}E_{ir}.
    \end{aligned}
    \right.
    $$
    Putting everything back together we get
    \begin{align*}
        &\bonj{X,\bonj{Y,E_{ij}}}\\
        =&\sum_{k=1}^n y_{ki}\left(\sum_{r=1}^{n}x_{rk}E_{rj}-x_{jr}E_{kr}\right)-y_{jk}\left(\sum_{r=1}^{n}x_{ri}E_{rk}-x_{kr}E_{ir}\right)\\
        =&\sum_{k=1}^n\sum_{r=1}^{n}y_{ki}x_{rk}E_{rj}-\sum_{k=1}^n\sum_{r=1}^{n}y_{ki}x_{jr}E_{kr}-\sum_{k=1}^n\sum_{r=1}^{n}y_{jk}x_{ri}E_{rk}+\sum_{k=1}^n\sum_{r=1}^{n}y_{jk}x_{kr}E_{ir}
    \end{align*}
    Now looking for the coefficient of $E_{ij}$ in order to later ask for the trace we get
    $$
    \left\lbrace
    \begin{aligned}
        &\sum_{k=1}^ny_{ki}x_{ik}\quad\text{by looking at the }r=i\text{ term of the first sum}.\\
        &-y_{ii}x_{jj}\word{from the second sum},\\
        &-y_{jj}x_{ii}\word{from the third, and}\\
        &\sum_{k=1}^ny_{jk}x_{kj}\quad\text{on the }j^{\text{th}}\text{ term of the last sum}.
    \end{aligned}
    \right.
    $$
    Now, the trace of $\Ad(X)\Ad(Y)$ would be obtained by summing over the diagonal entries of the matrix. But observe that we haven't \emph{flattened} any matrices at any point (in the sense that we haven't converted a matrix $A=(A_ij)$ into a vector $a_{(i-1)n+j}=A_{ij}$). So, even if 
    $$\Ad(X)\Ad(Y)\:\bC^{n^2}\to\bC^{n^2}$$
    and we could calculate the trace by summing the diagonal entries of the expression, this will instead be done by summing across all $(i,j)$. We could also interpret $\Ad(X)\Ad(Y)$ as a rank 3 tensor and then take the trace by summing across $2$ dimensions.\par
    From the previous discussion, we have that 
    \begin{align*}
        \tr(\Ad(X)\Ad(Y))&=\sum_{i,j=1}^n\left(\sum_{k=1}^ny_{ki}x_{ik}-y_{ii}x_{jj}-y_{jj}x_{ii}+\sum_{k=1}^ny_{jk}x_{kj}\right)\\
        &=\sum_{i,j,k=1}^ny_{ki}x_{ik}-\sum_{i,j=1}^ny_{ii}x_{jj}-\sum_{i,j=1}^ny_{jj}x_{ii}+\sum_{i,j,k=1}^ny_{jk}x_{kj}\\
        &=n\sum_{i,k=1}^ny_{ki}x_{ik}-\sum_{i=1}^ny_{ii}\sum_{j=1}^nx_{jj}-\sum_{j=1}^ny_{jj}\sum_{i=1}^nx_{ii}+n\sum_{j,k=1}^ny_{jk}x_{kj}\\
        &=n\tr(YX)-\tr(Y)\tr(X)-\tr(Y)\tr(X)+n\tr(YX)\\
        &=2n\tr(YX)-2\tr(Y)\tr(X)
    \end{align*}
    which gives us the identity
    $$\tr(\Ad(X)\Ad(Y))=2n\tr(YX)-2\tr(Y)\tr(X)$$
    and now, specializing to the case of $\gsl_n$, we have that $X,Y$ have zero trace so that the identity becomes
    $$\braket{X}{Y}=\tr(\Ad(X)\Ad(Y))=2n\tr(XY).$$
\end{ptcbr}

\begin{Ej}[Exercise 5]
    Generalize the computation we did in class to show that the Killing form for $\gsl_n$, when
restricted to the Cartan subalgebra $\lie{h}$, satisfies
\end{Ej}

\begin{ptcbr}
    I sadly didn't see the computation in class and thought that doing it the hard way around would be the best approach. The previous identity holds in all of $\gsl_n$ so in particular it holds for $\lie h^{\gsl}\leq \gsl_n$.
\end{ptcbr}
\begin{Ej}[Exercise 9]
    What is the size of the Weyl group of type $G_2$? Write out its elements as reduced words in
    the two simple reflections $s_1, s_2$ corresponding to the two simple roots of $G_2$.
\end{Ej}

\begin{ptcbr}
This Weyl group can be presented as
$$\quot{\gen(s_1,s_2)}{\genr{s_i^2,(s_1s_2)^6}}.$$
We may thus enumerate the elements of the group as 
\begin{multicols}{2}
    \begin{itemize}
        \item $e=(s_1s_2)^6=(s_2s_1)^6$
        \item $s_1$
        \item $s_2$
        \iffalse
        \item $s_1^{-1}=s_2(s_1s_2)^5$
        \item $s_2^{-1}=s_1(s_2s_1)^5$
        \fi
        \item $s_1s_2$
        \item $s_2s_1$
        \iffalse
        \item $(s_1s_2)^{-1}=(s_1s_2)^5$
        \item $(s_2s_1)^{-1}=(s_2s_1)^5$
        \fi
        \item $s_1s_2s_1$
        \item $s_2s_1s_2$
        \iffalse
        \item $(s_1s_2s_1)^{-1}=(s_2s_1s_2)(s_1s_2)^3$
        \item $(s_2s_1s_2)^{-1}=(s_1s_2s_1)(s_2s_1)^3$
        \fi
        \item $s_1s_2s_1s_2$
        \item $s_2s_1s_2s_1$
        \iffalse
        \item $(s_1s_2s_1s_2)^{-1}=(s_1s_2)^4$
        \item $(s_2s_1s_2s_1)^{-1}=(s_2s_1)^4$
        \fi
        \item $s_1s_2s_1s_2s_1$
        \item $s_2s_1s_2s_1s_2$
        \iffalse
        \item $(s_1s_2s_1s_2s_1)^{-1}=s_2(s_1s_2)^3$
        \item $(s_2s_1s_2s_1s_2)^{-1}=s_1(s_2s_1)^3$
        \fi
        \item $s_1s_2s_1s_2s_1s_2=(s_2s_1)^3$
    \end{itemize}
\end{multicols}
Observe that the remaining elements are determined by the braid relation.
\end{ptcbr}
\end{document} 