\documentclass[12pt]{memoir}

\def\nsemestre {II}
\def\nterm {Fall}
\def\nyear {2024}
\def\nprofesor {Maria Gillespie}
\def\nsigla {MATH601}
\def\nsiglahead {Advanced Combinatorics}
\def\nextra {HW4}
\def\nlang {ENG}
\input{../../headerVarillyDiff}
\usepackage{youngtab}

\begin{document}

\begin{Ej}
    Describe a ``ballot-type'' condition for a word of 1's and 2's to be lowest weight for $\gsl_2$, that is, that $F$ sends the word to $0$. Prove that your condition is correct. Do the same for $\gsl_3$ and the two lowering operators.
\end{Ej}

\begin{ptcbr}
   Recall a word is Yamanouchi when every suffix has partition content. This occures when reading right-to-left, $\#i's\geq\#i+1's$. The reverse condition of this is precisely being a ballot word, when every prefix left-to-right has more $i+1$'s than $i$'s.
\end{ptcbr}

\end{document} 
