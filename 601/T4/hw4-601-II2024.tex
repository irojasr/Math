\documentclass[12pt]{memoir}

\def\nsemestre {II}
\def\nterm {Fall}
\def\nyear {2024}
\def\nprofesor {Maria Gillespie}
\def\nsigla {MATH601}
\def\nsiglahead {Advanced Combinatorics}
\def\nextra {HW4}
\def\nlang {ENG}
\input{../../headerVarillyDiff}
\usepackage{youngtab}

\begin{document}

\begin{Ej}
    Describe a ``ballot-type'' condition for a word of 1's and 2's to be lowest weight for $\gsl_2$, that is, that $F$ sends the word to $0$. Prove that your condition is correct. Do the same for $\gsl_3$ and the two lowering operators.
\end{Ej}

\begin{ptcbr}
   The condition we are looking for is precisely being a ballot word, this occurs when every prefix left-to-right has more $2$'s than $1$'s. Observe that any word like this will be sent by $F_1$ to zero. $F_1$ will look for the last unmatched $1$, but having a greater number of $2$'s before it means that there will be no way for a $1$ to be unmatched. Thus, $F_1$ will send such a word to zero.\par
   Similarly for lowest weight words of $\gsl_3$, the condition is that in every prefix when reading left-to-right, we find more $3$'s than $2$'s and more $2$'s than $1$'s. Once again applying $F_1$ or $F_2$, we get nothing, because they will be looking for unmatched $1$'s or $2$'s respectively. But as the word is ballot, there are no unmatched $1$'s nor $2$'s. Observe also that it is not necessary for there to be double the amount of $2$'s because of the $1$'s and $3$'s. The $F_i$ operators do not interact with each other so there's no qualms about that.
\end{ptcbr}

\end{document} 
