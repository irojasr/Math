\documentclass[12pt]{memoir}

\def\nsemestre {II}
\def\nterm {Fall}
\def\nyear {2024}
\def\nprofesor {Maria Gillespie}
\def\nsigla {MATH601}
\def\nsiglahead {Advanced Combinatorics}
\def\nextra {HW4}
\def\nlang {ENG}
\input{../../headerVarillyDiff}
\usepackage{youngtab}

\begin{document}

\begin{Ej}
    Describe a ``ballot-type'' condition for a word of 1's and 2's to be lowest weight for $\gsl_2$, that is, that $F$ sends the word to $0$. Prove that your condition is correct. Do the same for $\gsl_3$ and the two lowering operators.
\end{Ej}

\begin{ptcbr}
   The condition we are looking for is precisely being a ballot word, this occurs when every prefix left-to-right has more $2$'s than $1$'s. Observe that any word like this will be sent by $F_1$ to zero. $F_1$ will look for the last unmatched $1$, but having a greater number of $2$'s before it means that there will be no way for a $1$ to be unmatched. Thus, $F_1$ will send such a word to zero.\par
   Similarly for lowest weight words of $\gsl_3$, the condition is that in every prefix when reading left-to-right, we find more $3$'s than $2$'s and more $2$'s than $1$'s. Once again applying $F_1$ or $F_2$, we get nothing, because they will be looking for unmatched $1$'s or $2$'s respectively. But as the word is ballot, there are no unmatched $1$'s nor $2$'s. Observe also that it is not necessary for there to be double the amount of $2$'s because of the $1$'s and $3$'s. The $F_i$ operators do not interact with each other so there's no qualms about that.
\end{ptcbr}

\begin{Ej}
    Write the element $E_{12}$ in $\gsl_3$ as an $8\x 8$ matrix in the adjoint representation, by computing
how $E_{12}$ acts on the basis $\set{E_{ij}\:\ i\neq j} \cup\set{H_{12}, H_{23}}$ defined in class, via the adjoint operator $\bonj{E_{12},-}$.
\end{Ej}
For this exercise we shall employ \ttt{bra-ket} notation. Here, $\braket{u}{v}$ is the inner product of vectors $u,v$ and $\ketbra{u}{v}$ is the rank one matrix whose entries are $u_iv_j$. The following property holds:
$$\ketbra{u}{v}\ketbra{x}{y}=\braket{v}{x}\ketbra{u}{y}.$$
Observe that our $E_{ij}$ matrices are in fact $\ketbra{e_i}{e_j}$ where $e_i$ is the standard basis vector. We will also take the liberty of invoking $E_{ii}$ as $H_{ij}=E_{ii}-E_{jj}$. This will be handy as 
$$\bonj{E_{ij},H_{rs}}=\bonj{E_{ij},E_{rr}-E_{ss}}=\bonj{E_{ij},E_{rr}}-\bonj{E_{ij},E_{ss}}.$$
With this, we may proceed:
\begin{ptcbr}
    If we take the Lie bracket of $E_{ij}$ with $E_{rs}$ we get
    \begin{align*}
        \bonj{E_{ij},E_{rs}}&=E_{ij}E_{rs}-E_{rs}E_{ij}\\
        &=\ketbra{e_i}{e_j}\ketbra{e_r}{e_s}-\ketbra{e_r}{e_s}\ketbra{e_i}{e_j}\\
        &=\dl_{jr}\ketbra{e_i}{e_s}-\dl_{is}\ketbra{e_r}{e_j}
    \end{align*}
    In our case this formula specializes to 
    $$\dl_{2r}\ketbra{e_1}{e_s}-\dl_{1s}\ketbra{e_r}{e_2}$$
    and for this matrix to not be zero we need either $r=2$ or $s=1$. This allows us to see 
        \begin{itemize}
            \item $[E_{12},E_{11}]=-\dl_{11}\ketbra{e_1}{e_2}=-E_{12}$.
            \item $[E_{12},E_{12}]=0$.
            \item $[E_{12},E_{13}]=0$.
            \item $[E_{12},E_{21}]=\dl_{22}\ketbra{e_1}{e_1}-\dl_{11}\ketbra{e_2}{e_2}=E_{11}-E_{22}=H_{12}$.
            \item $[E_{12},E_{22}]=\dl_{22}\ketbra{e_1}{e_2}=E_{12}$.
            \item $[E_{12},E_{23}]=\dl_{22}\ketbra{e_1}{e_3}=E_{13}$.
            \item $[E_{12},E_{31}]=-\dl_{13}\ketbra{e_3}{e_2}=-E_{32}$.
            \item $[E_{12},E_{32}]=0$.
            \item $[E_{12},E_{33}]=0$.
        \end{itemize}
        From this we have
        $$[E_{12},H_{12}]=-E_{12}-E_{12}=-2E_{12},\quad[E_{12},H_{23}]=E_{12}-0=E_{12},$$
        and ordering our basis as 
        $$\set{E_{12},E_{13},E_{21},E_{23},E_{31},E_{32},H_{12},H_{23}}$$
        our matrix will be 
        $$\begin{pmatrix}
            0&0&0&0&0&0&0&0\\
            0&0&0&0&0&0&0&0\\
            0&0&0&0&0&0&1&0\\
            0&1&0&0&0&0&0&0\\
            0&0&0&0&0&-1&0&0\\
            0&0&0&0&0&0&0&0\\
            -2&0&0&0&0&0&0&0\\
            1&0&0&0&0&0&0&0
        \end{pmatrix}^\sT=\begin{pmatrix}
            0&0&0&0&0&0&-2&1\\
            0&0&0&1&0&0&0&0\\
            0&0&0&0&0&0&0&0\\
            0&0&0&0&0&0&0&0\\
            0&0&0&0&0&0&0&0\\
            0&0&0&0&-1&0&0&0\\
            0&0&1&0&0&0&0&0\\
            0&0&0&0&0&0&0&0
        \end{pmatrix}$$
\end{ptcbr}

\begin{Ej}
    Show that the embedding $\gsl_2\hookto\gsl_3$ that sends 
    $$\twobytwo{a}{b}{c}{d}\mapsto\threebythree{a}{b}{0}{c}{d}{0}{0}{0}{0}$$
    is indeed an injective homomorphism of Lie algebras.
\end{Ej}

\begin{ptcbr}
    To show that this is Lie algebra homomorphism we must show that it respects the Lie bracket. Suppose $A,B\in\gsl_2$, where 
    $$A=\twobytwo{a}{b}{c}{d},\quad B=\twobytwo{e}{f}{g}{h}$$
    so that 
    $$\bonj{A,B}=AB-BA=\twobytwo{bg-cf}{af+bh-be-df}{ce+dg-ag-ch}{cf-bg}.$$
    When embedding these matrices into $\gsl_3$ we get 
    $$\iota A=\threebythree{a}{b}{0}{c}{d}{0}{0}{0}{0},\quad \iota B=\threebythree{e}{f}{0}{g}{h}{0}{0}{0}{0}.$$
    Taking their Lie bracket gives us precisely
    $$\bonj{\iota A,\iota B}=\threebythree{bg-cf}{af+bh-be-df}{0}{ce+dg-ag-ch}{cf-bg}{0}{0}{0}{0}=\iota\bonj{A,B}$$
    showing us that the mapping is indeed a morphism of Lie algebras.\par
    This map is certainly injective for if 
    $$\threebythree{a}{b}{0}{c}{d}{0}{0}{0}{0}=\threebythree{0}{0}{0}{0}{0}{0}{0}{0}{0}\To a=b=c=d=0\To A=0.$$
    
\end{ptcbr}

\begin{Ej}
    Recall that the complete homogeneous symmetric function $h_\mu(x_1,\dots,x_n)$, for a partition $\mu=(\mu_1,\dots,\mu_k)$,
can be defined as the product $h_{\mu_1}\dots h_{\mu_k}$ where $h_d$ is the sum of all monomials in $x_1,\dots,x_n$ of degree $d$.
\begin{enumerate}
    \item Which Schur function in $n$ variables is $h_d(x_1,\dots,x_n) $ equal to?
    \item Show that $h_\mu(x_1, x_2, x_3)$ is the character of the tensor product of the $k$ irreducible $\gsl_3$ representations
    $$V^{(\mu_1,0)}\oxyox V^{(\mu_k,0)}$$
    where $V^{(a,b)}$ denotes the irreducible representation with highest weight $(a, b)$.
\end{enumerate}
\end{Ej}

\begin{ptcbr}
    \begin{enumerate}
        \item We have $h_d=s_d$! This is because $s_d$'s monomials correspond to rows of length $d$ which can be filled as a SSYT. This is the same as all the possible monomials in $n$ variables in $h_d$.
        \item Recall that $\chi(V^{d,0})=s_d$ so in this case we have 
        $$\chi\left(V^{(\mu_1,0)}\oxyox V^{(\mu_k,0)}\right)=\prod_{j=1}^{k}\chi\left(V^{(\mu_j,0)}\right)=\prod_{j=1}^{k}s_{\mu_j}=\prod_{j=1}^{k}h_{\mu_j}=h_\mu$$
        where we use the fact that $s_d=h_d$ for row tableaux.
        \item Kostka numbers count SSYT of shape $\la$ with content $\mu$. When restricting to 3 variables we are only considering fillings of tableaux with $1,2$ and $3$. In this fashion, each term in the decomposition into irreducibles of the tensor product is counted $K_{\la\mu}$ times.
    \end{enumerate}
\end{ptcbr}
\end{document} 
