\documentclass[12pt]{memoir}

\def\nsemestre {II}
\def\nterm {Fall}
\def\nyear {2024}
\def\nprofesor {Maria Gillespie}
\def\nsigla {MATH601}
\def\nsiglahead {Advanced Combinatorics}
\def\nlang {ENG}
%\def\darktheme{}
\input{../../headerVarillyDiff}
\usepackage[enableskew]{youngtab}

\begin{document}
%\clearpage
\maketitle
%\thispagestyle{empty}
{\small 
\setlength{\parindent}{0em}
\setlength{\parskip}{1em}

This course will focus on the combinatorics of Young tableaux, crystal bases, root systems, Dynkin diagrams, and symmetric functions arising in representation theory of matrix groups and Lie algebras.

\subsubsection*{Requirements}
Familiarity with the basics of group theory and symmetric functions is helpful.
}
\newpage
\tableofcontents
%\begin{multicols}{2}
\chapter{}

\section{Day 1|20240819}

We will start by reviewing the representation theory of finite groups and the Lie group and Lie algebra representations. The objective is to classify semi-simple Lie algebras and groups. This classification is quite combinatorial. 

\subsection{Review of representation theory of finite groups}

Recall groups are sets $G$ endowed with a binary operation $\circ$ such that 
\begin{enumerate}
    \item There is an identity element $e$: $g\circ e=e\circ g=g$.
    \item Every element possesses an inverse. For each $g$, there is an $h$ such that $g\circ h=e=h\circ g$.
    \item The operation $\circ$ is associative.
\end{enumerate}

\begin{Ex}
    The \term{symmetric group} is the set of permutations of $\bonj{n}$. We denote it $(S_n,\circ)$ where our operation is composition. We will use this group quite a lot.
\end{Ex}

\begin{Ex}
We will be working with $\rGL_n(\bC)$ where $\bC$ will come in as more useful than $\bR$. The \term{general linear group} is characterized by the property that $\det(A)\neq 0$ for $A\in\rGL_n(\bC)$.
\end{Ex}

\begin{Ex}
    Given two groups we can construct $G\x H$ by doing operations pointwise. We can also take subgroups and quotient groups. 
\end{Ex}

\begin{Ex}
    Take the \term{special linear group} $\rSL_n(\bC)$ which is the set of matrices $A$ with $\det(A)=1$. This is a subgroup of $\rGL_n(\bC)$.
\end{Ex}

There's a lot more of matrix groups such as $\rSO_n(\bC)$, $\rSp_{2n}(\bC)$ and unitary groups $\rSU_n(\bC)$. 

\subsubsection{Groups which are representations of themselves}

Symmetry groups are groups of linear transformations of $\bC^n$ (some Euclidean space) that fix some shape. Any such group is a subgroup of $\rGL_n(\bC)$. Matrices here don't collapse points nor anything.

\begin{Ex}
    The symmetry group of a diamond in the plane can be found by analyzing the symmetries of the figure.\red{HMMM}
    The group in question is the Klein-4 group which can be seen as 
    $$\set{\id,r_x,r_y,r_xr_y}.$$
    Similarly we can see it as 
    $$\set{\id,(24),(13),(13)(24)}$$
\end{Ex}

\red{Fell asleep}

\section{Day 2|20240821}

We were looking at direct sums of representations. Recall representations are maps which take group elements to matrices. 

$$\rho\oplus\sg\:G\to\rGL_{n+m}(\bC)$$

and this map will send $g$ to a block matrix. A central question in representation theory is to classify the irreducible representations of some object. This is a central question because for finite groups, irreducible is the same as indecomposable.

\begin{Def}
A representation is \term{indecomposable} when it can't be written as a direct sum of smaller representations.
\end{Def}

Irreducible means that it has no non-trivial proper representations. This is analogous to the idea of prime and irreducible numbers. In the most general case where groups may be infinite, irreducible implies indecomposable. 

\subsection{Alternative definitions for representations}

We may define it as a vector space $V$ with an action $G\x V\to V$ so that
$$g(hv)=(gh)v$$
and it should be a linear action in the sense that $v\mapsto gv$ is a linear transformation.\par 

This is equivalent to the previous definition because $V$ can be seen as $\bC^n$. So the definition gives rise to a map 
$$G\to\Aut(V),\ g\mapsto g\.$$
Even more \emph{objecty} is the next definition. We can see a representation as a module over a group ring $\bC G$. This set is made up of formal linear combinations of elements of $G$.\par 
We endow it with a module structure, for any element $g\in G$ in particular in $\bC G$ we can make it a coefficient $gv\in V$ as a $\bC G$-module.

\subsection{Subrepresentations}

Now that we have all the algebraic structure we can use it to define subrepresentations. Because a subrepresentation will be a subspace which inherits the action for example. 

\begin{Def}
    $W\subseteq V$ is a \term{subrepresentation} of $G$ (when $V$ represents $G$) if 
    \begin{itemize}
        \item $W$ is a subspace of $V$, and
        \item $W$ is $G$-invariant in the sense that the image of $G\x W\to V$ is contained in $W$.
    \end{itemize}
    We will also say that $V$ is \term{irreducible} if there's no proper nonzero subrepresentation $W\subseteq V$.
\end{Def}

Sometimes it is possible to decompose a representation into a direct sum of subrepresentations.

\red{fell asleep}

\begin{Def}
    A \term{character} of a representation is the trace map $g\mapsto\tr(\rho(g))$.
\end{Def}

\textbf{Properties}
\begin{enumerate}
    \item $\chi_{V\oplus W}=\chi_V+\chi_W$. 
    \item $\chi_{V\ox W}=\chi_V\chi_W$.
    \item $\chi_V$ uniquely determines the representation.
\end{enumerate}

\section{Day 3|20240823}

\subsection{Lie groups}

\begin{Def}
    A \term{Lie group} is a real smooth manifold $G$ with a group structure such that 
    $$(g,h)\mapsto gh^{-1}$$
    is differentiable.
\end{Def}

A manifold is a set such that around each point there's a local neighborhood that's topologically equivalent to $\bR^n$. Elliptic curves are examples of manifolds.

\begin{Def}
    An \term{algebraic group} is an algebraic variety with a group structure. In this case the multiplication map should be algebraic.
\end{Def}

In certain specializations these two are the same object. In the case of complex Lie groups, we talk about smooth complex manifolds.

\begin{Ex}
    \begin{itemize}
        \item $(\bC^n,+)$ is a Lie group. But it's not compact. \red{sleepy sleepy}
        \item $\rGL_n$
    \end{itemize}
\end{Ex}

\begin{Lem}
(Zariski-)Closed subgroups of a Lie group are also Lie groups.
\end{Lem}

\begin{Ex}
    In particular $B_n$, the set of upper triangular matrices in $\rGL_n$, forms a Lie group. The torus $T_n$, the group of diagonal matrices, is also a Lie group.\par 
    It is called the torus because it's isomorphic to $(\bC\less0)^n$ and $\bC\less 0$ looks like a circle while $(\bC\less 0)^2$ is the product of two circles which is the torus.
\end{Ex}

\subsubsection{The Classical Groups}

The special linear group $\rSL_n$ consists of matrices whose determinant is 1. The classical groups are called clasiccal because they have very nice properties. In particular type $A$ is what we call $\rSL_n$.\par 
To talk about the special orthogonal group $\rSO_n$ we should first fix a symmetric bilinear form $(\.,\.)$ which is positive-definite. The \term{orthogonal group} $\rO_n$ consists of matrices which preserve this form. The special orthogonal group in particular is the subgroup of matrices with determinant $1$.

\begin{Rmk}
Over $\bR$, $\rO_n$ is actually the group of rigid transformations which is generated by reflections and rotations. For $\rSO_n$, it's only the rotations group.
\end{Rmk}

We can also alternatively define $\rO_n$ as 
$$\set{A\: A^\sT A=I}$$
because 
$$\bra{Av}\ket{Aw}=\bra{v}\ket{w}$$
and from this 
$$v^\sT A^\sT Aw=v^\sT w.$$
Comparing entry by entry we get the desired property.\par 
It's also a fact that $\rO_n$ is disconnected, one component is $\rSO_n$ and the other is the set of matrices with determinant $-1$. Finally \term{type B} means $\rSO_{\text{odd}}$ while $D$ means $\rSO_{\text{even}}$. The type $C$ groups are the symplectic groups.

\section{Day 4|20240826}

Continuing on with the classical groups, we will be talking about the \term{Symplectic group} of even dimension. We will be fixing a symplectic form which is a non-degenerate, skew-symmetric, bilinear form.

\begin{Ex}
    The dot product is not symplectic because it's symmetric.
\end{Ex}

\begin{Ex}
    Consider the form 
    $$v_1w_{2n}+v_2w_{2n-1}+\dots+v_nw_{n+1}-v_{n-1}w_n-v_{n+1}w_{n}-v_{n+2}w_{n-1}-\dots-v_{2n}w_{1}.$$
    If $\Om$ is such a matrix of a form, for example when $2n=6$ we have 
    $$
    \Om\defeq
    \begin{pmatrix}
        &&&&&1\\
        &&&&1&\\
        &&&1&&\\
        &&-1&&&\\
        &-1&&&&\\
        -1&&&&&
    \end{pmatrix}
    \To (v,w)=v^{\sT}\Om w
    $$
\end{Ex}

From this our first definition of the symplectic group is matrices which preserve this product. 

\begin{Def}
    The symplectic group $\rSp_{2n}$ is 
    $$\set{M\:(Mv,Mw)=(v,w)}$$
    or equivalently
    $$\set{M\: M^{\sT}\Om M=\Om}.$$
    We will simplify the notation to type $C$.
\end{Def}

\subsection{Representation of Lie groups}

\begin{Def}
    A representation of a Lie group is a map 
which is also differentiable and a group homomorphism.
\end{Def}

\section{Day 5|20240828}

For a partition $\la\vdash n$, we call $S^\la V$
$$\La^{\mu_1}V\ox\La^{\mu_2}V\oxyox\La^{\mu_k}V$$
where $\mu$ is the conjugate partition. 

\begin{Ex}
    For example if $\la=(5,4,1)$, then $\mu=(3,2,2,2,1)$ and so 
    $$S^{(5,4,1)}V=\La^3$$
    Elements can be written as a filling to the Young diagram. Such an element could be 
    $$(v_1\land v_2\land v_3)\ox(a\land b)\ox (c\land d)\ox (x\land y)\ox z$$
    and filling the diagram we have
    $$
    \young(r,qbdy,pacxz).
$$
It's important to familiarize ourselves with this idea so we will itechangebly talk about 
$$(e_1\w e_4\w e_3)\ox(e_1\w e_2)\ox (e_5\w e_3)\ox(e_2\w e_1)\ox e_2$$
and 
$$\young(3,4231,11522)=-\young(4,3231,11522)$$
The tableau $\young(1,12)$ is zero for example.
\end{Ex}

For a basis of $S^\la$, we can talk about it being spanned by elementary tableau where we order each column from least to greatest. These are called \term{column-strict tableau}. For example

$$\young(6,453,1212)$$

If $V$ is an $n$-dimensional vector space, then we have a largest element on our basis. This allows us to formulate the question:
\begin{significant}
How many column strict tableau are there with largest entry $n$? And shape $\la$.
\end{significant}
From this 
$$\binom{n}{\mu_1}\binom{n}{\mu_2}\dots\binom{n}{\mu_k}=S^\la V.$$

\begin{Def}
    The \term{Schur module} $V^\la$ is 
    $$V^\la=\quot{S^\la}{\genr{v_T-\sum_S v_S}}$$
    where the sum is over $S$'s obtained from $T$ by
    \begin{enumerate}
        \item Choose two columns of $C_1,C_2$ of $T$.
        \item Choose $k$ elements from $C_2$.
        \item Exchange them with $k$ elements from $C_1$ in all ways that preserve the order of the elements.
    \end{enumerate}
\end{Def}

\begin{Ex}
    Take $(4,3,3)$ with the filling 
    $$\young(576,244,1134)$$
    so choose the first and third columns as $C_1$ and $C_2$. One relation in $V_\la$
\end{Ex}

\begin{Th}
    The collection 
    $$\set{e_T\:\ T \text{semistandard} \sh(T)\vdash n}$$
    is a basis for the Schur module.
\end{Th}

\section{Day 6|20240830}

Last time we defined the Schur modules. These are 

$$S^\la V=\La^{\mu_1}V\oxyox \La^{\mu_r}$$

where $\mu=\la^\ast$ is the conjugate or transpose. Now $V^\la$ is $S^\la$ modded out by column exchanges. We will show that 
$$\set{e_T\: T\in SSYT(\la), \text{largest entry}\leq n}$$
is a basis for $V^\la$.

\begin{Ex}
    Consider the tableau
    $$\young(6,53,214)$$
    the second and third row are wrongfully ordered
\end{Ex}

\red{Sleepy sleepy}

We wil show that they are independent in the quotient. 

\begin{Ex}
    The idea for why $D_T$'s are independent. We can find lex orderings and make $D_T$ have nice leading term and then an ordering on the leading terms. E.g. $1,1+x,1+x+x^2,1+x+x^2+x^3$ are independent because the leading terms are all distinct.\par 
    In $V^{\young(a,bc)}$ we have 
    $$D_{\young(2,11)}=\det\twobytwo{z_{11}}{z_{12}}{z_{21}}{z_{22}}z_{11}=\dots$$
    And 
    $$D_{\young(2,12)}=z_{12}\det$$
    In the monomials $z_{11}^2z_{22}$ is larger than $z_{11}z_{12}z_{22}$ and that's how we show that they're independent of each other. This shows the elementary symmetric functions are independent. 
\end{Ex}

One exciting conclusion to look at it's characters. For a Lie group the right notion is to consider $H$ a maximal torus in a Lie group $G$. This is the maximal connecated, abelian Lie sub group.

\begin{Ex}
    $T_n\subseteq\rGL_n$ in this case $\chi_V\: H\to \bC$ where $h\mapsto\tr(h\text{ acts on } V)$. This $\chi_V$ determines $V$ and has nice properties with direct sum and tensor products.
    $$\chi_V\diag(x_1,\dots,x_n)$$
    is the trace of that matrix acting on $V^\la$. It suffices to look at a basis. For a given $e_T$ where $T$ is a SSYT, $X$ acts on each $e_i$ by doing $x_ie_i$. see
    $$\young(3,234,112)=x_1x_1x_2x_2x_3x_3x_4\.\young(3,234,112)$$
    now the trace is the sum of the eigenvalues and this is $x^T$. So 
    $$\sum_{T SSYT}x^T=s_\la(\un x).$$
\end{Ex}

\section{Day 7|20240904}

\begin{Th}
A representation of $\rGL_n$ is irreducible if and only if it has a unique highest weight vector.
\end{Th}

\begin{Def}
    A \term{weight vector} of $V$ is $v\in V$ such that for $x\in T_n$ (the torus), 
    $$x\.v=x_1^{\al_1}\dots x_n^{\al_n}v$$
    where $\al=(\al_1,\dots,\al_n)\in\bZ^n$ is weight.
\end{Def}

Recall that being in the torus meant $x=\threebythree{x_1}{\dots}{0}{\vdots}{\ddots}{\vdots}{0}{\dots}{x_n}$.

\begin{Def}
    A \term{highest weight vector} is a weight vector such that 
    $$B_n\.v=\bC^\ast\. v$$
    where $B_n$ is the Borel matrices compromised of upper triangular matrices.
\end{Def}

A representation is a sum of its weights: $V=\bigoplus_{\al}V_\al$ where $V_\al=\set{v\:\ x\.dot v=x^\al v}$.

\begin{Lem}
The only highest weight vector in $V^\la$ is $e_{T_0}$ where $T_E$
\end{Lem}
%%%%%%%%%%%% Contents end %%%%%%%%%%%%%%%%
\ifx\nextra\undefined
\printindex
\else\fi
\nocite{*}
\bibliographystyle{plain}
\bibliography{bibiCombiAvanzada.bib}
\end{document} 

