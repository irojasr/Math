\documentclass[12pt]{memoir}

\def\nsemestre {II}
\def\nterm {Fall}
\def\nyear {2024}
\def\nprofesor {Maria Gillespie}
\def\nsigla {MATH601}
\def\nsiglahead {Advanced Combinatorics}
\def\nlang {ENG}
%\def\darktheme{}
\input{../../headerVarillyDiff}
\usepackage[enableskew]{youngtab}

\begin{document}
%\clearpage
\maketitle
%\thispagestyle{empty}
{\small 
\setlength{\parindent}{0em}
\setlength{\parskip}{1em}

This course will focus on the combinatorics of Young tableaux, crystal bases, root systems, Dynkin diagrams, and symmetric functions arising in representation theory of matrix groups and Lie algebras.

\subsubsection*{Requirements}
Familiarity with the basics of group theory and symmetric functions is helpful.
}
\newpage
\tableofcontents
%\begin{multicols}{2}
\chapter{}

\section{Day 1|20240819}

We will start by reviewing the representation theory of finite groups and the Lie group and Lie algebra representations. The objective is to classify semi-simple Lie algebras and groups. This classification is quite combinatorial. 

\subsection{Review of representation theory of finite groups}

Recall groups are sets $G$ endowed with a binary operation $\circ$ such that 
\begin{enumerate}
    \item There is an identity element $e$: $g\circ e=e\circ g=g$.
    \item Every element possesses an inverse. For each $g$, there is an $h$ such that $g\circ h=e=h\circ g$.
    \item The operation $\circ$ is associative.
\end{enumerate}

\begin{Ex}
    The \term{symmetric group} is the set of permutations of $\bonj{n}$. We denote it $(S_n,\circ)$ where our operation is composition. We will use this group quite a lot.
\end{Ex}

\begin{Ex}
We will be working with $\rGL_n(\bC)$ where $\bC$ will come in as more useful than $\bR$. The \term{general linear group} is characterized by the property that $\det(A)\neq 0$ for $A\in\rGL_n(\bC)$.
\end{Ex}

\begin{Ex}
    Given two groups we can construct $G\x H$ by doing operations pointwise. We can also take subgroups and quotient groups. 
\end{Ex}

\begin{Ex}
    Take the \term{special linear group} $\rSL_n(\bC)$ which is the set of matrices $A$ with $\det(A)=1$. This is a subgroup of $\rGL_n(\bC)$.
\end{Ex}

There's a lot more of matrix groups such as $\rSO_n(\bC)$, $\rSp_{2n}(\bC)$ and unitary groups $\rSU_n(\bC)$. 

\subsubsection{Groups which are representations of themselves}

Symmetry groups are groups of linear transformations of $\bC^n$ (some Euclidean space) that fix some shape. Any such group is a subgroup of $\rGL_n(\bC)$. Matrices here don't collapse points nor anything.

\begin{Ex}
    The symmetry group of a diamond in the plane can be found by analyzing the symmetries of the figure.\red{HMMM}
    The group in question is the Klein-4 group which can be seen as 
    $$\set{\id,r_x,r_y,r_xr_y}.$$
    Similarly we can see it as 
    $$\set{\id,(24),(13),(13)(24)}$$
\end{Ex}

\red{Fell asleep}

\section{Day 2|20240821}

We were looking at direct sums of representations. Recall representations are maps which take group elements to matrices. 

$$\rho\oplus\sg\:G\to\rGL_{n+m}(\bC)$$

and this map will send $g$ to a block matrix. A central question in representation theory is to classify the irreducible representations of some object. This is a central question because for finite groups, irreducible is the same as indecomposable.

\begin{Def}
A representation is \term{indecomposable} when it can't be written as a direct sum of smaller representations.
\end{Def}

Irreducible means that it has no non-trivial proper representations. This is analogous to the idea of prime and irreducible numbers. In the most general case where groups may be infinite, irreducible implies indecomposable. 

\subsection{Alternative definitions for representations}

We may define it as a vector space $V$ with an action $G\x V\to V$ so that
$$g(hv)=(gh)v$$
and it should be a linear action in the sense that $v\mapsto gv$ is a linear transformation.\par 

This is equivalent to the previous definition because $V$ can be seen as $\bC^n$. So the definition gives rise to a map 
$$G\to\Aut(V),\ g\mapsto g\.$$
Even more \emph{objecty} is the next definition. We can see a representation as a module over a group ring $\bC G$. This set is made up of formal linear combinations of elements of $G$.\par 
We endow it with a module structure, for any element $g\in G$ in particular in $\bC G$ we can make it a coefficient $gv\in V$ as a $\bC G$-module.

\subsection{Subrepresentations}

Now that we have all the algebraic structure we can use it to define subrepresentations. Because a subrepresentation will be a subspace which inherits the action for example. 

\begin{Def}
    $W\subseteq V$ is a \term{subrepresentation} of $G$ (when $V$ represents $G$) if 
    \begin{itemize}
        \item $W$ is a subspace of $V$, and
        \item $W$ is $G$-invariant in the sense that the image of $G\x W\to V$ is contained in $W$.
    \end{itemize}
    We will also say that $V$ is \term{irreducible} if there's no proper nonzero subrepresentation $W\subseteq V$.
\end{Def}

Sometimes it is possible to decompose a representation into a direct sum of subrepresentations.

\red{fell asleep}

\begin{Def}
    A \term{character} of a representation is the trace map $g\mapsto\tr(\rho(g))$.
\end{Def}

\textbf{Properties}
\begin{enumerate}
    \item $\chi_{V\oplus W}=\chi_V+\chi_W$. 
    \item $\chi_{V\ox W}=\chi_V\chi_W$.
    \item $\chi_V$ uniquely determines the representation.
\end{enumerate}
%%%%%%%%%%%% Contents end %%%%%%%%%%%%%%%%
\ifx\nextra\undefined
\printindex
\else\fi
\nocite{*}
\bibliographystyle{plain}
\bibliography{bibiCombiAvanzada.bib}
\end{document} 

