%----------------------------------------------------------------------------------------
%	PACKAGES AND OTHER DOCUMENT CONFIGURATIONS
%----------------------------------------------------------------------------------------

\documentclass[12pt]{article}
\usepackage[spanish]{babel} %Tildes
\usepackage[extreme]{savetrees} %Espaciado e interlineado. Comentar si no gusta el interlineado.
\usepackage[utf8]{inputenc} %Encoding para tildes
\usepackage[breakable,skins]{tcolorbox} %Cajitas
\usepackage{fancyhdr} % Se necesita para el título arriba
\usepackage{lastpage} % Se necesita para poner el número de página
\usepackage{amsmath,amsfonts,amssymb,amsthm} %simbolos y demás
\usepackage{mathabx} %más símbolos
\usepackage{physics} %simbolos de derivadas, bra-ket.
\usepackage{multicol}
\usepackage[customcolors]{hf-tikz}
\usepackage[shortlabels]{enumitem}
\usepackage{tikz}
\usetikzlibrary{patterns}
%\def\darktheme
%%%%%%%%% === Document Configuration === %%%%%%%%%%%%%%

\pagestyle{fancy}
\setlength{\headheight}{14.49998pt} %NO MODIFICAR
\setlength{\footskip}{14.49998pt} %NO MODIFICAR

\ifx \darktheme\undefined

\lhead{Math261S11} % Nombre de autor
\chead{\textbf{Quiz 7}} % Titulo
\rhead{Name:\hspace*{5cm}}%\firstxmark} 
\lfoot{}%\lastxmark}
\cfoot{}
\rfoot{Page \thepage\ of\ \pageref{LastPage}} %A la derecha saldrá pág. 6 de 9. 
\else
\pagenumbering{gobble}
\pagecolor[rgb]{0,0,0}%{0.23,0.258,0.321}
\color[rgb]{1,1,1}
\fi

%%%%%%%%% === My T Color Box === %%%%%%%%%%%%%%

\ifx \darktheme\undefined
\newtcolorbox{ptcb}{
colframe = black,
colback = white,
breakable,
enhanced
}
\newtcolorbox{ptcbP}{
colframe = black,
colback = white,
coltitle = black,
colbacktitle = black!40,
title = Practice,
breakable,
enhanced
}

\else
\newtcolorbox{ptcb}{
colframe = white,
colback = black,
colupper = white,
breakable,
enhanced
}
\newtcolorbox{ptcbP}{
colframe = white,
colback = black,
colupper = white,
coltitle = white,
colbacktitle = black,
title = Practice,
breakable,
enhanced
}
\fi

%%%%%%%%% === Tikz para matrices === %%%%%%%%%%%%%%

\tikzset{
  style green/.style={
    set fill color=green!50!lime!60,
    set border color=white,
  },
  style cyan/.style={
    set fill color=cyan!90!blue!60,
    set border color=white,
  },
  style orange/.style={
    set fill color=orange!80!red!60,
    set border color=white,
  },
  row/.style={
    above left offset={-0.15,0.31},
    below right offset={0.15,-0.125},
    #1
  },
  col/.style={
    above left offset={-0.1,0.3},
    below right offset={0.15,-0.15},
    #1
  }
}

%%%%%%%%% === Theorems and suchlike === %%%%%%%%%%%%%%

\theoremstyle{plain}
\newtheorem{Th}{Theorem}  %%% Theorem 1.1
\newtheorem*{nTh}{Theorem}             %%% No-numbered Theorem
\newtheorem{Prop}[Th]{Proposition}     %%% Proposition 1.2
\newtheorem{Lem}[Th]{Lemma}             %%% Lemma 1.3
\newtheorem*{nLem}{Lemma}               %%% No-numbered Lemma
\newtheorem{Cor}[Th]{Corollary}        %%% Corollary 1.4
\newtheorem*{nCor}{Corollary}          %%% No-numbered Corollary

\theoremstyle{definition}
\newtheorem*{Def}{Definition}       %%% Definition 1.5
\newtheorem*{nonum-Def}{Definition}    %%% No number Definition
\newtheorem*{nEx}{Example}             %%% No number Example
\newtheorem{Ex}[Th]{Example}           %%% Example
\newtheorem{Ej}[Th]{Exercise}         %%% Exercise
\newtheorem*{nEj}{Exercise}           %%% No number Excercise
\newtheorem*{Not}{Notation}       %%% Definition 1.5

\theoremstyle{remark}
\newtheorem*{Rmk}{Remark}      %%%Remark 1.6

%\numberwithin{equation}{section}

\setlength{\parindent}{3ex}

%%====== Useful macros: =======%%%

\DeclareMathOperator{\gen}{gen}     %%%set generated by...
\DeclareMathOperator{\Rng}{Rng}     %%%rangomat
\DeclareMathOperator{\Nul}{Nul}     %%%rangomat
\DeclareMathOperator{\Proy}{Proy}   %%%proyección
\DeclareMathOperator{\id}{id}       %%%identity operator

\newcommand{\al}{\alpha}            %%%short for \alpha
\newcommand{\la}{\lambda}           %%%short for \lambda
\newcommand{\sg}{\sigma}            %%%short for \sigma
\newcommand{\te}{\theta}                %% short for  \theta
\renewcommand{\l}{\ell}

\newcommand{\thickhat}[1]{\mathbf{\hat{\text{$#1$}}}}
\newcommand{\ii}{\vu{\imath}}
\newcommand{\jj}{\vu{\jmath}}
\newcommand{\kk}{\thickhat{k}}

\newcommand{\bC}{\mathbb{C}}        %%%complex numbers
\newcommand{\bN}{\mathbb{N}}        %%%natural numbers
\newcommand{\bP}{\mathbb{P}}        %%%polynomials
\newcommand{\bR}{\mathbb{R}}        %%%real numbers
\newcommand{\bZ}{\mathbb{Z}}        %%%integer numbers
\newcommand{\cB}{\mathcal{B}}       %%%basis
\newcommand{\cC}{\mathcal{C}}       %%%basis
\newcommand{\cM}{\mathcal{M}}       %%%matrix family

\newcommand{\sT}{\mathsf{T}}        %%%traspuesta

\renewcommand{\geq}{\geqslant}      %%%(to save typing)
\renewcommand{\leq}{\leqslant}      %%%(to save typing)
\newcommand{\x}{\times}             %%%product
\renewcommand{\:}{\colon}           %%%colon in  f: A -> B
\newcommand{\isom}{\simeq}              %% isomorfismo

\newcommand{\un}[1]{\underline{#1}}
\newcommand{\half}{\frac12}

\newcommand*{\Cdot}{{\raisebox{-0.25ex}{\scalebox{1.5}{$\cdot$}}}}      %% cdot más grande
\renewcommand{\.}{\Cdot}                %% producto escalar

\newcommand{\twobyone}[2]{\begin{pmatrix} %% 2 x 1 matrix
  #1 \\ #2 \end{pmatrix}}
  \newcommand{\twobytwo}[4]{\begin{pmatrix} %% 2 x 2 matrix
    #1 & #2 \\ #3 & #4 \end{pmatrix}}
    \newcommand{\twobythree}[6]{\begin{pmatrix} %% 2 x 3 matrix
        #1 & #2 & #3\\ #4 & #5 & #6 \end{pmatrix}}
\newcommand{\threebyone}[3]{\begin{pmatrix} %% 3 x 1 matrix
  #1 \\ #2 \\ #3 \end{pmatrix}}
  \newcommand{\threebytwo}[6]{\begin{pmatrix} %% 3 x 1 matrix
    #1 & #2\\ #3 & #4\\ #5&#6 \end{pmatrix}}
\newcommand{\threebythree}[9]{\begin{pmatrix} %% 3 x 3 matrix
  #1 & #2 & #3 \\ #4 & #5 & #6 \\ #7 & #8 & #9 \end{pmatrix}}

\newcommand{\To}{\Rightarrow}

\newcommand{\vaf}{\overrightarrow}

\newcommand{\set}[1]{\{\,#1\,\}}    %% set notation
\newcommand{\Set}[1]{\biggl\{\,#1\,\biggr\}} %% set notation (large)
\newcommand{\red}[1]{\textcolor{red}{#1}}
\newcommand{\blu}[1]{\textcolor{blue}{#1}}
\newcommand{\word}[1]{\quad\text{#1}\quad} %% texto intercalado
\newcommand{\br}[1]{\left\langle #1\right\rangle}
\newcommand{\nb}{\nabla}
%----------------------------------------------------------------------------------------
%	ARTICLE CONTENTS
%----------------------------------------------------------------------------------------

\begin{document}
%\begin{multicols}{2}

\begin{Ej}
  In this exercise, we will review the use of polar coordinates.
  \begin{enumerate}
    \item Consider the following figure:

    \begin{center}


      \tikzset{every picture/.style={line width=0.75pt}} %set default line width to 0.75pt        

      \begin{tikzpicture}[x=0.75pt,y=0.75pt,yscale=-1,xscale=1]
      %uncomment if require: \path (0,300); %set diagram left start at 0, and has height of 300
      
      %Straight Lines [id:da9322038356101576] 
      \draw    (140,91) -- (140,191) ;
      %Straight Lines [id:da6751605298801924] 
      \draw    (90,141) -- (190,141) ;
      %Shape: Circle [id:dp2603483709032255] 
      \draw   (100,141) .. controls (100,118.91) and (117.91,101) .. (140,101) .. controls (162.09,101) and (180,118.91) .. (180,141) .. controls (180,163.09) and (162.09,181) .. (140,181) .. controls (117.91,181) and (100,163.09) .. (100,141) -- cycle ;
      %Straight Lines [id:da4876726638778297] 
      \draw    (90,191) -- (190,91) ;
      %Straight Lines [id:da013830272144174538] 
      \draw    (180,141) -- (190,130) ;
      %Straight Lines [id:da7786097088612032] 
      \draw    (140,141) -- (112,112) ;
      %Straight Lines [id:da879664606023387] 
      \draw    (135.4,146.2) -- (107,118) ;
      %Straight Lines [id:da6341813039022857] 
      \draw    (144.6,135.8) -- (117,108) ;
      %Straight Lines [id:da9855909775138032] 
      \draw    (150.2,130.6) -- (124.2,103.6) ;
      %Straight Lines [id:da4101753279794036] 
      \draw    (129.8,151) -- (104.2,124.4) ;
      %Straight Lines [id:da8030563437269853] 
      \draw    (154.6,125.4) -- (132.6,102) ;
      %Straight Lines [id:da6151907959632548] 
      \draw    (124.6,155.8) -- (106.31,136.87) -- (101.8,132) ;
      %Straight Lines [id:da9897479460199292] 
      \draw    (160.2,120.2) -- (142.2,101.2) ;
      %Straight Lines [id:da17658855094836046] 
      \draw    (120.6,161.8) -- (100,141) ;
      %Straight Lines [id:da1287264303313751] 
      \draw    (115,165.4) -- (101.6,151) ;
      %Straight Lines [id:da2670928208408272] 
      \draw    (165,116.6) -- (151.6,102.2) ;
      
      % Text Node
      \draw (192,141) node [anchor=west] [inner sep=0.75pt]    {$x$};
      % Text Node
      \draw (140,87.6) node [anchor=south] [inner sep=0.75pt]    {$y$};
      % Text Node
      \draw (192,87.6) node [anchor=south west] [inner sep=0.75pt]    {$y=x$};
      % Text Node
      \draw (192,126.6) node [anchor=south west] [inner sep=0.75pt]    {$x=2$};
      
      
      \end{tikzpicture}

  \end{center}
  Using the information provided in the figure, describe the \textbf{shaded} region in terms of polar coordinates.

  \item Consider the following integral:
  \[
  \int_{-\sqrt{2}}^{0}\int_{-\sqrt{4-x^2}}^x\dd y\dd x + \int_{0}^{\sqrt{2}}\int_{-\sqrt{4-x^2}}^{-x}\dd y\dd x.
  \]
  Sketch the region represented by this integral and express its bounds in polar coordinates.
\end{enumerate}
\end{Ej}

\begin{ptcb}
\vspace{15cm}
\end{ptcb}
\newpage
\begin{Ej}
  Consider the vector field $\mathbf{F}(x,y,z) = (z, 0, -x)$ and the curves $\mathcal{C}_1$ and $\mathcal{C}_2$ parametrized as follows:
  \[
  \left\lbrace
  \begin{aligned}
    &r_1(t) = (1-2t, 1, 1-2t),\quad 0 \leq t \leq 1, \\
    &r_2(s) = (s^3, 1, s^3),\quad -1 \leq s \leq 1.
  \end{aligned}
  \right.
  \]
  Perform the following tasks:
  \begin{enumerate}
    \item Compute the Jacobian matrix of $\mathbf{F}$, denoted $J\mathbf{F}$, and demonstrate that $\mathbf{F}$ is \textbf{not} conservative.
    \item Evaluate the line integrals
    \[
    \int_{\mathcal{C}_1} \mathbf{F} \cdot d\vec{x} \quad \text{and} \quad \int_{\mathcal{C}_2} \mathbf{F} \cdot d\vec{x}.
    \]
    \item Use the substitution $t = ({s^3 + 1})/{2}$ in the parametrization $r_1(t)$, and explain why the results of the integrals are identical, despite the fact that $\mathbf{F}$ is not conservative.
  \end{enumerate}
\end{Ej}

\begin{ptcb}
  \vspace{17cm}
\end{ptcb}
\iffalse
  x_{vals}=Sequence(x_L, x_R, density_{mesh})
  y_{vals}=Sequence(y_L, y_R, density_{mesh})
  z_{vals}=Sequence(z_L, z_R, density_{mesh})
  P(x,y,z)=-x^2+y^2+z^2
  Q(x,y,z)=x^2-y^2+z^2
  R(x,y,z)=x^2+y^2-z^2
  Curl_1(x,y,z)=Derivative(R,y)-Derivative(Q,z)
  Curl_2(x,y,z)=Derivative(P,z)-Derivative(R,x)
  Curl_3(x,y,z)=Derivative(Q,x)-Derivative(P,y)
  Zip(Zip(Zip(Vector((p, q, r), (p, q, r) + size_{arrows} (P(p, q, r), Q(p, q, r), R(p, q, r)) / sqrt((P(p, q, r))² + (Q(p, q, r))² + (R(p, q, r))²)), p, x_{vals}), q, y_{vals}), r, z_{vals})
  Zip(Zip(Zip(Vector((p, q, r), (p, q, r) + size_{arrows} (Curl_1(p, q, r), Curl_2(p, q, r), Curl_3(p, q, r)) / sqrt((Curl_1(p, q, r))² + (Curl_2(p, q, r))² + (Curl_3(p, q, r))²)), p, x_{vals}), q, y_{vals}), r, z_{vals})
\fi
%\end{multicols}
\end{document}