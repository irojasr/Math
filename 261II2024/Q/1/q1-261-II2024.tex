%----------------------------------------------------------------------------------------
%	PACKAGES AND OTHER DOCUMENT CONFIGURATIONS
%----------------------------------------------------------------------------------------

\documentclass[12pt]{article}
\usepackage[spanish]{babel} %Tildes
\usepackage[extreme]{savetrees} %Espaciado e interlineado. Comentar si no gusta el interlineado.
\usepackage[utf8]{inputenc} %Encoding para tildes
\usepackage[breakable,skins]{tcolorbox} %Cajitas
\usepackage{fancyhdr} % Se necesita para el título arriba
\usepackage{lastpage} % Se necesita para poner el número de página
\usepackage{amsmath,amsfonts,amssymb,amsthm} %simbolos y demás
\usepackage{mathabx} %más símbolos
\usepackage{physics} %simbolos de derivadas, bra-ket.
\usepackage{multicol}
\usepackage[customcolors]{hf-tikz}
\usepackage[shortlabels]{enumitem}
\usepackage{tikz}

%\def\darktheme
%%%%%%%%% === Document Configuration === %%%%%%%%%%%%%%

\pagestyle{fancy}
\setlength{\headheight}{14.49998pt} %NO MODIFICAR
\setlength{\footskip}{14.49998pt} %NO MODIFICAR

\ifx \darktheme\undefined

\lhead{Math261S11} % Nombre de autor
\chead{\textbf{Quiz 1}} % Titulo
\rhead{Name:\hspace*{5cm}}%\firstxmark} 
\lfoot{}%\lastxmark}
\cfoot{}
\rfoot{Page \thepage\ of\ \pageref{LastPage}} %A la derecha saldrá pág. 6 de 9. 
\else
\pagenumbering{gobble}
\pagecolor[rgb]{0,0,0}%{0.23,0.258,0.321}
\color[rgb]{1,1,1}
\fi

%%%%%%%%% === My T Color Box === %%%%%%%%%%%%%%

\ifx \darktheme\undefined
\newtcolorbox{ptcb}{
colframe = black,
colback = white,
breakable,
enhanced
}
\newtcolorbox{ptcbP}{
colframe = black,
colback = white,
coltitle = black,
colbacktitle = black!40,
title = Practice,
breakable,
enhanced
}

\else
\newtcolorbox{ptcb}{
colframe = white,
colback = black,
colupper = white,
breakable,
enhanced
}
\newtcolorbox{ptcbP}{
colframe = white,
colback = black,
colupper = white,
coltitle = white,
colbacktitle = black,
title = Practice,
breakable,
enhanced
}
\fi

%%%%%%%%% === Tikz para matrices === %%%%%%%%%%%%%%

\tikzset{
  style green/.style={
    set fill color=green!50!lime!60,
    set border color=white,
  },
  style cyan/.style={
    set fill color=cyan!90!blue!60,
    set border color=white,
  },
  style orange/.style={
    set fill color=orange!80!red!60,
    set border color=white,
  },
  row/.style={
    above left offset={-0.15,0.31},
    below right offset={0.15,-0.125},
    #1
  },
  col/.style={
    above left offset={-0.1,0.3},
    below right offset={0.15,-0.15},
    #1
  }
}

%%%%%%%%% === Theorems and suchlike === %%%%%%%%%%%%%%

\theoremstyle{plain}
\newtheorem{Th}{Theorem}  %%% Theorem 1.1
\newtheorem*{nTh}{Theorem}             %%% No-numbered Theorem
\newtheorem{Prop}[Th]{Proposition}     %%% Proposition 1.2
\newtheorem{Lem}[Th]{Lemma}             %%% Lemma 1.3
\newtheorem*{nLem}{Lemma}               %%% No-numbered Lemma
\newtheorem{Cor}[Th]{Corollary}        %%% Corollary 1.4
\newtheorem*{nCor}{Corollary}          %%% No-numbered Corollary

\theoremstyle{definition}
\newtheorem*{Def}{Definition}       %%% Definition 1.5
\newtheorem*{nonum-Def}{Definition}    %%% No number Definition
\newtheorem*{nEx}{Example}             %%% No number Example
\newtheorem{Ex}[Th]{Example}           %%% Example
\newtheorem{Ej}[Th]{Exercise}         %%% Exercise
\newtheorem*{nEj}{Exercise}           %%% No number Excercise
\newtheorem*{Not}{Notation}       %%% Definition 1.5

\theoremstyle{remark}
\newtheorem*{Rmk}{Remark}      %%%Remark 1.6

%\numberwithin{equation}{section}

\setlength{\parindent}{3ex}

%%====== Useful macros: =======%%%

\DeclareMathOperator{\gen}{gen}     %%%set generated by...
\DeclareMathOperator{\Rng}{Rng}     %%%rangomat
\DeclareMathOperator{\Nul}{Nul}     %%%rangomat
\DeclareMathOperator{\Proy}{Proy}   %%%proyección
\DeclareMathOperator{\id}{id}       %%%identity operator

\newcommand{\al}{\alpha}            %%%short for \alpha
\newcommand{\la}{\lambda}           %%%short for \lambda
\newcommand{\sg}{\sigma}            %%%short for \sigma
\newcommand{\te}{\theta}                %% short for  \theta
\renewcommand{\l}{\ell}

\newcommand{\thickhat}[1]{\mathbf{\hat{\text{$#1$}}}}
\newcommand{\ii}{\vu{\imath}}
\newcommand{\jj}{\vu{\jmath}}
\newcommand{\kk}{\thickhat{k}}

\newcommand{\bC}{\mathbb{C}}        %%%complex numbers
\newcommand{\bN}{\mathbb{N}}        %%%natural numbers
\newcommand{\bP}{\mathbb{P}}        %%%polynomials
\newcommand{\bR}{\mathbb{R}}        %%%real numbers
\newcommand{\bZ}{\mathbb{Z}}        %%%integer numbers
\newcommand{\cB}{\mathcal{B}}       %%%basis
\newcommand{\cC}{\mathcal{C}}       %%%basis
\newcommand{\cM}{\mathcal{M}}       %%%matrix family

\newcommand{\sT}{\mathsf{T}}        %%%traspuesta

\renewcommand{\geq}{\geqslant}      %%%(to save typing)
\renewcommand{\leq}{\leqslant}      %%%(to save typing)
\newcommand{\x}{\times}             %%%product
\renewcommand{\:}{\colon}           %%%colon in  f: A -> B
\newcommand{\isom}{\simeq}              %% isomorfismo

\newcommand{\un}[1]{\underline{#1}}
\newcommand{\half}{\frac12}

\newcommand*{\Cdot}{{\raisebox{-0.25ex}{\scalebox{1.5}{$\cdot$}}}}      %% cdot más grande
\renewcommand{\.}{\Cdot}                %% producto escalar

\newcommand{\twobyone}[2]{\begin{pmatrix} %% 2 x 1 matrix
  #1 \\ #2 \end{pmatrix}}
  \newcommand{\twobytwo}[4]{\begin{pmatrix} %% 2 x 2 matrix
    #1 & #2 \\ #3 & #4 \end{pmatrix}}
    \newcommand{\twobythree}[6]{\begin{pmatrix} %% 2 x 3 matrix
        #1 & #2 & #3\\ #4 & #5 & #6 \end{pmatrix}}
\newcommand{\threebyone}[3]{\begin{pmatrix} %% 3 x 1 matrix
  #1 \\ #2 \\ #3 \end{pmatrix}}
  \newcommand{\threebytwo}[6]{\begin{pmatrix} %% 3 x 1 matrix
    #1 & #2\\ #3 & #4\\ #5&#6 \end{pmatrix}}
\newcommand{\threebythree}[9]{\begin{pmatrix} %% 3 x 3 matrix
  #1 & #2 & #3 \\ #4 & #5 & #6 \\ #7 & #8 & #9 \end{pmatrix}}

\newcommand{\To}{\Rightarrow}

\newcommand{\vaf}{\overrightarrow}

\newcommand{\set}[1]{\{\,#1\,\}}    %% set notation
\newcommand{\Set}[1]{\biggl\{\,#1\,\biggr\}} %% set notation (large)
\newcommand{\red}[1]{\textcolor{red}{#1}}
\newcommand{\blu}[1]{\textcolor{blue}{#1}}
\newcommand{\word}[1]{\quad\text{#1}\quad} %% texto intercalado
\newcommand{\br}[1]{\left\langle #1\right\rangle}
%----------------------------------------------------------------------------------------
%	ARTICLE CONTENTS
%----------------------------------------------------------------------------------------

\begin{document}
%\begin{multicols}{2}

  \begin{Ej}
    Consider the vectors $\vec{u} = \br{1, 3, 4}$ and $\vec{v} = -2\jj + 3\kk$. Examine the following attempt to evaluate $3\vec{v} + 2\vec{u}$:
    \begin{ptcb}
    \begin{align}
        3\vec{v} + 2\vec{u} &= 3\br{1, 3, 4} + 2(-2\jj + 3\kk) \\
        &= (3 + 9 + 4) + 2(-2\jj + 3\kk) \\
        &= 16 + 2(-2\jj + 3\kk) \\
        &= 18(-2\jj + 3\kk) \\
        &= (-16\jj + 21\kk) \\
        &= \br{-16, 0, 21}
      \end{align}
    \end{ptcb}
    
    In the following space complete the tasks below:
    \begin{itemize}[itemsep=0em]
      \item Answer: should the result of this operation be a vector or a scalar?
      \item Identify at least 3 mistakes in the process. Refer to the equation numbers where errors occur.
    \end{itemize}
  \end{Ej}
  
  \begin{ptcb}
    \vspace{5.3cm}
  \end{ptcb}
  
  \begin{Ej}
    Consider the following pairs of vectors:
    \setcounter{equation}{0}
    \begin{gather}
      \ii + \sqrt{3}\jj \quad \text{and} \quad \sqrt{3}\ii + 3\jj \\
      2\ii + 4\jj + 6\kk \quad \text{and} \quad 4\ii + 6\jj + 8\kk \\
      \ii - \sqrt{3}\jj + \kk \quad \text{and} \quad -3\ii + 3\kk
    \end{gather}
  
    In the following space complete the tasks below:
    \begin{itemize}[itemsep=0em]
      \item Define parallel and orthogonal vectors.
      \item Identify which pair is parallel, which is orthogonal, and which is neither.
    \end{itemize}
  \end{Ej}
  
  \begin{ptcb}
    \vspace{5.3cm}
  \end{ptcb}
  \iffalse
  \begin{Ej}
    Consider the curve $r(t) = (t, (4 - t^2)^2)$ for $1 \leq t \leq 3$. 
    \begin{enumerate}
        \item Find a parametrization of the line segment between the endpoints of this curve.
        \item Find the velocity vector for this curve.
        \item Verify the following given solution and correct it if there are any mistakes:
        \begin{ptcb}
          \begin{align*}
            \text{Parametrization:} & \quad t(3) + (1 - t)(1) = 2t - 1 \\
            \text{Velocity vector:} & \quad v(t) = r'(t) = (1, -4t(4 - t^2))
          \end{align*}
        \end{ptcb}
            
    \end{enumerate}
    \end{Ej}

    \begin{ptcb}
\begin{enumerate}
  \item \textbf{Find the endpoints:}
  \begin{align*}
    r(1) &= (1, (4 - 1^2)^2) = (1, 9) \\
    r(3) &= (3, (4 - 3^2)^2) = (3, 25)
  \end{align*}

  \item \textbf{Parametrize the line segment between the endpoints:}
  \begin{align*}
    \text{The line segment is:} & \quad t(3, 25) + (1 - t)(1, 9) \\
    &= (3t + 1 - t, 25t + 9 - 9t) \\
    &= (2t + 1, 16t + 9), \quad 0 \leq t \leq 1
  \end{align*}

  \item \textbf{Find the velocity vector:}
  \begin{align*}
    v(t) &= r'(t) = \left( \frac{d}{dt} t, \frac{d}{dt} (4 - t^2)^2 \right) \\
    &= (1, -4t(4 - t^2))
  \end{align*}
\end{enumerate}
\textbf{Mistakes:}
\begin{enumerate}
  \item The endpoints are misinterpreted as they should be vectors. The points $t=1$ and $t=3$ represent the initial and end times. There's also no time interval for the new parametrization.
  \item The endpoints should be determined by evaluating the curve at the initial and ending times $t = 1$ and $t = 3$.
\end{enumerate}
    \end{ptcb}
    \fi
%\end{multicols}
\end{document}