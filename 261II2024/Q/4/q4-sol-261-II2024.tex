%----------------------------------------------------------------------------------------
%	PACKAGES AND OTHER DOCUMENT CONFIGURATIONS
%----------------------------------------------------------------------------------------

\documentclass[12pt]{article}
\usepackage[spanish]{babel} %Tildes
\usepackage[extreme]{savetrees} %Espaciado e interlineado. Comentar si no gusta el interlineado.
\usepackage[utf8]{inputenc} %Encoding para tildes
\usepackage[breakable,skins]{tcolorbox} %Cajitas
\usepackage{fancyhdr} % Se necesita para el título arriba
\usepackage{lastpage} % Se necesita para poner el número de página
\usepackage{amsmath,amsfonts,amssymb,amsthm} %simbolos y demás
\usepackage{mathabx} %más símbolos
\usepackage{physics} %simbolos de derivadas, bra-ket.
\usepackage{multicol}
\usepackage[customcolors]{hf-tikz}
\usepackage[shortlabels]{enumitem}
\usepackage{tikz}

%\def\darktheme
%%%%%%%%% === Document Configuration === %%%%%%%%%%%%%%

\pagestyle{fancy}
\setlength{\headheight}{14.49998pt} %NO MODIFICAR
\setlength{\footskip}{14.49998pt} %NO MODIFICAR

\ifx \darktheme\undefined

\lhead{Math261S11} % Nombre de autor
\chead{\textbf{Quiz 4 Solutions}} % Titulo
\rhead{Name:\hspace*{5cm}}%\firstxmark} 
\lfoot{}%\lastxmark}
\cfoot{}
\rfoot{Page \thepage\ of\ \pageref{LastPage}} %A la derecha saldrá pág. 6 de 9. 
\else
\pagenumbering{gobble}
\pagecolor[rgb]{0,0,0}%{0.23,0.258,0.321}
\color[rgb]{1,1,1}
\fi

%%%%%%%%% === My T Color Box === %%%%%%%%%%%%%%

\ifx \darktheme\undefined
\newtcolorbox{ptcb}{
colframe = black,
colback = white,
breakable,
enhanced
}
\newtcolorbox{ptcbP}{
colframe = black,
colback = white,
coltitle = black,
colbacktitle = black!40,
title = Practice,
breakable,
enhanced
}

\else
\newtcolorbox{ptcb}{
colframe = white,
colback = black,
colupper = white,
breakable,
enhanced
}
\newtcolorbox{ptcbP}{
colframe = white,
colback = black,
colupper = white,
coltitle = white,
colbacktitle = black,
title = Practice,
breakable,
enhanced
}
\fi

%%%%%%%%% === Tikz para matrices === %%%%%%%%%%%%%%

\tikzset{
  style green/.style={
    set fill color=green!50!lime!60,
    set border color=white,
  },
  style cyan/.style={
    set fill color=cyan!90!blue!60,
    set border color=white,
  },
  style orange/.style={
    set fill color=orange!80!red!60,
    set border color=white,
  },
  row/.style={
    above left offset={-0.15,0.31},
    below right offset={0.15,-0.125},
    #1
  },
  col/.style={
    above left offset={-0.1,0.3},
    below right offset={0.15,-0.15},
    #1
  }
}

%%%%%%%%% === Theorems and suchlike === %%%%%%%%%%%%%%

\theoremstyle{plain}
\newtheorem{Th}{Theorem}  %%% Theorem 1.1
\newtheorem*{nTh}{Theorem}             %%% No-numbered Theorem
\newtheorem{Prop}[Th]{Proposition}     %%% Proposition 1.2
\newtheorem{Lem}[Th]{Lemma}             %%% Lemma 1.3
\newtheorem*{nLem}{Lemma}               %%% No-numbered Lemma
\newtheorem{Cor}[Th]{Corollary}        %%% Corollary 1.4
\newtheorem*{nCor}{Corollary}          %%% No-numbered Corollary

\theoremstyle{definition}
\newtheorem*{Def}{Definition}       %%% Definition 1.5
\newtheorem*{nonum-Def}{Definition}    %%% No number Definition
\newtheorem*{nEx}{Example}             %%% No number Example
\newtheorem{Ex}[Th]{Example}           %%% Example
\newtheorem{Ej}[Th]{Exercise}         %%% Exercise
\newtheorem*{nEj}{Exercise}           %%% No number Excercise
\newtheorem*{Not}{Notation}       %%% Definition 1.5

\theoremstyle{remark}
\newtheorem*{Rmk}{Remark}      %%%Remark 1.6

%\numberwithin{equation}{section}

\setlength{\parindent}{3ex}

%%====== Useful macros: =======%%%

\DeclareMathOperator{\gen}{gen}     %%%set generated by...
\DeclareMathOperator{\Rng}{Rng}     %%%rangomat
\DeclareMathOperator{\Nul}{Nul}     %%%rangomat
\DeclareMathOperator{\Proy}{Proy}   %%%proyección
\DeclareMathOperator{\id}{id}       %%%identity operator

\newcommand{\al}{\alpha}            %%%short for \alpha
\newcommand{\la}{\lambda}           %%%short for \lambda
\newcommand{\sg}{\sigma}            %%%short for \sigma
\newcommand{\te}{\theta}                %% short for  \theta
\renewcommand{\l}{\ell}

\newcommand{\thickhat}[1]{\mathbf{\hat{\text{$#1$}}}}
\newcommand{\ii}{\vu{\imath}}
\newcommand{\jj}{\vu{\jmath}}
\newcommand{\kk}{\thickhat{k}}

\newcommand{\bC}{\mathbb{C}}        %%%complex numbers
\newcommand{\bN}{\mathbb{N}}        %%%natural numbers
\newcommand{\bP}{\mathbb{P}}        %%%polynomials
\newcommand{\bR}{\mathbb{R}}        %%%real numbers
\newcommand{\bZ}{\mathbb{Z}}        %%%integer numbers
\newcommand{\cB}{\mathcal{B}}       %%%basis
\newcommand{\cC}{\mathcal{C}}       %%%basis
\newcommand{\cM}{\mathcal{M}}       %%%matrix family

\newcommand{\sT}{\mathsf{T}}        %%%traspuesta

\renewcommand{\geq}{\geqslant}      %%%(to save typing)
\renewcommand{\leq}{\leqslant}      %%%(to save typing)
\newcommand{\x}{\times}             %%%product
\renewcommand{\:}{\colon}           %%%colon in  f: A -> B
\newcommand{\isom}{\simeq}              %% isomorfismo

\newcommand{\un}[1]{\underline{#1}}
\newcommand{\half}{\frac12}

\newcommand*{\Cdot}{{\raisebox{-0.25ex}{\scalebox{1.5}{$\cdot$}}}}      %% cdot más grande
\renewcommand{\.}{\Cdot}                %% producto escalar

\newcommand{\twobyone}[2]{\begin{pmatrix} %% 2 x 1 matrix
  #1 \\ #2 \end{pmatrix}}
  \newcommand{\twobytwo}[4]{\begin{pmatrix} %% 2 x 2 matrix
    #1 & #2 \\ #3 & #4 \end{pmatrix}}
    \newcommand{\twobythree}[6]{\begin{pmatrix} %% 2 x 3 matrix
        #1 & #2 & #3\\ #4 & #5 & #6 \end{pmatrix}}
\newcommand{\threebyone}[3]{\begin{pmatrix} %% 3 x 1 matrix
  #1 \\ #2 \\ #3 \end{pmatrix}}
  \newcommand{\threebytwo}[6]{\begin{pmatrix} %% 3 x 1 matrix
    #1 & #2\\ #3 & #4\\ #5&#6 \end{pmatrix}}
\newcommand{\threebythree}[9]{\begin{pmatrix} %% 3 x 3 matrix
  #1 & #2 & #3 \\ #4 & #5 & #6 \\ #7 & #8 & #9 \end{pmatrix}}

\newcommand{\To}{\Rightarrow}

\newcommand{\vaf}{\overrightarrow}

\newcommand{\set}[1]{\{\,#1\,\}}    %% set notation
\newcommand{\Set}[1]{\biggl\{\,#1\,\biggr\}} %% set notation (large)
\newcommand{\red}[1]{\textcolor{red}{#1}}
\newcommand{\blu}[1]{\textcolor{blue}{#1}}
\newcommand{\word}[1]{\quad\text{#1}\quad} %% texto intercalado
\newcommand{\br}[1]{\left\langle #1\right\rangle}
\newcommand{\nb}{\nabla}
%----------------------------------------------------------------------------------------
%	ARTICLE CONTENTS
%----------------------------------------------------------------------------------------

\begin{document}
%\begin{multicols}{2}

\begin{Ej}
  Short answers:
  \begin{enumerate}
   \item If $\vec{u}$ and $\vec{v}$ are parallel, what is the algebraic relation between them?
   \item Suppose a line $\l$ is normal to a plane $\pi$. What is the relation between $\vec{v}_\l$ and $\vec{n}_\pi$?
   \item If $f:\bR^3 \to \bR$ is a function, how many components does its gradient $\nabla f$ have?
   \item If $\nabla f(\vec{a}) \cdot \vec{u} < 0$, does $f$ increase or decrease in the direction of $\vec{u}$ at $\vec{a}$?
   \item If $f$ has $\det Hf > 0$ and $f_{xx} > 0$ at $\vec{x} = \vec{a}$, can you conclude if $f$ has a maximum, minimum, or saddle point at $\vec{a}$?
  \end{enumerate}
\end{Ej}

    \begin{ptcb}
     \begin{enumerate}
      \item $\vec{u}$ and $\vec v$ are multiples. This means $\vec{v}=c\vec u$ for some scalar $c$.
      \item If the line is normal to the plane, it's parallel to the normal vector of the plane. So $\vec{v}_\l\parallel\vec n_\pi$.
      \item A three-dimensional function has 3 entries on its gradient.
      \item This means that the directional derivative is negative in direction of $u$. So it's aligned opposite to $\nb f(\vec a)$, this means that it's decreasing in direction of $\vec u$ at $\vec a$.
      \item Positive determinant of Hessian implies that the function doesn't have a saddle point. As $f_{xx}$ is positive, $f$ looks like \emph{a happy face} in the $x$ direction, therefore it's a minimum. 
     \end{enumerate}
    \end{ptcb}

    \begin{Ej}
      Consider the function $f(x,y) = 8x + 8y$ on the region described by $\{4x^2 = y^2-1\}$. There are two critical points. Follow these steps to classify them:
      \begin{enumerate}
       \item Identify the function $f$ you're optimizing.
       \item Identify the constraint as $g = 0$.
       \item Write the Lagrange equation: $\nabla f = \lambda \nabla g$.
       \item Solve for $\vec{x}$ in terms of $\lambda$, then use $g = 0$ to find $\lambda$.
       \item Find $\vec{x}$ using $\lambda$ and classify the points by evaluating $f$.
      \end{enumerate}
   \end{Ej}
    \begin{ptcb}
      \begin{enumerate}
        \item The function to optimize is $f(x,y) = 8x+8y$.
        \item The constraint is 
        $$g(x,y)=4x^2 - y^2 +1=0.$$
        \item The Lagrange equation is 
        $$(8,8)=\la(8x,-2y).$$
        \item We may solve for $x,y$ via the equations
        $$
        \left\lbrace\begin{aligned}
          &8=8x\\
          &8=-2y
        \end{aligned}\right.
        \To
        \left\lbrace\begin{aligned}
          &\frac{1}{\la} =x\\
          &\frac{-4}{\la}= y
        \end{aligned}\right.
        $$
        So plugging these values into $g=0$ we get:
        $$4\left(\frac{1}{\la}\right)^2-\left(\frac{-4}{\la}\right)^2+1=0\To\frac{-12}{\la^2}=-1\To\la=\pm 2\sqrt{3}.$$
        \item The critical values are thus 
        $$(x_1,y_1)=\left(\frac{1}{2\sqrt{3}},\frac{-2}{\sqrt{3}}\right)\word{and}(x_2,y_2)=\left(\frac{-1}{2\sqrt{3}},\frac{2}{\sqrt{3}}\right).$$
        Plugging them into $f$ we get the values 
        $$f(x_1,y_1)=\frac{8}{2\sqrt{3}}+\frac{-16}{\sqrt{3}}=\frac{-12}{\sqrt{3}}\word{and}f(x_2,y_2)=\frac{-8}{2\sqrt{3}}+\frac{16}{\sqrt{3}}=\frac{12}{\sqrt{3}}$$
        where we see that $f(x_1,y_1)<f(x_2,y_2)$ so $(x_2,y_2)$ is a maximizing point whereas $(x_1,y_1)$ is a minimizing point.
      \end{enumerate}
    \end{ptcb}
%\end{multicols}
\end{document}