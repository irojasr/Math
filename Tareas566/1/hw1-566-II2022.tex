\documentclass[12pt]{memoir}

\def\nsemestre {II}
\def\nterm {Fall}
\def\nyear {2022}
\def\nprofesor {Jamie Juul}
\def\nsigla {MATH566}
\def\nsiglahead {Abstract Algebra}
\def\nextra {HW1}
\def\nlang {ENG}
\input{../../headerVarillyDiff}

\begin{document}
%\begin{multicols}{2}

\begin{Ej}Do the following exercises from Artin's Algebra:
  \begin{enumerate}[i)]
    \item Consider
    $$\vec a=\threebyone{a_1}{\vdots}{a_n},\ \vec b=(b_1,\dots,b_n)$$
    be a column and row vector respectively. Compute the products $\vec{a}\vec{b}$ and $\vec{b}\vec{a}$.
    \item Consider $A=\threebythree{1}{1}{1}{0}{1}{1}{0}{0}{1}$. Find a formula for $A^n$ and prove it by induction.
    \item Verify the rule 
    $$\det AB = (\det A)(\det B)$$ 
    for the matrices $A=\twobytwo{2}{3}{1}{4}$ and $B=\twobytwo{1}{1}{5}{-2}$.
  \end{enumerate}
\end{Ej}

\begin{ptcbr}
\begin{enumerate}[i)]
    \item $\vec a\vec b$ turns into a $[1\x 1]$ matrix. This is the inner product of this vectors. The result is $a_1b_1+\dots+a_nb_n$.\par 
    On the other hand $\vec b\vec a$ is a $[n\x n]$ matrix, the outer product of $\vec a$ and $\vec b$. This is a rank 1 matrix whose entries are given by $(a_ib_j)_{i,j\in[n]}$.
    \item Computing some powers of $A$ we see that
    $$A^2=\threebythree{1}{2}{3}{0}{1}{2}{0}{0}{1},\ A^3=\threebythree{1}{3}{6}{0}{1}{3}{0}{0}{1},\ A^4=\threebythree{1}{4}{10}{0}{1}{4}{0}{0}{1}.$$
    The numbers in the first row of $A$ are easy to characterize, one, $n$ itself, and the $n^{\text{th}}$ triangular number. The formula for $A^n$ must be 
    $$A^n=\threebythree{1}{n}{\binom{n+1}{2}}{0}{1}{n}{0}{0}{1}.$$
    With the matrices above we have the base case, so if we suppose that the formula is valid up to $n$, we would like to check it for $n+1$. In this case 
    $$A^nA=\threebythree{1}{n+1}{\binom{n+1}{2}+(n+1)}{0}{1}{n+1}{0}{0}{1}$$
    and since $\binom{n+1}{2}+(n+1)=\binom{n+2}{2}$, we have proven our result.
    \item We can see that
    $$AB=\twobytwo{17}{-4}{21}{-7}\To \det(AB)=-35.$$
    And by themselves, $\det A=5$ and $\det B= -7$ so the formula holds. 
\end{enumerate}
\end{ptcbr}

%\end{multicols}
\end{document} 