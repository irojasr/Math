%----------------------------------------------------------------------------------------
%	PACKAGES AND OTHER DOCUMENT CONFIGURATIONS
%----------------------------------------------------------------------------------------

\documentclass[12pt]{article}
\usepackage[spanish]{babel} %Tildes
\usepackage[extreme]{savetrees} %Espaciado e interlineado. Comentar si no gusta el interlineado.
\usepackage[utf8]{inputenc} %Encoding para tildes
\usepackage[breakable,skins]{tcolorbox} %Cajitas
\usepackage{fancyhdr} % Se necesita para el título arriba
\usepackage{lastpage} % Se necesita para poner el número de página
\usepackage{amsmath,amsfonts,amssymb,amsthm} %simbolos y demás
\usepackage{mathabx} %más símbolos
\usepackage{physics} %simbolos de derivadas, bra-ket.
\usepackage{multicol}
\usepackage[customcolors]{hf-tikz}
\usepackage[shortlabels]{enumitem}
\usepackage{tikz}

%\def\darktheme
%%%%%%%%% === Document Configuration === %%%%%%%%%%%%%%

\pagestyle{fancy}
\setlength{\headheight}{14.49998pt} %NO MODIFICAR
\setlength{\footskip}{14.49998pt} %NO MODIFICAR

\ifx \darktheme\undefined

\lhead{Math161S1} % Nombre de autor
\chead{\textbf{Quiz 2 - Solutions}} % Titulo
\rhead{Name: Solutions}%\firstxmark} 
\lfoot{}%\lastxmark}
\cfoot{}
\rfoot{Page \thepage\ of\ \pageref{LastPage}} %A la derecha saldrá pág. 6 de 9. 
\else
\pagenumbering{gobble}
\pagecolor[rgb]{0,0,0}%{0.23,0.258,0.321}
\color[rgb]{1,1,1}
\fi

%%%%%%%%% === My T Color Box === %%%%%%%%%%%%%%

\ifx \darktheme\undefined
\newtcolorbox{ptcb}{
colframe = black,
colback = white,
breakable,
enhanced
}
\newtcolorbox{ptcbP}{
colframe = black,
colback = white,
coltitle = black,
colbacktitle = black!40,
title = Practice,
breakable,
enhanced
}

\else
\newtcolorbox{ptcb}{
colframe = white,
colback = black,
colupper = white,
breakable,
enhanced
}
\newtcolorbox{ptcbP}{
colframe = white,
colback = black,
colupper = white,
coltitle = white,
colbacktitle = black,
title = Practice,
breakable,
enhanced
}
\fi

%%%%%%%%% === Tikz para matrices === %%%%%%%%%%%%%%

\tikzset{
  style green/.style={
    set fill color=green!50!lime!60,
    set border color=white,
  },
  style cyan/.style={
    set fill color=cyan!90!blue!60,
    set border color=white,
  },
  style orange/.style={
    set fill color=orange!80!red!60,
    set border color=white,
  },
  row/.style={
    above left offset={-0.15,0.31},
    below right offset={0.15,-0.125},
    #1
  },
  col/.style={
    above left offset={-0.1,0.3},
    below right offset={0.15,-0.15},
    #1
  }
}

%%%%%%%%% === Theorems and suchlike === %%%%%%%%%%%%%%

\theoremstyle{plain}
\newtheorem{Th}{Theorem}  %%% Theorem 1.1
\newtheorem*{nTh}{Theorem}             %%% No-numbered Theorem
\newtheorem{Prop}[Th]{Proposition}     %%% Proposition 1.2
\newtheorem{Lem}[Th]{Lemma}             %%% Lemma 1.3
\newtheorem*{nLem}{Lemma}               %%% No-numbered Lemma
\newtheorem{Cor}[Th]{Corollary}        %%% Corollary 1.4
\newtheorem*{nCor}{Corollary}          %%% No-numbered Corollary

\theoremstyle{definition}
\newtheorem*{Def}{Definition}       %%% Definition 1.5
\newtheorem*{nonum-Def}{Definition}    %%% No number Definition
\newtheorem*{nEx}{Example}             %%% No number Example
\newtheorem{Ex}[Th]{Example}           %%% Example
\newtheorem{Ej}[Th]{Exercise}         %%% Exercise
\newtheorem*{nEj}{Exercise}           %%% No number Excercise
\newtheorem*{Not}{Notation}       %%% Definition 1.5

\theoremstyle{remark}
\newtheorem*{Rmk}{Remark}      %%%Remark 1.6

%\numberwithin{equation}{section}

\setlength{\parindent}{3ex}

%%====== Useful macros: =======%%%

\DeclareMathOperator{\gen}{gen}     %%%set generated by...
\DeclareMathOperator{\Rng}{Rng}     %%%rangomat
\DeclareMathOperator{\Nul}{Nul}     %%%rangomat
\DeclareMathOperator{\Proy}{Proy}   %%%proyección
\DeclareMathOperator{\id}{id}       %%%identity operator

\newcommand{\al}{\alpha}            %%%short for \alpha
\newcommand{\la}{\lambda}           %%%short for \lambda
\newcommand{\sg}{\sigma}            %%%short for \sigma
\newcommand{\te}{\theta}                %% short for  \theta
\renewcommand{\l}{\ell}

\newcommand{\thickhat}[1]{\mathbf{\hat{\text{$#1$}}}}
\newcommand{\ii}{\vu{\imath}}
\newcommand{\jj}{\vu{\jmath}}
\newcommand{\kk}{\thickhat{k}}

\newcommand{\bC}{\mathbb{C}}        %%%complex numbers
\newcommand{\bN}{\mathbb{N}}        %%%natural numbers
\newcommand{\bP}{\mathbb{P}}        %%%polynomials
\newcommand{\bR}{\mathbb{R}}        %%%real numbers
\newcommand{\bZ}{\mathbb{Z}}        %%%integer numbers
\newcommand{\cB}{\mathcal{B}}       %%%basis
\newcommand{\cC}{\mathcal{C}}       %%%basis
\newcommand{\cM}{\mathcal{M}}       %%%matrix family

\newcommand{\sT}{\mathsf{T}}        %%%traspuesta

\renewcommand{\geq}{\geqslant}      %%%(to save typing)
\renewcommand{\leq}{\leqslant}      %%%(to save typing)
\newcommand{\x}{\times}             %%%product
\renewcommand{\:}{\colon}           %%%colon in  f: A -> B
\newcommand{\isom}{\simeq}              %% isomorfismo

\newcommand{\un}[1]{\underline{#1}}
\newcommand{\half}{\frac12}

\newcommand*{\Cdot}{{\raisebox{-0.25ex}{\scalebox{1.5}{$\cdot$}}}}      %% cdot más grande
\renewcommand{\.}{\Cdot}                %% producto escalar

\newcommand{\twobyone}[2]{\begin{pmatrix} %% 2 x 1 matrix
  #1 \\ #2 \end{pmatrix}}
  \newcommand{\twobytwo}[4]{\begin{pmatrix} %% 2 x 2 matrix
    #1 & #2 \\ #3 & #4 \end{pmatrix}}
    \newcommand{\twobythree}[6]{\begin{pmatrix} %% 2 x 3 matrix
        #1 & #2 & #3\\ #4 & #5 & #6 \end{pmatrix}}
\newcommand{\threebyone}[3]{\begin{pmatrix} %% 3 x 1 matrix
  #1 \\ #2 \\ #3 \end{pmatrix}}
  \newcommand{\threebytwo}[6]{\begin{pmatrix} %% 3 x 1 matrix
    #1 & #2\\ #3 & #4\\ #5&#6 \end{pmatrix}}
\newcommand{\threebythree}[9]{\begin{pmatrix} %% 3 x 3 matrix
  #1 & #2 & #3 \\ #4 & #5 & #6 \\ #7 & #8 & #9 \end{pmatrix}}

\newcommand{\To}{\Rightarrow}

\newcommand{\vaf}{\overrightarrow}

\newcommand{\set}[1]{\{\,#1\,\}}    %% set notation
\newcommand{\Set}[1]{\biggl\{\,#1\,\biggr\}} %% set notation (large)
\newcommand{\red}[1]{\textcolor{red}{#1}}
\newcommand{\blu}[1]{\textcolor{blue}{#1}}

%----------------------------------------------------------------------------------------
%	ARTICLE CONTENTS
%----------------------------------------------------------------------------------------

\begin{document}
%\begin{multicols}{2}

\begin{Ej}
  Explain what does the mnemonic \textbf{LIATE} or \textbf{LIPTE} means and how should one use it when integrating by parts.
\end{Ej}

\begin{ptcb}
  \textbf{LI(A/P)TE} is an acronym which means Logarithm, Inverse (trigonometric), Algebraic/Polynomial (function), Trigonometric, Exponential. It determines the order of precedence for which function to differentiate when using integration by parts.
  \end{ptcb}

  \begin{Ej}
    When using integration by parts, give an example of an integral where $\dd v=1\dd x$ and $u$ is the function we are looking to integrate. It is \textbf{not required} to solve the integral.
  \end{Ej}
  
  
  \begin{ptcb}
    Any integral of the form $\int\log(x)\dd x$, $\int\arcsin(x)\dd x$, $\int\arctan(x)\dd x$ is a valid answer. Also the integral $\int\log^2(x)\dd x$, the third question of this quiz, is a valid answer.\par 
    Answers such as $\int xe^x\dd x$, $\int x\sin(x)\dd x$ are not, because you have to take $\dd v=x\dd x$, for example.\par 
    There are some edge cases where another integral will be accepted as an answer.
    \end{ptcb}
  
    \begin{Ej}
Find an antiderivative for the function $\log^2(x)$ using integration by parts.
    \end{Ej}
    
    
    \begin{ptcb}
      By integration by parts:
      $$
      \left\lbrace
      \begin{aligned}
        &u=\log^2(x)\To\dd u=\frac{2\log(x)}{x}\dd x\\
        &\dd v=\dd x\To v=x
      \end{aligned}
      \right.
      \To I=x\log^2(x)-\int 2\log(x)\dd x=\un{x\log^2(x)-2x\log(x)+2x}
      $$
      \end{ptcb}
    
      
    \begin{Ej}
      Use integration by parts to evaluate the integral $\displaystyle\int\left(\frac{\log(x)}{x}\right)^3\dd x$.
          \end{Ej}
          
          
          \begin{ptcb}
            By integration by parts:
            $$
            \left\lbrace
            \begin{aligned}
              &u=\log^3(x)\To\dd u=\frac{3\log^2(x)}{x}\dd x\\
              &\dd v=\frac{1}{x^3}\dd x\To v=\frac{-1}{2x^2}
            \end{aligned}
            \right.
            \To I=\frac{-\log^3(x)}{2x^2}-\int \frac{-3\log^2(x)}{2x^3}\dd x=\frac{-\log^3(x)}{2x^2}+\frac{3}{2}I_2
            $$
            Applying parts to $I_2$ we get
            $$
            \left\lbrace
            \begin{aligned}
              &u_2=\log^2(x)\To\dd u_2=\frac{2\log(x)}{x}\dd x\\
              &\dd v_2=\frac{1}{x^3}\dd x\To v_2=\frac{-1}{2x^2}
            \end{aligned}
            \right.
            \To I_2=\frac{-\log^2(x)}{2x^2}-\int \frac{-\log(x)}{x^3}\dd x=-\frac{\log^2(x)}{2x^2}+I_3
            $$
            Once again we take
            $$
            \left\lbrace
            \begin{aligned}
              &u_3=\log(x)\To\dd u_3=\frac{1}{x}\dd x\\
              &\dd v_3=\frac{1}{x^3}\dd x\To v_3=\frac{-1}{2x^2}
            \end{aligned}
            \right.
            \To I_3=\frac{-\log(x)}{2x^2}-\int \frac{-1}{2x^3}\dd x=-\frac{\log(x)}{2x^2}+\frac{1}{2}\left(\frac{-1}{2x^2}\right)
            $$
            And collecting terms we get
            $$I=\frac{-\log^3(x)}{2x^2}+\frac{3}{2}\left\lbrack-\frac{\log^2(x)}{2x^2}+\left(-\frac{\log(x)}{2x^2}+\frac{1}{2}\left(\frac{-1}{2x^2}\right)\right)\right\rbrack=\un{\frac{-\log^3(x)}{2x^2}-\frac{3\log^2(x)}{4x^2}-\frac{3\log(x)}{4x^2}-\frac{3}{8x^2}}$$
            \end{ptcb}

%\end{multicols}
\end{document} 