%----------------------------------------------------------------------------------------
%	PACKAGES AND OTHER DOCUMENT CONFIGURATIONS
%----------------------------------------------------------------------------------------

\documentclass[12pt]{article}
\usepackage[spanish]{babel} %Tildes
\usepackage[extreme]{savetrees} %Espaciado e interlineado. Comentar si no gusta el interlineado.
\usepackage[utf8]{inputenc} %Encoding para tildes
\usepackage[breakable,skins]{tcolorbox} %Cajitas
\usepackage{fancyhdr} % Se necesita para el título arriba
\usepackage{lastpage} % Se necesita para poner el número de página
\usepackage{amsmath,amsfonts,amssymb,amsthm} %simbolos y demás
\usepackage{mathabx} %más símbolos
\usepackage{physics} %simbolos de derivadas, bra-ket.
\usepackage{multicol}
\usepackage[customcolors]{hf-tikz}
\usepackage[shortlabels]{enumitem}
\usepackage{tikz}
\usetikzlibrary{patterns}
\usepackage{siunitx}

%\def\darktheme
%%%%%%%%% === Document Configuration === %%%%%%%%%%%%%%

\pagestyle{fancy}
\setlength{\headheight}{14.49998pt} %NO MODIFICAR
\setlength{\footskip}{14.49998pt} %NO MODIFICAR

\ifx \darktheme\undefined

\lhead{Math161S1} % Nombre de autor
\chead{\textbf{Quiz 6}} % Titulo
\rhead{Name:\hspace*{5cm}}%\firstxmark} 
\lfoot{}%\lastxmark}
\cfoot{}
\rfoot{Page \thepage\ of\ \pageref{LastPage}} %A la derecha saldrá pág. 6 de 9. 
\else
\pagenumbering{gobble}
\pagecolor[rgb]{0,0,0}%{0.23,0.258,0.321}
\color[rgb]{1,1,1}
\fi

%%%%%%%%% === My T Color Box === %%%%%%%%%%%%%%

\ifx \darktheme\undefined
\newtcolorbox{ptcb}{
colframe = black,
colback = white,
breakable,
enhanced
}
\newtcolorbox{ptcbP}{
colframe = black,
colback = white,
coltitle = black,
colbacktitle = black!40,
title = Practice,
breakable,
enhanced
}

\else
\newtcolorbox{ptcb}{
colframe = white,
colback = black,
colupper = white,
breakable,
enhanced
}
\newtcolorbox{ptcbP}{
colframe = white,
colback = black,
colupper = white,
coltitle = white,
colbacktitle = black,
title = Practice,
breakable,
enhanced
}
\fi

%%%%%%%%% === Tikz para matrices === %%%%%%%%%%%%%%

\tikzset{
  style green/.style={
    set fill color=green!50!lime!60,
    set border color=white,
  },
  style cyan/.style={
    set fill color=cyan!90!blue!60,
    set border color=white,
  },
  style orange/.style={
    set fill color=orange!80!red!60,
    set border color=white,
  },
  row/.style={
    above left offset={-0.15,0.31},
    below right offset={0.15,-0.125},
    #1
  },
  col/.style={
    above left offset={-0.1,0.3},
    below right offset={0.15,-0.15},
    #1
  }
}

%%%%%%%%% === Theorems and suchlike === %%%%%%%%%%%%%%

\theoremstyle{plain}
\newtheorem{Th}{Theorem}  %%% Theorem 1.1
\newtheorem*{nTh}{Theorem}             %%% No-numbered Theorem
\newtheorem{Prop}[Th]{Proposition}     %%% Proposition 1.2
\newtheorem{Lem}[Th]{Lemma}             %%% Lemma 1.3
\newtheorem*{nLem}{Lemma}               %%% No-numbered Lemma
\newtheorem{Cor}[Th]{Corollary}        %%% Corollary 1.4
\newtheorem*{nCor}{Corollary}          %%% No-numbered Corollary

\theoremstyle{definition}
\newtheorem*{Def}{Definition}       %%% Definition 1.5
\newtheorem*{nonum-Def}{Definition}    %%% No number Definition
\newtheorem*{nEx}{Example}             %%% No number Example
\newtheorem{Ex}[Th]{Example}           %%% Example
\newtheorem{Ej}[Th]{Exercise}         %%% Exercise
\newtheorem*{nEj}{Exercise}           %%% No number Excercise
\newtheorem*{Not}{Notation}       %%% Definition 1.5

\theoremstyle{remark}
\newtheorem*{Rmk}{Remark}      %%%Remark 1.6

%\numberwithin{equation}{section}

\setlength{\parindent}{3ex}

%%====== Useful macros: =======%%%

\DeclareMathOperator{\gen}{gen}     %%%set generated by...
\DeclareMathOperator{\Rng}{Rng}     %%%rangomat
\DeclareMathOperator{\Nul}{Nul}     %%%rangomat
\DeclareMathOperator{\Proy}{Proy}   %%%proyección
\DeclareMathOperator{\id}{id}       %%%identity operator

\newcommand{\al}{\alpha}            %%%short for \alpha
\newcommand{\la}{\lambda}           %%%short for \lambda
\newcommand{\sg}{\sigma}            %%%short for \sigma
\newcommand{\te}{\theta}                %% short for  \theta
\renewcommand{\l}{\ell}

\newcommand{\thickhat}[1]{\mathbf{\hat{\text{$#1$}}}}
\newcommand{\ii}{\vu{\imath}}
\newcommand{\jj}{\vu{\jmath}}
\newcommand{\kk}{\thickhat{k}}

\newcommand{\bC}{\mathbb{C}}        %%%complex numbers
\newcommand{\bN}{\mathbb{N}}        %%%natural numbers
\newcommand{\bP}{\mathbb{P}}        %%%polynomials
\newcommand{\bR}{\mathbb{R}}        %%%real numbers
\newcommand{\bZ}{\mathbb{Z}}        %%%integer numbers
\newcommand{\cB}{\mathcal{B}}       %%%basis
\newcommand{\cC}{\mathcal{C}}       %%%basis
\newcommand{\cM}{\mathcal{M}}       %%%matrix family

\newcommand{\sT}{\mathsf{T}}        %%%traspuesta

\renewcommand{\geq}{\geqslant}      %%%(to save typing)
\renewcommand{\leq}{\leqslant}      %%%(to save typing)
\newcommand{\x}{\times}             %%%product
\renewcommand{\:}{\colon}           %%%colon in  f: A -> B
\newcommand{\isom}{\simeq}              %% isomorfismo

\newcommand{\un}[1]{\underline{#1}}
\newcommand{\half}{\frac12}

\newcommand*{\Cdot}{{\raisebox{-0.25ex}{\scalebox{1.5}{$\cdot$}}}}      %% cdot más grande
\renewcommand{\.}{\Cdot}                %% producto escalar

\newcommand{\twobyone}[2]{\begin{pmatrix} %% 2 x 1 matrix
  #1 \\ #2 \end{pmatrix}}
  \newcommand{\twobytwo}[4]{\begin{pmatrix} %% 2 x 2 matrix
    #1 & #2 \\ #3 & #4 \end{pmatrix}}
    \newcommand{\twobythree}[6]{\begin{pmatrix} %% 2 x 3 matrix
        #1 & #2 & #3\\ #4 & #5 & #6 \end{pmatrix}}
\newcommand{\threebyone}[3]{\begin{pmatrix} %% 3 x 1 matrix
  #1 \\ #2 \\ #3 \end{pmatrix}}
  \newcommand{\threebytwo}[6]{\begin{pmatrix} %% 3 x 1 matrix
    #1 & #2\\ #3 & #4\\ #5&#6 \end{pmatrix}}
\newcommand{\threebythree}[9]{\begin{pmatrix} %% 3 x 3 matrix
  #1 & #2 & #3 \\ #4 & #5 & #6 \\ #7 & #8 & #9 \end{pmatrix}}

\newcommand{\To}{\Rightarrow}

\newcommand{\vaf}{\overrightarrow}

\newcommand{\set}[1]{\{\,#1\,\}}    %% set notation
\newcommand{\Set}[1]{\biggl\{\,#1\,\biggr\}} %% set notation (large)
\newcommand{\red}[1]{\textcolor{red}{#1}}
\newcommand{\blu}[1]{\textcolor{blue}{#1}}

%----------------------------------------------------------------------------------------
%	ARTICLE CONTENTS
%----------------------------------------------------------------------------------------

\begin{document}
%\begin{multicols}{2}
Consider the following figures:
\begin{center}
  

\tikzset{every picture/.style={line width=0.75pt}} %set default line width to 0.75pt        

\begin{tikzpicture}[x=0.75pt,y=0.75pt,yscale=-1,xscale=1]
%uncomment if require: \path (0,300); %set diagram left start at 0, and has height of 300

%Straight Lines [id:da8082760698615621] 
\draw    (121,159.13) -- (121,39.13) ;
%Straight Lines [id:da42129239237432015] 
\draw    (161,119.13) -- (121,159.13) ;
%Straight Lines [id:da3084065729366139] 
\draw    (161,119.13) -- (201,159.13) ;
%Straight Lines [id:da2627146407642287] 
\draw    (121,39.13) -- (161,79.13) ;
%Straight Lines [id:da0739214290598118] 
\draw    (161,79.13) -- (201,39.13) ;
%Straight Lines [id:da972422384889911] 
\draw    (201,159.13) -- (201,39.13) ;
%Straight Lines [id:da23221479469104467] 
\draw    (250,80) -- (310,80) ;
%Straight Lines [id:da9575616549320587] 
\draw    (290,50) -- (350,50) ;
%Straight Lines [id:da44144192577022445] 
\draw    (250,80) -- (280,140) ;
%Straight Lines [id:da43153251742280596] 
\draw    (280,140) -- (310,80) ;
%Straight Lines [id:da9913243246187551] 
\draw    (320,110) -- (350,50) ;
%Straight Lines [id:da8790763336670147] 
\draw    (280,140) -- (320,110) ;
%Straight Lines [id:da1513042072498827] 
\draw    (310,80) -- (350,50) ;
%Straight Lines [id:da5867802834824685] 
\draw    (250,80) -- (270,65) ;
%Straight Lines [id:da04407353104888179] 
\draw    (270,80) -- (270,50) ;
%Straight Lines [id:da897836626384255] 
\draw    (280,80) -- (280,50) ;
%Straight Lines [id:da013774347148459265] 
\draw    (280,60) -- (290,50) ;
%Shape: Ellipse [id:dp7908402920091921] 
\draw   (270,50) .. controls (270,48.62) and (272.24,47.5) .. (275,47.5) .. controls (277.76,47.5) and (280,48.62) .. (280,50) .. controls (280,51.38) and (277.76,52.5) .. (275,52.5) .. controls (272.24,52.5) and (270,51.38) .. (270,50) -- cycle ;
%Straight Lines [id:da3119766425142587] 
\draw  [dash pattern={on 0.84pt off 2.51pt}]  (280,80) -- (280,140) ;
%Curve Lines [id:da7822185673649471] 
\draw    (295,110) .. controls (309.67,80.17) and (321,108.83) .. (335,80) ;

% Text Node
\draw (117.6,99.13) node [anchor=south] [inner sep=0.75pt]  [rotate=-270]  {$x=0$};
% Text Node
\draw (134.6,47.13) node [anchor=south] [inner sep=0.75pt]  [font=\scriptsize,rotate=-45]  {$x+y=6$};
% Text Node
\draw (178.6,129.53) node [anchor=south] [inner sep=0.75pt]  [font=\scriptsize,rotate=-45]  {$x+y=4$};
% Text Node
\draw (187.4,45.53) node [anchor=south] [inner sep=0.75pt]  [font=\scriptsize,rotate=-315]  {$y-x=2$};
% Text Node
\draw (146.11,127.84) node [anchor=south] [inner sep=0.75pt]  [font=\scriptsize,rotate=-315]  {$y=x$};
% Text Node
\draw (216.8,97.53) node [anchor=south] [inner sep=0.75pt]  [rotate=-270]  {$x=4$};
% Text Node
\draw (282,57.4) node [anchor=north west][inner sep=0.75pt]  [font=\scriptsize]  {$2\ \text{m}$};
% Text Node
\draw (282,90) node [anchor=west] [inner sep=0.75pt]  [font=\scriptsize]  {$3\ \text{m}$};
% Text Node
\draw (327,43) node [anchor=west] [inner sep=0.75pt]  [font=\scriptsize]  {$3\ \text{m}$};
% Text Node
\draw (302,128.4) node [anchor=north west][inner sep=0.75pt]  [font=\scriptsize]  {$8\ \text{m}$};
% Text Node
\draw (298,100.33) node [anchor=north west][inner sep=0.75pt]  [font=\scriptsize] [align=left] {Half};


\end{tikzpicture}

\end{center}
\begin{Ej}
 The figure on the left describes a cross-section of a solid of revolution bounded by the curves 
 $$\set{y=x,x+y=6},\ x\in[0,2],\ \text{and }\set{x+y=4,y-x=2},\ x\in[2,4].$$
 The density for such a cross section is given by the equation $\rho(y)=1-3y$. Express the mass of the solid of revolution obtained after rotating about the axis $y=8$ as a sum of integrals.\par $\lbrack \scriptsize\texttt{You may use any method, in any case your answer will involve more than one integral.}\rbrack$
\end{Ej}
\begin{ptcb}
\vspace*{7cm} 
\end{ptcb}

\begin{Ej}
  The figure on the right describes a tank filled up to half of the total height with water $(\rho=1000\ \si{\kilo\gram\per\metre\cubed})$. Find the work required to pump out the water from the tank.
\end{Ej}

\begin{ptcb}
  \vspace*{7cm} 
\end{ptcb}

%\end{multicols}
\end{document} 