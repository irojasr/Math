%----------------------------------------------------------------------------------------
%	PACKAGES AND OTHER DOCUMENT CONFIGURATIONS
%----------------------------------------------------------------------------------------

\documentclass[12pt]{article}
\usepackage[spanish]{babel} %Tildes
\usepackage[extreme]{savetrees} %Espaciado e interlineado. Comentar si no gusta el interlineado.
\usepackage[utf8]{inputenc} %Encoding para tildes
\usepackage[breakable,skins]{tcolorbox} %Cajitas
\usepackage{fancyhdr} % Se necesita para el título arriba
\usepackage{lastpage} % Se necesita para poner el número de página
\usepackage{amsmath,amsfonts,amssymb,amsthm} %simbolos y demás
\usepackage{mathabx} %más símbolos
\usepackage{physics} %simbolos de derivadas, bra-ket.
\usepackage{multicol}
\usepackage[customcolors]{hf-tikz}
\usepackage[shortlabels]{enumitem}
\usepackage{tikz}
\usetikzlibrary{patterns}
\usepackage{siunitx}

%\def\darktheme
%%%%%%%%% === Document Configuration === %%%%%%%%%%%%%%

\pagestyle{fancy}
\setlength{\headheight}{14.49998pt} %NO MODIFICAR
\setlength{\footskip}{14.49998pt} %NO MODIFICAR

\ifx \darktheme\undefined

\lhead{Math161S1} % Nombre de autor
\chead{\textbf{Quiz 6 - Solutions}} % Titulo
\rhead{Name:Solutions}%\firstxmark} 
\lfoot{}%\lastxmark}
\cfoot{}
\rfoot{Page \thepage\ of\ \pageref{LastPage}} %A la derecha saldrá pág. 6 de 9. 
\else
\pagenumbering{gobble}
\pagecolor[rgb]{0,0,0}%{0.23,0.258,0.321}
\color[rgb]{1,1,1}
\fi

%%%%%%%%% === My T Color Box === %%%%%%%%%%%%%%

\ifx \darktheme\undefined
\newtcolorbox{ptcb}{
colframe = black,
colback = white,
breakable,
enhanced
}
\newtcolorbox{ptcbP}{
colframe = black,
colback = white,
coltitle = black,
colbacktitle = black!40,
title = Practice,
breakable,
enhanced
}

\else
\newtcolorbox{ptcb}{
colframe = white,
colback = black,
colupper = white,
breakable,
enhanced
}
\newtcolorbox{ptcbP}{
colframe = white,
colback = black,
colupper = white,
coltitle = white,
colbacktitle = black,
title = Practice,
breakable,
enhanced
}
\fi

%%%%%%%%% === Tikz para matrices === %%%%%%%%%%%%%%

\tikzset{
  style green/.style={
    set fill color=green!50!lime!60,
    set border color=white,
  },
  style cyan/.style={
    set fill color=cyan!90!blue!60,
    set border color=white,
  },
  style orange/.style={
    set fill color=orange!80!red!60,
    set border color=white,
  },
  row/.style={
    above left offset={-0.15,0.31},
    below right offset={0.15,-0.125},
    #1
  },
  col/.style={
    above left offset={-0.1,0.3},
    below right offset={0.15,-0.15},
    #1
  }
}

%%%%%%%%% === Theorems and suchlike === %%%%%%%%%%%%%%

\theoremstyle{plain}
\newtheorem{Th}{Theorem}  %%% Theorem 1.1
\newtheorem*{nTh}{Theorem}             %%% No-numbered Theorem
\newtheorem{Prop}[Th]{Proposition}     %%% Proposition 1.2
\newtheorem{Lem}[Th]{Lemma}             %%% Lemma 1.3
\newtheorem*{nLem}{Lemma}               %%% No-numbered Lemma
\newtheorem{Cor}[Th]{Corollary}        %%% Corollary 1.4
\newtheorem*{nCor}{Corollary}          %%% No-numbered Corollary

\theoremstyle{definition}
\newtheorem*{Def}{Definition}       %%% Definition 1.5
\newtheorem*{nonum-Def}{Definition}    %%% No number Definition
\newtheorem*{nEx}{Example}             %%% No number Example
\newtheorem{Ex}[Th]{Example}           %%% Example
\newtheorem{Ej}[Th]{Exercise}         %%% Exercise
\newtheorem*{nEj}{Exercise}           %%% No number Excercise
\newtheorem*{Not}{Notation}       %%% Definition 1.5

\theoremstyle{remark}
\newtheorem*{Rmk}{Remark}      %%%Remark 1.6

%\numberwithin{equation}{section}

\setlength{\parindent}{3ex}

%%====== Useful macros: =======%%%

\DeclareMathOperator{\gen}{gen}     %%%set generated by...
\DeclareMathOperator{\Rng}{Rng}     %%%rangomat
\DeclareMathOperator{\Nul}{Nul}     %%%rangomat
\DeclareMathOperator{\Proy}{Proy}   %%%proyección
\DeclareMathOperator{\id}{id}       %%%identity operator

\newcommand{\al}{\alpha}            %%%short for \alpha
\newcommand{\la}{\lambda}           %%%short for \lambda
\newcommand{\sg}{\sigma}            %%%short for \sigma
\newcommand{\te}{\theta}                %% short for  \theta
\renewcommand{\l}{\ell}

\newcommand{\thickhat}[1]{\mathbf{\hat{\text{$#1$}}}}
\newcommand{\ii}{\vu{\imath}}
\newcommand{\jj}{\vu{\jmath}}
\newcommand{\kk}{\thickhat{k}}

\newcommand{\bC}{\mathbb{C}}        %%%complex numbers
\newcommand{\bN}{\mathbb{N}}        %%%natural numbers
\newcommand{\bP}{\mathbb{P}}        %%%polynomials
\newcommand{\bR}{\mathbb{R}}        %%%real numbers
\newcommand{\bZ}{\mathbb{Z}}        %%%integer numbers
\newcommand{\cB}{\mathcal{B}}       %%%basis
\newcommand{\cC}{\mathcal{C}}       %%%basis
\newcommand{\cM}{\mathcal{M}}       %%%matrix family

\newcommand{\sT}{\mathsf{T}}        %%%traspuesta

\renewcommand{\geq}{\geqslant}      %%%(to save typing)
\renewcommand{\leq}{\leqslant}      %%%(to save typing)
\newcommand{\x}{\times}             %%%product
\renewcommand{\:}{\colon}           %%%colon in  f: A -> B
\newcommand{\isom}{\simeq}              %% isomorfismo

\newcommand{\un}[1]{\underline{#1}}
\newcommand{\half}{\frac12}

\newcommand*{\Cdot}{{\raisebox{-0.25ex}{\scalebox{1.5}{$\cdot$}}}}      %% cdot más grande
\renewcommand{\.}{\Cdot}                %% producto escalar

\newcommand{\twobyone}[2]{\begin{pmatrix} %% 2 x 1 matrix
  #1 \\ #2 \end{pmatrix}}
  \newcommand{\twobytwo}[4]{\begin{pmatrix} %% 2 x 2 matrix
    #1 & #2 \\ #3 & #4 \end{pmatrix}}
    \newcommand{\twobythree}[6]{\begin{pmatrix} %% 2 x 3 matrix
        #1 & #2 & #3\\ #4 & #5 & #6 \end{pmatrix}}
\newcommand{\threebyone}[3]{\begin{pmatrix} %% 3 x 1 matrix
  #1 \\ #2 \\ #3 \end{pmatrix}}
  \newcommand{\threebytwo}[6]{\begin{pmatrix} %% 3 x 1 matrix
    #1 & #2\\ #3 & #4\\ #5&#6 \end{pmatrix}}
\newcommand{\threebythree}[9]{\begin{pmatrix} %% 3 x 3 matrix
  #1 & #2 & #3 \\ #4 & #5 & #6 \\ #7 & #8 & #9 \end{pmatrix}}

\newcommand{\To}{\Rightarrow}

\newcommand{\vaf}{\overrightarrow}

\newcommand{\set}[1]{\{\,#1\,\}}    %% set notation
\newcommand{\Set}[1]{\biggl\{\,#1\,\biggr\}} %% set notation (large)
\newcommand{\red}[1]{\textcolor{red}{#1}}
\newcommand{\blu}[1]{\textcolor{blue}{#1}}

%----------------------------------------------------------------------------------------
%	ARTICLE CONTENTS
%----------------------------------------------------------------------------------------

\begin{document}
%\begin{multicols}{2}

\begin{Ej}
  Suppose we have a triangular plate at the bottom of a pool. It is enclosed by the following curves:
  $$\set{y=x,\ y=2-x,\ y=0}.$$
  Suppose that the pool contains a fluid of density $\rho=1$ and the full depth of the pool is $10\ \si\metre$. Do the following:
  \begin{enumerate}[i)]
    \itemsep=-0.4em 
    \item Make a diagram which illustrates the situation described above. Highlight the enclosed region formed by the curves to show that that is the plate.
    \item Express the pressure as an integral. $\lbrack \scriptsize\texttt{Remember: Pressure is the integral of depth times width times the density.}\rbrack$
  \end{enumerate}
\end{Ej}
\begin{ptcb}
  \begin{center}
  

% Pattern Info
 
\tikzset{
  pattern size/.store in=\mcSize, 
  pattern size = 5pt,
  pattern thickness/.store in=\mcThickness, 
  pattern thickness = 0.3pt,
  pattern radius/.store in=\mcRadius, 
  pattern radius = 1pt}\makeatletter
  \pgfutil@ifundefined{pgf@pattern@name@_prup2w0so}{
  \pgfdeclarepatternformonly[\mcThickness,\mcSize]{_prup2w0so}
  {\pgfqpoint{-\mcThickness}{-\mcThickness}}
  {\pgfpoint{\mcSize}{\mcSize}}
  {\pgfpoint{\mcSize}{\mcSize}}
  {\pgfsetcolor{\tikz@pattern@color}
  \pgfsetlinewidth{\mcThickness}
  \pgfpathmoveto{\pgfpointorigin}
  \pgfpathlineto{\pgfpoint{\mcSize}{0}}
  \pgfpathmoveto{\pgfpointorigin}
  \pgfpathlineto{\pgfpoint{0}{\mcSize}}
  \pgfusepath{stroke}}}
  \makeatother
  \tikzset{every picture/.style={line width=0.75pt}} %set default line width to 0.75pt        
  
  \begin{tikzpicture}[x=0.75pt,y=0.75pt,yscale=-1,xscale=1]
  %uncomment if require: \path (0,300); %set diagram left start at 0, and has height of 300
  
  %Straight Lines [id:da44914162484239306] 
  \draw    (200,80) -- (200,160) ;
  %Straight Lines [id:da6844861001335495] 
  \draw    (200,160) -- (260,160) ;
  %Straight Lines [id:da8636784311251287] 
  \draw    (230,130) -- (200,160) ;
  %Straight Lines [id:da36342764321969756] 
  \draw    (230,130) -- (260,160) ;
  %Straight Lines [id:da3388825175894056] 
  \draw  [dash pattern={on 4.5pt off 4.5pt}]  (270,91) -- (230,130) ;
  %Straight Lines [id:da08376432369748443] 
  \draw  [dash pattern={on 4.5pt off 4.5pt}]  (260,160) -- (270,170) ;
  %Straight Lines [id:da32540555588554454] 
  \draw  [dash pattern={on 4.5pt off 4.5pt}]  (200,100) -- (230,130) ;
  %Shape: Triangle [id:dp502346937894611] 
  \draw  [pattern=_prup2w0so,pattern size=4.800000000000001pt,pattern thickness=0.75pt,pattern radius=0pt, pattern color={rgb, 255:red, 0; green, 0; blue, 0}] (230,130) -- (260,160) -- (200,160) -- cycle ;
  %Straight Lines [id:da13440250063976256] 
  \draw    (200,160) -- (200,165) ;
  %Straight Lines [id:da5727208732536868] 
  \draw    (260,160) -- (260,165) ;
  %Straight Lines [id:da3360882336521609] 
  \draw    (200,80) -- (200,60) ;
  %Shape: Wave [id:dp20509465106678593] 
  \draw  [color={rgb, 255:red, 0; green, 93; blue, 164 }  ,draw opacity=1 ] (200,62.5) .. controls (201.22,63.78) and (202.39,65) .. (203.75,65) .. controls (205.11,65) and (206.28,63.78) .. (207.5,62.5) .. controls (208.72,61.22) and (209.89,60) .. (211.25,60) .. controls (212.61,60) and (213.78,61.22) .. (215,62.5) .. controls (216.22,63.78) and (217.39,65) .. (218.75,65) .. controls (220.11,65) and (221.28,63.78) .. (222.5,62.5) .. controls (223.72,61.22) and (224.89,60) .. (226.25,60) .. controls (227.61,60) and (228.78,61.22) .. (230,62.5) .. controls (231.22,63.78) and (232.39,65) .. (233.75,65) .. controls (235.11,65) and (236.28,63.78) .. (237.5,62.5) .. controls (238.72,61.22) and (239.89,60) .. (241.25,60) .. controls (242.61,60) and (243.78,61.22) .. (245,62.5) .. controls (246.22,63.78) and (247.39,65) .. (248.75,65) .. controls (250.11,65) and (251.28,63.78) .. (252.5,62.5) .. controls (253.72,61.22) and (254.89,60) .. (256.25,60) .. controls (257.61,60) and (258.78,61.22) .. (260,62.5) .. controls (261.22,63.78) and (262.39,65) .. (263.75,65) .. controls (264.55,65) and (265.28,64.58) .. (266,63.97) ;
  %Straight Lines [id:da6096899471092039] 
  \draw    (200,60) -- (180,60.17) ;
  %Straight Lines [id:da5628101882615844] 
  \draw    (200,160) -- (180,160.17) ;
  %Straight Lines [id:da8965781839932772] 
  \draw  [dash pattern={on 4.5pt off 4.5pt}]  (190,60.08) -- (190,100) ;
  %Straight Lines [id:da6286332687541674] 
  \draw  [dash pattern={on 4.5pt off 4.5pt}]  (190,120.17) -- (190,160.08) ;
  
  % Text Node
  \draw (264.98,87.61) node [anchor=south east] [inner sep=0.75pt]  [font=\tiny,rotate=-315]  {$y=x$};
  % Text Node
  \draw (263.39,156.58) node [anchor=south east] [inner sep=0.75pt]  [font=\tiny,rotate=-45]  {$y=2-x$};
  % Text Node
  \draw (187,168.4) node [anchor=north west][inner sep=0.75pt]  [font=\tiny]  {$x=0$};
  % Text Node
  \draw (270,168.4) node [anchor=north east] [inner sep=0.75pt]  [font=\tiny]  {$x=2$};
  % Text Node
  \draw (177,108.4) node [anchor=north west][inner sep=0.75pt]  [font=\tiny]  {$10\text{ m}$};
  
  
  \end{tikzpicture}
  
  \end{center}
Notice that the curve $y=2-x$ is to the right of $y=x$. We need to express these functions in terms of $y$, so the curves are
$$y=2-x\To x=2-y,\ \text{and}\ y=x\To x=y$$
In this case $R-L=(2-y)-(y)=2-2y$. At height $y$ the depth $D$ is found by the equation
$$y+D=\text{total depth}\To y+D=10\To D=10-y.$$
The plate's max height occurs at the point where $2-x=x$ which is $x=1$. Plugging this in either curve's equation we get $y=1$. We conclude that the pressure is 
$$\int_0^1(10-y)(2-2y)(1)\dd y=29/3.$$
\end{ptcb}

\begin{Ej}
  Consider the region in the $1^{\text{st}}$ quadrant enclosed by the curves
  $$\set{x=1,\ x=2,\ y=x,\ y=x+2}.$$
  If we rotate the region about the axis $y=-1$ we obtain a solid of revolution.
  \begin{enumerate}[i)]
    \itemsep=-0.4em
    \item Sketch the region in question and make a rough sketch of how the solid of revolution looks like.
    \item Use the method of rings to find the volume of the solid.
  \end{enumerate}
\end{Ej}

\begin{ptcb}
\begin{center}
  

% Pattern Info
 
\tikzset{
  pattern size/.store in=\mcSize, 
  pattern size = 5pt,
  pattern thickness/.store in=\mcThickness, 
  pattern thickness = 0.3pt,
  pattern radius/.store in=\mcRadius, 
  pattern radius = 1pt}\makeatletter
  \pgfutil@ifundefined{pgf@pattern@name@_1k86imbef}{
  \pgfdeclarepatternformonly[\mcThickness,\mcSize]{_1k86imbef}
  {\pgfqpoint{-\mcThickness}{-\mcThickness}}
  {\pgfpoint{\mcSize}{\mcSize}}
  {\pgfpoint{\mcSize}{\mcSize}}
  {\pgfsetcolor{\tikz@pattern@color}
  \pgfsetlinewidth{\mcThickness}
  \pgfpathmoveto{\pgfpointorigin}
  \pgfpathlineto{\pgfpoint{\mcSize}{0}}
  \pgfpathmoveto{\pgfpointorigin}
  \pgfpathlineto{\pgfpoint{0}{\mcSize}}
  \pgfusepath{stroke}}}
  \makeatother
  \tikzset{every picture/.style={line width=0.75pt}} %set default line width to 0.75pt        
  
  \begin{tikzpicture}[x=0.75pt,y=0.75pt,yscale=-1,xscale=1]
  %uncomment if require: \path (0,300); %set diagram left start at 0, and has height of 300
  
  %Straight Lines [id:da7582877899405727] 
  \draw    (100,200) -- (100,100) ;
  %Straight Lines [id:da47942992210062685] 
  \draw    (160,200) -- (100,200) ;
  %Straight Lines [id:da9507520071256941] 
  \draw  [dash pattern={on 4.5pt off 4.5pt}]  (100,200) -- (160,140) ;
  %Straight Lines [id:da6480916102746426] 
  \draw  [dash pattern={on 4.5pt off 4.5pt}]  (100,160.25) -- (160,100) ;
  %Straight Lines [id:da4842649005345028] 
  \draw  [dash pattern={on 4.5pt off 4.5pt}]  (120,200) -- (120,100) ;
  %Straight Lines [id:da24789671258792412] 
  \draw  [dash pattern={on 4.5pt off 4.5pt}]  (140,200) -- (140,100) ;
  %Shape: Parallelogram [id:dp9392314243029221] 
  \draw  [pattern=_1k86imbef,pattern size=4.5pt,pattern thickness=0.75pt,pattern radius=0pt, pattern color={rgb, 255:red, 0; green, 0; blue, 0}] (140,120) -- (140,160.19) -- (120,180) -- (120,139.81) -- cycle ;
  %Shape: Parallelogram [id:dp43536009025700584] 
  \draw   (247,80) -- (247,120.19) -- (227,140) -- (227,99.81) -- cycle ;
  %Shape: Parallelogram [id:dp6454853241396137] 
  \draw   (247,199.81) -- (247,240) -- (227,220.19) -- (227,180) -- cycle ;
  %Straight Lines [id:da9309576327497984] 
  \draw  [dash pattern={on 0.84pt off 2.51pt}]  (247,160) -- (227,160) ;
  %Shape: Ellipse [id:dp8462443870077831] 
  \draw   (242,160) .. controls (242,138.01) and (244.24,120.19) .. (247,120.19) .. controls (249.76,120.19) and (252,138.01) .. (252,160) .. controls (252,181.99) and (249.76,199.81) .. (247,199.81) .. controls (244.24,199.81) and (242,181.99) .. (242,160) -- cycle ;
  %Shape: Ellipse [id:dp6714520665832591] 
  \draw   (235.75,160) .. controls (235.75,115.82) and (240.79,80) .. (247,80) .. controls (253.21,80) and (258.25,115.82) .. (258.25,160) .. controls (258.25,204.18) and (253.21,240) .. (247,240) .. controls (240.79,240) and (235.75,204.18) .. (235.75,160) -- cycle ;
  %Curve Lines [id:da25845073404391594] 
  \draw    (227,99.81) .. controls (210.25,119.88) and (210,200.38) .. (227,220.19) ;
  %Curve Lines [id:da49155696905062096] 
  \draw    (227,140) .. controls (222.5,142.88) and (223,177.38) .. (227,180) ;
  
  % Text Node
  \draw (160.99,143.82) node [anchor=north east] [inner sep=0.75pt]  [font=\tiny,rotate=-315]  {$x$};
  % Text Node
  \draw (160.99,103.82) node [anchor=north east] [inner sep=0.75pt]  [font=\tiny,rotate=-315]  {$x+2$};
  % Text Node
  \draw (120,203.4) node [anchor=north] [inner sep=0.75pt]  [font=\tiny]  {$1$};
  % Text Node
  \draw (140,203.4) node [anchor=north] [inner sep=0.75pt]  [font=\tiny]  {$2$};
  % Text Node
  \draw (180,157.4) node [anchor=north west][inner sep=0.75pt]  [font=\tiny]  {$y=-1$};
  
  
  \end{tikzpicture}
  
\end{center}
In this case, we don't need to find the intersections of the curves. The bounds of integration are given to us to be \un{$x=1$ and $x=2$}. We construct the radii:
$$R=U-\text{axis}=(x+2)-(-1)=x+3,\ r=D-\text{axis}=x-(-1)=x+1.$$
The area function is $A(x)=\pi[(x+3)^2+(x+1)^2]=\pi(2 x^2 + 8 x + 10)$ and so the volume is 
$$V=\int_1^2\pi(2 x^2 + 8 x + 10)\dd x=80\pi/3.$$
\end{ptcb}

%\end{multicols}
\end{document} 