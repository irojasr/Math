%----------------------------------------------------------------------------------------
%	PACKAGES AND OTHER DOCUMENT CONFIGURATIONS
%----------------------------------------------------------------------------------------

\documentclass[12pt]{article}
\usepackage[spanish]{babel} %Tildes
\usepackage[extreme]{savetrees} %Espaciado e interlineado. Comentar si no gusta el interlineado.
\usepackage[utf8]{inputenc} %Encoding para tildes
\usepackage[breakable,skins]{tcolorbox} %Cajitas
\usepackage{fancyhdr} % Se necesita para el título arriba
\usepackage{lastpage} % Se necesita para poner el número de página
\usepackage{amsmath,amsfonts,amssymb,amsthm} %simbolos y demás
\usepackage{mathabx} %más símbolos
\usepackage{physics} %simbolos de derivadas, bra-ket.
\usepackage{multicol}
\usepackage[customcolors]{hf-tikz}
\usepackage[shortlabels]{enumitem}
\usepackage{tikz}
\usetikzlibrary{patterns}
\usepackage{siunitx}

%\def\darktheme
%%%%%%%%% === Document Configuration === %%%%%%%%%%%%%%

\pagestyle{fancy}
\setlength{\headheight}{14.49998pt} %NO MODIFICAR
\setlength{\footskip}{14.49998pt} %NO MODIFICAR

\ifx \darktheme\undefined

\lhead{Math161S1} % Nombre de autor
\chead{\textbf{Quiz 4 - Solutions}} % Titulo
\rhead{Name:Solutions}%\firstxmark} 
\lfoot{}%\lastxmark}
\cfoot{}
\rfoot{Page \thepage\ of\ \pageref{LastPage}} %A la derecha saldrá pág. 6 de 9. 
\else
\pagenumbering{gobble}
\pagecolor[rgb]{0,0,0}%{0.23,0.258,0.321}
\color[rgb]{1,1,1}
\fi

%%%%%%%%% === My T Color Box === %%%%%%%%%%%%%%

\ifx \darktheme\undefined
\newtcolorbox{ptcb}{
colframe = black,
colback = white,
breakable,
enhanced
}
\newtcolorbox{ptcbP}{
colframe = black,
colback = white,
coltitle = black,
colbacktitle = black!40,
title = Practice,
breakable,
enhanced
}

\else
\newtcolorbox{ptcb}{
colframe = white,
colback = black,
colupper = white,
breakable,
enhanced
}
\newtcolorbox{ptcbP}{
colframe = white,
colback = black,
colupper = white,
coltitle = white,
colbacktitle = black,
title = Practice,
breakable,
enhanced
}
\fi

%%%%%%%%% === Tikz para matrices === %%%%%%%%%%%%%%

\tikzset{
  style green/.style={
    set fill color=green!50!lime!60,
    set border color=white,
  },
  style cyan/.style={
    set fill color=cyan!90!blue!60,
    set border color=white,
  },
  style orange/.style={
    set fill color=orange!80!red!60,
    set border color=white,
  },
  row/.style={
    above left offset={-0.15,0.31},
    below right offset={0.15,-0.125},
    #1
  },
  col/.style={
    above left offset={-0.1,0.3},
    below right offset={0.15,-0.15},
    #1
  }
}

%%%%%%%%% === Theorems and suchlike === %%%%%%%%%%%%%%

\theoremstyle{plain}
\newtheorem{Th}{Theorem}  %%% Theorem 1.1
\newtheorem*{nTh}{Theorem}             %%% No-numbered Theorem
\newtheorem{Prop}[Th]{Proposition}     %%% Proposition 1.2
\newtheorem{Lem}[Th]{Lemma}             %%% Lemma 1.3
\newtheorem*{nLem}{Lemma}               %%% No-numbered Lemma
\newtheorem{Cor}[Th]{Corollary}        %%% Corollary 1.4
\newtheorem*{nCor}{Corollary}          %%% No-numbered Corollary

\theoremstyle{definition}
\newtheorem*{Def}{Definition}       %%% Definition 1.5
\newtheorem*{nonum-Def}{Definition}    %%% No number Definition
\newtheorem*{nEx}{Example}             %%% No number Example
\newtheorem{Ex}[Th]{Example}           %%% Example
\newtheorem{Ej}[Th]{Exercise}         %%% Exercise
\newtheorem*{nEj}{Exercise}           %%% No number Excercise
\newtheorem*{Not}{Notation}       %%% Definition 1.5

\theoremstyle{remark}
\newtheorem*{Rmk}{Remark}      %%%Remark 1.6

%\numberwithin{equation}{section}

\setlength{\parindent}{3ex}

%%====== Useful macros: =======%%%

\DeclareMathOperator{\gen}{gen}     %%%set generated by...
\DeclareMathOperator{\Rng}{Rng}     %%%rangomat
\DeclareMathOperator{\Nul}{Nul}     %%%rangomat
\DeclareMathOperator{\Proy}{Proy}   %%%proyección
\DeclareMathOperator{\id}{id}       %%%identity operator

\newcommand{\al}{\alpha}            %%%short for \alpha
\newcommand{\la}{\lambda}           %%%short for \lambda
\newcommand{\sg}{\sigma}            %%%short for \sigma
\newcommand{\te}{\theta}                %% short for  \theta
\renewcommand{\l}{\ell}

\newcommand{\thickhat}[1]{\mathbf{\hat{\text{$#1$}}}}
\newcommand{\ii}{\vu{\imath}}
\newcommand{\jj}{\vu{\jmath}}
\newcommand{\kk}{\thickhat{k}}

\newcommand{\bC}{\mathbb{C}}        %%%complex numbers
\newcommand{\bN}{\mathbb{N}}        %%%natural numbers
\newcommand{\bP}{\mathbb{P}}        %%%polynomials
\newcommand{\bR}{\mathbb{R}}        %%%real numbers
\newcommand{\bZ}{\mathbb{Z}}        %%%integer numbers
\newcommand{\cB}{\mathcal{B}}       %%%basis
\newcommand{\cC}{\mathcal{C}}       %%%basis
\newcommand{\cM}{\mathcal{M}}       %%%matrix family

\newcommand{\sT}{\mathsf{T}}        %%%traspuesta

\renewcommand{\geq}{\geqslant}      %%%(to save typing)
\renewcommand{\leq}{\leqslant}      %%%(to save typing)
\newcommand{\x}{\times}             %%%product
\renewcommand{\:}{\colon}           %%%colon in  f: A -> B
\newcommand{\isom}{\simeq}              %% isomorfismo

\newcommand{\un}[1]{\underline{#1}}
\newcommand{\half}{\frac12}

\newcommand*{\Cdot}{{\raisebox{-0.25ex}{\scalebox{1.5}{$\cdot$}}}}      %% cdot más grande
\renewcommand{\.}{\Cdot}                %% producto escalar

\newcommand{\twobyone}[2]{\begin{pmatrix} %% 2 x 1 matrix
  #1 \\ #2 \end{pmatrix}}
  \newcommand{\twobytwo}[4]{\begin{pmatrix} %% 2 x 2 matrix
    #1 & #2 \\ #3 & #4 \end{pmatrix}}
    \newcommand{\twobythree}[6]{\begin{pmatrix} %% 2 x 3 matrix
        #1 & #2 & #3\\ #4 & #5 & #6 \end{pmatrix}}
\newcommand{\threebyone}[3]{\begin{pmatrix} %% 3 x 1 matrix
  #1 \\ #2 \\ #3 \end{pmatrix}}
  \newcommand{\threebytwo}[6]{\begin{pmatrix} %% 3 x 1 matrix
    #1 & #2\\ #3 & #4\\ #5&#6 \end{pmatrix}}
\newcommand{\threebythree}[9]{\begin{pmatrix} %% 3 x 3 matrix
  #1 & #2 & #3 \\ #4 & #5 & #6 \\ #7 & #8 & #9 \end{pmatrix}}

\newcommand{\To}{\Rightarrow}

\newcommand{\vaf}{\overrightarrow}

\newcommand{\set}[1]{\{\,#1\,\}}    %% set notation
\newcommand{\Set}[1]{\biggl\{\,#1\,\biggr\}} %% set notation (large)
\newcommand{\red}[1]{\textcolor{red}{#1}}
\newcommand{\blu}[1]{\textcolor{blue}{#1}}

%----------------------------------------------------------------------------------------
%	ARTICLE CONTENTS
%----------------------------------------------------------------------------------------

\begin{document}
%\begin{multicols}{2}

\begin{Ej}
  Suppose we are lifting an empty open box with a \emph{weightless rope} up a building and after some time it starts raining. We will calculate the work necessary to pull the box up given the following conditions:
  $$\left\lbrace
  \begin{aligned}
    &\text{Weight of box}=w_B=16\ \si\newton,\\
    &\text{Capacity of box}=\text{cap}_B=10\ \si\newton,\\
    &\text{Length of rope}=\l_R=15\ \si\metre.
  \end{aligned}
  \quad 
  \left\lbrace
\begin{aligned}
  &\text{Pulling velocity}=v_{\text{pull}}=1\ \si{\metre\per\second},\\
  &\text{Rain starts at }t_{\text{rain}}=5\ \si{\second},\\
  &\text{Rain speed}=v_{\text{rain}}=5\ \si{\newton\per\second}.
\end{aligned}
\right.\right.$$
  \begin{enumerate}[i)]
    \itemsep=-0.4em 
    \item Will the box be filled with water \textbf{before} reaching the top? If your answer is yes, at what height after beginning is the box full?
    \item Express the work required to pull as a sum of integrals. It is not necessary to solve them.
  \end{enumerate}
\end{Ej}
\begin{ptcb}
  \begin{enumerate}[i)]
    \itemsep=-0.4em 
    \item The box will be filled before reaching the top. Rain starts at $5\ \si\second$, at that moment the box is at height $5\ \si\metre$. Since the rain falls at $5\ \si{\newton\per\second}$, it takes $t_{\text{fill}}=(\text{cap}_B)/(v_{\text{rain}})=2\ \si\second$. Then the box will be filled at $7\ \si\metre$.
    \item We need to pull the box and the water, so the integral would be
    $$W=\underbrace{\int_0^{15} 16\dd y}_{\text{box}}+\underbrace{\int_5^{7}5y\dd y+\int_7^{15}10\dd y}_{\text{water}}
    $$
  \end{enumerate}
\end{ptcb}

\begin{Ej}
  Consider the region enclosed by the $x$-axis, and the lines $y=x$, $y=2-x$.
  \begin{enumerate}[i)]
    \itemsep=-0.4em
    \item Sketch the region in question and highlight the enclosed area.
    \item Suppose that the region defines a metal plate with density $\rho(x)=\sin(\pi x)$. Express the mass of the plate as an integral.
  \end{enumerate}
\end{Ej}

\begin{ptcb}
\begin{center}
  % Pattern Info

\tikzset{
  pattern size/.store in=\mcSize, 
  pattern size = 5pt,
  pattern thickness/.store in=\mcThickness, 
  pattern thickness = 0.3pt,
  pattern radius/.store in=\mcRadius, 
  pattern radius = 1pt}\makeatletter
  \pgfutil@ifundefined{pgf@pattern@name@_v2n2ski4c}{
  \pgfdeclarepatternformonly[\mcThickness,\mcSize]{_v2n2ski4c}
  {\pgfqpoint{-\mcThickness}{-\mcThickness}}
  {\pgfpoint{\mcSize}{\mcSize}}
  {\pgfpoint{\mcSize}{\mcSize}}
  {\pgfsetcolor{\tikz@pattern@color}
  \pgfsetlinewidth{\mcThickness}
  \pgfpathmoveto{\pgfpointorigin}
  \pgfpathlineto{\pgfpoint{\mcSize}{0}}
  \pgfpathmoveto{\pgfpointorigin}
  \pgfpathlineto{\pgfpoint{0}{\mcSize}}
  \pgfusepath{stroke}}}
  \makeatother
  \tikzset{every picture/.style={line width=0.75pt}} %set default line width to 0.75pt        
  
  \begin{tikzpicture}[x=0.75pt,y=0.75pt,yscale=-1,xscale=1]
  %uncomment if require: \path (0,300); %set diagram left start at 0, and has height of 300
  
  %Straight Lines [id:da44914162484239306] 
  \draw    (200,80) -- (200,160) ;
  %Straight Lines [id:da6844861001335495] 
  \draw    (200,160) -- (260,160) ;
  %Straight Lines [id:da8636784311251287] 
  \draw    (230,130) -- (200,160) ;
  %Straight Lines [id:da36342764321969756] 
  \draw    (230,130) -- (260,160) ;
  %Straight Lines [id:da3388825175894056] 
  \draw  [dash pattern={on 4.5pt off 4.5pt}]  (270,91) -- (230,130) ;
  %Straight Lines [id:da3545991837014174] 
  \draw  [dash pattern={on 4.5pt off 4.5pt}]  (200,160) -- (190,170) ;
  %Straight Lines [id:da08376432369748443] 
  \draw  [dash pattern={on 4.5pt off 4.5pt}]  (260,160) -- (270,170) ;
  %Straight Lines [id:da32540555588554454] 
  \draw  [dash pattern={on 4.5pt off 4.5pt}]  (190,90) -- (230,130) ;
  %Shape: Triangle [id:dp502346937894611] 
  \draw  [pattern=_v2n2ski4c,pattern size=4.800000000000001pt,pattern thickness=0.75pt,pattern radius=0pt, pattern color={rgb, 255:red, 0; green, 0; blue, 0}] (230,130) -- (260,160) -- (200,160) -- cycle ;
  %Straight Lines [id:da13440250063976256] 
  \draw    (200,160) -- (200,165) ;
  %Straight Lines [id:da5727208732536868] 
  \draw    (260,160) -- (260,165) ;
  
  % Text Node
  \draw (264.98,87.61) node [anchor=south east] [inner sep=0.75pt]  [font=\tiny,rotate=-315]  {$y=x$};
  % Text Node
  \draw (263.39,156.58) node [anchor=south east] [inner sep=0.75pt]  [font=\tiny,rotate=-45]  {$y=2-x$};
  % Text Node
  \draw (202,168.4) node [anchor=north west][inner sep=0.75pt]  [font=\tiny]  {$x=0$};
  % Text Node
  \draw (258,168.4) node [anchor=north east] [inner sep=0.75pt]  [font=\tiny]  {$x=2$};
  
  
  \end{tikzpicture}
\end{center}  
The mass of the plate is given by the integrals 
$$\int_{0}^1x\sin(\pi x)\dd x+\int_{1}^2(2-x)\sin(\pi x)\dd x.$$
Or with $y$ (right-minus-left) orientation
$$\int_0^1((2-y)-y)\left(\frac{1}{\pi}\arcsin(y)\right)\dd y.$$

\end{ptcb}

%\end{multicols}
\end{document} 