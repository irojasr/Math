%----------------------------------------------------------------------------------------
%	PACKAGES AND OTHER DOCUMENT CONFIGURATIONS
%----------------------------------------------------------------------------------------

\documentclass[12pt]{article}
%\usepackage[spanish]{babel} %Tildes
\usepackage[extreme]{savetrees} %Espaciado e interlineado. Comentar si no gusta el interlineado.
\usepackage[utf8]{inputenc} %Encoding para tildes
\usepackage[breakable,skins]{tcolorbox} %Cajitas
\usepackage{fancyhdr} % Se necesita para el título arriba
\usepackage{lastpage} % Se necesita para poner el número de página
\usepackage{amsmath,amsfonts,amssymb,amsthm} %simbolos y demás
\usepackage{mathabx} %más símbolos
\usepackage{physics} %simbolos de derivadas, bra-ket.
\usepackage{multicol}
\usepackage[customcolors]{hf-tikz}
\usepackage[shortlabels]{enumitem}
\usepackage{tikz}
\usetikzlibrary{patterns}
\usepackage{siunitx}

%\def\darktheme
%%%%%%%%% === Document Configuration === %%%%%%%%%%%%%%

\pagestyle{fancy}
\setlength{\headheight}{14.49998pt} %NO MODIFICAR
\setlength{\footskip}{14.49998pt} %NO MODIFICAR

\ifx \darktheme\undefined

\lhead{Math161S1} % Nombre de autor
\chead{\textbf{Quiz 10}} % Titulo
\rhead{Name:\hspace*{5cm}}%\firstxmark} 
\lfoot{}%\lastxmark}
\cfoot{}
\rfoot{Page \thepage\ of\ \pageref{LastPage}} %A la derecha saldrá pág. 6 de 9. 
\else
\pagenumbering{gobble}
\pagecolor[rgb]{0,0,0}%{0.23,0.258,0.321}
\color[rgb]{1,1,1}
\fi

%%%%%%%%% === My T Color Box === %%%%%%%%%%%%%%

\ifx \darktheme\undefined
\newtcolorbox{ptcb}{
colframe = black,
colback = white,
breakable,
enhanced
}
\newtcolorbox{ptcbP}{
colframe = black,
colback = white,
coltitle = black,
colbacktitle = black!40,
title = Practice,
breakable,
enhanced
}

\else
\newtcolorbox{ptcb}{
colframe = white,
colback = black,
colupper = white,
breakable,
enhanced
}
\newtcolorbox{ptcbP}{
colframe = white,
colback = black,
colupper = white,
coltitle = white,
colbacktitle = black,
title = Practice,
breakable,
enhanced
}
\fi

%%%%%%%%% === Tikz para matrices === %%%%%%%%%%%%%%

\tikzset{
  style green/.style={
    set fill color=green!50!lime!60,
    set border color=white,
  },
  style cyan/.style={
    set fill color=cyan!90!blue!60,
    set border color=white,
  },
  style orange/.style={
    set fill color=orange!80!red!60,
    set border color=white,
  },
  row/.style={
    above left offset={-0.15,0.31},
    below right offset={0.15,-0.125},
    #1
  },
  col/.style={
    above left offset={-0.1,0.3},
    below right offset={0.15,-0.15},
    #1
  }
}

%%%%%%%%% === Theorems and suchlike === %%%%%%%%%%%%%%

\theoremstyle{plain}
\newtheorem{Th}{Theorem}  %%% Theorem 1.1
\newtheorem*{nTh}{Theorem}             %%% No-numbered Theorem
\newtheorem{Prop}[Th]{Proposition}     %%% Proposition 1.2
\newtheorem{Lem}[Th]{Lemma}             %%% Lemma 1.3
\newtheorem*{nLem}{Lemma}               %%% No-numbered Lemma
\newtheorem{Cor}[Th]{Corollary}        %%% Corollary 1.4
\newtheorem*{nCor}{Corollary}          %%% No-numbered Corollary

\theoremstyle{definition}
\newtheorem*{Def}{Definition}       %%% Definition 1.5
\newtheorem*{nonum-Def}{Definition}    %%% No number Definition
\newtheorem*{nEx}{Example}             %%% No number Example
\newtheorem{Ex}[Th]{Example}           %%% Example
\newtheorem{Ej}[Th]{Exercise}         %%% Exercise
\newtheorem*{nEj}{Exercise}           %%% No number Excercise
\newtheorem*{Not}{Notation}       %%% Definition 1.5

\theoremstyle{remark}
\newtheorem*{Rmk}{Remark}      %%%Remark 1.6

%\numberwithin{equation}{section}

\setlength{\parindent}{3ex}

%%====== Useful macros: =======%%%

\DeclareMathOperator{\gen}{gen}     %%%set generated by...
\DeclareMathOperator{\Rng}{Rng}     %%%rangomat
\DeclareMathOperator{\Nul}{Nul}     %%%rangomat
\DeclareMathOperator{\Proy}{Proy}   %%%proyección
\DeclareMathOperator{\id}{id}       %%%identity operator

\newcommand{\al}{\alpha}            %%%short for \alpha
\newcommand{\la}{\lambda}           %%%short for \lambda
\newcommand{\sg}{\sigma}            %%%short for \sigma
\newcommand{\te}{\theta}                %% short for  \theta
\renewcommand{\l}{\ell}

\newcommand{\thickhat}[1]{\mathbf{\hat{\text{$#1$}}}}
\newcommand{\ii}{\vu{\imath}}
\newcommand{\jj}{\vu{\jmath}}
\newcommand{\kk}{\thickhat{k}}

\newcommand{\bC}{\mathbb{C}}        %%%complex numbers
\newcommand{\bN}{\mathbb{N}}        %%%natural numbers
\newcommand{\bP}{\mathbb{P}}        %%%polynomials
\newcommand{\bR}{\mathbb{R}}        %%%real numbers
\newcommand{\bZ}{\mathbb{Z}}        %%%integer numbers
\newcommand{\cB}{\mathcal{B}}       %%%basis
\newcommand{\cC}{\mathcal{C}}       %%%basis
\newcommand{\cM}{\mathcal{M}}       %%%matrix family

\newcommand{\sT}{\mathsf{T}}        %%%traspuesta

\renewcommand{\geq}{\geqslant}      %%%(to save typing)
\renewcommand{\leq}{\leqslant}      %%%(to save typing)
\newcommand{\x}{\times}             %%%product
\renewcommand{\:}{\colon}           %%%colon in  f: A -> B
\newcommand{\isom}{\simeq}              %% isomorfismo

\newcommand{\un}[1]{\underline{#1}}
\newcommand{\half}{\frac12}

\newcommand*{\Cdot}{{\raisebox{-0.25ex}{\scalebox{1.5}{$\cdot$}}}}      %% cdot más grande
\renewcommand{\.}{\Cdot}                %% producto escalar

\newcommand{\twobyone}[2]{\begin{pmatrix} %% 2 x 1 matrix
  #1 \\ #2 \end{pmatrix}}
  \newcommand{\twobytwo}[4]{\begin{pmatrix} %% 2 x 2 matrix
    #1 & #2 \\ #3 & #4 \end{pmatrix}}
    \newcommand{\twobythree}[6]{\begin{pmatrix} %% 2 x 3 matrix
        #1 & #2 & #3\\ #4 & #5 & #6 \end{pmatrix}}
\newcommand{\threebyone}[3]{\begin{pmatrix} %% 3 x 1 matrix
  #1 \\ #2 \\ #3 \end{pmatrix}}
  \newcommand{\threebytwo}[6]{\begin{pmatrix} %% 3 x 1 matrix
    #1 & #2\\ #3 & #4\\ #5&#6 \end{pmatrix}}
\newcommand{\threebythree}[9]{\begin{pmatrix} %% 3 x 3 matrix
  #1 & #2 & #3 \\ #4 & #5 & #6 \\ #7 & #8 & #9 \end{pmatrix}}

\newcommand{\To}{\Rightarrow}

\newcommand{\vaf}{\barerrightarrow}

\newcommand{\set}[1]{\{\,#1\,\}}    %% set notation
\newcommand{\Set}[1]{\biggl\{\,#1\,\biggr\}} %% set notation (large)
\newcommand{\red}[1]{\textcolor{red}{#1}}
\newcommand{\blu}[1]{\textcolor{blue}{#1}}

%----------------------------------------------------------------------------------------
%	ARTICLE CONTENTS
%----------------------------------------------------------------------------------------

\begin{document}
%\begin{multicols}{2}

\begin{Ej}
  Consider the figure which describes a region in the plane.
  \begin{figure}[h]
    \begin{center}

% Pattern Info
 
\tikzset{
  pattern size/.store in=\mcSize, 
  pattern size = 5pt,
  pattern thickness/.store in=\mcThickness, 
  pattern thickness = 0.3pt,
  pattern radius/.store in=\mcRadius, 
  pattern radius = 1pt}
  \makeatletter
  \pgfutil@ifundefined{pgf@pattern@name@_6vwej9fqd}{
  \pgfdeclarepatternformonly[\mcThickness,\mcSize]{_6vwej9fqd}
  {\pgfqpoint{0pt}{0pt}}
  {\pgfpoint{\mcSize}{\mcSize}}
  {\pgfpoint{\mcSize}{\mcSize}}
  {
  \pgfsetcolor{\tikz@pattern@color}
  \pgfsetlinewidth{\mcThickness}
  \pgfpathmoveto{\pgfqpoint{0pt}{\mcSize}}
  \pgfpathlineto{\pgfpoint{\mcSize+\mcThickness}{-\mcThickness}}
  \pgfpathmoveto{\pgfqpoint{0pt}{0pt}}
  \pgfpathlineto{\pgfpoint{\mcSize+\mcThickness}{\mcSize+\mcThickness}}
  \pgfusepath{stroke}
  }}
  \makeatother
  \tikzset{every picture/.style={line width=0.75pt}} %set default line width to 0.75pt        
  
  \begin{tikzpicture}[x=0.75pt,y=0.75pt,yscale=-1,xscale=1]
  %uncomment if require: \path (0,300); %set diagram left start at 0, and has height of 300
  
  %Shape: Arc [id:dp7124754957912127] 
  \draw  [draw opacity=0] (100,130) .. controls (83.43,130) and (70,116.57) .. (70,100) -- (100,100) -- cycle ; \draw   (100,130) .. controls (83.43,130) and (70,116.57) .. (70,100) ;  
  %Straight Lines [id:da8088218118996573] 
  \draw    (100,50) -- (100,150) ;
  %Straight Lines [id:da6233652144624046] 
  \draw    (150,100) -- (50,100) ;
  %Shape: Arc [id:dp390181704706716] 
  \draw  [draw opacity=0] (100,145) .. controls (75.15,145) and (55,124.85) .. (55,100) -- (100,100) -- cycle ; \draw   (100,145) .. controls (75.15,145) and (55,124.85) .. (55,100) ;  
  %Straight Lines [id:da6949708290712928] 
  \draw    (70,100) -- (75,95) ;
  %Straight Lines [id:da1489601834149794] 
  \draw    (55,100) -- (60,95) ;
  %Shape: Block Arc [id:dp2838446787529679] 
  \draw  [pattern=_6vwej9fqd,pattern size=6pt,pattern thickness=0.75pt,pattern radius=0pt, pattern color={rgb, 255:red, 0; green, 0; blue, 0}] (99.67,145) .. controls (75.06,144.82) and (55.14,124.88) .. (55,100.26) -- (70.67,100.17) .. controls (70.76,116.22) and (83.74,129.22) .. (99.78,129.33) -- cycle ;
  
  % Text Node
  \draw (60,95) node [anchor=south] [inner sep=0.75pt]  [font=\tiny]  {$-4$};
  % Text Node
  \draw (75,95) node [anchor=south] [inner sep=0.75pt]  [font=\tiny]  {$-3$};
  % Text Node
  \draw (150,96.6) node [anchor=south] [inner sep=0.75pt]  [font=\tiny]  {$x^{+}$};
  % Text Node
  \draw (102,50) node [anchor=west] [inner sep=0.75pt]  [font=\tiny]  {$y^{+}$};
  
  
  \end{tikzpicture}
\end{center}
  \end{figure}
  The marked points are $x=-4$ and $x=-3$. Describe the region in polar coordinates.
\end{Ej}

\begin{ptcb}
  \vspace*{3.25cm}
\end{ptcb}

\begin{Ej}
  Consider the complex number $z=e^{2\pi i/3}$. Show that the argument of $\bar{z}$ coincides with the argument $z^2$.
\end{Ej}

\begin{ptcb}
  \vspace*{3.25cm}
\end{ptcb}

\begin{Ej}
  Consider the complex number $w=2i$. What is the real part of $\frac{w}{1+w^2}$?
\end{Ej}

\begin{ptcb}
  \vspace*{3.25cm}
\end{ptcb}

\begin{Ej}
 Solve the equation $z+3\bar z=8-5i$ where $z$ is a complex number. 
\end{Ej}

\begin{ptcb}
  \vspace*{3.25cm}
\end{ptcb}


%\end{multicols}
\end{document} 