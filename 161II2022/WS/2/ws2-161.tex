%----------------------------------------------------------------------------------------
%	PACKAGES AND OTHER DOCUMENT CONFIGURATIONS
%----------------------------------------------------------------------------------------

\documentclass[12pt]{article}
\usepackage[spanish]{babel} %Tildes
\usepackage[extreme]{savetrees} %Espaciado e interlineado. Comentar si no gusta el interlineado.
\usepackage[utf8]{inputenc} %Encoding para tildes
\usepackage[breakable,skins]{tcolorbox} %Cajitas
\usepackage{fancyhdr} % Se necesita para el título arriba
\usepackage{lastpage} % Se necesita para poner el número de página
\usepackage{amsmath,amsfonts,amssymb,amsthm} %simbolos y demás
\usepackage{mathabx} %más símbolos
\usepackage{physics} %simbolos de derivadas, bra-ket.
\usepackage{multicol}
\usepackage[customcolors]{hf-tikz}
\usepackage[shortlabels]{enumitem}
\usepackage{tikz}

%\def\darktheme
%%%%%%%%% === Document Configuration === %%%%%%%%%%%%%%

\pagestyle{fancy}
\setlength{\headheight}{14.49998pt} %NO MODIFICAR
\setlength{\footskip}{14.49998pt} %NO MODIFICAR

\ifx \darktheme\undefined

\lhead{Math161S1} % Nombre de autor
\chead{\textbf{Week 2}} % Titulo
\rhead{}%\firstxmark} 
\lfoot{}%\lastxmark}
\cfoot{}
\rfoot{Page \thepage\ of\ \pageref{LastPage}} %A la derecha saldrá pág. 6 de 9. 
\else
\pagenumbering{gobble}
\pagecolor[rgb]{0,0,0}%{0.23,0.258,0.321}
\color[rgb]{1,1,1}
\fi

%%%%%%%%% === My T Color Box === %%%%%%%%%%%%%%

\ifx \darktheme\undefined
\newtcolorbox{ptcb}{
colframe = black,
colback = white,
breakable,
enhanced
}
\newtcolorbox{ptcbP}{
colframe = black,
colback = white,
coltitle = black,
colbacktitle = black!40,
title = Practice,
breakable,
enhanced
}

\else
\newtcolorbox{ptcb}{
colframe = white,
colback = black,
colupper = white,
breakable,
enhanced
}
\newtcolorbox{ptcbP}{
colframe = white,
colback = black,
colupper = white,
coltitle = white,
colbacktitle = black,
title = Practice,
breakable,
enhanced
}
\fi

%%%%%%%%% === Tikz para matrices === %%%%%%%%%%%%%%

\tikzset{
  style green/.style={
    set fill color=green!50!lime!60,
    set border color=white,
  },
  style cyan/.style={
    set fill color=cyan!90!blue!60,
    set border color=white,
  },
  style orange/.style={
    set fill color=orange!80!red!60,
    set border color=white,
  },
  row/.style={
    above left offset={-0.15,0.31},
    below right offset={0.15,-0.125},
    #1
  },
  col/.style={
    above left offset={-0.1,0.3},
    below right offset={0.15,-0.15},
    #1
  }
}

%%%%%%%%% === Theorems and suchlike === %%%%%%%%%%%%%%

\theoremstyle{plain}
\newtheorem{Th}{Theorem}  %%% Theorem 1.1
\newtheorem*{nTh}{Theorem}             %%% No-numbered Theorem
\newtheorem{Prop}[Th]{Proposition}     %%% Proposition 1.2
\newtheorem{Lem}[Th]{Lemma}             %%% Lemma 1.3
\newtheorem*{nLem}{Lemma}               %%% No-numbered Lemma
\newtheorem{Cor}[Th]{Corollary}        %%% Corollary 1.4
\newtheorem*{nCor}{Corollary}          %%% No-numbered Corollary

\theoremstyle{definition}
\newtheorem*{Def}{Definition}       %%% Definition 1.5
\newtheorem*{nonum-Def}{Definition}    %%% No number Definition
\newtheorem*{nEx}{Example}             %%% No number Example
\newtheorem{Ex}[Th]{Example}           %%% Example
\newtheorem{Ej}[Th]{Exercise}         %%% Exercise
\newtheorem*{nEj}{Exercise}           %%% No number Excercise
\newtheorem*{Not}{Notation}       %%% Definition 1.5

\theoremstyle{remark}
\newtheorem*{Rmk}{Remark}      %%%Remark 1.6

%\numberwithin{equation}{section}

\setlength{\parindent}{3ex}

%%====== Useful macros: =======%%%

\DeclareMathOperator{\gen}{gen}     %%%set generated by...
\DeclareMathOperator{\Rng}{Rng}     %%%rangomat
\DeclareMathOperator{\Nul}{Nul}     %%%rangomat
\DeclareMathOperator{\Proy}{Proy}   %%%proyección
\DeclareMathOperator{\id}{id}       %%%identity operator

\newcommand{\al}{\alpha}            %%%short for \alpha
\newcommand{\la}{\lambda}           %%%short for \lambda
\newcommand{\sg}{\sigma}            %%%short for \sigma
\newcommand{\te}{\theta}                %% short for  \theta
\renewcommand{\l}{\ell}

\newcommand{\thickhat}[1]{\mathbf{\hat{\text{$#1$}}}}
\newcommand{\ii}{\vu{\imath}}
\newcommand{\jj}{\vu{\jmath}}
\newcommand{\kk}{\thickhat{k}}

\newcommand{\bC}{\mathbb{C}}        %%%complex numbers
\newcommand{\bN}{\mathbb{N}}        %%%natural numbers
\newcommand{\bP}{\mathbb{P}}        %%%polynomials
\newcommand{\bR}{\mathbb{R}}        %%%real numbers
\newcommand{\bZ}{\mathbb{Z}}        %%%integer numbers
\newcommand{\cB}{\mathcal{B}}       %%%basis
\newcommand{\cC}{\mathcal{C}}       %%%basis
\newcommand{\cM}{\mathcal{M}}       %%%matrix family

\newcommand{\sT}{\mathsf{T}}        %%%traspuesta

\renewcommand{\geq}{\geqslant}      %%%(to save typing)
\renewcommand{\leq}{\leqslant}      %%%(to save typing)
\newcommand{\x}{\times}             %%%product
\renewcommand{\:}{\colon}           %%%colon in  f: A -> B
\newcommand{\isom}{\simeq}              %% isomorfismo

\newcommand{\un}[1]{\underline{#1}}
\newcommand{\half}{\frac12}

\newcommand*{\Cdot}{{\raisebox{-0.25ex}{\scalebox{1.5}{$\cdot$}}}}      %% cdot más grande
\renewcommand{\.}{\Cdot}                %% producto escalar

\newcommand{\twobyone}[2]{\begin{pmatrix} %% 2 x 1 matrix
  #1 \\ #2 \end{pmatrix}}
  \newcommand{\twobytwo}[4]{\begin{pmatrix} %% 2 x 2 matrix
    #1 & #2 \\ #3 & #4 \end{pmatrix}}
    \newcommand{\twobythree}[6]{\begin{pmatrix} %% 2 x 3 matrix
        #1 & #2 & #3\\ #4 & #5 & #6 \end{pmatrix}}
\newcommand{\threebyone}[3]{\begin{pmatrix} %% 3 x 1 matrix
  #1 \\ #2 \\ #3 \end{pmatrix}}
  \newcommand{\threebytwo}[6]{\begin{pmatrix} %% 3 x 1 matrix
    #1 & #2\\ #3 & #4\\ #5&#6 \end{pmatrix}}
\newcommand{\threebythree}[9]{\begin{pmatrix} %% 3 x 3 matrix
  #1 & #2 & #3 \\ #4 & #5 & #6 \\ #7 & #8 & #9 \end{pmatrix}}

\newcommand{\To}{\Rightarrow}

\newcommand{\vaf}{\overrightarrow}

\newcommand{\set}[1]{\{\,#1\,\}}    %% set notation
\newcommand{\Set}[1]{\biggl\{\,#1\,\biggr\}} %% set notation (large)
\newcommand{\red}[1]{\textcolor{red}{#1}}
\newcommand{\blu}[1]{\textcolor{blue}{#1}}

%----------------------------------------------------------------------------------------
%	ARTICLE CONTENTS
%----------------------------------------------------------------------------------------

\begin{document}
\begin{multicols}{2}
\section*{Trigonometric Substitution}
We have seen integrals of the type
$$
\int x\sqrt{1-x^2}\dd x,\ \int\frac{x}{\sqrt{1-x^2}}\dd x
$$
which we are able to solve by direct $u$-substitution. In both, by taking $u=1-x^2$, we obtain the following integrals
$$
\int \sqrt{u}\left(\frac{-\dd u}{2}\right),\ \int\frac{-\dd u}{2\sqrt{u}}.
$$
However now we are going to face integrals like 
$$\int\frac{\sqrt{1-x^2}}{x}\dd x.$$
The typical $u$-sub, $u=1-x^2$ doesn't work here, this integral becomes 
$$\int\frac{\sqrt{u}}{\sqrt{1-u}}\left(\frac{\dd u}{-2\sqrt{1-u}}\right)=-\frac{1}{2}\int\frac{\sqrt{u}}{1-u}\dd u.$$
We will need a new technique to work this type of integrals out.
\subsection*{Sine substitution}
For now, let's take $x=\sin(\te)$ without worrying too much from where that came from.

\begin{Ex} 
The integral $\int\frac{\sqrt{1-x^2}}{x}\dd x$ can be found using the substitution $x=\sin(\te)$.\par 
In this case $\dd x=\cos(\te)\dd\te$ after differentiating and so the integral becomes 
$$\int\frac{\sqrt{1-\sin^2(\te)}}{\sin(\te)}(\cos(\te)\dd\te).$$
By applying the Pythagorean identity we can simplify the square root:
$$\sin^2(\te)+\cos^2(\te)=1\To\cos^2(\te)=1-\sin^2(\te).$$
After taking this back into the integral we obtain
$$\int\frac{\sqrt{\cos^2(\te)}}{\sin(\te)}(\cos(\te)\dd\te)=\int \frac{\cos^2(\te)}{\sin(\te)}\dd\te$$
and by using the Pythagorean identity once more and changing the cosine to a sine we get 
$$\frac{\cos^2(\te)}{\sin(\te)}=\frac{1-\sin^2(\te)}{\sin(\te)}=\frac{1}{\sin(\te)}-\sin(\te)=\csc(\te)-\sin(\te).$$
By linearity the integral separates into 
$$\int\csc(\te)\dd\te-\int\sin(\te)\dd\te=\un{-\log[\csc(\te)+\cot(\te)]+\cos(\te)}.$$
The integral is not completely done at this point because we need to substitute back in our $x=\sin(\te)$ which would become a $\te=\arcsin(x)$. But \emph{for now} we will be satisfied.
\end{Ex}

\begin{Ex}
  Consider the following integral
  $$\int\frac{\dd x}{x^4\sqrt{4-x^2}}.$$
  Since we are thinking in terms of a sine substitution we can take $x=\sin(\te)$ once more, however that would turn our root into a 
  $4-\sin^2(\te)$ and \textbf{there's no identity for that expression}.\par
  We want that to become a $\cos^2(\te)$ in some way so if we multiply our sine like this\dots
  $$x=2\sin(\te)\To x^2=4\sin^2(\te)\To 4-x^2=4-4\sin^2(\te)$$
  then we can factor out the $4$ and work the integral as before!\par 
  Let us take the substitution
  $$x=2\sin(\te)\To\dd x=2\cos(\te)\dd\te$$
  and replacing inside the integral we obtain 
  \begin{align*}
    \int\frac{\dd x}{x^4\sqrt{4-x^2}}&=\int\frac{2\cos(\te)\dd\te}{(2\sin(\te))^4\sqrt{4-4\sin^2(\te)}}\\
    &=\int\frac{2\cos(\te)\dd\te}{16\sin^4(\te)\sqrt{4\cos^2(\te)}}\\
    &=\frac{1}{16}\int\csc^4(\te)\dd\te
  \end{align*}
This type of integral can be solved by separating the cosecant and using a trigonometric identity. Recall from the Pythagorean identity:
$$\sin^2(\te)+\cos^2(\te)=1\To 1+\cot^2(\te)=\csc^2(\te).$$
Then separating the cosecant into two squares we get:
\begin{align*}
  \frac{1}{16}\int\csc^4(\te)\dd\te&=\frac{1}{16}\int\csc^2(\te)\csc^2(\te)\dd\te\\
  &=\frac{1}{16}\int(1+\cot^2(\te))\csc^2(\te)\dd\te\\
  (\substack{u=\cot(\te)\\ \dd u=-\csc^2(\te)\dd\te})&=\frac{-1}{16}\int 1+u^2\dd u\\
  &=\frac{1}{16}\left(u+\frac{1}{3}u^3\right)\\
  &=\un{\frac{1}{16}\cot(\te)+\frac{1}{48}\cot^3(\te)}
\end{align*}
\end{Ex}

\begin{Prop}
Integrals with quadratic expressions inside of radicals such as \un{$\sqrt{a^2-b^2x^2}$} will be worked using the substitution \un{$x=\frac{a}{b}\sin(x)$}.
\end{Prop}

\begin{ptcbP}
Consider the following integral:
$$\int e^x\sqrt{1-9e^{2x}}\dd x.$$
\begin{enumerate}[i)]
  \itemsep=-0.4em
  \item Transform this integral by using a $u$-substitution into another one that could be done with a trigonometric substitution.
  \item Which of the following substitutions would lead to a correct answer? $u=3\sin(\te)$, $u=\sin(3\te)$ or $u=(1/3)\sin(\te)$. Discuss with your group members!
\end{enumerate}
\end{ptcbP}

\subsection*{Tangent Substitution}

When we have radicals involving an expression similar to $\sqrt{1+x^2}$ we will instead take $x=\tan(\te)$.
\begin{Rmk}
Notice the difference in the sign between the sine subs and the tangent subs. With sine it's a minus, and with tangent, a plus!
\end{Rmk}
\begin{Ex}
  Consider the following integral
  $$\int x^3\sqrt{1+x^2}\dd x.$$
  The expression should remind us of the Pythagorean identity in another way. We have that 
  $$\sin^2(\te)+\cos^2(\te)=1\To \tan^2(\te)+1=\sec^2(\te).$$
  Let us substitute then 
  $$x=\tan(\te)\To\dd x=\dd(\tan(\te))=\sec^2(\te)\dd\te.$$
  Replacing this into the integral we obtain
  $$\int(\tan(\te))^3\sqrt{1+\tan^2(\te)}(\sec^2(\te)\dd\te)=\int\tan^3(\te)\sec^3(\te)\dd\te.$$
  This trigonometric integral can be done by using the Pythagorean identity and separating the cubes into smaller powers:
  \begin{align*}
    \tan^3(\te)\sec^3(\te)&=\tan^2(\te)\sec^2(\te)(\tan(\te)\sec(\te))\\
    &=(\sec^2(\te)-1)\sec^2(\te)(\tan(\te)\sec(\te))\\
    &=(\sec^4(\te)-\sec^2(\te))(\sec(\te)\tan(\te)).
  \end{align*}
  This is now the argument of our integral. We have an expression involving the secant and it's derivative so we will take:
  $$u=\sec(\te)\To\dd u=\dd(\sec(\te))=(\sec(\te)\tan(\te))\dd\te.$$
  Our integral finally becomes 
  $$\int(u^4-u^2)\dd u=\un{\frac{\sec^5(\te)}{5}-\frac{\sec^3(\te)}{3}}$$
\end{Ex}

\begin{Prop}
  Integrals with quadratic expressions inside of radicals such as \un{$\sqrt{a^2+b^2x^2}$} will be worked using the substitution \un{$x=\frac{a}{b}\tan(x)$}.
\end{Prop}

\begin{ptcbP}
  Consider the integral 
  $$\int e^{4x}\sqrt{1+e^{2x}}\dd x.$$
  \begin{enumerate}
   \itemsep=-0.4em
   \item Would the substitution $u=e^{2x}$ transform this into a trigonometric integral?
   \item What if we took $e^x=\tan(\te)$ at once? Can the integral be solved?
   \item Compare with your group members either approach, first a $u$-sub and then a trig. one and the direct substitution $e^x=\tan(\te)$.
  \end{enumerate}
\end{ptcbP}

\subsection*{Secant Substitution}

The secant substitution comes into play with radicals of the form 
$$\sqrt{x^2-1}\To x=\sec(\te)$$
because of the identity $\tan^2(\te)=\sec^2(\te)-1$. When taking the secant inside, the root becomes a root of $\tan^2$.

\begin{Ex}
  Consider the integral 
  $$\int\frac{\dd x}{\sqrt{x^2-2x-3}}.$$
  This polynomial inside the radical can be simplified by completing the square:
  \begin{align*}
    x^2-2x-3=x^2-2x(+1-1)-3&=(x^2-2x+1)-1-3\\
    &=(x-1)^2-4.
  \end{align*}
  Even if we have that $-1$ we will take the whole of it to become our secant:
  \begin{align*}
    x-1=2\sec(\te)&\To \dd(x-1)=\dd(2\sec(\te))\\
    &\To\dd x=2\sec(\te)\tan(\te)\dd\te.
  \end{align*}
  Thus replacing in the integral we get 
  \begin{align*}
    \int\frac{\dd x}{\sqrt{(x-1)^2-4}}&=\int\frac{2\sec(\te)\tan(\te)}{\sqrt{(2\sec(\te))^2-4}}\dd\te\\
    &=\int\frac{2\sec(\te)\tan(\te)}{\sqrt{4\tan^2(\te)}}\dd\te\\
    &=\int\frac{2\sec(\te)\tan(\te)}{2\tan(\te)}\dd\te\\
    &=\int\sec(\te)\dd\te=\un{\log(\sec(\te)+\tan(\te))}
  \end{align*}
\end{Ex}

\begin{Prop}
  We have the following summary
  \vspace*{-0.5em}
  \begin{itemize}
    \itemsep=-0.2em
    \item $\sqrt{a^2-b^2x^2}$ reminds us of $1-\sin^2(\te)=\cos^2(\te)$ so we take $x=\sin(\te)\To\dd x=\cos(\te)\dd\te$.
    \item $\sqrt{a^2+b^2x^2}$ reminds us of $1+\tan^2(\te)=\sec^2(\te)$ so we take $x=\tan(\te)\To\dd x=\sec^2(\te)\dd\te$.
    \item $\sqrt{b^2x^2-a^2}$ reminds us of $\sec^2(\te)-1=\tan^2(\te)$ so we take $x=\sec(\te)\To\dd x=\sec(\te)\tan(\te)\dd\te$.
  \end{itemize}
\end{Prop}

\begin{Ej}
Compute the following integrals, in some of them, it will be necessary to complete the square:
\vspace*{-0.4em}
\begin{enumerate}[i)]
  \itemsep=-0.2em
  \item $\displaystyle\int\frac{x}{\sqrt{x^2-4x}}\dd x$
  \item $\displaystyle\int\sqrt{4x^2+4x+2}\dd x$
  \item $\displaystyle\int\frac{x^3}{\sqrt{16-9x^2}}\dd x$
\end{enumerate}
Is there an easier way to compute the last integral? Discuss with your group members to know each other's approach.
\end{Ej}
\vfill\null
\end{multicols}
\end{document} 