%----------------------------------------------------------------------------------------
%	PACKAGES AND OTHER DOCUMENT CONFIGURATIONS
%----------------------------------------------------------------------------------------

\documentclass[12pt]{article}
\usepackage[spanish]{babel} %Tildes
\usepackage[extreme]{savetrees} %Espaciado e interlineado. Comentar si no gusta el interlineado.
\usepackage[utf8]{inputenc} %Encoding para tildes
\usepackage[breakable,skins]{tcolorbox} %Cajitas
\usepackage{fancyhdr} % Se necesita para el título arriba
\usepackage{lastpage} % Se necesita para poner el número de página
\usepackage{amsmath,amsfonts,amssymb,amsthm} %simbolos y demás
\usepackage{mathabx} %más símbolos
\usepackage{physics} %simbolos de derivadas, bra-ket.
\usepackage{multicol}
\usepackage[customcolors]{hf-tikz}
\usepackage[shortlabels]{enumitem}
\usepackage{tikz}
\usetikzlibrary{patterns}
\usepackage{siunitx}

%\def\darktheme
%%%%%%%%% === Document Configuration === %%%%%%%%%%%%%%

\pagestyle{fancy}
\setlength{\headheight}{14.49998pt} %NO MODIFICAR
\setlength{\footskip}{14.49998pt} %NO MODIFICAR

\ifx \darktheme\undefined

\lhead{Math161S1} % Nombre de autor
\chead{\textbf{Week 5}} % Titulo
\rhead{}%\firstxmark} 
\lfoot{}%\lastxmark}
\cfoot{}
\rfoot{Page \thepage\ of\ \pageref{LastPage}} %A la derecha saldrá pág. 6 de 9. 
\else
\pagenumbering{gobble}
\pagecolor[rgb]{0,0,0}%{0.23,0.258,0.321}
\color[rgb]{1,1,1}
\fi

%%%%%%%%% === My T Color Box === %%%%%%%%%%%%%%

\ifx \darktheme\undefined
\newtcolorbox{ptcb}{
colframe = black,
colback = white,
breakable,
enhanced
}
\newtcolorbox{ptcbP}{
colframe = black,
colback = white,
coltitle = black,
colbacktitle = black!40,
title = Practice,
breakable,
enhanced
}

\else
\newtcolorbox{ptcb}{
colframe = white,
colback = black,
colupper = white,
breakable,
enhanced
}
\newtcolorbox{ptcbP}{
colframe = white,
colback = black,
colupper = white,
coltitle = white,
colbacktitle = black,
title = Practice,
breakable,
enhanced
}
\fi

%%%%%%%%% === Tikz para matrices === %%%%%%%%%%%%%%

\tikzset{
  style green/.style={
    set fill color=green!50!lime!60,
    set border color=white,
  },
  style cyan/.style={
    set fill color=cyan!90!blue!60,
    set border color=white,
  },
  style orange/.style={
    set fill color=orange!80!red!60,
    set border color=white,
  },
  row/.style={
    above left offset={-0.15,0.31},
    below right offset={0.15,-0.125},
    #1
  },
  col/.style={
    above left offset={-0.1,0.3},
    below right offset={0.15,-0.15},
    #1
  }
}

%%%%%%%%% === Theorems and suchlike === %%%%%%%%%%%%%%

\theoremstyle{plain}
\newtheorem{Th}{Theorem}  %%% Theorem 1.1
\newtheorem*{nTh}{Theorem}             %%% No-numbered Theorem
\newtheorem{Prop}[Th]{Proposition}     %%% Proposition 1.2
\newtheorem{Lem}[Th]{Lemma}             %%% Lemma 1.3
\newtheorem*{nLem}{Lemma}               %%% No-numbered Lemma
\newtheorem{Cor}[Th]{Corollary}        %%% Corollary 1.4
\newtheorem*{nCor}{Corollary}          %%% No-numbered Corollary

\theoremstyle{definition}
\newtheorem*{Def}{Definition}       %%% Definition 1.5
\newtheorem*{nonum-Def}{Definition}    %%% No number Definition
\newtheorem*{nEx}{Example}             %%% No number Example
\newtheorem{Ex}[Th]{Example}           %%% Example
\newtheorem{Ej}[Th]{Exercise}         %%% Exercise
\newtheorem*{nEj}{Exercise}           %%% No number Excercise
\newtheorem*{Not}{Notation}       %%% Definition 1.5

\theoremstyle{remark}
\newtheorem*{Rmk}{Remark}      %%%Remark 1.6

%\numberwithin{equation}{section}

\setlength{\parindent}{3ex}

%%====== Useful macros: =======%%%

\DeclareMathOperator{\gen}{gen}     %%%set generated by...
\DeclareMathOperator{\Rng}{Rng}     %%%rangomat
\DeclareMathOperator{\Nul}{Nul}     %%%rangomat
\DeclareMathOperator{\Proy}{Proy}   %%%proyección
\DeclareMathOperator{\id}{id}       %%%identity operator

\newcommand{\al}{\alpha}            %%%short for \alpha
\newcommand{\la}{\lambda}           %%%short for \lambda
\newcommand{\sg}{\ \sigma}            %%%short for \ \sigma
\newcommand{\te}{\theta}                %% short for  \theta
\renewcommand{\l}{\ell}

\newcommand{\thickhat}[1]{\mathbf{\hat{\text{$#1$}}}}
\newcommand{\ii}{\vu{\imath}}
\newcommand{\jj}{\vu{\jmath}}
\newcommand{\kk}{\thickhat{k}}

\newcommand{\bC}{\mathbb{C}}        %%%complex numbers
\newcommand{\bN}{\mathbb{N}}        %%%natural numbers
\newcommand{\bP}{\mathbb{P}}        %%%polynomials
\newcommand{\bR}{\mathbb{R}}        %%%real numbers
\newcommand{\bZ}{\mathbb{Z}}        %%%integer numbers
\newcommand{\cB}{\mathcal{B}}       %%%basis
\newcommand{\cC}{\mathcal{C}}       %%%basis
\newcommand{\cM}{\mathcal{M}}       %%%matrix family

\newcommand{\sT}{\mathsf{T}}        %%%traspuesta

\renewcommand{\geq}{\geqslant}      %%%(to save typing)
\renewcommand{\leq}{\leqslant}      %%%(to save typing)
\newcommand{\x}{\times}             %%%product
\renewcommand{\:}{\colon}           %%%colon in  f: A -> B
\newcommand{\isom}{\ \simeq}              %% isomorfismo

\newcommand{\un}[1]{\underline{#1}}
\newcommand{\half}{\frac12}

\newcommand*{\Cdot}{{\raisebox{-0.25ex}{\scalebox{1.5}{$\cdot$}}}}      %% cdot más grande
\renewcommand{\.}{\Cdot}                %% producto escalar

\newcommand{\twobyone}[2]{\begin{pmatrix} %% 2 x 1 matrix
  #1 \\ #2 \end{pmatrix}}
  \newcommand{\twobytwo}[4]{\begin{pmatrix} %% 2 x 2 matrix
    #1 & #2 \\ #3 & #4 \end{pmatrix}}
    \newcommand{\twobythree}[6]{\begin{pmatrix} %% 2 x 3 matrix
        #1 & #2 & #3\\ #4 & #5 & #6 \end{pmatrix}}
\newcommand{\threebyone}[3]{\begin{pmatrix} %% 3 x 1 matrix
  #1 \\ #2 \\ #3 \end{pmatrix}}
  \newcommand{\threebytwo}[6]{\begin{pmatrix} %% 3 x 1 matrix
    #1 & #2\\ #3 & #4\\ #5&#6 \end{pmatrix}}
\newcommand{\threebythree}[9]{\begin{pmatrix} %% 3 x 3 matrix
  #1 & #2 & #3 \\ #4 & #5 & #6 \\ #7 & #8 & #9 \end{pmatrix}}

\newcommand{\To}{\Rightarrow}

\newcommand{\vaf}{\overrightarrow}

\newcommand{\set}[1]{\{\,#1\,\}}    %% set notation
\newcommand{\Set}[1]{\biggl\{\,#1\,\biggr\}} %% set notation (large)
\newcommand{\bonj}[1]{\left\lbrack\,#1\,\right\rbrack} 
\newcommand{\red}[1]{\textcolor{red}{#1}}
\newcommand{\blu}[1]{\textcolor{blue}{#1}}

%----------------------------------------------------------------------------------------
%	ARTICLE CONTENTS
%----------------------------------------------------------------------------------------

\begin{document}
\begin{multicols}{2}
\section*{Applications of Integrals}

\subsection*{Volume and Solids of Revolution}

\begin{Def}
A \un{solid of revolution} is the figure obtained by rotating a curve about an axis. In most cases the axis will be a \emph{horizontal} or \emph{vertical} line.
\end{Def}

\begin{center}
  

\tikzset{every picture/.style={line width=0.75pt}} %set default line width to 0.75pt        

\begin{tikzpicture}[x=0.51pt,y=0.51pt,yscale=-1,xscale=1]
%uncomment if require: \path (0,300); %set diagram left start at 0, and has height of 300

%Straight Lines [id:da9632076834465861] 
\draw    (120,190) -- (220,190) ;
%Straight Lines [id:da9414532264559211] 
\draw    (100,100) -- (100,170) ;
%Curve Lines [id:da07409744866923951] 
\draw    (120,140) .. controls (170,190.25) and (170,190.75) .. (220,110) ;
%Straight Lines [id:da8008657359627083] 
\draw    (370,30) -- (370,100) ;
%Curve Lines [id:da26220849441656013] 
\draw    (390,70) .. controls (440,120.25) and (440,120.75) .. (490,40) ;
%Curve Lines [id:da8076923843049669] 
\draw    (350,70) .. controls (299.5,121.25) and (300,121.25) .. (250,40) ;
%Shape: Ellipse [id:dp9258034091478037] 
\draw   (350,70) .. controls (350,64.48) and (358.95,60) .. (370,60) .. controls (381.05,60) and (390,64.48) .. (390,70) .. controls (390,75.52) and (381.05,80) .. (370,80) .. controls (358.95,80) and (350,75.52) .. (350,70) -- cycle ;
%Shape: Ellipse [id:dp02758602989588721] 
\draw   (250,40) .. controls (250,17.91) and (303.73,0) .. (370,0) .. controls (436.27,0) and (490,17.91) .. (490,40) .. controls (490,62.09) and (436.27,80) .. (370,80) .. controls (303.73,80) and (250,62.09) .. (250,40) -- cycle ;
%Curve Lines [id:da9125656293980242] 
\draw    (304,106) .. controls (339.5,130.25) and (401,129.25) .. (435,105) ;
%Straight Lines [id:da6418589944936308] 
\draw    (320,220) -- (420,220) ;
%Curve Lines [id:da8620508653462946] 
\draw    (320,170) .. controls (370,220.25) and (370,220.75) .. (420,140) ;
%Curve Lines [id:da14386722617145953] 
\draw    (320,270) .. controls (370,219.75) and (370,219.75) .. (420,300) ;
%Shape: Ellipse [id:dp6886424457945872] 
\draw   (310,220) .. controls (310,192.39) and (314.48,170) .. (320,170) .. controls (325.52,170) and (330,192.39) .. (330,220) .. controls (330,247.61) and (325.52,270) .. (320,270) .. controls (314.48,270) and (310,247.61) .. (310,220) -- cycle ;
%Shape: Ellipse [id:dp6179748136071703] 
\draw   (415,220) .. controls (415,175.82) and (417.24,140) .. (420,140) .. controls (422.76,140) and (425,175.82) .. (425,220) .. controls (425,264.18) and (422.76,300) .. (420,300) .. controls (417.24,300) and (415,264.18) .. (415,220) -- cycle ;
%Straight Lines [id:da6957230422586694] 
\draw    (240,160) -- (298.34,121.11) ;
\draw [shift={(300,120)}, rotate = 146.31] [color={rgb, 255:red, 0; green, 0; blue, 0 }  ][line width=0.75]    (10.93,-3.29) .. controls (6.95,-1.4) and (3.31,-0.3) .. (0,0) .. controls (3.31,0.3) and (6.95,1.4) .. (10.93,3.29)   ;
%Straight Lines [id:da7890424774403921] 
\draw    (240,160) -- (298.34,198.89) ;
\draw [shift={(300,200)}, rotate = 213.69] [color={rgb, 255:red, 0; green, 0; blue, 0 }  ][line width=0.75]    (10.93,-3.29) .. controls (6.95,-1.4) and (3.31,-0.3) .. (0,0) .. controls (3.31,0.3) and (6.95,1.4) .. (10.93,3.29)   ;

% Text Node
\draw (218,193.4) node [anchor=north east] [inner sep=0.75pt]    {$y=b$};
% Text Node
\draw (102,103.4) node [anchor=north west][inner sep=0.75pt]    {$x=a$};
% Text Node
\draw (350,15) node [anchor=north west][inner sep=0.75pt]    {$x=a$};
% Text Node
\draw (480,210) node [anchor=north east] [inner sep=0.75pt]    {$y=b$};
% Text Node
\draw (172,233) node [anchor=north west][inner sep=0.75pt]   [align=left] {{\fontfamily{pcr}\selectfont Horizontal}\\{\fontfamily{pcr}\selectfont Rotation}};
% Text Node
\draw (135,23) node [anchor=north west][inner sep=0.75pt]   [align=left] {{\fontfamily{pcr}\selectfont Vertical}\\{\fontfamily{pcr}\selectfont Rotation}};


\end{tikzpicture}

\end{center}

\subsubsection*{Method of the Rings}

To obtain the volume of the solid we will cut cross-sectional areas \emph{perpendicular} to the axis of rotation and integrate through the bounds of the curve.\par 
These cross-sectional areas will take the form of \textbf{disks}, whose areas can be calculated using the formula 
$$\un{\pi(R^2-r^2)}.$$
Our task is to determine the radii $R$ and $r$ as a function of $x$ or $y$.

\begin{Ex} 
We will find the volume of the solid obtained by rotating the region bounded by the curves 
$$
\left\lbrace
\begin{aligned}
  &f(x)=\sqrt[3]{x},\\
  &g(x)=x/4,
\end{aligned}
\right.
$$
inside the first quadrant through the $x$-axis. 
\begin{enumerate}[i)]
  \itemsep=-0.4em
  \item First we find the bounds of our curves. This is done by equating both expressions:
  \begin{align*}
    \sqrt[3]{x}=x/4&\iff x=x^3/64\To x^2=64\ \text{or}\ x=0\\
    &\iff x\in\set{0,\pm 8}.
  \end{align*}
  Since we are in the first quadrant the intersections must be \un{$x=0$ and $x=8$}.
  \item After graphing these curves we see that the upper curve is $f(x)$ and the lower is $g(x)$, so we obtain 
  $$R=f(x),\ r=g(x).$$
  \item The area of the larger disk is $\pi R^2$ and from that amount we subtract the area of the smaller disk $\pi r^2$ to obtain the cross-sectional disk's area:
  $$A(x)=\pi(x^{2/3}-x^2/16).$$
  \item Finally we integrate through the bounds we found to obtain the volume:
  $$V=\int_0^8\pi(x^{2/3}-x^2/16)\dd x=\frac{128\pi}{15}.$$
\end{enumerate}
\end{Ex}

\begin{Ex}
  Let us rotate the region from the last example about the line $x=-2$.\par 
  In this case this is now a \textbf{vertical rotation}, and we must switch our equations to be in terms of $y$.
  \begin{enumerate}[i)]
    \itemsep=-0.4em
    \item We switch the equations:
    $$
    \left\lbrace
    \begin{aligned}
      &y=\sqrt[3]{x}\iff x=y^3\To h(y)=y^3,\\
      &y=x/4\iff x=4y\To k(y)=4y.
    \end{aligned}
    \right.
    $$
    \item The intersections have not changed, but their $y$-coordinates are 
    $$y^3=4y\To y^2=4\ \text{or}\ y=0 \iff y\in\set{0,\pm 2}.$$
    Since we are in the first quadrant, the $y$-bounds are \un{$y=0$ and $y=2$}.
    \item $h$ and $k$ are the same curves as $f$ and $g$ just that now $k$ is the right one and $h$ is the left one. Thus 
    $$
    \left\lbrace
    \begin{aligned}
      &R=\text{Right}-\text{axis}=4y-(-2),\\
      &r=\text{Left}-\text{axis}=y^3-(-2).
    \end{aligned}
    \right.
    $$
    \item The area of the cross-sectional disk is 
    $$A(y)=\pi\left\lbrack(4y+2)^2-(y^3+2)^2\right\rbrack.$$
    \item We integrate the area through $y=0$ and $y=2$ to obtain 
    $$V=\int_0^2\pi (16 y + 16 y^2 - 4 y^3 - y^6)\dd y=\frac{848\pi}{21}$$
  \end{enumerate}
\end{Ex}
\begin{ptcb}
\begin{Rmk}
When rotation about a line $x=a$ we are doing a \textbf{vertical rotation}. While rotations about $y=b$ are \textbf{horizontal rotations}. Here $a,b$ are any real number.\par 
When doing a \textbf{vertical rotation} about the axis $x=a$ the radii will be:
$$
\left\lbrace
\begin{aligned}
&R=\text{Right}-\text{axis}=\text{Right}-a,\\
 &r=\text{Left}-\text{axis}=\text{Left}-a.
\end{aligned}
\right.
 $$
On the other hand, for \textbf{horizontal rotations} about $y=b$ we have 
$$
\left\lbrace
\begin{aligned}
&R=\text{Up}-\text{axis}=\text{Up}-b,\\
 &r=\text{Down}-\text{axis}=\text{Down}-b.
\end{aligned}
\right.
 $$
\end{Rmk}
\end{ptcb}
\begin{Ej}
  Determine the volume of the solid obtained by rotating the region bounded by the curves $y=6e^{-2x}$, $y=6+4x-x^2$ and $x=1$ about the axis $y=-1$. $\bonj{V=\frac{937}{15}+\frac{12}{e^2}+\frac{9}{e^4}}\pi$.
  \end{Ej}

  \begin{Ej}
    Determine the volume of the solid obtained by rotating the region bounded by the curves $x=y^2-4$ and $x=6-3y$ about the axis $x=24$. $\bonj{V=(31556/15)\pi}$.
    \end{Ej}

    \begin{Ej}
      Determine the volume of the solid obtained by rotating the triangle bounded by $y=2x+1$, $x=4$ and $y=3$ about the axis $x=-4$. $\bonj{V=126\pi}$.
      \end{Ej}
\newpage
\subsubsection*{Method of the Cylinders}

We know cut cross-sectional areas \emph{parallel} to the axis of rotation. These areas look like \emph{cylinders} (or shells). Their area is given by 
$$\un{2\pi r h}$$
and once again we find $r$ and $h$ as functions of $x$ and $y$.

\begin{Ex}
  Let us find the solid of revolution formed by rotating the region bounded by the curves
  $$
\left\lbrace
\begin{aligned}
  &f(x)=(x-1)(x-3)^2,\\
  &y=0\ (x\ \text{axis}),
\end{aligned}
\right.
$$
inside the first quadrant about the axis $x=0$ ($y$-axis).
\begin{enumerate}[i)]
  \itemsep=-0.4em
  \item We first find the bounds of integration by equating the expressions: 
  $$(x-1)(x-3)^2=0\iff x=1\ \text{or}\ x=3.$$
  So the intersections are \un{$x=1$ and $x=3$}.
  \item We graph the curves to see that $f(x)$ lies above the $x$ axis inside the interval $[1,3]$ but now we have that the height of the cylinders will be 
  $$h(x)=\text{Up}-\text{Down}=(x-1)(x-3)^2-0.$$
  The radius $r$ is the distance from the axis $x=0$ to the cylinders, this distance is precisely $x$.
  \item Thus the area of the cylinder at any point will be 
  $$A(x)=2\pi(x)[(x-1)(x-3)^2].$$
  \item We integrate the area through $x=1$ and $x=3$ to obtain the volume: 
  $$V=\int_1^3[2\pi(x^4-7x^3+15x^2-9x)]\dd x=24\pi/5.$$
\end{enumerate}
\end{Ex}

\begin{Rmk}
  If we had used the method of the rings in the last example, we would have to express $f(x)$ in terms of $y$, and inverting a cubic polynomial is not an easy task.
\end{Rmk}

\begin{Ex}
  We will find the volume of the solid of revolution formed by rotating the region bounded by 
  $$
  \left\lbrace
  \begin{aligned}
    &x=(y-2)^2,\\
    &y=x,
  \end{aligned}
  \right.
  $$
  inside the first quadrant, about the axis $y=-1$. 
  \begin{enumerate}[i)]
    \itemsep=-0.4em
    \item We first get the intersections:
    $$(y-2)^2=y\iff y^2-5y+4=0\iff y\in\set{1,4}.$$
    Our bounds will be \un{$y=1$ and $y=4$}.
    \item After graphing the curves, we see that the \un{rightmost one} is $x=y$ and the \un{left one} is $x=(y-2)^2$. Thus 
    $$h(y)=\text{Right}-\text{Left}=y-(y-2)^2.$$
    In this case the radius is the distance from the axis $y=-1$ to the point on the cylinder which is precisely $y$. So \un{$r=y-(-1)$}. 
    \item The area of the cylinder is 
    $$2\pi rh=2\pi(y+1)[y-(y-2)^2].$$
    \item We integrate to obtain the volume 
    $$V=\int_1^42\pi(-y^3+4y^2+y-4)\dd y=63\pi/2.$$
  \end{enumerate}
\end{Ex}

\begin{ptcb}
  \begin{Rmk}
 In general for the method of the cylinders, when doing a \textbf{vertical rotation} about the axis $x=a$ we have:
 $$
 \left\lbrace
 \begin{aligned}
  &r=\text{Up}-\text{Down},\\
  &h=x-\text{axis}=x-a.
 \end{aligned}
 \right.
  $$
  On the other hand, for \textbf{horizontal rotations} about $y=b$ we have
  $$
  \left\lbrace
  \begin{aligned}
   &r=\text{Right}-\text{Left},\\
   &h=y-\text{axis}=y-b.
  \end{aligned}
  \right.
   $$ 
  
  \end{Rmk}
  \end{ptcb}

\begin{Ej}
  Find the volume of the solid obtained by rotating the region bounded by the curves
  $$
  \left\lbrace
  \begin{aligned}
   &y=e^{x/2}/(x+2),\\
   &y=5-x/4,\\
   &x=-1,\ \text{and}\ x=6,
  \end{aligned}
  \right.
   $$ 
   about the axis $x=-2$. $\bonj{V=2\pi(392/3+2/\sqrt{e}-2e^3)}$.
\end{Ej}

\begin{Ej}
  Determine the volume of the solid obtained by rotating the region bounded by the curves 
  $$
  \left\lbrace
  \begin{aligned}
   &x=y^2-4,\\
   &x=6-3y,
  \end{aligned}
  \right.
   $$ 
  about the axis $y=-8$. $\bonj{V=(4459/6)\pi}$.
  \end{Ej}

  \begin{Ej}
    Determine the volume of the solid obtained by rotating the region bounded by the curves
    $$
  \left\lbrace
  \begin{aligned}
   &y=x^2-6x+9,\\
   &y=-x^2+6x-1,
  \end{aligned}
  \right.
   $$ 
    about the axis $x=8$. $\bonj{V=(640/3)\pi}$.
    \end{Ej}

\end{multicols}
\end{document} 