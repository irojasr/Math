%----------------------------------------------------------------------------------------
%	PACKAGES AND OTHER DOCUMENT CONFIGURATIONS
%----------------------------------------------------------------------------------------

\documentclass[12pt]{article}
\usepackage[spanish]{babel} %Tildes
\usepackage[extreme]{savetrees} %Espaciado e interlineado. Comentar si no gusta el interlineado.
\usepackage[utf8]{inputenc} %Encoding para tildes
\usepackage[breakable,skins]{tcolorbox} %Cajitas
\usepackage{fancyhdr} % Se necesita para el título arriba
\usepackage{lastpage} % Se necesita para poner el número de página
\usepackage{amsmath,amsfonts,amssymb,amsthm} %simbolos y demás
\usepackage{mathabx} %más símbolos
\usepackage{physics} %simbolos de derivadas, bra-ket.
\usepackage{multicol}
\usepackage[customcolors]{hf-tikz}
\usepackage[shortlabels]{enumitem}
\usepackage{tikz}
\usetikzlibrary{patterns}
\usepackage{siunitx}

%\def\darktheme
%%%%%%%%% === Document Configuration === %%%%%%%%%%%%%%

\pagestyle{fancy}
\setlength{\headheight}{14.49998pt} %NO MODIFICAR
\setlength{\footskip}{14.49998pt} %NO MODIFICAR

\ifx \darktheme\undefined

\lhead{Math161S1} % Nombre de autor
\chead{\textbf{Week 4}} % Titulo
\rhead{}%\firstxmark} 
\lfoot{}%\lastxmark}
\cfoot{}
\rfoot{Page \thepage\ of\ \pageref{LastPage}} %A la derecha saldrá pág. 6 de 9. 
\else
\pagenumbering{gobble}
\pagecolor[rgb]{0,0,0}%{0.23,0.258,0.321}
\color[rgb]{1,1,1}
\fi

%%%%%%%%% === My T Color Box === %%%%%%%%%%%%%%

\ifx \darktheme\undefined
\newtcolorbox{ptcb}{
colframe = black,
colback = white,
breakable,
enhanced
}
\newtcolorbox{ptcbP}{
colframe = black,
colback = white,
coltitle = black,
colbacktitle = black!40,
title = Practice,
breakable,
enhanced
}

\else
\newtcolorbox{ptcb}{
colframe = white,
colback = black,
colupper = white,
breakable,
enhanced
}
\newtcolorbox{ptcbP}{
colframe = white,
colback = black,
colupper = white,
coltitle = white,
colbacktitle = black,
title = Practice,
breakable,
enhanced
}
\fi

%%%%%%%%% === Tikz para matrices === %%%%%%%%%%%%%%

\tikzset{
  style green/.style={
    set fill color=green!50!lime!60,
    set border color=white,
  },
  style cyan/.style={
    set fill color=cyan!90!blue!60,
    set border color=white,
  },
  style orange/.style={
    set fill color=orange!80!red!60,
    set border color=white,
  },
  row/.style={
    above left offset={-0.15,0.31},
    below right offset={0.15,-0.125},
    #1
  },
  col/.style={
    above left offset={-0.1,0.3},
    below right offset={0.15,-0.15},
    #1
  }
}

%%%%%%%%% === Theorems and suchlike === %%%%%%%%%%%%%%

\theoremstyle{plain}
\newtheorem{Th}{Theorem}  %%% Theorem 1.1
\newtheorem*{nTh}{Theorem}             %%% No-numbered Theorem
\newtheorem{Prop}[Th]{Proposition}     %%% Proposition 1.2
\newtheorem{Lem}[Th]{Lemma}             %%% Lemma 1.3
\newtheorem*{nLem}{Lemma}               %%% No-numbered Lemma
\newtheorem{Cor}[Th]{Corollary}        %%% Corollary 1.4
\newtheorem*{nCor}{Corollary}          %%% No-numbered Corollary

\theoremstyle{definition}
\newtheorem*{Def}{Definition}       %%% Definition 1.5
\newtheorem*{nonum-Def}{Definition}    %%% No number Definition
\newtheorem*{nEx}{Example}             %%% No number Example
\newtheorem{Ex}[Th]{Example}           %%% Example
\newtheorem{Ej}[Th]{Exercise}         %%% Exercise
\newtheorem*{nEj}{Exercise}           %%% No number Excercise
\newtheorem*{Not}{Notation}       %%% Definition 1.5

\theoremstyle{remark}
\newtheorem*{Rmk}{Remark}      %%%Remark 1.6

%\numberwithin{equation}{section}

\setlength{\parindent}{3ex}

%%====== Useful macros: =======%%%

\DeclareMathOperator{\gen}{gen}     %%%set generated by...
\DeclareMathOperator{\Rng}{Rng}     %%%rangomat
\DeclareMathOperator{\Nul}{Nul}     %%%rangomat
\DeclareMathOperator{\Proy}{Proy}   %%%proyección
\DeclareMathOperator{\id}{id}       %%%identity operator

\newcommand{\al}{\alpha}            %%%short for \alpha
\newcommand{\la}{\lambda}           %%%short for \lambda
\newcommand{\sg}{\ \sigma}            %%%short for \ \sigma
\newcommand{\te}{\theta}                %% short for  \theta
\renewcommand{\l}{\ell}

\newcommand{\thickhat}[1]{\mathbf{\hat{\text{$#1$}}}}
\newcommand{\ii}{\vu{\imath}}
\newcommand{\jj}{\vu{\jmath}}
\newcommand{\kk}{\thickhat{k}}

\newcommand{\bC}{\mathbb{C}}        %%%complex numbers
\newcommand{\bN}{\mathbb{N}}        %%%natural numbers
\newcommand{\bP}{\mathbb{P}}        %%%polynomials
\newcommand{\bR}{\mathbb{R}}        %%%real numbers
\newcommand{\bZ}{\mathbb{Z}}        %%%integer numbers
\newcommand{\cB}{\mathcal{B}}       %%%basis
\newcommand{\cC}{\mathcal{C}}       %%%basis
\newcommand{\cM}{\mathcal{M}}       %%%matrix family

\newcommand{\sT}{\mathsf{T}}        %%%traspuesta

\renewcommand{\geq}{\geqslant}      %%%(to save typing)
\renewcommand{\leq}{\leqslant}      %%%(to save typing)
\newcommand{\x}{\times}             %%%product
\renewcommand{\:}{\colon}           %%%colon in  f: A -> B
\newcommand{\isom}{\ \simeq}              %% isomorfismo

\newcommand{\un}[1]{\underline{#1}}
\newcommand{\half}{\frac12}

\newcommand*{\Cdot}{{\raisebox{-0.25ex}{\scalebox{1.5}{$\cdot$}}}}      %% cdot más grande
\renewcommand{\.}{\Cdot}                %% producto escalar

\newcommand{\twobyone}[2]{\begin{pmatrix} %% 2 x 1 matrix
  #1 \\ #2 \end{pmatrix}}
  \newcommand{\twobytwo}[4]{\begin{pmatrix} %% 2 x 2 matrix
    #1 & #2 \\ #3 & #4 \end{pmatrix}}
    \newcommand{\twobythree}[6]{\begin{pmatrix} %% 2 x 3 matrix
        #1 & #2 & #3\\ #4 & #5 & #6 \end{pmatrix}}
\newcommand{\threebyone}[3]{\begin{pmatrix} %% 3 x 1 matrix
  #1 \\ #2 \\ #3 \end{pmatrix}}
  \newcommand{\threebytwo}[6]{\begin{pmatrix} %% 3 x 1 matrix
    #1 & #2\\ #3 & #4\\ #5&#6 \end{pmatrix}}
\newcommand{\threebythree}[9]{\begin{pmatrix} %% 3 x 3 matrix
  #1 & #2 & #3 \\ #4 & #5 & #6 \\ #7 & #8 & #9 \end{pmatrix}}

\newcommand{\To}{\Rightarrow}

\newcommand{\vaf}{\overrightarrow}

\newcommand{\set}[1]{\{\,#1\,\}}    %% set notation
\newcommand{\Set}[1]{\biggl\{\,#1\,\biggr\}} %% set notation (large)
\newcommand{\red}[1]{\textcolor{red}{#1}}
\newcommand{\blu}[1]{\textcolor{blue}{#1}}

%----------------------------------------------------------------------------------------
%	ARTICLE CONTENTS
%----------------------------------------------------------------------------------------

\begin{document}
\begin{multicols}{2}
\section*{Applications of Integrals}

\subsection*{Work}

When a force is applied to an object, energy is transferred from or into it. \emph{Work} is transferred energy and \emph{doing work} is the act of transferring that energy.
\iffalse
\par
Recall that work was defined by the formula \un{$W=F\.d\.\cos(\te)$}
where $\te$ was the angle between the force and the displacement.\par 
The force might now be \emph{variable} across the displacement so we will extend this definition.
\fi
\begin{Def}
  If $F=F(x)$ is a force moves an object moved along a path with endpoints $x_i$ and $x_f$, then the work done by $F$ is 
  $$\int_{x_i}^{x_f}F(x)\dd x.$$
\end{Def}
\vspace*{-0.5em}
\begin{Ex}
Suppose a box weighs $19\ \si{\newton}$. It is hanging by an infinitesimally thin rope which is $6\ \si{\metre}$ long. We will find the work necessary to pull the box up.
\vspace*{-0.5em}
\begin{center}
  \tikzset{every picture/.style={line width=0.75pt}} %set default line width to 0.75pt        

\begin{tikzpicture}[x=0.75pt,y=0.75pt,yscale=-1,xscale=1]
%uncomment if require: \path (0,300); %set diagram left start at 0, and has height of 300

%Flowchart: Process [id:dp0938356142417498] 
\draw   (130,70) -- (190,70) -- (190,130) -- (130,130) -- cycle ;
%Flowchart: Internal Storage [id:dp3615213765362848] 
\draw   (100,100) -- (120,100) -- (120,120) -- (100,120) -- cycle ; \draw   (102.5,100) -- (102.5,120) ; \draw   (100,102.5) -- (120,102.5) ;
%Straight Lines [id:da7491030256750448] 
\draw    (110,100) -- (110,50) ;
%Shape: Circle [id:dp03924037009573356] 
\draw   (110,50) .. controls (110,44.48) and (114.48,40) .. (120,40) .. controls (125.52,40) and (130,44.48) .. (130,50) .. controls (130,55.52) and (125.52,60) .. (120,60) .. controls (114.48,60) and (110,55.52) .. (110,50) -- cycle ;
%Straight Lines [id:da4615590205830298] 
\draw    (120,50) -- (130,70) ;
%Straight Lines [id:da3793288770432668] 
\draw  [dash pattern={on 4.5pt off 4.5pt}]  (120,40) -- (150,40) ;
%Straight Lines [id:da9709726917379455] 
\draw    (130,130) -- (140,140) ;
%Straight Lines [id:da6023724791929683] 
\draw    (140,130) -- (150,140) ;
%Straight Lines [id:da9529302717550674] 
\draw    (150,130) -- (160,140) ;
%Straight Lines [id:da770400140639595] 
\draw    (160,130) -- (170,140) ;
%Straight Lines [id:da2505283172970174] 
\draw    (170,130) -- (180,140) ;
%Straight Lines [id:da9364074980742183] 
\draw    (180,130) -- (190,140) ;
%Flowchart: Internal Storage [id:dp581159856988392] 
\draw   (240,100) -- (260,100) -- (260,120) -- (240,120) -- cycle ; \draw   (242.5,100) -- (242.5,120) ; \draw   (240,102.5) -- (260,102.5) ;
%Shape: Circle [id:dp48388887820853155] 
\draw   (250,50) .. controls (250,44.48) and (254.48,40) .. (260,40) .. controls (265.52,40) and (270,44.48) .. (270,50) .. controls (270,55.52) and (265.52,60) .. (260,60) .. controls (254.48,60) and (250,55.52) .. (250,50) -- cycle ;
%Straight Lines [id:da2773804301272753] 
\draw [color={rgb, 255:red, 0; green, 93; blue, 164 }  ,draw opacity=1 ]   (278,110) -- (222,110) ;
\draw [shift={(220,110)}, rotate = 360] [color={rgb, 255:red, 0; green, 93; blue, 164 }  ,draw opacity=1 ][line width=0.75]    (10.93,-3.29) .. controls (6.95,-1.4) and (3.31,-0.3) .. (0,0) .. controls (3.31,0.3) and (6.95,1.4) .. (10.93,3.29)   ;
\draw [shift={(280,110)}, rotate = 180] [color={rgb, 255:red, 0; green, 93; blue, 164 }  ,draw opacity=1 ][line width=0.75]    (10.93,-3.29) .. controls (6.95,-1.4) and (3.31,-0.3) .. (0,0) .. controls (3.31,0.3) and (6.95,1.4) .. (10.93,3.29)   ;
%Straight Lines [id:da7212288171804997] 
\draw [color={rgb, 255:red, 0; green, 93; blue, 164 }  ,draw opacity=1 ]   (250,52) -- (250,110) ;
\draw [shift={(250,50)}, rotate = 90] [color={rgb, 255:red, 0; green, 93; blue, 164 }  ,draw opacity=1 ][line width=0.75]    (10.93,-3.29) .. controls (6.95,-1.4) and (3.31,-0.3) .. (0,0) .. controls (3.31,0.3) and (6.95,1.4) .. (10.93,3.29)   ;
%Straight Lines [id:da36750862068781376] 
\draw [color={rgb, 255:red, 0; green, 93; blue, 164 }  ,draw opacity=1 ]   (270,90) -- (270,72) ;
\draw [shift={(270,70)}, rotate = 90] [color={rgb, 255:red, 0; green, 93; blue, 164 }  ,draw opacity=1 ][line width=0.75]    (10.93,-3.29) .. controls (6.95,-1.4) and (3.31,-0.3) .. (0,0) .. controls (3.31,0.3) and (6.95,1.4) .. (10.93,3.29)   ;
%Straight Lines [id:da509957260474805] 
\draw [color={rgb, 255:red, 0; green, 93; blue, 164 }  ,draw opacity=1 ]   (270,90) -- (288,90) ;
\draw [shift={(290,90)}, rotate = 180] [color={rgb, 255:red, 0; green, 93; blue, 164 }  ,draw opacity=1 ][line width=0.75]    (10.93,-3.29) .. controls (6.95,-1.4) and (3.31,-0.3) .. (0,0) .. controls (3.31,0.3) and (6.95,1.4) .. (10.93,3.29)   ;
%Straight Lines [id:da3891799375056051] 
\draw    (250,110) -- (270,120) ;
%Straight Lines [id:da7785508826611864] 
\draw    (240,40) -- (250,50) ;

% Text Node
\draw (292,86.6) node [anchor=south west] [inner sep=0.75pt]  [font=\small,color={rgb, 255:red, 0; green, 93; blue, 164 }  ,opacity=1 ]  {$x$};
% Text Node
\draw (272,66.6) node [anchor=south west] [inner sep=0.75pt]  [font=\small,color={rgb, 255:red, 0; green, 93; blue, 164 }  ,opacity=1 ]  {$y$};
% Text Node
\draw (272,123.4) node [anchor=north west][inner sep=0.75pt]  [font=\scriptsize]  {$( 0,0)$};
% Text Node
\draw (240,36.6) node [anchor=south] [inner sep=0.75pt]  [font=\scriptsize]  {$y_{f}$};


\end{tikzpicture}

\end{center}
\vspace*{-1.2em}
We place the (0,0) coordinate at the center of box assuming its mass is concentrated at this point. Displacement goes from $y_i=0\ \si\metre$ to $y_f=6\ \si\metre$.\par 
As only the gravitational force acts of the box, the work required to move the box from $y_i$ to $y_f$ is equal to:
$$W=\int_{y_i=0}^{y_f=6}(19\ \si{\newton})(\dd y\ \si{\metre})=\int_0^619\dd y\ \ \si{\newton\metre}=\un{114\ \si{\joule}}.$$
\end{Ex}
\vspace*{-1em}
\begin{Rmk}
We will not worry about units, in this example all units were included to illustrate.
\end{Rmk}

\begin{Ex}
Suppose the rope is \textbf{not weightless}. The \emph{linear density} of the rope is $\rho_R=5\ \si{\newton\per\metre}$, and it's pulled upwards at a speed of $1\ \si{\metre\per\second}$. \textbf{As it's pulled up, mass decreases. Therefore less work is required to pull the rope}. We can summarize:
  $$
  \left\lbrace
  \begin{aligned}
    &\text{Weight box}=F_B=7\ \si\newton\\
    &\text{Length rope}=\l_R=10\ \si\metre\\
    &\text{Weight rope}=F_R=???
  \end{aligned}
  \right.
  $$
At the starting point, the whole weight of the rope contributes to the work. And at the end, there's no more rope. Since we are pulling with constant velocity we can model the situation as follows:
\begin{center}
 

\tikzset{every picture/.style={line width=0.75pt}} %set default line width to 0.75pt        

\begin{tikzpicture}[x=0.75pt,y=0.75pt,yscale=-1,xscale=1]
%uncomment if require: \path (0,300); %set diagram left start at 0, and has height of 300

%Straight Lines [id:da3963628725639554] 
\draw [color={rgb, 255:red, 0; green, 93; blue, 164 }  ,draw opacity=1 ]   (120,260) -- (208,260) ;
\draw [shift={(210,260)}, rotate = 180] [color={rgb, 255:red, 0; green, 93; blue, 164 }  ,draw opacity=1 ][line width=0.75]    (10.93,-3.29) .. controls (6.95,-1.4) and (3.31,-0.3) .. (0,0) .. controls (3.31,0.3) and (6.95,1.4) .. (10.93,3.29)   ;
%Straight Lines [id:da4395954669823825] 
\draw [color={rgb, 255:red, 0; green, 93; blue, 164 }  ,draw opacity=1 ]   (120,260) -- (120,192) ;
\draw [shift={(120,190)}, rotate = 90] [color={rgb, 255:red, 0; green, 93; blue, 164 }  ,draw opacity=1 ][line width=0.75]    (10.93,-3.29) .. controls (6.95,-1.4) and (3.31,-0.3) .. (0,0) .. controls (3.31,0.3) and (6.95,1.4) .. (10.93,3.29)   ;
%Straight Lines [id:da860990618167697] 
\draw    (120,220) -- (190,260) ;
%Straight Lines [id:da9243809284032045] 
\draw [line width=1.5]    (120,220) -- (130,210) ;
%Straight Lines [id:da13975859575108562] 
\draw [line width=1.5]    (190,250) -- (190,265) ;
%Straight Lines [id:da16507195175156264] 
\draw [line width=1.5]    (120,220) -- (115,220) ;
%Straight Lines [id:da738681080138905] 
\draw [line width=1.5]    (115,265) -- (120,260) ;

% Text Node
\draw (118,193.4) node [anchor=north east] [inner sep=0.75pt]  [font=\footnotesize]  {$W_{R}( y)$};
% Text Node
\draw (212,256.6) node [anchor=south west] [inner sep=0.75pt]  [font=\footnotesize]  {$y$};
% Text Node
\draw (132,210) node [anchor=west] [inner sep=0.75pt]  [font=\scriptsize]  {$W_{R} =\text{total weight}$};
% Text Node
\draw (190,246.6) node [anchor=south] [inner sep=0.75pt]  [font=\scriptsize]  {$W_{R} =0$};
% Text Node
\draw (113,261.6) node [anchor=south east] [inner sep=0.75pt]  [font=\scriptsize]  {$y=0$};
% Text Node
\draw (188,265) node [anchor=east] [inner sep=0.75pt]  [font=\scriptsize]  {$y=\text{top}$};


\end{tikzpicture}

\end{center}
The total weight of the rope is
 $$W_{R,i}=\rho_R\.\l_R=50\ \si\newton.$$ 
 And at the end $W_{R,f}=0\ \si\newton$. The speed at which the rope moves is $1\ \si{\metre\per\second}$ so the rate of change in speed is the same as in distance.\par
 We can model the weight at any height $y$ using the linear equation
$$W_R(y)=\left(\frac{y_2-y_1}{x_2-x_1}\right)y+b=\left(\frac{0-50}{10-0}\right)y+50=-5y+50.$$
Notice that this can also be modeled by the equation
$$W_R(y)=\rho_R(\l_R-y)=5(10-y).$$
Thus the work is 
$$W=\underbrace{\int_0^{10} 7\dd y}_{\text{box}}+\underbrace{\int_0^{10}(-5y+50)\dd y}_{\text{rope}}=70+250=\un{320\ \si\joule}.$$
\end{Ex}
\vspace*{-1.2em}
\begin{Ex}
  Suppose it's now raining at a rate of $2\ \si{\newton\per\second}$, we are now raising a \textbf{open} box at $1\ \si{\metre\per\second}$ with the following conditions:
  $$
  \left\lbrace
  \begin{aligned}
    &\text{Weight box}=F_B=12\ \si\newton,\ \text{Box capacity}=\text{cap}_B=10\ \si\newton\\
    &\text{Length rope}=\l_R=8\ \si\metre,\ \text{Density rope}=\rho_R=4\ \si{\newton\per\metre}
  \end{aligned}
  \right.
  $$
Since it's \textbf{raining}, the box is getting filled with water whose weight will contribute to the work. We have to model this as with the weight of the rope.\par 
Initially, $W_{W,0}=0\ \si\newton$ since the box is empty, and the box can't hold more than $10\ \si\newton$, then that's $W_{W,f}$.\par 
However the box takes $t_{\text{fill}}=\frac{\text{cap}_B}{v_{\text{fill}}}=\frac{10}{2}=5\ \si\second$ to fill. And by that time, the box has been raised just $5\ \si\metre$. The final $3\ \si\metre$ will be with the full box of water. We can model as follows:
\begin{center}

  \tikzset{every picture/.style={line width=0.75pt}} %set default line width to 0.75pt        
  
  \begin{tikzpicture}[x=0.75pt,y=0.75pt,yscale=-1,xscale=1]
  %uncomment if require: \path (0,300); %set diagram left start at 0, and has height of 300
  
  %Straight Lines [id:da15700705660149783] 
  \draw [color={rgb, 255:red, 0; green, 93; blue, 164 }  ,draw opacity=1 ]   (141,255) -- (229,255) ;
  \draw [shift={(231,255)}, rotate = 180] [color={rgb, 255:red, 0; green, 93; blue, 164 }  ,draw opacity=1 ][line width=0.75]    (10.93,-3.29) .. controls (6.95,-1.4) and (3.31,-0.3) .. (0,0) .. controls (3.31,0.3) and (6.95,1.4) .. (10.93,3.29)   ;
  %Straight Lines [id:da5562690961417642] 
  \draw [color={rgb, 255:red, 0; green, 93; blue, 164 }  ,draw opacity=1 ]   (141,255) -- (141,187) ;
  \draw [shift={(141,185)}, rotate = 90] [color={rgb, 255:red, 0; green, 93; blue, 164 }  ,draw opacity=1 ][line width=0.75]    (10.93,-3.29) .. controls (6.95,-1.4) and (3.31,-0.3) .. (0,0) .. controls (3.31,0.3) and (6.95,1.4) .. (10.93,3.29)   ;
  %Straight Lines [id:da13246208790111447] 
  \draw [color={rgb, 255:red, 0; green, 192; blue, 243 }  ,draw opacity=1 ]   (141,255) -- (181,220) ;
  %Straight Lines [id:da8370610207479219] 
  \draw [line width=1.5]    (211,255) -- (211,260) ;
  %Straight Lines [id:da7987174000572641] 
  \draw [line width=1.5]    (141,220) -- (136,220) ;
  %Straight Lines [id:da7852929622056828] 
  \draw [line width=1.5]    (136,260) -- (141,255) ;
  %Straight Lines [id:da0935167374438195] 
  \draw [line width=1.5]    (181,255) -- (181,260) ;
  %Straight Lines [id:da22488315166851947] 
  \draw  [dash pattern={on 4.5pt off 4.5pt}]  (181,220) -- (181,255) ;
  %Straight Lines [id:da9750439645026963] 
  \draw  [dash pattern={on 4.5pt off 4.5pt}]  (181,220) -- (141,220) ;
  %Straight Lines [id:da9547915482339251] 
  \draw  [dash pattern={on 4.5pt off 4.5pt}]  (211,255) -- (211,220) ;
  %Straight Lines [id:da34539064819325316] 
  \draw [color={rgb, 255:red, 0; green, 192; blue, 243 }  ,draw opacity=1 ]   (181,220) -- (211,220) ;
  %Flowchart: Internal Storage [id:dp5950588097969658] 
  \draw   (290,245) -- (310,245) -- (310,265) -- (290,265) -- cycle ; \draw   (292.5,245) -- (292.5,265) ; \draw   (290,247.5) -- (310,247.5) ;
  %Straight Lines [id:da427731061971393] 
  \draw    (300,245) -- (300,195) ;
  %Shape: Circle [id:dp05625319262059647] 
  \draw   (300,195) .. controls (300,189.48) and (304.48,185) .. (310,185) .. controls (315.52,185) and (320,189.48) .. (320,195) .. controls (320,200.52) and (315.52,205) .. (310,205) .. controls (304.48,205) and (300,200.52) .. (300,195) -- cycle ;
  %Straight Lines [id:da8938442782471943] 
  \draw  [dash pattern={on 4.5pt off 4.5pt}]  (310,185) -- (340,185) ;
  %Straight Lines [id:da27717882254280335] 
  \draw    (280,237.5) -- (290,247.5) ;
  %Straight Lines [id:da4011726694279778] 
  \draw    (282.5,235) -- (292.5,245) ;
  %Straight Lines [id:da3706252992377512] 
  \draw    (280,237.5) -- (282.5,235) ;
  %Curve Lines [id:da8882778826060533] 
  \draw [color={rgb, 255:red, 142; green, 216; blue, 248 }  ,draw opacity=1 ]   (290,255) .. controls (299.89,240.33) and (299.89,259.89) .. (310,250) ;
  %Straight Lines [id:da0773536146448437] 
  \draw [color={rgb, 255:red, 142; green, 216; blue, 248 }  ,draw opacity=1 ]   (280,195) -- (280,200) ;
  %Straight Lines [id:da9207130736915694] 
  \draw [color={rgb, 255:red, 142; green, 216; blue, 248 }  ,draw opacity=1 ]   (285,200) -- (285,205) ;
  %Straight Lines [id:da9090552402507512] 
  \draw [color={rgb, 255:red, 142; green, 216; blue, 248 }  ,draw opacity=1 ]   (290,195) -- (290,200) ;
  %Straight Lines [id:da2354364760922001] 
  \draw [color={rgb, 255:red, 142; green, 216; blue, 248 }  ,draw opacity=1 ]   (295,200) -- (295,205) ;
  %Straight Lines [id:da8442899057976425] 
  \draw [color={rgb, 255:red, 142; green, 216; blue, 248 }  ,draw opacity=1 ]   (280,205) -- (280,210) ;
  %Straight Lines [id:da5830278233345441] 
  \draw [color={rgb, 255:red, 142; green, 216; blue, 248 }  ,draw opacity=1 ]   (285,210) -- (285,215) ;
  %Straight Lines [id:da6458508857548457] 
  \draw [color={rgb, 255:red, 142; green, 216; blue, 248 }  ,draw opacity=1 ]   (290,205) -- (290,210) ;
  %Straight Lines [id:da389081993863875] 
  \draw [color={rgb, 255:red, 142; green, 216; blue, 248 }  ,draw opacity=1 ]   (295,210) -- (295,215) ;
  %Straight Lines [id:da4698331801209108] 
  \draw [color={rgb, 255:red, 142; green, 216; blue, 248 }  ,draw opacity=1 ]   (280,175) -- (280,180) ;
  %Straight Lines [id:da25806935831367994] 
  \draw [color={rgb, 255:red, 142; green, 216; blue, 248 }  ,draw opacity=1 ]   (285,180) -- (285,185) ;
  %Straight Lines [id:da6303963191886901] 
  \draw [color={rgb, 255:red, 142; green, 216; blue, 248 }  ,draw opacity=1 ]   (290,175) -- (290,180) ;
  %Straight Lines [id:da8058252545225999] 
  \draw [color={rgb, 255:red, 142; green, 216; blue, 248 }  ,draw opacity=1 ]   (295,180) -- (295,185) ;
  %Straight Lines [id:da41335487690708383] 
  \draw [color={rgb, 255:red, 142; green, 216; blue, 248 }  ,draw opacity=1 ]   (280,185) -- (280,190) ;
  %Straight Lines [id:da8633769129467583] 
  \draw [color={rgb, 255:red, 142; green, 216; blue, 248 }  ,draw opacity=1 ]   (285,190) -- (285,195) ;
  %Straight Lines [id:da9798026448186634] 
  \draw [color={rgb, 255:red, 142; green, 216; blue, 248 }  ,draw opacity=1 ]   (290,185) -- (290,190) ;
  %Straight Lines [id:da024069444268022044] 
  \draw [color={rgb, 255:red, 142; green, 216; blue, 248 }  ,draw opacity=1 ]   (295,190) -- (295,195) ;
  %Straight Lines [id:da11115710015755886] 
  \draw [color={rgb, 255:red, 142; green, 216; blue, 248 }  ,draw opacity=1 ]   (300,195) -- (300,200) ;
  %Straight Lines [id:da0034173566274477984] 
  \draw [color={rgb, 255:red, 142; green, 216; blue, 248 }  ,draw opacity=1 ]   (305,200) -- (305,205) ;
  %Straight Lines [id:da20435728551083132] 
  \draw [color={rgb, 255:red, 142; green, 216; blue, 248 }  ,draw opacity=1 ]   (310,195) -- (310,200) ;
  %Straight Lines [id:da7546729993720354] 
  \draw [color={rgb, 255:red, 142; green, 216; blue, 248 }  ,draw opacity=1 ]   (315,200) -- (315,205) ;
  %Straight Lines [id:da29136940835025094] 
  \draw [color={rgb, 255:red, 142; green, 216; blue, 248 }  ,draw opacity=1 ]   (300,205) -- (300,210) ;
  %Straight Lines [id:da260240529304232] 
  \draw [color={rgb, 255:red, 142; green, 216; blue, 248 }  ,draw opacity=1 ]   (305,210) -- (305,215) ;
  %Straight Lines [id:da9089362475647194] 
  \draw [color={rgb, 255:red, 142; green, 216; blue, 248 }  ,draw opacity=1 ]   (310,205) -- (310,210) ;
  %Straight Lines [id:da5279985875925712] 
  \draw [color={rgb, 255:red, 142; green, 216; blue, 248 }  ,draw opacity=1 ]   (315,210) -- (315,215) ;
  %Straight Lines [id:da019203355274531653] 
  \draw [color={rgb, 255:red, 142; green, 216; blue, 248 }  ,draw opacity=1 ]   (300,175) -- (300,180) ;
  %Straight Lines [id:da38350473434085086] 
  \draw [color={rgb, 255:red, 142; green, 216; blue, 248 }  ,draw opacity=1 ]   (305,180) -- (305,185) ;
  %Straight Lines [id:da26049019601589496] 
  \draw [color={rgb, 255:red, 142; green, 216; blue, 248 }  ,draw opacity=1 ]   (310,175) -- (310,180) ;
  %Straight Lines [id:da10169294222349134] 
  \draw [color={rgb, 255:red, 142; green, 216; blue, 248 }  ,draw opacity=1 ]   (315,180) -- (315,185) ;
  %Straight Lines [id:da4774013918670772] 
  \draw [color={rgb, 255:red, 142; green, 216; blue, 248 }  ,draw opacity=1 ]   (300,185) -- (300,190) ;
  %Straight Lines [id:da18166056962042387] 
  \draw [color={rgb, 255:red, 142; green, 216; blue, 248 }  ,draw opacity=1 ]   (305,190) -- (305,195) ;
  %Straight Lines [id:da7486340902139925] 
  \draw [color={rgb, 255:red, 142; green, 216; blue, 248 }  ,draw opacity=1 ]   (310,185) -- (310,190) ;
  %Straight Lines [id:da8122007096155761] 
  \draw [color={rgb, 255:red, 142; green, 216; blue, 248 }  ,draw opacity=1 ]   (315,190) -- (315,195) ;
  %Straight Lines [id:da27845691815504225] 
  \draw [color={rgb, 255:red, 142; green, 216; blue, 248 }  ,draw opacity=1 ]   (280,224) -- (280,229) ;
  %Straight Lines [id:da22483615439832927] 
  \draw [color={rgb, 255:red, 142; green, 216; blue, 248 }  ,draw opacity=1 ]   (285,229) -- (285,234) ;
  %Straight Lines [id:da9314600903057564] 
  \draw [color={rgb, 255:red, 142; green, 216; blue, 248 }  ,draw opacity=1 ]   (290,224) -- (290,229) ;
  %Straight Lines [id:da6443981467998208] 
  \draw [color={rgb, 255:red, 142; green, 216; blue, 248 }  ,draw opacity=1 ]   (295,229) -- (295,234) ;
  %Straight Lines [id:da27883780649178425] 
  \draw [color={rgb, 255:red, 142; green, 216; blue, 248 }  ,draw opacity=1 ]   (280,234) -- (280,239) ;
  %Straight Lines [id:da9830125711701108] 
  \draw [color={rgb, 255:red, 142; green, 216; blue, 248 }  ,draw opacity=1 ]   (285,239) -- (285,244) ;
  %Straight Lines [id:da059436680418371246] 
  \draw [color={rgb, 255:red, 142; green, 216; blue, 248 }  ,draw opacity=1 ]   (290,234) -- (290,239) ;
  %Straight Lines [id:da7406688076690306] 
  \draw [color={rgb, 255:red, 142; green, 216; blue, 248 }  ,draw opacity=1 ]   (295,239) -- (295,244) ;
  %Straight Lines [id:da5021144385128589] 
  \draw [color={rgb, 255:red, 142; green, 216; blue, 248 }  ,draw opacity=1 ]   (280,204) -- (280,209) ;
  %Straight Lines [id:da13013399963985517] 
  \draw [color={rgb, 255:red, 142; green, 216; blue, 248 }  ,draw opacity=1 ]   (285,209) -- (285,214) ;
  %Straight Lines [id:da31807111982140657] 
  \draw [color={rgb, 255:red, 142; green, 216; blue, 248 }  ,draw opacity=1 ]   (290,204) -- (290,209) ;
  %Straight Lines [id:da8566541277995903] 
  \draw [color={rgb, 255:red, 142; green, 216; blue, 248 }  ,draw opacity=1 ]   (295,209) -- (295,214) ;
  %Straight Lines [id:da49151551348589684] 
  \draw [color={rgb, 255:red, 142; green, 216; blue, 248 }  ,draw opacity=1 ]   (280,214) -- (280,219) ;
  %Straight Lines [id:da37398838677422996] 
  \draw [color={rgb, 255:red, 142; green, 216; blue, 248 }  ,draw opacity=1 ]   (285,219) -- (285,224) ;
  %Straight Lines [id:da11597922438255104] 
  \draw [color={rgb, 255:red, 142; green, 216; blue, 248 }  ,draw opacity=1 ]   (290,214) -- (290,219) ;
  %Straight Lines [id:da1639408375872644] 
  \draw [color={rgb, 255:red, 142; green, 216; blue, 248 }  ,draw opacity=1 ]   (295,219) -- (295,224) ;
  %Straight Lines [id:da37273686152799934] 
  \draw [color={rgb, 255:red, 142; green, 216; blue, 248 }  ,draw opacity=1 ]   (300,224) -- (300,229) ;
  %Straight Lines [id:da7739729511968407] 
  \draw [color={rgb, 255:red, 142; green, 216; blue, 248 }  ,draw opacity=1 ]   (305,229) -- (305,234) ;
  %Straight Lines [id:da07783085603350193] 
  \draw [color={rgb, 255:red, 142; green, 216; blue, 248 }  ,draw opacity=1 ]   (310,224) -- (310,229) ;
  %Straight Lines [id:da9286536241197245] 
  \draw [color={rgb, 255:red, 142; green, 216; blue, 248 }  ,draw opacity=1 ]   (315,229) -- (315,234) ;
  %Straight Lines [id:da6159451473475648] 
  \draw [color={rgb, 255:red, 142; green, 216; blue, 248 }  ,draw opacity=1 ]   (300,234) -- (300,239) ;
  %Straight Lines [id:da4863729535434409] 
  \draw [color={rgb, 255:red, 142; green, 216; blue, 248 }  ,draw opacity=1 ]   (305,239) -- (305,244) ;
  %Straight Lines [id:da176648234986301] 
  \draw [color={rgb, 255:red, 142; green, 216; blue, 248 }  ,draw opacity=1 ]   (310,234) -- (310,239) ;
  %Straight Lines [id:da6718808401972798] 
  \draw [color={rgb, 255:red, 142; green, 216; blue, 248 }  ,draw opacity=1 ]   (315,239) -- (315,244) ;
  %Straight Lines [id:da7570212189836263] 
  \draw [color={rgb, 255:red, 142; green, 216; blue, 248 }  ,draw opacity=1 ]   (300,204) -- (300,209) ;
  %Straight Lines [id:da08471876032913417] 
  \draw [color={rgb, 255:red, 142; green, 216; blue, 248 }  ,draw opacity=1 ]   (305,209) -- (305,214) ;
  %Straight Lines [id:da20269011315953023] 
  \draw [color={rgb, 255:red, 142; green, 216; blue, 248 }  ,draw opacity=1 ]   (310,204) -- (310,209) ;
  %Straight Lines [id:da9074082142190854] 
  \draw [color={rgb, 255:red, 142; green, 216; blue, 248 }  ,draw opacity=1 ]   (315,209) -- (315,214) ;
  %Straight Lines [id:da8197724867867013] 
  \draw [color={rgb, 255:red, 142; green, 216; blue, 248 }  ,draw opacity=1 ]   (300,214) -- (300,219) ;
  %Straight Lines [id:da7164179752295212] 
  \draw [color={rgb, 255:red, 142; green, 216; blue, 248 }  ,draw opacity=1 ]   (305,219) -- (305,224) ;
  %Straight Lines [id:da15678182924124595] 
  \draw [color={rgb, 255:red, 142; green, 216; blue, 248 }  ,draw opacity=1 ]   (310,214) -- (310,219) ;
  %Straight Lines [id:da3956755372041971] 
  \draw [color={rgb, 255:red, 142; green, 216; blue, 248 }  ,draw opacity=1 ]   (315,219) -- (315,224) ;
  %Straight Lines [id:da8936547547943101] 
  \draw [color={rgb, 255:red, 142; green, 216; blue, 248 }  ,draw opacity=1 ]   (320,210) -- (320,228) ;
  \draw [shift={(320,230)}, rotate = 270] [color={rgb, 255:red, 142; green, 216; blue, 248 }  ,draw opacity=1 ][line width=0.75]    (10.93,-3.29) .. controls (6.95,-1.4) and (3.31,-0.3) .. (0,0) .. controls (3.31,0.3) and (6.95,1.4) .. (10.93,3.29)   ;
  
  % Text Node
  \draw (139,188.4) node [anchor=north east] [inner sep=0.75pt]  [font=\footnotesize]  {$W_{B}( y)$};
  % Text Node
  \draw (233,251.6) node [anchor=south west] [inner sep=0.75pt]  [font=\footnotesize]  {$y$};
  % Text Node
  \draw (211,216.6) node [anchor=south] [inner sep=0.75pt]  [font=\scriptsize]  {$W_{W} =\text{cap}_{B}$};
  % Text Node
  \draw (134,256.6) node [anchor=south east] [inner sep=0.75pt]  [font=\scriptsize]  {$y=0$};
  % Text Node
  \draw (213,263.4) node [anchor=north west][inner sep=0.75pt]  [font=\scriptsize]  {$y=\text{top}$};
  % Text Node
  \draw (181,263.4) node [anchor=north] [inner sep=0.75pt]  [font=\scriptsize]  {$y=5$};
  % Text Node
  \draw (322,233.4) node [anchor=north west][inner sep=0.75pt]  [font=\scriptsize,color={rgb, 255:red, 142; green, 216; blue, 248 }  ,opacity=1 ]  {$2N/s$};
  
  
  \end{tikzpicture}
  \end{center}
  \vspace*{-0.8em}
By using a linear relation we can see that the water's weight is 
$$
\left\lbrace
\begin{aligned}
  &W_W(y)=\frac{10-0}{5-0}y+b=2y,\ 0\leq y\leq 5\\
  &W_W(y)=10,\ 5\leq y\leq 8\\
\end{aligned}
\right.
$$
Finally the work is 
$$W=\underbrace{\int_0^{8} 12\dd y}_{\text{box}}+\underbrace{\int_0^{8}4(8-y)\dd y}_{\text{rope}}+\underbrace{\int_0^{5}2y\dd y+\int_5^{8}10\dd y}_{\text{water}}=\un{279\ \si\joule}.$$
\end{Ex}
\vspace*{-1em}
\begin{ptcbP}
Instead of raining, the box is leaking $(2\ \si{\newton\per\second})$, what is the work required to raise it?
\end{ptcbP}

\subsection*{Mass}

Recall that the density $\rho$ of a solid with mass $m$ and volume $V$ is given by $\rho= m/V$. In our planar case we will consider area $A$ instead of volume $V$. This means that $\rho=m/A$, and in consequence $m=\rho A$.\par 
The problem of finding the mass of an object with \emph{variable density} can be solved by integrating.

\begin{Ex}
  Suppose we have a parabolic plate $y=4-x^2$ bounded by the $x$-axis. 
  \begin{center}
    

% Pattern Info
 
\tikzset{
  pattern size/.store in=\mcSize, 
  pattern size = 5pt,
  pattern thickness/.store in=\mcThickness, 
  pattern thickness = 0.3pt,
  pattern radius/.store in=\mcRadius, 
  pattern radius = 1pt}
  \makeatletter
  \pgfutil@ifundefined{pgf@pattern@name@_6zsndcugr}{
  \pgfdeclarepatternformonly[\mcThickness,\mcSize]{_6zsndcugr}
  {\pgfqpoint{0pt}{0pt}}
  {\pgfpoint{\mcSize}{\mcSize}}
  {\pgfpoint{\mcSize}{\mcSize}}
  {
  \pgfsetcolor{\tikz@pattern@color}
  \pgfsetlinewidth{\mcThickness}
  \pgfpathmoveto{\pgfqpoint{0pt}{\mcSize}}
  \pgfpathlineto{\pgfpoint{\mcSize+\mcThickness}{-\mcThickness}}
  \pgfpathmoveto{\pgfqpoint{0pt}{0pt}}
  \pgfpathlineto{\pgfpoint{\mcSize+\mcThickness}{\mcSize+\mcThickness}}
  \pgfusepath{stroke}
  }}
  \makeatother
  \tikzset{every picture/.style={line width=0.75pt}} %set default line width to 0.75pt        
  
  \begin{tikzpicture}[x=0.75pt,y=0.75pt,yscale=-1,xscale=1]
  %uncomment if require: \path (0,300); %set diagram left start at 0, and has height of 300
  
  %Straight Lines [id:da01403342387511397] 
  \draw    (100,200) -- (200,200) ;
  %Shape: Parabola [id:dp623511170858644] 
  \draw  [pattern=_6zsndcugr,pattern size=7.5pt,pattern thickness=0.75pt,pattern radius=0pt, pattern color={rgb, 255:red, 0; green, 0; blue, 0}] (120,200) .. controls (140,120) and (160,120) .. (180,200) ;
  %Straight Lines [id:da22173840233412623] 
  \draw [color={rgb, 255:red, 0; green, 93; blue, 164 }  ,draw opacity=1 ]   (212,200) -- (308,200) ;
  \draw [shift={(310,200)}, rotate = 180] [color={rgb, 255:red, 0; green, 93; blue, 164 }  ,draw opacity=1 ][line width=0.75]    (10.93,-3.29) .. controls (6.95,-1.4) and (3.31,-0.3) .. (0,0) .. controls (3.31,0.3) and (6.95,1.4) .. (10.93,3.29)   ;
  \draw [shift={(210,200)}, rotate = 0] [color={rgb, 255:red, 0; green, 93; blue, 164 }  ,draw opacity=1 ][line width=0.75]    (10.93,-3.29) .. controls (6.95,-1.4) and (3.31,-0.3) .. (0,0) .. controls (3.31,0.3) and (6.95,1.4) .. (10.93,3.29)   ;
  %Shape: Parabola [id:dp9632171600419536] 
  \draw   (230,200) .. controls (250,120) and (270,120) .. (290,200) ;
  %Straight Lines [id:da9535989798691262] 
  \draw [color={rgb, 255:red, 0; green, 93; blue, 164 }  ,draw opacity=1 ]   (260,200) -- (260,122) ;
  \draw [shift={(260,120)}, rotate = 90] [color={rgb, 255:red, 0; green, 93; blue, 164 }  ,draw opacity=1 ][line width=0.75]    (10.93,-3.29) .. controls (6.95,-1.4) and (3.31,-0.3) .. (0,0) .. controls (3.31,0.3) and (6.95,1.4) .. (10.93,3.29)   ;
  %Straight Lines [id:da7982538020098101] 
  \draw    (260,140) -- (270,140) ;
  %Straight Lines [id:da9664445066717691] 
  \draw    (260,200) -- (270,190) ;
  %Straight Lines [id:da018976009185489473] 
  \draw    (285,205) -- (290,200) ;
  %Straight Lines [id:da9822517712817638] 
  \draw    (225,205) -- (230,200) ;
  
  % Text Node
  \draw (272,136.6) node [anchor=south west] [inner sep=0.75pt]  [font=\scriptsize]  {$y=4$};
  % Text Node
  \draw (270,186.6) node [anchor=south] [inner sep=0.75pt]  [font=\tiny]  {$( 0,0)$};
  % Text Node
  \draw (227,208.4) node [anchor=north west][inner sep=0.75pt]  [font=\tiny]  {$x=-2$};
  % Text Node
  \draw (287,208.4) node [anchor=north west][inner sep=0.75pt]  [font=\tiny]  {$x=2$};
  
  
  \end{tikzpicture}
  
  \end{center}
  If the plate's density was $\rho(x)=2x+1$ then 
  \begin{align*}
    m&=\int_{-2}^2\rho(x)(\text{up}-\text{down})\dd x\\
    &=\int_{-2}^2(2x+1)\lbrack(4-x^2)-0\rbrack \dd x=32/3.
  \end{align*}
\end{Ex}

\begin{Rmk}
  We know that 
  $$\text{Area}=\text{width}\.\text{height}.$$
  Sometimes we will use up-down orientation, where we will use 
  $$\text{Area}=\dd x\.(\text{up}-\text{down})=(g(x)-f(x))\dd x,$$
  where $g(x)\geq f(x)$.\par 
  Or in the other case
  $$\text{Area}=(\text{right}-\text{left})\.\dd y=(g(y)-f(y))\dd y$$
  where $g(y)\geq f(y)$.
\end{Rmk}

\begin{Ex}
  Suppose that our plate is bounded by the curve $x=\cos(y)$ and the $y$-axis. If the plate has density $\rho(y)=y+1$, then we can find its mass by setting up the integral in the other way.
  \begin{center}
    

% Pattern Info
 
\tikzset{
  pattern size/.store in=\mcSize, 
  pattern size = 5pt,
  pattern thickness/.store in=\mcThickness, 
  pattern thickness = 0.3pt,
  pattern radius/.store in=\mcRadius, 
  pattern radius = 1pt}
  \makeatletter
  \pgfutil@ifundefined{pgf@pattern@name@_bzkjfgaxs}{
  \pgfdeclarepatternformonly[\mcThickness,\mcSize]{_bzkjfgaxs}
  {\pgfqpoint{0pt}{0pt}}
  {\pgfpoint{\mcSize}{\mcSize}}
  {\pgfpoint{\mcSize}{\mcSize}}
  {
  \pgfsetcolor{\tikz@pattern@color}
  \pgfsetlinewidth{\mcThickness}
  \pgfpathmoveto{\pgfqpoint{0pt}{\mcSize}}
  \pgfpathlineto{\pgfpoint{\mcSize+\mcThickness}{-\mcThickness}}
  \pgfpathmoveto{\pgfqpoint{0pt}{0pt}}
  \pgfpathlineto{\pgfpoint{\mcSize+\mcThickness}{\mcSize+\mcThickness}}
  \pgfusepath{stroke}
  }}
  \makeatother
  \tikzset{every picture/.style={line width=0.75pt}} %set default line width to 0.75pt        
  
  \begin{tikzpicture}[x=0.75pt,y=0.75pt,yscale=-1,xscale=1]
  %uncomment if require: \path (0,300); %set diagram left start at 0, and has height of 300
  
  %Straight Lines [id:da12651440644236667] 
  \draw    (140,140) -- (140,240) ;
  %Shape: Parabola [id:dp7666003413147079] 
  \draw  [pattern=_bzkjfgaxs,pattern size=7.5pt,pattern thickness=0.75pt,pattern radius=0pt, pattern color={rgb, 255:red, 0; green, 0; blue, 0}] (140,160) .. controls (166.67,180) and (166.67,200) .. (140,220) ;
  %Straight Lines [id:da725213386458611] 
  \draw [color={rgb, 255:red, 0; green, 93; blue, 164 }  ,draw opacity=1 ]   (250,142) -- (250,238) ;
  \draw [shift={(250,240)}, rotate = 270] [color={rgb, 255:red, 0; green, 93; blue, 164 }  ,draw opacity=1 ][line width=0.75]    (10.93,-3.29) .. controls (6.95,-1.4) and (3.31,-0.3) .. (0,0) .. controls (3.31,0.3) and (6.95,1.4) .. (10.93,3.29)   ;
  \draw [shift={(250,140)}, rotate = 90] [color={rgb, 255:red, 0; green, 93; blue, 164 }  ,draw opacity=1 ][line width=0.75]    (10.93,-3.29) .. controls (6.95,-1.4) and (3.31,-0.3) .. (0,0) .. controls (3.31,0.3) and (6.95,1.4) .. (10.93,3.29)   ;
  %Shape: Parabola [id:dp12165755900306918] 
  \draw   (250,160) .. controls (276.9,179.69) and (277.13,199.69) .. (250.7,220) ;
  %Straight Lines [id:da9650170220417672] 
  \draw [color={rgb, 255:red, 0; green, 93; blue, 164 }  ,draw opacity=1 ]   (250,190) -- (298,190) ;
  \draw [shift={(300,190)}, rotate = 180] [color={rgb, 255:red, 0; green, 93; blue, 164 }  ,draw opacity=1 ][line width=0.75]    (10.93,-3.29) .. controls (6.95,-1.4) and (3.31,-0.3) .. (0,0) .. controls (3.31,0.3) and (6.95,1.4) .. (10.93,3.29)   ;
  %Straight Lines [id:da8789159890287226] 
  \draw    (245,215) -- (250.06,219.94) ;
  %Straight Lines [id:da5047388227408768] 
  \draw    (245,155) -- (250.06,159.94) ;
  
  % Text Node
  \draw (243,151.6) node [anchor=south east] [inner sep=0.75pt]  [font=\tiny]  {$y=\pi /2$};
  % Text Node
  \draw (142,143.4) node [anchor=north west][inner sep=0.75pt]  [font=\scriptsize]  {$x=0$};
  % Text Node
  \draw (162,186.6) node [anchor=south west] [inner sep=0.75pt]  [font=\scriptsize]  {$x=\cos( y)$};
  % Text Node
  \draw (243.33,209.93) node [anchor=south east] [inner sep=0.75pt]  [font=\tiny]  {$y=-\pi /2$};
  
  
  \end{tikzpicture}
  
  \end{center}
  In this case 
  $$
  \left\lbrace
  \begin{aligned}
    &\text{left}= x\ \text{axis} = 0,\\
    &\text{right}=\text{curve} = \cos(y)
  \end{aligned}
  \right.
  $$
  the $\dd y$ is the height which goes from $-\pi/2$ to $\pi/2$ and so 
  $$m=\int_{-\pi/2}^{\pi/2}\rho(y)\cos(y)\dd y=\int_{-\pi/2}^{\pi/2}(y+1)\cos(y)\dd y=2.$$
\end{Ex}

It might be the case that our plate might not have an axis as its boundary. Consider the following example:

\begin{Ex}
  Let us find the mass of the plate with density $\rho(x)=6-20x$ bounded by the equations
  $$
  \left\lbrace
  \begin{aligned}
    &f(x)=x+2,\\
    &g(x)=-x^2+x+6.
  \end{aligned}
  \right.
  $$
  \begin{center}
   

% Pattern Info
 
\tikzset{
  pattern size/.store in=\mcSize, 
  pattern size = 5pt,
  pattern thickness/.store in=\mcThickness, 
  pattern thickness = 0.3pt,
  pattern radius/.store in=\mcRadius, 
  pattern radius = 1pt}\makeatletter
  \pgfutil@ifundefined{pgf@pattern@name@_rw7xxjw74}{
  \pgfdeclarepatternformonly[\mcThickness,\mcSize]{_rw7xxjw74}
  {\pgfqpoint{-\mcThickness}{-\mcThickness}}
  {\pgfpoint{\mcSize}{\mcSize}}
  {\pgfpoint{\mcSize}{\mcSize}}
  {\pgfsetcolor{\tikz@pattern@color}
  \pgfsetlinewidth{\mcThickness}
  \pgfpathmoveto{\pgfpointorigin}
  \pgfpathlineto{\pgfpoint{\mcSize}{0}}
  \pgfpathmoveto{\pgfpointorigin}
  \pgfpathlineto{\pgfpoint{0}{\mcSize}}
  \pgfusepath{stroke}}}
  \makeatother
  \tikzset{every picture/.style={line width=0.75pt}} %set default line width to 0.75pt        
  
  \begin{tikzpicture}[x=0.75pt,y=0.75pt,yscale=-1,xscale=1]
  %uncomment if require: \path (0,300); %set diagram left start at 0, and has height of 300
  
  %Straight Lines [id:da04112332353175807] 
  \draw    (290,194.99) -- (295,189.99) ;
  %Straight Lines [id:da7142455523953686] 
  \draw    (200,189.99) -- (265,124.98) ;
  %Shape: Parabola [id:dp025976677821947014] 
  \draw  [pattern=_rw7xxjw74,pattern size=6pt,pattern thickness=0.75pt,pattern radius=0pt, pattern color={rgb, 255:red, 0; green, 0; blue, 0}] (265,124.98) .. controls (234.92,98.9) and (213.25,120.58) .. (200,189.99) ;
  %Straight Lines [id:da39841673674945366] 
  \draw [color={rgb, 255:red, 0; green, 93; blue, 164 }  ,draw opacity=1 ]   (290,190) -- (370,190) ;
  %Straight Lines [id:da429189967737566] 
  \draw    (295,189.99) -- (360,124.98) ;
  %Shape: Parabola [id:dp1986384203660576] 
  \draw   (360,124.98) .. controls (329.92,98.9) and (308.25,120.58) .. (295,189.99) ;
  %Straight Lines [id:da6366955993003761] 
  \draw [color={rgb, 255:red, 0; green, 93; blue, 164 }  ,draw opacity=1 ]   (330,190) -- (330,102) ;
  \draw [shift={(330,100)}, rotate = 90] [color={rgb, 255:red, 0; green, 93; blue, 164 }  ,draw opacity=1 ][line width=0.75]    (10.93,-3.29) .. controls (6.95,-1.4) and (3.31,-0.3) .. (0,0) .. controls (3.31,0.3) and (6.95,1.4) .. (10.93,3.29)   ;
  %Straight Lines [id:da22226597836229933] 
  \draw  [dash pattern={on 4.5pt off 4.5pt}]  (360,124.98) -- (360,195) ;
  
  % Text Node
  \draw (234.5,160.88) node [anchor=north west][inner sep=0.75pt]  [font=\scriptsize]  {$f( x)$};
  % Text Node
  \draw (224.19,118.27) node [anchor=south east] [inner sep=0.75pt]  [font=\scriptsize]  {$g( x)$};
  % Text Node
  \draw (292,198.39) node [anchor=north west][inner sep=0.75pt]  [font=\scriptsize]  {$x=-2$};
  % Text Node
  \draw (360,198.4) node [anchor=north] [inner sep=0.75pt]  [font=\scriptsize]  {$x=2$};
  
  
  \end{tikzpicture}
  
  
  \end{center}
  To find the limits in the $x$-axis we have to equate both expressions. 
  \begin{align*}
    f(x)=g(x)&\iff x+2=-x^2+x+6\\
    &\iff0=-x^2+4\iff x=\pm 2.
  \end{align*}
  It follows that the integral runs from $-2$ to $2$ and since the upper limit of the plate is $g$ and the lower is $f$, we obtain the following expression for the mass
  \begin{align*}
    m&=\int_{-2}^2\rho(x)(g(x)-f(x))\dd x\\
    &=\int_{-2}^2(6-20x)\lbrack(-x^2+x+6)-(x+2)\rbrack\dd x=64.\\
  \end{align*}
\end{Ex}

\begin{ptcbP}
  Consider a plate bounded by the horizontal lines $y=1$ and $y=4$ and the oblique lines $y=2x+1$ and $y=2x+5$.
  \vspace*{-0.5em}
  \begin{enumerate}[i)]
    \itemsep=-0.4em 
    \item Draw the enclosed region.
    \item If the density of the plate is $\rho(x)=x^2$, find its mass.
    \item Now assume that $\rho$ is in terms of $y$, $\rho(y)=2y-1$. To calculate mass, does the setup of the integral change?
  \end{enumerate}
\end{ptcbP}

\begin{Ej}
Suppose you are pushing a box which weighs $20\ \si\newton$ across a path of length $15\ \si\metre$, at a speed of $2\ \si{\metre\per\second}$. If it's raining at a speed of $4\ \si{\newton\per\second}$ and the box has a capacity of $10\ \si{\newton}$, what is the work done on the box after displacing it?
\end{Ej}

\begin{Ej}
  Find the mass of a plate with density $\rho(x)=1$ bounded below by the curves $y=2x$ and $y=8-2x$, and above by the curve $y=12x-3x^2$.
\end{Ej}
\end{multicols}
\end{document} 