%----------------------------------------------------------------------------------------
%	PACKAGES AND OTHER DOCUMENT CONFIGURATIONS
%----------------------------------------------------------------------------------------

\documentclass[12pt]{article}
\usepackage[spanish]{babel} %Tildes
\usepackage[extreme]{savetrees} %Espaciado e interlineado. Comentar si no gusta el interlineado.
\usepackage[utf8]{inputenc} %Encoding para tildes
\usepackage[breakable,skins]{tcolorbox} %Cajitas
\usepackage{fancyhdr} % Se necesita para el título arriba
\usepackage{lastpage} % Se necesita para poner el número de página
\usepackage{amsmath,amsfonts,amssymb,amsthm} %simbolos y demás
\usepackage{mathabx} %más símbolos
\usepackage{physics} %simbolos de derivadas, bra-ket.
\usepackage{multicol}
\usepackage[customcolors]{hf-tikz}
\usepackage[shortlabels]{enumitem}
\usepackage{tikz}

%\def\darktheme
%%%%%%%%% === Document Configuration === %%%%%%%%%%%%%%

\pagestyle{fancy}
\setlength{\headheight}{14.49998pt} %NO MODIFICAR
\setlength{\footskip}{14.49998pt} %NO MODIFICAR

\ifx \darktheme\undefined

\lhead{Math161S1} % Nombre de autor
\chead{\textbf{Week 3}} % Titulo
\rhead{}%\firstxmark} 
\lfoot{}%\lastxmark}
\cfoot{}
\rfoot{Page \thepage\ of\ \pageref{LastPage}} %A la derecha saldrá pág. 6 de 9. 
\else
\pagenumbering{gobble}
\pagecolor[rgb]{0,0,0}%{0.23,0.258,0.321}
\color[rgb]{1,1,1}
\fi

%%%%%%%%% === My T Color Box === %%%%%%%%%%%%%%

\ifx \darktheme\undefined
\newtcolorbox{ptcb}{
colframe = black,
colback = white,
breakable,
enhanced
}
\newtcolorbox{ptcbP}{
colframe = black,
colback = white,
coltitle = black,
colbacktitle = black!40,
title = Practice,
breakable,
enhanced
}

\else
\newtcolorbox{ptcb}{
colframe = white,
colback = black,
colupper = white,
breakable,
enhanced
}
\newtcolorbox{ptcbP}{
colframe = white,
colback = black,
colupper = white,
coltitle = white,
colbacktitle = black,
title = Practice,
breakable,
enhanced
}
\fi

%%%%%%%%% === Tikz para matrices === %%%%%%%%%%%%%%

\tikzset{
  style green/.style={
    set fill color=green!50!lime!60,
    set border color=white,
  },
  style cyan/.style={
    set fill color=cyan!90!blue!60,
    set border color=white,
  },
  style orange/.style={
    set fill color=orange!80!red!60,
    set border color=white,
  },
  row/.style={
    above left offset={-0.15,0.31},
    below right offset={0.15,-0.125},
    #1
  },
  col/.style={
    above left offset={-0.1,0.3},
    below right offset={0.15,-0.15},
    #1
  }
}

%%%%%%%%% === Theorems and suchlike === %%%%%%%%%%%%%%

\theoremstyle{plain}
\newtheorem{Th}{Theorem}  %%% Theorem 1.1
\newtheorem*{nTh}{Theorem}             %%% No-numbered Theorem
\newtheorem{Prop}[Th]{Proposition}     %%% Proposition 1.2
\newtheorem{Lem}[Th]{Lemma}             %%% Lemma 1.3
\newtheorem*{nLem}{Lemma}               %%% No-numbered Lemma
\newtheorem{Cor}[Th]{Corollary}        %%% Corollary 1.4
\newtheorem*{nCor}{Corollary}          %%% No-numbered Corollary

\theoremstyle{definition}
\newtheorem*{Def}{Definition}       %%% Definition 1.5
\newtheorem*{nonum-Def}{Definition}    %%% No number Definition
\newtheorem*{nEx}{Example}             %%% No number Example
\newtheorem{Ex}[Th]{Example}           %%% Example
\newtheorem{Ej}[Th]{Exercise}         %%% Exercise
\newtheorem*{nEj}{Exercise}           %%% No number Excercise
\newtheorem*{Not}{Notation}       %%% Definition 1.5

\theoremstyle{remark}
\newtheorem*{Rmk}{Remark}      %%%Remark 1.6

%\numberwithin{equation}{section}

\setlength{\parindent}{3ex}

%%====== Useful macros: =======%%%

\DeclareMathOperator{\gen}{gen}     %%%set generated by...
\DeclareMathOperator{\Rng}{Rng}     %%%rangomat
\DeclareMathOperator{\Nul}{Nul}     %%%rangomat
\DeclareMathOperator{\Proy}{Proy}   %%%proyección
\DeclareMathOperator{\id}{id}       %%%identity operator

\newcommand{\al}{\alpha}            %%%short for \alpha
\newcommand{\la}{\lambda}           %%%short for \lambda
\newcommand{\sg}{\sigma}            %%%short for \sigma
\newcommand{\te}{\theta}                %% short for  \theta
\renewcommand{\l}{\ell}

\newcommand{\thickhat}[1]{\mathbf{\hat{\text{$#1$}}}}
\newcommand{\ii}{\vu{\imath}}
\newcommand{\jj}{\vu{\jmath}}
\newcommand{\kk}{\thickhat{k}}

\newcommand{\bC}{\mathbb{C}}        %%%complex numbers
\newcommand{\bN}{\mathbb{N}}        %%%natural numbers
\newcommand{\bP}{\mathbb{P}}        %%%polynomials
\newcommand{\bR}{\mathbb{R}}        %%%real numbers
\newcommand{\bZ}{\mathbb{Z}}        %%%integer numbers
\newcommand{\cB}{\mathcal{B}}       %%%basis
\newcommand{\cC}{\mathcal{C}}       %%%basis
\newcommand{\cM}{\mathcal{M}}       %%%matrix family

\newcommand{\sT}{\mathsf{T}}        %%%traspuesta

\renewcommand{\geq}{\geqslant}      %%%(to save typing)
\renewcommand{\leq}{\leqslant}      %%%(to save typing)
\newcommand{\x}{\times}             %%%product
\renewcommand{\:}{\colon}           %%%colon in  f: A -> B
\newcommand{\isom}{\simeq}              %% isomorfismo

\newcommand{\un}[1]{\underline{#1}}
\newcommand{\half}{\frac12}

\newcommand*{\Cdot}{{\raisebox{-0.25ex}{\scalebox{1.5}{$\cdot$}}}}      %% cdot más grande
\renewcommand{\.}{\Cdot}                %% producto escalar

\newcommand{\twobyone}[2]{\begin{pmatrix} %% 2 x 1 matrix
  #1 \\ #2 \end{pmatrix}}
  \newcommand{\twobytwo}[4]{\begin{pmatrix} %% 2 x 2 matrix
    #1 & #2 \\ #3 & #4 \end{pmatrix}}
    \newcommand{\twobythree}[6]{\begin{pmatrix} %% 2 x 3 matrix
        #1 & #2 & #3\\ #4 & #5 & #6 \end{pmatrix}}
\newcommand{\threebyone}[3]{\begin{pmatrix} %% 3 x 1 matrix
  #1 \\ #2 \\ #3 \end{pmatrix}}
  \newcommand{\threebytwo}[6]{\begin{pmatrix} %% 3 x 1 matrix
    #1 & #2\\ #3 & #4\\ #5&#6 \end{pmatrix}}
\newcommand{\threebythree}[9]{\begin{pmatrix} %% 3 x 3 matrix
  #1 & #2 & #3 \\ #4 & #5 & #6 \\ #7 & #8 & #9 \end{pmatrix}}

\newcommand{\To}{\Rightarrow}

\newcommand{\vaf}{\overrightarrow}

\newcommand{\set}[1]{\{\,#1\,\}}    %% set notation
\newcommand{\Set}[1]{\biggl\{\,#1\,\biggr\}} %% set notation (large)
\newcommand{\red}[1]{\textcolor{red}{#1}}
\newcommand{\blu}[1]{\textcolor{blue}{#1}}

%----------------------------------------------------------------------------------------
%	ARTICLE CONTENTS
%----------------------------------------------------------------------------------------

\begin{document}
\begin{multicols}{2}
\section*{Partial Fractions}
Integrals involving quadratic polynomials can be solved in many different ways, not only trigonometric substitution. The partial fractions method involves separating a polynomial into its \un{irreducible} factors.
\begin{Ex} 
Consider the integral 
$$\int\frac{\dd x}{(x-5)(x-4)}.$$
We will split the integrand into a couple of fractions whose denominators are the factors of the original denominator.
$$\frac{1}{(x-5)(x-4)}=\frac{A}{x-5}+\frac{B}{x-4},$$
where $A,B$ are constants. To find these constants we multiply both side of the equation by the original denominator $(x-5)(x-4)$:
\begin{align*}
  1=\frac{(x-5)(x-4)}{(x-5)(x-4)}&=\frac{A(x-5)(x-4)}{x-5}+\frac{B(x-5)(x-4)}{x-4}\\
  &=A(x-4)+B(x-5).
\end{align*}
Expanding the expression in the right we obtain the equality
$$1=Ax-4A+Bx-5B\To 1=(A+B)x+(-4A-5B)$$
and since the expression in the left is also a polynomial we can equate coefficients.
$$
\left\lbrace
\begin{aligned}
  &1=-4A-5B\\
  &0=A+B
\end{aligned}
\right.\To
\left\lbrace
\begin{aligned}
  &1=-4A-5B\\
  &B=-A
\end{aligned}
\right.
$$
We can replace the second equation into the first one
$$1=-4A-5(-A)\To\un{1=A}\To \un{B=-1}.$$
Thus it follows that 
$$\frac{1}{(x-5)(x-4)}=\frac{(1)}{x-5}+\frac{(-1)}{x-4},$$
and we can separate this expression using linearity of the integral:
\begin{align*}
  \int\frac{\dd x}{(x-5)(x-4)}&=\int\left(\frac{1}{x-5}+\frac{-1}{x-4}\right)\dd x\\
  &=\int\frac{\dd x}{x-5}-\int\frac{\dd x}{x-4}\\
  &=\un{\log(x-5)-\log(x-4)}
\end{align*}
\end{Ex}

We will not always find ourselves with a factored polynomial. Consider the following example:

\begin{Ex}
To compute the integral of $\frac{x+1}{(x-1)(x-2)}$ we will once again separate into partial fractions. Since the factors of the denominator are linear, we will separate into two fractions:
\begin{align*}
  &\frac{x+1}{(x-1)(x-2)}=\frac{A}{x-1}+\frac{B}{x-2}\\
  \To&x+1=A(x-2)+B(x-1)\\
  \To&x+1=(A+B)x+(-2A-B).
\end{align*}
This time the linear coefficient on the left hand side is \textbf{not zero}. Anyways equating coefficients is analogous
$$
\left\lbrace
\begin{aligned}
  &1=-2A-B\\
  &1=A+B
\end{aligned}
\right.\To
\left\lbrace
\begin{aligned}
  &1=-2A-B\\
  &B=1-A
\end{aligned}
\right.\To
1=-2A-(1-A)
$$
We solve to obtain $\un{A=-2}$ and $\un{B=3}$.\par 
Replacing in the partial fraction decomposition we obtain 
$$\frac{x+1}{(x-1)(x-2)}=\frac{(-2)}{x-1}+\frac{(3)}{x-2}.$$
As we did before we can separate the integral by linearity
\begin{align*}
  \int\frac{x+1}{(x-1)(x-2)}\dd x&=\int\left(\frac{-2}{x-1}+\frac{3}{x-2}\right)\dd x\\
  &=\int\frac{-2}{x-1}\dd x+\int\frac{3}{x-2}\dd x\\
  &=\un{-2\log(x-1)+3\log(x-2)}
\end{align*}
\end{Ex}

It can also occur that the polynomial in denominator is not factored. In that case, we have to factor and then decompose into partial fractions.

\begin{Ex}
Consider the rational function 
$$f(x)=\frac{3x^2+2x+1}{x^3-6x^2+11x-6}$$ 
To separate, we have to factor the denominator into irreducibles and so we use the rational roots theorem. \par
The rational roots of the denominator are in the set 
  $$R=\set{\pm 1,\pm 2,\pm 3},$$
  and after trying $x=1$ we see that it is indeed a root. Thus 
  \begin{align*}
    x^3-6x^2+11x-6&=(x-1)(x^2-5x+6)\\
    &=(x-1)(x-2)(x-3)
  \end{align*}
  We can now separate into partial fractions
  \begin{align*}
    &\frac{3x^2+2x+1}{(x-1)(x-2)(x-3)}=\frac{A}{x-1}+\frac{B}{x-2}+\frac{C}{x-3}.
  \end{align*}
  Cross multiplying the right hand side results in 
  \begin{gather*}
    A(x-2)(x-3)+B(x-1)(x-3)+C(x-1)(x-2)\\
    =(A+B+C)x^2+(-5A-4B-3C)x+(6A+3B+2C)
  \end{gather*}
  We can now collect coefficients and mount our system of equations 
  $$
\left\lbrace
\begin{aligned}
  &3=A+B+C\\
  &2=-5A-4B-3C\\
  &1=6A+3B+2C
\end{aligned}
\right.\To
\left\lbrace
\begin{aligned}
  &A=3\\
  &B=-17\\
  &C=17
\end{aligned}
\right.
$$
We can now separate into our partial fractions to obtain 
$$\int f(x)\dd x=\un{3\log(x-1)-17\log(x-2)+17\log(x-3)}.$$
\end{Ex}
\newpage
\subsection*{Higher Degrees and Repeated Roots}
Let us summarize the terms in the partial fraction decomposition according the possible cases:\par
\begin{ptcb}
\begin{center}
  \begin{tabular}{l|l}
    \vspace*{1mm}Denominator & Partial Fraction Decomposition \\
    \vspace*{1mm}
    $ax+b$ & $\displaystyle\frac{A}{ax+b}$ \\
    \vspace*{1mm}
    $(ax+b)^n$ & $\displaystyle\frac{A_1}{ax+b}+\frac{A_2}{(ax+b)^2}+\dots+\frac{A_n}{(ax+b)^n}$ \\
    \vspace*{1mm}
    $ax^2+bx+c$&  $\displaystyle\frac{Ax+B}{ax^2+bx+c}$ \\
    \vspace*{1mm}
    $(ax^2+bx+c)^n$&  $\displaystyle\frac{A_1x+B_1}{ax^2+bx+c}+\dots+\frac{A_nx+B_n}{(ax^2+bx+c)^n}$
    \end{tabular}
\end{center}
\end{ptcb}
  \begin{Ex}
    Let us decompose $f(x)=\frac{x^2-4x+12}{(x-2)^2(x-4)}$.
    $$\frac{x^2-4x+12}{(x-2)^2(x-4)}=\frac{A}{x-2}+\frac{B}{(x-2)^2}+\frac{C}{x-4}$$
    according to our table since we have a repeated root. 
    Collecting terms on the right hand side we get the expression
    \begin{align*}
      &A(x-2)(x-4)+B(x-4)+C(x-2)^2\\
      =&(A+C)x^2+(-6A+B-4C)x+(8A-4B+4C)
    \end{align*}
    Equating coefficients we get the system 
    $$
\left\lbrace
\begin{aligned}
  &1=A+C\\
  &-4=-6A+B-4C\\
  &12=8A-4B+4C
\end{aligned}
\right.\To
\left\lbrace
\begin{aligned}
  &A=-2\\
  &B=4\\
  &C=3
\end{aligned}
\right.
$$
Thus it follows that 
$$\frac{x^2-4x+12}{(x-2)^2(x-4)}=\frac{-2}{x-2}+\frac{4}{(x-2)^2}+\frac{3}{x-4}.$$
If we wanted to integrate $f$, we could use the linearity on the integral to get 
$$\int f(x)\dd x=\un{-2\log(x-2)+\frac{4}{x-2}+3\log(x-4)}.$$
  \end{Ex}
In case we have an irreducible quadratic, it's necessary to separate according to the $3^{\text{rd}}$ and $4^{\text{th}}$ rows of the table.
\begin{Ex}
  Let $f(x)=\frac{5x^3+x^2-x+7}{(x-2)(x^2+x+1)^2}$, we decompose into partial fractions:
  $$f(x)=\frac{A}{x-2}+\frac{Bx+C}{x^2+x+1}+\frac{Dx+E}{(x^2+x+1)^2}.$$
  Multiplying by $(x-2)(x^2+x+1)^2$ we get the expression
  \vspace*{-0.9em}
  \begin{center}
    \resizebox{0.91\hsize}{!}{
      $A(x^2+x+1)^2+(Bx+C)(x-2)(x^2+x+1)+(Dx+E)(x-2)$   
            }
  \end{center}
  \vspace*{-0.9em}
 After expanding and collecting in terms of $x$ we obtain
 $$
 \left\lbrace
 \begin{aligned}
   &0=A+B\\
   &5=2A-B+C\\
   &1=3A-B-C+D\\
   &-1=2A-2B-C-2D+E\\
   &7=A-2C-2E
 \end{aligned}
 \right.\To
 \left\lbrace
 \begin{aligned}
   &A=1\\
   &B=-1\\
   &C=2\\
   &D=-1\\
   &E=-5
 \end{aligned}
 \right.
 $$
 This means that 
 $$f(x)=\frac{1}{x-2}+\frac{-x+2}{x^2+x+1}+\frac{-x-5}{(x^2+x+1)^2}.$$
 Let us integrate the quadratic fractions:
 \begin{align*}
  \int\frac{-x+2}{x^2+x+1}\dd x&=\int\frac{-2x+4\textbf{-5+5}}{2(x^2+x+1)}\dd x\\
  &=-\frac{1}{2}\int\frac{2x+1}{x^2+x+1}\dd x+\frac{5}{2}\int\frac{\dd x}{x^2+x+1}\\
  &=\un{-\frac{1}{2}\log(x^2+x+1)+\frac{5}{\sqrt{3}}\arctan\left(\frac{2x+1}{\sqrt{3}}\right)}
 \end{align*}
 We deal with other one in a similar manner
 \vspace*{0.2em}
 \begin{align*}
  \int\frac{-x-5}{(x^2+x+1)^2}\dd x&=\int\frac{-2x-10\textbf{+9-9}}{2(x^2+x+1)^2}\dd x\\
  &=\frac{1}{2}\int\frac{-(2x+1)}{(x^2+x+1)^2}\dd x-\frac{9}{2}\int\frac{\dd x}{(x^2+x+1)^2}\\
  &=\un{-\frac{1}{2(x^2+x+1)}}-\frac{9}{2}I.
 \end{align*}
 \vspace*{0.2em}
 For the last integral $I$ we shall complete the square
 $$x^2+x+1=x^2+x+\frac{1}{4}+\frac{3}{4}=\left(x+\frac{1}{2}\right)^2+\frac{3}{4}.$$
 With the trigonometric substitution $x+\frac{1}{2}=\frac{\sqrt{3}}{2}\tan(\te)$ we can simplify the integral into  \begin{align*}
  I&=\int\frac{(\sqrt{3}/2)\sec^2(\te)\dd\te}{(9/16)\sec^4(\te)}=\frac{8\sqrt{3}}{9}\int\cos^2(\te)\dd\te\\
  &=\frac{8\sqrt{3}}{9}\int\left(\frac{1+\cos(2\te)}{2}\right)\dd\te=\frac{8\sqrt{3}}{9}\left(\frac{\te}{2}+\frac{\sin(2\te)}{4}\right)\\
  &=\un{\frac{4\sqrt{3}}{9}\arctan\left(\frac{2x+1}{\sqrt{3}}\right)+\frac{2x+1}{3(x^2+x+1)}}
 \end{align*}
 The result of this integral is the collected sum of all the underlined terms.
\end{Ex}
\begin{Ej}
Compute the following integrals using their partial fraction decomposition. \emph{In some of the integrals it might be possible to simplify the fraction first.}
\vspace*{-0.4em}
\begin{multicols*}{2}
  \begin{enumerate}[i)]
    \itemsep=-0.2em
    \item \resizebox{0.75\hsize}{!}{$\displaystyle\int\frac{2u^2+3u+1}{(u^2+2u+5)(u^2+3u+2)}\dd u$}
    \item $\displaystyle\int\frac{8}{{3x^3+7x^2+4x}}\dd x$
    \item $\displaystyle\int\frac{t^2+2t-8}{t^3-6t^2+4t-24}\dd t$
    \item $\displaystyle\int\frac{4x-3}{{x^3-3x^2}}\dd x$
    \item $\displaystyle\int\frac{2s-2}{s^4-1}\dd s$
    \item \resizebox{0.8\hsize}{!}{$\displaystyle\int\frac{3x^5-4x^4+x^3+x^2-24x-2}{(x-1)^2(x^2+4)^2}\dd x$}
  \end{enumerate}
\end{multicols*}

\end{Ej}
\end{multicols}
\end{document} 