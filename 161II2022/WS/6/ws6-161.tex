%----------------------------------------------------------------------------------------
%	PACKAGES AND OTHER DOCUMENT CONFIGURATIONS
%----------------------------------------------------------------------------------------

\documentclass[12pt]{article}
%\usepackage[spanish]{babel} %Tildes
\usepackage[extreme]{savetrees} %Espaciado e interlineado. Comentar si no gusta el interlineado.
\usepackage[utf8]{inputenc} %Encoding para tildes
\usepackage[breakable,skins]{tcolorbox} %Cajitas
\usepackage{fancyhdr} % Se necesita para el título arriba
\usepackage{lastpage} % Se necesita para poner el número de página
\usepackage{amsmath,amsfonts,amssymb,amsthm} %simbolos y demás
\usepackage{mathabx} %más símbolos
\usepackage{physics} %simbolos de derivadas, bra-ket.
\usepackage{multicol}
\usepackage[customcolors]{hf-tikz}
\usepackage[shortlabels]{enumitem}
\usepackage{tikz}
\usetikzlibrary{patterns}
\usepackage{siunitx}

%\def\darktheme
%%%%%%%%% === Document Configuration === %%%%%%%%%%%%%%

\pagestyle{fancy}
\setlength{\headheight}{14.49998pt} %NO MODIFICAR
\setlength{\footskip}{14.49998pt} %NO MODIFICAR

\ifx \darktheme\undefined

\lhead{Math161S1} % Nombre de autor
\chead{\textbf{Worksheet 6}} % Titulo
\rhead{}%\firstxmark} 
\lfoot{}%\lastxmark}
\cfoot{}
\rfoot{Page \thepage\ of\ \pageref{LastPage}} %A la derecha saldrá pág. 6 de 9. 
\else
\pagenumbering{gobble}
\pagecolor[rgb]{0,0,0}%{0.23,0.258,0.321}
\color[rgb]{1,1,1}
\fi

%%%%%%%%% === My T Color Box === %%%%%%%%%%%%%%

\ifx \darktheme\undefined
\newtcolorbox{ptcb}{
colframe = black,
colback = white,
breakable,
enhanced
}
\newtcolorbox{ptcbP}{
colframe = black,
colback = white,
coltitle = black,
colbacktitle = black!40,
title = Practice,
breakable,
enhanced
}

\else
\newtcolorbox{ptcb}{
colframe = white,
colback = black,
colupper = white,
breakable,
enhanced
}
\newtcolorbox{ptcbP}{
colframe = white,
colback = black,
colupper = white,
coltitle = white,
colbacktitle = black,
title = Practice,
breakable,
enhanced
}
\fi

%%%%%%%%% === Tikz para matrices === %%%%%%%%%%%%%%

\tikzset{
  style green/.style={
    set fill color=green!50!lime!60,
    set border color=white,
  },
  style cyan/.style={
    set fill color=cyan!90!blue!60,
    set border color=white,
  },
  style orange/.style={
    set fill color=orange!80!red!60,
    set border color=white,
  },
  row/.style={
    above left offset={-0.15,0.31},
    below right offset={0.15,-0.125},
    #1
  },
  col/.style={
    above left offset={-0.1,0.3},
    below right offset={0.15,-0.15},
    #1
  }
}

%%%%%%%%% === Theorems and suchlike === %%%%%%%%%%%%%%

\theoremstyle{plain}
\newtheorem{Th}{Theorem}  %%% Theorem 1.1
\newtheorem*{nTh}{Theorem}             %%% No-numbered Theorem
\newtheorem{Prop}[Th]{Proposition}     %%% Proposition 1.2
\newtheorem{Lem}[Th]{Lemma}             %%% Lemma 1.3
\newtheorem*{nLem}{Lemma}               %%% No-numbered Lemma
\newtheorem{Cor}[Th]{Corollary}        %%% Corollary 1.4
\newtheorem*{nCor}{Corollary}          %%% No-numbered Corollary

\theoremstyle{definition}
\newtheorem*{Def}{Definition}       %%% Definition 1.5
\newtheorem*{nonum-Def}{Definition}    %%% No number Definition
\newtheorem*{nEx}{Example}             %%% No number Example
\newtheorem{Ex}[Th]{Example}           %%% Example
\newtheorem{Ej}[Th]{Exercise}         %%% Exercise
\newtheorem*{nEj}{Exercise}           %%% No number Excercise
\newtheorem*{Not}{Notation}       %%% Definition 1.5

\theoremstyle{remark}
\newtheorem*{Rmk}{Remark}      %%%Remark 1.6

%\numberwithin{equation}{section}

\setlength{\parindent}{3ex}

%%====== Useful macros: =======%%%

\DeclareMathOperator{\gen}{gen}     %%%set generated by...
\DeclareMathOperator{\Rng}{Rng}     %%%rangomat
\DeclareMathOperator{\Nul}{Nul}     %%%rangomat
\DeclareMathOperator{\Proy}{Proy}   %%%proyección
\DeclareMathOperator{\id}{id}       %%%identity operator

\newcommand{\al}{\alpha}            %%%short for \alpha
\newcommand{\la}{\lambda}           %%%short for \lambda
\newcommand{\sg}{\ \sigma}            %%%short for \ \sigma
\newcommand{\te}{\theta}                %% short for  \theta
\renewcommand{\l}{\ell}

\newcommand{\thickhat}[1]{\mathbf{\hat{\text{$#1$}}}}
\newcommand{\ii}{\vu{\imath}}
\newcommand{\jj}{\vu{\jmath}}
\newcommand{\kk}{\thickhat{k}}

\newcommand{\bC}{\mathbb{C}}        %%%complex numbers
\newcommand{\bN}{\mathbb{N}}        %%%natural numbers
\newcommand{\bP}{\mathbb{P}}        %%%polynomials
\newcommand{\bR}{\mathbb{R}}        %%%real numbers
\newcommand{\bZ}{\mathbb{Z}}        %%%integer numbers
\newcommand{\cB}{\mathcal{B}}       %%%basis
\newcommand{\cC}{\mathcal{C}}       %%%basis
\newcommand{\cM}{\mathcal{M}}       %%%matrix family

\newcommand{\sT}{\mathsf{T}}        %%%traspuesta

\renewcommand{\geq}{\geqslant}      %%%(to save typing)
\renewcommand{\leq}{\leqslant}      %%%(to save typing)
\newcommand{\x}{\times}             %%%product
\renewcommand{\:}{\colon}           %%%colon in  f: A -> B
\newcommand{\isom}{\ \simeq}              %% isomorfismo

\newcommand{\un}[1]{\underline{#1}}
\newcommand{\half}{\frac12}

\newcommand*{\Cdot}{{\raisebox{-0.25ex}{\scalebox{1.5}{$\cdot$}}}}      %% cdot más grande
\renewcommand{\.}{\Cdot}                %% producto escalar

\newcommand{\twobyone}[2]{\begin{pmatrix} %% 2 x 1 matrix
  #1 \\ #2 \end{pmatrix}}
  \newcommand{\twobytwo}[4]{\begin{pmatrix} %% 2 x 2 matrix
    #1 & #2 \\ #3 & #4 \end{pmatrix}}
    \newcommand{\twobythree}[6]{\begin{pmatrix} %% 2 x 3 matrix
        #1 & #2 & #3\\ #4 & #5 & #6 \end{pmatrix}}
\newcommand{\threebyone}[3]{\begin{pmatrix} %% 3 x 1 matrix
  #1 \\ #2 \\ #3 \end{pmatrix}}
  \newcommand{\threebytwo}[6]{\begin{pmatrix} %% 3 x 1 matrix
    #1 & #2\\ #3 & #4\\ #5&#6 \end{pmatrix}}
\newcommand{\threebythree}[9]{\begin{pmatrix} %% 3 x 3 matrix
  #1 & #2 & #3 \\ #4 & #5 & #6 \\ #7 & #8 & #9 \end{pmatrix}}

\newcommand{\To}{\Rightarrow}

\newcommand{\vaf}{\overrightarrow}

\newcommand{\set}[1]{\{\,#1\,\}}    %% set notation
\newcommand{\Set}[1]{\biggl\{\,#1\,\biggr\}} %% set notation (large)
\newcommand{\bonj}[1]{\left\lbrack\,#1\,\right\rbrack} 
\newcommand{\red}[1]{\textcolor{red}{#1}}
\newcommand{\blu}[1]{\textcolor{blue}{#1}}

%----------------------------------------------------------------------------------------
%	ARTICLE CONTENTS
%----------------------------------------------------------------------------------------

\begin{document}
\begin{multicols}{2}
\section*{Review}

\subsection*{Limit Comparison Test}

\begin{Ex} 
  Consider the series $\displaystyle\sum_{n=1}^\infty \frac{7}{2^n-5\sin^m(n^3)}$, where $m>0$ is a positive integer. We will study its convergence using the Limit Comparison Test. 
  \begin{enumerate}[i)]
    \itemsep=-0.4em
    \item The fastest growing term is the denominator is $2^n$ because sine is always bounded by its amplitude, in this case $5$.
    \item Comparing the general terms we have 
    \begin{align*}
      \frac{a_n}{b_n}&=\frac{\frac{7}{2^n-5\sin^m(n^3)}}{\frac{1}{2^n}}=\frac{7}{\frac{1}{2^n}\left(2^n-5\sin^m(n^3)\right)}\\
      &=\frac{7}{1-\frac{5\sin^m(n^3)}{2^n}}.
    \end{align*}
    \item We are interested in $\frac{5\sin^m(n^3)}{2^n}$, so we will take the absolute value and analyze:
    $$\left|\frac{5\sin^m(n^3)}{2^n}\right|= \frac{5|\sin^m(n^3)|}{2^n}\leq\frac{5}{2^n}\to0.$$
    So the original term goes to zero as well. This means that $\frac{a_n}{b_n}$ goes to $7$.
    \item As $7>0$, by the Limit Comparison Test, both series behave the same and thus our original series converges.
  \end{enumerate}
\end{Ex}

\subsection*{Ratio Test}
\begin{Ex} 
  Consider the series $\displaystyle\sum_{n=0}^\infty \frac{r^n}{n!}$, $r>0$. We will study its convergence using the ratio test:
  \begin{enumerate}[i)]
    \itemsep=-0.4em 
    \item The general term is $a_n=\frac{r^n}{n!}$.
    \item The next term is $a_{n+1}=\frac{r^{n+1}}{(n+1)!}$ .
    \item Thus combining these two we get the consecutive ratio as:
    $$\left|\frac{a_{n+1}}{a_n}\right|=\left|\frac{\frac{r^{n+1}}{(n+1)!}}{\frac{r^n}{n!}}\right|=\left|\frac{r}{n+1}\right|.$$
    \item As $r$ is a positive constant, the limit doesn't mind it. So when taking the limit of the consecutive ratio we get
    $$\left|\frac{r}{n+1}\right|\to 0,\ n\to\infty.$$
    And so, as $0\in[0,1[$, we that our series converges after applying the ratio test.
  \end{enumerate}

  \end{Ex}
\begin{Rmk}
  It is important to note that this problem can also be solved using the root test.\par 
  We have that the $n^{\text{th}}$ root of the general term is 
  $$\sqrt[n]{\left|\frac{r^n}{n!}\right|}=\frac{r}{\sqrt[n]{n!}}$$
  and that limit will be analyzed on its own.\par 
  We will use the fact that the exponential and logarithm functions are inverses of one another:
  $$\sqrt[n]{n!}=e^{\log(\sqrt[n]{n!})}=e^{\frac{1}{n}\log(n!)}.$$
  The quantity $\log(n!)$ can be managed using a result called \emph{Stirling's Approximation}. This states that for large values of $n$ it holds that $\log(n!)\approx n\log(n)$. So using this we have that 
  $$\frac{1}{n}\log(n!)\approx\frac{1}{n}(n\log(n))=\log(n).$$
  As $n$ grows, $\log(n)\to\infty$ and so $\frac{1}{n}\log(n!)\to\infty$. In this sense, $\sqrt[n]{n!}\to\infty$ and so the expression $\frac{r}{\sqrt[n]{n!}}$ goes to zero. 
\end{Rmk}

\begin{Rmk}
  It is \emph{not imperative} to know the Stirling Approximation, I've used it here for the sake of argument to prove that the root test also works.
\end{Rmk}

\subsection*{Root Test}
\begin{Ex} 
Consider the series $\displaystyle\sum_{n=1}^\infty \frac{n^2}{\left(2+\frac{1}{n}\right)^n}$, we will study its convergence using the root test:
\begin{enumerate}[i)]
  \itemsep=-0.4em 
  \item The general term is $a_n=\frac{n^2}{\left(2+\frac{1}{n}\right)^n}$.
  \item Thus the $n\textsuperscript{th}$ root is 
  $$\sqrt[n]{|a_n|}=\sqrt[n]{\left|\frac{n^2}{\left(2+\frac{1}{n}\right)^n}\right|}=\frac{n^{\frac{2}{n}}}{2+\frac{1}{n}}.$$
  \item We can take the limit of the $n\textsuperscript{th}$ root as follows:
  $$\lim_{n\to\infty}\frac{n^{\frac{2}{n}}}{2+\frac{1}{n}}=\frac{\lim_{n\to\infty} (n^{\frac{2}{n}})}{2+\lim_{n\to\infty}\frac{1}{n}}=\frac{1}{2+0}=\frac12.$$
  And so, as $\half\in[0,1[$, we have that by the root test, our series converges. 
\end{enumerate}
\end{Ex}
\begin{Rmk}
Here we have used the fact that $\sqrt[n]{n}\to 1$ as $n$ grows. This can be proven as follows:
$$\sqrt[n]{n}=e^{\log(\sqrt[n]{n})}=e^{\frac{1}{n}\log(n)}.$$
So taking the limit of the exponential now lets us analyze the limit on the inside:
$$\lim_{n\to\infty}\frac{\log(n)}{n}=\frac{\infty}{\infty}.$$
As this is an indeterminate form we pass the discrete variable to a continuous one and apply L'Hôpital's rule: 
$$\frac{\log(x)}{x}\xrightarrow[]{L'H}\frac{\frac{1}{x}}{1}=\frac{1}{x}\to 0.$$
So this means that in discrete variables $\frac{\log(n)}{n}\to0$. In total the whole expression evaluates to 
$$\lim_{n\to\infty}e^{\frac{1}{n}\log(n)}=e^0=1\To \sqrt[n]{n}\to 1.$$
\end{Rmk}
\newpage
\section*{Exercises}

These exercises were chosen with \emph{malice prépensée}; two series which look quite similar might require the same test (or might not). [\emph{Hints and procedures to this problems will be given upon request.}]\par 

Study the convergence of the following series:
\vspace*{-0.4em}
\begin{enumerate}[i)]
  \itemsep=-0.4em
  \item $\displaystyle\sum_{n=1}^\infty \frac{\sin(kn)}{n^2}$, where $k$ is a positive integer.%C LCT n^-2
  \item $\displaystyle\sum_{n=2}^\infty \frac{1}{\sqrt[3]{n^2-1}}$.%D compare with n^-2/3
  \item $\displaystyle\sum_{n=0}^\infty \frac{1}{\sqrt[3]{n^2+1}}$.%D compare with n^-2/3
  \item $\displaystyle\sum_{n=0}^\infty \frac{n^k}{n!}$, where $k$ is a positive integer..%C ratio test
  \item $\displaystyle\sum_{n=0}^\infty \frac{n^n}{n!}$.%D div, ratio, root
  \item $\displaystyle\sum_{n=1}^\infty \frac{n!}{n^n}$.%C ratio, root
  \item $\displaystyle\sum_{n=1}^\infty \frac{2^nn!}{n^n}$.\label{ex:fact2}%C ratio, root
  \item $\displaystyle\sum_{n=1}^\infty \frac{3^nn!}{n^n}$.\label{ex:fact3}%D div, ratio, root
  \item $\displaystyle\sum_{n=0}^\infty \frac{n^n}{(n+2)^{n+2}}$.%D div, ratio, root
  \item $\displaystyle\sum_{n=1}^\infty \frac{n^n}{(2n)^{2n}}$.%C ratio, root
  \item $\displaystyle\sum_{n=1}^\infty \frac{n^{2n}}{(2n)^{n}}$.%D div, ratio, root
\end{enumerate}

\begin{Ej}
If you have successfully solved items \ref{ex:fact2} and \ref{ex:fact3} then it should be intuitive that there's a value $2<a<3$ such that the series 
$$\sum_{n=1}^\infty \frac{a^nn!}{n^n}$$
switches from convergent to divergent.
\vspace*{-0.4em}
  \begin{enumerate}[i)]
    \itemsep=-0.4em
  \item Show that for $a<e$ the series converges.
  \item For $a>e$ show the opposite, the series diverges.
\end{enumerate}
\end{Ej}

Studying the series $\displaystyle\sum_{n=1}^\infty \frac{e^nn!}{n^n}$ requires the use of \emph{Stirling's Approximation} which we will not use. \textbf{If you wish to discuss this series let me know}.
\begin{Ej}
  For $p,q$ positive integers consider
  $$\sum_{n=0}^\infty\frac{(p+1)(p+2)\cdots(p+n)}{(q+1)(q+2)\cdots(q+n)}.$$
  \vspace*{-0.4em}
  \begin{enumerate}[i)]
    \itemsep=-0.4em
    \item Suppose $q>p+1$, show that the series converges. 
    \item On the other case, $q\leq p+1$, show that the series diverges. 
  \end{enumerate}
\end{Ej}

\begin{Ej}
  The following is an example of a series for which the root test works but the ratio test fails. Consider the series
  $$\sum_{n=0}^\infty\frac{1}{3^{n+(-1)^n}}$$
  \vspace*{-0.4em}
  \begin{enumerate}[i)]
    \itemsep=-0.4em
    \item Find the limit obtained after applying the ratio test.
    \item Show that with the root test, this series converges. %https://www.wolframalpha.com/input?i=lim_%7Bn-%3E%5Cinfty%7D%28%7C3%5E%7Bn%2B%28-1%29%5En%7D%7C%29%5E%7B1%2Fn%7D
  \end{enumerate}
\end{Ej}
\end{multicols}
\end{document} 