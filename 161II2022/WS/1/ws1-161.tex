%----------------------------------------------------------------------------------------
%	PACKAGES AND OTHER DOCUMENT CONFIGURATIONS
%----------------------------------------------------------------------------------------

\documentclass[12pt]{article}
\usepackage[spanish]{babel} %Tildes
\usepackage[extreme]{savetrees} %Espaciado e interlineado. Comentar si no gusta el interlineado.
\usepackage[utf8]{inputenc} %Encoding para tildes
\usepackage[breakable,skins]{tcolorbox} %Cajitas
\usepackage{fancyhdr} % Se necesita para el título arriba
\usepackage{lastpage} % Se necesita para poner el número de página
\usepackage{amsmath,amsfonts,amssymb,amsthm} %simbolos y demás
\usepackage{mathabx} %más símbolos
\usepackage{physics} %simbolos de derivadas, bra-ket.
\usepackage{multicol}
\usepackage[customcolors]{hf-tikz}
\usepackage[shortlabels]{enumitem}
\usepackage{tikz}

%\def\darktheme
%%%%%%%%% === Document Configuration === %%%%%%%%%%%%%%

\pagestyle{fancy}
\setlength{\headheight}{14.49998pt} %NO MODIFICAR
\setlength{\footskip}{14.49998pt} %NO MODIFICAR

\ifx \darktheme\undefined

\lhead{Math161S1} % Nombre de autor
\chead{\textbf{Week 1}} % Titulo
\rhead{}%\firstxmark} 
\lfoot{}%\lastxmark}
\cfoot{}
\rfoot{Page \thepage\ of\ \pageref{LastPage}} %A la derecha saldrá pág. 6 de 9. 
\else
\pagenumbering{gobble}
\pagecolor[rgb]{0,0,0}%{0.23,0.258,0.321}
\color[rgb]{1,1,1}
\fi

%%%%%%%%% === My T Color Box === %%%%%%%%%%%%%%

\ifx \darktheme\undefined
\newtcolorbox{ptcb}{
colframe = black,
colback = white,
breakable,
enhanced
}
\newtcolorbox{ptcbP}{
colframe = black,
colback = white,
coltitle = black,
colbacktitle = black!40,
title = Practice,
breakable,
enhanced
}

\else
\newtcolorbox{ptcb}{
colframe = white,
colback = black,
colupper = white,
breakable,
enhanced
}
\newtcolorbox{ptcbP}{
colframe = white,
colback = black,
colupper = white,
coltitle = white,
colbacktitle = black,
title = Practice,
breakable,
enhanced
}
\fi

%%%%%%%%% === Tikz para matrices === %%%%%%%%%%%%%%

\tikzset{
  style green/.style={
    set fill color=green!50!lime!60,
    set border color=white,
  },
  style cyan/.style={
    set fill color=cyan!90!blue!60,
    set border color=white,
  },
  style orange/.style={
    set fill color=orange!80!red!60,
    set border color=white,
  },
  row/.style={
    above left offset={-0.15,0.31},
    below right offset={0.15,-0.125},
    #1
  },
  col/.style={
    above left offset={-0.1,0.3},
    below right offset={0.15,-0.15},
    #1
  }
}

%%%%%%%%% === Theorems and suchlike === %%%%%%%%%%%%%%

\theoremstyle{plain}
\newtheorem{Th}{Theorem}  %%% Theorem 1.1
\newtheorem*{nTh}{Theorem}             %%% No-numbered Theorem
\newtheorem{Prop}[Th]{Proposition}     %%% Proposition 1.2
\newtheorem{Lem}[Th]{Lemma}             %%% Lemma 1.3
\newtheorem*{nLem}{Lemma}               %%% No-numbered Lemma
\newtheorem{Cor}[Th]{Corollary}        %%% Corollary 1.4
\newtheorem*{nCor}{Corollary}          %%% No-numbered Corollary

\theoremstyle{definition}
\newtheorem*{Def}{Definition}       %%% Definition 1.5
\newtheorem*{nonum-Def}{Definition}    %%% No number Definition
\newtheorem*{nEx}{Example}             %%% No number Example
\newtheorem{Ex}[Th]{Example}           %%% Example
\newtheorem{Ej}[Th]{Exercise}         %%% Exercise
\newtheorem*{nEj}{Exercise}           %%% No number Excercise
\newtheorem*{Not}{Notation}       %%% Definition 1.5

\theoremstyle{remark}
\newtheorem*{Rmk}{Remark}      %%%Remark 1.6

%\numberwithin{equation}{section}

\setlength{\parindent}{3ex}

%%====== Useful macros: =======%%%

\DeclareMathOperator{\gen}{gen}     %%%set generated by...
\DeclareMathOperator{\Rng}{Rng}     %%%rangomat
\DeclareMathOperator{\Nul}{Nul}     %%%rangomat
\DeclareMathOperator{\Proy}{Proy}   %%%proyección
\DeclareMathOperator{\id}{id}       %%%identity operator

\newcommand{\al}{\alpha}            %%%short for \alpha
\newcommand{\la}{\lambda}           %%%short for \lambda
\newcommand{\sg}{\sigma}            %%%short for \sigma
\newcommand{\te}{\theta}                %% short for  \theta
\renewcommand{\l}{\ell}

\newcommand{\thickhat}[1]{\mathbf{\hat{\text{$#1$}}}}
\newcommand{\ii}{\vu{\imath}}
\newcommand{\jj}{\vu{\jmath}}
\newcommand{\kk}{\thickhat{k}}

\newcommand{\bC}{\mathbb{C}}        %%%complex numbers
\newcommand{\bN}{\mathbb{N}}        %%%natural numbers
\newcommand{\bP}{\mathbb{P}}        %%%polynomials
\newcommand{\bR}{\mathbb{R}}        %%%real numbers
\newcommand{\bZ}{\mathbb{Z}}        %%%integer numbers
\newcommand{\cB}{\mathcal{B}}       %%%basis
\newcommand{\cC}{\mathcal{C}}       %%%basis
\newcommand{\cM}{\mathcal{M}}       %%%matrix family

\newcommand{\sT}{\mathsf{T}}        %%%traspuesta

\renewcommand{\geq}{\geqslant}      %%%(to save typing)
\renewcommand{\leq}{\leqslant}      %%%(to save typing)
\newcommand{\x}{\times}             %%%product
\renewcommand{\:}{\colon}           %%%colon in  f: A -> B
\newcommand{\isom}{\simeq}              %% isomorfismo

\newcommand{\un}[1]{\underline{#1}}
\newcommand{\half}{\frac12}

\newcommand*{\Cdot}{{\raisebox{-0.25ex}{\scalebox{1.5}{$\cdot$}}}}      %% cdot más grande
\renewcommand{\.}{\Cdot}                %% producto escalar

\newcommand{\twobyone}[2]{\begin{pmatrix} %% 2 x 1 matrix
  #1 \\ #2 \end{pmatrix}}
  \newcommand{\twobytwo}[4]{\begin{pmatrix} %% 2 x 2 matrix
    #1 & #2 \\ #3 & #4 \end{pmatrix}}
    \newcommand{\twobythree}[6]{\begin{pmatrix} %% 2 x 3 matrix
        #1 & #2 & #3\\ #4 & #5 & #6 \end{pmatrix}}
\newcommand{\threebyone}[3]{\begin{pmatrix} %% 3 x 1 matrix
  #1 \\ #2 \\ #3 \end{pmatrix}}
  \newcommand{\threebytwo}[6]{\begin{pmatrix} %% 3 x 1 matrix
    #1 & #2\\ #3 & #4\\ #5&#6 \end{pmatrix}}
\newcommand{\threebythree}[9]{\begin{pmatrix} %% 3 x 3 matrix
  #1 & #2 & #3 \\ #4 & #5 & #6 \\ #7 & #8 & #9 \end{pmatrix}}

\newcommand{\To}{\Rightarrow}

\newcommand{\vaf}{\overrightarrow}

\newcommand{\set}[1]{\{\,#1\,\}}    %% set notation
\newcommand{\Set}[1]{\biggl\{\,#1\,\biggr\}} %% set notation (large)
\newcommand{\red}[1]{\textcolor{red}{#1}}
\newcommand{\blu}[1]{\textcolor{blue}{#1}}

%----------------------------------------------------------------------------------------
%	ARTICLE CONTENTS
%----------------------------------------------------------------------------------------

\begin{document}
\begin{multicols}{2}
\section*{Integration by Parts}

The substitution formula (\emph{$u$-sub}) works as an analog to the chain rule in differentiation. The integration by parts formula is analogous to the \emph{product rule}.

\begin{Rmk}
  Suppose $f,g$ are differentiable. Then 
  $$\dv{x}(fg)=f\dv{x}(g)+g\dv{x}(f)$$
  and integrating both sides we obtain:
  $$fg=\int\dv{x}(fg)=\int f(g')\dd x+\int (f')g\dd x.$$
  Rearranging the equality we obtain the formula
  $$\int f(g')\dd x=fg-\int (f')g\dd x.$$
\end{Rmk}

\begin{Def}
  The \un{integration by parts} formula for two functions $u$ and $v$ is given by
  $$\int u\dd v=uv-\int v\dd u,$$
  where $\dd u=\dv{u}{x}\dd x$ and likewise for $v$.\par 
  An easy way to remember the right hand side of this formula is with the mnemonic \un{\emph{ultraviolet voodoo}}. 
\end{Def}

\begin{Ex} 
If we are asked to integrate $xe^x$ by itself, we can't do it. However with the formula we can:
$$\int xe^x\dd x,\ \text{with } u=x,\ \dd v=e^x\dd x.$$
\emph{The $u$ is the function which is easy to differentiate and the $v$ is the one which is easier to integrate.}\par 
We obtain $\dd u=\dd x$ by differentiating and $v = e^x$ after integrating. Thus rearranging the integral we get
$$\int xe^x\dd x=xe^x-\int e^x\dd x.$$
The last integral we can compute so in the end we obtain
$$\int xe^x\dd x=xe^x-\int e^x\dd x=\un{xe^x-e^x}.$$
\end{Ex}

\begin{Ex}
  Consider the following integral 
  $$\int x^2 e^x\dd x.$$
  To integrate we have to apply the same formula. Here we take 
  $$u=x^2\ \text{and } \dd v=e^x\dd x.$$
  We differentiate $u$ and integrate $\dd v$ to obtain 
  $$\dd u=2x\dd x\ \text{and } v=e^x.$$
  Given this we can arrange the integration by parts formula as follows:
  $$\int x^2 e^x\dd x= \underbrace{x^2}_{u}\underbrace{e^x}_{v}-\int \underbrace{e^x}_{v}\underbrace{2x\dd x}_{\dd u}.$$
  We can factor out a two from the last integral, and using the result from the previous example we get
  $$\int x^2 e^x\dd x= x^2e^x-2(xe^x-e^x)=\un{x^2e^x-2xe^x+2e^x}.$$
\end{Ex}

\begin{ptcbP}
In a group with your classmates calculate the following integral
$$\int x^3e^x\dd x.$$
\emph{This calculation can be made easier by using the result of the previous examples.}\par 
How about calculating 
$$\int x^ne^x\dd x,\ \text{for } n=4,\dots,7?$$
\end{ptcbP}

\begin{Ex}\label{ex:intxsinx}
  Consider the integral 
  $$\displaystyle \int x\sin(x)\dd x.$$
  Like in the previous cases, we take 
  $$u=x\ \text{and } \dd v=\sin(x)\dd x$$
  therefore
  $$\dd u=\dd x\ \text{and } v=-\cos(x).$$
  The formula gives us 
  $$\int x\sin(x)\dd x=x(-\cos(x))-\int (-\cos(x))\dd x.$$
  Integrating the cosine and taking out the minuses gives us 
  $$\int x\sin(x)\dd x=\un{-x\cos(x)+\sin(x)}.$$
  However, we haven't asked ourselves what happens if we take 
  $$u=\sin(x)\ \text{and } \dd v= x\dd x.$$
  In this case we get 
  $$\dd u=\cos(x)\dd x\ \text{and } v=\frac{x^2}{2}$$
  and thus after arranging the integral with the formula we get
  $$\int x\sin(x)\dd x=\sin(x)\left(\frac{x^2}{2}\right)-\int \frac{x^2}{2}\cos(x)\dd x.$$
\end{Ex}

We have ran into a problem! Using the formula the other way didn't make this integral simpler to calculate. \emph{That's our objective.} There's no definitive way of choosing who's $u$ and who's $v$ but an easy heuristic is as follows:

\begin{Prop}
The order in which to choose who's $u$ is 
  \begin{itemize}
    \itemsep=-0.4em
    \item (\textbf{L})ogarithms
    \item (\textbf{I})nverse Trigonometrics
    \item (\textbf{A})lgebraic functions
    \item (\textbf{T})rigonometric functions
    \item (\textbf{E})xponentials
  \end{itemize}
The mnemonic to remember these is \textbf{LIATE}.
\end{Prop}

Usually following this idea we will get a simpler integral to calculate after applying the formula. Lets use this idea to integrate the following:

\begin{Ex}
  Let us calculate the indefinite integral 
  $$\int \log(x)\dd x.$$
Since the function is a logarithm, it's a top priority to differentiate that. Thus $u=\log(x)$, \un{but\dots who's $\dd v$}? There's always a hidden one $(1)$ multiplying right there, so we take $\dd v=1\dd x=\dd x$. After differentiating and integrating we get
$$\dd u=\frac 1x\dd x\ \text{and } v=x.$$
The integral becomes
$$\int \log(x)\dd x=\log(x)\.(x)-\int x\left(\frac{1}{x}\right)\dd x=\un{x\log(x)-x}$$
\end{Ex}

\begin{Rmk}
  Don't forget to consider integrating 1! This works sometimes!
\end{Rmk}

\begin{ptcbP}
  Calculate the integral of $\arcsin(x)$.
  \emph{Like the one we just did, use the same idea!}
\end{ptcbP}

\begin{Ex}
  We can now calculate the integral 
  $$\int x\log (x)\dd x,\ \text{let }
  \left\lbrace
  \begin{aligned}
    &u=\log(x)\To\dd u=1/x\dd x,\\
    &\dd v= x\dd x\To v=(1/2)x^2,
  \end{aligned}
  \right.
  $$
  and then after rearranging we get
  \begin{align*}
    \int x\log (x)\dd x&=\half x^2\log(x)-\int\left(\frac{1}{2}x^2\right)\left(\frac{1}{x}\dd x\right)\\
    &=\un{\half x^2\log(x)-\frac{1}{4}x^2}.
  \end{align*}
\end{Ex}

We could generalize this to any integer power of $x$, but how about a rational, or even an \emph{irrational} power of $x$?

\begin{Ex}
  Let us find the indefinite integral 
  $$\int x^{\sqrt{5}}\log(x)\dd x.$$
  In fact, the power at which $x$ is raised \emph{does not matter}. The process is the same! Let 
  $$
  \left\lbrace
  \begin{aligned}
    &u=\log(x)\To\dd u=1/x\dd x,\\
    &\dd v= x^{\sqrt{5}}\dd x\To v=(1/(\sqrt{5}+1))x^{\sqrt{5}+1}.
  \end{aligned}
  \right.
  $$
Rearranging we get
$$\int x^{\sqrt{5}}\log(x)\dd x=\frac{(x^{\sqrt{5}+1})\log(x)}{\sqrt{5}+1}-\int\left(\frac{x^{\sqrt{5}+1}}{\sqrt{5}+1}\right)\left(\frac{1}{x}\dd x\right).$$
The integral on the right is the integral of a power of $x$ so in the end, the result is 
$$\int x^{\sqrt{5}}\log(x)\dd x=\frac{(x^{\sqrt{5}+1})\log(x)}{\sqrt{5}+1}-\frac{x^{\sqrt{5}+1}}{(\sqrt{5}+1)^2}.$$
\end{Ex}
\vfill\null\columnbreak
\begin{Ex}
  In this example we don't consider a power of $x$ multiplying another function. Let's calculate
  $$\int e^x\sin(x)\dd x.$$
  By using the \textbf{LIATE} mnemonic we choose the following
  $$
  \left\lbrace
  \begin{aligned}
    &u=\sin(x)\To\dd u=\cos(x)\dd x,\\
    &\dd v= e^x\dd x\To v=e^x.
  \end{aligned}
  \right.
  $$
  We obtain 
  $$\int e^x\sin(x)\dd x=e^x\sin(x)-\int\cos(x)e^x\dd x.$$
  Applying the formula once more with this last integral:
  $$
  \left\lbrace
  \begin{aligned}
    &u_2=\cos(x)\To\dd u_2=-\sin(x)\dd x,\\
    &\dd v_2= e^x\dd x\To v_2=e^x.
  \end{aligned}
  \right.
  $$
 If $I$ is the original integral we get:
 $$I=e^x\sin(x)-\left(e^x\cos(x)+\int e^x\sin(x)\dd x\right).$$
This last integral is the one we are looking for. We can rearrange this equation as follows:
\begin{gather*}
  I=e^x\sin(x)-e^x\cos(x)-I\\
  \To I=\un{(1/2)(e^x\sin(x)-e^x\cos(x))}
\end{gather*}
\end{Ex}

\begin{ptcbP}
  With an analogous reasoning calculate the integral
  $$\int e^{2x}\sin(3x)\dd x.$$
\end{ptcbP}
\begin{Ej}
  Compute the following integrals:
  \vspace{-0.4em}
  \begin{itemize}
    \itemsep=-0.3em
    \item $\int x^2\cos(x)\dd x$ (\emph{Hint: Use Example \ref{ex:intxsinx}}).
    \item $\int\log^2(x)\dd x$.
    \item $\int x^\al\log(x)\dd x$, $\al\neq 1$ is any real number.
    \item $\int \sqrt{x}\log(3x)\dd x$.
  \end{itemize}
\end{Ej}
\vfill\null
\end{multicols}
\end{document} 