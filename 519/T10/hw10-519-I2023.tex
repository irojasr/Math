\documentclass[12pt]{memoir}

\def\nsemestre {I}
\def\nterm {Spring}
\def\nyear {2023}
\def\nprofesor {Jeff Achter}
\def\nsigla {MATH519}
\def\nsiglahead {Complex Analysis}
\def\nextra {HW10}
\def\nlang {ENG}
\input{../../headerVarillyDiff}

\begin{document}

\begin{Ej}[5.10(a) Stein\& Shakarchi]
   Find the Hadamard product for $e^z-1$.
\end{Ej}

\begin{ptcbr}
Recall Hadamard's theorem states that if $f$ is an entire function with order of growth $\rho$ and $k=\floor{\rho}$ then 
$$f(z)=e^{p(z)}z^m\prod_{n=1}^{\infty}E_k\left(\frac{z}{a_n}\right)$$
where $(a_n)$ is the collection of non-null zeroes of $f$, $p$ has degree at most $k$ and $m=\ord(f,0)$.\par 
In our case $e^z-1$ has order of growth 1 and it has simple zeroes at $z=2\pi i n$ for $n\in\bZ$. In particular the order of zero is one. This means that 
$$e^z-1=e^{a_1z+a_0}z\prod_{n\in\bZ\less\set{0}}\left(1-\frac{z}{2\pi i n}\right)e^{z/2\pi i n}.$$
To simplify this product we multiply opposites across the origin:
\begin{align*}
\bonj{\left(1-\frac{z}{2\pi i n}\right)e^{z/2\pi i n}}\bonj{\left(1-\frac{z}{2\pi i (-n)}\right)e^{z/2\pi i (-n)}}&=\left(1+\left(\frac{z}{2\pi i n}\right)^2\right)e^{z/2\pi i n}e^{-z/2\pi i n}\\
&=1+\frac{z^2}{4\pi^2n^2}   
\end{align*}
So we get
$$e^z-1=e^{a_1z+a_0}z\prod_{n=1}^\infty\left(1+\frac{z^2}{4\pi^2n^2} \right).$$ 
Dividing both sides by $z$ we get 
$$\frac{e^z-1}{z}=e^{a_1z+a_0}\prod_{n=1}^\infty\left(1+\frac{z^2}{4\pi^2n^2} \right)$$
and as $z$ approaches $0$ we get that 
$$1=e^{a_0}(1)\To a_0=0.$$
Expanding the exponential function as a Taylor series and comparing coefficients we get the following:
$$z+\frac{z^2}{2}+O(z^3)=(1+a_1z+\frac{(a_1z)^2}{2}+O(z^3))z\prod_{n=1}^\infty\left(1+\frac{z^2}{4\pi^2n^2} \right)$$
Thus we obtain 
$$z+\frac{z^2}{2}+O(z^3)=z+a_1z^2+O(z^3)\To a_1=\frac{1}{2}.$$
In conclusion we have 
$$e^z-1=e^{z/2}z\prod_{n=1}^\infty\left(1+\frac{z^2}{4\pi^2n^2} \right).$$
\end{ptcbr}

\begin{Ej}[5.11 Stein\& Shakarchi]
   Show that if $f$ is an entire function of finite order that omits two values, then $f$ is constant. This result remains true for any entire function and is known as Picard's little theorem.
\hint{If $f$ misses $a$, then $f (z)-a$ is of the form $e^{p(z)}$ where $p$ is a polynomial.}

\end{Ej}

\begin{ptcbr}
Assume $f$ omits two values $a,b$ which means that 
$$f(z)-a=e^{p(z)},\word{and}f(z)-b=e^{q(z)}\word{for some}p,q\word{polynomials}.$$
From this, we may subtract one equation from the other to get 
$$b-a=e^{p(z)}-e{q(z)}$$
   \end{ptcbr}
\end{document} 
