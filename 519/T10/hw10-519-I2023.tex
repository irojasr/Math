\documentclass[12pt]{memoir}

\def\nsemestre {I}
\def\nterm {Spring}
\def\nyear {2023}
\def\nprofesor {Jeff Achter}
\def\nsigla {MATH519}
\def\nsiglahead {Complex Analysis}
\def\nextra {HW10}
\def\nlang {ENG}
\input{../../headerVarillyDiff}

\begin{document}

\begin{Ej}[5.10(a) Stein\& Shakarchi]
   Find the Hadamard product for $e^z-1$.
\end{Ej}

\begin{ptcbr}
Recall Hadamard's theorem states that if $f$ is an entire function with order of growth $\rho$ and $k=\floor{\rho}$ then 
$$f(z)=e^{p(z)}z^m\prod_{n=1}^{\infty}E_k\left(\frac{z}{a_n}\right)$$
where $(a_n)$ is the collection of non-null zeroes of $f$, $p$ has degree at most $k$ and $m=\ord(f,0)$.\par 
In our case $e^z-1$ has order of growth 1 and it has simple zeroes at $z=2\pi i n$ for $n\in\bZ$. In particular the order of zero is one. This means that 
$$e^z-1=e^{a_1z+a_0}z\prod_{n\in\bZ\less\set{0}}\left(1-\frac{z}{2\pi i n}\right)e^{z/2\pi i n}.$$
To simplify this product we multiply opposites across the origin:
\begin{align*}
\bonj{\left(1-\frac{z}{2\pi i n}\right)e^{z/2\pi i n}}\bonj{\left(1-\frac{z}{2\pi i (-n)}\right)e^{z/2\pi i (-n)}}&=\left(1+\left(\frac{z}{2\pi i n}\right)^2\right)e^{z/2\pi i n}e^{-z/2\pi i n}\\
&=1+\frac{z^2}{4\pi^2n^2}   
\end{align*}
So we get
$$e^z-1=e^{a_1z+a_0}z\prod_{n=1}^\infty\left(1+\frac{z^2}{4\pi^2n^2} \right).$$ 
Dividing both sides by $z$ we get 
$$\frac{e^z-1}{z}=e^{a_1z+a_0}\prod_{n=1}^\infty\left(1+\frac{z^2}{4\pi^2n^2} \right)$$
and as $z$ approaches $0$ we get that 
$$1=e^{a_0}(1)\To a_0=0.$$
Expanding the exponential function as a Taylor series and comparing coefficients we get the following:
$$z+\frac{z^2}{2}+O(z^3)=(1+a_1z+\frac{(a_1z)^2}{2}+O(z^3))z\prod_{n=1}^\infty\left(1+\frac{z^2}{4\pi^2n^2} \right)$$
Thus we obtain 
$$z+\frac{z^2}{2}+O(z^3)=z+a_1z^2+O(z^3)\To a_1=\frac{1}{2}.$$
In conclusion we have 
$$e^z-1=e^{z/2}z\prod_{n=1}^\infty\left(1+\frac{z^2}{4\pi^2n^2} \right).$$
\end{ptcbr}

\begin{Ej}[5.11 Stein\& Shakarchi]
   Show that if $f$ is an entire function of finite order that omits two values, then $f$ is constant. This result remains true for any entire function and is known as Picard's little theorem.
\hint{If $f$ misses $a$, then $f (z)-a$ is of the form $e^{p(z)}$ where $p$ is a polynomial.}

\end{Ej}

\begin{ptcbr}
Assume $f$ omits two values $a,b$ which means that 
$$f(z)-a=e^{p(z)},\word{and}f(z)-b=e^{q(z)}\word{for some}p,q\word{polynomials}.$$
From this, we may subtract one equation from the other to get 
$$b-a=e^{p(z)}-e{q(z)}$$
\iffalse
and now taking the limit as $z\to\infty$ on both sides we see that 
$$\lim_{z\to\infty}(e^{p(z)}-e{q(z)})=b-a$$
is finite limit. 
\fi
and now we may differentiate both sides of the equation to obtain 
$$0=p'(z)e^{p(z)}-q'(z)e^{q(z)}.$$
As this equation holds for \emph{all} $z\in\bC$ it must happen that $p'(z)$ and $q'(z)$ have the same zeroes with the same multiplicities. Thus $q'(z)=cp'(z)$ for some non-zero $c\in\bC$. Returning to our equation we have 
$$e^{p(z)}p'(z)=cp'(z)e^{q(z)}\To e^{p(z)}=ce^{q(z)}\To ce^{q(z)}-e^{q(z)}=b-a.$$
Differentiating this equation we obtain 
$$(c-1)q'(z)e^{q(z)}=0\To q'(z)=0\To q\word{is constant}.$$
This allows us to conclude that $f$ is constant as $f=e^q+b$.\par 
If it occured that $c=1$, then $p'=q'$ and so $q(z)=p(z)+d$ for some $d\in\bC$. Replacing this in the equation we have 
$$e^{p(z)}-e^{p(z)+d}=b-a\To(1-e^d)e^{p(z)}p'(z)=0\To p'(z)=0\To p\word{is constant}$$
and once again we deduce $f$ is constant. Finally if it was the case that $d=0$, then $p=q$ but this means that 
$$f(z)-e^{p(z)}=a=b$$ 
and this can't happen as $a,b$ are different values of $\bC$. In conclusion we have that $f$ is constant.
   \end{ptcbr}

\begin{Ej}
   Assume $\Re(s)=\sg>0$. For $n,N\in\bN$ define 
   $$\dl_n(s)=\frac{1}{n^s}-\int\limits_n^{n+1}\frac{\dd x}{x^s}=\int\limits_n^{n+1}\left(\frac{1}{n^s}-\frac{1}{x^s}\right)\dd x\word{and}F_N(s)=\sum_{n=1}^{N}\dl_n(s).$$
   \begin{enumerate}[i)]
      \itemsep=-0.4em
      \item Show that $|\dl_n(s)|\leq\frac{|s|}{n^{\Re(s)+1}}$. \hint{Represent the integrand in the definition of $\dl_n$ using the
      observation that $\int_n^x\frac{\dd u}{u^{s+1}}=\frac{-1}{s}(x^{-s}-n^{-s})$.}
      \item Show that $(F_N(s))$ converges uniformly on any half-plane of the form $\Re(s)\geq \al>0$.
      \item Show that $\zeta(s)-\frac{1}{s-1}$ is bounded and holomorphic near $s=1$. \hint{Use the fact that $\frac{1}{s-1}=\int_1^\infty x^{-s}\dd x$.}
   \end{enumerate}
\end{Ej}

\begin{ptcbr}
   \begin{enumerate}[i)]
      \itemsep=-0.4em
      \item Observe that from the hint we have 
      $$s\int\limits_n^x\frac{\dd u}{u^{s+1}}=\left(\frac{1}{n^s}-\frac{1}{x^s}\right).$$
      Then replacing this expression inside $\dl_n$ we get 
      $$ |\dl_n(s)|=\left|\int\limits_{n}^{n+1}s\int\limits_n^x\frac{\dd u}{u^{s+1}}\dd x\right|\leq \int\limits_{n}^{n+1}|s|\left|\int\limits_n^x\frac{\dd u}{u^{s+1}}\right|\dd x$$
      and we can estimate the inner integral by taking the integrand's sup-norm. We have 
      $$\left|\int\limits_n^x\frac{\dd u}{u^{s+1}}\right|\leq\sup_{n\leq u\leq x}\left|\frac{1}{u^{s+1}}\right|(x-n)$$
      and for positive real numbers $u$ we can estimate $|u^z|$ as 
      $$|u^x||u^{iy}|=u^x|e^{iy\log(u)}|=u^x$$ 
      so this means that 
      $$\sup_{n\leq u\leq x}\left|\frac{1}{u^{s+1}}\right|(x-n)=\sup_{n\leq u\leq x}\frac{1}{u^{\sg+1}}(x-n)\leq\frac{x-n}{n^{\sg+1}}$$
      where the last inequality holds because $\frac{1}{u^{\sg+1}}$ is a decreasing function. Returning to our $\dl_n$ estimate we have
      $$\int\limits_{n}^{n+1}|s|\left|\int\limits_n^x\frac{\dd u}{u^{s+1}}\right|\dd x\leq \int\limits_{n}^{n+1}|s|\frac{x-n}{n^{\sg+1}}\dd x=\frac{|s|}{n^{\sg+1}}\frac{1}{2}\leq\frac{|s|}{n^{\sg+1}}.$$
      \item We now consider the partial sums $F_N=\sum_{n=1}^{N}\dl_n$. Observe that we may bound $F_N$ with the countable triangle inequality:
      $$|F_N(s)|\leq \sum_{n=1}^{N}|\dl_n(s)|\leq\sum_{n=1}^{N}\frac{|s|}{n^{\sg+1}}$$
      and this is a series of real numbers which converges when $\sg>0$. This implies that $F_N$ converges uniformly for $\sg\geq\al>0$.
      \item Observe that from our initial identity 
      $$\dl_n(s)=\frac{1}{n^s}-\int\limits_n^{n+1}\frac{\dd x}{x^s}$$
      we can sum up to $N$ to obtain
      $$F_N(s)=\sum_{n=1}^{N}\frac{1}{n^s}-\sum_{n=1}^{N}\left(\int\limits_n^{n+1}\frac{\dd x}{x^s}\right)=\sum_{n=1}^{N}\frac{1}{n^s}-\int\limits_{1}^{N+1}\frac{\dd x}{x^s}.$$
      Letting $N$ grow without bound we arrive at 
      $$\lim_{N\to\infty}F_N(s)=\zeta(s)-\int\limits_1^{\infty}\frac{\dd x}{x^s}=\zeta(s)-\frac{1}{s-1}.$$
      As the properties of the sequence $(F_N(s))$ are preserved through uniform limit, we have that $F_\infty(s)$ is bounded and holomorphic for $\sg>0$. So our expression for the equality is only valid wherever the improper integral converges, and this is where $\sg>1$. So we obtain the desired result as $F_\infty$ is $\zeta(s)-\frac{1}{s-1}$.
   \end{enumerate}
\end{ptcbr}
\end{document} 
