\documentclass[12pt]{memoir}

\def\nsemestre {I}
\def\nterm {Spring}
\def\nyear {2023}
\def\nprofesor {Jeff Achter}
\def\nsigla {MATH519}
\def\nsiglahead {Complex Analysis}
\def\nlang {ENG}
\input{../../headerVarillyDiff}

\begin{document}
%\clearpage
\maketitle
%\thispagestyle{empty}
{\small
\setlength{\parindent}{0em}
\setlength{\parskip}{1em}

This course is an introduction to analytic functions of a single complex variable.  The subject is beautiful.-- it turns out that a function with a complex derivative is highly structured -- and enjoys a give and take with many other areas of mathematics.

\subsubsection*{Requirements}
Knowledge of convergence of sequences, series: limits, continuity, differentiation, integration of one-variable functions is required.
}
\newpage
\tableofcontents
%\begin{multicols}{2}
\chapter{First Midterm}

\section{Interim| HW1}

\begin{Ej}[1.1 Stein \& Shakarchi]
Describe geometrically the sets of points $z$ in the complex plane defined by the following relations:
\begin{enumerate}
    \itemsep=-0.4em
    \item $|z-z_1|=|z-z_2|$ where $z_1,z_2\in\bC$.
    \item $1/z=\ov z$.
    \item $\Re(z)=3$
    \item $\Re(z)>c$, (resp.,$\geq c$) where $c\in\bR$.
    \item $\Re(az+b)>0$ where $a,b\in\bC$.
    \item $|z|=\Re(z)+1$.
    \item $\Im(z)=c$ with $c\in\bR$.
\end{enumerate}
\end{Ej}

\begin{ptcbr}
    \begin{enumerate}[i)]
        \itemsep=-0.4em
        \item The first set is the set of points at the same distance from $z_1$ and $z_2$. If we consider the line segment $z_1z_2$, then the set in question is the bisector of that line segment.
        \item Note that
        $$1/z=\ov z\iff 1=\ov zz\iff 1=|z|^2\iff 1=|z|,$$
        thus the set is the unit circle.
        \item The set is a perpendicular line to the real axis at $z=3$.
        \item This infinite set is an infinite half plane to the right (but not including) of the line $z=c$. In the other case, we do include the line in question.
        \item Let us rephrase this inequality in terms of real numbers. Call $a=a_1+ia_2$, $b=b_1+ib_2$ and $z=x+iy$. Then 
        $$\Re(az+b)=\Re[a_1 x - a_2 y + b_1 + i (a_2 x +  a_1 y +  b_2)],$$
        thus our desired inequality is true whenever $a_1 x - a_2 y + b_1>0$. Solving for $y$ we get $y>(a_1x+b_1)/a_2$, which is the half plane located above the line $y=(a_1x+b_1)/a_2$.
        \item The equation in question is equivalent to 
        $$\Re(z)^2+\Im(z)^2=(\Re(z)+1)^2.$$
        To ease the notation, assume $z=x+iy$. Then the equation reads 
        $$x^2+y^2=x^2+2x+1\iff y^2=2x+1\iff x=(y^2-1)/2.$$
        It holds the the parabola in question contains the points which satisfy the equation.
        \item This set is a line parallel to the real axis at $z=c$
    \end{enumerate}
\end{ptcbr}

\begin{Ej}
    Do the following:
    \begin{enumerate}[i)]
        \itemsep=-0.4em
        \item Show that the complex conjugation map $\kp:\bC\to\bC,\ z\mapsto\ov z$ is an involution, i.e., a ring homomorphism such that $\kp\circ\kp=\id$.
        \item Suppose $a\in\bR,\ z\in\bC$. Show that 
        $$\Re(az)=a\Re(z),\word{and}\Im(az)=a\Im(z).$$
    \end{enumerate}
\end{Ej}

\begin{ptcbr}
    Let us take $z=x+iy$ with $x,y\in\bR$.
    \begin{enumerate}[i)]
        \itemsep=-0.4em
        \item We have $\ov z=x+i(-y)=x-iy$. Once more we get $\ov{\ov z}=x-i(-y)=x+iy=z$. Thus $\ov{\ov z}=z$ for any $z\in\bC$. In conclusion $\ov{\ov \.}=\id$.
        \item It holds that 
        \begin{align*}
            &\Re(az)=\Re(ax+aiy)=ax=a\Re(z),\\
            &\Im(az)=\Im(ax+aiy)=ay=a\Im(z).
        \end{align*}
    \end{enumerate}
\end{ptcbr}

\begin{Ej}
    Do the following:
    \begin{enumerate}[i)]
        \itemsep=-0.4em
        \item Prove that $|z+w|^2=|z|^2+|w|^2+2\Re(z\ov w)$.
        \item Use this to prove the parallelogram rule: $|z+w|^2+|z-w|^2=2(|z|^2+|w|^2)$.
    \end{enumerate}
\end{Ej}

\begin{ptcbr}
    \begin{enumerate}[i)]
        \itemsep=-0.4em
        \item Note that 
        $$|z+w|^2=(z+w)\ov{(z+w)}=(z+w)(\ov z+\ov w)=z\ov{z}+w\ov{z}+z\ov{w}+w\ov w.$$
        The number $w\ov z$ is the conjugate of $z\ov w$, and summing a number and its conjugate returns twice its real part. Thus we get the desired identity. 
        \item As the past identity holds for all complex numbers, it holds when $w=-w$. This means that 
        $|z-w|^2=|z|^2+|-w|^2+2\Re(z(\ov{-w}))=|z|^2+|w|^2-2\Re(z\ov w)$
        and summing this together with the first identity gives us the parallelogram law.
    \end{enumerate}
\end{ptcbr}

\begin{Ej}[1.5 Stein \& Shakarchi]
    A set $\Om$ is said to be pathwise connected if any two points in $\Om$ can be joined by a (piecewise-smooth) curve entirely contained in $\Om$. The purpose of this exercise is to prove that an open set $\Om$ is pathwise connected if and only if $\Om$ is connected.
    \begin{enumerate}[i)]
        \itemsep=-0.4em
        \item Suppose first that $\Om$ is open and pathwise connected, and that it can be written as $\Om$ = $\Om_1\cup\Om_2$ where $\Om_1$ and $\Om_2$ are disjoint non-empty open sets. Choose two points $w_1\in\Om_1$ and $w_2\in\Om_2$ and let $\ga$ denote a curve in $\Om$ joining $w_1$ to $w_2$. Consider a parametrization $z:\bonj{0,1}\to\Om$ of this curve with $z(0) = w_1$ and $z(1) = w_2$, and let
        $$t_\ast = \sup_{0\leq t\leq 1}\set{t\:\forall s [(0\leq s<t)\To (z(s)\in\Om_1)]}.$$
        Arrive at a contradiction by considering the point $z(t_\ast)$.
        \item Conversely, suppose that $\Om$ is open and connected. Fix a point $w\in\Om$ and let $\Om_1\subseteq\Om$ denote the set of all points that can be joined to $w$ by a curve contained in $\Om$. Also, let $\Om_2\subseteq\Om$ denote the set of all points that cannot be joined to $w$ by a curve in $\Om$. Prove that both $\Om_1$ and $\Om_2$ are open, disjoint and their union is $\Om$. Finally, since $\Om_1$ is non-empty (why?) conclude that $\Om$ = $\Om$1 as desired.
    \end{enumerate}
    \end{Ej}

\begin{ptcbr}
    \begin{enumerate}[i)]
        \itemsep=-0.4em
        \item Recall first, that by definition of supremum we have that if $S$ is our set, then 
        $$\exists s\in S(s>t_\ast-\eps)$$
        for $\eps>0$. Following the idea, we consider the point $z(t_\ast)$. We have two options to place $z(t_\ast)$, either in $\Om_1$ or $\Om_2$.\par 
        Let's start by definition of supremum \red{FINISH}
        \item Take $\Om_1,\Om_2$ as in the statement. Then $\Om_1$ is non-empty as $w\in\Om_1$ because it's connected to itself through a trivial path. Suppose now that $z\in\Om_1$ and that $r>0$. Take $x\in B(z,r)$, then there exists a line-segment between $z$ and $x$ and there's a smooth curve which connects $z\in\Om_1$ with $w$. Thus the piecewise-continuous path from $x$ to $z$ and from $z$ to $w$ is a path which connects $x$ and $w$. As $x$ is arbitrary, it follows that $B(z,r)\subseteq \Om_1$. Formally, if $\ga:[0,1]\to\Om_1$ is the map which parametrizes the curve between $z$ and $w$ and $r:[0,1]\to B(z,r)$ is the map $t\mapsto tz+(1-t)x$, then the curve from $x$ to $w$ is parametrized by the function 
        $$f=\begin{cases}
            
        \end{cases}$$
    \end{enumerate}
\end{ptcbr}

\begin{Ej}[1.7 Stein \& Shakarchi]
    The family of mappings introduced here plays an important role in complex analysis. These mappings, sometimes called \textbf{Blaschke factors}, will reappear in various applications in later chapters.
    \begin{enumerate}[i)]
        \itemsep=-0.4em
        \item Let $z,w\in\bC$ such that $\ov{z}w\neq 1$. Prove that 
        $$\left|\frac{w-z}{1-\ov w z}\right|<1$$
        if $|z|<1$ and $|w|<1$, and also that 
        $$\left|\frac{w-z}{1-\ov w z}\right|=1$$
        if $|z|=1$ or $|w|=1$. \hint{Why can one assume that $z$ is real? I then suffices to prove that $(r-w)(r-\ov w)\leq (1-rw)(1-r\ov w)$ with equality for appropriate $r$ and $|w|$.}\aside{Here is an alternate approach, which you may use if you like. Fix $w\in\bC$ with $w<1$, and consider the function $z\mapsto \frac{w-z}{1-\ov w z}$. What is $\ov{f(z)}$? By computing $f(z)\ov{f(z)}$, show that $|z|=1$ implies $|f(z)|=1$. Find a point $z$ with $|z|<1$ such that $|f(z)|<1$. Since $f$ is continuous, this shows that $f$ takes the unit disc to itself. (Why?)}
        \item Prove that for a fixed $w\in\bD$, the mapping $F\:z\mapsto\frac{w-z}{1-\ov w z}$ satisfies the following:
        \begin{enumerate}[a)]
            \itemsep=-0.4em
            \item $F$ maps the unit disc to itself (that is, $F:\bD\to\bD$), and is holomorphic.
            \item $F$ interchanges $0$ and $w$. 
            \item $|F(z)|=1$ if $|z|=1$.
            \item $F$ is bijective. \hint{Calculate $F\circ F$.}
        \end{enumerate}
    \end{enumerate}
    \end{Ej}

    \begin{ptcbr}
        \begin{enumerate}[i)]
            \itemsep=-0.4em
            \item The inequality in question is equivalent to 
            $$0\leq|w-z|<|1-\ov wz|.$$
            Since the quantities are positive, we can square them and preserve the order. It holds that 
            $$0\leq|w-z|^2<|1-\ov wz|^2\iff 0\leq (w-z)\ov{(w-z)}<(1-\ov wz)\ov{(1-\ov wz)},$$
            Simplifying this expression we get 
            \begin{align*}
                &(w-z)(\ov w-\ov z)<(1-\ov wz)(1-w\ov z)\\
                \iff&w\ov w-w\ov z -z\ov w+z\ov z<1-w\ov z-\ov wz+\ov wzw\ov z\\
                \iff&|w|^2+|z|^2<1+|w|^2|z|^2\\
                \iff&0<(1-|w|^2)(1-|z|^2).
            \end{align*}
            The inequality is true whenever both moduli are less than one, and whenever either is one equality is achieved.
            \item Now we suppose $w\in\bD$ which means that $|w|<1$. Taking $z\in\bD$ and applying $F$ gives us the quantity $\frac{w-z}{1-\ov w z}$ which by the previous argument, has modulus less than $1$ whenever $w,z$ do.\par 
            The function $F$ is holomorphic because it is a quotient of holomorphic functions. The denominator is never zero inside the domain because that would mean that $1=\ov w z$. And taking moduli in both sides of the equation gives us 
            $$1=|1|=|w||z|<1$$
            which is impossible.\par 
            Now $F(0)=\frac{w-0}{1-0}=w$ and $F(w)=\frac{w-w}{1-|w|^2}=0$. The denominator in the last expression is never zero because $|w|<1$.\par 
            By the second part of the previous argument it holds that $|z|=1$ immediately gives us $|F(z)|=1$. And finally we will see that $F$ is an involution:
            $$F(F(z))=F\left(\frac{w-z}{1-\ov w z}\right)=\frac{w-\left(\frac{w-z}{1-\ov w z}\right)}{1-\ov w\left(\frac{w-z}{1-\ov w z}\right)}.$$ 
            Homogenizing and clearing denominators we get 
            $$\frac{w(1-\ov wz)-w+z}{1-\ov w z-\ov w(w-z)}=\frac{-w\ov wz+z}{1-\ov ww}=\frac{(-w\ov w+1)z}{1-\ov ww}=z.$$
            This means that $F$ is it's own inverse and therefore, $F$ is bijective. 
        \end{enumerate}
    \end{ptcbr}
\section{Day 1| 20230120}

\subsection{The Complex Numbers}

To construct the complex numbers we take the real numbers, adjoin a variable and mod out by $\genr{x^2+1}$. We can also define $\bC$ as $\set{a+bi:\ a,b\in\bR}$ with the property $i^2=-1$. This means that we can multiply complex numbers in the following way:
$$(a+bi)(c+di)=ac+(bc+ad)i+bdi^2=(ac-bd)+(ad+bc)i.$$
Also as $x^2+1$ is irreducible in $\bR[x]$, $\bC$ is a finite field extension of $\bR$ of degree 2. As a 2-dimensional vector space $\set{1,i}$ is a basis for $\bC$.\par 
The map $a+bi\mapsto\twobyone{a}{b}$ is not a ring homomorphism, it's a bijection with a bit of structure. The map $z\mapsto \al z$, when $\al=a+bi$, is a linear map with the following action over the basis 
\begin{align*}
    &\al\. 1=\al\To[\al]\twobyone{1}{0}=\twobyone{a}{b}\\
    &\al\. i=-b+ai\To[\al]\twobyone{0}{1}=\twobyone{-b}{a}
\end{align*}
which means that $[\al]=\twobytwo{a}{-b}{b}{a}$. The converse, if we have a $\bR$-linear transformation, then it's $\bC$-linear if and only if it looks like $\twobytwo{a}{-b}{b}{a}$.

\begin{Def}
The \term{complex conjugation} map is $a+bi\mapsto a-bi$, or $z\mapsto\ov z$.
\end{Def}

This map is $\bR$-linear but not $\bC$-linear. 

\begin{Ex}
For $\al=a+bi$, we have 
$$\ov{2\al}=\ov{2(a+bi)}=\ov{2a+2bi}=2a-2bi=2\ov{al}.$$
Whereas if instead 
$$\ov{i\al}=\ov{ai-b}=-b-ai\neq i\ov{\al}=b+ai.$$
\end{Ex}

As a $\bR$-linear map, we can identify with the matrix $\twobytwo{1}{0}{0}{-1}$. By looking at the shape of this matrix we can see that it is not $\bC$-linear.

\begin{Lem}
The map $z\mapsto\ov z$ is a ring homomorphism
\end{Lem}

\begin{ptcbp}
$\ov{z+w}=\ov z+\ov w$ and $\ov{zw}=\ov z\ov w$.
\end{ptcbp}

With the complex conjugation we can pick out the real and imaginary parts of $\al=a+bi$. 
$$\al+\ov\al=2\Re(\al),\quad \al-\ov\al=2i\Im(\al)$$
\subsubsection{A Notion of Size}
Can't do geometry without one. Notice that for $z=a+bi$
$$z\ov z=a^2+b^2>0.$$
From a complex number we have extracted a positive quantity.

\begin{Def}
    The \term{complex modulus} of $z$ is $|z|=\sqrt{z\ov z}$.
\end{Def}

The fact that every number has $n$ roots is very important in complex analysis.\par 
As a vector in the plane, the norm of $z$ is $|z|$
\begin{center}
    INC FIG
\end{center}
This means that $a+bi\mapsto\twobyone{a}{b}$ is an isometry. In this sense the distance between two complex numbers is $d(z,w)=|z-w|$.

\subsubsection{Polar Coordinates (\emph{ad hoc})}

For $\te\in\bR$, define 
$$\exp(i\te)=e^{i\te}=\cos(\te)+i\sin(\te)\To |\exp(i\te)|=\sqrt{\cos^2(\te)+\sin^2(\te)}=1.$$
Every point in the unit circle is of the form $e^{i\te}$ and vice-versa.
\begin{center}
    INC FIG
\end{center}
For non-zero complex numbers, $z=|z|e^{i\te}$ for some $\te$.

\begin{Def}
    For a complex number $z=re^{i\te}$, an \term{argument} of $z$ is $\te$. 
\end{Def}
To have a well defined function, we mod out by multiples of $2\pi$: $$\arg:\bC\less\set{0}\to\quot{\bR}{2\pi\bZ},$$
and we obtain a group isomorphism. In general, ``lengths multiply, angles add.''\par 
For inverses if $z=re^{i\te}$, then $\frac{1}{z}=\frac1re^{-i\te}$.

\begin{Def}
    The \term{upper-half plane} is $\bH=\set{\Im(z)>0}$.
\end{Def}

\begin{Lem}
    If $H$ is a half plane $\Im(z-\bt/\ga)>0$
\end{Lem}

\section{Day 2| 20230123}

Recall the complex conjugation map and the modulus of a complex number. This gives us an isometry between $\bR^2$ and $\bC$. Let us prove the lemma from last time. 

\begin{Lem}
    If $H\subseteq\bC$ is a half plane, then there exist $\bt,\ga\in\bC$ such that 
    $$H=\Set{z:\ \Im\left(\frac{z-\bt}{\ga}\right)>0}.$$
\end{Lem}
\begin{center}
    INC FIG
\end{center}
Pick a point $\bt\in H$, then translate $H$ to the origin by $z\mapsto z-\bt$. The plane is now rotated by $\te$ at the origin so we should rotate every point. Then $z\in H-\bt$ whenever $ze^{-i\te}\in\bH$. \red{REDO}\par 
Let us see an application, for a polynomial, the coefficients determine the roots. The following lemma is a technical lemma.

\begin{Lem}
    Suppose $p\in\bC[z]$ and $H$ is a half plane which contains all the roots of $p$. Then $H$ contains all the roots of $p'$.
\end{Lem}

\begin{ptcbp}
    We can assume $p$ is monic, so suppose $\al_1,\dots,\al_d$ are the roots of $\bC$. This means that 
    $$p(z)=\prod_{k=1}^d(z-\al_k)\To p'(z)=\sum_{k=1}^d\frac{p(z)}{z-\al_k}\To\frac{p'(z)}{p(z)}=\sum_{k=1}^d\frac{1}{z-\al_k}.$$
    Now suppose that $H$ contains all $\al_k$ and suppose $z_0\not\in H$, if we show $p'(z_0)\neq 0$ we are done because all the points which make $p'$ vanish won't be outside $H$.\par 
    Describe $H$ by the previous lemma, there exist $\bt,\ga$ such that points in $H$ satisfy the inequality $\Im\left(\frac{z-\bt}{\ga}\right)>0$. As $z_0$ is not in $H$, then $\Im\left(\frac{z_0-\bt}{\ga}\right)<0$. For each $k\in[d]$, we have that 
    $$z_0-\al_k=z_0-\bt+\bt-\al_k=(z_0-\bt)-(\al_k-\bt)$$
    so by taking imaginary parts 
    $$\Im\left(\frac{z-\al_k}{\ga}\right)=\Im\left(\frac{z-\al_k}{\ga}\right)-\Im\left(\frac{z-\al_k}{\ga}\right)$$ 
    The quantity on the right is negative because it's a negative number minus a positive. So it holds that $\Im\left(\frac{\ga}{z-\al_k}\right)>0$. With this we can calculate the following:
    $$\Im\left(\ga\frac{p'(z_0)}{p(z_0)}\right)=\Im\left(\sum_{k=1}^d\frac{\ga}{z_0-\al_k}\right)>0$$
    so in particular this number is non-zero. Thus $p'(z_0)\neq 0$ 
\end{ptcbp}

\begin{Def}
    A set $S\subseteq\bR^n$ is \term{convex} if for any two points $x,y\in S$, the line segment between $x$ and $y$ is also contained in $S$. This is 
    $$\set{ty+(1-t)x:\ x,y\in S}\subseteq S.$$
    The \term{convex hull} of $S$ is the intersection of all convex sets containing $S$. 
\end{Def}

In the case of a finite set of complex numbers, the convex hull can be found by intersecting half-planes which contain them.

\begin{Cor}[Gauss-Lucas]
The roots of $p'(z)$ are contained in the convex hull of the roots of $p(z)$. 
\end{Cor}

\subsection{Metric Spaces}

\begin{Def}
    A \term{metric space} is a set with a distance function.
\end{Def}

\begin{Ex}
    $\bR^n$ is a metric space with $d(x,y)=\norm{x-y}$. Subsets of metric spaces with an induced distance are metric spaces. 
\end{Ex}

\begin{itemize}
    \item nbhd
    \item open and closed
    \item Cauchy
\end{itemize}

\begin{Def}
    Cauchy sequence
\end{Def}
%%%%%%%%%%%% Contents end %%%%%%%%%%%%%%%%
\ifx\nextra\undefined
\printindex
\else\fi
\nocite{*}
\bibliographystyle{plain}
\bibliography{bibiComAnal.bib}
\end{document} 
