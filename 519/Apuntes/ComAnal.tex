\documentclass[12pt]{memoir}

\def\nsemestre {I}
\def\nterm {Spring}
\def\nyear {2023}
\def\nprofesor {Jeff Achter}
\def\nsigla {MATH519}
\def\nsiglahead {Complex Analysis}
\def\nlang {ENG}

\makeatletter
\ifx \nauthor\undefined
  \def\nauthor{Ignacio Rojas}
\else
\fi

\ifx \nextra \undefined
\ifx \nlang \undefined
\author{Basado en las clases impartidas por \nprofesor \\\small Notas tomadas por \nauthor}
\else
\author{Based on the lectures by \nprofesor \\\small Notes written by \nauthor}
\fi
\else
\author{\nauthor}
\fi
\date{\nterm\ \nyear}

%%%%%%%%%%%%%
%% 1. Pacotes
%%%%%%%%%%%%%

\usepackage{alltt}
\usepackage{amsfonts}
\usepackage{amsmath}
\usepackage{amssymb}
\usepackage{amsthm}
\usepackage{algorithm}
\usepackage[noend]{algpseudocode}
\usepackage{array}
\newcommand\hmmax{0} % default 3
\newcommand\bmmax{0} % default 4 %%tex.se/3676,219310
%\usepackage{bbold}
\usepackage{bm}
\usepackage{booktabs}
%\usepackage{caption}
%\usepackage{cancel}
%\usepackage{dsfont}
\usepackage{esint}
\usepackage{fancyhdr}
\usepackage{graphicx}
\usepackage[utf8]{inputenc}
\usepackage{listings}
\usepackage{mathabx}
\usepackage[cal=euler]{mathalfa}
%\usepackage[cal=euler,frak=euler]{mathalfa} % mathcal (JIRR) precisabamos correr initexmf --mkmaps en cmd JCVDG
\usepackage{mathdots}
\usepackage{mathrsfs}
%\usepackage{mathtools}
\usepackage{microtype}
\usepackage{multicol}
\usepackage{multirow}
\usepackage[theoremfont,largesc,tighter,osf]{newpxtext} %JCV Diff
\let\widering\undefined
%\usepackage[bigdelims,vvarbb]{newpxmath} %JCVDG
%por alguna razón esto afectaba las tildes en \min, \lim y demás
%\usepackage{pdflscape}
\usepackage{pgfplots}
\usepackage{physics}
\usepackage{siunitx}
\usepackage{slashed}
%\usepackage{stmaryrd}
%\SetSymbolFont{stmry}{bold}{U}{stmry}{m}{n}
%\usepackage{subfigure}
\usepackage{subcaption}
\usepackage{tabularx}
\usepackage[breakable,skins]{tcolorbox}
\usepackage{textcomp} %%JCVDG
\usepackage{tikz}
\usepackage{tkz-euclide}
\usepackage[normalem]{ulem}
\usepackage[all]{xy}
\usepackage{imakeidx}
\ifx \nlang \undefined
\usepackage[spanish]{babel}
\else\fi 
\usepackage{wrapfig}

%%%%%%%%%%%%%%%%%%%%
%% 2. Document Setup
%%%%%%%%%%%%%%%%%%%%

\ifx \nextra \undefined
    \ifx \nlang \undefined
    \makeindex[intoc, title=Índice Analítico] %Título de índice analítico
    %El índice general es aquel en el que se indican los capítulos, títulos y subtítulos del libro.
    %Índice onomástico es donde aparece el nombre de personas mencionadas en el texto, por orden alfabético con el número de las páginas donde aparecen.
    %El índice analítico se refiere a los temas y conceptos que aparecen en el libro
    \indexsetup{othercode={\fancyhead[LE]{\emph{Índice Analítico}}}}
    \else
    \makeindex[intoc, title=Index] 
    \indexsetup{othercode={\fancyhead[LE]{\emph{Index}}}}
    \fi
  \usepackage[pdftex,
    hidelinks,
    pdfauthor={\nauthor},
    pdfsubject={Notas: \nsiglahead\ \nsemestre-\nyear},
    pdftitle={Semestre \nsemestre\ - \nsigla},
  pdfkeywords={UCR Costa Rica Matem\'aticas Mate \nsemestre\ \nterm\ \nyear\ \nsiglahead}]{hyperref}
  \title{\nsigla\ --- \nsiglahead}
\else
  \usepackage[pdftex,
     hidelinks,
    pdfauthor={\nauthor},
    pdfsubject={\nextra \nsiglahead\ \nsemestre-\nyear},
    pdftitle={Semestre \nsemestre\ - \nsigla},
  pdfkeywords={UCR Costa Rica Matem\'aticas Mate \nsemestre\ \nterm\ \nyear\ \nsiglahead\ \nextra}]{hyperref}

  \title{\nsigla\ --- \nsiglahead \\ {\Large \nextra}}
  \renewcommand\printindex{}
\fi

\pgfplotsset{compat=1.12}


\pagestyle{fancy}
\setlength{\headheight}{15.72pt} %preceding warning said make it at least this


\ifx \nsiglahead \undefined
\def\nsiglahead{\nsigla}
\fi

\lhead{} %%%empty lhead
\rfoot{\thepage}

\ifx \nextra \undefined
  \chead{
    \ifnum\thepage=1
    \else
      \ifx \nlang \undefined
      \textbf{Notas \nsiglahead\ \nsemestre-\nyear}
      \else
      \textbf{Notes \nsiglahead\ \nsemestre-\nyear}
      \fi
    \fi}
  \rhead{}%\firstxmark} % Top right header
\else
%    \chead{
%    \ifnum\thepage=1
%    \else
%      \textbf{Notas \nsiglahead\ \nsemestre-\nyear \ (\nextra)}
%    \fi}
     \chead{
       \textbf{\nextra\ \nsigla\ \nsemestre-\nyear}
     }
     \rhead{
       \textbf{\nauthor}
     }
\fi
\lfoot{}%\lastxmark} % Bottom left footer
\cfoot{} % Bottom center footer

\usetikzlibrary{arrows.meta}
\usetikzlibrary{decorations.markings}
\usetikzlibrary{decorations.pathmorphing}
\usetikzlibrary{positioning}
\usetikzlibrary{fadings}
\usetikzlibrary{intersections}
\usetikzlibrary{cd}

\ifx \nhtml \undefined
\else
  \renewcommand\printindex{}
  \DisableLigatures[f]{family = *}
  \let\Contentsline\contentsline
  \renewcommand\contentsline[3]{\Contentsline{#1}{#2}{}}
  \renewcommand{\@dotsep}{10000}
  \newlength\currentparindent
  \setlength\currentparindent\parindent

  \newcommand\@minipagerestore{\setlength{\parindent}{\currentparindent}}
  \usepackage[active,tightpage,pdftex]{preview}
  \renewcommand{\PreviewBorder}{0.1cm}

  \newenvironment{stretchpage}%
  {\begin{preview}\begin{minipage}{\hsize}}%
    {\end{minipage}\end{preview}}
  \AtBeginDocument{\begin{stretchpage}}
  \AtEndDocument{\end{stretchpage}}

  \newcommand{\@@newpage}{\end{stretchpage}\begin{stretchpage}}

  \let\@real@section\section
  \renewcommand{\section}{\@@newpage\@real@section}
  \let\@real@subsection\subsection
  \renewcommand{\subsection}{\@ifstar{\@real@subsection*}{\@@newpage\@real@subsection}}
\fi
\ifx \ntrim \undefined
\usepackage[shortlabels]{enumitem} %mfw package order matters por savetrees
\else
  \usepackage{geometry}
  \geometry{
    papersize={379pt, 699pt},
    textwidth=345pt,
    textheight=596pt,
    left=17pt,
    top=54pt,
    right=17pt
  }
  \headwidth=345pt
 \usepackage[extreme]{savetrees}
\fi

\ifx \darktheme\undefined
\else
\pagecolor[rgb]{0.2,0.231,0.302}%{0.23,0.258,0.321}
\color[rgb]{1,1,1}
\fi

\ifx \nextra \undefined
\let\@real@maketitle\maketitle
\renewcommand{\maketitle}{\@real@maketitle\begin{center}\begin{minipage}[c]{0.9\textwidth}\centering\footnotesize 
  \ifx \nlang \undefined
  Estas notas no están respaldadas por los profesores y han sido modificadas (a menudo de manera significativa) después de las clases. No están lejos de ser representaciones precisas de lo que realmente se dio en clase y en particular todos los errores son casi seguramente míos.
  \else 
  Please note that these notes were not provided or endorsed by the lecturer and have been significantly altered after the class. They may not accurately reflect the content covered in class and any errors are solely my responsibility.
  \fi
\end{minipage}\end{center}}
\else
\fi

\def\moverlay{\mathpalette\mov@rlay}
\def\mov@rlay#1#2{\leavevmode\vtop{%
   \baselineskip\z@skip \lineskiplimit-\maxdimen
   \ialign{\hfil$\m@th#1##$\hfil\cr#2\crcr}}}
\newcommand{\charfusion}[3][\mathord]{
    #1{\ifx#1\mathop\vphantom{#2}\fi
        \mathpalette\mov@rlay{#2\cr#3}
      }
    \ifx#1\mathop\expandafter\displaylimits\fi}

%%%%%%%%%%%%%%%%%%%%%%%%%%%%%%
%% 2.1 Some internal machinery
%%%%%%%%%%%%%%%%%%%%%%%%%%%%%%

\makeatletter
\renewcommand{\section}{\@startsection{section}{1}{\z@}%
							 {-3.25ex \@plus -1ex \@minus -.2ex}%
							 {1.5ex \@plus.2ex}%
							 {\normalfont\large\bfseries}}
\renewcommand{\subsection}{\@startsection{subsection}{2}{\z@}%
							 {-3.25ex \@plus -1ex \@minus -.2ex}%
							 {1.5ex \@plus .2ex}%
               {\normalfont\normalsize\bfseries}}
\newcommand*{\defeq}{\!\mathrel{\rlap{%
             \raisebox{0.3ex}{$\m@th\cdot$}}%
             \raisebox{-0.3ex}{$\m@th\cdot$}}%
                    =\!}
\makeatother
\ifx\ntrim\undefined
\newcommand{\coursetitle}{\nsigla: \nsiglahead}
\ifx\nextra\undefined
\pagestyle{ruled}
\makeoddhead{ruled}{\coursetitle}{}{\rightmark}
\else\fi
\settypeblocksize{49pc}{37pc}{*}
\setlrmargins{*}{*}{1.2}
\setulmargins{*}{*}{0.8}
\setheadfoot{16pt}{30pt}
\setheaderspaces{*}{1.5pc}{1}
\setmarginnotes{1pt}{1pt}{1pt}
\checkandfixthelayout

\setlength{\unitlength}{3pt}
\setlength{\hfuzz}{1pt}

\setlength{\fboxsep}{6pt}

\setlength{\footskip}{17pt}

\linespread{1.1}
\else\fi
\renewcommand{\cftdotsep}{\cftnodots} %%% no dots in ToC
\setpnumwidth{2em}  %%% width of page-number box in ToC


\newcommand{\stophere}{\relax} %% can be changed to `\endinput'
% \newcommand{\stophere}{\endinput} %% can be changed to `\relax'


\DeclareRobustCommand{\qned}{\ifmmode
  \else \leavevmode\unskip\penalty9999 \hbox{}\nobreak\hfill \fi
  \quad\hbox{\qnedsymbol}}
\newcommand{\qnedsymbol}{$\boxminus$} %% No-proofs end with `\qned'

\DeclareRobustCommand{\qef}{\ifmmode
  \else \leavevmode\unskip\penalty9999 \hbox{}\nobreak\hfill \fi
  \quad\hbox{\qefsymbol}}
\newcommand{\qefsymbol}{$\lozenge$} %% Examples end with `\qef'
\def\enddefn{\qef\endtrivlist}      %% `\qef' automático en defns
\def\endejem{\qef\endtrivlist}      %% `\qef' automático en ejemplos

\newcommand{\hideqed}{\renewcommand{\qed}{}} %% to suppress `\qed'
\newcommand{\hideqef}{\renewcommand{\qef}{}} %% to suppress `\qef'

% \newcommand{\ldbrack}{\ensuremath{[\mskip-2.5mu[}} %% corchetes [[
% \newcommand{\rdbrack}{\ensuremath{]\mskip-2.5mu]}} %% corchetes ]]

\newcommand{\stroke}{\mathbin|}     %% (for `\bbraket' and such)

\newcommand{\rtri}{\blacktriangleright} %% (for `\marker' and such)
\newcommand{\tribar}{|\mkern-2mu|\mkern-2mu|} %% norma triple: |||


%% Formatting changes:

\renewcommand{\labelitemi}{$\diamond$} %% instead of bullets

\renewcommand{\theenumi}{\alph{enumi}}  %% use lowercase letters
\renewcommand{\labelenumi}{\textup{(\theenumi)}} %% inside parentheses

%%%%%%%%%%%%%%
%% 2.2. Colors
%%%%%%%%%%%%%%

\definecolor{MATLABgreen}{RGB}{28,172,0} % color values Red, Green, Blue
\definecolor{MATLABlila}{RGB}{170,55,241}
\definecolor{dankBlue}{RGB}{51,60,77} % color values Red, Green, Blue
\definecolor{dankBlueLite}{RGB}{82,97,125} % color values Red, Green, Blue
\definecolor{celesUCR}{RGB}{0,192,243}
\definecolor{azulUCR}{RGB}{0,93,164}
\definecolor{verdeUCR}{RGB}{109,192,103}
\definecolor{yelloUCR}{RGB}{255,224,106}

%%%%%%%%%%%%%%%%%%%%%%%%%%%
%% 3. Theorems and suchlike
%%%%%%%%%%%%%%%%%%%%%%%%%%%

\ifx\nlang\undefined

\theoremstyle{plain}
\ifx \nextra \undefined
\newtheorem{Th}{Teorema}[section]      %%% Theorem 1.1.1
\newtheorem{Tmon}[Th]{Teoremón}
\newtheorem{Prop}[Th]{Proposición}     %%% Proposition 1.1.2
\newtheorem{Lem}[Th]{Lema}             %%% Lemma 1.1.3
\newtheorem{Cor}[Th]{Corolario}        %%% Corollary 1.1.4
\else
\newtheorem{Th}{Teorema}               %%% Theorem 1.1.1
\newtheorem{Tmon}{Teoremón}
\newtheorem{Prop}{Proposición}         %%% Proposition 1.1.2
\newtheorem{Lem}{Lema}                 %%% Lemma 3
\newtheorem{Cor}{Corolario}            %%% Corollary 4
\fi
\newtheorem*{nonum-Th}{Teorema}        %%% No-numbered Theorem
\newtheorem*{nonum-Cor}{Corolario}     %%% No-numbered Corollary

\theoremstyle{definition}
\ifx \nextra \undefined
\newtheorem{Def}[Th]{Definición}       %%% Definition 1.1.5
\newtheorem{Ex}[Th]{Ejemplo}           %%% Example 1.1.6
\newtheorem{Ej}[Th]{Ejercicio}         %%% Ejercicio 1.1.7
\else
\newtheorem{Def}{Definición}           %%% Definition 5
\newtheorem{Ex}{Ejemplo}               %%% Example 6
\newtheorem{Ej}{Ejercicio}             %%% Ejercicio 7
\fi
\newtheorem{Hec}[Th]{Hecho}            %%% Hecho 1.1.8
\newtheorem*{nonum-Def}{Definición}    %%% No number Definition
\newtheorem*{nonum-Ex}{Ejemplo}        %%% No number Example
\newtheorem*{nonum-Ej}{Ejercicio}      %%% No number Ejercicio
\newtheorem*{nonum-Hec}{Hecho}         %%% No number Fact


\theoremstyle{remark}
\newtheorem{Rmk}[Th]{Observación}      %%%Remark 1.1.9
\newtheorem*{nonum-Rmk}{Observación}         %%% No number Fact
\newtheorem*{Notn}{Notaci\'on}        %% Notaciones
\newtheorem*{Warn}{Advertencia}       %% Advertencias
\newtheorem*{Qn}{Pregunta}            %% Pregunta

\else

\theoremstyle{plain}
\ifx \nextra \undefined
\newtheorem{Th}{Theorem}[section]      %%% Theorem 1.1.1
\newtheorem{Tmon}[Th]{Teoremón}
\newtheorem{Prop}[Th]{Proposition}     %%% Proposition 1.1.2
\newtheorem{Lem}[Th]{Lemma}             %%% Lemma 1.1.3
\newtheorem{Cor}[Th]{Corollary}        %%% Corollary 1.1.4
\else
\newtheorem{Th}{Theorem}               %%% Theorem 1.1.1
\newtheorem{Tmon}{Teoremón}
\newtheorem{Prop}{Proposition}         %%% Proposition 1.1.2
\newtheorem{Lem}{Lemma}                 %%% Lemma 3
\newtheorem{Cor}{Corollary}            %%% Corollary 4
\fi
\newtheorem*{nonum-Th}{Theorem}        %%% No-numbered Theorem
\newtheorem*{nonum-Cor}{Corollary}     %%% No-numbered Corollary

\theoremstyle{definition}
\ifx \nextra \undefined
\newtheorem{Def}[Th]{Definition}       %%% Definition 1.1.5
\newtheorem{Ex}[Th]{Example}           %%% Example 1.1.6
\newtheorem{Ej}[Th]{Exercise}         %%% Exercise 1.1.7
\else
\newtheorem{Def}{Definition}           %%% Definition 5
\newtheorem{Ex}{Example}               %%% Example 6
\newtheorem{Ej}{Exercise}             %%% Exercise 7
\fi
\newtheorem{Hec}[Th]{Fact}            %%% Fact 1.1.8
\newtheorem*{nonum-Def}{Definition}    %%% No number Definition
\newtheorem*{nonum-Ex}{Example}        %%% No number Example
\newtheorem*{nonum-Ej}{Exercise}      %%% No number Exercise
\newtheorem*{nonum-Hec}{Fact}         %%% No number Fact


\theoremstyle{remark}
\newtheorem{Rmk}[Th]{Remark}      %%%Remark 1.1.9
\newtheorem*{nonum-Rmk}{Remark}         %%% No number Fact
\newtheorem*{Notn}{Notation}        %% Notaciones
\newtheorem*{Warn}{Warning}       %% Warnings
\newtheorem*{Qn}{Question}            %% Question

\fi 

\numberwithin{equation}{section}

\setlength{\parindent}{3ex}

% \renewcommand{\labelitemi}{--}
% \renewcommand{\labelitemii}{$\circ$}
% \renewcommand{\labelenumi}{(\roman{*})}

%\let\stdsection\section
%\renewcommand\section{\newpage\stdsection}

\newcommand\qedsym{\hfill\ensuremath{\square}}
% Strike through
\def\st{\bgroup \ULdepth=-.55ex \ULset}

%%%%%%%%% === My T Color Box === %%%%%%%%%%%%%%

\ifx\nlang\undefined
\ifx \darktheme\undefined
\newtcolorbox{ptcb}{
colframe = black,
colback = white,
breakable,
enhanced
}
\newtcolorbox{ptcbp}{
colframe = black,
colback = white,
coltitle = black,
colbacktitle = black!40,
title = Prueba,
breakable,
enhanced
}
\newtcolorbox{ptcbr}{
colframe = blue,
colback = white,
coltitle = blue,
colbacktitle = blue!40,
title = Respuesta,
breakable,
enhanced
}
\else
\newtcolorbox{ptcb}{
colframe = white,
colback = dankBlue,
colupper = white,
breakable,
enhanced
}
\newtcolorbox{ptcbp}{
colframe = white,
colback = dankBlue,
colupper = white,
coltitle = white,
colbacktitle = dankBlueLite,
title = Prueba,
breakable,
enhanced
}
\newtcolorbox{ptcbr}{
colframe = white,
colback = white,
coltitle = blue,
colbacktitle = blue!40,
title = Respuesta,
breakable,
enhanced
}
\fi

\else
\ifx \darktheme\undefined
\newtcolorbox{ptcb}{
colframe = black,
colback = white,
breakable,
enhanced
}
\newtcolorbox{ptcbp}{
colframe = black,
colback = white,
coltitle = black,
colbacktitle = black!40,
title = Proof,
breakable,
enhanced
}
\newtcolorbox{ptcbr}{
colframe = blue,
colback = white,
coltitle = blue,
colbacktitle = blue!40,
title = Answer,
breakable,
enhanced
}
\else
\newtcolorbox{ptcb}{
colframe = white,
colback = dankBlue,
colupper = white,
breakable,
enhanced
}
\newtcolorbox{ptcbp}{
colframe = white,
colback = dankBlue,
colupper = white,
coltitle = white,
colbacktitle = dankBlueLite,
title = Proof,
breakable,
enhanced
}
\newtcolorbox{ptcbr}{
colframe = white,
colback = white,
coltitle = blue,
colbacktitle = blue!40,
title = Answer,
breakable,
enhanced
}
\fi
\fi


%%%%%%%%% === Listings === %%%%%%%%%%%%%%
\lstset{basicstyle=\ttfamily,breaklines=true}

\lstset{language=Matlab,%
    %basicstyle=\color{red},
    breaklines=true,%
    morekeywords={matlab2tikz},
    keywordstyle=\color{blue},%
    morekeywords=[2]{1}, keywordstyle=[2]{\color{black}},
    identifierstyle=\color{black},%
    stringstyle=\color{MATLABlila},
    commentstyle=\color{MATLABgreen},%
    showstringspaces=false,%without this there will be a symbol in the places where there is a space
    numbers=left,%
    numberstyle={\tiny \color{black}},% size of the numbers
    numbersep=9pt, % this defines how far the numbers are from the text
   % emph=[1]{for,end,break,function,if,elseif,else},emphstyle=[1]\color{blue}, %some words to emphasise
    %emph=[2]{word1,word2}, emphstyle=[2]{style},
}

%%%%%%%%%%%%%%%%%%%%%%%%%%
%% 4. Simple abbreviations
%%%%%%%%%%%%%%%%%%%%%%%%%%

%%% Operator names:

\DeclareMathOperator{\area}{area}
\DeclareMathOperator{\card}{card}
\DeclareMathOperator{\ccl}{ccl}
\DeclareMathOperator{\ch}{ch}
\DeclareMathOperator{\cl}{cl}
\DeclareMathOperator{\coker}{coker}
\DeclareMathOperator{\Conv}{Conv}   %%Convex hull
\DeclareMathOperator{\cosec}{cosec}
\DeclareMathOperator{\cosech}{cosech}
\DeclareMathOperator{\covol}{covol}
\DeclareDocumentCommand\curl{}{\operatorname{curl}} 
\DeclareMathOperator{\diag}{diag}
\DeclareMathOperator{\diam}{diam}
\DeclareMathOperator{\Diff}{Diff}
\DeclareDocumentCommand\div{}{\operatorname{div}} 
\DeclareMathOperator{\energy}{energy}
\DeclareMathOperator{\erfc}{erfc}
\DeclareMathOperator{\Ext}{Ext}
\DeclareMathOperator{\fst}{fst}
\DeclareMathOperator{\Fit}{Fit}
\DeclareMathOperator{\gr}{gr}
\DeclareMathOperator{\hcf}{hcf}
\DeclareMathOperator{\Hilb}{Hilb} %Hilbert scheme
\DeclareMathOperator{\id}{id}
\DeclareMathOperator{\Ind}{Ind}
\DeclareMathOperator{\Int}{Int}
\DeclareMathOperator{\Isom}{Isom}
\DeclareMathOperator{\lcm}{lcm}
\DeclareMathOperator{\length}{length}
\DeclareMathOperator{\Lie}{Lie}
\DeclareMathOperator{\like}{like}
\DeclareMathOperator{\Lk}{Lk}
\DeclareMathOperator{\Maps}{Maps}
\DeclareMathOperator{\mcd}{mcd}
\DeclareMathOperator{\mcm}{mcm}
\DeclareMathOperator{\Min}{Min}
\DeclareMathOperator{\orb}{orb}
\DeclareMathOperator{\ord}{ord}
\DeclareMathOperator{\otp}{otp}
\DeclareMathOperator{\pr}{pr}       %% proyector
\DeclareMathOperator{\poly}{poly}
\DeclareMathOperator{\rel}{rel}
\DeclareMathOperator{\Rad}{Rad}
\DeclareMathOperator*{\res}{res}
\DeclareMathOperator{\Ric}{Ric}
\DeclareMathOperator{\rk}{rk}
\DeclareMathOperator{\Rees}{Rees}
\DeclareMathOperator{\Root}{Root}
\DeclareMathOperator{\rot}{rot}         %% rotacional
\DeclareMathOperator{\spn}{span}
\DeclareMathOperator{\St}{St}
\DeclareMathOperator{\supp}{supp}
\DeclareMathOperator{\Syl}{Syl}
\DeclareMathOperator{\Sym}{Sym}
\DeclareMathOperator{\vol}{vol}

% not-math
\newcommand{\bolds}[1]{{\bfseries #1}}
\newcommand{\cat}[1]{\mathsf{#1}}
\newcommand{\ph}{\,\cdot\,}
\newcommand{\term}[1]{\un{#1}\index{#1}}
\newcommand{\phantomeq}{\hphantom{{}={}}}
\newcommand{\ttt}{\texttt}
\newcommand{\red}[1]{\textcolor{red}{#1}}
\newcommand{\prp}[1]{\textcolor{purple}{#1}}
\newcommand{\blu}[1]{\textcolor{azulUCR}{#1}}
\newcommand{\green}[1]{\textcolor{verdeUCR}{#1}}
\newcommand{\yelo}[1]{\textcolor{yelloUCR}{#1}}
\newcommand{\cele}[1]{\textcolor{celesUCR}{#1}}

%functions
\DeclareMathOperator{\sgn}{sgn}
\newcommand*{\Cdot}{{\raisebox{-0.25ex}{\scalebox{1.5}{$\cdot$}}}}      %% cdot más grande
\newcommand{\ind}{\mathbf{1}}       %%%indicator function
\newcommand{\mm}{\mathfrak{m}}      %%%metric


% Greek letters:

\newcommand{\al}{\alpha}                %% short for  \alpha
\newcommand{\bt}{\beta}                 %% short for  \beta
\newcommand{\Dl}{\Delta}                %% short for  \Delta
\newcommand{\dl}{\delta}                %% short for  \delta
\newcommand{\eps}{\varepsilon}          %% short for  \varepsilon
\newcommand{\Ga}{\Gamma}                %% short for  \Gamma
\newcommand{\ga}{\gamma}                %% short for  \gamma
\newcommand{\kp}{\kappa}                %% short for  \kappa
\newcommand{\La}{\Lambda}               %% short for  \Lambda
\newcommand{\la}{\lambda}               %% short for  \lambda
\newcommand{\Om}{\Omega}                %% short for  \Omega
\newcommand{\om}{\omega}                %% short for  \omega
\newcommand{\Sg}{\Sigma}                %% short for  \Sigma
\newcommand{\sg}{\sigma}                %% short for  \sigma
\newcommand{\Te}{\Theta}                %% short for  \Theta
\newcommand{\te}{\theta}                %% short for  \theta
\newcommand{\ups}{\upsilon}             %% short for  \upsilon
\newcommand{\vf}{\varphi}               %% short for  \varphi
\newcommand{\ze}{\zeta}                 %% short for  \zeta
\newcommand{\vsg}{\varsigma}            %% short for  \varsigma
\newcommand{\vte}{\vartheta}            %% short for  \vartheta

%Boldface letters

\newcommand{\bA}{\mathbb{A}}        %% antisimetrizador
\newcommand{\bB}{\mathbb{B}}        %% bola unitaria
\newcommand{\bC}{\mathbb{C}}    %%% números complejos
\newcommand{\bCP}{\mathbb{CP}}  %%% espacio proyectivo complejo
\newcommand{\bD}{\mathbb{D}}        %% Poincaré disk
\newcommand{\bE}{\mathbb{E}}
\newcommand{\bF}{\mathbb{F}}        %% un cuerpo
\newcommand{\bH}{\mathbb{H}}        %% cuaterniones
\newcommand{\bI}{\mathbb{I}}        %% ideal de zeros
\newcommand{\bK}{\mathbb{K}}            %% ein korper
\newcommand{\bN}{\mathbb{N}}    %%% números naturales
\newcommand{\bP}{\mathbb{P}}        %% números enteros positivos
\newcommand{\bQ}{\mathbb{Q}}    %%% números racionales
\newcommand{\bR}{\mathbb{R}}    %%% números reales
\newcommand{\bRP}{\mathbb{RP}}  %%% espacio proyectivo real
\newcommand{\bS}{\mathbb{S}}    %%% esfera
\newcommand{\bT}{\mathbb{T}}        %% círculo o toro
\newcommand{\bV}{\mathbb{V}}        %% lugar geométrico de ceros
\newcommand{\bZ}{\mathbb{Z}}    %%% números enteros

%Script letters:

\newcommand{\cA}{\mathcal{A}}           %% formas diferenciales
\newcommand{\cB}{\mathcal{B}}           %% una base vectorial
\newcommand{\cC}{\mathcal{C}}           %% otra base vectorial
\newcommand{\cD}{\mathcal{D}}           %% funciones de prueba
\newcommand{\cE}{\mathcal{E}}           %% un modulo proyectivo
\newcommand{\cF}{\mathcal{F}}           %% espacio de Fock
\newcommand{\cG}{\mathcal{G}}           %% funtor de Gelfand
\newcommand{\cH}{\mathcal{H}}           %% espacio de Hilbert
\newcommand{\cI}{\mathcal{I}}           %% un funtor de inclusion
\newcommand{\cJ}{\mathcal{J}}           %% otro funtor
\newcommand{\cK}{\mathcal{K}}           %% otro espacio de Hilbert
\newcommand{\cL}{\mathcal{L}}           %% operadores lineales
\newcommand{\cM}{\mathcal{M}}           %% multiplicadores
\newcommand{\cN}{\mathcal{N}}           %% funciones nulas
\newcommand{\cO}{\mathcal{O}}           %% funciones de crec-to lento
\newcommand{\cP}{\mathcal{P}}           %% una particion
\newcommand{\cR}{\mathcal{R}}           %% funciones representativas
\newcommand{\cQ}{\mathcal{Q}}           %% otra particion
\newcommand{\cS}{\mathcal{S}}           %% funciones de Schwartz
\newcommand{\cT}{\mathcal{T}}           %% una topologia
\newcommand{\cU}{\mathcal{U}}           %% cubrimiento abierto
\newcommand{\cV}{\mathcal{V}}           %% vecindarioas
\newcommand{\cW}{\mathcal{W}}           %% grupo de Weyl
\newcommand{\cZ}{\mathcal{Z}}           %% topología de Zariski

%%% Fraktur letters:

\newcommand{\gA}{\mathfrak{A}}      %% un atlas
\newcommand{\g}{\mathfrak{g}}       %% un álgebra de Lie
\newcommand{\gB}{\mathfrak{B}}      %% otro atlas
\newcommand{\ggl}{\mathfrak{gl}}    %% álg de Lie general lineal
\newcommand{\gsl}{\mathfrak{sl}}    %% álg de Lie especial lineal
\newcommand{\gso}{\mathfrak{so}}    %% álg de Lie especial ortogonal
\newcommand{\gsu}{\mathfrak{su}}    %% álg de Lie especial unitaria
\newcommand{\gX}{\mathfrak{X}}      %% campos vectoriales

%%% Roman letters:

\newcommand{\dR}{\mathrm{dR}}       %% cohomología de de Rham
\newcommand{\rGL}{\mathrm{GL}}      %% grupo general lineal
\newcommand{\rO}{\mathrm{O}}        %% grupo ortogonal
\newcommand{\rSL}{\mathrm{SL}}      %% grupo especial lineal
\newcommand{\rSO}{\mathrm{SO}}      %% grupo ortogonal especial
\newcommand{\rSp}{\mathrm{Sp}}      %% grupo simpléctico
\newcommand{\rSU}{\mathrm{SU}}      %% grupo unitario especial
\newcommand{\rU}{\mathrm{U}}        %% grupo unitario
\newcommand{\rUH}{\mathrm{UH}}      %% cuaterniones unitarias
\newcommand{\rT}{\mathrm{T}}        %% grupo triangular

% Sanserif letters:

\newcommand{\sA}{\mathsf{A}}            %% algebras de Lie A_n
\newcommand{\sB}{\mathsf{B}}            %% grupo como categoria
\newcommand{\sC}{\mathsf{C}}            %% una categoria
\newcommand{\sD}{\mathsf{D}}            %% otra categoria
\newcommand{\sE}{\mathsf{E}}            %% otra categoria mas
\newcommand{\sF}{\mathsf{F}}            %% algebra de Lie F_4
\newcommand{\sG}{\mathsf{G}}            %% algebra de Lie G_2
\newcommand{\sJ}{\mathsf{J}}            %% un poset
\newcommand{\sK}{\mathsf{K}}            %% un poset
\newcommand{\sL}{\mathcal{L}}           %% derivada de Lie
\newcommand{\sN}{\mathsf{N}}            %% categoría con objetos \bN
\newcommand{\sT}{\mathsf{T}}            %% transpuesta

%%% Boldface letters:

\bmdefine{\CC}{C}                       %% C negrilla
\bmdefine{\cc}{c}
%\bmdefine{\dd}{d}                       %% d negrilla
\bmdefine{\ee}{e}                       %% vector e
\bmdefine{\eeps}{\varepsilon}           %% basic form \eps
\bmdefine{\FF}{F}                       %% vector F
\bmdefine{\ff}{f}                       %% vector f
\bmdefine{\ii}{i}                       %% cuaternion i
\bmdefine{\jj}{j}                       %% cuaternion j
\bmdefine{\kk}{k}                       %% cuaternion k
\bmdefine{\lla}{\lambda}                %% sucesion \la
\bmdefine{\mmu}{\mu}                    %% sucesion \mu
\bmdefine{\pp}{p}                       %% vector p
\bmdefine{\qq}{q}                       %% vector q
\bmdefine{\rr}{r}                       %% vector r
\bmdefine{\ssg}{\sigma}                 %% vector \sg
%\bmdefine{\sss}{s}
%\bmdefine{\ttt}{t}
\bmdefine{\VV}{V}                       %% V negrilla
\bmdefine{\xx}{x}                       %% sucesion x
\bmdefine{\xxi}{\xi}                    %% vector \xi
\bmdefine{\yy}{y}                       %% sucesion y
\bmdefine{\zz}{z}                       %% sucesion z

% Matrix groups
\DeclareMathOperator{\GL}{GL}   %%% grupo general lineal
\DeclareMathOperator{\Or}{O}    %%% grupo ortogonal
\DeclareMathOperator{\PGL}{PGL} %%% grupo proyectivo lineal
\DeclareMathOperator{\PSL}{PSL} %%% grupo proyectivo lineal especial
\DeclareMathOperator{\PSO}{PSO} %%% grupo proyectivo ortogonal
\DeclareMathOperator{\PSU}{PSU} %%% grupo proyectivo unitario
\DeclareMathOperator{\SL}{SL}   %%% grupo especial lineal
\DeclareMathOperator{\SO}{SO}   %%% grupo especial ortogonal
\DeclareMathOperator{\SU}{SU}   %%% grupo especial unitario

% Numericc
\newcommand{\argmin}{\text{argm\'in}}
\DeclareMathOperator{\dof}{dof}

%% Brackets
\newcommand{\conj}[1]{\left\lbrace#1\right\rbrace}
\newcommand{\bonj}[1]{\left\lbrack#1\right\rbrack}
\newcommand{\obonj}[1]{\left\rbrack#1\right\lbrack}
\newcommand{\rbonj}[1]{\left\rbrack#1\right\rbrack}
\newcommand{\lbonj}[1]{\left\lbrack#1\right\lbrack}
\newcommand{\snm}[1]{\|#1\|}           %small norma
\newcommand{\nm}[1]{\left\|#1\right\|} %norma pegadita
\newcommand{\pnm}[1]{\biggl|\biggl|#1\biggr|\biggr|}
\let\oldvec=\vec
\renewcommand{\vec}[1]{\mathbf{#1}}
\newcommand\quot[2]{
        \mathchoice
            {% \displaystyle
                \text{\raise1ex\hbox{$#1$}\Big/\lower1ex\hbox{$#2$}}%
            }
            {% \textstyle
                {^{ #1}/_{ #2}}
            }
            {% \scriptstyle
                {^{ #1}/_{ #2}}
            }
            {% \scriptscriptstyle
                {^{ #1}/_{ #2}}
            }
    }
%\newcommand*\quot[2]{{^{\textstyle #1}\big/_{\textstyle #2}}}
\newcommand*\squot[2]{{^{ #1}/_{ #2}}}%%%small quotient
\newcommand{\multinom}[2]{\ensuremath{\left(\kern-.3em\left(\genfrac{}{}{0pt}{}{#1}{#2}\right)\kern-.3em\right)}}

% Probability
\DeclareMathOperator{\Bernoulli}{Bernoulli}
\DeclareMathOperator{\betaD}{beta}
\DeclareMathOperator{\bias}{bias}
\DeclareMathOperator{\binomial}{binomial}
\DeclareMathOperator{\corr}{corr}
\DeclareMathOperator{\cov}{cov}
\DeclareMathOperator{\gammaD}{gamma}
\DeclareMathOperator{\mse}{mse}
\DeclareMathOperator{\multinomial}{multinomial}
\DeclareMathOperator{\Poisson}{Poisson}
\DeclareMathOperator{\Var}{Var}     %%%variance
\DeclareMathOperator{\Cov}{Cov}     %%%Covariance
\renewcommand{\mid}{\;\ifnum\currentgrouptype=16 \middle\fi|\;}

% Combinatorics
\DeclareMathOperator{\ins}{ins}   % insertion tableaux
\DeclareMathOperator{\asc}{asc}   % ascents
\DeclareMathOperator{\rw}{rw}     % reading word
\DeclareMathOperator{\rev}{rev}     % reading word
\DeclareMathOperator{\rect}{rect} % rectification of young tableau
\DeclareMathOperator{\sh}{sh}     % shape of young tableau
\DeclareMathOperator{\std}{std}   % standarization
\DeclareMathOperator{\Fl}{\mathcal{F}\ell}       %% conjunto de Flags
\DeclareMathOperator{\Frob}{Frob} % Frobenius map

% Algebra
\DeclareMathOperator{\Ad}{Ad}       %% acción adjunta
\DeclareMathOperator{\adj}{adj}
\DeclareMathOperator{\Ann}{Ann}     %% aniquilador o anulador de módulos
\DeclareMathOperator{\Ass}{Ass}     %% ideales asociados
\DeclareMathOperator{\Aut}{Aut}
\DeclareMathOperator{\Bl}{\mathcal{B}\!\ell}       %% blowup de un espacio
\DeclareMathOperator{\Char}{char}
\DeclareMathOperator{\codim}{codim}
\DeclareMathOperator{\disc}{disc}
\DeclareMathOperator{\dom}{dom}
\DeclareMathOperator{\End}{End}     %%%space of endomorphisms
\DeclareMathOperator{\Fix}{Fix}
\DeclareMathOperator{\Frac}{Frac}
\DeclareMathOperator{\Gal}{Gal}
\DeclareMathOperator{\gen}{gen}     %%%set generated by...
\DeclareMathOperator{\Gr}{Gr}       %%%Grassmannian
\DeclareMathOperator{\Hom}{Hom}
\DeclareMathOperator{\Hurw}{Hurw}
\DeclareMathOperator{\image}{image}
\DeclareMathOperator{\Mor}{Mor}
\DeclareMathOperator{\Nil}{Nil}
\DeclareMathOperator{\Orb}{Orb}
\DeclareMathOperator{\Pic}{Pic}     %%% grupo de Picard 
\DeclareMathOperator{\Quot}{Quot}
\DeclareMathOperator{\Spec}{Spec}
\DeclareMathOperator{\Stab}{Stab}
\DeclareMathOperator{\Taut}{Taut}

% Analysis
\DeclareMathOperator*{\esssup}{ess\hspace{0.5mm}sup}
\DeclareMathOperator*{\essinf}{ess\hspace{0.5mm}inf}
%\DeclareMathOperator{\Int}{Int}     %%%interior vacilon funcional

\newcommand{\loc}{\text{loc}}
\newcommand{\LB}{\cL_\cB}           %%%bounded linear operator

% Logic
\newcommand{\cleq}{\preccurlyeq}
\newcommand{\cgeq}{\succcurlyeq}

% Others
\renewcommand{\ev}{\operatorname{ev}}     %%%evalutation previously expectation value physics package
\newcommand{\bigcupdot}{\charfusion[\mathop]{\bigcup}{\Cdot}} %%JCVDG
%\renewcommand{\bigcupdot}{\charfusion[\mathop]{\bigcup}{\Cdot}}
\newcommand{\cupdot}{\charfusion[\mathbin]{\cup}{\Cdot}}
\newcommand{\exterior}{\mathchoice{{\textstyle\bigwedge}}{{\bigwedge}}{{\textstyle\wedge}}{{\scriptstyle\wedge}}}
\newcommand{\hol}{\mathfrak{hol}}
\newcommand{\Id}{\mathrm{Id}}
\newcommand{\lie}[1]{\mathfrak{#1}}
\newcommand{\qeq}{\mathrel{``{=}"}}
\newcommand{\wsto}{\stackrel{\mathrm{w}^*}{\to}}
\newcommand{\wt}{\mathrm{wt}}

%\let\Im\relax
%\let\Re\relax

%%% Shorter symbol names:

\newcommand{\bull}{{\scriptstyle\bullet}}  %% vertice en figuras
\newcommand{\del}{\partial}             %% short for  \partial
\newcommand{\downto}{\downarrow}        %% limite a la derecha
\newcommand{\dsp}{\displaystyle}        %% despliegue en texto
\renewcommand{\geq}{\geqslant}          %% mayor o igual (variante)
\newcommand{\hookto}{\hookrightarrow}     %% inclusion arrow
\newcommand{\isom}{\simeq}              %% isomorfismo
\renewcommand{\l}{\ell}                   %% ele cursiva
\renewcommand{\leq}{\leqslant}          %% menor o igual (variante)
\newcommand{\less}{\setminus}           %% set difference
\newcommand{\otto}{\leftrightarrow}     %% bijection
\newcommand{\ox}{\otimes}               %% producto tensorial
\newcommand{\rt}{\triangleleft}         %% un orden parcial
\newcommand{\rteq}{\trianglelefteq}     %% normal subgroup
\newcommand{\up}{{\mathord{\uparrow}}}  %% espinor `up'
\newcommand{\upto}{\uparrow}            %% left hand limit
\newcommand{\w}{\wedge}                 %% producto exterior
\newcommand{\wto}{\rightharpoonup}      %% convergencia debil
\newcommand{\x}{\times}                 %% producto vectorial
\renewcommand{\.}{\Cdot}                %% producto escalar
\renewcommand{\:}{\mathbin{:}}          %% colon in  f: A -> B
\newcommand{\into}{\rightarrowtail}     %% injection arrow
\newcommand{\lr}{\dashv}                %% adjunction
\newcommand{\lt}{\triangleright}        %% a left action
\newcommand{\lteq}{\trianglerighteq}    %% normal supergroup
\newcommand{\nb}{\nabla}                %% homomorfismo de suma
\newcommand{\nisom}{\not\simeq}         %% negacion de isomorfismo
%\newcommand{\oast}{\circledast}         %% variante de * (ya existe en stmaryrd)
\newcommand{\onto}{\twoheadrightarrow}  %% surjection arrow
\newcommand{\opp}{\circ}                %% objeto opuesto
\newcommand{\ottto}{\longleftrightarrow} %% bijection in display
\newcommand{\pullb}{\lrcorner}          %% simbolo de pullback
\newcommand{\pushf}{\ulcorner}          %% simbolo de pushout
\newcommand{\rx}{\rtimes}               %% producto semidirecto
\newcommand{\To}{\Rightarrow}           %% entre funtores
\newcommand{\tofro}{\rightleftarrows}   %% pair of opposed maps
\newcommand{\toto}{\rightrightarrows}   %% pair of parallel maps

\renewcommand{\2}{\flat}                  %% marcador de sucesiones
\newcommand{\3}{\sharp}                 %% marcador de sucesiones
\newcommand{\4}{\natural}               %% marcador de morfismos
% \newcommand{\5}{\diamond}               %% for roots of trees
% \newcommand{\7}{\dagger}                %% adjunto de operador
\newcommand{\8}{\bullet}                %% anonymous degree

%%% Useful abbreviations:

\newcommand{\Coo}{\cC^\infty}         %% funciones suaves
\newcommand{\ctr}{\mathbin{\lrcorner\,}} %% contraction symbol
\newcommand{\nbf}{{\vec\nabla}}     %% short for  \vec\nabla

\newcommand{\as}{\quad\text{cuando}\enspace} %% `cuando' en límites
\newcommand{\bCoo}{{\bC_\infty}}    %% esfera de Riemann
% \newcommand{\bRoo}{{\bR_\infty}}    %% círculo real extendido

%%% Repeated relations:

\newcommand{\cupycup}{\cup\cdots\cup} %% unión repetida
\newcommand{\capycap}{\cap\cdots\cap} %% intersección repetida
\newcommand{\sys}{\subset\cdots\subset}%% subconjunto propio repetido
\newcommand{\subysub}{\subseteq\cdots\subseteq} %%subconjunto repetido
\newcommand{\oxyox}{\otimes\cdots\otimes} %% prod tensorial repetido
\newcommand{\wyw}{\wedge\cdots\wedge} %% producto exterior repetido
\newcommand{\opyop}{\oplus\cdots\oplus} %% suma directa repetida
\newcommand{\xyx}{\times\cdots\times} %% producto directo repetido

%%% Arrows with riders:

\newcommand{\longto}{\mathop{\longrightarrow}\limits}

%%% Small fractions in displays:

\newcommand{\half}{{\mathchoice{\nhalf}{\thalf}{\shalf}{\shalf}}} %%display text script script^2
\newcommand{\happi}{{\tfrac{\pi}{2}}} %% small fraction  \pi/2
\newcommand{\quarter}{\tfrac{1}{4}} %% small fraction  1/4
\newcommand{\nhalf}{\frac{1}{2}}
\newcommand{\shalf}{{\scriptstyle\frac{1}{2}}} %% tiny fraction 1/2
\newcommand{\thalf}{{\tfrac{1}{2}}} %% small fraction  1/2
\renewcommand{\third}{\tfrac{1}{3}}   %% small fraction  1/3 %Hay que renew porque mathabx toma second y third como x'' y x''' por ejemplo

\newcommand{\ihalf}{{\tfrac{i}{2}}} %% small fraction  i/2

%%%%%%%%%%%%%%%%%%%%%%%%%%%%%
%% 5. Commands with arguments
%%%%%%%%%%%%%%%%%%%%%%%%%%%%%

%%% Accent-like commands, abbreviated:

\newcommand{\ov}{\overline}        %% short for  \overline
\newcommand{\un}{\underline}       %% short for  \underline
\newcommand{\wh}{\widehat}          %% short for  \widehat

%%% Separate words in displays:

\newcommand{\word}[1]{\quad\text{#1}\quad} %% texto intercalado

%%% Webpage locator:

\newcommand{\zelda}[1]{$\langle${\footnotesize\texttt{#1}}$\rangle$}

%% Symbol placement:

\newcommand{\pre}[1]{{}^{#1\!}} %% upper left exponent

%%% Proof-part labels:

\newcommand{\Adiff}[2]{\ensuremath{\Ad\,(\mathrm{#1})\Longleftrightarrow
    (\mathrm{#2})}:\enspace}
\newcommand{\Adimp}[2]{\ensuremath{\Ad\,(\mathrm{#1})\Longrightarrow
    (\mathrm{#2})}:\enspace}
\newcommand{\Adit}[1]{\ensuremath{\Ad\,(\mathrm{#1})}:\enspace}

%%% Enclose one argument with delimiters:

\newcommand{\bool}[1]{\llbracket#1\rrbracket} %% condición booleana
\newcommand{\combo}[1]{\operatorname{co}(#1)} %% convex combo
\newcommand{\lin}[1]{\operatorname{lin}\langle#1\rangle} %% `span'
\newcommand{\set}[1]{\{\,#1\,\}}    %% set notation

\newcommand{\floor}[1]{\lfloor#1\rfloor} %% mayor entero <= x
\newcommand{\Set}[1]{\biggl\{\,#1\,\biggr\}} %% set notation (large)
\newcommand{\roof}[1]{\lceil#1\rceil} %% menor entero >= x
\newcommand{\genr}[1]{\left\langle #1\right\rangle}     %% grupo generado por #1

%%% Asides:

\newcommand{\aside}[1]{$\llbracket$\,#1\,$\rrbracket$} % nota lateral
\ifx \nlang \undefined
\newcommand{\hint}[1]{$\llbracket$\,In\-di\-ca\-ci\'on: #1\,$\rrbracket$}
\else
\newcommand{\hint}[1]{$\llbracket$\,Hint: #1\,$\rrbracket$}
\fi 


%%% Matrices:

\newcommand{\onebytwo}[2]{\begin{pmatrix} %% 1 x 2 matrix
  #1 & #2 \end{pmatrix}}
\newcommand{\onebythree}[3]{\begin{pmatrix} %% 1 x 3 matrix
  #1 & #2 & #3 \end{pmatrix}}
\newcommand{\onebyfour}[4]{\begin{pmatrix} %% 1 x 4 matrix
  #1 & #2 & #3 & #4 \end{pmatrix}}
\newcommand{\twobyone}[2]{\begin{pmatrix} %% 2 x 1 matrix
   #1 \\ #2 \end{pmatrix}}
\newcommand{\twobytwo}[4]{\begin{pmatrix} %% 2 x 2 matrix
   #1 & #2 \\ #3 & #4 \end{pmatrix}}
\newcommand{\twobythree}[6]{\begin{pmatrix} %% 2 x 3 matrix
    #1 & #2 & #3\\ #4 & #5 & #6 \end{pmatrix}}
\newcommand{\threebyone}[3]{\begin{pmatrix} %% 3 x 1 matrix
   #1 \\ #2 \\ #3 \end{pmatrix}}
\newcommand{\threebythree}[9]{\begin{pmatrix} %% 3 x 3 matrix
   #1 & #2 & #3 \\ #4 & #5 & #6 \\ #7 & #8 & #9 \end{pmatrix}}
\newcommand{\fourbyone}[4]{\begin{pmatrix} %% 2 x 1 matrix
   #1 \\ #2 \\ #3 \\ #4 \end{pmatrix}}
%\newcommand{\fourbyfour}[16]{\begin{pmatrix} %% 4 x 4 matrix
%  #1 & #2 & #3 & #4\\ #5 & #6 & #7 & #8 \\ #9 & #10 & #11 & #12 \\ #13 & #14 & #15 & #16 \end{pmatrix}}
\newcommand{\nbyn}[9]{\begin{pmatrix} %% 4 x 4 matrix with prefilled entries
  #1 & #2 & \cdots & #3\\ #4 & #5 & \cdots & #6 \\ \vdots & \vdots & \ddots & \vdots \\ #7 & #8 & \cdots & #9 \end{pmatrix}}

%%%%%%%%%%%%%%%%%%%%%%%%%%%%
%% 6. Hyphenation exceptions
%%%%%%%%%%%%%%%%%%%%%%%%%%%%

\hyphenation{auto-va-lor auto-va-lo-res auto-vec-tor auto-vec-to-res
car-di-na-li-dad ce-rra-da ce-rra-do ce-rra-das ce-rra-dos cons-tan-te
cons-tan-tes cons-truc-ci cons-truir con-ti-nua con-ti-nua-mente
con-ti-nuas con-ti-nui-dad con-ti-nuo con-ti-nuos co-rres-pon-den-cia
co-rres-pon-de co-rres-pon-den co-rres-pon-dien-te
co-rres-pon-dien-tes co-va-rian-te cual-quier cual-quiera
cu-bri-mien-to desa-rro-lla-do desa-rro-llar des-pu dia-go-nal
dia-go-na-les di-fe-ren-cia-ble di-fe-ren-cia-bles di-fe-ren-cial
di-fe-ren-cia-les di-fe-ren-te di-fe-ren-tes dis-cre-ta dis-cre-tas
dis-cre-to dis-cre-tos di-vi-si-bi-li-dad di-vi-si-ble ele-men-tal
ele-men-ta-les ele-men-to ele-men-tos equi-va-len-cia equi-va-lente
equi-va-lentes equi-va-rian-te equi-va-rian-tes eu-cli-dia-na
eu-cli-dia-nas eu-cli-dia-no eu-cli-dia-nos Fi-gu-ra Gal-ois
gal-oi-sia-na ge-ne-rada ge-ne-rado ge-ne-ra-dor ge-ne-ra-do-res
ge-ne-ral ge-ne-ra-les ge-ne-ra-li-dad ge-ne-ra-li-za ge-ne-ra-li-zan
ge-ne-ran ge-ne-rar geo-me-tr geo-me-try Ha-da-mard ho-meo-mor-fis-mo
ho-meo-mor-fo idea-les in-de-pen-dien-te in-de-pen-dien-tes
in-va-rian-cia in-va-rian-te in-va-rian-tes li-ne-a-les
li-ne-al-men-te ma-ne-ra me-dian-te mo-der-no nin-gu-no nues-tra
nues-tro nu-me-ra-ble ope-ra-ci ope-ra-cio-nes ope-ra-dor
ope-ra-do-res or-to-go-nal par-ti-cu-lar pro-ce-di-mien-to pro-duc-to
pro-duc-tos pro-pie-dad pro-pie-da-des pro-po-si-ci re-fe-ren-cia
re-fle-xi-va re-fle-xi-vas re-fle-xi-vo re-fle-xi-vos re-so-lu-ble
res-pec-ti-va-men-te res-pec-ti-vo res-pec-ti-vos res-pec-to
sa-tis-fa-ce sepa-ra-ble sepa-ra-bles si-guien-te si-guien-tes
subes-pa-cio subes-pa-cios te-dra-edro te-tra-edros tri-vial
tri-via-les uti-lidad va-lo-res va-ria-ble va-ria-bles va-rie-dad
va-rie-da-des ve-cin-da-rio ve-cin-da-rios vec-to-rial vec-to-ria-les
vice-versa}


%%% TikZ arrows and such

\pgfarrowsdeclarecombine{twolatex'}{twolatex'}{latex'}{latex'}{latex'}{latex'}
\tikzset{->/.style = {decoration={markings,
                                  mark=at position 1 with {\arrow[scale=2]{latex'}}},
                      postaction={decorate}}}
\tikzset{<-/.style = {decoration={markings,
                                  mark=at position 0 with {\arrowreversed[scale=2]{latex'}}},
                      postaction={decorate}}}
\tikzset{<->/.style = {decoration={markings,
                                   mark=at position 0 with {\arrowreversed[scale=2]{latex'}},
                                   mark=at position 1 with {\arrow[scale=2]{latex'}}},
                       postaction={decorate}}}
\tikzset{->-/.style = {decoration={markings,
                                   mark=at position #1 with {\arrow[scale=2]{latex'}}},
                       postaction={decorate}}}
\tikzset{-<-/.style = {decoration={markings,
                                   mark=at position #1 with {\arrowreversed[scale=2]{latex'}}},
                       postaction={decorate}}}
\tikzset{->>/.style = {decoration={markings,
                                  mark=at position 1 with {\arrow[scale=2]{latex'}}},
                      postaction={decorate}}}
\tikzset{<<-/.style = {decoration={markings,
                                  mark=at position 0 with {\arrowreversed[scale=2]{twolatex'}}},
                      postaction={decorate}}}
\tikzset{<<->>/.style = {decoration={markings,
                                   mark=at position 0 with {\arrowreversed[scale=2]{twolatex'}},
                                   mark=at position 1 with {\arrow[scale=2]{twolatex'}}},
                       postaction={decorate}}}
\tikzset{->>-/.style = {decoration={markings,
                                   mark=at position #1 with {\arrow[scale=2]{twolatex'}}},
                       postaction={decorate}}}
\tikzset{-<<-/.style = {decoration={markings,
                                   mark=at position #1 with {\arrowreversed[scale=2]{twolatex'}}},
                       postaction={decorate}}}

\tikzset{circ/.style = {fill, circle, inner sep = 0, minimum size = 3}}
\tikzset{scirc/.style = {fill, circle, inner sep = 0, minimum size = 1.5}}
\tikzset{mstate/.style={circle, draw, blue, text=black, minimum width=0.7cm}}

\tikzset{eqpic/.style={baseline={([yshift=-.5ex]current bounding box.center)}}}
\tikzset{commutative diagrams/.cd,cdmap/.style={/tikz/column 1/.append style={anchor=base east},/tikz/column 2/.append style={anchor=base west},row sep=tiny}}

\definecolor{mblue}{rgb}{0.2, 0.3, 0.8}
\definecolor{morange}{rgb}{1, 0.5, 0}
\definecolor{mgreen}{rgb}{0.1, 0.4, 0.2}
\definecolor{mred}{rgb}{0.5, 0, 0}

\def\drawcirculararc(#1,#2)(#3,#4)(#5,#6){%
    \pgfmathsetmacro\cA{(#1*#1+#2*#2-#3*#3-#4*#4)/2}%
    \pgfmathsetmacro\cB{(#1*#1+#2*#2-#5*#5-#6*#6)/2}%
    \pgfmathsetmacro\cy{(\cB*(#1-#3)-\cA*(#1-#5))/%
                        ((#2-#6)*(#1-#3)-(#2-#4)*(#1-#5))}%
    \pgfmathsetmacro\cx{(\cA-\cy*(#2-#4))/(#1-#3)}%
    \pgfmathsetmacro\cr{sqrt((#1-\cx)*(#1-\cx)+(#2-\cy)*(#2-\cy))}%
    \pgfmathsetmacro\cA{atan2(#2-\cy,#1-\cx)}%
    \pgfmathsetmacro\cB{atan2(#6-\cy,#5-\cx)}%
    \pgfmathparse{\cB<\cA}%
    \ifnum\pgfmathresult=1
        \pgfmathsetmacro\cB{\cB+360}%
    \fi
    \draw (#1,#2) arc (\cA:\cB:\cr);%
}
\newcommand\getCoord[3]{\newdimen{#1}\newdimen{#2}\pgfextractx{#1}{\pgfpointanchor{#3}{center}}\pgfextracty{#2}{\pgfpointanchor{#3}{center}}}

\newcommand\qedshift{\vspace{-17pt}}
\newcommand\fakeqed{\pushQED{\qed}\qedhere}

\def\Xint#1{\mathchoice
   {\XXint\displaystyle\textstyle{#1}}%
   {\XXint\textstyle\scriptstyle{#1}}%
   {\XXint\scriptstyle\scriptscriptstyle{#1}}%
   {\XXint\scriptscriptstyle\scriptscriptstyle{#1}}%
   \!\int}
\def\XXint#1#2#3{{\setbox0=\hbox{$#1{#2#3}{\int}$}
     \vcenter{\hbox{$#2#3$}}\kern-.5\wd0}}
\def\ddashint{\Xint=}
\def\dashint{\Xint-}

\newcommand\separator{{\centering\rule{2cm}{0.2pt}\vspace{2pt}\par}}

\newenvironment{own}{\color{gray!70!black}}{}

\newcommand\makecenter[1]{\raisebox{-0.5\height}{#1}}

\mathchardef\mdash="2D

\newenvironment{significant}{\begin{center}\begin{minipage}{0.9\textwidth}\centering\em}{\end{minipage}\end{center}}
\DeclareRobustCommand{\rvdots}{%
  \vbox{
    \baselineskip4\p@\lineskiplimit\z@
    \kern-\p@
    \hbox{.}\hbox{.}\hbox{.}
  }}
\DeclareRobustCommand\tph[3]{{\texorpdfstring{#1}{#2}}}
\def\BState{\State\hskip-\ALG@thistlm}

\makeatother 

\begin{document}
%\clearpage
\maketitle
%\thispagestyle{empty}
{\small
\setlength{\parindent}{0em}
\setlength{\parskip}{1em}

This course is an introduction to analytic functions of a single complex variable.  The subject is beautiful.-- it turns out that a function with a complex derivative is highly structured -- and enjoys a give and take with many other areas of mathematics.

\subsubsection*{Requirements}
Knowledge of convergence of sequences, series: limits, continuity, differentiation, integration of one-variable functions is required.
}
\newpage
\tableofcontents
%\begin{multicols}{2}
\chapter{First Midterm}

\section{Interim| HW1}

\begin{Ej}[1.1 Stein \& Shakarchi]
Describe geometrically the sets of points $z$ in the complex plane defined by the following relations:
\begin{enumerate}
    \itemsep=-0.4em
    \item $|z-z_1|=|z-z_2|$ where $z_1,z_2\in\bC$.
    \item $1/z=\ov z$.
    \item $\Re(z)=3$
    \item $\Re(z)>c$, (resp.,$\geq c$) where $c\in\bR$.
    \item $\Re(az+b)>0$ where $a,b\in\bC$.
    \item $|z|=\Re(z)+1$.
    \item $\Im(z)=c$ with $c\in\bR$.
\end{enumerate}
\end{Ej}

\begin{ptcbr}
    \begin{enumerate}[i)]
        \itemsep=-0.4em
        \item The first set is the set of points at the same distance from $z_1$ and $z_2$. If we consider the line segment $z_1z_2$, then the set in question is the bisector of that line segment.
        \item Note that
        $$1/z=\ov z\iff 1=\ov zz\iff 1=|z|^2\iff 1=|z|,$$
        thus the set is the unit circle.
        \item The set is a perpendicular line to the real axis at $z=3$.
        \item This infinite set is an infinite half plane to the right (but not including) of the line $z=c$. In the other case, we do include the line in question.
        \item Let us rephrase this inequality in terms of real numbers. Call $a=a_1+ia_2$, $b=b_1+ib_2$ and $z=x+iy$. Then 
        $$\Re(az+b)=\Re[a_1 x - a_2 y + b_1 + i (a_2 x +  a_1 y +  b_2)],$$
        thus our desired inequality is true whenever $a_1 x - a_2 y + b_1>0$. Solving for $y$ we get $y>(a_1x+b_1)/a_2$, which is the half plane located above the line $y=(a_1x+b_1)/a_2$.
        \item The equation in question is equivalent to 
        $$\Re(z)^2+\Im(z)^2=(\Re(z)+1)^2.$$
        To ease the notation, assume $z=x+iy$. Then the equation reads 
        $$x^2+y^2=x^2+2x+1\iff y^2=2x+1\iff x=(y^2-1)/2.$$
        It holds the the parabola in question contains the points which satisfy the equation.
        \item This set is a line parallel to the real axis at $z=c$
    \end{enumerate}
\end{ptcbr}

\begin{Ej}
    Do the following:
    \begin{enumerate}[i)]
        \itemsep=-0.4em
        \item Show that the complex conjugation map $\kp:\bC\to\bC,\ z\mapsto\ov z$ is an involution, i.e., a ring homomorphism such that $\kp\circ\kp=\id$.
        \item Suppose $a\in\bR,\ z\in\bC$. Show that 
        $$\Re(az)=a\Re(z),\word{and}\Im(az)=a\Im(z).$$
    \end{enumerate}
\end{Ej}

\begin{ptcbr}
    Let us take $z=x+iy$ with $x,y\in\bR$.
    \begin{enumerate}[i)]
        \itemsep=-0.4em
        \item We have $\ov z=x+i(-y)=x-iy$. Once more we get $\ov{\ov z}=x-i(-y)=x+iy=z$. Thus $\ov{\ov z}=z$ for any $z\in\bC$. In conclusion $\ov{\ov \.}=\id$.
        \item It holds that 
        \begin{align*}
            &\Re(az)=\Re(ax+aiy)=ax=a\Re(z),\\
            &\Im(az)=\Im(ax+aiy)=ay=a\Im(z).
        \end{align*}
    \end{enumerate}
\end{ptcbr}

\begin{Ej}
    Do the following:
    \begin{enumerate}[i)]
        \itemsep=-0.4em
        \item Prove that $|z+w|^2=|z|^2+|w|^2+2\Re(z\ov w)$.
        \item Use this to prove the parallelogram rule: $|z+w|^2+|z-w|^2=2(|z|^2+|w|^2)$.
    \end{enumerate}
\end{Ej}

\begin{ptcbr}
    \begin{enumerate}[i)]
        \itemsep=-0.4em
        \item Note that 
        $$|z+w|^2=(z+w)\ov{(z+w)}=(z+w)(\ov z+\ov w)=z\ov{z}+w\ov{z}+z\ov{w}+w\ov w.$$
        The number $w\ov z$ is the conjugate of $z\ov w$, and summing a number and its conjugate returns twice its real part. Thus we get the desired identity. 
        \item As the past identity holds for all complex numbers, it holds when $w=-w$. This means that 
        $|z-w|^2=|z|^2+|-w|^2+2\Re(z(\ov{-w}))=|z|^2+|w|^2-2\Re(z\ov w)$
        and summing this together with the first identity gives us the parallelogram law.
    \end{enumerate}
\end{ptcbr}

\begin{Ej}[1.5 Stein \& Shakarchi]
    A set $\Om$ is said to be pathwise connected if any two points in $\Om$ can be joined by a (piecewise-smooth) curve entirely contained in $\Om$. The purpose of this exercise is to prove that an open set $\Om$ is pathwise connected if and only if $\Om$ is connected.
    \begin{enumerate}[i)]
        \itemsep=-0.4em
        \item Suppose first that $\Om$ is open and pathwise connected, and that it can be written as $\Om$ = $\Om_1\cup\Om_2$ where $\Om_1$ and $\Om_2$ are disjoint non-empty open sets. Choose two points $w_1\in\Om_1$ and $w_2\in\Om_2$ and let $\ga$ denote a curve in $\Om$ joining $w_1$ to $w_2$. Consider a parametrization $z:\bonj{0,1}\to\Om$ of this curve with $z(0) = w_1$ and $z(1) = w_2$, and let
        $$t_\ast = \sup_{0\leq t\leq 1}\set{t\:\forall s [(0\leq s<t)\To (z(s)\in\Om_1)]}.$$
        Arrive at a contradiction by considering the point $z(t_\ast)$.
        \item Conversely, suppose that $\Om$ is open and connected. Fix a point $w\in\Om$ and let $\Om_1\subseteq\Om$ denote the set of all points that can be joined to $w$ by a curve contained in $\Om$. Also, let $\Om_2\subseteq\Om$ denote the set of all points that cannot be joined to $w$ by a curve in $\Om$. Prove that both $\Om_1$ and $\Om_2$ are open, disjoint and their union is $\Om$. Finally, since $\Om_1$ is non-empty (why?) conclude that $\Om$ = $\Om$1 as desired.
    \end{enumerate}
    \end{Ej}

\begin{ptcbr}
    \begin{enumerate}[i)]
        \itemsep=-0.4em
        \item 
        \iffalse
        Recall first, that by definition of supremum we have that if $S$ is our set, then 
        $$\exists s\in S(s>t_\ast-\eps)$$
        for $\eps>0$. Following the idea, we consider the point $z(t_\ast)$. We have two options to place $z(t_\ast)$, either in $\Om_1$ or $\Om_2$.\par 
        Let's start by definition of supremum \red{FINISH}
        Let us proceed as mentioned by assuming $\Om$ is disconnected.
        For the point $z(t_\ast)$ we have two options, either it's in $\Om_1$ or $\Om_2$.\par 
        If it ocurred that $z(t_\ast)\in\Om_1$ then, as $\Om_1$ is open, there exists $r>0$ such that $B(z(t_\ast),r)\subseteq \Om_1$. 
        \fi 
        We will proceed using a topological argument instead of a metric one. As the function $\ga$ is continuous, it pulls back $\Om_1$ and $\Om_2$ into $[0,1]$ as open sets. However, as the sets are disjoint, their inverse images are disjoint as well. In other words, we have found two open disjoint sets which separate $[0,1]$: 
        $$[0,1]=\ga^{-1}[\Om_1]\cupdot\ga^{-1}[\Om_2].$$
        But this is impossible because $[0,1]$ is a connected set. Thus, our assumption that $\Om$ was disconnected must be false. We conclude that path-connectedness implies connectedness.
        \item Take $\Om_1,\Om_2$ as in the statement. Then $\Om_1$ is non-empty as $w\in\Om_1$ because it's connected to itself through a trivial path. Suppose now that $z\in\Om_1$ and that $d(z,\del\Om_1)>r>0$. Take $x\in B(z,r)$, then there exists a line-segment between $z$ and $x$ and there's a smooth curve which connects $z\in\Om_1$ with $w$. Thus the piecewise-continuous path from $x$ to $z$ and from $z$ to $w$ is a path which connects $x$ and $w$. As $x$ is arbitrary, it follows that $B(z,r)\subseteq \Om_1$, and thus $\Om_1$ is open.\par 
        Formally, if $\ga:[0,1]\to\Om_1$ is the map which parametrizes the curve between $z$ and $w$ and $r:[0,1]\to B(z,r)$ is the map $t\mapsto tz+(1-t)x$, then the curve from $x$ to $w$ is parametrized by the function 
        $$f=\begin{cases}
            2tz+(1-2t)x,\ t\in[0,1/2],\\
            \ga(2t-1),\ t\in[1/2,1].
        \end{cases}$$
        On the other hand take a point $z\in\Om_2$ and let $d(z,\del\Om_2)>r>0$. Consider a point $x\in B(z,r)$ and assume by way of contradiction that such $x$ can be connected to $w$ by a curve which can be parametrized by a smooth function $\ga$. As the ball is convex, we can connect $z$ to $x$ and then to $w$ creating a path between $z$ and $w$. This is impossible as $z$ cannot be connected to $w$ by a path, thus our assumption must be false. It holds that $x$ cannot be connected to $w$ by a path and thus $x\in\Om_2$. Therefore $\Om_2$ is also open. We conclude that $\Om=\Om_1\cup\Om_2$ is a union of two disjoint open sets, and since $\Om$ is connected, it must hold that $\Om_2$ is empty. We conclude that $\Om$ is path-connected. 
    \end{enumerate}
\end{ptcbr}

\begin{Ej}[1.7 Stein \& Shakarchi]
    The family of mappings introduced here plays an important role in complex analysis. These mappings, sometimes called \textbf{Blaschke factors}, will reappear in various applications in later chapters.
    \begin{enumerate}[i)]
        \itemsep=-0.4em
        \item Let $z,w\in\bC$ such that $\ov{z}w\neq 1$. Prove that 
        $$\left|\frac{w-z}{1-\ov w z}\right|<1$$
        if $|z|<1$ and $|w|<1$, and also that 
        $$\left|\frac{w-z}{1-\ov w z}\right|=1$$
        if $|z|=1$ or $|w|=1$. \hint{Why can one assume that $z$ is real? I then suffices to prove that $(r-w)(r-\ov w)\leq (1-rw)(1-r\ov w)$ with equality for appropriate $r$ and $|w|$.}\aside{Here is an alternate approach, which you may use if you like. Fix $w\in\bC$ with $w<1$, and consider the function $z\mapsto \frac{w-z}{1-\ov w z}$. What is $\ov{f(z)}$? By computing $f(z)\ov{f(z)}$, show that $|z|=1$ implies $|f(z)|=1$. Find a point $z$ with $|z|<1$ such that $|f(z)|<1$. Since $f$ is continuous, this shows that $f$ takes the unit disc to itself. (Why?)}
        \item Prove that for a fixed $w\in\bD$, the mapping $F\:z\mapsto\frac{w-z}{1-\ov w z}$ satisfies the following:
        \begin{enumerate}[a)]
            \itemsep=-0.4em
            \item $F$ maps the unit disc to itself (that is, $F:\bD\to\bD$), and is holomorphic.
            \item $F$ interchanges $0$ and $w$. 
            \item $|F(z)|=1$ if $|z|=1$.
            \item $F$ is bijective. \hint{Calculate $F\circ F$.}
        \end{enumerate}
    \end{enumerate}
    \end{Ej}

    \begin{ptcbr}
        \begin{enumerate}[i)]
            \itemsep=-0.4em
            \item The inequality in question is equivalent to 
            $$0\leq|w-z|<|1-\ov wz|.$$
            Since the quantities are positive, we can square them and preserve the order. It holds that 
            $$0\leq|w-z|^2<|1-\ov wz|^2\iff 0\leq (w-z)\ov{(w-z)}<(1-\ov wz)\ov{(1-\ov wz)},$$
            Simplifying this expression we get 
            \begin{align*}
                &(w-z)(\ov w-\ov z)<(1-\ov wz)(1-w\ov z)\\
                \iff&w\ov w-w\ov z -z\ov w+z\ov z<1-w\ov z-\ov wz+\ov wzw\ov z\\
                \iff&|w|^2+|z|^2<1+|w|^2|z|^2\\
                \iff&0<(1-|w|^2)(1-|z|^2).
            \end{align*}
            The inequality is true whenever both moduli are less than one, and whenever either is one equality is achieved.
            \item Now we suppose $w\in\bD$ which means that $|w|<1$. Taking $z\in\bD$ and applying $F$ gives us the quantity $\frac{w-z}{1-\ov w z}$ which by the previous argument, has modulus less than $1$ whenever $w,z$ do.\par 
            The function $F$ is holomorphic because it is a quotient of holomorphic functions. The denominator is never zero inside the domain because that would mean that $1=\ov w z$. And taking moduli in both sides of the equation gives us 
            $$1=|1|=|w||z|<1$$
            which is impossible.\par 
            Now $F(0)=\frac{w-0}{1-0}=w$ and $F(w)=\frac{w-w}{1-|w|^2}=0$. The denominator in the last expression is never zero because $|w|<1$.\par 
            By the second part of the previous argument it holds that $|z|=1$ immediately gives us $|F(z)|=1$. And finally we will see that $F$ is an involution:
            $$F(F(z))=F\left(\frac{w-z}{1-\ov w z}\right)=\frac{w-\left(\frac{w-z}{1-\ov w z}\right)}{1-\ov w\left(\frac{w-z}{1-\ov w z}\right)}.$$ 
            Homogenizing and clearing denominators we get 
            $$\frac{w(1-\ov wz)-w+z}{1-\ov w z-\ov w(w-z)}=\frac{-w\ov wz+z}{1-\ov ww}=\frac{(-w\ov w+1)z}{1-\ov ww}=z.$$
            This means that $F$ is it's own inverse and therefore, $F$ is bijective. 
        \end{enumerate}
    \end{ptcbr}
\section{Day 1| 20230120}

\subsection{The Complex Numbers}

To construct the complex numbers we take the real numbers, adjoin a variable and mod out by $\genr{x^2+1}$. We can also define $\bC$ as $\set{a+bi:\ a,b\in\bR}$ with the property $i^2=-1$. This means that we can multiply complex numbers in the following way:
$$(a+bi)(c+di)=ac+(bc+ad)i+bdi^2=(ac-bd)+(ad+bc)i.$$
Also as $x^2+1$ is irreducible in $\bR[x]$, $\bC$ is a finite field extension of $\bR$ of degree 2. As a 2-dimensional vector space $\set{1,i}$ is a basis for $\bC$.\par 
The map $a+bi\mapsto\twobyone{a}{b}$ is not a ring homomorphism, it's a bijection with a bit of structure. The map $z\mapsto \al z$, when $\al=a+bi$, is a linear map with the following action over the basis 
\begin{align*}
    &\al\. 1=\al\To[\al]\twobyone{1}{0}=\twobyone{a}{b}\\
    &\al\. i=-b+ai\To[\al]\twobyone{0}{1}=\twobyone{-b}{a}
\end{align*}
which means that $[\al]=\twobytwo{a}{-b}{b}{a}$. The converse, if we have a $\bR$-linear transformation, then it's $\bC$-linear if and only if it looks like $\twobytwo{a}{-b}{b}{a}$.

\begin{Def}
The \term{complex conjugation} map is $a+bi\mapsto a-bi$, or $z\mapsto\ov z$.
\end{Def}

This map is $\bR$-linear but not $\bC$-linear. 

\begin{Ex}
For $\al=a+bi$, we have 
$$\ov{2\al}=\ov{2(a+bi)}=\ov{2a+2bi}=2a-2bi=2\ov{al}.$$
Whereas if instead 
$$\ov{i\al}=\ov{ai-b}=-b-ai\neq i\ov{\al}=b+ai.$$
\end{Ex}

As a $\bR$-linear map, we can identify with the matrix $\twobytwo{1}{0}{0}{-1}$. By looking at the shape of this matrix we can see that it is not $\bC$-linear.

\begin{Lem}
The map $z\mapsto\ov z$ is a ring homomorphism
\end{Lem}

\begin{ptcbp}
$\ov{z+w}=\ov z+\ov w$ and $\ov{zw}=\ov z\ov w$.
\end{ptcbp}

With the complex conjugation we can pick out the real and imaginary parts of $\al=a+bi$. 
$$\al+\ov\al=2\Re(\al),\quad \al-\ov\al=2i\Im(\al)$$
\subsubsection{A Notion of Size}
Can't do geometry without one. Notice that for $z=a+bi$
$$z\ov z=a^2+b^2>0.$$
From a complex number we have extracted a positive quantity.

\begin{Def}
    The \term{complex modulus} of $z$ is $|z|=\sqrt{z\ov z}$.
\end{Def}

The fact that every number has $n$ roots is very important in complex analysis.\par 
As a vector in the plane, the norm of $z$ is $|z|$
\begin{center}
    INC FIG
\end{center}
This means that $a+bi\mapsto\twobyone{a}{b}$ is an isometry. In this sense the distance between two complex numbers is $d(z,w)=|z-w|$.

\subsubsection{Polar Coordinates (\emph{ad hoc})}

For $\te\in\bR$, define 
$$\exp(i\te)=e^{i\te}=\cos(\te)+i\sin(\te)\To |\exp(i\te)|=\sqrt{\cos^2(\te)+\sin^2(\te)}=1.$$
Every point in the unit circle is of the form $e^{i\te}$ and vice-versa.
\begin{center}
    INC FIG
\end{center}
For non-zero complex numbers, $z=|z|e^{i\te}$ for some $\te$.

\begin{Def}
    For a complex number $z=re^{i\te}$, an \term{argument} of $z$ is $\te$. 
\end{Def}
To have a well defined function, we mod out by multiples of $2\pi$: $$\arg:\bC\less\set{0}\to\quot{\bR}{2\pi\bZ},$$
and we obtain a group isomorphism. In general, ``lengths multiply, angles add.''\par 
For inverses if $z=re^{i\te}$, then $\frac{1}{z}=\frac1re^{-i\te}$.

\begin{Def}
    The \term{upper-half plane} is $\bH=\set{\Im(z)>0}$.
\end{Def}

\begin{Lem}
    If $H$ is a half plane $\Im(z-\bt/\ga)>0$
\end{Lem}

\section{Day 2| 20230123}

Recall the complex conjugation map and the modulus of a complex number. This gives us an isometry between $\bR^2$ and $\bC$. Let us prove the lemma from last time. 

\begin{Lem}
    If $H\subseteq\bC$ is a half plane, then there exist $\bt,\ga\in\bC$ such that 
    $$H=\Set{z:\ \Im\left(\frac{z-\bt}{\ga}\right)>0}.$$
\end{Lem}
\begin{center}
    INC FIG
\end{center}
Pick a point $\bt\in H$, then translate $H$ to the origin by $z\mapsto z-\bt$. The plane is now rotated by $\te$ at the origin so we should rotate every point. Then $z\in H-\bt$ whenever $ze^{-i\te}\in\bH$. \red{REDO}\par 
Let us see an application, for a polynomial, the coefficients determine the roots. The following lemma is a technical lemma.

\begin{Lem}
    Suppose $p\in\bC[z]$ and $H$ is a half plane which contains all the roots of $p$. Then $H$ contains all the roots of $p'$.
\end{Lem}

\begin{ptcbp}
    We can assume $p$ is monic, so suppose $\al_1,\dots,\al_d$ are the roots of $\bC$. This means that 
    $$p(z)=\prod_{k=1}^d(z-\al_k)\To p'(z)=\sum_{k=1}^d\frac{p(z)}{z-\al_k}\To\frac{p'(z)}{p(z)}=\sum_{k=1}^d\frac{1}{z-\al_k}.$$
    Now suppose that $H$ contains all $\al_k$ and suppose $z_0\not\in H$, if we show $p'(z_0)\neq 0$ we are done because all the points which make $p'$ vanish won't be outside $H$.\par 
    Describe $H$ by the previous lemma, there exist $\bt,\ga$ such that points in $H$ satisfy the inequality $\Im\left(\frac{z-\bt}{\ga}\right)>0$. As $z_0$ is not in $H$, then $\Im\left(\frac{z_0-\bt}{\ga}\right)<0$. For each $k\in[d]$, we have that 
    $$z_0-\al_k=z_0-\bt+\bt-\al_k=(z_0-\bt)-(\al_k-\bt)$$
    so by taking imaginary parts 
    $$\Im\left(\frac{z-\al_k}{\ga}\right)=\Im\left(\frac{z-\al_k}{\ga}\right)-\Im\left(\frac{z-\al_k}{\ga}\right)$$ 
    The quantity on the right is negative because it's a negative number minus a positive. So it holds that $\Im\left(\frac{\ga}{z-\al_k}\right)>0$. With this we can calculate the following:
    $$\Im\left(\ga\frac{p'(z_0)}{p(z_0)}\right)=\Im\left(\sum_{k=1}^d\frac{\ga}{z_0-\al_k}\right)>0$$
    so in particular this number is non-zero. Thus $p'(z_0)\neq 0$ 
\end{ptcbp}

\begin{Def}
    A set $S\subseteq\bR^n$ is \term{convex} if for any two points $x,y\in S$, the line segment between $x$ and $y$ is also contained in $S$. This is 
    $$\set{ty+(1-t)x:\ x,y\in S}\subseteq S.$$
    The \term{convex hull} of $S$ is the intersection of all convex sets containing $S$. 
\end{Def}

In the case of a finite set of complex numbers, the convex hull can be found by intersecting half-planes which contain them.

\begin{Cor}[Gauss-Lucas]
The roots of $p'(z)$ are contained in the convex hull of the roots of $p(z)$. 
\end{Cor}

\subsection{Metric Spaces}

\begin{Def}
    A \term{metric space} is a set with a distance function.
\end{Def}

\begin{Ex}
    $\bR^n$ is a metric space with $d(x,y)=\norm{x-y}$. Subsets of metric spaces with an induced distance are metric spaces. 
\end{Ex}

\begin{itemize}
    \item nbhd
    \item open and closed
    \item Cauchy
\end{itemize}

\begin{Def}
    Cauchy sequence
\end{Def}

\section{Day 3| 20230125}

The defining property of $\bR$ is that it is complete. In that sense it is possible to prove that $\bR^n$ is also complete.

\subsection{Derivatives}

Recall a real function $g$ is differentiable at $x_0$ if there exists a real number $a$ such that
$$g(x)=g(x_0)+a(x-x_0)+\psi(x),\ \frac{\psi(x)}{x-x_0}\xrightarrow[x\to x_0]0.$$
In the same sense a multivariable function is differentiable when there exists a linear transformation such that a similar condition holds. 

\begin{Def}
$f$ has complex derivative iff real derivative and Cauchy-Riemann equations
\end{Def}

\begin{Ex}
    The map $z\mapsto\ov z$ is not complex-differentiable. First by matrix definition and second with limit. 
\end{Ex}

\section{Day 4| 20230127}

\begin{Lem}
If $\sum_{n\geq 0} z_n$ is absolutely convergent, then it's convergent.
\end{Lem}

\begin{ptcbp}
If $s_n$ is a partial sum, then 
$$|s_n-s_m|=\left|\sum_{i=m+1}^{n}z_i\right|\leq\sum_{i=m+1}^{n}|z_i|<\eps$$
because $\sum|z_n|$ is Cauchy.
\end{ptcbp}

\subsection{Power Series}

\begin{Def}
    A \term{power series} (centered at $0$) is an expression of the form $\sum_{n\geq 0}a_nz^n$.
\end{Def}

\begin{Ex}
    The power series for the exponential function is $e^z=\sum_{n\geq 0}\frac{z^n}{n!}$. 
\end{Ex}

\begin{Th}[Cauchy-Hadamard]\label{thm-cauchy-hadamard}
Suppose $\sum_{n\geq 0}a_nz^n$ has radius of convergence $\frac{1}{r}=\limsup|a_n|^\frac{1}{n}$. Then the series converges for $|z|<r$ and diverges for $|z|>r$.
\end{Th}

\begin{ptcbp}
    
\end{ptcbp}

\section{Day 5| 20230130}

Last time with Hadamard's criterion we learned something that we \emph{already know}. Recall that for radii less than the radius of convergence, power series converge.\par 
As a corollary we can prove the following:

\begin{Cor}
    Suppose $f(z)=\sum_{n\geq 0}a_nz^n$ has radius of convergence $R$. Then the following holds:
    \begin{enumerate}[i)]
        \itemsep=-0.4em
        \item The formal derivative of $f$, 
        $$g(z)=\sum_{n\geq 1}na_nz^{n-1}$$
        converges absolutely and uniformly on $B(0,R)$.
        \item $f'(z)=g(z)$.
    \end{enumerate}
\end{Cor}

\begin{ptcbp}
Notice that 
$$\lim_{n\to\infty}\sqrt[n]{n}=1\To g\ \text{converges}.$$
This is because $\limsup|na_n|^{1/n}=\limsup|a_n|^{1/n}$.\par 
Call $S_N$ the $N^{\text{th}}$ partial sum of $f$. For $r<R$, suppose $|z-z_0|<r$. Then 
$$\left|\frac{f(z)-f(z_0)}{z-z_0}-g(z)\right|=\left|\frac{S_n(z)-E_n(z)-S_n(z_0)+E_n(z_0)-g(z_0)(z-z_0)}{z-z_0}\right|.$$
Let us now add zero carefully and apply the triangle inequality. The previous term is less than 
$$\left|\frac{S_N(z)-S_N(z_0)}{z-z_0}-S_N'(z_0)\right|+|S_N'(z_0)-g(z_0)|+\left|\frac{E_N(z)-E_N(z_0)}{z-z_0}\right|.$$
The last term which contains the errors can be written as 
$$\left|\sum_{n\geq N}\frac{a_n(z^n-z_0^n)}{z-z_0}\right|\leq \sum_{n\geq N}n|a_n|r^{n-1}$$
and for large $N$, this quantity is small. With a similar reasoning we get that 
$$|S_N'(z_0)-g(z_0)|\leq \sum_{n\geq N}n|a_nz^{n-1}|.$$
For $z$ close to $z_0$, the first term is small as well.
\end{ptcbp}

\begin{Cor}
    A complex power series is infinitely differentiable. 
\end{Cor}

\begin{Lem}
The power series of the exponential function satisfies the equality $e^{z+w}=e^ze^w$. 
\end{Lem}

\begin{ptcbp}
    \begin{align*}
        e^ze^w&=\left(\sum_{n\geq 0}\frac{z^n}{n!}\right)\left(\sum_{n\geq 0}\frac{w^n}{n!}\right)\\
        &=\sum_{n\geq 0}\sum_{k+\l=n}\frac{z^k}{k!}\frac{w^\l}{\l!}\\
        &=\sum_{n\geq 0}\frac{1}{n!}\sum_{k+\l=n}\frac{n!}{k!\l!}z^{k+\l}\\
        &\sum_{n\geq0}\frac{1}{n!}(z+w)^n=e^{z+w}.
    \end{align*}
\end{ptcbp}

\begin{Th}
    $e^{i\te}=\cos(\te)+i\sin(\te)$
\end{Th}

\begin{Lem}
    $\te\in\bR$, then $e^{i\te}=1$ iff $\te\in2\pi\bZ$. 
\end{Lem}

\begin{Prop}
If $z=re^{i\al}$, then $\bt$ is an argument of $z$ iff $\al-\bt\in 2\pi\bZ$.
\end{Prop}

\begin{Cor}
    There is a grp isom $\bC^\x\to\bR_{\geq 0}\x\bR/2\pi\bZ$.
\end{Cor}
\section{Interim 2| HW2}

\begin{Ej}
    Suppose $S\subseteq \bC$ is a domain and $f\: S\to\bC$ is differentiable at $z_0\in S$.
    \begin{enumerate}[i)]
        \itemsep=-0.4em
        \item Compute $f'(z_0)$ along a a trajectory $z_0+\Dl x$ where $\Dl x\to 0$. Show that 
        $$f'(z_0)=u_x(z_0)+iv_y(z_0).$$
        \item Compute $f'(z_0)$ along a a trajectory $z_0+i\Dl y$ where $\Dl y\to 0$. Show that 
        $$f'(z_0)=(1/i)(u_y(z_0)+iv_x(z_0)).$$
        \item Conclude that $f$ satisfies the Cauchy-Riemann equations. 
    \end{enumerate}
\end{Ej}
\begin{ptcbr}
By definition, for $h\in\bC$, we have 
$$\lim_{h\to 0}\frac{f(z_0+h)-f(z_0)}{h}=f'(z_0)$$
whenever $f$ is differentiable at $z_0$. 
\begin{enumerate}[i)]
    \itemsep=-0.4em
    \item Take $h=\Dl x$, a number with no imaginary part. Then separating $f$ into its real and imaginary parts we have 
    \begin{align*}
        f'(z_0)&=\lim_{\Dl x\to 0}\frac{f(z_0+\Dl x)-f(z_0)}{\Dl x}\\
        &=\lim_{\Dl x\to 0}\frac{u(x_0+\Dl x,y_0)+iv(x_0+\Dl x,y_0)-u(x_0,y_0)-iv(x_0,y_0)}{\Dl x}\\
        &=\lim_{\Dl x\to 0}\frac{u(x_0+\Dl x,y_0)-u(x_0,y_0)}{\Dl x}+i\hspace{-2mm}\lim_{\Dl x\to 0}\frac{v(x_0+\Dl x,y_0)-v(x_0,y_0)}{\Dl x}\\
        &=\pdv{x}u(x_0,y_0)+i\pdv{x}v(x_0,y_0)
    \end{align*}
    \item On the flipside, take $h=i\Dl y$ with $\Dl y\to 0$. We once again separate $f$ as follows:
    \begin{align*}
        f'(z_0)&=\lim_{\Dl y\to 0}\frac{f(z_0+i\Dl y)-f(z_0)}{i\Dl y}\\
        &=\lim_{\Dl y\to 0}\frac{u(x_0,y_0+\Dl y)+iv(x_0,y_0+\Dl y)-u(x_0,y_0)-iv(x_0,y_0)}{i\Dl y}\\
        &=\lim_{\Dl y\to 0}\frac{u(x_0,y_0+\Dl y)-u(x_0,y_0)}{i\Dl y}+i\hspace{-2mm}\lim_{\Dl y\to 0}\frac{v(x_0,y_0+\Dl y)-v(x_0,y_0)}{i\Dl y}\\
        &=\frac1i\pdv{y}u(x_0,y_0)+\pdv{y}v(x_0,y_0)\\
        &=\pdv{y}v(x_0,y_0)-i\pdv{y}u(x_0,y_0)
    \end{align*}
    \item As the derivatives along both trajectories should match, we have that 
    $$\pdv{x}u(x_0,y_0)+i\pdv{x}v(x_0,y_0)=\pdv{y}v(x_0,y_0)-i\pdv{y}u(x_0,y_0).$$
    Two complex numbers are equal whenever both the real and imaginary parts coincide, so it must hold that 
    $$\pdv{x}u(x_0,y_0)=\pdv{y}v(x_0,y_0),\quad \pdv{x}v(x_0,y_0)=-\pdv{y}u(x_0,y_0).$$
    If a function is holomorphic for every $z$, then this translates to $u_x=v_y$ and $v_x=-u_y$.
\end{enumerate}
\end{ptcbr}

\begin{Ej}[\cite{Stein} 1.13]
    Suppose $f$ is holomorphic in an open set $\Om$. Prove that in any one of the following cases:
     $$\Re(f)\ \text{is constant;}\quad \Im(f)\ \text{is constant;}\quad |f|\ \text{is constant;}$$
    one can conclude that $f$ is constant.
\end{Ej}

\begin{ptcbr}
    As $f$ is holomorphic, $f$ satisfies the Cauchy-Riemann equations. This means that if $f=u+iv$, then 
    $$u_x=v_y,\quad v_x=-u_y.$$
    \begin{itemize}
        \itemsep=-0.4em
        \item If either $u$ or $v$ are constant, then $u_x,u_y$ or $v_x,v_y$ are both zero. In either of those case, by the Cauchy-Riemann equations we can conclude that the other pair of derivatives is zero respectively. 
        \item If the complex modulus is constant, then $|f|^2=u^2+v^2$ is constant as well. Differentiating the expression with respect to both variables gives us 
        $$\begin{cases}
            &2uu_x+2vv_x=0\\
            &2uu_y+2vv_y=0
        \end{cases}\To
        \begin{cases}
            &uu_x+vv_x=0\\
            &uu_y+vv_y=0
        \end{cases} $$
        Now, for sake of argument suppose $u$ isn't zero. Applying the Cauchy-Riemann equations we can restate the first equation as follows:
        $$
        \begin{cases}
            &uv_y+v(-u_y)=0\\
            &uu_y+vv_y=0
        \end{cases} \To
        \begin{cases}
            &v_y=\frac{v}{u}u_y\\
            &uu_y+vv_y=0
        \end{cases} $$
        Substituting the first equation into the second we obtain
        $$ uu_y+v\left(\frac{v}{u}u_y\right)=\left(\frac{u^2+v^2}{u}\right)u_y=0$$
        from which follows that either $u^2+v^2=0$ or $u_y=0$. In the first case, as $u$ is a non-zero real function, it is impossible for the sum to be zero. So it must hold that $u_y=0$.\par 
        Doing a similar process by solving for $u_y$ on the second equation we reach the condition that $v_y=0$ as well. From here, using the Cauchy-Riemann equations we see that all partial derivatives of $u$ and $v$ are zero as we wished.\par 
        In the case that $u=0$, we refer to the first case, where $u$ is a constant. 
    \end{itemize}
    Finally we conclude that $f$ is constant in any case.
\end{ptcbr}
\begin{Ej}
    Prove the following:
    \begin{enumerate}[i)]
        \itemsep=-0.4em
        \item The power series $\sum_{n\geq 0}nz^n$ doesn't converge for any point on the unit circle.
        \item The power series $\sum_{n\geq 0}\frac{z^n}{n^2}$ converges for \emph{every} point in the unit circle.
        \item The power series $\sum_{n\geq 0}\frac{z^n}{n}$ converges for {every} point in the unit circle, \emph{except} $z=1$.
        \end{enumerate}
\end{Ej}

\begin{ptcbr}
    \begin{enumerate}[i)]
        \itemsep=-0.4em
        \item We will prove that the series in question isn't Cauchy. Consider $S_m$, the $m^{\text{th}}$ partial sum, then 
        $$|S_{m+1}-S_m|=m+1$$
        because $z$ has complex modulus 1. Recall that a sequence of complex numbers $(z_n)$ is a Cauchy sequence whenever %https://math.stackexchange.com/questions/4479987/negation-of-cauchy-criterion
        $$\forall \eps \exists N\left[\forall m\forall n(m\geq N\land n\geq N\land \eps>0\To |z_m-z_n|<\eps)\right].$$
        In order to prove that $(S_m)$ isn't Cauchy we must contradict this statement. Thus we must find an $\eps_0>0$ such that for all $N$, there are $m,n$ for which $|S_m-S_n|>\eps_0$.\par 
        Take $\eps_0=1$, $m$ any sufficiently large natural number and $n=m+1$ as we did before. Thus we have that $m+1>1$ which lets us conclude that $(S_m)$ isn't Cauchy. There are no non-Cauchy convergent sequences in $\bC$ so it must hold that our series diverges given the condition that $|z|=1$. 
        \item Recall the Weierstrass M-test which states that if $(f_n(z))$ is a sequence of functions and there are $M_n>0$ such that $|f_n(z)|\leq M_n$ and $\sum M_n$ is a convergent series, then $\sum f_n$ converges uniformly.\par 
        In this case, pick $M_n=\frac{1}{n^2}$. The series $\sum \frac{1}{n^2}$ converges as it is a $p$-series. Then 
        $$|z|=1\To\left|\frac{z^n}{n^2}\right|\leq\frac{1}{n^2}$$
        and thus we can conclude that $\sum\frac{z^n}{n^2}$ converges uniformly for points in the unit circle. 
        \item The series in question is the harmonic series when $z=1$ so it diverges. We will prove that when $|z|=1$, but $z\neq 1$ this series is Cauchy. So let us fix $z$ with $|z|=1$ and call $S_m=\sum_{k=0}^m\frac{z^k}{k}$, then let $\eps>0$. Assume $n>m$ for sake of argument and then
        \begin{align*}
            |S_n-S_{m-1}|&=\left|\sum_{k=m}^n\frac{z^k}{k}\right|\\
            &=\left|\frac{1}{n}\sum_{k=1}^nz^k-\frac{1}{m}\sum_{k=1}^{m-1}z^k-\sum_{k=m}^{n-1}\left(\frac{1}{k+1}-\frac1k\right)\sum_{j=1}^k z^j\right|\\
            &\leq \frac{1}{n}\left|\frac{z^{n+1}-z}{z-1}\right|+\frac{1}{m}\left|\frac{z^{m}-z}{z-1}\right|+\sum_{k=m}^{n-1}\left|\frac{z^{n+1}-z}{(z-1)(k^2+k)}\right|\\
            &\leq \frac{2}{n}\left|\frac{1}{z-1}\right|+\frac{2}{m}\left|\frac{1}{z-1}\right|+\sum_{k=m}^{n-1}\left|\frac{1}{z-1}\right|\frac{2}{k^2+k}
            %&\leq \left|\frac{1}{z-1}\right|\left(\frac{1}{n}+\frac{1}{m}+\sum_{k=m}^{n-1}\frac{1}{k^2+k}\right).
        \end{align*}
        Now let us state a couple of facts:
        \begin{itemize}
            \item $\left|\frac{1}{z-1}\right|$ might be arbitrarily large, but $z$ is fixed. This means that $\left|\frac{1}{z-1}\right|$ is finite.
            \item Call $\widetilde{S}_r=\sum_{k=1}^{r}\frac{2}{k^2+k}$, it is important to note that this a sequence of \emph{positive numbers}. $\widetilde{S}_\infty$ converges after comparing with $\sum_{k=1}^\infty\frac{1}{k^2}$.
        \end{itemize}
        We will name $M=\left|\frac{1}{z-1}\right|$ so that the last expression can be written as follows:
        $$\frac{2M}{n}+\frac{2M}{m}+M(\widetilde{S}_{n-1}-\widetilde{S}_{m-1}).$$
        Now, as $\frac{1}{n}$ converges to zero, there exists $N_1\in\bN$ such that 
        $$n\geq N_1\To \frac{1}{n}<\frac{\eps}{6M},\ \eps>0$$
        On the other hand as $\widetilde{S}_r$ converges, there exists an $N_2\in\bN$ such that 
        $$m,n\geq N_2\To |\widetilde{S}_m-\widetilde{S}_n|<\frac{\eps}{3M},\ \eps>0$$
        Pick $N=\max{N_1,N_2}$ and let $\eps>0$, then whenever $m,n\geq N$, the following holds 
        $$\frac{2M}{n}+\frac{2M}{m}+M(\widetilde{S}_{n-1}-\widetilde{S}_{m-1})\leq 2M\frac{\eps}{6M}+2M\frac{\eps}{6M}+M\frac{\eps}{3M}=\eps.$$
        Therefore the series in question is Cauchy and we can conclude that it converges.
    \end{enumerate}
\end{ptcbr}
\begin{Ej}
    Let $\al\in\bC,\ r>0$ and $\ga_r\: [0,2\pi[\to\bC$ given by $t\mapsto re^{it}+\al$. Let $n\in\bN$, calculate the integral $\int\limits_{\ga_1(0)}z^n\dd z$.
\end{Ej}


\begin{ptcbr}
    We can parametrize with $t\mapsto e^{it}$ with $r\in[0,2\pi[$ so that 
    $$\int\limits_{\ga_1(0)}z^n\dd z=\int\limits_0^{2\pi}(e^{it})^n(ie^{it}\dd t)=\int\limits_0^{2\pi}ie^{i(n+1)t}\dd t=\eval{\frac{e^{i(n+1)t}}{n+1}}_{0}^{2\pi}=\frac{e^{2\pi i(n+1)}}{n+1}-\frac{1}{n+1}=0.$$
\end{ptcbr}

\begin{Ej}
    Consider the following three groups:
    \begin{itemize}
        \itemsep=-0.4em
        \item $\bC^\x$ with multiplication as binary operation.
        \item $\bR_{>0}$ with multiplication as binary operation.
        \item $\quot{\bR}{2\pi\bZ}$ with addition as binary operation.
    \end{itemize}
    Show that 
    $$\al\: \bC^\x\to \bR_{>0}\oplus\quot{\bR}{2\pi\bZ},\ z\mapsto(|z|,\arg(z))$$
    is a group isomorphism as follows:
    \begin{enumerate}[i)]
        \itemsep=-0.4em
        \item Show that $\al$ is a group homomorphism. \hint{This comes down to show that $|zw|=|z||w|$ and $\arg(zw)=\arg(z)+\arg(w)$.}
        \item Show that $\al$ is surjective. \hint{For $r\in\bR_{>0}$ and representative $\te\in\bR$ show that there is some $\bC^{\x}$ such that $|z|=r$ and $\arg(z)=\te$.}
        \item Show that $\al$ is injective. \hint{Suppose $\al(z)=(1,0)$, then show that $z=1$.}
    \end{enumerate}
\end{Ej}

\begin{ptcbr}
    \begin{enumerate}[i)]
        \itemsep=-0.4em
        \item The function $\al$ is a homomorphism because 
        \begin{align*}
            \al(wz)&=(|wz|,\arg(wz))=(|w||z|,\arg(w)+\arg(z))\\
        &=(|w|,\arg(w))\circ(|z|,\arg(z))=\al(w)\circ\al(z)
        \end{align*}
        where $\circ$ is the group operation in the direct product. To prove the equalities hold, take $wz=r_1e^{i\te_1}$, $w=r_2e^{i\te_2}$ and $z=r_3e^{i\te_3}$. Then 
        $$wz=(r_2e^{i\te_2})(r_3e^{i\te_3})=(r_2r_3)e^{i(\te_2+\te_3)}=r_1e^{i\te_1},$$
        and as the polar representation of a complex number is unique we have that $r_1=r_2r_3$ and $\te_1=\te_2+\te_3$.
        \item Take $(r,\te)$ in the codomain of $\al$. As $r>0$, we can write it as $x^2+y^2$ for $x,y\in\bR$. Given that condition we may find the angle by the relation $\tan(\te)=\frac{y}{x}$. Taking $r$ and $\te$ as given lets us construct a complex number $z=x+iy$ such that $\al(z)=(r,\te)$.
        \item If it happened that $r=1$ and $\te=0$, then the complex number in question could be represented as $1\.e^0=1$. Thus $z=1$. This means that $\ker(\al)=\set{\id}$ and thus, as $\al$ is a morphism, it's also injective.
    \end{enumerate}
\end{ptcbr}

\section{Day 6| 20230201}

\begin{Def}
    A \term{parametrization} of a curve is a function $z\: [a,b]\to\bC$.\par 
    It is smooth if it's differentiable and piecewise smooth if for a partition of $[a,b]$, $z$ is smooth on the parts.
\end{Def}

\begin{Ex}
    The function $z\:[0,2\pi]\to\bC,\ t\mapsto e^{it}$ is a parametrization of the unit circle. 
\end{Ex}

\begin{Def}
    Two parametrizations $w,z$ are equivalent if there exists a bijection $[a,b]\to[c,d]$ such that $w(s)=z(s(t))$.
\end{Def}

\begin{Ex}
    An equivalent parametrization of $e^{it}$ is $[0,1]\to\bC,\ t\mapsto e^{2\pi i t}$. 
\end{Ex}

The reverse parametrization of $z\:[a,b]\to\bC$ is $z^-\:[-b,-a]\to\bC,\ t\mapsto z(-t)$. A curve is closed if it starts where it ends. Simple curves don't cross themselves.

\begin{Def}
    The integral over a curve $\ga$ is 
    $$\int\limits_\ga f(z)\dd z=\int\limits_a^bf(z(t))z'(t)\dd t$$
    where $z\:[a,b]\to\ga$ parametrizes the curve.
\end{Def}

\begin{Ex}
    Consider the integral of $\ov z$ over the unit circle. This is 
    $$\int\limits_{\set{|z|=1}}\ov z\dd z=\int\limits_0^{2\pi}\ov{e^{it}}ie^{-it}\dd t=\int\limits_{0}^{2\pi}i\dd t=2\pi i.$$
\end{Ex}

Integrals over the complex numbers obey the same properties as over the real numbers. The arc length of a curve is the same as in multivariable calculus. The integral also obeys the triangle inequality.

\begin{Def}
    A \term{domain} is a non-empty, open and connected subset of $\bC$.
\end{Def}

\begin{Lem}
    If $F,f$ are functions defined on $\Om$, a domain, with $F'=f$, and $w,z\in\Om$, then 
    $$\int\limits_\ga f(z)\dd z=F(z)-F(w)$$
    where $\ga\subseteq\Om$ is a curve connecting $w$ to $z$. 
\end{Lem}

\begin{Cor}
If the curve is closed the integral is zero.
\end{Cor}

As a consequence, the function $\ov z$ has no antiderivative in any ball around the origin. 

\begin{Lem}
    Suppose $f$ is holomorphic on $\Om$ and $f'=0$ on $\Om$. Then $f$ is constant.
\end{Lem}

\begin{ptcbp}
    If $w,z\in\Om$, then 
    $$0=\int f'=f(z)-f(w)\To f(z)=f(w)$$
    so $f$ must be constant.
\end{ptcbp}

Next time: Goursat's theorem.

\section{Day 7| 20230203}

\section{Day 8| 20230206}

Last time we proved Morera's theorem. Recall Goursat's theorem, which tells us that along a contour a holomorphic function has zero integral. From this it is extracted that holomorphic functions have primitives.

\begin{Cor}
    If $f$ is holomorphic on an open disk $D$ and $\ga\subseteq D$ is a closed contour, then $\int\limits_\ga f(z)\dd z=0$.
\end{Cor}

\begin{ptcbp}
    As $f$ admits a primitive $F$ on $D$, the integral in question is $F(end)-F(begin)=0$.
\end{ptcbp}

\subsection{Toy Contours}

A \term{toy contour} is a closed curve with well defined interior and exterior and it's \emph{easy to describe${}^{\text{TM}}$}.

\begin{Ex}
    Squares are toy contours. A hollow circle with a rectangle is a toy contour.
\end{Ex}

\begin{Th}
    If $\ga$ is a toy contour, $f$
 is holomorphic on an open set containing $\ga$ in its interior, then $\int f(z)\dd z=0$.
\end{Th}

\begin{Ex}
    Let us calculate $\int\limits_0^\infty\frac{1-\cos(x)}{x^2}\dd x$.\par 
    For that, consider the function $z\mapsto \frac{1-e^{iz}}{z^2}$. DRAW FIG. The integral can be separated into 
    $$\int\limits_{-R}^{-\eps}f+\int\limits_{\ga_\eps}^{}f+\int\limits_{\eps}^{R}f+\int\limits_{\ga_R}f.$$
    We can parametrize $\ga_R$ by $t\mapsto R^{it}$ with $t\in\bonj{0,\pi}$. Computing the integral we get the function 
    $$\frac{1-\exp(iRe^{it})}{(Re^{it})^2}\To |f(t)|\leq \frac{1+|\exp(iRe^{it})|}{|(Re^{it})^2|}\leq\frac{2}{R^2}.$$
\end{Ex}
\section{Interim 3| HW3}

\begin{Ej}[Exercise 2]
    Evaluate the integral $\int\limits_{\ga_1(0)}\Re(z)\dd x$ in two ways:
    \begin{enumerate}[i)]
        \itemsep=-0.4em
        \item Directly using the definition. \hint{You can model your calculation on the work we did in
        class to compute the integral of $\ov z$.}
        \item Using the fact that $\Re(z)=\frac{z+\ov z}{2}$.
    \end{enumerate}
\end{Ej}

\begin{ptcbr}
    Both integrals will use the parametrization $t\mapsto e^{it}$ with $t\in[0,2\pi[$.
    \begin{enumerate}[i)]
        \itemsep=-0.4em
        \item The first integral is 
        \begin{align*}
            \int\limits_{0}^{2\pi}\Re(e^{it})(ie^{it})\dd t&=\int\limits_{0}^{2\pi}ie^{it}\cos(t)\dd t=\int\limits_{0}^{2\pi}(i\cos^2(t)-\sin(t)\cos(t))\dd t\\
            &=i\int\limits_{0}^{2\pi}cos^2(t)\dd t-\int\limits_{0}^{2\pi}\sin(t)\cos(t)\dd t=i\pi+0=i\pi.
        \end{align*}
        \item The second integral is 
        $$\int\limits_{\ga_1(0)}^{}\frac{z+\ov z}{2}=\frac{1}{2}\int\limits_{\ga_1(0)}^{}z\dd z+\frac{1}{2}\int\limits_{\ga_1(0)}^{}\ov z\dd z=0+\half\int\limits_{0}^{2\pi}e^{-it}(ie^{it})\dd t=\half(2\pi i)=i\pi.$$
    \end{enumerate}
    Both calculations coincide in the value of $i\pi$.
\end{ptcbr}

\begin{Ej}
    Suppose $f$ is defined on a domain $\Om$ with $\ga\subseteq\Om$, a closed contour. Additionally, suppose that for $\eps>0$, there exists a polynomial $P_\eps(z)$ such that $|f(z)-P_\eps(z)|<\eps$ for $z\in\ga$. Show that $\int\limits_\ga |f(z)|\dd z=0$ and thus conclude that $\int\limits_{\ga}^{}f(z)\dd z=0$. 
\end{Ej}

\begin{ptcb}
Consider $\ga$ an arbitrary, but fixed contour inside $\Om$. For $\eps/\l(\ga)>0$, there exists $P_{\eps/\l(\ga)}(z)$ such that
$$|f(z)-P_{\eps/\l(\ga)}(z)|<\frac{\eps}{\l(\ga)}.$$
Then, applying the triangle inequality we have 
\begin{align*}
    \int\limits_{\ga}^{}f(z)\dd z\leq\int\limits_{\ga}^{}|f(z)|\dd z\leq \int\limits_{\ga}^{}|f(z)-P_{\eps/\l(\ga)}(z)|\dd z+\int\limits_{\ga}^{}|P_{\eps/\l(\ga)}(z)|\dd z 
\end{align*}
The first integral can be bounded in the contour $\ga$ by hypothesis as follows: 
$$\int\limits_{\ga}^{}|f(z)-P_{\eps/\l(\ga)}(z)|\dd z\leq\sup_{z\in\ga}|f(z)-P_{\eps/\l(\ga)}(z)|\int\limits_{\ga}\dd z\leq \frac{\eps}{\l(\ga)}\l(\ga)=\eps.$$
The other integral can't be shown to be zero using the theorems we have at hand.
\end{ptcb}

\begin{Lem}
    Suppose $f$ is holomorphic, then there are two possibilities: 
    \begin{enumerate}[i)]
        \itemsep=-0.4em
        \item Either $|f|$ is holomorphic and therefore constant (from which we conclude that $f$ is constant). 
        \item Or $|f|$ is not holomorphic.
    \end{enumerate}
\end{Lem}

\begin{ptcbp}
The function $|f|$ is a real valued complex function. This means that 
$$|f(z)|=g(z)+ih(z),\word{with} h=0.$$
If $|f(z)|$ was holomorphic, it would satisfy the Cauchy-Riemann equations which means that $g_x=g_y=0$. This can be used to conclude that $f$ is also constant (by a previous homework exercise.)\par 
In the other case, $|f(z)|$ is simply not holomorphic.
\end{ptcbp}

By the previous lemma, assuming $P_\eps$ is not a constant polynomial, $|P_\eps|$ is not holomorphic. We cannot state results at this moment about the integral $\int\limits_{\ga}^{}|P_{\eps/\l(\ga)}(z)|\dd z$.\par 
As an alternative approach without showing that $\int\limits_{\ga}^{}|f(z)|\dd z$ we have the following:

\begin{ptcbr}
Recall the reverse triangle inequality: 
$$|x+y|\leq |x|+|y|\To |x+y|-|y|\leq |x|\xrightarrow[]{x=\tilde{x}-y}|\tilde{x}|-|y|\leq|\tilde{x}-y|.$$
Now take $P_{\eps/\l(\ga)}(z)$ as before, then 
$$\left|\int\limits_\ga (f(z)-P_{\eps/\l(\ga)}(z))\dd z\right|\geq\left|\int\limits_\ga f(z)\dd z\right|-\left|\int\limits_\ga P_{\eps/\l(\ga)}(z)\dd z\right| $$
and the rightmost integral is zero as $P_{\eps/\l(\ga)}$ is a polynomial and therefore, a holomorphic function. Then, applying the integral triangle inequality we have 
$$\left|\int\limits_\ga f(z)\dd z\right|\leq\left|\int\limits_\ga (f(z)-P_{\eps/\l(\ga)}(z))\dd z\right|\leq\int\limits_\ga |f(z)-P_{\eps/\l(\ga)}(z)|\dd z\leq\eps.$$
The last integral is smaller than $\eps$ by the previous attempted argument. As $\eps>0$ is arbitrary and $\left|\int\limits_\ga f(z)\dd z\right|$ is a real number, the only possibility is that it's equal to zero. The only complex number with zero modulus is the origin, so we conclude that $\int\limits_{\ga}^{}f(z)\dd z=0$.
\end{ptcbr}

\begin{Ej}
Prove that $\int\limits_0^{\infty}\sin(x^2)\dd x=\int\limits_0^{\infty}\cos(x^2)\dd x=\frac{\sqrt{2\pi}}{4}$. These are the Fresnel integrals. Here, $\int\limits_0^{\infty}$ is interpreted as $\lim_{R\to\infty}\int\limits_0^{R}$.\hint{Integrate the function $e^{-z^2}$ over the path in Figure 14. Recall that $\int\limits_{-\infty}^{\infty}e^{-x^2}\dd x=\sqrt{\pi}$.}
\end{Ej}

\begin{ptcbr}
The path in question is boundary of the circular sector of fixed radius from $\te=0$ to $\te=\frac{\pi}{4}$. Call $\ga$ the curve which bounds the sector. Now notice that 
$$\cos(x^2)+i\sin(x^2)=e^{ix^2},$$
so we will work with the function $e^{iz^2}$ through the circular sector in question. We have that $\int\limits_\ga e^{iz^2}\dd z$ is zero because the exponential function is holomorphic, but also that integral can be broken down into three pieces as follows:
$$\int\limits_{0}^{R}e^{ix^2}\dd x+\int\limits_{0}^{\pi/4}\exp\bonj{i(Re^{it})^2}(iRe^{it})+\int\limits_{0}^{1}\exp\bonj{i((1-t)Re^{i\frac{\pi}{4}})^2}(-Re^{i\frac{\pi}{4}})\dd t.$$
The second integrand can be bounded as follows: 
\begin{align*}
    \left|\exp\bonj{i(Re^{it})^2}(iRe^{it})\right|&\leq R|\exp\bonj{i(Re^{it})^2}|=R|\exp\bonj{iR^2(\cos(2t)+i\sin(2t))}|\\
    &=R|\exp\bonj{iR^2\cos(2t)-R^2\sin(2t)}|\\
    &=Re^{-R^2\sin(2t)}|e^{i(R^2\cos(2t))}|\\
    &=Re^{-R^2\sin(2t)}\xrightarrow[R\to\infty]{}0
\end{align*}
So, as $R$ grows, we can bound the second integral by small quantity which decreases to zero. The third integral can be manipulated as follows, first consider the exponent in the integrand:
$$i((1-t)Re^{i\frac{\pi}{4}})^2=i(1-t)^2R^2e^{\frac{i\pi}{2}}=-(1-t)^2R^2.$$
Taking the substitution $u=(1-t)R$ we have $\dd u=-R\dd t$ and as $t\to 0$, $u\to R$ while $t\to 1\To u\to 0$. The third integral can be written as 
$$\int\limits_{R}^{0}e^{-u^2}e^{i\frac{\pi}{4}}\dd u=-e^{\frac{i\pi}{4}}\int\limits_{0}^{R}e^{-u^2}\dd u.$$
We have the following equation at this point 
$$\int\limits_{0}^{R}e^{ix^2}\dd x-e^{\frac{i\pi}{4}}\int\limits_{0}^{R}e^{-u^2}\dd u=0$$
where we now take the limit as $R\to\infty$. From here we get 
$$\int\limits_{0}^{\infty}e^{ix^2}\dd x=e^{\frac{i\pi}{4}}\frac{\sqrt{\pi}}{2}.$$
Taking real and imaginary parts of the integral in question gives us the desired result.
\end{ptcbr}

\section{Day 9| 20230208}

\subsection{The Cauchy Integral Formula}

Suppose $f$ is holomorphic on a domain $\Om$ which contains a disk $D$ with boundary $C$. Then for $z\in D$ 
$$f(z)=\frac{1}{2\pi i}\oint\limits_C\frac{f(w)}{w-z}\dd w.$$

\begin{ptcbp}
    Consider the function $g(w)=\frac{f(w)}{w-z}$ and a keyhole contour around $z$. We know that inside the contour $g$ is holomorphic, so this means that $\oint\limits_{\Ga(\dl,\eps)}g(w)\dd w=0$.\par 
    Away from $z$, $g$ is continuous, so this means that 
    $$\lim_{\dl\to 0}\oint\limits_{\Ga(\dl,\eps)}g(w)\dd w=\oint\limits_{-B(z,\eps)}g(w)\dd w+\oint\limits_Cg(w)\dd w$$
    where $B(z,\eps)$ is the ball of radius $\eps$ centered at $z$. We can now write 
    $$\oint\limits_{-B(z,\eps)}g(w)\dd w=\oint\limits_{-B(z,\eps)}\frac{f(w)-f(z)}{w-z}\dd w+\oint\limits_{-B(z,\eps)}\frac{f(z)}{w-z}\dd w$$
    so as $f$ is holomorphic we can bound $\frac{f(w)-f(z)}{w-z}$ with $\sup_{B}|f'(z)|$ (\red{WATCH OUT}). On the other hand 
    $$\oint\limits_{-B(z,\eps)}\frac{f(z)}{w-z}\dd w=f(z)\int\limits_{0}^{2\pi}\frac{(-i\eps e^{-it})}{z+\eps e^{-it}-z}\dd t=f(z)\int\limits_{0}^{2\pi}-i\dd t=f(z)(-2\pi i).$$
    We conclude that 
    $$\oint\limits_Cg(w)\dd w=2\pi i f(z)$$
    which we can rearrange to the desired equality.
\end{ptcbp}

\begin{Ex}
    With the formula we can compute 
    $$\oint\limits_{B(0,1)}\frac{z}{2z+1}=\frac{-i\pi}{2}.$$
\end{Ex}

\begin{Th}
    Suppose $f$ is holomorphic on $\Om$, then 
    \begin{enumerate}[i)]
        \itemsep=-0.4em
        \item $f$ is infinitely differentiable.
        \item If $C$ is a curve inside $\Om$, 
        $$f^{(n)}(z)=\frac{n!}{2\pi i}\oint\limits_C\frac{f(w)}{(w-z)^{n+1}}\dd w.$$
    \end{enumerate}
\end{Th}

\section{Interim 4| HW4}

\begin{Ej}
    In this problem, we'll use the Cauchy integral formula to show that an analytic function has a power series representation.\par
    Suppose $f$ is analytic on an open set containing $\ov{B}(0,R)$. We will show that, on $B(0,R)$, there is an equality of functions
    $$f(z)=\sum_{n\geq 0}a_nz^n,\word{where}a_n=\frac{f^{(n)}(0)}{n!}.$$
    Let $\cC=\del B(0,R)$ be the circle of radius $R$ centered at the origin oriented in positive direction. Fix $z$ with $|z|<r=R$.
    \begin{enumerate}[i)]
        \itemsep=-0.4em
        \item Show that 
        $$f(z)=\frac{1}{2\pi i}\oint\limits_\cC g(w)f(w)\dd w,\word{where}g(w)=\frac{1}{w}\frac{1}{1-z/w}.$$
        \item Let $N\in\bN$. Show that 
        $$g(w)=\sum_{n=0}^{N-1}\frac{z^n}{w^{n+1}}+\frac{z^N}{(w-z)w^N}.$$
        \item Show that 
        $$f(z)=\sum_{n=0}^{N-1}\frac{f^{(n)}(0)}{n!}z^n+\rho_N(z),\word{where}\rho_N(z)=\frac{z^N}{2\pi i}\oint\limits_\cC\frac{f(w)}{(w-z)w^N}\dd w.$$
        \item Let $M=\sup_{z\in\cC}|f(z)|$, show that 
        $$|\rho_N(z)|\leq\frac{rM}{R-r}\left(\frac{r}{R}\right)^{N-1}.$$
        \hint{If $w\in\cC$, then $|w-z|\geq R-r$.}
        \item Show that $\lim_{N\to\infty}\rho_N(z)=0$.
    \end{enumerate}
\end{Ej}

\begin{ptcbr}
    \begin{enumerate}[i)]
        \itemsep=-0.4em
        \item By Cauchy's formula for $z\in B(0,R)$ we have 
        $$f(z)=\frac{1}{2\pi i}\oint\limits_\cC\frac{f(w)}{w-z}\dd z.$$
        Taking the $\frac{1}{w-z}$ and factoring a $w$ on the bottom we get 
        $$\frac{1}{w-z}=\frac{1}{w(1-z/w)}=\frac{1}{w}\frac{1}{1-z/w}.$$
        Replacing inside the integral we get the desired equality. 
        \item Consider the geometric series with common term $\frac{z}{w}$, we have the following:
        $$
        \left\lbrace
        \begin{aligned}
            &\sum_{n=0}^\infty \left(\frac
            {z}{w}\right)^n=\frac{1}{1-z/w}\\
            &\sum_{n=0}^{N-1} \left(\frac
            {z}{w}\right)^n=\frac{1-(z/w)^N}{1-(z/w)}
        \end{aligned}
        \right.
        $$
        Now notice that 
        \begin{align*}
            &\frac{1}{1-z/w}-\frac{1-(z/w)^N}{1-(z/w)}=\frac{(z/w)^N}{1-(z/w)}\\
            \To&g(w)-\frac{1}{w}\sum_{n=0}^{N-1} \left(\frac
            {z}{w}\right)^n=\frac{(z/w)^N}{w-z}\\
            \To&g(w)=\sum_{n=0}^{N-1}\frac{z^n}{w^{n+1}}+\frac{z^N}{(w-z)w^N}.
        \end{align*}
        \item Replacing the last value on the identity of the first item we get 
        $$f(z)=\frac{1}{2\pi i}\oint\limits_\cC \left(\sum_{n=0}^{N-1}\frac{z^n}{w^{n+1}}+\frac{z^N}{(w-z)w^N}\right)f(w)\dd w$$
        and as the sum is finite, it commutes with the integral without any restrictions. Exchanging the integral we get 
        $$\sum_{n=0}^{N-1}\frac{1}{2\pi i}\oint\limits_\cC \left(\frac{z^n}{w^{n+1}}f(w)\dd w\right)+\frac{1}{2\pi i}\oint\limits_\cC \frac{z^N}{(w-z)w^N}f(w)\dd w.$$
        Using Cauchy's integral formula for derivatives we have 
        \begin{align*}
        &f^{(n)}(0)=\frac{n!}{2\pi i}\oint\limits_\cC\frac{f(w)}{(w-0)^{n+1}}\dd w\\
        \To&\frac{f^{(n)}(0)}{n!}=\frac{1}{2\pi i}\oint\limits_\cC\frac{f(w)}{w^{n+1}}\dd w.
        \end{align*}
        So when factoring out the $z$'s from the previous expressions we obtain 
        $$\sum_{n=0}^{N-1}z^n\frac{f^{(n)}(0)}{n!}+\frac{z^N}{2\pi i}\oint\limits_\cC \frac{f(w)}{(w-z)w^N}\dd w.$$
        This is the desired expression. 
        \item For this item, it is paramount to remember that $|z|=r<R$ and points $w\in\cC$ satisfy $|w|=R$. Now look at the following diagram: 
        \begin{center}




        \tikzset{every picture/.style={line width=0.75pt}} %set default line width to 0.75pt        

        \begin{tikzpicture}[x=0.75pt,y=0.75pt,yscale=-1,xscale=1]
        %uncomment if require: \path (0,300); %set diagram left start at 0, and has height of 300
        
        %Shape: Circle [id:dp4346947048117715] 
        \draw   (200,150) .. controls (200,122.39) and (222.39,100) .. (250,100) .. controls (277.61,100) and (300,122.39) .. (300,150) .. controls (300,177.61) and (277.61,200) .. (250,200) .. controls (222.39,200) and (200,177.61) .. (200,150) -- cycle ;
        %Shape: Circle [id:dp5633091016774019] 
        \draw   (225,150) .. controls (225,136.19) and (236.19,125) .. (250,125) .. controls (263.81,125) and (275,136.19) .. (275,150) .. controls (275,163.81) and (263.81,175) .. (250,175) .. controls (236.19,175) and (225,163.81) .. (225,150) -- cycle ;
        %Straight Lines [id:da044579420049741914] 
        \draw  [dash pattern={on 0.84pt off 2.51pt}]  (270.25,104.56) -- (260.13,127.28) ;
        %Straight Lines [id:da9218408814716863] 
        \draw  [dash pattern={on 0.84pt off 2.51pt}]  (295.8,130.2) -- (260.13,127.28) ;
        %Straight Lines [id:da5249492794602243] 
        \draw  [dash pattern={on 4.5pt off 4.5pt}]  (309.8,149) -- (280.6,131) ;
        %Straight Lines [id:da7471615005961604] 
        \draw  [dash pattern={on 4.5pt off 4.5pt}]  (265.19,115.92) -- (241.8,90.6) ;
        %Shape: Circle [id:dp8921119746382541] 
        \draw   (259.3,127) .. controls (259.3,126.45) and (259.75,126) .. (260.3,126) .. controls (260.85,126) and (261.3,126.45) .. (261.3,127) .. controls (261.3,127.55) and (260.85,128) .. (260.3,128) .. controls (259.75,128) and (259.3,127.55) .. (259.3,127) -- cycle ;
        %Shape: Circle [id:dp188848317748754] 
        \draw   (295.1,130.3) .. controls (295.1,129.75) and (295.55,129.3) .. (296.1,129.3) .. controls (296.65,129.3) and (297.1,129.75) .. (297.1,130.3) .. controls (297.1,130.85) and (296.65,131.3) .. (296.1,131.3) .. controls (295.55,131.3) and (295.1,130.85) .. (295.1,130.3) -- cycle ;
        
        % Text Node
        \draw (257.13,124.88) node [anchor=south] [inner sep=0.75pt]  [font=\tiny]  {$z$};
        % Text Node
        \draw (297.8,133.8) node [anchor=south west] [inner sep=0.75pt]  [font=\tiny]  {$w$};
        % Text Node
        \draw (311.8,157.93) node [anchor=south west] [inner sep=0.75pt]  [font=\tiny]  {$d( w,z)$};
        % Text Node
        \draw (246.8,94.2) node [anchor=south east] [inner sep=0.75pt]  [font=\tiny]  {$R-r$};
        
        
        \end{tikzpicture}
        
        


        \end{center}
        Recall that for a closed set $C$ $d(z,C)=\min_{w\in C}\set{d(w,z)}$. In the case of the disk of radius $R$ we have that 
        $$d(z,\cC)=R-r\geq d(z,w),\word{where}w\in\cC.$$
        This means that 
        $$|w-z|\geq R-r\To\frac{1}{|w-z|}\leq\frac{1}{R-r}.$$
        Now taking the complex modulus of $\rho_N$ for $z$ with $|z|=r$, we have
        $$|\rho_N(z)|\leq\frac{r^n}{2\pi}\oint\limits_\cC\frac{|f(z)|}{|w-z||w|^N}\dd w\leq \frac{r^N}{2\pi}\frac{M}{(R-r)R^N}(2\pi R)=\frac{Mr}{R-r}\left(\frac{r}{R}\right)^{N-1}.$$
        \item It suffices to show that $|\rho_N|\to 0$ as this implies $\rho_N\to0$. Notice that $\frac{r}{R}<1$ because $r<R$. So this means that $\left(\frac{r}{R}\right)^{N-1}<\frac{\eps(R-r)}{rM}$ where $\eps>0$ and $N$ is large enough. Thus for such an $N$ we have 
        $$|\rho_N(z)|\leq \frac{rM}{R-r}\left(\frac{r}{R}\right)^{N-1}<\frac{rM}{R-r}\frac{\eps(R-r)}{rM}=\eps.$$
        We thus conclude that $|\rho_N(z)|\to 0$ as $N\to\infty$ and therefore $\rho_N\to 0$ as well.
    \end{enumerate}
\end{ptcbr}
\begin{Ej}
    For $0\leq r\leq n$, the binomial coefficient $\binom{n}{r}$ is defined by $\binom{n}{r}=\frac{n!}{r!(n-r)!}$.
    \begin{enumerate}[i)]
        \itemsep=-0.4em
        \item Let $\ga=\del B(0,1)$, the positive, circular arc around $0$ with radius $1$. Show that 
        $$\binom{n}{r}=\frac{1}{2\pi i}\oint\limits_\ga \frac{(1+z)^n}{z^{r+1}}\dd z.$$
        \aside{In fact, this works for any simple closed curve around the origin.}
        \item Use the previous item to show that $\binom{n}{r}\leq 2^n$. \emph{This can also be shown directly by computing $(1+1)^n$.}
    \end{enumerate}
\end{Ej}

\begin{ptcbr}
    \begin{enumerate}[i)]
        \itemsep=-0.4em
        \item Notice that the integral in question, by the binomial theorem is: 
        $$\frac{1}{2\pi i}\oint\limits_\ga \frac{1}{z^{r+1}}\left(\sum_{k=0}^{n}\binom{n}{k}z^k\right)\dd z=\frac{1}{2\pi i}\sum_{k=0}^n\binom{n}{k}\oint\limits_{\ga}z^{k-r-1}\dd z.$$
        Notice that most of the terms in the sum are actually zero because %$z^{k-r-1}$ is holomorphic about the origin.
        $$\oint\limits_{\ga}z^{k-r-1}\dd z=(2\pi i)\dl_{kr}.$$
        With this, we have the desired equality as the expression is equal to 
        $$\frac{1}{2\pi i}\binom{n}{r}\oint\limits_{\ga}z^{-1}\dd z=\binom{n}{r}.$$
        \item Now bounding terms in the integral we have that 
        $$\binom{n}{r}\leq\frac{1}{2\pi}(2^n)\text{len}(\ga)=2^n.$$
    \end{enumerate}
\end{ptcbr}
\begin{Ej}[\cite{Stein} 2.7]
    Suppose $f\:\bD\to\bC$ is holomorphic. Show that the diameter $d=\sup_{z,w\in\bD}|f(z)-f(w)|$ of the image of $f$ satisfies $2|f'(0)|\leq d$.\par
    Moreover, it can be shown that equality holds precisely when $f$ is linear, $f(z)=az+b$.
    \aside{In connection with this result, see the relationship between the diameter of a curve and Fourier series described in Problem 1, Chapter 4, Book I.}
    \hint{$2f'(0)=\frac{1}{2\pi i}\oint\limits_{|\ze|=r}\frac{f(\ze)-f(-\ze)}{\ze^2}\dd\ze$ when $0<r<1$.}
\end{Ej}
%https://math.stackexchange.com/questions/1375438/stein-shakarchi-complex-analysis-ch-2-ex-7?noredirect=1&lq=1
\begin{ptcbr}
    To show the desired identity for $f'(0)$ it suffices to use Cauchy's formula:
    $$f'(z)=\frac{1}{2\pi i}\oint\limits_{|w|=r}\frac{f(w)}{(w-z)^2}\dd w\To f'(0)=\frac{1}{2\pi i}\oint\limits_{|w|=r}\frac{f(w)}{w^2}\dd w.$$
    Now take the change of variable $w=-u\To \dd w=-\dd u$. The curve through which we are integrating remains the same, as $|w|=|-u|=|u|$. Thus 
    $$f'(0)=\frac{1}{2\pi i}\oint\limits_{|u|=r}\frac{-f(u)}{u^2}\dd u.$$
    Renaming variables and adding the last two results we get 
    $$2f'(0)=\frac{1}{2\pi i}\oint\limits_{|\ze|=r}\frac{f(\ze)-f(-\ze)}{\ze^2}\dd\ze.$$
    Taking the complex modulus we see that 
    $$2|f'(0)|\leq\frac{1}{2\pi}\oint\limits_{|\ze|=r}\frac{|f(\ze)-f(-\ze)|}{|\ze|^2}\dd\ze\leq \frac{1}{2\pi r^2}\sup_{|\ze|=r}|f(\ze)-f(-\ze)|(2\pi r)$$
    As $\set{|\ze|=r}\subseteq\bD$ we have that 
    $$\sup_{|\ze|=r}|f(\ze)-f(-\ze)|\leq d\To 2|f'(0)|\leq\frac{d}{r}.$$
    As the last inequality holds for all $0<r<1$, in particular we have that 
    $$|f'(0)|\leq\inf_{0<r<1}\frac{d}{r}=d$$
    which is the desired inequality.
    %https://math.stackexchange.com/questions/3619008/2-leftf0-right-sup-z-omega-in-mathbbd-leftfz-f-omega-right?noredirect=1&lq=1
    %https://math.stackexchange.com/questions/339569/a-result-similar-to-schwarz-lemma
    %http://manetheren.bigw.org/~ray/diampblm.pdf
\end{ptcbr}
\begin{Ej}
    Let $\Om$ be a bounded open subset of $\bC$, and $\vf\:\Om\to\Om$ a holomorphic function. Prove that if there exists a point $z_0\in\Om$ such that $\vf(z_0)=z_0$ and $\vf'(z_0)=1$,
    then $\vf$ is linear.
\hint{Why can one assume that $z_0=0$? Write $z+a_nz^n+O(z^{n+1})$ near $0$, and prove that if $\vf_k=\vf\circ\vf\circ\dots\circ\vf$ (where $\vf$ appears $k$ times), then $\vf_k(z)=ka_nz^n+O(z^{n+1})$. Apply the Cauchy inequalities and let $k\to\infty$ to conclude the proof.}
\end{Ej}

\begin{ptcbr}
    %https://math.stackexchange.com/questions/275270/holomorphic-function-varphi-with-fixed-point-z-0-such-that-varphiz-o-1
   Let us begin by making some observations:
   \begin{itemize}
    \itemsep=-0.4em
    \item As $\vf$ is holomorphic inside $\Om$, it is analytic inside $\Om$ and thus we may write it as a power series centered at $z_0$
    $$\vf(z)=\sum_{n\geq 0}^{}a_n(z-z_0)^n.$$
    The conditions in the problem imply that $a_0=z_0$ and $a_1=1$ which means that 
    $$\vf(z)=z_0+(z-z_0)+\sum_{n\geq 2}^{}a_n(z-z_0)^n.$$
    \item We can thus consider the function $\vf(z)-z_0$ and move the whole domain to the origin by translating by $z_0$. This way we can reinterpret 
    $$\vf(z)=z+\sum_{n\geq 2}a_nz^n.$$
    \item Suppose $m\geq 2$ is the smallest integer such that $a_m\neq 0$ in the expansion of $\vf$, we this we have  
    $$\vf(z)=z+a_mz^m+O(z^{m+1}).$$
    \item $\vf$ is a bounded function because $\text{im}(\vf)\subseteq\Om$ and as $\Om$ is bounded, there exists $r>0$ such that $\Om\subseteq B(0,r)$. this means that $\norm{\vf}_{\infty}\leq r$. 
   \end{itemize}
   With this in hand we are ready to proceed. Consider the composition of $\vf$ with itself. We have 
   \begin{align*}
   &\vf\left(z+a_mz^{m}+O(z^{m+1})\right)\\
   =&\left(z+a_mz^{m}+O(z^{m+1})\right)+a_m\left(z+a_mz^{m}+O(z^{m+1})\right)^m+O(z^{m+1})\\
   =&z+a_mz^m+a_mz^m+O(z^{m+1})\\
   =&z+2a_mz^m+O(z^{m+1}),
   \end{align*}
   where the second-to-last equality comes after expanding the $m^{\text{th}}$ power and realizing that all the other terms in the expansion belong in $O(z^{m+1})$. Inductively we have 
   \begin{align*}
    &\vf\left(z+(k-1)a_mz^{m}+O(z^{m+1})\right)\\
    =&\left(z+(k-1)a_mz^{m}+O(z^{m+1})\right)+a_m\left(z+(k-1)a_mz^{m}+O(z^{m+1})\right)^m+O(z^{m+1})\\
    =&z+(k-1)a_mz^m+a_mz^m+O(z^{m+1})\\
    =&z+ka_mz^m+O(z^{m+1}),
    \end{align*}
    so in general $\vf_k$ is what we expect it to be. Notice that $vf_k$ is still a function from $\Om$ to $\Om$ which means it's uniformly bounded independent of $k$. Now by using the Cauchy inequality we have 
    $$\left|D^m\vf_k(z)\right|=km!|a_m|\leq\frac{m!}{R^m}\norm{\vf_k}_{\infty}\To |a_m|\leq \frac{\norm{\vf_k}_\infty}{kR^m}\xrightarrow{k\to\infty}0.$$
    As $m$ is the smallest index such that $a_m$ isn't zero, we conclude that there can be no such smallest index. All the $a_m$ are zero except for the linear coefficient and thus we conclude that $\vf$ is a linear function.
\end{ptcbr}

\section{Day 11| 20230213}

We have used the Cauchy integral formula for derivatives. As a corollary we have that 

\begin{Cor}
    If $f$ is holomorphic about $z_0\in\Om$, then $f$ is analytic at $z_0$.
\end{Cor}

\begin{Th}
    Suppose $f$ is holomorphic on a domain $\Om$ with $(z_n)\subseteq\Om$ such that $z_n\to z_0\in\Om$. Suppose that $f(z_n)=0$ for all $n$, then $f$ is identically zero.
\end{Th}

\begin{ptcbp}
    About $z_0$ we have 
    $$f(z)=\sum_{n\geq m}^{}a_n(z-z_0)^n=(z-z_0)^m\sum_{n\geq m}^{}a_n(z-z_0)^{n-m}=(z-z_0)^mg(z),\ a_m\neq 0.$$
    We also have that $g(z_0)=a_m\neq 0$. As $g$ is holomorphic, $|z-z_0|<\eps$ implies $g(z)\neq 0$. But for large enough $k$, if $|z_k-z_0|<\eps$, if $f(z_k)=0$, then 
    $$f(z_k)=(z_k-z_0)^mg(z_k).$$
\end{ptcbp}

\begin{Th}
    Suppose $f,g$ are holomorphic on a domain $\Om$. Suppose also that there is a non-constant sequence $(z_n)\subseteq\Om$ such that $z_n\to z\in\Om$ and $f(z_n)=g(z_n)$. Then $f=g$.
\end{Th}

\begin{ptcbp}
    Letting $h=f-g$, we see that $h$ is identically zero with the previous result so $f=g$.
\end{ptcbp}

\begin{Def}
    We say a that a sequence of functions $(f_n)$ converges to $f$ pointwise on $S$, if 
    $$\forall z(f_n(z)\xrightarrow[]{n\to\infty} f(z)).$$
    Meanwhile the convergence is uniform if 
    $$\norm{f_n-f}_\infty\xrightarrow[]{n\to\infty}0.$$
\end{Def}

\begin{Ex}
    The sequence of functions $(z^n)$ converges pointwise to $\dl_{z0}$ which is not continuous. Uniform convergence preserves continuity so $(z^n)$ is not uniformly convergent.\par
    Power series converge uniformly.
\end{Ex}

\section{Day 12| 20230215}

\begin{Th}
    Suppose $(f_n)$ is a sequence of holomorphic functions such that $f_n\to f$ uniformly on compact sets. Then $f$ is holomorphic.
\end{Th}

\begin{Lem}[Uniform Convergence Theorem]
    $f_n\to f$ uniformly $\To$ $\int f_k\to \int f$.
\end{Lem}

\begin{ptcbp}
    By Morera's theorem, it's enough to show that for triangular contours in $\Om$, $\int\limits_Tf(z)\dd z=0$. The triangle is a compact set and so we are done.
\end{ptcbp}

\red{FINISH}

\section{Day 13| 20230217}

Recall the Schwarz reflection principle. We will move into meromorphic functions now, so to begin we will assume $f$ is holomorphic on a domanin $\Om$ and that it's not identically zero.

\section{Day 14| 20230220}

Cauchy residue formula for one residue and multiple residues.

\begin{Ex}
    We will compute the following principal value:
    $$\lim_{R\to\infty}\frac{-R}{R}\frac{x^2}{x^6+1}\dd x.$$
    The function $f(z)=\frac{z^2}{z^6+1}$ has poles at the roots of $z^6+1$. This means that 
    $$z^6+1=0\iff e^{6it}=e^{\pi i+2\pi ik},\ k\in\bZ.$$
    So solutions to this equation are 
    $$t=\frac{\pi}{6}+\frac{2\pi ik}{6},\ k\in\bZ.$$
    We will only use $0\leq k\leq 5$ in this case.\par
    We now integrate over the half disk contour $\cC$ with radius $R$ of the upper half-plane. By the residue theorem we have that 
    $$\int\limits_\cC\frac{z^2}{z^6+1}\dd z=\sum_{k=0}^{5}\res\left(\frac{z^2}{z^6+1},e^{\frac{\pi}{6}+\frac{2\pi ik}{6}}\right).$$ 
\end{Ex}


\section{Interim 5|HW5}

\begin{Ej}[Stein \& Shakarchi 2.15]
    Suppose $f$ is a non-vanishing continuous function on $\ov\bD$ that is holomorphic in $\bD$. Suppose that 
    $$|f(z)|=1\word{when}|z|=1$$\hint{You will need to use the fact that, away from $0$, $z\mapsto\frac1z$ is continuous; so $z$ and $z_0$ are close if and only if $\frac1z$ and $\frac{1}{z_0}$ are close.}
\end{Ej}
%%%%%%%%%%%% Contents end %%%%%%%%%%%%%%%%
\ifx\nextra\undefined
\printindex
\else\fi
\nocite{*}
\bibliographystyle{plain}
\bibliography{bibiComAnal.bib}
\end{document} 
