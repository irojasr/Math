\documentclass[12pt]{memoir}

\def\nsemestre {I}
\def\nterm {Spring}
\def\nyear {2023}
\def\nprofesor {Jeff Achter}
\def\nsigla {MATH519}
\def\nsiglahead {Complex Analysis}
\def\nextra {HW8}
\def\nlang {ENG}
\input{../../headerVarillyDiff}

\begin{document}

\begin{Ej}
    Let $\Om$ be a domain containing the unit circle $C_1(0)$. Show that there is no function $F(z)$ holomorphic on $\Om$ such that $e^{F(z)}=z$ on $\Om$. \hint{What would $F'(z)$ be? Can you prove that $F'(z)$ admits no primitive on $\Om$?}
\end{Ej}

\begin{ptcbr}
Suppose by way of contradiction that $F$ is a holomorphic function which satisfies the equation. Differentiating we get 
$$e^{F(z)}F'(z)=1\to F'(z)=\frac{1}{e^{F(z)}}=\frac1z.$$
In $\bC\less\rbonj{-\infty,0}$ we have that $\frac{1}{z}$ admits $\log(z)$ as a primitive. So let us define $G(z)=F(z)-\log(z)$, differentiating $G$ we obtain $\frac1z-\frac1z=0$ so this means that $G$ is constant.\par 
From this we deduce that $F(z)=\log(z)+C$ where $C$ is constant. However $F$ is defined in a domain which contains the complete unit circle and the logarithm can only be defined as a multi-valued function in that region, not as a particular branch. This contradicts the fact that $F$ is a function.\par 
Therefore our assumption that $F$ is holomorphic and satisfies the equation is false. In conclusion, there does not exist such a function.
\end{ptcbr}

\begin{Ej}
    Let $\Om$ be a domain with $0\not\in\Om$.
    \begin{enumerate}
        \itemsep=-0.4em
        \item Suppose $f,g$ are continuous branches of the logarithm on $\Om$. Show that there is some integer $n$ such that $g(z)=f(z)+2\pi i n$. \hint{$\Om$ is connected.}
        \item Suppose $f(z)$ is a continuous branch of the logarithm. Show that $f(z)$ is holomorphic. \hint{$\Om$ can be covered by simply connected domains.}
    \end{enumerate}
\end{Ej}

\begin{ptcbr}
    \begin{enumerate}
        \itemsep=-0.4em
        \item If $f,g$ are continuous branches of the logarithm then 
        $$f'(z)=g'(z)=\frac{1}{z},\word{for}z\in\Om.$$
        With this $f-g=c$ where $c$ is a constant. Now 
        $$z=e^{g(z)}=e^{g(z)+c}=e^c z\To e^c=1\To c=2\pi i n$$
        for some $n\in\bZ$. 
        \item Let $f$ be a continuous branch of the logarithm. This is, $e^{f(z)}=z$, so we may apply the inverse function theorem to $e^z$ on the domain $\Om$.
        For any $w_0\in\Om$, let $f(w_0)=z_0$. Then we have that $\dv{e^z}{z}(z_0)=e^{z_0}\neq 0$ so by the inverse function theorem there exists a neighborhood $U\ni z_0$ such that $e^z$ is locally invertible. Also it must occur that the inverse is holomorphic, so $f$ is holomorphic at every point of $\Om$ as our point was arbitrary. 
    \end{enumerate}
\end{ptcbr}

\begin{Ej}[5.1  Stein \& Shakarchi]
    Give another proof of Jensen's formula in the unit disc using the functions
(called Blaschke factors)
$$\psi_\al(z)=\frac{\al-z}{1-\ov\al z}.$$
\hint{The function $f /(\psi_{z_1}\dots\psi_{z_n})$ is nowhere vanishing.}
\end{Ej}

\begin{ptcbr}
    Following the proof outline we will consider the function 
    $$g=\frac{f}{\psi_1\dots\psi_n}$$
    where $\psi_j(z)=\frac{z-z_j}{1-\ov z_j z}$ is the Blaschke factor associated to the root $z_j$ of $f$. In this case the roots of $f$ lie inside the unit circle and $g$'s zeroes lie outside the unit circle as they are of the form $\frac{1}{\ov z_j}$.\par
    Now, we know that the formula holds for nowhere vanishing functions, and also that the formula is multiplicative. So it suffices to prove it for the Blaschke factors.\par 
    Consider $\psi_j(z)$, the formula states 
    $$\log|\psi_j(0)|=\sum_{z\in Z}\log\left(\frac{|z|}{1}\right)+\frac{1}{2\pi}\int\limits_0^{2\pi}\log|\psi_j(1\.e^{i\te})|\dd\te$$
    where the set $Z$ is the set of zeroes of $\psi_j$. In this case, we only have one zero: $z_j$, so the sum becomes a unique term.\par 
    For the integral, we ought to remember that Blaschke factors map the unit circle to the unit circle. This means that there is a $\vf$ such that $e^{i\vf}=\psi_j(e^{i\te})$. Replacing this into the integral we get 
    $$\int\limits_0^{2\pi}\log|\psi_j(e^{i\te})|\dd\te=\int\limits_0^{2\pi}\log|e^{i\vf}|\dd\te=\int\limits_0^{2\pi}\log(1)\dd\te=0.$$
    So what the formula tells us is that 
    $$\log|\psi_j(0)|=\log|z_j|$$
    but this is true because $\psi_j$ exchanges $0$ and $z_j$.\par 
    Finally as the formula is multiplicative, it holds for $f=g\psi_1\dots\psi_n$ and so we have proven it in the case of the unit circle.
\end{ptcbr}
\end{document} 
