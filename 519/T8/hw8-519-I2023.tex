\documentclass[12pt]{memoir}

\def\nsemestre {I}
\def\nterm {Spring}
\def\nyear {2023}
\def\nprofesor {Jeff Achter}
\def\nsigla {MATH519}
\def\nsiglahead {Complex Analysis}
\def\nextra {HW8}
\def\nlang {ENG}
\input{../../headerVarillyDiff}

\begin{document}

\begin{Ej}
    Let $\Om$ be a domain containing the unit circle $C_1(0)$. Show that there is no function $F(z)$ holomorphic on $\Om$ such that $e^{F(z)}=z$ on $\Om$. \hint{What would $F'(z)$ be? Can you prove that $F'(z)$ admits no primitive on $\Om$?}
\end{Ej}

\begin{ptcbr}
Suppose $F$ is a function which satisfies the equation. Differentiating we get 
$$e^{F(z)}F'(z)=1\to F'(z)=\frac{1}{e^{F(z)}}=\frac1z.$$
In $\bC\less\rbonj{-\infty,0}$ we have that $\frac{1}{z}$ admits $\log(z)$ as a primitive. So let us define $G(z)=F(z)-\log(z)$, this function has derivative $0$ so $G$ is constant.\par 
This means that $F(z)=\log(z)+C$
\end{ptcbr}

\begin{Ej}
    Let $\Om$ be a domain with $0\not\in\Om$.
    \begin{enumerate}
        \itemsep=-0.4em
        \item Suppose $f,g$ are continuous branches of the logarithm on $\Om$. Show that there is some integer $n$ such that $g(z)=f(z)+2\pi i n$. \hint{$\Om$ is connected.}
        \item Suppose $f(z)$ is a continuous branch of the logarithm. Show that $f(z)$ is holomorphic. \hint{$\Om$ can be covered by simply connected domains.}
    \end{enumerate}
\end{Ej}

\begin{ptcbr}
    \begin{enumerate}
        \itemsep=-0.4em
        \item If $f,g$ are continuous branches of the logarithm then 
        $$f'(z)=g'(z)=\frac{1}{z},\word{for}z\in\Om.$$
        With this $f-g=c$ where $c$ is a constant. Now 
        $$z=e^{g(z)}=e^(g(z)+c)=e^c z\To e^c=1\To c=2\pi i n$$
        for some $n\in\bZ$. 
        \item Let $f$ be a continuous branch of the logarithm. This is, $e^{f(z)}=z$, so we may apply the inverse function theorem to $e^z$ on the domain $\Om$.
        For any $w_0\in\Om$, let $f(w_0)=z_0$. Then we have that $\dv{e^z}{z}(z_0)=e^{z_0}\neq 0$ so by the inverse function theorem there exists a neighborhood $U\ni z_0$ such that $e^z$ is locally invertible. The theorem also guarantees that the inverse must be holomorphic, so it must happen that $f$ is holomorphic at every point of $\Om$. 
    \end{enumerate}
\end{ptcbr}
\end{document} 
