\documentclass[12pt]{memoir}

\def\nsemestre {I}
\def\nterm {Spring}
\def\nyear {2023}
\def\nprofesor {Jeff Achter}
\def\nsigla {MATH519}
\def\nsiglahead {Complex Analysis}
\def\nextra {HW6}
\def\nlang {ENG}
\input{../../headerVarillyDiff}

\begin{document}

\begin{Ej}
    Prove that all entire functions that are also injective take the form $f (z) = az + b $with $a, b \in\bC$, and $a \neq 0$. \hint{Apply the Casorati-Weierstrass theorem to $f(1/z)$.}
\end{Ej}

\begin{ptcbr}
The function $g(z)=f(1/z)$ has a singularity at $z=0$. If it were removable, then $g$ is bounded on $B(0,R)$ for some $R>0$.\par 
This means that $f$ is bounded outside $B(0,R)$, but as $f$ is entire, it's continuous and so it's bounded \emph{inside} $B(0,R)$.\par 
By Liouville's theorem $f$ is constant. But that contradicts the fact that $f$ is injective.\par 
Now assume $g$ has an essential singularity at $z=0$. By the Casorati-Weierstrass theorem, we have a neighborhood of the origin $B(0,R)$ with $R>0$, such that $g\bonj{B(0,R)}$ is dense in $\bC$. This means that $f\bonj{\set{|z|>R}}$ is dense in $\bC$ and we have that $f[B(0,R)]$ is an open set.\par 
Recall now that dense subsets of $\bC$ intersect every non trivial open set, in particular this means that
$$f[B(0,R)]\cap f\bonj{\set{|z|>R}}\neq \emptyset$$
and so for any $w\in f[B(0,R)]\cap f\bonj{\set{|z|>R}}$ we can find $z_1$ with $|z_1|<R$ and $z_2$ with $|z_2|>R$ such that
$$f(z_1)=f(z_2)=w,\word{and}z_1\neq z_2.$$
This contradicts the fact that $f$ is injective. Thus, the only type of singularity that may occur at $z=0$ is a pole.\par 
From here we see that in the Taylor expansion of $f$, the analytic part coincides with $g$'s principal part. As $g$'s principal part must be finite, $f$ must be a polynomial. The degree of $f$ can't be larger than $1$ because $f$ is injective, it can't also be $0$ because $f$ is injective and so we conclude that $f$ is linear as desired.
\end{ptcbr}

\begin{Ej}
    As in class, consider the unit sphere
    $$X=\set{(a,b,c)\:\ a^2+b^2+c^2=1}\subseteq\bR^3$$
    Let $N = (0, 0, 1), S = (0, 0, -1), U_N = X\less N, U_S = X\less S$. Consider the following three
    charts on $X$:
    \begin{itemize}
        \item $\phi_N\: U_N\to\bC,\ (a,b,c)\mapsto\frac{a+ib}{1-c}$.
        \item $\phi_S\: U_S\to\bC,\ (a,b,c)\mapsto\frac{a+ib}{1+c}$.
        \item $\psi_S\: U_S\to\bC,\ (a,b,c)\mapsto\frac{a-ib}{1+c}$.
    \end{itemize}
    Do the following:
    \begin{enumerate}[i)]
        \itemsep=-0.4em
        \item The inverse of $\phi_N$ is 
        $$\phi_N^{-1}(z)=\left(\frac
        {2\Re(z)}{|z|^2+1},\frac
        {2\Im(z)}{|z|^2+1},\frac
        {|z|^2-1}{|z|^2+1}\right).$$
        Calculate $\phi_S^{-1}$ and $\psi_S^{-1}$.
        \item Among the three charts $\set{(U_N,\phi_N),(U_S,\phi_S),(U_S,\psi_S)}$, one pair is compatible and the other two are not. Which is which? Why?\par 
        \hint{Remember a function is holomorphic if and only if $\del_{\ov z}f=0$.}
    \end{enumerate}
\end{Ej}

\begin{ptcbr}
The function $g(z)=f(1/z)$ has a singularity at $z=0$. If it were removable, then $g$ is bounded on $B(0,R)$ for some $R>0$.\par 
This means that $f$ is bounded outside $B(0,R)$, but as $f$ is entire, it's continuous and so it's bounded \emph{inside} $B(0,R)$.\par 
By Liouville's theorem $f$ is constant. But that contradicts the fact that $f$ is injective.\par 
Now assume $g$ has an essential singularity at $z=0$. By the Casorati-Weierstrass theorem, we have a neighborhood of the origin $B(0,R)$ with $R>0$, such that $g\bonj{B(0,R)}$ is dense in $\bC$. This means that $f\bonj{\set{|z|>R}}$ is dense in $\bC$
\end{ptcbr}
\end{document} 
