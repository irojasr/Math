\documentclass[12pt]{memoir}

\def\nsemestre {I}
\def\nterm {Spring}
\def\nyear {2023}
\def\nprofesor {Jeff Achter}
\def\nsigla {MATH519}
\def\nsiglahead {Complex Analysis}
\def\nextra {HW11}
\def\nlang {ENG}
\input{../../headerVarillyDiff}

\begin{document}

\begin{Ej}[7.8 Stein\& Shakarchi]
    The function $\zeta$ has infinitely many zeros in the critical strip. This can be seen as follows.
    \begin{enumerate}[i)]
        \item Let 
        $$F(s)=\xi(1/2+s),\word{where}\xi(s)=\pi^{-s/2}\Ga(s/2)\zeta(s).$$
        Show that $F(s)$ is an even function of $s$ and as a result, there exists $G$ such that $G(s^2)=F(s)$.
        \item Show that the function $(s-1)\zeta(s)$ is an entire function of growth order $1$, that is 
        $$|(s-1)\zeta(s)|\leq A_\eps e^{a_\eps|s|^{1+\eps}}.$$
        As a consequence $G(s)$ is of growth order $1/2$.
        \item Deduce from the above that $\zeta$ has infinitely many zeros in the critical strip.
    \end{enumerate}
    \hint{To prove the first two parts use the functional equation for $\zeta(s)$. For the last one, use a result of Hadamard, which states that an entire function with fractional order has infinitely many zeros (Exercise 14 in Chapter 5).}
\end{Ej}

\begin{ptcbr}
    \begin{enumerate}[i)]
        \item Observe that 
        \begin{align*}    
        F(-s)&=\xi(1/2-s)=\xi\left(\half-\half+\half-s\right)=\xi\left(1-\half-s\right)\\
        &=\xi\left(1-\left(\half+s\right)\right)=\xi\left(\half+s\right)
        \end{align*}
        where the last equality comes from the identity $\xi(s)=\xi(1-s)$ for all $s\in\bC$.
        \item We know that $\zeta(s)$ has a pole of order $1$ at $s=1$ and that's it's only pole. So the function $(s-1)\zeta(s)$ is holomorphic on the whole plain which means it's entire. \red{Show order of growth}
        \item Finally our function has non-integral order so it has an infinite number of roots. This follows from a exercise where we use Hadamard's factorization theorem.
    \end{enumerate}
\end{ptcbr}

\begin{Ej}[7.6 Stein\& Shakarchi]
    Read [SS]7.6, assume its result, and proceed as follows. Let $\dl$ be the function defined in [SS]7.6:
    $$\dl(a)=\left\lbrace
    \begin{aligned}
        &1\quad 1<a\\
        &\half\quad a=1\\
        &0\quad 0\leq a<1\\
    \end{aligned}
    \right.$$
    Fix a positive real number $X$ which is not an integer.
    \begin{enumerate}[i)]
        \item Show that $\Psi(X)=\sum_{n\geq 1}\La(n)\dl\left(\frac{X}{n}\right)$.
        \item Consider $G(s)=\frac{X^s}{s}\left(\frac{-\zeta'(s)}{\zeta(s)}\right)$, show that 
        $$\Psi(X)=\frac{1}{2\pi i}\hspace*{-1.5em}\int\limits_{\set{\Re(s)=c}}\hspace*{-1.5em}G(s)\dd s.$$
    \end{enumerate}
    \hint{Assume you can exchange summation and integration; you will need to use our formula from class for $L(\zeta(s))$, which is also in [SS] Chapter 7, section 2.}
\end{Ej}

\begin{ptcbr}
\begin{enumerate}[i)]
    \item Observe that the $\dl$ function can be expressed as a sum of indicator functions:
    $$\dl(a)=\ind_{\set{a>1}}+\half\ind_{\set{a=1}}+0\ind{\set{0\leq a<1}}.$$
    In this sense we have that for a fixed $n$, the function $\dl\left(\frac{X}{n}\right)$ is 
    $$\dl\left(\frac{X}{n}\right)=\ind_{\set{X>n}}+\half\ind_{\set{X=n}}+0\ind_{\set{X<n}}.$$
    So reminding ourselves that $X$ is a positive non-integer we have that $\dl(X/n)$ is never $\half$. Now fix $X$ so that 
    $$\sum_{n\geq 1}\La(n)\dl\left(\frac{X}{n}\right)=\sum_{n<X}\La(n)+0\sum_{n>X}\La(n).$$
    As the rest of the sum is zero because of the indicator, we have that the whole sum actually is $\sum_{n<X}\La(n)$ which is precisely our $\Psi$ function.
    \item If we now have the integral in question, we may replace $G$ by its definition and see that 
    $$\frac{1}{2\pi i}\hspace*{-1.5em}\int\limits_{\set{\Re(s)=c}}\hspace*{-1.5em}\frac{X^s}{s}\left(\frac{-\zeta'(s)}{\zeta(s)}\right)\dd s=\frac{1}{2\pi i}\hspace*{-1.5em}\int\limits_{\set{\Re(s)=c}}\hspace*{-1.5em}X^s\frac{\dd s}{s}\sum_{n=1}^\infty\frac{\La(n)}{n^s}.$$
    Subtly applying the dominated convergence theorem we may interchange the series with th integral to obtain 
    $$\sum_{n=1}^\infty\La(n)\frac{1}{2\pi i}\hspace*{-1.5em}\int\limits_{\set{\Re(s)=c}}\hspace*{-1.5em}\left(\frac{X}{n}\right)^s\frac{\dd s}{s}$$
    and by exercise $7.6$ we know that the integral in question is 
    $$\frac{1}{2\pi i}\hspace*{-1.5em}\int\limits_{\set{\Re(s)=c}}\hspace*{-1.5em}\left(\frac{X}{n}\right)^s\frac{\dd s}{s}=\ind_{\set{X>n}}$$
    which means that the whole expression is 
    $$\sum_{n=1}^\infty\La(n)\ind_{\set{X>n}}=\sum_{n<X}\La(n)$$
    as desired. As both expressions equal the same sum, we have that $\Psi$ is the integral in question.
\end{enumerate}
\end{ptcbr}

\begin{Ej}
    One uses the results of the previous problems in the following way.
    \begin{enumerate}[i)]
        \item Show that $\res(G,1)=X$. \hint{Use the fact that $\zeta(s)$ has a pole at $ s = 1$ of order $1$.}
        \item Show that $\res(G,0)=\lim_{s\to 0}\frac{-\zeta'(s)}{\zeta(s)}$. \emph{It turns out that this is $-\log(2\pi)$.}
        \item Show that $\sum\res(G,\rho)=-\frac{1}{2}\log(1-X^{-2})$, where the sum is over the trivial zeros of $\zeta(s)$.
    \end{enumerate}
    \emph{From here, moving c “all the way to the left” means that we pick up all the residues of $G(s)$, and we
    are left with von Mangoldt's explicit formula:}
    $$\psi(X)=X-\sum\frac{X^\rho}{\rho}-\frac{\zeta'(0)}{\zeta(0)}-\half\log(1-X^{-2})$$
    \emph{where the sum is over all critical zeroes of $\zeta(s)$.}
\end{Ej}

\begin{ptcbr}
    \begin{enumerate}[i)]
        \item Observe that 
        $$\res(G,1)=\lim_{s\to 1}(s-1)\frac{X^s}{s}\left(\frac{-\zeta'(s)}{\zeta(s)}\right)=\lim_{s\to 1}\frac{X^s}{s}\lim_{s\to 1}(s-1)(L(\zeta(s))).$$
        The limit on the right is the residue at $s=1$ of the logarithmic derivative of $\zeta$, it is know that this residue is the order of the point in question of the function. This means that 
        $$\res(G,1)=X\.-\ord(\zeta,1)=X\.1=X.$$
        \item In this case, we have that 
        $$\res(G,0)=\lim_{s\to 0}(s)\frac{X^s}{s}\left(\frac{-\zeta'(s)}{\zeta(s)}\right)=\left(\lim_{s\to 0}X^s\right)\left(\lim_{s\to 0}\frac{-\zeta'(s)}{\zeta(s)}\right)$$
        and the left limit turns to $1$ so we obtain the desired result.
        \item It is a subtle observation that 
        $$-\half\log(1-X^{-2})=\half\sum_{n\geq 1}\frac{\left(\frac
        {1}{X^2}\right)^n}{n}=\sum_{n\geq 1}\frac{X^{-2n}}{2n}.$$
        Now, the trivial zeroes of the zeta function are at $s=-2n$, so 
        $$\res(G,-2n)=\lim_{s\to-2n}(s+2n)\frac{X^s}{s}\left(\frac{-\zeta'(s)}{\zeta(s)}\right)=\left(\lim_{s\to-2n}\frac{X^s}{s}\right)(-\ord(\zeta,-2n))$$
        where the limit evaluates to $\frac{X^{-2n}}{-2n}$ and the order is $1$ so we obtain $\frac{X^{-2n}}{2n}$ and summing through all trivial zeroes we obtain the desired result.
    \end{enumerate}
\end{ptcbr}
\end{document} 
