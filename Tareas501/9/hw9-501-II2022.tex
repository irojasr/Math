\documentclass[12pt]{memoir}

\def\nsemestre {II}
\def\nterm {Fall}
\def\nyear {2022}
\def\nprofesor {Maria Gillespie}
\def\nsigla {MATH501}
\def\nsiglahead {Combinatorics}
\def\nextra {HW9}
\def\nlang {ENG}
\input{../../headerVarillyDiff}
\DeclareMathOperator{\des}{des}
\DeclareMathOperator{\inv}{inv}
\DeclareMathOperator{\exc}{exc}
\DeclareMathOperator{\maj}{maj}
\usepackage{halloweenmath}

\begin{document}

\begin{Ej}[Exercise 2]
    Let $d>1$ be a positive integer. A \emph{$d$-ary De Bruijn sequence} of degree $n$ is a sequence of length $d^n$ containing every length $n$ sequence in $([d-1]^\ast)^n$ exactly once as a circular factor.
    \begin{enumerate}[i)]
        \itemsep=-0.4em 
        \item Show that there always exists a $d$-ary De Bruijn sequence of degree $n$ for any $n$.
        \item Find the number of $d$-ary De Bruijn sequences that begin with $n$ zeroes. \hint{You may want to consult the computation for $d=2$ given at the end of chapter 5.}
    \end{enumerate}
\end{Ej}
\begin{nonum-Rmk}
For this exercise, we will shift notation up by one and instead of considering the alphabet set as $[d-1]^\ast=\set{0,1,\dots,d-1}$, we will consider $[d]=\set{1,\dots,d}$. We will also call $B(d,n)$ the set of De Bruijn sequences of length $d^n$ with alphabet $[d]$.\par 
Also, we will define the Bruijn graph $G_{d,n}$ as follows: 
$$
\left\lbrace
\begin{aligned}
    &V=[d]^n=\set{(s_1,\dots,s_n):\ \forall i(s_i\in[d])},\\
    &E=\set{((s_1,\dots,s_n),(t_1,\dots,t_n))\:\ \forall i(1\leq i\leq n-1\To t_i=s_{i+1})\land t_n\in[d]}.\\
\end{aligned}
\right.
$$
That is, the edge set is formed by pairs of strings of the form $(s_1,s_2,\dots,s_n)$ and $(s_2,s_3,\dots,s_n,t)$ where $t\in[d]$. We are shifting all indices in our string and adding a new admissible character.
\end{nonum-Rmk}
\begin{ptcbr}
\begin{enumerate}[i)]
    \itemsep=-0.4em 
    \item First, let us prove that $G_{d,n}$ is Eulerian. Consider any vertex $v=(s_1,\dots,s_n)\in G_{d,n}$, it holds that $d_{\text{out}}(v)=d_{\text{in}}(v)=d$.\par 
    First, consider the edges out of $v$: any out-neighbor of $v$ is a vertex $(s_2,\dots,s_n,t)$ with $t\in[d]$. Since there are $d$ options for the character $t$, $v$ sends an edge to each one of them.\par 
    Similarly, every edge which connects to $v$ comes from a vertex of the form $(t,s_1,\dots,s_{n-1})$. There are $d$ vertices of that form in $G_{d,n}$. We conclude that $d_{\text{out}}(v)=d_{\text{in}}(v)=d$.\par 
    The De-Bruijn graph is also strongly connected: any two vertices $u,v$ can be reached from one another after deleting and appending sufficient characters. It follows that $G_{d,n}$ is Eulerian for any $d$ and any $n$.\par 
    Take an Eulerian cycle in a De Bruijn graph $G_{d,n-1}$. Such cycle traverses all the edges of our graph, and given that $d_{\text{out}}(v)=d$ for all $v$, it holds that 
    $$|E(G_{d,n-1})|=d\.|G_{d,n-1}|=d\.d^{n-1}=d^n.$$
    Labeling the edges by the character it appends to each vertex, we get a string of length $d^n$ which contains all possible substrings of length $n$ in $d$ characters. A $d$-ary de Bruijn sequence is minimal with respect to this property so it must hold that the sequence generated is a De Bruijn sequence.
    \item This problem is asking us to find all Eulerian walks which begin on any edge out of the vertex $(1,1,\dots,1)$\footnote{Once again we remind ourselves of the preference of notation.}\par 
    Ideas:Recursion, BEST theorem, ch9 aigner
\end{enumerate}
\end{ptcbr}

\begin{Ej}
    An \emph{undirected Eulerian tour} is a tour on the edges of an undirected graph using every undirected edge exactly once (in just one direction). Derive necessary and sufficient conditions for the existence of an undirected Eulerian tour in an undirected graph. Prove your result
\end{Ej}

\begin{ptcbr}
    An Eulerian tour must begin and end somewhere. All the other vertices are part of the walk, so we must \emph{enter and exit} every one of those vertices. We should at least be able to \emph{exit} the first vertex and \emph{enter} the last one. Thus the result should be $G$ has Eulerian tour if and only if
    \begin{itemize}
        \itemsep=-0.4em
        \item Either all the vertices have even degree.
        \item Or exactly two vertices have odd degree. 
    \end{itemize}
    In the first case we have an Eulerian \emph{circuit} which is also a tour. On the second case, we can only find a tour from $u$ to $v$.
    Let us prove that 
\begin{center}
    \emph{A connected graph has an Eulerian circuit if and only iff all the vertices have even degree.}
\end{center}
From which we will deduce
\begin{center}
    \emph{A connected graph has an Eulerian tour if and only if two vertices have odd degree.}
\end{center}
Suppose $G$ has an Eulerian circuit labeled by the vertices $x_{1}x_{2}\dots x_mx_1$ where it is possible that there are repetitions among this sequence. Then, any vertex that appears $k$ times in this sequence will have degree $2k$ because it will be entered and existed according to the circuit.\par 
To prove the other direction we proceed by induction on the number of edges. Suppose $G$ is a graph with $m$ edges whose vertices all have even degrees. In the base case, when $m=0$
\end{ptcbr}

\begin{Ej}[Exercise 4]
    The adjacency matrix of a directed graph $D$ is $A$ such that $a_{ij}=\bool{(i,j)\in E}$.
    \begin{enumerate}[i)]
        \itemsep=-0.4em 
        \item Show that the $(i,j)^{\text{th}}$ entry of $A^k$ is the number of directed paths of length $k$ from $v_i$ to $v_j$ in $D$.
        \item Verify that the following equality holds, where we consider both sides as formal power series in $x$ with coefficients in the ring $M_n(\bQ)$: 
        $$(I-xA)^{-1}=1+xA+x^2A^2+x^3A^3+\dots$$
        \item Using the previous part along with the explicit formula for the inverse of a matrix (in terms of cofactors), show that the generating function for the number of paths $p_{i,j}(n)$ of length $n$ from $v_i$ to $v_j$ is 
        $$\sum_{n=0}^\infty p_{i,j}(n)x^n=\frac{(-1)^{i+j}\det\bonj{(I-xA)^{(j,i)}}}{\det(I-xA)},$$
        where $A^{(i,j)}$ is the $(i,j)^{\text{th}}$ minor of $A$.\par
        This result is called the \textbf{transfer-matrix method}, as it gives a method of proving a sequence has a rational generating function, by showing that the sequence counts paths in a certain directed graph.
        \item Let $b_n$ be the number of sequences of length $n + 1$ with entries from $[3]$ that
        start with $1$, end with $3$, and do not contain the subsequences $22$ or $23$. Find a closed formula for
        the generating function of $b_n$ using the transfer-matrix method, by constructing a directed graph
        in which certain paths are counted by $b_n$. You may use a computer to calculate the determinants,
        but you must write out the directed graph and the corresponding matrices.
    \end{enumerate}
\end{Ej}

\begin{ptcbr}
    \begin{enumerate}[i)]
        \itemsep=-0.4em 
        \item Suppose $D$ is a directed graph with $n$ vertices, we will proceed by induction and use $A^2$ as a base case. The $(i,j)^{\text{th}}$ entry of $A^2$ is given by 
        $$\sum_{k=1}^na_{ik}a_{kj}=a_{i1}a_{1j}+a_{i2}a_{2j}+\dots+a_{in}a_{nj},$$
        and every term $a_{ik}a_{kj}$ counts the number of edges from $v_i$ to $v_k$ times the number of edges from $v_k$ to $v_j$. But a length 2 path from $v_i$ to $v_j$ can go through any other vertex $u\in D$. Since each path must be different, we sum each possibility to get the complete number.\par 
        Suppose now that the $(i,j)^{\text{th}}$ entry of $A^m$ is the number of paths of length $m$ from $v_i$ to $v_j$ for $m\leq k-1$. Now 
        $$A_k=(A^{k-1})A\To (A^k)_{ij}=\sum_{\l=1}^n(A^{k-1})_{il}A_{\l j}.$$
        We can thus decompose the $(i,j)^{\text{th}}$ entry of $A^k$ into a sum of terms of the form $(A^{k-1})_{il}A_{\l j}$. By induction hypothesis $(A^{k-1})_{il}$ is the number of paths from $v_i$ to $v_\l$ and adding the edge which could go from $v_\l$ to $v_j$ we get a new path. However $v_\l$ can be any vertex in $D$, so summing the possibilities we get the complete number of length $k$ walks.\par 
        Thus we conclude that the $(i,j)^{\text{th}}$ entry of $A^k$ is the number of paths of length $k$ from $v_i$ to $v_j$ in $D$.
        \item We need to prove that the inverse of $I-xA$ is the matrix $1+xA+x^2A^2+x^3A^3+\dots$ and to do that we will multiply them:
        \begin{align*}
            &(I-xA)(1+xA+x^2A^2+x^3A^3+\dots)\\
            =&(1+xA+x^2A^2+x^3A^3+\dots)-(xA-x^2A^2-x^3A^3-x^4A^4-\dots)=I.
        \end{align*}
        We do not worry about convergence issues because this is a formal power series.
        \item The $(i,j)^{\text{th}}$ entry of $\sum_{n=0}^\infty A^nx^n$ is $\sum_{n=0}^\infty p_{i,j}(n)x^n$ and recalling that this matrix is actually $(I-xA)^{-1}$, we must find the entries of this matrix. By the inverse formula for a matrix we get 
        $$(I-xA)^{-1}=\frac{1}{\det(I-xA)}\adj(I-xA)=\frac{1}{\det(I-xA)}\text{cof}(I-xA)^\sT.$$
        The $(i,j)^{\text{th}}$ entry of the cofactors matrix is $(-1)^{i+j}\det[(I-xA)^{(i,j)}]$, the determinant of the minor matrix obtained by deleting the $i^{\text{th}}$ row and $j^{\text{th}}$ column. Thus the $(i,j)^{\text{th}}$ of the transpose is obtained by switching the indices.\par 
        We conclude that the $(i,j)^{\text{th}}$ entry of $\sum_{n=0}^\infty A^nx^n$ is 
        $$\left(\frac{1}{\det(I-xA)}\text{cof}(I-xA)^\sT\right)_{i,j}=\frac{1}{\det(I-xA)}(-1)^{i+j}\det[(I-xA)^{(j,i)}].$$
        \item Consider the following graph which encodes the construction of our string:
        \begin{center}
            % https://tikzcd.yichuanshen.de/#N4Igdg9gJgpgziAXAbVABwnAlgFyxMJZAJgBoBGAXVJADcBDAGwFcYkRyQBfU9TXfIRQAGUsOp0mrdsW68QGbHgJFRxCQxZtEIAMzcJMKAHN4RUADMAThAC2SUSBwQk5GgCMYYKEgC0u4R5LG3tEMicXRHIgkGs7JHDnBw8vH0R-QPk40LcI5JBPbz8AmOz8pMRHRggINCIATlELJjgYCUZ6T0YABX5lIRArLGMACxw5YPiwmgrw6trVUmbGVvbOmB6+wXYh0fHSkPLI8MK0jK5KLiA
\begin{tikzcd}
    2 \arrow[rrd]                                                        &  &                                                                                              \\
                                                                                     &  & 1 \arrow[llu, bend right] \arrow[loop, distance=2em, in=35, out=325] \arrow[lld] \\
    3 \arrow[uu] \arrow[rru, bend right] \arrow[loop, distance=2em, in=305, out=235] &  &                                                                                             
    \end{tikzcd}
        \end{center}
        Finding the amount of strings of length $n+1$ beginning with $1$, ending in $3$ is the same as counting walks from vertex $1$ to $3$ of length $n$. This means that $b_n=p_{1,3}(n)$ and we know that that sequence's generating function is 
        $$\frac{1}{\det(I-xA)}(-1)^{1+3}\det[(I-xA)^{(3,1)}].$$
        To find a closed form we start by noting that the adjacency matrix of our graph is 
        $$A=\threebythree{1}{1}{1}{1}{0}{0}{1}{1}{1}\To I-xA=\threebythree{1-x}{-x}{-x}{-x}{1}{0}{-x}{-x}{1-x}.$$
        We expand by minors using the third row so that we can calculate the minor determinant at once:
        \begin{align*}
            &\det(I-xA)\\
            =&(-x)\det\twobytwo{-x}{-x}{1}{0}-(-x)\det\twobytwo{1-x}{-x}{-x}{0}+(1-x)\det\twobytwo{1-x}{-x}{-x}{1}\\
            =&(-x)(0-(-x))+(x)(0-x^2)+(1-x)[(1-x)-x^2]\\
            =&1-2x-x^2
        \end{align*}
    \end{enumerate}
    We conclude that the generating function is $\frac{-x}{1-2x-x^2}$.
    \end{ptcbr}
\end{document}
