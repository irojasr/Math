\documentclass[12pt]{memoir}

\def\nsemestre {II}
\def\nterm {Fall}
\def\nyear {2022}
\def\nprofesor {Maria Gillespie}
\def\nsigla {MATH501}
\def\nsiglahead {Combinatorics}
\def\nextra {HW11}
\def\nlang {ENG}
\input{../../headerVarillyDiff}
\DeclareMathOperator{\des}{des}
\DeclareMathOperator{\inv}{inv}
\DeclareMathOperator{\exc}{exc}
\DeclareMathOperator{\maj}{maj}
\usepackage{halloweenmath}

\begin{document}

\begin{Ej}[Exercise 1 and 2]
    Let $P$ be a finite poset and $L=\cJ(P)$ be the corresponding distributive lattice. If $X\subseteq P$ is a lower-order ideal, then use the corresponding lowercase letter $x$ to denote the associated element of $L$. 
    \begin{enumerate}[i)]
        \itemsep=-0.4em
        \item Show that $x$ covers $y$ in $L$ if and only if $Y=X\less\set{m}$, where $m$ is a maximal element of $X$.
        \item Show that $x$ is join-irreducible in $L$ if and only if $X$ is a principal ideal of $P$. %https://math.stackexchange.com/questions/4008962/
    \end{enumerate}
 \end{Ej}

 We will use the following lemma which is Proposition 5.1.2 in Sagan. 

 \begin{Lem}
If $P$ is finite and $I$ is an ideal, then $I=\gen(S)$ where $S$ is the set of maximal elements in $I$. 
 \end{Lem}

\begin{ptcbp}
If $x\in I$, consider $X\set{y\in I:\ y\geq x}$, the set of upper bounds of $x$. This set is non-empty as $x\geq x$ and there must exist a maximal element of $X$ since $P$ is finite. Call $s\in X$ such a maximal element. If $s$ is not maximal in $I$, then $\exists \tilde{s}\in I(\tilde{s}\geq s)$ and this means that $\tilde{s}\geq x$. This contradicts the maximality of $s$ in $X$.\par 
This proves that $x\in\gen(S)$ as $s$ is a maximal element of $I$.\par 
On the other hand if $x\in\gen(S)$ then, $x\leq s$ for some $s\in S\subseteq I$. As $I$ is an order ideal and $s\in I$, then $x\in I$. 
\end{ptcbp}

Recall as well, $x$ is \emph{join-irreducible} when $x=y\lor z$ implies $x=y$ or $x=z$. Also, when ordering with $\subseteq$, join irreducibility means that $A= B\cup C$ implies $A=B$ or $A=C$. 

\begin{ptcbr}
    \begin{enumerate}[i)]
        \itemsep=-0.4em
        \item Let us begin by assuming that $x$ covers $y$ in $L$ this means that $x\supseteq y$ and $\not\exists z(x>z>y)$. In terms of elements, this means that $|x\less y|=1$. Take $n$ to be the element we removed from $x$, if $n$ is not maximal, then $y$ is no longer an order ideal as $n\leq s$ where $s$ is a maximal element in $y$.\par 
        On the other hand $X\less\set{m}\subseteq X$, so we only have to prove that $X\less\set{m}$ is an ideal. As $X$ is an ideal, $X$ is generated by its maximal elements $S$. We claim that $X\less\set{m}$ is an ideal generated by $S\less{m}$ and the elements covered by $m$. Thus, take an element $n\leq m$, then we have two cases:
        \begin{itemize}
            \itemsep=-0.4em 
            \item Either $n$ is maximal in $X\less\set{m}$ or, 
            \item there exists $s\in X$ another maximal element such that $n\leq s$ in which case $n\in\gen(s)$.
        \end{itemize} 
        Thus $X\less\set{m}$ is an order ideal and so $x$ covers $y=X\less\set{m}$. 
        \item Suppose $x$ is join-irreducible, now for every element $s\in x$ it holds that $\gen(s)\subseteq x$. On the other hand $x\subseteq\bigcup_{s\in x}\gen(s)$. Thus 
        $$x-\bigcup_{s\in x}\gen(s)\To\exists s'\in x(x=\gen(s'))$$
        because $x$ is join irreducible. Finally $x$ is principal.\par 
        On the other hand suppose $x=\gen(s)$ and now suppose $x=y\cup z$ where $y,z\in L$. Thus 
        $$s\in x\To s\in y\lor\ s\in z\To \gen(s)\subseteq y\lor\ \gen(s)\subseteq z\To x\subseteq y\lor\ x\subseteq z.$$
        As $x=y\cup z$, then $y\subseteq x$ and $z\subseteq x$. So either $x=y$ or $x=z$ in which case, $x$ is join-irreducible.
    \end{enumerate}
\end{ptcbr}

\begin{Ej}[Exercise 7]
Define $Q^P=\set{f:P\to Q\mid\ f\ \text{is order-preserving}}$ along with the order $f\leq_{Q^P}g$ if the inequality holds pointwise. Prove that for any posets $P,Q,R$ we have:
\begin{enumerate}[i)]
    \itemsep=-0.4em
    \item $R^{P+Q}\isom R^P\x R^Q$.
    \item $(R^Q)^P\isom R^{Q\x P}$.
\end{enumerate}    
\end{Ej}

First recall that $P+Q$ is the poset $(P\cup Q,\leq)$ where $s\leq t$ when $s\leq_P t$ or $s\leq_Q t$. If the elements are in distinct sets, they are not comparable.\par 
The poset $P\x Q$'s order relation is determined component-wise.\footnote{\textbf{Sarah} was the one who got me into this problem because I went to ask her about problem 1, but she asked about this problem. We solved it partially and finished the last part with \textbf{Kyle} and then I asked her about 1.a.}

\begin{ptcbr}
\begin{enumerate}[i)]
    \itemsep=-0.4em
    \item Consider the functions 
    \begin{align*}
        &\vf: R^{P+Q}\to R^P\x R^Q,\ f\mapsto (f,f),\word{and}\\
        &\psi: R^P\x R^Q\to R^{P+Q},\ (g,h)\mapsto\left\lbrace\begin{aligned}
            g,\ x\in P,\\
            h,\ x\in Q.
        \end{aligned}\right.
    \end{align*}
    We can see that 
    $$\vf(\psi(g,h))=\left(\left\lbrace\begin{aligned}
        g,\ x\in P\\
        h,\ x\in Q
    \end{aligned}\right.,\left\lbrace\begin{aligned}
        g,\ x\in P\\
        h,\ x\in Q
    \end{aligned}\right.\right),$$
    and as the first entry is always a function which acts on $P$, then the first entry is always $g$. Likewise for $h$. On the flipside, 
    $$\psi(\vf(f))=\left\lbrace\begin{aligned}
        f,\ x\in P\\
        f,\ x\in Q
    \end{aligned}\right.$$
    but as $f$ is a function defined on $P+Q$, this function acts the same as $f$. We conclude that the composition of the maps is the identity and so they are inverses of each other.\par 
    Now we will prove they are \emph{order-preserving}. 
    \begin{itemize}
        \itemsep=-0.4em 
        \item If $f_1\leq f_2$, then we want to show that $(f_1,f_1)\leq(f_2,f_2)$. For this to happen, the order relation must hold component-wise. But that relation is precisely $f_1\leq f_2$. 
        \item On the other hand suppose $(g_1,h_1)\leq(g_2,h_2)$, then we want to see that $\psi(g_1,h_1)\leq\psi(g_2,h_2)$. Thus it must hold that 
        $$\forall x\in P+Q(\psi(g_1,h_1)(x)\leq\psi(g_2,h_2)(x)).$$
        There are two possibilities, either $x\in P$ or $x\in Q$. If $x\in P$ then $\psi(g_i,h_i)(x)=g_i(x)$ and we have $g_1\leq g_2$. The other case is similar.
    \end{itemize}
    We conclude that $R^{P+Q}\isom R^P\x R^Q$ as posets.
    \item In the same fashion we begin by considering functions
    \begin{align*}
        &\chi: (R^Q)^P\to R^{Q\x P},\ f\mapsto [f(-)](-),\word{and}\\
        &\la: R^{Q\x P}\to (R^Q)^P,\ g\mapsto g(-,p).
    \end{align*}
    The action of the function $[f(-)](-)$ is $(q,p)\mapsto [f(p)](q)\in R$ and the action of $g(-,p)$ is $q\mapsto g(q,p)$. It is important to note that $f(p)$ is an element of $R^Q$, which means that $f(p)$ is a function which takes inputs of $Q$ and outputs elements of $R$\footnote{The best analogy I could come up when discussing with \textbf{Sarah} was to use a matrix with variables. We can evaluate the matrix at any point, that is the $f(p)$. But the matrix itself is a linear transformation acting on vectors. That is the $f(p)(q)$.}. Also $\la$ is independent of the choice of $p$ as long as $p\in P$,this is because $\la$ just evaluates the second entry of $g$ first.\par 
    As before, consider the composition of these maps:
    $$\la(\chi(f))=[f(p)](-).$$
    This is the same function $f$ just that we have evaluated the $p$ first so that we obtain a function $f(p)$ from $Q$ to $R$.\par 
    On the other hand 
    $$\chi(\la(g))=[g(-,p)](-)=g(-,p)$$
    where $[g(-,p)](q)=g(q,p)$. As $p$ is arbitrary, it holds that $\chi\circ\la=\id_{R^{Q\x P}}$. It only remains to prove that $\chi$ and $\la$ are order-preserving.
    \begin{itemize}
        \itemsep=-0.4em
        \item Suppose $f_1\leq f_2$ in $(R^Q)^P$. Then for all $p$, $f_1(p)\leq f_2(p)$. As this are functions from $Q$ to $R$, then for all $q$: $f_1(p)(q)\leq f_2(p)(q)$. This is $\chi(f_1)\leq\chi(f_2)$.
        \item Now take $g_1\leq g_2$ in $R^{Q\x P}$. Then for all $(q,p)$ $g_1(q,p)\leq g_2(q,p)$, in particular for all $p$: $g_1(-,p)\leq g_2(-,p)$ as functions. This means that $\la(g_1)\leq\la(g_2)$.
    \end{itemize}
\end{enumerate} 
\end{ptcbr}

\begin{Ej}[Exercise 9]
    \emph{True or false:} if every chain and every antichain of a poset $P$ is finite, then $P$ is finite. \hint{The minimal elements of any poset form an
    antichain, as do the elements that cover any given element}
\end{Ej}

Ah! I have no words to express the joy of working on this problem. Truly thank you for it, the statement is true and we will prove it using the infinite Ramsey Theorem:

\begin{Th}
Suppose $X$ is infinite, then for a coloring $c:\binom{X}{n}\to [r]$, there exists $Y\subseteq X$ such that $\binom{Y}{n}$ is monochromatic.
\end{Th}

I hope you will allow me to reference a proof for this statement instead of writing it down here. One proof can be found in Bollobás' Modern Graph Theory in page 186, Theorem 4.\par 
Let us now proceed with proving the statement.

\begin{ptcbr}
    The statement of the problem is equivalent to
    \begin{significant}
        \small If $P$ is an infinite poset, then $P$ contains an infinite chain or an infinite antichain.
    \end{significant}
    Consider a coloring $\binom{P}{2}$, color $(x,y)$ red if $x$ and $y$ are comparable and blue if they aren't. By the infinite Ramsey Theorem, there exists $Y\subseteq X$ such that all the pairs of elements in $Y$ are either all red or all blue. This means that all elements in $Y$ are comparable or incomparable. In the first case we have an infinite chain and in the latter, an infinite antichain.\par 
    Indeed having a set of incomparable elements gives us an antichain, but it might not be obvious how having comparable elements gives us a chain. First, as all elements are comparable, we have a total order on the set. And given that total order we can \emph{linearly} order the elements in order to form the desired chain. 
\end{ptcbr}
\iffalse
\begin{ptcbr}
    Proving this statement is equivalent to proving the following
    \begin{significant}
        \small If $P$ is an infinite poset, then $P$ contains an infinite chain or an infinite antichain.
    \end{significant}
    For that effect, suppose $P$ is an infinite poset and consider the set of minimal elements $\cM$ of $P$. As $P$ is infinite, then either $\cM$ is infinite or it isn't. If it is, we are done. $\cM$ is the desired infinite antichain.\par 
    On the other hand $\cM$ is finite, but as $P$ is infinite, there must exist an element $m$ such that the principal filter (the upper-order ideal) at $m$ is infinite. Consider the cover of $m$, these elements also form an antichain. If they weren't then two elements would be comparable \red{CHECK DETAIL} then one of them wouldn't cover $m$.\par 
    Call $\cC(m)$ the cover of $m$, then either $\cC(m)$ is infinite or it isn't. If it is, we have found the desired antichain. If it's not, we can iterate this process finding an element $m'\in\cC(m)$ such that the principal filter at $m'$ is infinite. Then there are two possibilities
    \begin{itemize}
        \itemsep=-0.4em
        \item Either we find an infinite antichain as a cover of the \emph{next} minimal element at some step.
        \item Or we don't, but in that sense we continue finding an $m_{i+1}\in\cC(m_i)$ such that $m_{i+1}\geq m_i\geq\dots\geq m$ is a chain. The chain $(m_i)$ is an infinite chain as the antichain finding process never ends. 
    \end{itemize} 
    Thus we conclude that there must exist an infinite chain or an infinite antichain.
\end{ptcbr}
\fi
\end{document}
