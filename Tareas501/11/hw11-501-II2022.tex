\documentclass[12pt]{memoir}

\def\nsemestre {II}
\def\nterm {Fall}
\def\nyear {2022}
\def\nprofesor {Maria Gillespie}
\def\nsigla {MATH501}
\def\nsiglahead {Combinatorics}
\def\nextra {HW11}
\def\nlang {ENG}
\input{../../headerVarillyDiff}
\DeclareMathOperator{\des}{des}
\DeclareMathOperator{\inv}{inv}
\DeclareMathOperator{\exc}{exc}
\DeclareMathOperator{\maj}{maj}
\usepackage{halloweenmath}

\begin{document}

\begin{Ej}[Exercise 1 and 2]
    Let $P$ be a finite poset and $L=\cJ(P)$ be the corresponding distributive lattice. If $X\subseteq P$ is a lower-order ideal, then use the corresponding lowercase letter $x$ to denote the associated element of $L$. 
    \begin{enumerate}[i)]
        \itemsep=-0.4em
        \item Show that $x$ covers $y$ in $L$ if and only if $Y=X\less\set{m}$, where $m$ is a maximal element of $X$.
        \item Show that $x$ is join-irreducible in $L$ if and only if $X$ is a principal ideal of $P$.
    \end{enumerate}
 \end{Ej}
\iffalse
 We will use the following lemma which is Proposition 5.1.2 in Sagan. 

 \begin{Lem}
If $P$ is finite and $I$ is an ideal, then $I=\gen(S)$ where $S$ is the set of maximal elements in $I$. 
 \end{Lem}

\begin{ptcbp}
If $x\in I$, consider $X\set{y\in I:\ y\geq x}$, the set of upper bounds of $x$. This set is non-empty as $x\geq x$ and there must exist a maximal element of $X$ since $P$ is finite. Call $s\in X$ such a maximal element. If $s$ is not maximal in $I$, then $\exists \tilde{s}\in I(\tilde{s}\geq s)$ and this means that $\tilde{s}\geq x$. This contradicts the maximality of $s$ in $X$.\par 
This proves that $x\in\gen(S)$ as $s$ is a maximal element of $I$.\par 
On the other hand if $x\in\gen(S)$ then, $x\leq s$ for some $s\in S\subseteq I$. As $I$ is an order ideal and $s\in I$, then $x\in I$. 
\end{ptcbp}
\fi
\begin{ptcbr}
    \begin{enumerate}[i)]
        \itemsep=-0.4em
        \item Let us begin by assuming that $x$ covers $y$ in $L$ this means that $x\supseteq y$ and $\not\exists z(x>z>y)$. In terms of elements, this means that $|x\less y|=1$. Take $n$ to be the element we removed from $x$, if $n$ is not maximal, then $y$ is no longer an order ideal as $n\leq s$ where $s$ is a maximal element in $y$.\par 
        On the other hand $X\less\set{m}\subseteq X$, so we only have to prove that $X\less\set{m}$ is an ideal. \red{FINISH}
        \item 
    \end{enumerate}
\end{ptcbr}
\end{document}
