\documentclass[12pt]{memoir}

\def\nsemestre {II}
\def\nterm {Fall}
\def\nyear {2022}
\def\nprofesor {Maria Gillespie}
\def\nsigla {MATH501}
\def\nsiglahead {Combinatorics}
\def\nextra {HW14}
\def\nlang {ENG}
\input{../../headerVarillyDiff}
\DeclareMathOperator{\inv}{inv}
\begin{document}

\begin{Ej}[Exercise 1, Bonus]
What is your favorite bijection, and why? (This problem does not count towards your 10 point total problem limit; it will be an extra fun point added on top to whatever you hand in for real below.)
\end{Ej}

\begin{ptcbr}
I have a whole set of bijections which I may call my favorite. Consider $(f_d)$ where $d\in\ttt{DATES}$, the set \ttt{DATES} consists of all the days where I went to class. For each $d$, we have 
\begin{align*}
f_d:\set{\ttt{students in 502 class on day }d}&\to\set{\text{seats in room E204}},\\
\ttt{student}&\mapsto\text{preferred seat on day }d.    
\end{align*}
The function $f_d$ is a bijection when restricted to its image because no two \ttt{grad students} can sit in the same chair.\par
In particular, amongst the $f_d$'s, the ones I'm most partial to, are the ones in $\ttt{DATES}'$ where this is the subset of days in which I did not fall asleep in class.\par 
These are my favorite bijections because they are unique and different. I'm almost sure\footnote{the set where I'm not sure has measure zero.} no one in class has a bijection like this as their favorite. Also, the idea of writing this made me chuckle, it's not everyday that you can have fun like this.\footnote{\textbf{Everyone} in class collaborated with me on this problem. If it weren't for them, I wouldn't have been able to do this problem quite literally.} 
\end{ptcbr}

\begin{Ej}[Exercise 2]
Apply Franklin's involution to $(7,6,4,3)$ and check that applying it again returns the original partition.
\end{Ej}

\begin{ptcbr}
    
\begin{center}
    \tikzset{every picture/.style={line width=0.75pt}} %set default line width to 0.75pt        

\begin{tikzpicture}[x=0.75pt,y=0.75pt,yscale=-1,xscale=1]
%uncomment if require: \path (0,300); %set diagram left start at 0, and has height of 300

%Shape: Square [id:dp2529280990567253] 
\draw   (200,200) -- (205,200) -- (205,205) -- (200,205) -- cycle ;
%Shape: Square [id:dp37427252398278177] 
\draw   (195,200) -- (200,200) -- (200,205) -- (195,205) -- cycle ;
%Shape: Square [id:dp01964466980827817] 
\draw   (190,200) -- (195,200) -- (195,205) -- (190,205) -- cycle ;
%Shape: Square [id:dp12237354903136533] 
\draw   (185,200) -- (190,200) -- (190,205) -- (185,205) -- cycle ;
%Shape: Square [id:dp32297381585497553] 
\draw   (180,200) -- (185,200) -- (185,205) -- (180,205) -- cycle ;
%Shape: Square [id:dp38530822774551465] 
\draw   (175,200) -- (180,200) -- (180,205) -- (175,205) -- cycle ;
%Shape: Square [id:dp14536512886724084] 
\draw   (170,200) -- (175,200) -- (175,205) -- (170,205) -- cycle ;
%Shape: Square [id:dp025506581215240187] 
\draw   (195,195) -- (200,195) -- (200,200) -- (195,200) -- cycle ;
%Shape: Square [id:dp5212039933003418] 
\draw   (190,195) -- (195,195) -- (195,200) -- (190,200) -- cycle ;
%Shape: Square [id:dp796826429947336] 
\draw   (185,195) -- (190,195) -- (190,200) -- (185,200) -- cycle ;
%Shape: Square [id:dp08162216583941317] 
\draw   (180,195) -- (185,195) -- (185,200) -- (180,200) -- cycle ;
%Shape: Square [id:dp6945093194132337] 
\draw   (175,195) -- (180,195) -- (180,200) -- (175,200) -- cycle ;
%Shape: Square [id:dp6801096253140368] 
\draw   (170,195) -- (175,195) -- (175,200) -- (170,200) -- cycle ;
%Shape: Square [id:dp23092166677901793] 
\draw   (185,190) -- (190,190) -- (190,195) -- (185,195) -- cycle ;
%Shape: Square [id:dp5949608168505787] 
\draw   (180,190) -- (185,190) -- (185,195) -- (180,195) -- cycle ;
%Shape: Square [id:dp9596347021729035] 
\draw   (175,190) -- (180,190) -- (180,195) -- (175,195) -- cycle ;
%Shape: Square [id:dp028730561372985353] 
\draw   (170,190) -- (175,190) -- (175,195) -- (170,195) -- cycle ;
%Shape: Square [id:dp9177511094583908] 
\draw   (180,185) -- (185,185) -- (185,190) -- (180,190) -- cycle ;
%Shape: Square [id:dp5435578712504716] 
\draw   (175,185) -- (180,185) -- (180,190) -- (175,190) -- cycle ;
%Shape: Square [id:dp9780162105022518] 
\draw   (170,185) -- (175,185) -- (175,190) -- (170,190) -- cycle ;
%Straight Lines [id:da5503062620320955] 
\draw    (210,195) -- (243,195) ;
\draw [shift={(245,195)}, rotate = 180] [color={rgb, 255:red, 0; green, 0; blue, 0 }  ][line width=0.75]    (10.93,-3.29) .. controls (6.95,-1.4) and (3.31,-0.3) .. (0,0) .. controls (3.31,0.3) and (6.95,1.4) .. (10.93,3.29)   ;
%Shape: Square [id:dp05236524770784046] 
\draw  [dash pattern={on 0.84pt off 2.51pt}] (280,200) -- (285,200) -- (285,205) -- (280,205) -- cycle ;
%Shape: Square [id:dp27910670153823713] 
\draw   (275,200) -- (280,200) -- (280,205) -- (275,205) -- cycle ;
%Shape: Square [id:dp8852142880369072] 
\draw   (270,200) -- (275,200) -- (275,205) -- (270,205) -- cycle ;
%Shape: Square [id:dp8605932563685457] 
\draw   (265,200) -- (270,200) -- (270,205) -- (265,205) -- cycle ;
%Shape: Square [id:dp7182620210563464] 
\draw   (260,200) -- (265,200) -- (265,205) -- (260,205) -- cycle ;
%Shape: Square [id:dp6522809031085295] 
\draw   (255,200) -- (260,200) -- (260,205) -- (255,205) -- cycle ;
%Shape: Square [id:dp7255816052863471] 
\draw   (250,200) -- (255,200) -- (255,205) -- (250,205) -- cycle ;
%Shape: Square [id:dp26506249449354846] 
\draw  [dash pattern={on 0.84pt off 2.51pt}][line width=0.75]  (275,195) -- (280,195) -- (280,200) -- (275,200) -- cycle ;
%Shape: Square [id:dp6707896375377687] 
\draw   (270,195) -- (275,195) -- (275,200) -- (270,200) -- cycle ;
%Shape: Square [id:dp9880846834577579] 
\draw   (265,195) -- (270,195) -- (270,200) -- (265,200) -- cycle ;
%Shape: Square [id:dp09009817259226027] 
\draw   (260,195) -- (265,195) -- (265,200) -- (260,200) -- cycle ;
%Shape: Square [id:dp9023716509729187] 
\draw   (255,195) -- (260,195) -- (260,200) -- (255,200) -- cycle ;
%Shape: Square [id:dp5372965902156774] 
\draw   (250,195) -- (255,195) -- (255,200) -- (250,200) -- cycle ;
%Shape: Square [id:dp6285416848945435] 
\draw   (265,190) -- (270,190) -- (270,195) -- (265,195) -- cycle ;
%Shape: Square [id:dp9505737844092881] 
\draw   (260,190) -- (265,190) -- (265,195) -- (260,195) -- cycle ;
%Shape: Square [id:dp7657422533799325] 
\draw   (255,190) -- (260,190) -- (260,195) -- (255,195) -- cycle ;
%Shape: Square [id:dp6380211804246454] 
\draw   (250,190) -- (255,190) -- (255,195) -- (250,195) -- cycle ;
%Shape: Square [id:dp6576919327721391] 
\draw   (260,185) -- (265,185) -- (265,190) -- (260,190) -- cycle ;
%Shape: Square [id:dp3354293549592997] 
\draw   (255,185) -- (260,185) -- (260,190) -- (255,190) -- cycle ;
%Shape: Square [id:dp1286346751412586] 
\draw   (250,185) -- (255,185) -- (255,190) -- (250,190) -- cycle ;
%Curve Lines [id:da8585209952669695] 
\draw    (285,195) .. controls (294.8,166.38) and (284.62,155.82) .. (266.14,173.86) ;
\draw [shift={(265,175)}, rotate = 314.41] [color={rgb, 255:red, 0; green, 0; blue, 0 }  ][line width=0.75]    (10.93,-3.29) .. controls (6.95,-1.4) and (3.31,-0.3) .. (0,0) .. controls (3.31,0.3) and (6.95,1.4) .. (10.93,3.29)   ;
%Shape: Square [id:dp1256791046628849] 
\draw   (255,180) -- (260,180) -- (260,185) -- (255,185) -- cycle ;
%Shape: Square [id:dp4708965383648591] 
\draw   (250,180) -- (255,180) -- (255,185) -- (250,185) -- cycle ;
%Straight Lines [id:da7741501846787047] 
\draw    (290,195.04) -- (323,195.04) ;
\draw [shift={(325,195.04)}, rotate = 180] [color={rgb, 255:red, 0; green, 0; blue, 0 }  ][line width=0.75]    (10.93,-3.29) .. controls (6.95,-1.4) and (3.31,-0.3) .. (0,0) .. controls (3.31,0.3) and (6.95,1.4) .. (10.93,3.29)   ;
%Shape: Square [id:dp990540540897183] 
\draw   (360,200.04) -- (365,200.04) -- (365,205.04) -- (360,205.04) -- cycle ;
%Shape: Square [id:dp28629642116721676] 
\draw   (355,200.04) -- (360,200.04) -- (360,205.04) -- (355,205.04) -- cycle ;
%Shape: Square [id:dp8747458957107797] 
\draw   (350,200.04) -- (355,200.04) -- (355,205.04) -- (350,205.04) -- cycle ;
%Shape: Square [id:dp6844710379897743] 
\draw   (345,200.04) -- (350,200.04) -- (350,205.04) -- (345,205.04) -- cycle ;
%Shape: Square [id:dp6779475050787578] 
\draw   (340,200.04) -- (345,200.04) -- (345,205.04) -- (340,205.04) -- cycle ;
%Shape: Square [id:dp05169893034449036] 
\draw   (335,200.04) -- (340,200.04) -- (340,205.04) -- (335,205.04) -- cycle ;
%Shape: Square [id:dp1526067888494178] 
\draw   (330,200.04) -- (335,200.04) -- (335,205.04) -- (330,205.04) -- cycle ;
%Shape: Square [id:dp20491374952000418] 
\draw  [line width=0.75]  (355,195.04) -- (360,195.04) -- (360,200.04) -- (355,200.04) -- cycle ;
%Shape: Square [id:dp042419564779356156] 
\draw   (350,195.04) -- (355,195.04) -- (355,200.04) -- (350,200.04) -- cycle ;
%Shape: Square [id:dp7965667197766826] 
\draw   (345,195.04) -- (350,195.04) -- (350,200.04) -- (345,200.04) -- cycle ;
%Shape: Square [id:dp8554446826633839] 
\draw   (340,195.04) -- (345,195.04) -- (345,200.04) -- (340,200.04) -- cycle ;
%Shape: Square [id:dp7149669705499673] 
\draw   (335,195.04) -- (340,195.04) -- (340,200.04) -- (335,200.04) -- cycle ;
%Shape: Square [id:dp5754448169137754] 
\draw   (330,195.04) -- (335,195.04) -- (335,200.04) -- (330,200.04) -- cycle ;
%Shape: Square [id:dp9064150808017724] 
\draw   (345,190.04) -- (350,190.04) -- (350,195.04) -- (345,195.04) -- cycle ;
%Shape: Square [id:dp4578176803938725] 
\draw   (340,190.04) -- (345,190.04) -- (345,195.04) -- (340,195.04) -- cycle ;
%Shape: Square [id:dp43618130058262405] 
\draw   (335,190.04) -- (340,190.04) -- (340,195.04) -- (335,195.04) -- cycle ;
%Shape: Square [id:dp08716130852213189] 
\draw   (330,190.04) -- (335,190.04) -- (335,195.04) -- (330,195.04) -- cycle ;
%Shape: Square [id:dp17195860575317323] 
\draw   (340,185.04) -- (345,185.04) -- (345,190.04) -- (340,190.04) -- cycle ;
%Shape: Square [id:dp6792579403673551] 
\draw   (335,185.04) -- (340,185.04) -- (340,190.04) -- (335,190.04) -- cycle ;
%Shape: Square [id:dp6550543486796916] 
\draw   (330,185.04) -- (335,185.04) -- (335,190.04) -- (330,190.04) -- cycle ;
%Curve Lines [id:da79352721640113] 
\draw    (345,175) .. controls (375.18,165.59) and (383.86,175.01) .. (366.12,193.84) ;
\draw [shift={(365,195)}, rotate = 314.41] [color={rgb, 255:red, 0; green, 0; blue, 0 }  ][line width=0.75]    (10.93,-3.29) .. controls (6.95,-1.4) and (3.31,-0.3) .. (0,0) .. controls (3.31,0.3) and (6.95,1.4) .. (10.93,3.29)   ;
%Shape: Square [id:dp5885708305169732] 
\draw  [dash pattern={on 0.84pt off 2.51pt}] (335,180.04) -- (340,180.04) -- (340,185.04) -- (335,185.04) -- cycle ;
%Shape: Square [id:dp11105600087840672] 
\draw  [dash pattern={on 0.84pt off 2.51pt}] (330,180.04) -- (335,180.04) -- (335,185.04) -- (330,185.04) -- cycle ;

% Text Node
\draw (167,208.4) node [anchor=north west][inner sep=0.75pt]  [font=\tiny]  {$( 7,6,4,3)$};
% Text Node
\draw (247,208.4) node [anchor=north west][inner sep=0.75pt]  [font=\tiny]  {$( 6,5,4,3,2)$};
% Text Node
\draw (327,208.44) node [anchor=north west][inner sep=0.75pt]  [font=\tiny]  {$( 7,6,4,3)$};


\end{tikzpicture}

\end{center}

For the partition $(7,6,4,3)$ the smallest part is $r=3$ and we have 
$$7=1+6\neq 2+4$$
so in this case $s=2$. As $s<r$ we move $2$ boxes from the staircase to the new top row. This gives us $(6,5,4,3,2)$. Now the smallest part is $r=2$ and we have that 
$$6=1+5=2+4=3+3=4+2\neq 5+0$$
so in this case $s=5$. Therefore we move the $r$'s row and place each of the blocks on the first $r$ rows. This gives us $(7,6,4,3)$ back.

\end{ptcbr}

\begin{Ej}[Exercise 3, Sagan 3.16(c)]
Show that the generating function for the number of partitions of $n$ with $\la_1=k$ equals the generating function for the number of partitions of $n$ with exactly $k$ parts and that both are equal to the product $x^k/[(1-x)\dots(1-x^k)]$.
\end{Ej}

\begin{ptcbr}
Let us call $p(n,k)$ the number of partitions of $n$ with exactly $k$ parts and $\wh{p}(n,k)$, the number of partitions of $n$ with $\la_1=k$. To show that their generating functions are equal, it suffices to prove that $p=\wh{p}$ for all $n,k$.\par
To that effect, take $\la\vdash n$ with $k$ parts. When conjugating $\la$ we see that $\la^\ast$ now has largest part equal to $k$. In terms of their Young tableaux, $\la$ has dimensions $k\x\la_1$ while $\la^\ast$ has dimensions $\la_1\x k$.\par 
Conjugation of partitions is a bijective map, which means that every partition with $k$ parts is uniquely associated to a partition with largest part $k$ and back. In conclusion it must hold that $p=\wh{p}$ and therefore the generating functions are equal.\par
To see the product formula let us consider instead the number of partitions with largest part \emph{at most} $k$. The generating product function for that sequence is 
\begin{align*}
    &\left((1+x+x^2+\dots)((x^2)^0+(x^2)^1+(x^2)^2+\dots)\dots((x^k)^0+(x^k)^1+(x^k)^2+\dots)\right)\\
    =&\frac{1}{(1-x)(1-x^2)\dots(1-x^k)}.
\end{align*}
The last infinite sum counts the number parts in $\la$ equal to $k$. If we wish to guarantee that there is one part equal to $k$ we must multiply that factor by $x^k$. Therefore we have that the product which counts $\wh{p}$ is 
\begin{align*}
    &\left((1+x+x^2+\dots)((x^2)^0+(x^2)^1+(x^2)^2+\dots)\dots((x^k)^1+(x^k)^2+(x^k)^3+\dots)\right)\\
    =&\frac{x^k}{(1-x)(1-x^2)\dots(1-x^k)}.
\end{align*}
And since $p=\wh{p}$ it holds that both functions are equal to the product.
\end{ptcbr}

\begin{Ej}[Exercise 3, Sagan 3.16(d)]
    The Durfee square of $\la$ is the largest square partition $(d^d)$ such that $(d^d)\subseteq \la$. Use this concept to prove
    $$\sum_{n=0}^\infty p(n)x^n=\sum_{d=0}^\infty \frac{x^{d^2}}{(1-x)^2(1-x^2)^2\dots(1-x^d)^2}.$$
\end{Ej}

\begin{ptcbr}
What the problem is telling us is to count the total number of partitions in a different way than the one we know.\par 
For that effect consider a partition $\la$ with its corresponding Young diagram. Such a partition can be decomposed into a Durfee square, a \emph{right} component and an \emph{up} component. Still viewing such components as Young diagrams, we can see a characteristic which defines them:
\vspace*{-0.4em}
\begin{itemize}
    \itemsep=-0.4em
    \item The \emph{right} component is a partition of $n-d$ with at most $d$ parts.
    \item The \emph{up} component is a partition (of some number) with largest part at most $d$. 
\end{itemize} 
By the previous problem we know that these types of partitions are in correspondence and each is counted by a product generating function. So for all sizes of Durfee squares we can count all partitions these way: fix a Durfee square size, then attach the \emph{right} and \emph{up} components to the Young diagram. Since each partition necessarily contains a Durfee square, we add a factor of $x^{d^2}$ while each of the two components are each counted by $1/((1-x)\dots(1-x^d))$. Multiplying these together gives us 
the desired identity 
$$\sum_{n=0}^\infty p(n)x^n=\sum_{d=0}^\infty \frac{x^{d^2}}{(1-x)^2(1-x^2)^2\dots(1-x^d)^2}.$$
\end{ptcbr}
\newpage
\begin{Ej}[Exercise 3, Sagan 3.17]
    Let $a_n$ be the number of partitions of $n$ such that any part $i$
 is repeated fewer than $i$ times and let $b_n$ be the number of partitions such that no part is a square. Using generating functions, show that $a_n=b_n$.
\end{Ej}

\begin{ptcbr}
    We can expand the product generating function for $a_n$ in the following way 
    \begin{align*}
        a_n\xrightarrow[]{ogf}&(1)((x^2)^0+(x^2)^1)((x^3)^0+(x^3)^1+(x^3)^2)((x^4)^0+(x^4)^1+(x^4)^2+(x^4)^3)\dots\\
        &=\prod_{i=1}^\infty\sum_{j=0}^{i-1}x^{ij}.
    \end{align*}
    The sum inside the product is a geometric sum equal to $\frac{1-x^{i^2}}{1-x^i}$. So it holds that 
    $$\displaystyle a_n\xrightarrow[]{ogf}\prod_{i=1}^\infty\frac{1-x^{i^2}}{1-x^i}=\frac{\displaystyle \prod_{i=1}^\infty\frac{1}{1-x^i}}{\displaystyle \prod_{i=1}^\infty\frac{1}{1-x^{i^2}}}.$$
    The last product is the generating function which counts all partitions divided by the function which counts partitions of square numbers. In total, the quotient counts the number of partitions of numbers which are not squares. Thus 
    $$b_n\xrightarrow[]{ogf}\frac{\displaystyle \prod_{i=1}^\infty\frac{1}{1-x^i}}{\displaystyle \prod_{i=1}^\infty\frac{1}{1-x^{i^2}}}$$ 
    and as $a_n$ and $b_n$ have the same generating function, it must hold that $a_n=b_n$.
\end{ptcbr}
\newpage
\begin{Ej}[Exercise 3, Sagan 3.18]
    If $m\geq 2$, use generating functions to prove that the number of partitions where each part is repeated fewer than $m$ times equals the number of partitions of $n$ into parts not divisible by $m$. 
\end{Ej}

\begin{ptcbr}
    Call $a_n$ the number of partitions of $n$ where each part is repeated fewer than $m$ times and $b_n$ the number of partitions of $n$ into parts not divisible by $m$. By a similar reasoning to the last problem we have that 
  \begin{align*}
    a_n\xrightarrow[]{ogf}&((x^1)^0+\dots+(x^1)^{m-1})((x^2)^0+\dots+(x^2)^{m-1})((x^3)^0+\dots+(x^3)^{m-1})\dots\\
    &=\prod_{i=1}^\infty\sum_{j=0}^{m-1}x^{ij}=\prod_{i=1}^\infty\frac{1-x^{im}}{1-x^i}.
  \end{align*}
  This last product can be seen to be equal to 
  $$\frac{\displaystyle \prod_{i=1}^\infty\frac{1}{1-x^i}}{\displaystyle \prod_{i=1}^\infty\frac{1}{1-x^{im}}}$$
  where the product on the top counts the number of total partitions and the one on the bottom counts partitions of numbers whose parts are multiples of $m$. Removing such factors from the top leaves us with the partitions of numbers whose parts are not divisible by $m$. This means that 
  $$b_n\xrightarrow[]{ogf}\frac{\displaystyle \prod_{i=1}^\infty\frac{1}{1-x^i}}{\displaystyle \prod_{i=1}^\infty\frac{1}{1-x^{im}}}\To a_n=b_n.$$
\end{ptcbr}
\end{document}
