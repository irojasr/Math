\documentclass[12pt]{memoir}

\def\nsemestre {II}
\def\nterm {Fall}
\def\nyear {2022}
\def\nprofesor {Maria Gillespie}
\def\nsigla {MATH501}
\def\nsiglahead {Combinatorics}
\def\nextra {HW8}
\def\nlang {ENG}
\input{../../headerVarillyDiff}
\DeclareMathOperator{\des}{des}
\DeclareMathOperator{\inv}{inv}
\DeclareMathOperator{\exc}{exc}
\DeclareMathOperator{\maj}{maj}
\usepackage{halloweenmath}

\begin{document}

\begin{Ej}
    A \emph{plane tree} is a tree along with, for every vertex $v$, a cyclic ordering of the nodes $w$ connected to $v$ by
an edge. Use species to find a formula for the number of plane trees on $n$ vertices, using the following
steps:
\begin{enumerate}[i)]
    \itemsep=-0.4em 
    \item Build your intuition by directly counting the plane trees on n vertices for $n =
    0, 1, 2, 3, 4, 5$. Write out the corresponding first handful of terms for the exponential generating function.
    \item Let $\cP\cL$ denote the species of plane trees. Let $\cR$ be the species of rooted plane trees, where we choose one of the $n$ vertices to be the root. Relate $\cR$ and $\cP\cL$ with a species identity.
\end{enumerate}
\end{Ej}

\begin{ptcbr}
    \begin{enumerate}[i)]
        \itemsep=-0.4em
        \item \footnote{From counting wrong to not understanding species, \textbf{Kelsey}, \textbf{Kaylee}, \textbf{Sarah} and \textbf{Sam} helped me out with this problem.}From zero to two we have only one option. On 3 vertices we have 3 ways to label our vertices.\par 
        Four vertices give us 2 different trees. The first one $P_4$ can be labeled in 12 ways by choosing the root in 4 ways and 3 ways for a neighbor. While $K_{1,3}$ we have $4$ ways to pick the root and $2$ more to pick the neighbors. In total for four vertices we have $20$ ways.\par 
        Finally on $5$ vertices we have $3$ unlabeled trees.
        \begin{itemize}
            \itemsep=-0.4em
            \item On $K_{1,4}$ there's $5$ ways to label a root and $4!/4$ ways to order the neighbors. 
            \item In $P_5$ we have $5$ ways to choose the root again and $4!/2$ ways to order the neighbors. 
            \item Finally we have a \emph{chair} graph. There's no symmetry in this graph so in total we have $5!$ ways to choose the root vertex $v$ and its neighbors.
        \end{itemize}
        In array function notation we have:
        $$\begin{bmatrix}
            0&1&2&3&4&5\\
            1&1&1&3&20&210
        \end{bmatrix}$$
        \item We claim that the formula is $X(\cP\cL)'=\cR$. This is because when we differentiate we take one of the vertices to be a starred vertex outside our labels. With $X$ we are labeling it as the root.
    \end{enumerate}
\end{ptcbr}

\begin{Ej}[Exercise 6]
  Each of $n$ (distinguishable) telephone poles is painted red, white, blue, or
  yellow. An odd number are painted blue and an even number are painted yellow. In how many ways can this be done?
\end{Ej}

\begin{ptcbr}
Let us consider a variable $z$ whose exponent will count the number of telephone poles painted in any color. Since we have no restrictions on our red and white telephone poles, the exponential generating function which encodes the number of ways to paint the poles in ONE of those colors is 
$$1+z+\frac{z^2}{2}+\frac{z^3}{3!}+\dots=e^z$$
because there is only one way to paint in one color all of the poles. The function which counts the number of ways to paint in either red or white is 
$$(e^z)(e^z)=e^{2z}=1+2z+2^2\frac{z^2}{2}+2^3\frac{z^3}{3!}+\dots$$
In general the coefficients count the number of functions $f$ 
$$f:\set{\text{pole}_1,\text{pole}_2,\dots,\text{pole}_n}\to\set{\text{red},\text{white}}.$$
Both sets are distinguishable so the number is $2^n$.\par 
If we are painting only an odd number of poles in blue, then the exponential generating function looks like 
$$0+z+0\frac{z^2}{2}+\frac{z^3}{3!}+\dots=\sinh(z),$$
because if we are only painting poles in color blue, then there's no way to paint $2$ poles with only color blue since we need to paint an odd number of poles blue. By a similar reasoning we arrive at the fact the e.g.f. which encodes yellow colored telephone poles is $\cosh(z)$.\par 
The function which encodes painting in any of these colors is 
\begin{align*}
    e^z\cosh(z)\sinh(z)&=e^z\left(\frac{e^z+e^{-z}}{2}\right)\left(\frac{e^z-e^{-z}}{2}\right),\\
    &=\frac{1}{4}e^z(e^{2z}-e^{-2z}),\\
    &=\frac{1}{4}(e^{3z}-e^{-z}).
\end{align*}
The value $t_n$ which counts the way to paint given the conditions is the coefficient which accompanies $z^n/n!$ in the expansion of $\frac{1}{4}(e^{3z}-e^{-z})$. We can find that value as follows:
$$\frac{1}{4}(e^{3z}-e^{-z})=\sum_{n=0}^\infty\frac{3^n}{4}\frac{z^n}{n!}-\sum_{n=0}^\infty\frac{(-1)^n}{4}\frac{z^n}{n!}=\sum_{n=0}^\infty\frac{3^n-(-1)^n}{4}\frac{z^n}{n!}.$$
Also, $3\equiv -1\bmod 4$ so $t_n$ is always an integer. We conclude that $t_n=\frac{3^n-(-1)^n}{4}$. 
\end{ptcbr}

\begin{Ej}[Exercise 7]
    Suppose now the colors orange and purple are also used. The number of
orange poles plus the number of purple poles is even. Now how many ways are there?
\end{Ej}

\begin{ptcbr}
    Consider the now the number of ways to paint an even number of poles with ONLY purple and orange. That is equivalent to partition a set of $2n$ poles into two and that amount is $S(2n,2)$. This means the function which encodes the number of ways to paint in this way is 
    $$\sum_{n=0}^\infty S(2n,2)\frac{z^{2n}}{(2n)!}=\sum_{n=0}^\infty (2^{2n-1}-1)\frac{z^{2n}}{(2n)!}=\half\sum_{n=0}^\infty\frac{(2z)^{2n}}{(2n)!}-\sum_{n=0}^\infty \frac{z^{2n}}{(2n)!}.$$
    We recognize this as the function $\half\cosh(2z)-\cosh(z)$. Let us apply some trigonometric identities to have this functions in terms of exponentials:
    \begin{align*}
        \half\cosh(2z)-\cosh(z)&=\half\left(\frac{e^{2z}+e^{-2z}}{2}\right)-\left(\frac{e^{z}+e^{-z}}{2}\right)\\
        &=\frac{1}{4}(e^{2z}+e^{-2z}-2e^z-2e^{-z}).
    \end{align*}
    Then, the generating function which encodes the total number of possibilities is 
    \begin{align*}
    &\frac{1}{4}(e^{3z}-e^{-z})\.\frac{1}{4}(e^{2z}+e^{-2z}-2e^z-2e^{-z})\\
    =&\frac{1}{16}(e^{5z}+e^z-2e^{4z}-2e^{2z}-e^z-e^{-3z}+2+2e^{-2z})\\
    =&\frac{1}{16}(e^{5z}-2e^{4z}-2e^{2z}+2+2e^{-2z}-e^{-3z})\\
    =&\frac{1}{16}\sum_{n=0}^\infty(5^n-2(4^n)-2(2^n)+2\dl_{n0}+2(-2)^n-(-3)^n)\frac{z^n}{n!}
    \end{align*}
    We conclude that the number of painting in this way is 
    $$t_n=\frac{1}{16}(5^n-2(4^n)-2(2^n)+2\dl_{n0}+2(-2)^n-(-3)^n).\footnote{This quantity is not always an integer. When the number of poles is odd, this element lives in $\half\bZ$. I've re-done the process and haven't been able to spot out my mistake.}$$
\end{ptcbr}

\end{document}
