\documentclass[12pt]{memoir}

\def\nsemestre {II}
\def\nterm {Fall}
\def\nyear {2022}
\def\nprofesor {Maria Gillespie}
\def\nsigla {MATH501}
\def\nsiglahead {Combinatorics}
\def\nextra {HW6}
\def\nlang {ENG}
\input{../../headerVarillyDiff}
\DeclareMathOperator{\des}{des}
\DeclareMathOperator{\inv}{inv}
\DeclareMathOperator{\exc}{exc}
\DeclareMathOperator{\maj}{maj}
\usepackage{lipsum}
\begin{document}

\begin{Ej}[Exercise 3]
   Read the definition of alternating permutations in Section 1.4 of Stanley, and write out all 16 alternating permutations of size 5. Then, read Proposition 1.6.1 and its proof, and either (a)
comment on how awesome it is and what you learned from it, OR (b) point out something you didn't fully understand about the proof.
\end{Ej}

\begin{ptcbr}
An \un{alternating permutation} is a permutation that, when written in list notation, goes (\emph{down-up-down-up\dots}). On the other hand a \un{reverse-alternating permutatio} goes (\emph{up-down-up-down\dots}).\par 
As in Section 1.4, call $E_n$ the set of alternating permutations of order $n$, we can construct some elements in $E_5$ with elements of $E_4$. We have the following permutations in $S_4$ which are alternating and reverse-alternating
\begin{center}
   \begin{tabular}{l|l}
      \emph{down-up} & \emph{up-down} \\
      \hline
      2143&1324   \\
      3142&1423   \\
      3241&2314   \\
        4132&2413   \\
        4231&3412  
      \end{tabular}
\end{center}
For this permutations we can either add a 5 at the end of the \emph{down-up}'s or a 5 at the beginning of the \emph{up-down}'s. We get some \emph{down-up} permutations in $S_5$ this way:

$$\set{21435,31425,32415,41325,42315,51324,51423,52314,52413,53412}\subseteq E_5.$$

We can also include 5 in the middle of the \emph{down-up}'s and reverse the last two elements. This adds other permutations to our count:

$$\set{\un{21534},31524,32514,41523,42513}\subseteq E_5.$$

By doing the same process with the \emph{up-down} permutations, but instead reversing the order of the first two elements we get the permutations 

$$\set{31524,41523,32514,42513,\un{43512}}\subseteq E_5.$$

This process gives us only one different permutation from the last process, in fact, the reverse process also gives us a permutation which is not on this set. So in total we have

\begin{align*}
   E_5=&\set{21435,31425,32415,41325,42315}\cup\set{51324,51423,52314,52413,53412}\\
   &\cup\set{31524,32514,41523,42513}\cup\set{21534,43512}.
\end{align*}

These are the 16 alternating permutations in question.\par 
Regarding the proof, what surprises me the most is the way that the permutations are counted. I only used the permutations one step before to mine, but in the proof, all the other permutations of lower orders are used.\par 
On questions or doubts I still find the following troubling:

\begin{enumerate}[i)]
   \itemsep=-0.4em
   \item I need to wrap the recursion around my head, I admit that still can't grasp it fully. 
   \item Also, we get the factor of $2E_{n+1}$ since we are using both sets of \emph{down-up}'s and \emph{up-down}'s in the count, right?
\end{enumerate}
The differential equation is not a problem, \textbf{but afterwards...} The alternative proof after deriving the identity for secant, I don't understand how Stanley rewrites that equation as the one he names (1.56).\par 
I also need to review what \emph{descent set} $D$ means, because I know what $\des(\pi)$ means, but not exactly what the descent set.\par 
At the beginning of section 1.6 Stanley mentions that $\pi$ is \emph{alternating} or \emph{down-up} when $D(\pi)=\set{2n-1}_{n\geq 1}\cap[n-1]$. This equivalence makes intuitive sense for me, but I need to prove it.\footnote{I know I was only supposed to comment on the proposition, but why stop there? I must admit that afterwards when $f_k(n)$ is defined I got completely lost, I can't imagine a permutation with such descents.}
\end{ptcbr}

\begin{Ej}[Exercise 5]
   A \textbf{binary tree} of length $n$ constructed recursively as follows.\par 
   \vspace*{-0.6em}

   \begin{itemize}
      \itemsep=-0.4em
      \item The empty set is a binary tree of length 0.
      \item Otherwise a binary tree has a \emph{root vertex} $v$, a \emph{left subtree} $T_1$ and a \emph{right subtree} $T_2$, each of which is also a binary tree having a root vertex.\par 
      \vspace*{-0.4em}
      We draw the root vertex at the top with an edge going down to the root vertices of $T_1, T_2$. Then draw each tree recursively in the same manner.
   \end{itemize}
   \vspace*{-0.6em}

   Prove that the number of binary trees on $n$ vertices is the $n^{\text{th}}$ Catalan number $C_n$.\hint{Show that they satisfy the recursion for the Dyck paths}
   \end{Ej}
   
   \begin{ptcbr}
Let us call $f(n)$ the number of binary trees on $n$ vertices. The initial condition is $f(0)=1$ because the empty set is a binary tree.\par 
To create a binary tree with $n+1$ vertices we choose the root and then we still have $n$ vertices to go.\par 
Fix $\l$ to be the number of vertices we assign to the left tree then, the the remaining $n-\l$ vertices go to the right tree. The number of ways to build right and left subtrees this way is $f(\l)f(n-\l)$.\par 
However, running $\l$ through all possible options of $n$ gives us a plethora of disjoint events. We can sum those  possibilities to get the total number of binary trees on $n+1$ vertices which is 
$$f(n+1)=\sum_{\l=0}^{n}f(\l)f(n-\l).$$
It follows that $f(n)=C_n$. 
      \end{ptcbr}

\begin{Ej}[Exercise 6]
   A \textbf{triangulation} of a convex $(n+2)$-gon is a collection $(n-1)$ diagonals that do not intersect each other. Show that the number of triangulations of a convex $(n+2)$-gon is the $n^{\text{th}}$ Catalan numbers $C_n$. \hint{Show that they satisfy the recursion for the Dyck paths}
   \end{Ej}
   
   \begin{ptcbr}
      In the same way we chose a \emph{left} and \emph{right} trees, here we will chose L-and-R triangulations.\par 
      The initial conditions don't match up until $T(2)=2=C_2$. Nonetheless we will form the recurrence for $n\geq 2$. Suppose we want to find $T(n)$, for that effect let us construct an $(n+2)$-gon labeling the vertices $1$ through $n+2$:
      \begin{center}

         \tikzset{every picture/.style={line width=0.75pt}} %set default line width to 0.75pt        
         
         \begin{tikzpicture}[x=0.75pt,y=0.75pt,yscale=-1,xscale=1]
         %uncomment if require: \path (0,300); %set diagram left start at 0, and has height of 300
         
         %Straight Lines [id:da14032885786543736] 
         \draw    (45.5,33.75) -- (93.5,17) ;
         %Straight Lines [id:da18483984818254018] 
         \draw    (93.5,17) -- (140,29.5) ;
         %Straight Lines [id:da8230294269720508] 
         \draw    (140,29.5) -- (163,65.5) ;
         %Straight Lines [id:da7802267531570728] 
         \draw    (113.5,158.5) -- (152.5,140) ;
         %Straight Lines [id:da7717797702169771] 
         \draw    (26,77) -- (45.5,33.75) ;
         %Straight Lines [id:da6555889756570468] 
         \draw    (76.5,150.5) -- (113.5,158.5) ;
         %Straight Lines [id:da6388975755574837] 
         \draw    (235.5,30.25) -- (293.5,17) ;
         %Straight Lines [id:da14610674625034537] 
         \draw    (293.5,17) -- (340,29.5) ;
         %Straight Lines [id:da8227274811207295] 
         \draw    (340,29.5) -- (363,65.5) ;
         %Curve Lines [id:da8690431311277056] 
         \draw  [dash pattern={on 4.5pt off 4.5pt}]  (363,65.5) .. controls (376.5,103.75) and (373.5,113.75) .. (352.5,140) ;
         %Straight Lines [id:da2772128926893782] 
         \draw    (313.5,158.5) -- (352.5,140) ;
         %Straight Lines [id:da5820372167779093] 
         \draw    (226,77) -- (235.5,30.25) ;
         %Curve Lines [id:da32660025377744484] 
         \draw  [dash pattern={on 4.5pt off 4.5pt}]  (226,77) .. controls (225,112.25) and (231,93.75) .. (242.5,114.75) .. controls (254,135.75) and (237.5,140.75) .. (254,149.25) ;
         %Straight Lines [id:da6218021333670047] 
         \draw    (254,149.25) -- (313.5,158.5) ;
         %Straight Lines [id:da9052448245737825] 
         \draw    (293.5,17) -- (313.5,158.5) ;
         %Straight Lines [id:da35479330057110614] 
         \draw    (340,29.5) -- (313.5,158.5) ;
         %Curve Lines [id:da3753608201652048] 
         \draw  [dash pattern={on 4.5pt off 4.5pt}]  (163,65.5) .. controls (170.25,86.04) and (172.74,98.44) .. (170.24,109.58) .. controls (168.08,119.18) and (162.22,127.85) .. (152.5,140) ;
         %Curve Lines [id:da17362192875108295] 
         \draw  [dash pattern={on 4.5pt off 4.5pt}]  (26,77) .. controls (25,112.25) and (33,95.75) .. (47.5,116.25) .. controls (62,136.75) and (43.5,140.25) .. (76.5,150.5) ;
         
         % Text Node
         \draw (95.5,13.6) node [anchor=south west] [inner sep=0.75pt]  [font=\scriptsize]  {$v_{1}$};
         % Text Node
         \draw (142,26.1) node [anchor=south west] [inner sep=0.75pt]  [font=\scriptsize]  {$v_{2}$};
         % Text Node
         \draw (165,62.1) node [anchor=south west] [inner sep=0.75pt]  [font=\scriptsize]  {$v_{3}$};
         % Text Node
         \draw (154.5,143.4) node [anchor=north west][inner sep=0.75pt]  [font=\scriptsize]  {$v_{k-1}$};
         % Text Node
         \draw (115.5,161.9) node [anchor=north west][inner sep=0.75pt]  [font=\scriptsize]  {$v_{k}$};
         % Text Node
         \draw (74.5,153.9) node [anchor=north east] [inner sep=0.75pt]  [font=\scriptsize]  {$v_{k+1}$};
         % Text Node
         \draw (24,73.6) node [anchor=south east] [inner sep=0.75pt]  [font=\scriptsize]  {$v_{n+1}$};
         % Text Node
         \draw (43.5,30.35) node [anchor=south east] [inner sep=0.75pt]  [font=\scriptsize]  {$v_{n+2}$};
         % Text Node
         \draw (295.5,13.6) node [anchor=south west] [inner sep=0.75pt]  [font=\scriptsize]  {$v_{1}$};
         % Text Node
         \draw (342,26.1) node [anchor=south west] [inner sep=0.75pt]  [font=\scriptsize]  {$v_{2}$};
         % Text Node
         \draw (365,62.1) node [anchor=south west] [inner sep=0.75pt]  [font=\scriptsize]  {$v_{3}$};
         % Text Node
         \draw (354.5,143.4) node [anchor=north west][inner sep=0.75pt]  [font=\scriptsize]  {$v_{k-1}$};
         % Text Node
         \draw (315.5,161.9) node [anchor=north west][inner sep=0.75pt]  [font=\scriptsize]  {$v_{k}$};
         % Text Node
         \draw (252,152.65) node [anchor=north east] [inner sep=0.75pt]  [font=\scriptsize]  {$v_{k+1}$};
         % Text Node
         \draw (224,73.6) node [anchor=south east] [inner sep=0.75pt]  [font=\scriptsize]  {$v_{n+1}$};
         % Text Node
         \draw (233.5,26.85) node [anchor=south east] [inner sep=0.75pt]  [font=\scriptsize]  {$v_{n+2}$};
         % Text Node
         \draw (51.5,80.4) node [anchor=north west][inner sep=0.75pt]    {$( n+2) -\text{gon}$};
         % Text Node
         \draw (253.59,25) node [anchor=north west][inner sep=0.75pt]  [font=\tiny,rotate=-65.26]  {$[( n+2) -k+1+1] -\text{gon}$};
         % Text Node
         \draw (324.8,142.2) node [anchor=north west][inner sep=0.75pt]  [font=\footnotesize,rotate=-282.57]  {$( k-1) -\text{gon}$};
         
         
         \end{tikzpicture}
         \end{center} 
      We take the first two vertices of our polygon, note that this choice is independent of the actual number of triangulations, and from them we pick vertex $k$ to draw a triangle.\par 
      Given an \emph{nice orientation} we can see that we have a \emph{left}-gon and a \emph{right}-gon. The right one contains vertices from $2$ through $k$ which amount to $k-2+1=k-1$ vertices. While the left one runs from $k$ to $n+2$ and $1$. The number of vertices on the left is $(n+2)-k+1+1$. So the number of triangulations given that vertex $k$ we chose is 
      $$T[(k-1)-2]T[(n-k+4)-2].$$
      Summing through all the possible choices of $k$, we run from $3$ through $n+2$. This means that 
      $$T(n)=\sum_{k=3}^{n+2}T(k-3)T(n-k+2)\xrightarrow[\substack{k\to 3\\\To\l\to 0\\k\to n+2\\\To\l\to n-1}]{\l=k-3}\sum_{\l=0}^{n-1}T(\l)T(n-\l-1)=\sum_{\l=0}^{n-1}T(\l)T[(n-1)-\l].$$ 
      This is precisely the recurrence which defines the Catalan numbers and so $T(n)=C_n$ for $n\geq 2$.
      \end{ptcbr}

   \begin{Ej}[Exercise 8]
         A \textbf{derangement} of $[n]$ is a permutation $\pi\in S_n$ with no fixed points. That is $\forall i(\pi(i)\neq i)$. Let $D_n$ be the number of derangements of $[n]$. Prove that 
         $$\sum_{n=0}^\infty\frac{D_n}{n!}x^n=\frac{e^{-x}}{1-x}.$$  
         \end{Ej}
%https://math.stackexchange.com/questions/203899/combinatorial-argument-to-prove-the-recurrence-relation-for-number-of-derangemen
%https://math.stackexchange.com/questions/240357/exponential-generating-function-for-derangements
   \begin{ptcbr}
      Let us begin by establishing a recurrence relation for $D_n$. We will do this by considering \ttt{grad students} and their \ttt{preffered place to sit at}. Then the number $D_n$ is the number of ways not \ttt{grad student} sits at their \ttt{preffered desk}.\par 
      Suppose that the first \ttt{grad student} enters the room and sits on desk $i$. When the $i^{\text{th}}$ \ttt{grad student} enters the room there are two possibilities:
      \begin{itemize}
         \itemsep=-0.4em
         \item They sit on desk 1, and then the problem reduces to the case with $n-2$ \ttt{grad students}.
         \item Otherwise we may relabel \ttt{grad student} $i$ as the first \ttt{grad student} and then say that desk 1 is $i$'s preferred desk. This reduces to the case of $n-1$ \ttt{grad students}.
      \end{itemize}
      Since this events are disjoint, the possibilities for each must be summed. But our choice for the first one's preference was arbitrary, there are other $n-1$ possible choices. It follows that 
      $$D_n=(n-1)(D_{n-1}+D_{n-2}).$$
      Let us shift indices to obtain the recurrence $D_{n+2}=(n+1)(D_{n+1}+D_n)$. By taking the exponential generating function on both sides we get 
      $$\sum_{n=0}^\infty D_{n+2}\frac{x^n}{n!}=\sum_{n=0}^\infty (n+1)D_{n+1}\frac{x^n}{n!}+\sum_{n=0}^\infty (n+1)D_{n}\frac{x^n}{n!}$$
      and if we call $\cD(x)$, $D_n$'s e.g.f. then this equation translates to the differential equation
      \begin{align*}
         D^2\cD(x)&=(xD+1)D\cD(x)+(xD+1)\cD(x),\\
         &=xD^2\cD(x)+D\cD(x)+xD\cD(x)+\cD(x).
      \end{align*}  
      Let's switch notation to make this equation a bit more refreshing to the eyes, say $y=y(x)=\cD(x)$ and we will change the differential operator by primes:
      $$y''=xy''+y'+xy'+y.$$
      Let us gather two initial conditions for this equation. By evaluating $\cD$ and $\cD'$ at $x=0$ we recover the following
      $$\cD(0)=D_0=1,\ \cD'(0)=D_1=0$$
      where we take $D_0=1$ by convention and $D_1$ tells us that there are no permutations on ${1}$ which do not fix $1$.\par 
      Now we can solve the differential equation as follows:
      \begin{align*}
         (1-x)y''-y'=xy'+y&\To\dv{x}\left((1-x)y'\right)=\dv{x}(xy),\\
         &\To (1-x)y'=xy+c_1,\\
         (x\to 0)&\To (1-0)(0)=(0)(1)+c_1\To c_1=0,\\
         &\To (1-x)y'=xy,\\
         &\To \frac{y'}{y}=\frac{x}{1-x}=-\bonj{\frac{-x+1-1}{1-x}}=-\left(1-\frac{1}{1-x}\right),\\
         &\To \log(y)=-x-\log(1-x)+c_2,\\
         (x\to 0)&\To \log(1)=0-0+c_2\To c_2=0,\\
         &\To y=e^{-x}\frac{1}{1-x}.
      \end{align*}
      We can conclude that $\cD(x)=\frac{e^{-x}}{1-x}$ as we wanted.
   \end{ptcbr}
\end{document}
