\documentclass[12pt]{memoir}

\def\nsemestre {II}
\def\nterm {Fall}
\def\nyear {2022}
\def\nprofesor {Maria Gillespie}
\def\nsigla {MATH501}
\def\nsiglahead {Combinatorics}
\def\nextra {HW1}
\def\nlang {ENG}
\input{../../headerVarillyDiff}

\begin{document}

\begin{Ej}[Exercise 1]
    Prove that addition and multiplication of cardinalities satisfies the distributive law, that is, that 
    $$|A|(|B|+|C|)=|A|\.|B|+|A|\.|C|,$$
    using the definition of addition and multiplication of cardinalities that we defined in class.
\end{Ej}

\begin{ptcbr}
To prove the equality using the definition of cardinalities, we would like to exhibit a bijection from a set whose size is the amount on the left to a set whose size is the amount on the right.\par 
Consider any sets $A,B$ and $C$. The following table summarizes the information we are working with
\begin{center}
\begin{tabular}{lll}
    Set & Cardinality & Elements \\
    \hline
    $A\x(B\cupdot C)$ &  $|A|(|B|+|C|)$ &  $(a,(x,n))$\\
    $(A\x B)\cupdot(A\x C)$& $|A|\.|B|+|A|\.|C|$ & $((a,x),n)$
\end{tabular}
\end{center}
with $n\in\set{0,1}$ and $n=0\To x\in B$, while $n=1\To x\in C$. This is because the disjoint union can be constructed in the following way
$$B\cupdot C=(B\x\set{0})\cup(C\x\set{1}).$$
Because of this construction it doesn't matter if $B$ and $C$ share elements since they will be labeled.\par 
Now the function 
$$f:A\x(B\cupdot C)\to (A\x B)\cupdot(A\x C),\ (a,(x,n))\mapsto ((a,x),n)$$
can be proven to be well-defined, injective and surjective. It is therefore bijective and its inverse is the function that follows the rule $((a,x),n)\mapsto (a,(x,n))$.\par 
Since we have found the bijection in question, it follows that both sets have the same cardinalities and thus the quantities in question are equal.
\end{ptcbr}

\begin{Ej}[Exercise 2]
Prove the binomial theorem using a combinatorial argument as follows.
\begin{enumerate}[i)]
    \itemsep=-0.4em
    \item Show that, for all positive integers $s, t$, and $n$, we have
    $$\sum_{k=0}^n\binom{n}{k}s^kt^{n-k}=(s+t)^n.$$
    In particular, do not treat $s$ and $t$ as variables; rather, interpret $(s+t)^n$ as counting something
    parameterized by the integers $s, t, n$ and show that the right hand side counts the same thing.
    \item Defining the polynomials $p(x) = (x + 1)^n$ and $q(x) = \sum_{k=0}^n\binom{n}{k}x^k$, we have that $p(s) = q(s)$
    for all positive integers s. Use the fact that polynomials in one variable that agree on infinitely many
    values must be the same to conclude that $p(x) = q(x)$ as polynomials.
    \item Finally, plug in $x/y$ and clear the denominators on both sides of the equation $p(x/y) = q(x/y)$ to show that the binomial theorem holds.
\end{enumerate}
\end{Ej}

\begin{ptcbr}
    \begin{enumerate}
        \itemsep=-0.4em
        \item 
        %Consider finite sets $S,T$ with cardinalities $s,t$ respectively. The quantity $s+t$ is the cardinality of the set $S\cupdot T$ which can be realized as $(S\x\set{0})\cup(T\x\set{1})$. Now the $n^{\text{th}}$ power of that quantity counts the number of elements in the $n$-fold cartesian product of $S\cupdot T$.\par 
        %We can expand the sum of the equality to begin analyzing the sets
        %$$\binom{n}{0}t^n+\binom{n}{1}st^{n-1}+\dots+\binom{n}{n-1}s^{n-1}t+\binom{n}{n}s^n.$$
        Consider finite sets $S,T$ with cardinalities $s,t$ respectively. The quantity $(s+t)^n$ counts the number of 
    \end{enumerate}
\end{ptcbr}


\end{document} 
