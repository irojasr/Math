\documentclass[12pt]{memoir}

\def\nsemestre {II}
\def\nterm {Fall}
\def\nyear {2022}
\def\nprofesor {Maria Gillespie}
\def\nsigla {MATH501}
\def\nsiglahead {Combinatorics}
\def\nextra {HW4}
\def\nlang {ENG}
\input{../../headerVarillyDiff}
\DeclareMathOperator{\des}{des}
\DeclareMathOperator{\inv}{inv}
\DeclareMathOperator{\exc}{exc}
\DeclareMathOperator{\maj}{maj}
\begin{document}

\begin{Ej}[Exercise 2]
   Prove the $q$-binomial theorem 
   $$\prod_{j=1}^n(1+xq^j)=\sum_{k=0}^nq^{\frac{k(k+1)}{2}}\binom{\vec{n}}{\vec{k}}x^k.$$
   You may either use an induction or combinatorial argument.
\end{Ej}
%http://www-groups.mcs.st-andrews.ac.uk/~pjc/Teaching/MT5821/1/l6.pdf
\begin{ptcbr}
    By induction on $n$, when $n=1$ we get 
    $$(1+xq)=q^0(1)x^0+q(1)x=1+qx.$$
    Suppose that the statement is true for every natural number up to $n-1$, then 
    $$\prod_{j=1}^n(1+xq^j)=\left(\prod_{j=1}^{n-1}(1+xq^j)\right)(1+xq^n)=\left(\sum_{k=0}^{n-1}q^{\frac{k(k+1)}{2}}\binom{\vec{n-1}}{\vec{k}}x^k\right)(1+xq^n)$$
    Multiplying the terms we get the following
    $$\sum_{k=0}^{n-1}\left(q^{\frac{k(k+1)}{2}}\binom{\vec{n-1}}{\vec{k}}x^k+q^{\frac{k(k+1)}{2}+n}\binom{\vec{n-1}}{\vec{k}}x^{k+1}\right),$$
    and when collecting coefficients for the $x^k$ term, the resulting quantity is 
    \begin{align*}
        &q^{\frac{k(k+1)}{2}}\binom{\vec{n-1}}{\vec{k}}+q^{\frac{(k-1)k}{2}+n}\binom{\vec{n-1}}{\vec{k-1}}\\
        =&q^{\frac{k(k+1)}{2}}\left\lbrack\binom{\vec{n-1}}{\vec{k}}+q^{\binom{k}{2}+n-\binom{k+1}{2}}\binom{\vec{n-1}}{\vec{k-1}}\right\rbrack\\
        =&q^{\frac{k(k+1)}{2}}\left\lbrack\binom{\vec{n-1}}{\vec{k}}+q^{n-k}\binom{\vec{n-1}}{\vec{k-1}}\right\rbrack.
    \end{align*}
    In brackets, the expression is precisely $\binom{\vec n}{\vec k}$ after applying Pascal's second quantum identity. Thus we get the desired result.
    \iffalse
    Consider the function $p(x)=\prod_{j=1}^n(1+xq^j)$, and let us evaluate at $x=qx$:
    \begin{align*}
        p(qx)&=(1+(qx)q)(1+(qx)q^2)\dots(1+(qx)q^n)\\
        &=(1+xq^2)(1+xq^3)\dots(1+xq^{n+1})\\
        \To(1+xq)p(qx)&=\left\lbrack(1+xq)\dots(1+xq^{n})\right\rbrack(1+xq^{n+1})\\
        \To(1+xq)p(qx)&=p(x)(1+xq^{n+1}).
    \end{align*}
    Now, if we expand $p$ as a sum of monomials and call $a_k=a_k(q)$ the coefficient of $x^k$, we can compare coefficients on both sides:
    \fi
\end{ptcbr}

\begin{Ej}[Exercise 3]
Consider a noncommutative ring over R generated by two variables x
and y, having the ``noncommutative multiplication'' rule $yx=qxy$ where $q\in\bR$ is nonzero.\par 
Assume that multiplication is still distributive across addition, and all constants commute with each other and all variables.\par 
For instance,
$$x\.3=3\.x,\quad x^2y\neq yx^2.$$
Prove that in this algebra we have the ``quantum binomial theorem'':
$$(x+y)^n=\sum_{k=0}^n\binom{\vec n}{\vec k}x^ky^{n-k}.$$
\end{Ej}

\begin{ptcbr}
    Let us view any $n$-term multiplication from the $(x+y)^n$ expanded out without applying any commutation rule. The products would be strings of $x$ and $y$.\par
    For a fixed $k$, we count an inversion of this string as using the quantum commutative rule we see that $s=q^{\inv(s)}x^ky^{n-k}$. However, there are $\binom{n}{k}$ ways to pick out our $k$ $x$'s and along that, there are $\binom{\vec{n}}{\vec{k}}$ possibilities of inversions among all of the strings of length $n$ with $k$ $x$'s.\par 
    Summing across all possible combinations of strings we get the desired sum. And since the product on the left also counts all the possible combinations of strings with their respective inversions, we conclude that this quantities must be equal since they count the same objects.

    \iffalse
    By induction once again, when $n=1$ we get 
    $$x+y=(1)(1)y+(1)x(1)=y+x.$$
    Now let us assume that the identity is valid up to $n-1$, then 
    \begin{align*}
        (x+y)^n&=\left(\sum_{k=0}^{n-1}\binom{\vec{n-1}}{\vec k}x^ky^{(n-1)-k}\right)(x+y)\\
        &=\sum_{k=0}^{n-1}\binom{\vec{n-1}}{\vec k}x^ky^{(n-1)-k}x+\sum_{k=0}^{n-1}\binom{\vec{n-1}}{\vec k}x^ky^{n-k}.
    \end{align*}
    On the first sum, we have to use our commutation relation on $y^{(n-1)-k}x$, this is 
    $$\underbrace{yy\dots y}_{n-1-k\ \text{times}}\hspace{-0.72em}x=q\hspace{-0.72em}\underbrace{yy\dots y}_{n-2-k\ \text{times}}\hspace{-0.72em}xy=q^2\hspace{-0.72em}\underbrace{yy\dots y}_{n-3-k\ \text{times}}\hspace{-0.72em}xy^2=q^{(n-1)-k}xy^{(n-1)-k},$$
    where we have implicitly used induction on the commutation relation. We have obtained the following expression
    $$\sum_{k=0}^{n-1}\binom{\vec{n-1}}{\vec k}q^{(n-1)-k}x^{k+1}y^{(n-1)-k}+\sum_{k=0}^{n-1}\binom{\vec{n-1}}{\vec k}x^ky^{n-k}$$
\fi
\end{ptcbr}

\begin{Ej}
    For a permutation $w\in S_n$, define $\des(w)$ to be the number of descents of $w$ (as opposed to the sum of the descents, which is denoted $\text{maj}$). Define an excedance of $w$ to be an index $i$ such that $w_i>i$.\par 
    Show that the statistics $\des$ and $\exc$ are equidistributed on $S_n$.
\end{Ej}

\begin{ptcbr}
    To show that $\des$ and $\exc$ are equidistributed, it suffices to show that there exists a bijective function $\vf:S_n\to S_n$ such that the following diagram commutes:
    % https://tikzcd.yichuanshen.de/#N4Igdg9gJgpgziAXAbVABwnAlgFyxMJZABgBpiBdUkANwEMAbAVxiRAGUB9QgX1PUy58hFACZSARiq1GLNgB15AWzo4AFgCMNAAgBaIPgOx4CRMqOn1mrRB24H+IDMeFEJky7JsgD0mFABzeCJQADMAJwglJDIQHAgkdxAGOg0YBgAFQRMRZJhQnBBqKzlbWARDEAiopHE4hMQkku8YAA8AYyLk1PSsl1NbcKwAtULK6ujEWPja6jUYOig2SDBWagYsVbYoOjh5peKvBXl6cLQ1LC6UtMzs10Hh0YcwyMmAZmoZxA-kiAg0IiiADsZFCjDgMGk116dwGICGIzGFB4QA
    \begin{center}
        \begin{tikzcd}
            S_n \arrow[rrd, "\des"] \arrow[dd, "\varphi"', no head, dashed] &\\
            & {} \arrow[loop, distance=3.5em, in=220, out=125] & \mathbb Z \\
S_n \arrow[rru, "\exc"']&&          
            \end{tikzcd}
    \end{center}
Let us call $F$ Foata's bijection on $S_n$, which converts a cycle to list notation\footnote{Albeit a bit more convoluted after rearranging.}, and $g$ the function which sends a permutation to its inverse, this last function is invertible because $S_n$ is a group.\par 
The claim is that $(g\circ F^{-1})$\footnote{\textbf{Ian} came up with the idea of using Foata's bijection and then reversing the order of the cycles and lastly applying the inverse of Foata's bijection. Formally that idea is expressed at that product.} is our desired $\vf$. First, this function is a composition of bijections so it's bijective.\par 
Consider any permutation $\pi\in S_n$ written in list notation. Now $F^{-1}(\pi)$ is a permutation written in canonical disjoint cycle notation, including the $1$-cycles. This means that 
$$F^{-1}(\pi)=c_1c_2\dots c_r\To g(F^{-1}(\pi))=c_1^{-1}c_2^{-1}\dots c_r^{-1}$$
where the relative positions of the cycles aren't altered because they are disjoint and therefore commute.\par 
Consider the the pair $\pi_i\pi_{i+1}$ inside of $\pi$. Suppose it was a descent, this means that $\pi_i>\pi_{i+1}$. After mapping through $g\circ F^{-1}$, this pair must be adjacent in one of the cycles, say $c_j^{-1}$ in such a way that 
$$c_j^{-1}=(\dots\pi_{i+1}\pi_i\dots)$$
and what this means is that $\pi_{i+1}\mapsto\pi_i$. Therefore this pair has become an excedance because $\pi_i>\pi_{i+1}$\footnote{\textbf{Sam} and \textbf{Ian} were kind enough to explain why the process worked after I did various examples to check that it indeed worked. The part that was killing me was that I when I returned the last permutation to list notation I thought that it could become something like $\pi_{i+1}\pi_i\dots$ and so $1\mapsto \pi_{i+1}$ for example.}. After replacing the $>$ signs with $\leq$ signs we can show that non-descents don't become excedances. This means that the amount of excedances after applying $(g\circ F^{-1})$ is the same as the number of descents in the original $\pi$.\par 
We conclude that $\des(\pi)=\exc(\vf(\pi))$ and this means that $\des$ and $\exc$ are equidistributed.
\iffalse
Finally after dropping the parenthesis with Foata's bijection we get another permutation $\vf(\pi)\in S_n$. All of $\vf(\pi)$'s descents have switched to excedances without generating more\footnote{This is the key idea which \textbf{Sam} explained to me.}. This happens because all of the indices at which a descent occurred, have inverted their positions.
\fi
\end{ptcbr}
 \end{document} 
