\documentclass[12pt]{memoir}

\def\nsemestre {II}
\def\nterm {Fall}
\def\nyear {2022}
\def\nprofesor {Maria Gillespie}
\def\nsigla {MATH501}
\def\nsiglahead {Combinatorics}
\def\nextra {HW4}
\def\nlang {ENG}
\input{../../headerVarillyDiff}

\begin{document}

\begin{Ej}[Exercise 2]
   Prove the $q$-binomial theorem 
   $$\prod_{j=1}^n(1+xq^j)=\sum_{k=0}^nq^{\frac{k(k+1)}{2}}\binom{\vec{n}}{\vec{k}}x^k.$$
   You may either use an induction or combinatorial argument.
\end{Ej}

\begin{ptcbr}
    Consider the function $p(x)=\prod_{j=1}^n(1+xq^j)$, and let us evaluate at $x=qx$:
    \begin{align*}
        p(qx)&=(1+(qx)q)(1+(qx)q^2)\dots(1+(qx)q^n)\\
        &=(1+xq^2)(1+xq^3)\dots(1+xq^{n+1})\\
        \To(1+xq)p(qx)&=\left\lbrack(1+xq)\dots(1+xq^{n})\right\rbrack(1+xq^{n+1})\\
        \To(1+xq)p(qx)&=p(x)(1+xq^{n+1}).
    \end{align*}
    Now, if we expand $p$ as a sum of monomials and call $a_k=a_k(q)$ the coefficient of $x^k$, we can compare coefficients on both sides:
\end{ptcbr}

 \end{document} 
