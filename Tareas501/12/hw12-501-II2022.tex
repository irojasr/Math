\documentclass[12pt]{memoir}

\def\nsemestre {II}
\def\nterm {Fall}
\def\nyear {2022}
\def\nprofesor {Maria Gillespie}
\def\nsigla {MATH501}
\def\nsiglahead {Combinatorics}
\def\nextra {HW12}
\def\nlang {ENG}
\input{../../headerVarillyDiff}

\begin{document}

\begin{Ej}[Exercise 1]
    Among a family of 20 people, 9 of them like chocolate ice cream, 7 of them like vanilla ice cream, and 2 like both chocolate and vanilla. How many don't like either flavor of ice cream? Explain how the Inclusion-Exclusion principle applies here.
 \end{Ej}

\begin{ptcbr}
   Call $F$ the set of family members, $C$ the set of people who like chocolate ice cream and $V$ the ones who like vanilla. The set $F$ can be partitioned as 
   $$F=(C\cup V)\cupdot (C\cup V)^c$$
   where $(C\cup V)^c$ is the set of family members who dislike both flavors. This is the set we are looking for.\par 
   By Inclusion-Exclusion applied to the two sets $C,V$ we have 
   $$|C\cup V|-|C|-|V|+|C\cap V|=0\To |C\cup V|=9+7-2=14.$$
   As the whole family has 20 members and 14 of them like either flavor of ice cream, then 6 of them dislike both flavors.\par 
   We must use the Inclusion-Exclusion formula to find the size of the union, by summing only $|C|$ and $|V|$ we would be overcounting the the ones who like both flavors.
\end{ptcbr}

\begin{Ej}
    Let $P$ be a poset in which every interval $[x, y]$ is finite. Show that, in the incidence algebra $\mathscr{I}(P)$:
    \vspace*{-0.4em}
    \begin{enumerate}
        \itemsep=-0.4em
        \item $f$ is invertible if and only if $\forall x(f(x,x)\neq 0)$.
        \item $fg=\dl\iff gf=\dl$, this is, inverses are two sided.
        \item If $f$ is invertible then $f^{-1}$ is unique.
    \end{enumerate}
\end{Ej}

Really quickly, recall that the incidence algebra is the set of \emph{interval functions} from $P$ to $\bC$. In other words, we can describe $\mathscr{I}(P)$ as 
$$\mathscr{I}(P)=\set{f:P^2\to\bC:\ x>y\To f(x,y)=0}.$$
\begin{ptcbr}
    \begin{enumerate}
        \itemsep=-0.4em
        \item Suppose $f$ is invertible with $fg=\dl$. If $x\in P$:
        $$(f\.g)(x,x)=f(x,x)g(x,x)=\dl(x,x)=1.$$
        This means that, as complex numbers, $f(x,x)g(x,x)=1$ thus none can be zero and $f(x,x)=\frac{1}{g(x,x)}$.\par 
        On the other hand, suppose $f(x,x)\neq 0$. We will construct an inverse for $f$ inductively using the fact the every interval is finite.\par
        Our base case is $|[x,y]|=1$, then $x=y$ and $g(x,x)=\frac{1}{f(x,x)}$. Suppose that we have an interval $[x,y]$ of length $n$ and for intervals of length less than $n$ $g(x,y)$ is the inverse of $f(x,y)$. So 
        \begin{align*}
            \dl(x,y)=(fg)(x,y)\iff &0=\sum_{x\leq z\leq y}f(x,z)g(z,y)\\
            \iff &0=f(x,x)g(x,y)+\sum_{x< z\leq y}f(x,z)g(z,y)\\
            \iff &-f(x,x)g(x,y)=\sum_{x< z\leq y}f(x,z)g(z,y)\\
            \iff &g(x,y)=\frac{-1}{f(x,x)}\sum_{x< z\leq y}f(x,z)g(z,y)\\
        \end{align*}
        Thus it holds that when $f(x,x)\neq 0$, we can solve the previous equation to obtain an expression for the inverse of $f$. By induction, it follows that $f$ is invertible. 
    \end{enumerate}
\end{ptcbr}
\end{document}
