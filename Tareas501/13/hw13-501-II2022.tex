\documentclass[12pt]{memoir}

\def\nsemestre {II}
\def\nterm {Fall}
\def\nyear {2022}
\def\nprofesor {Maria Gillespie}
\def\nsigla {MATH501}
\def\nsiglahead {Combinatorics}
\def\nextra {HW13}
\def\nlang {ENG}
\input{../../headerVarillyDiff}
\DeclareMathOperator{\inv}{inv}
\begin{document}

The purpose of the whole exercise 34 is to prove that the Vandermonde determinant is equal to the product of binomials. My initial thought when I saw part $(a)$ was, ``\emph{oh, that expression is Vandermonde's determinant, I can just try to relate the tournament to something about powers of monomials.}'' But then I saw the last part and understood that the objective was not to use the fact that the determinant of the matrix of monomials is what it is, instead we see that \emph{that} matrix is related to our problem and that the product is also related and therefore the matrix's determinant is what it should be. So without further ado:

\begin{Ej}[Exercise 3, Stanley 2.34(a)]
    Show that $\sum_Tw(T)=\prod_{i,j\in[n]}(x_j-x_i)$
    where the sum is taken over all $2^{\binom{n}{2}}$ tournaments on $[n]$.
 \end{Ej}

\begin{ptcbr}
As every pair of vertices is joined by exactly one edge and there are no loops, every tournament on $[n]$ has $\binom{n}{2}$ edges. This means that the sum $\sum_Tw(T)$ has $2^{\binom{n}{2}}$ terms.\par 
On the right hand side, the product in question has $\binom{n}{2}$ factors of the form $(x_j-x_i)$ and when expanded as a sum of monomials we find $2^{\binom{n}{2}}$ summands. Each of the monomials in the expanded result of the product corresponds to a weight of a tournament.\par 
The reasoning for that conclusion is as follows. Pick two vertices $i<j$, there are two possibilities to pick a weighted edge between these two:
\vspace*{-0.4em}
\begin{itemize}
    \itemsep=-0.4em
    \item Either we have $i\to j$ with weight $x_j$, 
    \item or we have $i\leftarrow j$ with weight $-x_i$.
\end{itemize}
This determines the choice of monomial in the $(x_j-x_i)$ factor. In the same way, picking either $x_j$ or $-x_i$ in the expansion determines the orientation of the edge between $i$ and $j$. No tournaments are missing nor we have more than necessary as the amount matches up. Since each of the $2^{\binom{n}{2}}$ possible weights is also counted in $\prod_{i,j\in[n]}(x_j-x_i)$, it holds that 
$$\sum_Tw(T)=\prod_{i,j\in[n]}(x_j-x_i).$$
\end{ptcbr}
\newpage
\begin{Ej}[Exercise 4, Stanley 2.34(b)]
    A tournament is transitive if there's a permutation $\pi\in S_n$ such that $\pi(i)<\pi(j)$ when $i\to j$. Show that a non-transitive tournament contains a 3 cycle. 
\end{Ej}

Sometimes I've got to lower my head and admit I made a mistake. \textbf{Kyle} showed me what was wrong with my initial argument and then proceeded to tell me what was the correct thing to do. This was my original proof:
\iffalse
Triangle-free graphs are so \emph{fun}, the first result I saw in graph theory was Mantel's theorem. I still remember that class very vividly, my professor asked us for examples of $K_3$-free graphs and the best I could come up with was a path graph. Eventually other people came up with bipartite graphs and then he stated, ``\emph{The most edges that a $K_3$-graph can have is $n^2/4$ and the extremal graph is $K_{n/2,n/2}$.}''\par 
It is not the same case for directed graphs but still, this problem is related to triangle-free graphs. 
\fi
\begin{ptcb}
    This statement in question is equivalent to $T$ is (3-cycle)-free implies $T$ is transitive. The transitive condition is also equivalent to $i\to j,\ j\to k\in E\To i\to k\in E$. So suppose we have three vertices $i,j,k\in T$ such that $i\to j$ and $j\to k$ are edges of $T$.\par 
    If it were the case that $k\to i$ was an edge in $T$, this would contradict the fact that $T$ is (3-cycle)-free, which is not the case. So there's no $k\to i$ edge.\par 
    As $T$ is a tournament, there \emph{must} be an edge between $i$ and $k$, so the only possibility is that the edge $i\to k$ is in $T$. And so, we have proven that $T$ is transitive as 
    $$i\to j,\ j\to k\in E\To i\to k\in E.$$
\end{ptcb}

The problem with my argument is that I just handwave the fact that usual transitivity is equivalent to book-transitivity. However it is not as trivial to see that the equivalence is true. We will not be seeing that in the proof, but instead we will be proving the statement:
\vspace*{-2em} 
\begin{significant}
$T$ is (3-cycle)-free implies $T$ is book-transitive.
\end{significant}

\begin{ptcbr}
    We must construct a permutation $\pi$ such that $\pi(i)<\pi(j)$ whenever $i\to j$. This is done by induction on $|T|$, the interesting base case is $|T|=3$.\par 
    For that tournament, at least one vertex $v$ must have $d_{\text{in}}(v)=2$, else there would be a 3-cycle. Assign $\pi(v)=3$ and then consider the induced subgraph by the remaining vertices, one sends an edge to another so assign the endpoint vertex to $2$ and the last one to $1$. Thus we have constructed a permutation which respects the condition in question.\par 
    Suppose by induction, that all (3-cycle)-free tournaments with $n$ vertices are book-transitive. We claim that there exists a vertex with in-degree $n$ and to prove this we suppose it is not the case. Instead assume by means of contradiction that all vertices have incoming and outgoing edges.\par 
    As $T$ is finite, this means that there is a directed cycle in $T$ of the form 
    $$v_1\to v_2\to v_3\to\dots\to v_k\to v_1.$$
    We can take the minimal directed cycle and then use the following key observation. As $T$ is (3-cycle)-free, there must be an edge from $v_1$ to $v_3$ because $T$ is a tournament. Otherwise the edge would be $v_3\to v_1$ which contradicts $T$ being triangle free. Thus, no length $k\geq 3$ directed cycle exists in $T$. In consequence there must be a vertex $v$ of $T$ with in-degree $n$, that is, all other vertices point to it.\par 
    Consider $\wh{T}=T\less\set{v}$. As $\wh{T}$ has $n$ vertices, by induction hypothesis, it's book-transitive. We can find $\wh{\pi}\in S_n$ such that $\wh{\pi}(i)<\wh{\pi}(j)$ whenever $i\to j$.\par 
    Now define $\pi:T\to[n+1]$ in such a way that 
    $$\pi\mid_{\wh{T}}=\wh{\pi}\word{and}\pi(v)=n+1.$$
    As $\pi(v)$ is the greatest value and every vertex points to $v$, the book-transitive condition still holds in $T$. By induction we have proven that triangle-free tournaments are book-transitive.
    %https://math.stackexchange.com/questions/3610269/if-a-tournament-graph-has-no-cycles-of-length-3-prove-that-it-is-a-partial-or
\end{ptcbr}

\begin{Ej}[Exercise 4, Stanley 2.34(c)]
The relation $T\otto T'$ means that $T'$ was obtained by reversing a 3-cycle in $T$. Show that if $T\otto T'$ then $w(T')=-w(T)$. 
\end{Ej}

\begin{ptcbr}
    Suppose $\Dl ijk$ is the cycle we will be reversing and consider the induced subgraph generated by $i,j$ and $k$. There are only two possibilities for $\Dl ijk$, either 
    $$i\to j\to k\to i\word{or}i\to k\to j\to i.$$
    As one of the reverse of the other, we will see that their weights differ by sign. The weight of the first one is $x_jx_k(-x_i)$ and the second one is $x_k(-x_j)(-x_i)$. Thus 
    $$-x_ix_jx_k=w(\Dl ijk)=-w(\Dl ikj)=(-1)^2x_ix_jx_k.$$
    In the whole tournament, the weight remains unaltered, the part of the monomial which changes is $x_ix_jx_k$. So it must hold that $w(T')=-w(T)$.
\end{ptcbr}
\newpage
\begin{Ej}[Exercise 4, Stanley 2.34(d)]
    Show that if $T\otto T'$ then $T,T'$ share the number of 3-cycles.  
    \end{Ej}
    
\begin{ptcbr}
    Once again, pick $\Dl ijk$ to be the triangle to be reversed. As $T$ is a tournament, every vertex $v$ in $T\less\set{i,j,k}$ is connected to $\Dl ijk$. There are three cases for each vertex, we will see that after reversing $\Dl ijk$ we still have the same number of triangles:
    \begin{itemize}
        \itemsep=-0.4em
        \item If all edges point to or from $\Dl ijk$ to $v$, then there's no triangle between $v$ and either pair of the vertices in the triangle. Reversing the order inside $\Dl ijk$ has no effect and thus the count remains constant.
        \item If one edge from $v$ points to $i$, without losing generality, then $\Dl ijk$ and $\Dl vij$ are triangles. Reversing the order we get two triangles, namely $\Dl ikj$ and $\Dl vik$. So the count remains the same. 
        \item If two edges point from $v$ to $\Dl ijk$, once again suppose without loss of generality they point to $i$ and $j$. We have the original triangle and $\Dl vjk$ which disappears after reversing orientation but in turn we get $\Dl vik$. 
    \end{itemize}
    As any of the vertices of $T$ falls in any of these three categories, and the count remains the same, we have that the number of triangles of the reversed tournament is the same as in the original case.
\end{ptcbr}
\end{document}
