\documentclass[12pt]{memoir}

\def\nsemestre {II}
\def\nterm {Fall}
\def\nyear {2022}
\def\nprofesor {Maria Gillespie}
\def\nsigla {MATH501}
\def\nsiglahead {Combinatorics}
\def\nextra {HW7}
\def\nlang {ENG}
\input{../../headerVarillyDiff}
\DeclareMathOperator{\des}{des}
\DeclareMathOperator{\inv}{inv}
\DeclareMathOperator{\exc}{exc}
\DeclareMathOperator{\maj}{maj}

\begin{document}

\begin{Ej}[Exercise 3]
   Let $c_n$ be the number of sequences of length $n$ in which:
   \begin{itemize}
    \itemsep=-0.4em
    \item Each number is one of $0,1,2,3$.
    \item No two $3$'s are consecutive.
   \end{itemize}
   For instance $0221030132$ is a valid sequence but $033112333$ is not.
   \begin{enumerate}[i)]
    \itemsep=-0.4em 
    \item Find a recursion for $c_n$ (this should be similar to the Fibonacci recurrence). Remember to include the initial conditions!
    \item Use the recursion to find a closed form for the generating function of $c_n$.
    \item Use the formula discussed in class for solving linear recurrences to find an explicit formula for $c_n$ in terms of $n$.
   \end{enumerate}
\end{Ej}

\begin{ptcbr}
\begin{enumerate}[i)]
   \itemsep=-0.4em 
   \item First, we can observe that the total number of sequences of length $n$ with characters $0,\dots,3$ is $4^n$. Dividing this amount into conditioned sequences and forbidden sequences we get the following 
   $$4^n=c_n+f_n,$$
   where $c_n$ is the quantity we are looking for and $f_n$ is the number of \emph{forbidden} sequences of length $n$. This is, sequences which do have $33$ as a substring.\par 
   After considering the initial conditions:
   $$f_0=0,\ f_1=0,\ f_2=1,\ f_3=7,\ f_4=40,$$
   it is possible to conjecture that 
   $$f_{n+2}=3f_{n+1}+3f_n+4^n.$$
   To prove this recurrence we will append a digit to a length $(n+1)$ sequence. There are several ways to this:
   \begin{itemize}
      \itemsep=-0.4em
      \item If the digit we are appending is either $0$, $1$ or $2$, we are not adding any more forbidden substrings. So for each of those digits, our count of forbidden sequences goes up by $f_{n+1}$. Right now we have $3f_{n+1}$ forbidden sequences.
      \item If the digit we are appending is $3$, there are two cases:
      \begin{itemize}
         \itemsep=-0.4em
         \item Either the last digit of the $(n+1)$ sequence is $0$, $1$ or $2$ in which case we are not adding more forbidden strings. Each of those possibilities accounts for $f_n$ forbidden sequences. This adds up to our past total to get $3f_{n+1}+3f_n$.
         \item If the last digit of the $(n+1)$ sequence is $3$, then we just added a forbidden substring. Counting this is the same as counting all $(n+2)$ strings which end in $33$. This amount is $4^n$.
      \end{itemize}
   \end{itemize}
   In total, we have $3f_{n+1}+3f_n+4^n$ forbidden sequences of length $n$. Then by our initial relation we have 
   $$c_n=4^n-f_n=-(3f_{n+1}+3f_n),\ f_0=0,\ f_1=0.\footnote{Even though the recurrence is not in terms of $c$'s, it's still a recursive formula. The derivation for $c_n$'s recursive formula lies below this answer.}$$
   \item Let us derive a generating function for $f_n$ and from that we will obtain $c_n$'s generating function.\par 
   Call $F$, $(f_n)$'s generating function, then the recurrence 
   $$f_{n+2}=3f_{n+1}+3f_n+4^n$$
   translates to the equation 
   $$\frac{F(x)-f_0-f_1x}{x^2}=\frac{3(F(x)-f_0)}{x}+3F(x)+\frac{1}{1-4x}.$$
   Applying the initial conditions we get 
   $$\frac{F(x)}{x^2}=\frac{3F(x)}{x}+3F(x)+\frac{1}{1-4x}.$$
   We can solve for $F$ to obtain 
   \begin{align*}
      &F(x)\left(\frac{1}{x^2}-\frac{3}{x}-3\right)=\frac{1}{1-4x},\\
      \To&F(x)\left(\frac{1-3x-3x^2}{x^2}\right)=\frac{1}{1-4x},\\
      \To&F(x)=\frac{1}{1-4x}\left(\frac{x^2}{1-3x-3x^2}\right).\\
   \end{align*}
   Now let us factor $1-3x-3x^2$ by taking $x=\frac{-3\pm\sqrt{21}}{6}$, where $\al$ is the root with $+$ while $\bt$, the one with negative. Then 
   \begin{align*}
      (1-3x-3x^2)&=-3(x-\al)(x-\bt)=-3\al\bt\left(1-\frac{x}{\al}\right)\left(1-\frac{x}{\bt}\right)\\
      &=\frac{-3}{ab}(1-ax)(1-bx),
   \end{align*}
   where $a,\al$ and $b,\bt$ are pairs of reciprocals. We can now continue to solve $F$ as a sum of partial fractions as follows
   $$F(x)=\frac{-abx^2}{3(1-4x)(1-ax)(1-bx)}=\frac{A}{1-ax}+\frac{B}{1-bx}+\frac{C}{1-4x}.$$
   Homogenizing the denominator on the equation to the right we get
   $$\frac{-ab}{3}x^2=A(1-4x)(1-bx)+B(1-4x)(1-ax)+C(1-ax)(1-bx).$$
   Since this equation holds for any value of $x$, we might substitute certain values to get cleaner equations for $A,B$ and $C$:
   $$
\left\lbrace
\begin{aligned}
   (x=\al)\To&\frac{-ab}{3}\al^2=A(1-4\al)(1-b\al)=A\al^2(4-a)(b-a),\\
   \To&\frac{-ab}{3(4-a)(b-a)}=A,\\
   (x=\bt)\To&\frac{-ab}{3}\bt^2=B(1-4\bt)(1-a\bt)=B\bt^2(4-b)(a-b),\\
   \To&\frac{-ab}{3(4-b)(a-b)}=B,\\
   (x=1/4)\To&\frac{-ab}{3\.16}=C(1-a/4)(1-b/4)=\frac{C}{16}(4-a)(4-b),\\
   \To&\frac{-ab}{3(4-a)(4-b)}=C.\\
\end{aligned}
\right.
$$
Comparing coefficients $(1-ax)(1-bx)=1-(a+b)x+abx^2$, we have that $a+b=3$ and $ab=-3$. Expanding $(4-a)(4-b)=16-4(a+b)+ab=16-12-3=1$. From this we get \un{$C=1$}. Also, using the polarization identity it holds that 
$$|a+b|^2-|a-b|^2=4ab\To|a-b|^2=-(-12-9)=21\To|a-b|=\sqrt{21}.$$
We also have that $\bt<-\frac{1}{2}<\al$, so $b>-2>a$. This means that $a-b=-\sqrt{21}$. From this we can replace in $A$ and $B$'s expressions:
$$A=$$

   \item formula
\end{enumerate}
\end{ptcbr}

\begin{ptcb}
   We can also construct the recurrence in terms of the allowed sequences $c_n$. Take any length $n$ allowed sequence, then there are two possibilities:
   \begin{itemize}
      \itemsep=-0.4em
      \item The last digit is $0$, $1$ or $2$, then the rest of the sequence is a length $(n-1)$ allowed sequence. For each digit we count $c_{n-1}$ allowed sequences. So in total we have $3c_{n-1}$ allowed sequences.
      \item If the last digit is $3$, then the second-to-last digit can't be three. There are only $3$ other possibilities: $0$, $1$ or $2$. For each of these the remaining length $n$ sequence has to fullfil the condition. Which means we count $c_{n-2}$ allowed sequences per digit.
   \end{itemize}
   This total amounts to $c_n=3c_{n-1}+3c_{n-2}$. With this recurrence we have counted all the possibilities since the only options for the last digit are the ones mentioned above. 
\end{ptcb}

\end{document}
