\documentclass[12pt]{memoir}

\def\nsemestre {II}
\def\nterm {Fall}
\def\nyear {2022}
\def\nprofesor {Maria Gillespie}
\def\nsigla {MATH501}
\def\nsiglahead {Combinatorics}
\def\nextra {HW10}
\def\nlang {ENG}
\input{../../headerVarillyDiff}
\DeclareMathOperator{\des}{des}
\DeclareMathOperator{\inv}{inv}
\DeclareMathOperator{\exc}{exc}
\DeclareMathOperator{\maj}{maj}
\usepackage{halloweenmath}

\begin{document}

\begin{Ej}[Exercise 1]
    Find the largest possible size
    of a matching for $P_n$, and find the smallest possible size of a maximal matching for $P_n$. Express your
    answers in terms of $n$ (they may depend on the parity of $n$ or its residue modulo 3).
\end{Ej}

\begin{ptcbr}
    
\end{ptcbr}

\begin{Ej}[Exercise 3]
    Show that the number of spanning trees of $K_{m,n}$ is $m^{n-1}n^{m-1}$.
\end{Ej}

We will follows the steps described in problem 5.66 in Stanley Vol.2.

\begin{ptcbr}
    The adjacency matrix of $K_{m,n}$ can be written in block form:
    $$A=\twobytwo{0_{m\x m}}{\ind_{m\x n}}{\ind_{n\x m}}{0_{n\x n}}.$$
    Here $0$ is the zeroes matrix and $\ind$ is the ones matrix. The vertices of $K_{m,n}$ have degree either $n$ or $m$ so the Laplacian matrix of $K_{m,n}$ is
    $$L=D-A=\twobytwo{nI_{m\x m}}{0_{m\x n}}{0_{n\x m}}{mI_{n\x n}}-\twobytwo{0_{m\x m}}{\ind_{m\x n}}{\ind_{n\x m}}{0_{n\x n}}=\twobytwo{nI_{m\x m}}{-\ind_{m\x n}}{-\ind_{n\x m}}{mI_{n\x n}}.$$
    With this in hand, let us proceed with the computations:
    \begin{enumerate}[i)]
        \itemsep=-0.4em
        \item The matrix $L-mI$ is precisely 
        $$L-mI=\twobytwo{(n-m)I_{m\x m}}{-\ind_{m\x n}}{-\ind_{n\x m}}{0_{n\x n}}$$
        whose last $n$ rows are all identical. We row reduce this matrix in the following way: 
        \begin{itemize}
            \itemsep=-0.4em
            \item Eliminate the last $n-1$ rows subtracting row $n+1$ from them. We initially guess that the rank of this matrix will be $m+1$.
            \item Divide the first $m$ rows by $n-m$ and then eliminate the ones in the $(m+1)^{\text{th}}$ row by subtracting the first $m$ rows from that one. 
            \item Our last row is now $(0,\dots,0,\frac{m}{n-m},\dots,\frac{m}{m-n})$ which we will convert to a row of ones after dividing by $r/(n-m)$. 
            \item We can use the last row to eliminate the $-\ind_{m\x n}$ block on top. 
        \end{itemize}
        The resulting matrix is $\ttt{rref}(L-mI)$, the rank of this matrix is $m+1$ so the rank of $L-mI$ is also $m+1$.\par 
        By the rank nullity theorem, $\dim\ker(L-mI)+(m+1)=m+n$ and so the geometric multiplicity of $m$ is $n-1$. Thus there are \emph{at least} $(n-1)$ eigenvalues equal to $m$.
        \item With the same reasoning we can prove that the geometric multiplicity of $n$ is $m-1$. In which case, there would be at least $m-1$ eigenvalues of $L$ equal to $n$.
        \item The matrix $L$ can have at most $m+n$ eigenvalues, assuming that the algebraic and geometric multiplicities coincide for $m,n$, so summing the multiplicities we get 
        $$(m-1)+(n-1)+\text{remaining}=m+n\To \text{remaining}=2.$$
    \end{enumerate}
\end{ptcbr}
\end{document}
