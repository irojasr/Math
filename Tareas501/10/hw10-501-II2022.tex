\documentclass[12pt]{memoir}

\def\nsemestre {II}
\def\nterm {Fall}
\def\nyear {2022}
\def\nprofesor {Maria Gillespie}
\def\nsigla {MATH501}
\def\nsiglahead {Combinatorics}
\def\nextra {HW10}
\def\nlang {ENG}
\input{../../headerVarillyDiff}
\DeclareMathOperator{\des}{des}
\DeclareMathOperator{\inv}{inv}
\DeclareMathOperator{\exc}{exc}
\DeclareMathOperator{\maj}{maj}
\usepackage{halloweenmath}

\begin{document}

\begin{Ej}[Exercise 1]
    Find the largest possible size
    of a matching for $P_n$, and find the smallest possible size of a maximal matching for $P_n$. Express your
    answers in terms of $n$ (they may depend on the parity of $n$ or its residue modulo 3).
\end{Ej}

\begin{ptcbr}
    Call $m_n$ the minimal number of edges in a maximal matching in $P_n$, and $M_n$ the maximum number of edges in a matching for $P_n$. We summarize the results found by counting the numbers by hand as follows:
    $$
    \begin{matrix}
        n&1&2&3&4&5&6&7&8&9\\
        m_n&0&1&1&1&2&2&2&3&3\\
        M_n&0&1&1&2&2&3&3&4&4
    \end{matrix}\hspace*{1em}
    \begin{matrix}
        10&11&12&13&14&15&16&17&18\\
        3&4&4&4&5&5&5&6&6\\
        5&5&6&6&7&7&8&8&9
    \end{matrix} 
    $$
    We conjecture that $m_n=\floor{\frac{n+1}{3}}$ and $M_n=\floor{\frac{(n+1/2)}{2}}$. These quantities can also be expressed in another way.\par 
    Recall that any integer can be uniquely written as $pr+s$ where $p$ is a prime, $r\geq 0$ and $s\in[p]$. Then we have the following formulas for $m_n$ and $M_n$: 
    $$m_{3r+s+1}=r+1,\ m_1=0,\ M_{2r+s+1}=r+1,\ M_1=0.$$
    The procedure to pick the edges is as follows:
    \begin{itemize}
        \itemsep=-0.4em
        \item For the maximum number of edges we greedily pick edges for our matching. Start with the first edge and then pick the next available edge which can be added to the matching. On $P_{2n}$ we have picked $n$ edges for out matching including the last edge. Adding one edge by considering $P_{2n+1}$ doesn't alter our count because the edge to its immediate left is already taken. However adding another edge in $P_{2n+2}$ adds an edge which is not adjacent to the last edge of $P_{2n}$, so we add that to our matching.
        \item For the minimum number of edges in a maximal matching we must greedily \emph{avoid edges}. As soon as we can move our edge to the right or add a new edge we will do so. 
        \begin{center}
            

\tikzset{every picture/.style={line width=0.75pt}} %set default line width to 0.75pt        

\begin{tikzpicture}[x=0.75pt,y=0.75pt,yscale=-1,xscale=1]
%uncomment if require: \path (0,235); %set diagram left start at 0, and has height of 235

%Straight Lines [id:da01791683848914527] 
\draw    (210,100) -- (240,80) ;
%Straight Lines [id:da6282646308802451] 
\draw    (240,80) .. controls (242.31,79.53) and (243.7,80.46) .. (244.16,82.77) .. controls (244.62,85.08) and (246.01,86.01) .. (248.32,85.55) .. controls (250.63,85.08) and (252.02,86.01) .. (252.48,88.32) .. controls (252.94,90.63) and (254.33,91.56) .. (256.64,91.09) .. controls (258.95,90.63) and (260.34,91.56) .. (260.8,93.87) .. controls (261.26,96.18) and (262.65,97.11) .. (264.96,96.64) .. controls (267.27,96.17) and (268.66,97.1) .. (269.12,99.41) -- (270,100) -- (270,100) ;
%Straight Lines [id:da27448383524296815] 
\draw    (270,100) -- (300,80) ;
%Straight Lines [id:da19434111959020117] 
\draw    (300,100) -- (330,80) ;
%Straight Lines [id:da4948845639195685] 
\draw    (330,80) .. controls (332.31,79.53) and (333.7,80.46) .. (334.16,82.77) .. controls (334.62,85.08) and (336.01,86.01) .. (338.32,85.55) .. controls (340.63,85.08) and (342.02,86.01) .. (342.48,88.32) .. controls (342.94,90.63) and (344.33,91.56) .. (346.64,91.09) .. controls (348.95,90.63) and (350.34,91.56) .. (350.8,93.87) .. controls (351.26,96.18) and (352.65,97.11) .. (354.96,96.64) .. controls (357.27,96.17) and (358.66,97.1) .. (359.12,99.41) -- (360,100) -- (360,100) ;
%Straight Lines [id:da8873549310246829] 
\draw    (360,100) -- (390,80) ;
%Straight Lines [id:da28693933659154225] 
\draw    (390,80) .. controls (392.31,79.53) and (393.7,80.46) .. (394.16,82.77) .. controls (394.62,85.08) and (396.01,86.01) .. (398.32,85.55) .. controls (400.63,85.08) and (402.02,86.01) .. (402.48,88.32) .. controls (402.94,90.63) and (404.33,91.56) .. (406.64,91.09) .. controls (408.95,90.63) and (410.34,91.56) .. (410.8,93.87) .. controls (411.26,96.18) and (412.65,97.11) .. (414.96,96.64) .. controls (417.27,96.17) and (418.66,97.1) .. (419.12,99.41) -- (420,100) -- (420,100) ;
%Straight Lines [id:da04078489023421694] 
\draw    (430,100) -- (460,80) ;
%Straight Lines [id:da7641399983185369] 
\draw    (460,80) .. controls (462.31,79.53) and (463.7,80.46) .. (464.16,82.77) .. controls (464.62,85.08) and (466.01,86.01) .. (468.32,85.55) .. controls (470.63,85.08) and (472.02,86.01) .. (472.48,88.32) .. controls (472.94,90.63) and (474.33,91.56) .. (476.64,91.09) .. controls (478.95,90.63) and (480.34,91.56) .. (480.8,93.87) .. controls (481.26,96.18) and (482.65,97.11) .. (484.96,96.64) .. controls (487.27,96.17) and (488.66,97.1) .. (489.12,99.41) -- (490,100) -- (490,100) ;
%Straight Lines [id:da9051125265457196] 
\draw    (490,100) -- (520,80) ;
%Straight Lines [id:da017039075711011042] 
\draw    (580,80) .. controls (579.54,82.31) and (578.15,83.24) .. (575.84,82.77) .. controls (573.53,82.31) and (572.14,83.24) .. (571.68,85.55) .. controls (571.22,87.86) and (569.83,88.79) .. (567.52,88.32) .. controls (565.21,87.85) and (563.82,88.78) .. (563.36,91.09) .. controls (562.9,93.4) and (561.51,94.33) .. (559.2,93.87) .. controls (556.89,93.4) and (555.5,94.33) .. (555.04,96.64) .. controls (554.58,98.95) and (553.19,99.88) .. (550.88,99.41) -- (550,100) -- (550,100) ;
%Straight Lines [id:da7606697279360992] 
\draw    (520,80) -- (550,100) ;

\end{tikzpicture}

        \end{center}
        In the figure, the wavy edges are the edges in our matching. This is a demonstration of the process of \emph{moving to the right} between $P_5$ and $P_6$. This algorithm guarantees that $P_n$ will have a matching of size $m_n$.
    \end{itemize}
\end{ptcbr}

\begin{Ej}[Exercise 3]
    Show that the number of spanning trees of $K_{m,n}$ is $m^{n-1}n^{m-1}$.
\end{Ej}

We will follows the steps described in problem 5.66 in Stanley Vol.2.

\begin{ptcbr}
    The adjacency matrix of $K_{m,n}$ can be written in block form:
    $$A=\twobytwo{0_{m\x m}}{\ind_{m\x n}}{\ind_{n\x m}}{0_{n\x n}}.$$
    Here $0$ is the zeroes matrix and $\ind$ is the ones matrix. The vertices of $K_{m,n}$ have degree either $n$ or $m$ so the Laplacian matrix of $K_{m,n}$ is
    $$L=D-A=\twobytwo{nI_{m\x m}}{0_{m\x n}}{0_{n\x m}}{mI_{n\x n}}-\twobytwo{0_{m\x m}}{\ind_{m\x n}}{\ind_{n\x m}}{0_{n\x n}}=\twobytwo{nI_{m\x m}}{-\ind_{m\x n}}{-\ind_{n\x m}}{mI_{n\x n}}.$$
    As $L$ is a symmetric matrix, it is diagonalizable. This will come in handy when finding the amount of eigenvalues. With this in hand, let us proceed with the computations:
    \begin{enumerate}[i)]
        \itemsep=-0.4em
        \item The matrix $L-mI$ is precisely 
        $$L-mI=\twobytwo{(n-m)I_{m\x m}}{-\ind_{m\x n}}{-\ind_{n\x m}}{0_{n\x n}}$$
        whose last $n$ rows are all identical. We row reduce this matrix in the following way: 
        \begin{itemize}
            \itemsep=-0.3em
            \item Eliminate the last $n-1$ rows subtracting row $n+1$ from them. We initially guess that the rank of this matrix will be $m+1$.
            \item Divide the first $m$ rows by $n-m$ and then eliminate the ones in the $(m+1)^{\text{th}}$ row by subtracting the first $m$ rows from that one. 
            \item Our last row is now $(0,\dots,0,\frac{m}{n-m},\dots,\frac{m}{m-n})$ which we will convert to a row of ones after dividing by $m/(n-m)$. 
            \item We can use the last row to eliminate the $-\ind_{m\x n}$ block on top. 
        \end{itemize}
        The resulting matrix is $\ttt{rref}(L-mI)$, the rank of this matrix is $m+1$ so the rank of $L-mI$ is also $m+1$.\par 
        By the rank nullity theorem, $\dim\ker(L-mI)+(m+1)=m+n$ and so the geometric multiplicity of $m$ is $n-1$. Thus there are \emph{at least} $(n-1)$ eigenvalues equal to $m$. As $L$ is diagonalizable, the algebraic and geometric multiplicities must coincide, so there are \emph{exactly} $(n-1)$ eigenvalues equal to $m$. 
        \item With the same reasoning we can prove that the geometric multiplicity of $n$ is $m-1$. In which case, there are $m-1$ eigenvalues of $L$ equal to $n$.
        \item The matrix $L$ can have at most $m+n$ eigenvalues, summing the multiplicities we get 
        $$(m-1)+(n-1)+\text{remaining}=m+n\To \text{remaining}=2.$$
        To find the remaining eigenvalues we will consult the determinant and the trace of $L$. As $L$ is a block matrix whose diagonal is made of square blocks we have 
        \begin{align*}
            \det L&=\twobytwo{nI_{m\x m}}{-\ind_{m\x n}}{-\ind_{n\x m}}{mI_{n\x n}}\\
            &=\det(nI_{m\x m})\det(mI_{n\x n}-(-\ind_{n\x m})(nI_{m\x m})^{-1}(-\ind_{m\x n}))\\
            &=n^m\det(mI_{n\x n}-(1/n)(\ind_{m\x n})^\sT(\ind_{m\x n}))
        \end{align*}
        The matrix $(\ind_{m\x n})^\sT(\ind_{m\x n})$ is an $[n\x n]$ with entries $\braket{\ind_{1\x m}}{\ind_{m\x 1}}=m$. Thus the matrix inside the determinant is 
        $$mI_{n\x n}-(m/n)\ind_{n\x n}\To (mI_{n\x n}-(m/n)\ind_{n\x n})\ind_{n\x 1}=0.$$
        As the rows of both matrices sum to the same value $m$, then the corresponding ones vector $\ind_{n\x 1}$ is in their kernel. Our matrix in question has non-trivial kernel and thus is singular. Then it has determinant zero. It follows that $L$ has determinant zero\footnote{I should've realized earlier without invoking the Schur Decomposition, that the rows of $L$ sum to the same value. This means that $\ind_{(m+n)\x 1}$ is in the kernel and thus $L$ is singular. Still, it was a fun exercise to compute that determinant.} and so it has zero as an eigenvalue.\par 
        The trace of our matrix is $\tr(L)=nm+mn=2mn$. And as the sum of the eigenvalues is the trace, we have that 
        $$(m-1)n+(n-1)m+0+\text{last}=2mn\To\text{last}=m+n.$$
        \item Finally, using the Matrix-Tree Theorem we conclude that the number of spanning trees is 
        $$\frac{1}{m+n}(m+n)m^{n-1}n^{m-1}=m^{n-1}n^{m-1}.$$
    \end{enumerate}
\end{ptcbr}

\begin{Ej}[Exercise 4]
    Let $m$ and $n$ be positive integers with $m < n$. How many saturated matchings does the complete bipartite graph $K_{m,n}$ have?
\end{Ej}

\begin{ptcbr}
Call $M\cupdot N$ our partition of the vertices. Then any saturated matching must saturate $M$. Pick any vertex $v\in M$, then we have $n$ possibilities from where to pick our first edge for the matching. Now pick another vertex $u\in M\less\set{v}$, we have $n-1$ possibilities to pick another edge for our matching because we have already picked one edge and $u$ can't be connected to that same vertex on our matching.\par
Iterating this process we see that the total number of ways to construct our matching is 
$$n(n-1)(n-2)\dots(n-m+1)=n^{\un{m}}.$$
\end{ptcbr}


\begin{Ej}[Exercise 9]
   An undirected graph is $k$-regular if every vertex has degree $k$.
   \begin{enumerate}[i)]
    \itemsep=-0.4em
    \item Show that a bipartite $k$-regular graph must have the same number of vertices of each color in a two-coloring.
    \item Show that such a graph has a perfect matching (that saturates both vertex colors).
   \end{enumerate} 
\end{Ej}
%https://www.youtube.com/watch?v=73u0OQCR2rs
%https://www.youtube.com/watch?v=f9VTbSCBq0Y
\begin{ptcbr}
    Call $G=U\cupdot V$ our graph. We will begin by proving that $U$ and $V$ must be of the same size given that $G$ is regular. The number of edges in $G$ can be counted in two ways:
    \begin{itemize}
        \itemsep=-0.4em
        \item Every vertex sends $k$ edges from $U$ to $V$ so there are in total $k|U|$ edges.
        \item By the same reasoning, counting on the other side, there are $k|V|$ edges in total. 
    \end{itemize}
    This means that $k|U|=k|V|$ and thus $|U|=|V|$.
    \begin{enumerate}[i)]
        \itemsep=-0.4em
        \item Color the vertices according to which set they are in. Paint a vertex red if it's in $U$ and blue if it's in $V$. No other two-coloring is proper because if we paint any vertex in $V$ red, it will be connected to all the other red vertices in $U$. 
        \item To find a matching, let us verify Hall's condition. Pick any subset of $S\subseteq U$ and look at edges coming out of it, there are $k|S|$ edges which land in $N(S)$. Now 
        $$E(N(S))=\set{\text{edges to }S}\cup\set{\text{edges to }U\less S}.$$
        There are two possibilities:
        \begin{enumerate}[a)]
            \itemsep=-0.4em
            \item If the second set is empty then $E(N(S))=E(S)$ but we can count $E(N(S))=k|N(S)|$, so $|N(S)|=|S|$ and thus we have Hall's condition.
            \item On the other hand $|E(N(S))|> |E(S)|$ and so $k|N(S)|>k|S|$, so once again we have Hall's condition.
        \end{enumerate}
        By Hall's theorem, there must exist a perfect matching, as such matching saturates $U$ and therefore saturates $V$.
    \end{enumerate}
\end{ptcbr}
\end{document}
