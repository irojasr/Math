\documentclass[12pt]{memoir}

\def\nsemestre {II}
\def\nterm {Fall}
\def\nyear {2022}
\def\nprofesor {Mark Shoemaker}
\def\nsigla {MATH672}
\def\nsiglahead {Algebraic Geometry}
\def\nextra {HW6}
\def\nlang {ENG}
\input{../../headerVarillyDiff}

\begin{document}

\begin{Ej}
 Let $H\subseteq\bP^n$ be a hyperplane. $H$ is defined as the zero locus of a linear equation 
 $$H=\bV(a_0x_0+\dots+a_nx_n),\ a_0,\dots,a_n\in\bC.$$
 Prove that $H\isom\bP^{n-1}$. (Also, be comfortable with the fact that $\bP^n\less L\isom\bA^n$.) 
\end{Ej}

\begin{ptcbr}
    
  \end{ptcbr}

\begin{Ej}
  Let $\nu_2:\bP^2\to\bP^5$ be the Veronese embedding of degree 2. Write out the equations defining the image of $\nu_2$.
\end{Ej}

\begin{ptcbr}
  Recall that this particular Veronese map takes a point in $\bP^2$ to all possible monomials of degree $2$ in $\bP^5$. This means that 
  $$\nu_2(\bonj{u:v:w})=\bonj{u^2:v^2:w^2:uv:vw:wu}.$$
  Recall that the image of the Veronese map is defined using a multi-indexed array, so let us relabel in that sense:
  $$\bonj{u^2:v^2:w^2:uv:vw:wu}=\bonj{z_{2,0,0}:z_{0,2,0}:z_{0,0,2}:z_{1,1,0}:z_{0,1,1}:z_{1,0,1}}.$$
  The defining equations for the image are given by 
  $$z_Iz_J=z_Kz_L,\ I+J=K+L,$$
  where $I,\dots,L$ are our multi-indices. The only ways to sum our multi-indices in a non-trivial manner are:
  $$
  \left\lbrace
  \begin{aligned}
    &(2,0,0)+(0,1,1)=(1,1,0)+(1,0,1)\\
    &(0,2,0)+(1,0,1)=(1,1,0)+(0,1,1)\\
    &(0,0,2)+(1,1,0)=(1,0,1)+(0,1,1)
  \end{aligned}
  \right.\ 
  \left\lbrace
  \begin{aligned}
    &(2,0,0)+(0,2,0)=(1,1,0)+(1,1,0)\\
    &(2,0,0)+(0,0,2)=(1,0,1)+(1,0,1)\\
    &(0,2,0)+(0,0,2)=(0,1,1)+(0,1,1)
  \end{aligned}
  \right.
  $$
  If we name $[a:b:c:d:e:f]$ the coordinates of $\bP^5$, then we have the following system of equations 
  $$
  \left\lbrace
  \begin{aligned}
    &ae=df\\
    &bf=de\\
    &cd=ef\\
  \end{aligned}
  \right.\quad 
  \left\lbrace
  \begin{aligned}
    &ab=d^2\\
    &bc=e^2\\
    &ca=f^2\\
  \end{aligned}
  \right.
  $$
  These are the equations which define $\Im(\nu_2)$.
\end{ptcbr}

\begin{Ej}
  Recall that $\bP^5$ parametrizes the space of all conics in $\bP^2$. Show that the image of the Veronese embedding $\nu_2:\bP^2\to\bP^5$ is exactly those conics which are ``double lines''.
\hint{If $L = \bV(ax + by + cz)$, what is the (degree 2) equation of the associated
double line
}\end{Ej}

\begin{Ej}
  Given four points and a line in $\bP^2$, show that typically
two conics pass through the four points and are tangent to the line. Under
what special conditions on the positions of the points and the line do we
fail to get exactly two? \hint{As we will discuss in detail in Section 6.1,
a line is tangent to a conic if the defining quadratic function has a double
root when restricted to the line; on the other hand, a quadratic polynomial
has a double root if and only if its discriminant, a degree two polynomial
in its coefficients, is zero.}
\end{Ej}
%https://math.stackexchange.com/questions/2815990/determine-conics-by-four-points-and-a-tangent-line
\end{document} 