\documentclass[12pt]{memoir}

\def\nsemestre {II}
\def\nterm {Fall}
\def\nyear {2022}
\def\nprofesor {Mark Shoemaker}
\def\nsigla {MATH672}
\def\nsiglahead {Algebraic Geometry}
\def\nextra {HW7}
\def\nlang {ENG}
\input{../../headerVarillyDiff}
\usepackage{halloweenmath}
\begin{document}

\begin{Ej}
	Find an example of two curves in $\bP^2$ that have the same
	degree but are not isomorphic.
\end{Ej}

\begin{ptcbr}

\end{ptcbr}

\begin{Ej}
	Do the following:
	\begin{enumerate}
		\item Find the Hilbert polynomial $P$ of a $k$-dimensional linear
		      subvariety of $\bP^n$.
		\item  Describe the Hilbert scheme of varieties in $\bP^n$ with Hilbert
		      polynomial $P$.
	\end{enumerate}

\end{Ej}

\begin{ptcbr}

\end{ptcbr}

\begin{Ej}
	Assume that the variety $V\subseteq\bP^n$ has the Hilbert polynomial
	$P(n)$. Calculate the Hilbert polynomial of the image variety $\nu_d(V)\subseteq \bP^{\binom{n+d}{d}-1}$ of the Veronese map. \hint{Do the case of $V=\bP^1$ first.}
\end{Ej}

\begin{ptcbr}
Recall that the Hilbert function for $\bP^1$ is the dimension of, $R_m$, the $m^{\text{th}}$ graded piece of $\bC[x,y]$. The homogenous polynomials in $\bC[x,y]$ have 
$$\set{x^m,x^{m-1}y,\dots,xy^{m-1},y^m}$$
as a basis. So in this case $m\mapsto \dim(R_m)=m+1$ is the Hilbert function of $\bP^1$. Let us now consider the image of $\bP^1$ through the $d\textsuperscript{th}$ Veronese embedding.
\end{ptcbr}

\begin{Ej}
	Using the theorem describing the defining equations for
	$T_pV$ in terms of the equations for $V$, compute the tangent spaces of the
	curves in examples (1), (2), and (3) at the origin.
\end{Ej}

\begin{ptcbr}
\begin{enumerate}
	\item The curve in question is $\bV(y-x^2)$, our function is $P_1(x,y)=y-x^2$ then $\nb P_1(x,y)=(-2x,1)$. The tangent space at the origin is the zero locus of 
	$$\braket{\nb P_1(0,0)}{(x,y)-(0,0)}=\braket{(0,1)}{(x,y)}=y.$$
	This coincides with our original finding because $\bV(y)$ is precisely the $x$-axis which is tangent to the parabola at the origin.
	\item Now we are working with $\bV(y^2-x^2-x^3)$, then $P_2(x,y)=y^2-x^2-x^3$. The differential in this case is 
	$$\nb P_2(x,y)=(-2x-3x^2,2y)\xrightarrow{\eps_0}\nb P_2(0,0)=(0,0)$$
	and so the variety in question is the zero locus of the zero function. As the whole of $\bA^2$ is such set, we can see that this makes sense because the origin is a singular point of our variety. 
	\item Finally let us consider $\bV(y^2-x^3)$. In this case 
	$$\braket{\nb P_3(0,0)}{(x,y)-(0,0)}=\braket{(-3x^2,2y)}{}$$
\end{enumerate}
\end{ptcbr}
\begin{Ej}
	Let $V\subseteq\bP^n$ be a hypersurface defined by a homogeneous
	irreducible polynomial $F$. Find an explicit description of the tangent space
	to $V$ at a point $p$. What conditions on $p$ ensure that the tangent space to
	$V$ at $p$ has dimension $n - 1$?
\end{Ej}

\begin{ptcbr}

\end{ptcbr}

\end{document}