\documentclass[12pt]{memoir}

\def\nsemestre {II}
\def\nterm {Fall}
\def\nyear {2022}
\def\nprofesor {Mark Shoemaker}
\def\nsigla {MATH672}
\def\nsiglahead {Algebraic Geometry}
\def\nextra {HW7}
\def\nlang {ENG}
\input{../../headerVarillyDiff}
\usepackage{halloweenmath}
\begin{document}

\begin{Ej}
	Find an example of two curves in $\bP^2$ that have the same
	degree but are not isomorphic.
\end{Ej}

\begin{ptcbr}
Let us consider the curves $V_1=\bV(xy)$ and $V_2=\bV(xy-z^2)$. To find the degrees of these curves we will calculate their Hilbert polynomials. To that effect let us decompose $\bC[x,y,z]$ into equally graded parts and then use the relations in our ideals:
$$\bC[x,y,z]=\bC\oplus \gen(x,y,z)\oplus\gen(x^2,y^2,z^2,xy,xz,yz)\oplus\dots$$
And so, applying the relation $xy=0$ we lose an $xy$ in the $R_2$ component. Looking at the degree 3 component we get 
$$\gen(x^3,y^3,z^3,\un{x^2y},x^2z,\un{y^2x},y^2z,z^2x,z^2y,\un{xyz}),$$
where the underlined elements are the generators we lose. We can see that the elements we have lost are the degree 1 generators multiplied by $xy$. Likewise in the case of $R_2$ we lost the $xy$ when we multiplied $1$ by it. Therefore, the amount of generators of $R_m$ in $\bC[V_1]$ will be $\multinom{2+1}{m}-\multinom{2+1}{m-2}$. This quantity is 
\begin{align*}
	\binom{2+m}{m}-\binom{2+m-2}{m-2}&=\frac{(m+2)!}{2m!}-\frac{m!}{2(m-2)!}\\
	&=\frac{(m+2)(m+1)}{2}-\frac{m(m-1)}{2}\\
	&=2m+1,
\end{align*}
and so if the degree of the Hilbert polynomial is $k$, then $\deg(V)=k!a_k$. It holds that the degree of $\bV(xy)$ is 2. This can also be seen by intersecting a \emph{general line} through the variety.\par 
On the other hand, when taking the quotient by $\gen(xy-z^2)$ and doing the same process we are losing\footnote{Not exactly losing, I think a better word or description would be \emph{adding a trivial generator to our set.}} the same amount (albeit different ones) of generators on each step. Thus the Hilbert polynomial for $V_2$ is also $2m+1$.\par 
Finally, notice that $V_1$ is a reducible variety as $V_1=\bV(x)\cup\bV(y)$ and $V_2$ is irreducible. Should there be an isomorphism between these varieties, it should preserve reducibility. This is impossible so it holds that $V_1$ and $V_2$ are not isomorphic, but they have the same degrees.
\end{ptcbr}

\begin{Ej}
	Do the following:
	\begin{enumerate}
		\item Find the Hilbert polynomial $P$ of a $k$-dimensional linear
		      subvariety of $\bP^n$.
		\item  Describe the Hilbert scheme of varieties in $\bP^n$ with Hilbert
		      polynomial $P$.
	\end{enumerate}

\end{Ej}

\begin{ptcbr}

\end{ptcbr}

\begin{Ej}
	Assume that the variety $V\subseteq\bP^n$ has the Hilbert polynomial
	$P(n)$. Calculate the Hilbert polynomial of the image variety $\nu_d(V)\subseteq \bP^{\binom{n+d}{d}-1}$ of the Veronese map. \hint{Do the case of $V=\bP^1$ first.}
\end{Ej}

\begin{ptcbr}
Recall that the Hilbert function for $\bP^1$ is the dimension of, $R_m$, the $m^{\text{th}}$ graded piece of $\bC[x,y]$. The homogenous polynomials in $\bC[x,y]$ have 
$$\set{x^m,x^{m-1}y,\dots,xy^{m-1},y^m}$$
as a basis. So in this case $m\mapsto \dim(R_m)=m+1$ is the Hilbert function of $\bP^1$. Let us now consider the image of $\bP^1$ through the $d\textsuperscript{th}$ Veronese embedding.
\end{ptcbr}

\begin{Ej}
	Using the theorem describing the defining equations for
	$T_pV$ in terms of the equations for $V$, compute the tangent spaces of the
	curves in examples (1), (2), and (3) at the origin.
\end{Ej}

\begin{ptcbr}
\begin{enumerate}
	\item The curve in question is $\bV(y-x^2)$, our function is $P_1(x,y)=y-x^2$ then $\nb P_1(x,y)=(-2x,1)$. The tangent space at the origin is the zero locus of 
	$$\braket{\nb P_1(0,0)}{(x,y)-(0,0)}=\braket{(0,1)}{(x,y)}=y.$$
	This coincides with our original finding because $\bV(y)$ is precisely the $x$-axis which is tangent to the parabola at the origin.
	\item Now we are working with $\bV(y^2-x^2-x^3)$, then $P_2(x,y)=y^2-x^2-x^3$. The differential in this case is 
	$$\nb P_2(x,y)=(-2x-3x^2,2y)\xrightarrow{\eps_0}\nb P_2(0,0)=(0,0)$$
	and so the variety in question is the zero locus of the zero function. As the whole of $\bA^2$ is such set, we can see that this makes sense because the origin is a singular point of our variety. 
	\item Finally let us consider $\bV(y^2-x^3)$. In this case 
	$$\braket{\nb P_3(0,0)}{(x,y)-(0,0)}=\braket{(-3(0)^2,2(0))}{(x,y)}=0,$$
	and once again our tangent space is the whole affine plane. This is agrees with what we have seen, the curve has a singular point at the origin.
\end{enumerate}
\end{ptcbr}
\begin{Ej}
	Let $V\subseteq\bP^n$ be a hypersurface defined by a homogeneous
	irreducible polynomial $F$. Find an explicit description of the tangent space
	to $V$ at a point $p$. What conditions on $p$ ensure that the tangent space to
	$V$ at $p$ has dimension $n - 1$?
\end{Ej}

\begin{ptcbr}
Let us begin by considering an affine chart $U_i\isom\bA^n$ which contains $p$. Our projective variety $V$ becomes an affine variety $V\cap U_i$ which is the zero locus of the de-homogenized polynomial $\widetilde{F}=F\eval_{x_i=1}$.\par 
We can now describe the tangent space at $p$ as 
$$T_p(V\cap U_i)=\bV\left(\braket{\nb\widetilde{F}(p)}{\vec{x}-p}\right).$$
The projective closure of this affine algebraic variety is the \emph{projective tangent space} of $V$ at $p$. To find this, let us simplify notation a bit by calling $L$ the linear polynomial in question.
\begin{itemize}
	\item We can see that $L$ is an irreducible polynomial through a degree argument. If $L$ were reducible then $L=pq$ and $\deg(L)=\deg(p)+\deg(q)$. As the degree is an integer, $p$ or $q$ must be a linear polynomial and the other a constant. 
	\item Now the polynomial ring we are working in is a UFD so irreducibles are prime, then it holds that $\gen(L)$ is a prime ideal and therefore radical. 
	\item Recall, by the projective closure theorem, the ideal generated by the homogenization of \emph{all} elements of $\sqrt{\gen(L)}$ is $\bI(\ov{V})$. But as $\sqrt{\gen(L)}=\gen(L)$ we have that $\bI(\ov{V})$ is generated by elements of the form $\pre{h}(p\.L)$ where the homogenization is taken with respect to the variable $x_i$.
\end{itemize}
In summary the tangent space is the zero locus of $\gen(\pre{h}(p\.L))$ where $p$ is any polynomial and $L$ is the differential of $F$.\par 
Now, as $F$ is an homogeneous irreducible polynomial, the variety $V$ has dimension $n-1$. For the tangent space to have that same dimension, it must hold that $p$ is a \emph{smooth point} of $V$. For this to happen $p$ must not be a \emph{singular point} and this happens when 
$$p\not\in\bV(\del_0F,\del_1F,\dots,\del_nF).$$
\end{ptcbr}

\end{document}