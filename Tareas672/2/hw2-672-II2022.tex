\documentclass[12pt]{memoir}

\def\nsemestre {II}
\def\nterm {Fall}
\def\nyear {2022}
\def\nprofesor {Mark Shoemaker}
\def\nsigla {MATH672}
\def\nsiglahead {Algebraic Geometry}
\def\nextra {HW1}
\def\nlang {ENG}
\input{../../headerVarillyDiff}

\begin{document}

\begin{Ej}[2.1.1]
  Suppose $R$ is a ring, prove the following:
  \vspace{-0.4em}
  \begin{enumerate}
    \itemsep=-0.4em
    \item Every maximal ideal is prime.
    \item Every prime ideal is radical.
    \item If $I\rteq R$, then $\sqrt{I}\rteq R$.
  \end{enumerate}
\end{Ej}

\begin{ptcbr}
\begin{enumerate}[i)]
  \itemsep=-0.4em
    \item Let $\lie m\rteq R$, maximal, and $r,s\in R$ with $rs\in\lie m$. Aiming for a contradiction let us suppose that neither $r$ nor $s$ lie inside of $\lie m$.\par 
    Since $r\not\in\lie m$, then the ideal generated by $r$ and $\lie m$ is the whole ring $R$. This means that $ar+m_1=1$ for $a\in R$ and $m_1\in\lie m$. Likewise for some $b\in R$ and $m_2\in\lie m$ it follows that $bs+m_2=1$. Now 
    $$1=(ar+m_1)(bs+m_2)=(ab)rs+arm_2+bsm_1+m_1m_2$$
    and this last expression is a combination of elements in $\lie m$. It follows that $1\in\lie m$, but this means that $R=\lie m$. This is impossible, so it must follow that $r\in\lie m$ or $s\in\lie m$.
    \item Suppose $\lie p\rteq R$ is prime. We want to show that for any $p\in\sqrt{\lie p}$, $p\in\lie p$. As $p\in\sqrt{\lie p}$, then $p^n\in\lie p$. We will prove that $p^n\in\lie p\To p\in\lie p$.\par 
    By induction, our base case is $n=1$, but there is nothing to prove there. So let us suppose that $p^{n+1}\in\lie p$. Then 
    $$p^{n+1}=pp^n\in\lie p\To (p\in\lie p)\lor(p^n\in\lie p).$$
    If the first statement holds, we are done. If the second one holds, we are done by induction hypothesis.
    \item First, let us show that the radical is non-empty. Since $0\in I$, then $0=0^n\in I$ and it follows that $0\in I$.\par 
    Suppose now that $x,y\in\sqrt{I}$, we will show that $x+y, xy\in\sqrt{I}$, thus proving that $\sqrt I$ is a subring. Our hypothesis tells us that $x^m,y^n\in I$ for some $m,n\in\bN$. In this case we have that
    $$(x+y)^{m+n}=\sum_{k=0}^{m+n}x^ky^{m+n-k},$$
    and $x^ky^{m+n-k}\in\sqrt{I}$ because 
    $$k<n\iff k+(m+n)<n+(m+n)\iff (m+n)<m+2n-k$$
    which means that in case that one of our elements is not inside, then the other one will surely be. \par 
    Now $(x^m)^n,(y^n)^m\in I$ and then $(xy)^{mn}\in I$ and so $xy\in\sqrt{I}$.\par 
    Finally if $r\in R$ and $x\in\sqrt{I}$, then $rx^m\in I$. It follows that $(rx)^m\in I$ and so $rx\in\sqrt I$, proving that $\sqrt{I}$ is absorbent. We conclude that $\sqrt{I}$ is an ideal of $R$ whenever $I$ is.
\end{enumerate}
\end{ptcbr}

\begin{Ej}[2.1.2]
  Suppose $R$ is a ring, prove the following:
  \vspace{-0.4em}
  \begin{enumerate}
    \itemsep=-0.4em
    \item $\lie m$ is maximal $\iff\ \quot{R}{\lie m}$ is a field. 
    \item $\lie p$ is prime $\iff\ \quot{R}{\lie p}$ is an integral domain. 
  \end{enumerate}
\end{Ej}

\begin{ptcbr}
  \begin{enumerate}
    \itemsep=-0.4em
    \item $(\To)$ If $\lie m$ is a proper maximal ideal, take $r\in R\less\lie m$. Then the ideal generated by $r$ and $\lie m$ is the whole ring $R$. It follows that for some $a\in R$ and $m\in\lie m$, $ar+m=1$. If we translate this expression to the quotient ring we obtain
    $$ar+m\equiv 1\bmod \lie m\To ar\equiv 1\bmod\lie m.$$ %%CHANGE BMOD TO MOD
    Since $r$ was arbitrary, we have found an inverse $a$ for any element $r$ inside the quotient ring. Since $\quot{R}{\lie m}$ is already a commutative ring with identity, and now we have inverses, it follows that $\quot{R}{\lie m}$ is a field.\par 
    $(\Leftarrow)$ On the other hand suppose $\quot{R}{\lie m}$ is a field.
  \end{enumerate}
\end{ptcbr}
\end{document} 