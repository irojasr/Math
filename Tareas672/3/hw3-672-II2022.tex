\documentclass[12pt]{memoir}

\def\nsemestre {II}
\def\nterm {Fall}
\def\nyear {2022}
\def\nprofesor {Mark Shoemaker}
\def\nsigla {MATH672}
\def\nsiglahead {Algebraic Geometry}
\def\nextra {HW3}
\def\nlang {ENG}
\input{../../headerVarillyDiff}

\begin{document}

\begin{Ej}
  Do the following:
  \begin{enumerate}[i)]
    \itemsep=-0.4em 
    \item Let $q=(a_1,\dots,a_n)$ be a point in $\bA^n$. Using the fact that $I(q)$ is a maximal ideal in $\bC[x_1,\dots,x_n]$, prove that the coordinate ring of $q$ is isomorphic to $\bC$.
    \item If $i:\set{q}\to\bA^n$ is the inclusion map, show that the pullback homomorphism
    $$i^\3 : \bC[x_1,\dots,x_n] \to \bC[q] = \bC$$
    sends a function $f (x_1,\dots,x_n)$ to the complex number $f (a_1,\dots,a_n)$ obtained by evaluating at that point.
  \end{enumerate}
   
\end{Ej}

\begin{ptcbr}
  \begin{enumerate}[i)]
    \itemsep=-0.4em 
  \item The ideal $I(q)$ is in fact $\gen(x_1-a_1,\dots,x_n-a_n)$, a maximal ideal in $\bC[x_1,\dots,x_n]$. Then the coordinate ring of $\set{q}$ is precisely
  $$\bC[q]=\quot{\bC[x_1,\dots,x_n]}{I(\set{q})}=\quot{\bC[x_1,\dots,x_n]}{\gen(x_1-a_1,\dots,x_n-a_n)}.$$
  The evaluation homomorphism $\eps_{q}$ with help of the $1^{\text{{st}}}$ isomorphism theorem gives us the desired isomorphism. This is clearly a surjective map since we can get to any complex number by solving a linear equation and its kernel is the aforementioned ideal.
  \item Since the inclusion mapping is a morphism of algebraic varieties, then it induces a pullback homomorphism between the coordinate rings. By definition its action is as follows:
  $$i^\3:\bC[\bA^n]\to\bC[q],\ g\mapsto g\circ i.$$
  Let us unpack the terminology. First, the inclusion homomorphism is the identity mapping restricted to $\set{q}$. Then the pullback can be expressed as 
  $$i^\3:\bC[x_1,\dots,x_n]\to\bC,\ g(\vec{z})\mapsto g(\id\hspace{-0.3em}\mid\hspace{-0.3em}_{\set{q}}(\vec{z})).$$
  In this sense the action of $g\circ i$ is 
  $$\bA^n\xrightarrow[\vec{z}\mapsto q]{\id\hspace{-0.3em}\mid\hspace{-0.3em}_{\set{q}}}\bA^n\xrightarrow[g\mapsto g(-)]{g}\bC\To\bA^n\xrightarrow[g\mapsto g(q)]{g\circ i}\bC,$$
  and thus, since the action of this map is the same as $\eps_q$, we conclude that $i^\3=\eps_q$.
  \end{enumerate}
\end{ptcbr}

\begin{Ej}
  Prove that if $F : V \to W$ is an isomorphism of affine algebraic varieties, then the pullback homomorphism is a ring isomorphism.
\end{Ej}

\begin{ptcbr}
  The pullback homomorphism is precisely $F^\3:\bC[W]\to\bC[V]$ such that $g\mapsto g\circ F$. Since the pullback is already a ring homomorphism, it suffices to show that it is invertible by explicitly constructing an inverse.\par 
  The map $F^{-1}:W\to V$ defines a pullback from $\bC[V]$ to $\bC[W]$ such that $(F^{-1})^\3(h)=h\circ F^{-1}$. Now this last map is an inverse to $F^{\3}$ since 
  $$F^\3((F^{-1})^\3(h))=F^\3(h\circ F^{-1})=(h\circ F^{-1})\circ F=h,$$
  and likewise on the other side. It follows that $(F^{-1})^\3=(F^\3)^{-1}$.
\end{ptcbr}

\begin{Ej}
  Let $V\subseteq \bA^n$, $W\subseteq \bA^m$ be affine algebraic varieties. Let $\widetilde{F}:\bA^n\to\bA^m$ be a morphism. Show that 
  $$\widetilde{F}(V)\subseteq W\iff \widetilde{F}^\3:\bC[y_1,\dots,y_m]\to\bC[x_1,\dots,x_n]\ \text{sends } I(W)\ \text{to } I(V).$$
  \hint{W=V(I(W)).}
\end{Ej}

\begin{ptcbr}
  \begin{enumerate}[i)]
    \itemsep=-0.4em
    \item Let us consider $q\in\widetilde{F}^\3(I(W))$, we want to show that $q(\vec a)=0$ for $\vec a\in V$.\par 
    By definition of direct image, there exists a polynomial $p\in I(W)$ such that $\widetilde{F}^\3(p)=q$, thus if $\vec a\in V$
    $$q(\vec{a})=\widetilde{F}^\3(p)(\vec a)=p(F(\vec a))=0$$
    where the last equality follows from the fact that $F(\vec a)\in W$ and $p(\vec x)=0$ for any $\vec x\in W$. 
    \item On the other hand, let us consider a $\vec b\in\widetilde{F}(V)$. We want to show that $\vec b\in W$. By the Nullstellensatz, this is equivalent to showing that $\vec b\in V(I(W))$. Or equivalently, that $q(\vec b)=0$ for $q\in I(W)$.\par 
    Once again, using the definition of the direct image, we may find an element $\vec a\in V$ such that $\widetilde{F}(\vec a)=\vec b$. Thus if we take $q\in I(W)$, then
    $$q(\vec b)=q(\widetilde{F}(\vec{a}))=(q\circ\widetilde{F})(\vec{a})=(\widetilde{F}^\3(q))(\vec{a})=0$$
    where the last equality follows from the fact that $\widetilde{F}^\3(q)$ is a polynomial in $I(V)$.
  \end{enumerate}
\end{ptcbr}

I wanted to answer the last exercise directly without separating the directions in the equivalence. However the proof would go along the lines of 
$$\widetilde F(V)\subseteq W \iff I(F(V))\supseteq I(W)\iff I(V)\supseteq \widetilde{F}^\3(I(W))$$
using $I$ as a contravariant functor. I feel that this diagram comes in handy:
\begin{figure}[h]
\centering
  % https://tikzcd.yichuanshen.de/#N4Igdg9gJgpgziAXAbVABwnAlgFyxMJZABgBoBGAXVJADcBDAGwFcYkQA1EAX1PU1z5CKAEykR1Ok1bsAOrIC29HAAsARgDNgAQW4A9Qr37Y8BIgFYKkhizaIQ8pas079CnnxAYTQogGZSYmtpOxAAdQ9jQTMUAIkaGxl7R2V1DQACAEkACjCASkivAVNhZDEAFmDbOUVUzXSAYWQAD0pC72jSyz8qpJB6Afbi3xQyHoSQmqc04EzubI4C7kkYKABzeCJQDQAnCHdEMhAcCCRyCer7LELd-bOaE6QxKUuHWQB3LFg8RlhgADFuDc9gcAsdTohnox6GoYIwAArDGIgRgwDQ4EAXPrXIwgW4HI6PRB+XH4pBgonlUkg+7gpDmal3SEPCEANkZBJZSAA7BykKyuYhLC8+vIfrB0v89PI4Cp6Ds0MCmeVBezPGTENzBQz1TTECq6ZrltwgA
\begin{tikzcd}
  &  &  & W \arrow[rrd, "i"] \arrow[dd, "I"] & &\\
V \arrow[rrd, "i"] \arrow[rrru] \arrow[dd, "I"] & & & & & \bA^m \arrow[dd, "I"] \\
  &  & \bA^n \arrow[rrru, "\widetilde{F}"] \arrow[dd, "I"] & I(W) \arrow[rrd, "i"] \arrow[llld] & & \\
{I}(V) \arrow[rrd, "i"] & & & & & {\bC[y_1,\dots,y_m]} \arrow[llld, "\tilde F^\sharp"] \\
  &  & {\bC[x_1,\dots,x_n]} & & & 
\end{tikzcd}
\end{figure}
My question is, can this statement be proven in a direct way? Using the equivalence of categories maybe?
\begin{Ej}[2.6.1]
  Prove that the spectrum $\Spec(R)$ of a commutative ring $R$ can be given the structure of a topological space whose closed sets are of the form $V(I)=\set{\lie p\in\Spec(R):\ \lie p\supseteq I}$ for $I\rteq R$.
\end{Ej}

\begin{ptcbr}
  First, the whole set and the empty set are closed:
  \begin{itemize}
    \itemsep=-0.4em
    \item Since every prime ideal $\lie p$ is an ideal, we get $0\in\lie p$ and then $\set{0}\subseteq \lie p$. Then $V(0)=\set{\lie p\in\Spec(R):\ \lie p\supseteq \set 0}$, and this is the whole set. Thus $V(0)=\Spec(R)$.
    \item On the other hand, if $V(I)=\emptyset$, then, there's no prime ideal which contains $I$ besides the whole ring. It follows that $V(R)=\emptyset$.
  \end{itemize}
  Let us now prove that the finite union of closed sets is closed. This is, we are looking for $K\rteq R$ such that $V(I)\cup V(J)=V(K)$. Let us take $K=IJ$, then 
  $$IJ\leq I,J\To V(I)\cup V(J)\subseteq V(IJ),\ (\lie p\supseteq IJ)\To[(\lie p\supseteq I)\lor(\lie p\supseteq J)]$$
  and the last assertion gives us the other inclusion. We conclude that $V(IJ)=V(I)\cup V(J)$ and by induction we can prove the equality for a countable number of ideals.\par 
  Finally, consider a collection of ideals $(I_\al)_{\al\in\cA}$. We want to find an ideal $J\rteq R$ such that $\bigcap_{\al\in\cA}V(I_\al)=V(J)$. For that effect we shall take $J=\sum_{\al\in\cA}I_\al$.\par
  Since $J$ is the smallest ideal which contains all of the $I_\al$'s, then 
  $$\forall\al\left(V(J)\subseteq V(I_\al)\right)\To V(J)\subseteq \bigcap_{\al\in\cA}V(I_\al).$$
  On the other hand, by minimality of $J$, 
  $$\forall\al(\lie p\supseteq I_\al)\To\lie p\supseteq \sum I_\al$$
  and this guarantees the other side of the inclusion.\par 
  We conclude that in fact the Zariski topology defined on $\Spec(R)$ is in fact a topology.
\end{ptcbr}

\begin{Ej}[2.5.(1,2)]
  Do the following:
  \begin{enumerate}[i)]
    \itemsep=-0.4em
    \item Show that the pullback $\bC[W]\xrightarrow{F^\3}\bC[V]$ is injective if and only if $F$ is \emph{dominant}. This is, $F(V)$ is dense in $W$.
    \item Show that the pullback $\bC[W]\xrightarrow{F^\3}\bC[V]$ is surjective if and only if $F$ defines an isomorphism between $V$ and some algebraic subvariety of $W$.
  \end{enumerate}
 \end{Ej}
%https://math.stackexchange.com/questions/3452958/pullback-of-f-is-injective-if-and-only-if-the-image-of-f-is-dense-proof-check?rq=1
 \begin{ptcbr}
  \begin{enumerate}[i)]
    \itemsep=-0.4em
    \item Recall $F$ is dense in $E$ whenever $\ov F=E$. The Zariski closure operator is $V(I(\.))$, so it is equivalent to show that $F^\3$ is injective if and only if $V(I(F(V)))=W$. 
    \item (Hmmm?)
  \end{enumerate}
  I must admit that this time I've got to optimize since there's 161 exam on Thursday and I haven't started my Combinatorics homework due Friday. Today is (202209131754).
 \end{ptcbr}
\end{document} 