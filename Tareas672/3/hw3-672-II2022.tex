\documentclass[12pt]{memoir}

\def\nsemestre {II}
\def\nterm {Fall}
\def\nyear {2022}
\def\nprofesor {Mark Shoemaker}
\def\nsigla {MATH672}
\def\nsiglahead {Algebraic Geometry}
\def\nextra {HW2}
\def\nlang {ENG}
\input{../../headerVarillyDiff}

\begin{document}

\begin{Ej}[2.1.1]
  Suppose $R$ is a ring, prove the following:
  \vspace{-0.4em}
  \begin{enumerate}[i)]
    \itemsep=-0.4em
    \item Every maximal ideal is prime.
    \item Every prime ideal is radical.
    \item If $I\rteq R$, then $\sqrt{I}\rteq R$.
  \end{enumerate}
\end{Ej}

\begin{ptcbr}
\begin{enumerate}[i)]
  \itemsep=-0.4em
    \item Let $\lie m\rteq R$, maximal, and $r,s\in R$ with $rs\in\lie m$. Aiming for a contradiction let us suppose that neither $r$ nor $s$ lie inside of $\lie m$.\par 
    Since $r\not\in\lie m$, then the ideal generated by $r$ and $\lie m$ is the whole ring $R$. This means that $ar+m_1=1$ for $a\in R$ and $m_1\in\lie m$. Likewise for some $b\in R$ and $m_2\in\lie m$ it follows that $bs+m_2=1$. Now 
    $$1=(ar+m_1)(bs+m_2)=(ab)rs+arm_2+bsm_1+m_1m_2$$
    and this last expression is a combination of elements in $\lie m$. It follows that $1\in\lie m$, but this means that $R=\lie m$. This is impossible, so it must follow that $r\in\lie m$ or $s\in\lie m$.
    \item Suppose $\lie p\rteq R$ is prime. We want to show that for any $p\in\sqrt{\lie p}$, $p\in\lie p$. As $p\in\sqrt{\lie p}$, then $p^n\in\lie p$. We will prove that $p^n\in\lie p\To p\in\lie p$.\par 
    By induction, our base case is $n=1$, but there is nothing to prove there. So let us suppose that $p^{n+1}\in\lie p$. Then 
    $$p^{n+1}=pp^n\in\lie p\To (p\in\lie p)\lor(p^n\in\lie p).$$
    If the first statement holds, we are done. If the second one holds, we are done by induction hypothesis.
    \item First, let us show that the radical is non-empty. Since $0\in I$, then $0=0^n\in I$ and it follows that $0\in I$.\par 
    Suppose now that $x,y\in\sqrt{I}$, we will show that $x+y, xy\in\sqrt{I}$, thus proving that $\sqrt I$ is a subring. Our hypothesis tells us that $x^m,y^n\in I$ for some $m,n\in\bN$. In this case we have that
    $$(x+y)^{m+n}=\sum_{k=0}^{m+n}\binom{m+n}{k}x^ky^{m+n-k},$$
    and $x^ky^{m+n-k}\in\sqrt{I}$ because 
    $$k<n\iff k+(m+n)<n+(m+n)\iff (m+n)<m+2n-k$$
    which means that in case that one of our elements is not inside, then the other one will surely be. \par 
    Now $(x^m)^n,(y^n)^m\in I$ and then $(xy)^{mn}\in I$ and so $xy\in\sqrt{I}$.\par 
    Finally if $r\in R$ and $x\in\sqrt{I}$, then $rx^m\in I$. It follows that $(rx)^m\in I$ and so $rx\in\sqrt I$, proving that $\sqrt{I}$ is absorbent. We conclude that $\sqrt{I}$ is an ideal of $R$ whenever $I$ is.
\end{enumerate}
\end{ptcbr}

\begin{Ej}[2.1.2]
  Suppose $R$ is a ring, prove the following:
  \vspace{-0.4em}
  \begin{enumerate}[i)]
    \itemsep=-0.4em
    \item $\lie m$ is maximal $\iff\ \quot{R}{\lie m}$ is a field. 
    \item $\lie p$ is prime $\iff\ \quot{R}{\lie p}$ is an integral domain. 
  \end{enumerate}
\end{Ej}
%https://math.stackexchange.com/questions/147047/shorter-proof-of-r-i-is-a-field-if-and-only-if-i-is-maximal
%https://proofwiki.org/wiki/Prime_Ideal_iff_Quotient_Ring_is_Integral_Domain
\begin{ptcbr}
  \begin{enumerate}[i)]
    \itemsep=-0.4em
    \item $(\To)$ If $\lie m$ is a proper maximal ideal, take $r\in R\less\lie m$. Then the ideal generated by $r$ and $\lie m$ is the whole ring $R$. It follows that for some $a\in R$ and $m\in\lie m$, $ar+m=1$. If we translate this expression to the quotient ring we obtain
    $$ar+m\equiv 1\bmod \lie m\To ar\equiv 1\bmod\lie m.$$ %%CHANGE BMOD TO MOD
    Since $r$ was arbitrary, we have found an inverse $a$ for any element $r$ inside the quotient ring. Since $\quot{R}{\lie m}$ is already a commutative ring with identity, and now we have inverses, it follows that $\quot{R}{\lie m}$ is a field.\par 
    $(\Leftarrow)$ On the other hand suppose $\quot{R}{\lie m}$ is a field. Let $I$ be an ideal of $R$ which properly contains $\lie m$. If $r\in I\less\lie m$, then there exists $s\in R$ such that $rs\equiv 1\bmod\lie m$. This means that 
    $$rs-1\in\lie m\subsetneq I\To rs-(rs-1)\in I\To 1\in I\To I=R.$$
    Since $I$ is arbitrary, it follows that no proper ideal besides the whole ring contains $\lie m$. This means that $\lie m$ is maximal.
    \item $(\To)$ If $\lie p$ is a prime ideal, suppose $rs\equiv 0\bmod\lie p$. This means that $rs\in\lie p$, from which follows that $r\in\lie p$ or $s\in\lie p$ because $\lie p$ is a prime ideal. In any case this means that
    $$r\equiv 0\bmod\lie p\quad\lor s\equiv 0\bmod\lie p$$
    and therefore $\quot{R}{\lie p}$ has no zero divisors.\par
    $(\Leftarrow)$ The opposite direction is quite similar. Consider $rs\in\lie p$ for some $r,s\in R$. Then 
    $$rs\in\lie p\To rs\equiv 0\bmod\lie p\To (r\equiv 0\bmod\lie p)\lor(s\equiv 0\bmod\lie p)\To(r\in\lie p)\lor(s\in\lie p)$$
    where the second implication follows from the fact that $\quot{R}{\lie p}$ has no zero divisors. We conclude that $\lie p$ is prime.
  \end{enumerate}
\end{ptcbr}

\begin{Ej}
  Suppose $f:R\to S$ is a ring homomorphism and $\lie q\rteq S$ is a prime ideal. Show that $f^{-1}[\lie q]\rteq R$ is a prime ideal.
\end{Ej}
%https://yutsumura.com/the-inverse-image-of-an-ideal-by-a-ring-homomorphism-is-an-ideal/

\begin{ptcbr}
  We will first show that ring homomorphisms take ideals back into ideals, and then that they take primes back into primes.\par 
  Suppose $r_1,r_2\in f^{-1}[\lie q]$. We want to see that this an absorbent subring. Our hypothesis tells us that 
  \begin{align*}
    f(r_1),f(r_2)\in\lie q\To&f(r_1)+f(r_2),f(r_1)f(r_2)\in\lie q\\
    \To&f(r_1+r_2),f(r_1r_2)\in\lie q\\
    \To&r_1+r_2,r_1r_2\in f^{-1}[\lie q].
  \end{align*}
  This lets us conclude that $f^{-1}[\lie q]$ is a subring of $R$. To prove it is absorbent, suppose that $r\in R$ and $p\in f^{-1}[\lie q]$. This means that $f(r)\in S$ and $f(p)\in\lie q$, and since $\lie q$ is a prime ideal in $S$, it follows that $f(r)f(p)\in\lie q$. We conclude that $rp\in f^{-1}[\lie q]$, and thus, this set is an ideal.\par 
  Let us now consider $r_1,r_2$ as before, but now with the hypothesis that $r_1r_2\in f^{-1}[\lie q]$. This means that 
  \begin{align*}
    f(r_1)f(r_2)=f(r_1r_2)\in\lie q&\To (f(r_1)\in\lie q)\lor(f(r_2)\in\lie q)\\
    &\To(r_1\in f^{-1}[\lie q])\lor(r_1\in f^{-1}[\lie q]).
  \end{align*}
  We conclude that $f^{-1}[\lie q]$ is also prime. 
\end{ptcbr}
\newpage
\begin{Ej}[2.3.3]
  Show that if $I\rteq\bC[\vec{x}]$ is radical, then $I=\hspace{-1em}\displaystyle\bigcap_{\substack{\lie m\supseteq I\\
  \lie m\ \text{maximal}}}\hspace{-1em}\lie m$.
\end{Ej}
%https://math.stackexchange.com/questions/732907/intersection-of-all-maximal-ideals-containing-a-given-ideal
\begin{ptcbr}
 We have the following:
  $$\hspace{-1em}\displaystyle\bigcap_{\substack{\lie m\supseteq I\\
  \lie m\ \text{maximal}}}\hspace{-1em}\lie m
  %=\hspace{-1em}\displaystyle\bigcap_{\substack{\lie m\supseteq I\\\lie m\ \text{maximal}}}\hspace{-1em}\sqrt{\lie m}
  =\sqrt{\hspace{-0.1em}\displaystyle\bigcap_{\substack{\lie m\supseteq I\\
  \lie m\ \text{maximal}}}\hspace{-1em}\lie m},$$
  the left-to-right inclusion is immediate because $I\subseteq\sqrt{I}$. On the other hand
  $$x\in\sqrt{\bigcap_{\al\in\cA}\lie m_\al}\To x^n\in \bigcap_{\al\in\cA}\lie m_\al\To\forall \al(x^n\in\lie m_\al)\To\forall \al(x\in\lie m_\al)\To x\in\bigcap_{\al\in\cA}\lie m_\al.$$
  Now by the Nullstellensatz,
  $$\sqrt{\hspace{-0.1em}\displaystyle\bigcap_{\substack{\lie m\supseteq I\\
  \lie m\ \text{maximal}}}\hspace{-1em}\lie m}=I\left(V\left(\hspace{-0.1em}\displaystyle\bigcap_{\substack{\lie m\supseteq I\\
  \lie m\ \text{maximal}}}\hspace{-1em}\lie m\right)\right),$$
  %$$I\left(V\left(\hspace{-0.1em}\displaystyle\bigcap_{\substack{\lie m\supseteq I\\\lie m\ \text{maximal}}}\hspace{-1em}\lie m\right)\right)=I\left(\hspace{-0.1em}\bigcup_{\substack{\lie m\supseteq I\\ \lie m\ \text{maximal}}}\hspace{-1em}V(\lie m)\right)$$ 
  and we can assume that the maximal ideals are all distinct. Otherwise the intersection would count repetitions or the whole ring which doesn't affect the result.\par 
  This means that these ideals are comaximal, and \un{the intersection of comaximal ideals is equal to their product}. This means that 
  $$V\left(\hspace{-0.1em}\bigcap_{\substack{\lie m\supseteq I\\
  \lie m\ \text{maximal}}}\hspace{-1em}\lie m\right)=V\left(\hspace{-0.1em}\prod_{\substack{\lie m\supseteq I\\
  \lie m\ \text{maximal}}}\hspace{-1em}\lie m\right)=\hspace{-0.1em}\bigcup_{\substack{\lie m\supseteq I\\
  \lie m\ \text{maximal}}}\hspace{-1em}V(\lie m)$$
  %\footnote{In the case we are using a countable $\cA$, the result is immediate by induction. If not, $\cA$ can be well ordered and then by transfinite induction the result is true.}.
  %https://math.stackexchange.com/questions/3258486/are-index-sets-always-well-ordered
  %https://en.wikipedia.org/wiki/Transfinite_induction
  where in the last step we used the fact that $V(\prod)=\bigcup V$. Now, geometrically, \un{the variety associated to a maximal ideal corresponds to a point}, it follows that 
  $$\hspace{-0.1em}\bigcup_{\substack{\lie m\supseteq I\\
  \lie m\ \text{maximal}}}\hspace{-1em}V(\lie m)=\bigcup_{\vec{a}\in V(I)}\set{\vec a}=V(I).$$
  Finally taking the ideal of functions which annihilate on this set, we get 
  $$I[V(I)]=\sqrt{I}=I.$$
  Once again in the second-to-last equality we have applied the Nullstellensatz. We conclude our desired equality $\hspace{-1em}\displaystyle\bigcap_{\substack{\lie m\supseteq I\\
  \lie m\ \text{maximal}}}\hspace{-1em}\lie m=I$.
\end{ptcbr}

\begin{Ej}
  Prove that the coordinate ring of an affine algebraic variety is:
  \vspace*{-0.4em}
  \begin{enumerate}[i)]
    \itemsep=-0.4em
    \item reduced;
    \item fin. gen. as $\bC$-algebra;
    \item Noetherian.
  \end{enumerate}
\end{Ej}
%https://math.stackexchange.com/questions/2827266/if-not-exists-ri-nilpotent-then-r-i-is-an-integral-domain


\begin{Lem}[2.1.5]
  A quotient ring $\quot{R}{I}$ is reduced if and only if $I$ is radical.
  \end{Lem}

\begin{ptcbp}
\begin{enumerate}
  \item[$(\To)$] Suppose $\quot{R}{I}$ is reduced and let us take an $r\in R$ such that $r^n\in I$ for some $n$. Inside the quotient we have 
  $$r^n\equiv 0\bmod I\To r\equiv 0\bmod I$$
  because inside the quotient there are no nilpotent elements besides zero itself. The last equivalence means that $r\in I$ and so $I$ is radical.
  \item[$(\Leftarrow)$] Now let us take a nilpotent element in the quotient with the assumption that $I$ is radical. Suppose $r$ is such element, this means that 
  $$r^n\equiv 0\bmod I$$
  and so $r^n\in I$. But, as $I$ is a radical ideal, it follows that $r\in I$. In the terms of the quotient this means that $r\equiv 0\bmod I$ and so, $\quot{R}{I}$ has no nilpotent elements besides zero.
\end{enumerate}
\end{ptcbp}

\begin{ptcbr}
  Let our variety be $V$. Any algebraic variety can be seen as the zero locus of an ideal $I$ of polynomials. Therefore, $V=V(I_0)$ where $I_0\rteq\bC[\vec{x}]$.
\vspace*{-0.4em}
  \begin{enumerate}[i)]
    \itemsep=-0.4em
    \item The coordinate ring of $V$ is $\bC[V]:=\quot{\bC[\vec{x}]}{I(V)}$ and $I(V)=I(V(I_0))=\sqrt{I_0}$ by the Nullstellensatz. The radical of $I_0$ is a radical ideal and by the last exercise, it follows that $\bC[V]$ is reduced, as it the quotient of a ring by a radical ideal.
    \item The set $\set{x_1\bmod\sqrt{I_0},x_2\bmod\sqrt{I_0},\dots,x_n\bmod\sqrt{I_0}}$ generates the coordinate ring as a $\bC$-algebra. 
    \item Since $\bC[\vec{x}]$ is already Noetherian by Hilbert's Basis Theorem, it follows that a quotient of this ring by an ideal must be Noetherian.
  \end{enumerate}
\end{ptcbr}

\begin{Ej}[2.3.4]
  Prove that the Zariski topology on an affine algebraic variety is compact: Every open cover has a finite subcover.
\end{Ej}
%https://math.stackexchange.com/questions/1938690/why-is-a-zariski-closed-set-compact-under-the-zariski-topology/1940667

To prove this result we will use an auxiliary definition.

\begin{Def}
A topological space $X$ is Noetherian if the descending chain condition holds for closed sets of $X$. This is, if every descending chain 
$$X\supseteq F_1\supseteq F_2\supseteq\dots$$
becomes stationary.
\end{Def}

\begin{Prop}
  Every subspace of a Noetherian space $X$ is Noetherian.
  \end{Prop}

\begin{ptcbp}
  If $Y$ is a subspace, a closed set $F_i$ can be realized as $Y\cap\ov{F}_i$ where the closure is over $X$. Thus a descending chain can be seen as a descending chain of $\ov F_i$'s over $X$, which is already Noetherian. Thus $
  Y$ is Noetherian.
\end{ptcbp}
\begin{Prop}
Every Noetherian space $X$ is (quasi-)compact\footnote{Just a comment, I'm assuming ``quasi-compact'' is the finite subcover condition while ``compact'' is quasi-compactness + Hausdorff condition.}.
\end{Prop}

\begin{ptcbp}
  Suppose $(U_\al)$ is an open cover for $X$. Let $\cU$ be the collection of finite unions of elements in $(U_\al)$. Taking complements, we can form a chain of closed subsets which will have a minimal element as our space is Noetherian. Once again taking complements, that element becomes the maximal element of $(U_\al)$. If we name that element $U$, then it must happen that $U=X$, else, there would be a $U_\al$ such that $U\supsetneq U\cup U_\al\in\cU$ contradicting $U$'s maximality. It follows that $(U_\al)$ has a finite subcovering.
\end{ptcbp}

\begin{ptcbr}
  Any descending chain of closed subsets in $\bA^n$ corresponds to an ascending chain of radical ideals in $\bC[\vec{x}]$ which is Noetherian by Hilbert's Basis Theorem. It follows that $\bA^n$ is a Noetherian space and therefore compact.\par 
  By the first proposition, it follows that any variety is Noetherian. By the second proposition, any variety is compact.
\end{ptcbr}
\end{document} 