\documentclass[12pt]{memoir}

\def\nsemestre {II}
\def\nterm {Fall}
\def\nyear {2022}
\def\nprofesor {Mark Shoemaker}
\def\nsigla {MATH672}
\def\nsiglahead {Algebraic Geometry}
\def\nextra {HW3}
\def\nlang {ENG}
\input{../../headerVarillyDiff}

\begin{document}

\begin{Ej}
  Do the following:
  \begin{enumerate}[i)]
    \itemsep=-0.4em 
    \item Let $q=(a_1,\dots,a_n)$ be a point in $\bA^n$. Using the fact that $I(q)$ is a maximal ideal in $\bC[x_1,\dots,x_n]$, prove that the coordinate ring of $q$ is isomorphic to $\bC$.
    \item If $i:\set{q}\to\bA^n$ is the inclusion map, show that the pullback homomorphism
    $$i^\3 : \bC[x_1,\dots,x_n] \to \bC[q] = \bC$$
    sends a function $f (x_1,\dots,x_n)$ to the complex number $f (a_1,\dots,a_n)$ obtained by evaluating at that point.
  \end{enumerate}
   
\end{Ej}

\begin{ptcbr}
  \begin{enumerate}[i)]
    \itemsep=-0.4em 
  \item The ideal $I(q)$ is in fact $\gen(x_1-a_1,\dots,x_n-a_n)$, a maximal ideal in $\bC[x_1,\dots,x_n]$. Then the coordinate ring of $\set{q}$ is precisely
  $$\bC[q]=\quot{\bC[x_1,\dots,x_n]}{I(\set{q})}=\quot{\bC[x_1,\dots,x_n]}{\gen(x_1-a_1,\dots,x_n-a_n)}.$$
  The evaluation homomorphism $\eps_{q}$ with help of the $1^{\text{{st}}}$ isomorphism theorem gives us the desired isomorphism. This is clearly a surjective map since we can get to any complex number by solving a linear equation and its kernel is the aforementioned ideal.
  \item Since the inclusion mapping is a morphism of algebraic varieties, then it induces a pullback homomorphism between the coordinate rings. By definition its action is as follows:
  $$i^\3:\bC[\bA^n]\to\bC[q],\ g\mapsto g\circ i.$$
  Let us unpack the terminology. First, the inclusion homomorphism is the identity mapping restricted to $\set{q}$. Then the pullback can be expressed as 
  $$i^\3:\bC[x_1,\dots,x_n]\to\bC,\ g(\vec{z})\mapsto g(\id\hspace{-0.3em}\mid\hspace{-0.3em}_{\set{q}}(\vec{z})).$$
  In this sense the action of $g\circ i$ is 
  $$\bA^n\xrightarrow[\vec{z}\mapsto q]{\id\hspace{-0.3em}\mid\hspace{-0.3em}_{\set{q}}}\bA^n\xrightarrow[g\mapsto g(-)]{g}\bC\To\bA^n\xrightarrow[g\mapsto g(q)]{g\circ i}\bC,$$
  and thus, since the action of this map is the same as $\eps_q$, we conclude that $i^\3=\eps_q$.
  \end{enumerate}
\end{ptcbr}

\begin{Ej}
  Prove that if $F : V \to W$ is an isomorphism of affine algebraic varieties, then the pullback homomorphism is a ring isomorphism.
\end{Ej}

\begin{ptcbr}
  The pullback homomorphism is precisely $F$
\end{ptcbr}

\begin{Ej}
  Let $V\subseteq \bA^n$, $W\subseteq \bA^m$ be affine algebraic varieties. Let $\tilde{F}:\bA^n\to\bA^m$ be a morphism. Show that 
  $$\tilde{F}(V)\subseteq W\iff \tilde{F}^\3:\bC[y_1,\dots,y_m]\to\bC[x_1,\dots,x_n]\ \text{sends } I(W)\ \text{to } I(V).$$
  \hint{W=V(I(W)).}
\end{Ej}

\begin{Ej}[2.6.1]
  Prove that $\Spec(R)$ of a commutative ring $R$ can be given the structure of a topological space whose closed sets are of the form $V(I)=\set{\lie p\in\Spec(R):\ \lie p\supseteq I}$ for $I\rteq R$.
\end{Ej}

\begin{Ej}[2.5.(1,2)]
  Do the following:
  \begin{enumerate}[i)]
    \itemsep=-0.4em
    \item Show that the pullback $\bC[W]\xrightarrow{F^\3}\bC[V]$ is injective if and only if $F$ is \emph{dominant}. This is, $F(V)$ is dense in $W$.
    \item Show that the pullback $\bC[W]\xrightarrow{F^\3}\bC[V]$ is surjective if and only if $F$ defines an isomorphism between $V$ and some algebraic subvariety of $W$.
     
  \end{enumerate}
 \end{Ej}
\end{document} 