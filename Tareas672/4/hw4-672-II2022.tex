\documentclass[12pt]{memoir}

\def\nsemestre {II}
\def\nterm {Fall}
\def\nyear {2022}
\def\nprofesor {Mark Shoemaker}
\def\nsigla {MATH672}
\def\nsiglahead {Algebraic Geometry}
\def\nextra {HW4}
\def\nlang {ENG}
\input{../../headerVarillyDiff}

\begin{document}

\begin{Ej}
 Define a \emph{line} in $\bP^2$ to be a closed subset of the form $L=\set{[x:y:z]:\ ax+by+cz=0}$ for some constants $a,b,c\in\bC$, not all zero.
 \begin{enumerate}[i)]
  \itemsep=-0.4em
  \item If $(a,b,c)=(1,0,0)$, we saw in class that $\bP^2\less L=\set{[x:y:z]:\ x\neq 0}=U_x$ could be identified with $\bC^2$.\par 
  Similarly, show that for any line $L$ there is a bijection $\bP^2\less L\isom \bC^2$.
  \item Prove that any two distinct lines $L_1$ and $L_2$ intersect in a single point.
  \item Prove that there is a unique line $L$ through any two distinct points in $\bP^2$.
 \end{enumerate}
\end{Ej}

\begin{ptcbr}
  \begin{enumerate}[i)]
    \itemsep=-0.4em
    \item constants
    \item Let $L_1,L_2\subseteq\bP^2$ be two distinct lines with direction $(a,b,c)$ and $(d,e,f)$. Since they are distinct this means that $\not\exists\la((d,e,f)=\la(a,b,c))$. A point $[x:y:z]$ in the intersection of $L_1$ and $L_2$ must satisfy the system of equations
    $$
    \left\lbrace
    \begin{aligned}
      &ax+by+cz=0,\\
      &dx+ey+fz=0.
    \end{aligned}
    \right.
    $$
    Solutions to this system of equations are parametrized in terms of $z$ in the following manner
    $$[x:y:z]=\bonj{\frac{bf-ce}{ae-bd}z:\frac{cd-af}{ae-bd}z:z},$$
    and ordinarily this would give us an infinite number of solutions. However in $\bP^2$ this corresponds to the point $[bf-ce:cd-af:ae-bd]$.
    \item Let us now consider two points $[x:y:z],[u:v:w]\in\bP^2$ which are distinct. Once again, consider a system of equations 
    $$
    \left\lbrace
    \begin{aligned}
      &ax+by+cz=0,\\
      &au+bv+cw=0.
    \end{aligned}
    \right.
    $$
    There is an infinite number of solutions to this system for $(a,b,c)\in\bC^3$.aaaaaaa
  \end{enumerate}
  \end{ptcbr}

  \begin{Ej}
    Consider the sequence $(p_n)_{n\in\bN}\subseteq\bC^3$ with $p_n=(n^3,2n^2,3n^3)$. Identifying $\bC^3$ with $\set{x_0\neq 0}\subseteq\bP^3$, what is the limit of $p_n$ as $n\to\infty$?
   \end{Ej}

  \begin{ptcbr}
    We can identify $p_n$ with the sequence $\widetilde{p}_n=[n^3:2n^2:3n^3:1]$. Now for $n\neq 0$ it holds that 
    $$\widetilde{p}_n=\bonj{1:\frac2n:3:\frac{1}{n^3}}\xrightarrow[n\to\infty]{}[1:0:3:0].$$
    This coincides with the limit of $p_n$ in the usual sense which is $\infty$ and $[1:0:3:0]$ is a point at infinity.
  \end{ptcbr}

  \begin{Ej}
    In $\bA^2$, let $V=\bV(x)$, $W=\bV(x-1)$ and $Z=\bV(y-x^2)$. Let $\ov V,\ov W$ and $\ov Z$ denote their respective \emph{projective closures} in $\bP^2$. Find the points in the intersections $\ov V\cap \ov W,\ \ov V\cap \ov Z$ and $\ov W\cap \ov Z$.
  \end{Ej}

  \begin{ptcbr}
    First, let us parametrize the varieties in question as points of $\bA^2$:
    $$
    \left\lbrace
    \begin{aligned}
      &\bV(x)=\set{x=0}=\set{(0,t):\ t\in\bC},\\
      &\bV(x-1)=\set{x=1}=\set{(1,t):\ t\in\bC},\\
      &\bV(y-x^2)=\set{y=x^2}=\set{(t,t^2):\ t\in\bC}.
    \end{aligned}
    \right.
    $$
    For each one of those sets, their projective closure corresponds to the embedding of the points inside $\bP^2$ along with their limit points. In the case of $V$ we have 
    $$\ov V=\set{[0:t:1]:\ t\in\bC}\cup\set{\text{limit points}}=\set{[0:t:1]:\ t\in\bC}\cup\set{[0:1:0]}.$$
    Likewise we have 
    $$
    \left\lbrace
    \begin{aligned}
      &\ov W=\set{[1:t:1]:\ t\in\bC}\cup\set{[0:1:0]},\\
      &\ov W=\set{[1:t:1]:\ t\in\bC}\cup\set{[0:1:0]}
    \end{aligned}
    \right.
    $$
  \end{ptcbr}
\end{document} 