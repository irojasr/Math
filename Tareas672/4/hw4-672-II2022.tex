\documentclass[12pt]{memoir}

\def\nsemestre {II}
\def\nterm {Fall}
\def\nyear {2022}
\def\nprofesor {Mark Shoemaker}
\def\nsigla {MATH672}
\def\nsiglahead {Algebraic Geometry}
\def\nextra {HW4}
\def\nlang {ENG}
\input{../../headerVarillyDiff}

\begin{document}

\begin{Ej}
 Define a \emph{line} in $\bP^2$ to be a closed subset of the form $L=\set{[x:y:z]:\ ax+by+cz=0}$ for some constants $a,b,c\in\bC$, not all zero.
 \begin{enumerate}[i)]
  \itemsep=-0.4em
  \item If $(a,b,c)=(1,0,0)$, we saw in class that $\bP^2\less L=\set{[x:y:z]:\ x\neq 0}=U_x$ could be identified with $\bC^2$.\par 
  Similarly, show that for any line $L$ there is a bijection $\bP^2\less L\isom \bC^2$.
  \item Prove that any two distinct lines $L_1$ and $L_2$ intersect in a single point.
  \item Prove that there is a unique line $L$ through any two distinct points in $\bP^2$.
 \end{enumerate}
\end{Ej}

\begin{ptcbr}
  \end{ptcbr}

  
\end{document} 