\documentclass[12pt]{memoir}

\def\nsemestre {II}
\def\nterm {Fall}
\def\nyear {2022}
\def\nprofesor {Mark Shoemaker}
\def\nsigla {MATH672}
\def\nsiglahead {Algebraic Geometry}
\def\nextra {HW1}
\def\nlang {ENG}
\input{../../headerVarillyDiff}

\begin{document}
%\begin{multicols}{2}

\begin{Ej}
  Do the following:
  \begin{enumerate}[i)]
    \item Give a simple description of the closed sets in $\bA^1$ (with respect to the Zariski
    topology).
    \item Use your previous answer to prove that $\bA^1$ is not Hausdorff.
  \end{enumerate}
\end{Ej}

\begin{ptcbr}
\begin{enumerate}[i)]
    \item If we consider $\bA^1$ over an algebraically closed field $k$ of characteristic zero then every closed set is of the form $V(I)$ where $I\in\Spec(k[x])$. Since $k[x]$ is a PID, then $I=\gen(p)$ for a polynomial $p\in k[x]$. Then $V(I)$ would be the set of roots inside $k$ of $p$. Since $p$ is arbitrary, every closed set $V(I)$ of $\bA^1$ is a finite set.\par 
    This means that the open sets are the complement of the finite sets. In essence, the Zariski topology coincides with the cofinite topology over $\bA^1$.
    \item The cofinite topology is not Hausdorff, so it follows that the Zariski topology isn't Hausdorff as well.
    %https://math.stackexchange.com/questions/906497/infinite-topological-space-with-cofinite-topology-is-not-hausdorff
\end{enumerate}
\end{ptcbr}

\begin{Ej}
  Show that the Zariski topology on $\bA^2$ is not the product topology on $\bA^1\x\bA^1$. (Hint: Consider the diagonal.)
\end{Ej}

\begin{Ej}
  Let $F:V\to W$ be a morphism of affine algebraic varieties.
 Prove that $F$ is continuous in the Zariski topology. 
\end{Ej}

\begin{ptcbr}
  Recall that a function is continuous if the inverse image of a closed set is once again a closed set.\par 
  Suppose $V_0\subseteq W$ is a closed set, we would like to see that $F^{-1}[V_0]\subseteq V$ is a closed set as well. Since $V_0$ is closed, then there exists an ideal $I$ such that $V_0=V(I)$. We can decompose $V(I)=\bigcap_{G\in I}V(G)$ and use the fact that the inverse image of an intersection is the intersection of inverse images to show our result.\par 
FINISH
\end{ptcbr}

\begin{Ej}
  Show that the twisted cubic $V$ of Figure 1.5 is isomorphic to the affine line by constructing an explicit isomorphism from $\bA^1$ to $V$. (Hint: See Exercise 1.2.3)
\end{Ej}

\begin{Ej}
 Show that if $F:X\to Y$ is a surjective morphism of affine
 algebraic varieties, then the dimension of$ X $is at least as large as the dimension of $Y$. 
\end{Ej}
%\end{multicols}
\end{document} 