\documentclass[12pt]{memoir}

\def\nsemestre {I}
\def\nterm {Spring}
\def\nyear {2023}
\def\nprofesor {Mark Shoemaker}
\def\nsigla {MATH673}
\def\nsiglahead {Algebraic Geometry}
\def\nextra {HW9}
\def\nlang {ENG}
\input{../../headerVarillyDiff}
\DeclareMathOperator{\sh}{sh}
\begin{document}

\begin{Ej}[6.2.D Vakil]
    Show that an $\cO_X$-module $\cF$ is quasicoherent if and only if for each such distinguished $\Spec(A_f)\hookto\Spec(A)$, $\al\:\cF(\Spec(A))_f\to\cF(\Spec(A_f))$ is an isomorphism.
\end{Ej}

\begin{ptcbr}
Let us assume $\cF$ is quasicoherent, this that when restricted to affine sets we have 
$$\cF\mid_{\Spec(A)}\isom\widetilde{M}\word{where}M\in\cat{Mod}_A.$$
Observe now that when restricting to $\Spec(A)$ we have
$$\cF(\Spec(A))_f=\cF\mid_{\Spec(A)}(\Spec(A))_f$$
and since $\cF$ is quasicoherent we have the following isomorphism:
$$\cF\mid_{\Spec(A)}(\Spec(A))_f\isom\widetilde{M}(\Spec(A))_f=M_f.$$
But from our original discussion on the module sheaf, we have that for the distinguished open $D(f)$ this is 
$$M_f=\widetilde{M}(D(f))\isom\cF\mid_{\Spec(A)}(D(f)).$$
We finally use the key fact that $D(f)=\Spec(A_f)$ so 
$$\cF\mid_{\Spec(A)}(D(f))=\cF\mid_{\Spec(A)}(\Spec(A_f))=\cF(\Spec(A_f))$$
and with this we arrive at the desired conclusion that both sets are isomorphic to each other.\par 
On the other hand, suppose we have the isomorphism. We wish to show that $\cF$ is quasicoherent, so we must find an isomorphism of sheaves $\vf\:\widetilde{M}\to\cF$ where $M$ is an $A$-module, such that for open sets $U\subseteq V\subseteq X$, the following diagram commutes:
\begin{center}
    % https://tikzcd.yichuanshen.de/#N4Igdg9gJgpgziAXAbVABwnAlgFyxMJZABgBpiBdUkANwEMAbAVxiRAB12B3LWPB2MACyAXwAUANQCUIEaXSZc+QigCM5KrUYs2nAMYAxSTLkLseAkTKrN9Zq0QduvGP0GixAVRPyQGc8pE6jbUdjqO+kbespowUADm8ESgAGYAThAAtkgATNQ4EEgAzKHaDk40KV4y1Ax0AEYwDAAKihYqIGlY8QAWOLK+6VlIZCAFSOpa9rrslcYDqRnZiKPjiDmmIEPLk2tFIhQiQA
\begin{tikzcd}
    \widetilde{M}(V) \arrow[r, "\vf(V)"] \arrow[d] & \cF(V) \arrow[d] \\
    \widetilde{M}(U) \arrow[r, "\vf(U)"']          & \cF(U)          
    \end{tikzcd}
\end{center}
As $\cF$ is an $\cO_X$-module\footnote{A sheaf of $\cO_X$-modules, for each open set $U$, $\cF(U)$ is an $\cO_X(U)$-module.} we may take $\cF(X)$ as the module $M$ in question. We claim that $\cF\isom\widetilde{M}$ where $M$ is the module $\cF(X)$.\par 
We may cover $X$ by distinguished opens and rewrite the previous diagram for $D(f)\subseteq D(g)$: 
\begin{center}
    % https://tikzcd.yichuanshen.de/#N4Igdg9gJgpgziAXAbVABwnAlgFyxMJZABgBpiBdUkANwEMAbAVxiRAB12B3LWPB2MACyAXwAUAETEBzAJSyQI0uky58hFAEZyVWoxZtOAYwBikmfMXKQGbHgJEym3fWatEHbrxj9Bo8wBmlkoqdupE2s7UrgYexmZSQQoiujBQ0vBEoAEAThAAtkgATNQ4EEgAzNH67p40AYGW1Ax0AEYwDAAKqvYaIDlY0gAWOFbZeYWIZCBlSNp6bobs9eZyyda5BUjTs4hFISCbk-O7FSkiQA
\begin{tikzcd}
    \widetilde{M}(D(g)) \arrow[r, "\vf(D(g))"] \arrow[d] & \cF(D(g)) \arrow[d] \\
    \widetilde{M}(D(f)) \arrow[r, "\vf(D(f))"']          & \cF(D(f))          
    \end{tikzcd}
\end{center}
However, we know that when acting on distinguished opens, the sheaf behaves like $\widetilde{M}(D(g))=M_g$. Also, recall that $D(g)=\Spec(A_g)$ so that our diagram may rewritten as 
\begin{center}
    % https://tikzcd.yichuanshen.de/#N4Igdg9gJgpgziAXAbVABwnAlgFyxMJZABgBpiBdUkANwEMAbAVxiRAB12BjAMQAoAGgEoA+gHMQAX1LpMufIRQBGclVqMWbTrz6cAymhhc+AQXFChUmSAzY8BImSVr6zVog7d+wkQDMrsnYKRCrO1K6aHtr8+obGZr4WUmowUGLwRKC+AE4QALZIAEzUOBBIAMzhGu6eNL58ACJ8iZbUDHQARjAMAApy9oog2VhiABY4ASA5+UhkIKVIKupuWux1jXxiSdJZuQWIcwuIhTtTe4slZYjlkhSSQA
\begin{tikzcd}
    \cF(X)_g \arrow[r, "\vf(D(g))"] \arrow[d] & \cF(\Spec(A_g)) \arrow[d] \\
    \cF(X)_f \arrow[r, "\vf(D(f))"']          & \cF(\Spec(A_f))          
    \end{tikzcd}
\end{center}
If we additionally assume that $X$ is an affine scheme, we have 
\begin{center}
    % https://tikzcd.yichuanshen.de/#N4Igdg9gJgpgziAXAbVABwnAlgFyxMJZABgBpiBdUkANwEMAbAVxiRAB12BjAMQApOAZTQwufAIIBKSQH0A5iAC+pdJlz5CKAIzkqtRizadeA9sNET50pSpAZseAkTJa99Zq0Qdu-ISLFSsgBmNqoOGkQ6rtTuhl7Gvmb+lkHWinowUHLwRKBBAE4QALZIAEzUOBBIAMwxBp7eNEF8ACJ8qZIg1Ax0AEYwDAAKao6aIPlYcgAWOKEgBcVIZCCVSDr6HkbsTa18cmm2CyWIy6uIpcp5hcfrZ9XpikA
\begin{tikzcd}
    \cF(\Spec(A))_g \arrow[r, "\vf(D(g))"] \arrow[d] & \cF(\Spec(A_g)) \arrow[d] \\
    \cF(\Spec(A))_f \arrow[r, "\vf(D(f))"']          & \cF(\Spec(A_f))          
    \end{tikzcd}
\end{center}
and by the hypothesis, both columns of this diagram are isomorphic and those isomorphisms are compatible with the restriction $D(f)\subseteq D(g)$.\par 
If $X$ were not affine, we could cover by affine sets and glue together the morphisms? \red{I'm actually not sure, would it be possible to work at the level of stalks?}
\end{ptcbr}

\begin{Ej}[6.3.A Vakil]
    Show that a sequence of quasicoherent sheaves $\cF\to\cG\to\cH$
    on $X$ is exact if and only if it is exact on every open set in any given affine cover of X.\par 
    (In particular, taking sections over an affine open Spec A is an exact functor from the category of quasicoherent sheaves on X to the category of A-modules. Recall that taking sections is only left-exact in general, see section 2.6.F.)\par
    Thus we may check injectivity or surjectivity of a morphism of quasicoherent sheaves by checking on an affine cover of our choice.
\end{Ej}

\begin{ptcbr}
    Suppose that our sequence is exact for the covering of $X=\bigcup D(f_i)$. As the sheaves are quasicoherent we have that 
    $$\cF(\Spec(A_f))\isom\cF(\Spec(A))_f$$
    so our sequence being exact is equivalent to the exactness of 
    $$\cF(\Spec(A))_f\to\cG(\Spec(A))_f\to\cH(\Spec(A))_f.$$
    From here we could state the localization is exact or we can rip it into $-\ox_AA_f$ and then the tensor product is an exact functor. From this we see the equivalence with the exactness of 
    $$\cF(\Spec(A))\to\cG(\Spec(A))\to\cH(\Spec(A))$$ 
    \red{I'd love to claim that this sequence being exact means that the big sequence is exact. Also I can't claim the same from the other side.}
\end{ptcbr}

\begin{Ej}[6.4.B Vakil]
    Do the following:
    \begin{enumerate}[i)]
        \itemsep=-0.4em
        \item If $f\in A$ and $M$ is a finitely presented $A$-module, show that $M_f$ is a finitely presented $A_f$-module.
        \item If $\genr{f_1,\dots,f_n}=A$ and $M_{f_i}$ is a finitely presented $A_{f_i}$-module for all $i$, show that $M$ is a finitely presented $A$-module. \hint{Show first that $M$ is finitely genrated so that we may write $M=\coker(\al)$ with $\al\: N\hookto A^{\oplus n}$ for some submodule $N\subseteq A^{\oplus n}$. We wish to show that $N$ is finitely generateed. Localize at $f_i$ and show $N_{f_i}$ is $A_{f_i}$-finitely generated then apply the previous exercise to $N$.}
    \end{enumerate}
\end{Ej}

\begin{ptcbr}
    \begin{enumerate}
        \itemsep=-0.4em
        \item If $M$ is finitely presented, then we have an exact sequence 
        $$A^q\to A^p\to M\to 0$$
        As localization is an exact functor we have that the sequence 
        $$A^q_f\to A^p_f\to M_f\to 0$$
        is also an exact sequence. This means that $M_f$ is also finitely presented.
        \item We have the exact sequence for all $i$:
        $$A^q_{f_i}\to A^p_{f_i}\to M_{f_i}\to 0$$
        so in particular $A^p_{f_i}\to M_{f_i}\to 0$ is exact. From this, using the fact that finite generation satisfies the Affine Communication lemma we have that $M$ is a finitely generated $A$-module.\par 
        As suggested, we may see $M\isom\coker(\al)$ for $\al\: N\hookto A^n$. Localizing at each $f_i$ we obtain $N_{f_i}$
    \end{enumerate}
\end{ptcbr}
\end{document} 