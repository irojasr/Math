\documentclass[12pt]{memoir}

\def\nsemestre {I}
\def\nterm {Spring}
\def\nyear {2023}
\def\nprofesor {Mark Shoemaker}
\def\nsigla {MATH673}
\def\nsiglahead {Algebraic Geometry}
\def\nextra {HW10}
\def\nlang {ENG}
\input{../../headerVarillyDiff}
\DeclareMathOperator{\sh}{sh}
\begin{document}

\begin{Ej}[14.2.B Vakil]
    Suppose $\cF,\cG$ are locally free sheaves on $X$ of rank $m$ and $n$ respectively. Show that $\cH om_{\cO_X}(\cF,\cG)$ is a locally free sheaf of rank $mn$.
\end{Ej}

\begin{ptcbr}
Observe that if $\cF$ is locally free of rank $m$ then, there is a basis $(U_i)$ of $X$ such that $\cF\mid_{U_i}\isom \cO_{U_i}^m$. Similarly $\cG\mid_{V_j}\isom \cO_{V_i}^n$.\par 
Now the collection $(U_i\cap V_j)_{i,j}$ is also a cover for $X$. From this we get that 
$$\cF\mid_{U_i\cap V_j}\isom \cO_{U_i\cap V_j}^m,\word{and}\cG\mid_{U_i\cap V_j}\isom \cO_{U_i\cap V_j}^n.$$
Observe now that for open sets $W\subseteq U_i\cap V_j$ we have that 
$$\cH om(\cF,\cG)(W)=\Hom(\cF\mid_W,\cG\mid_W)\isom\cH om(\cO_W^m,\cO_W^n)\isom\cO^{mn}_W.$$
This means that the $\Hom$-sheaf locally looks like copies $\cO$ which means it's locally free. 
\end{ptcbr}

\begin{Ej}[14.2.C Vakil]
If $\cE$ is a locally free sheaf on $X$ of rank $n$, then $\cE^\lor=\cH om(\cE,\cO_X)$ is also locally free of rank $n$. This is the \term{dual} of $\cE$. 
\begin{enumerate}[i)]
    \itemsep=-0.4em
    \item Given transition functions for $\cE$, describe the transition functions for $\cE$ for $\cE^\lor$. \aside{Note that if $\cE$ is rank 1, i.e., invertible, the transition functions of the dual are the inverse of the transition functions of the original.}
    \item Show $\cE\isom\cE^{\lor\lor}$. \aside{Caution: your argument showing that there is a canonical isomorphism $\cF\to(\cF^\lor)^\lor$ better not also show that there is an isomorphism $\cF\to\cF^\lor$! We will see an example in 15.1 of a locally free $\cF$ that is not isomorphic to its dual: the invertible sheaf $\cO(1)$ on $\bP^n$.}
\end{enumerate}
\end{Ej}

\begin{ptcbr}
    
\end{ptcbr}

\begin{Ej}
    Show that every invertible sheaf on $\bP^1_k$ is of the form $O(n)$ for some $n$. \hint{Use the classification of finitely generated modules over a principal ideal domain to show that all invertible sheaves on $\bA^1_k$ are trivial. Reduce to determining possible transition functions between the two open subsets in the standard cover of $\bP^1_k$.}
\end{Ej}

\begin{ptcbr}
    We begin by following the hint 
\end{ptcbr}
\end{document} 