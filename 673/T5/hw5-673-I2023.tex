\documentclass[12pt]{memoir}

\def\nsemestre {I}
\def\nterm {Spring}
\def\nyear {2023}
\def\nprofesor {Mark Shoemaker}
\def\nsigla {MATH673}
\def\nsiglahead {Algebraic Geometry}
\def\nextra {HW5}
\def\nlang {ENG}
\input{../../headerVarillyDiff}
\DeclareMathOperator{\sh}{sh}
\begin{document}
%\begin{multicols}{2}

\begin{Ej}[3.2.E Vakil]
    Show that we have identified all the prime ideals of $\bC[x,y]$.\par
    \hint{Suppose $\lie p$ is a prime ideal that is not principal. Show you can find $f,g\in\lie p$ with no common factor. By considering the Euclidean algorithm in the Euclidean domain $\bC(x)[y]$, show that you can find a nonzero $h\in\gen(f,g)\subseteq\lie p$. Using primality, show that one of the linear factors of $h$, say $(x-a)$, is in $\lie p$. Similarly show there is some $(y-b)\in\lie p$.
    }
\end{Ej}

The example in the book before the exercise describes $\bA^2_\bC=\Spec\bC[x,y]$. The example shows that 
\begin{itemize}
    \itemsep=-0.4em
    \item $0$ is a prime ideal. 
    \item Ideals of the form $\gen(x-a,y-b)$ with $a,b\in\bC$ are prime. Even more, that they are maximal. 
    \item And finally ideals of the form $\gen(f)$ with an irreducible $f$ are also prime.
\end{itemize}

The hint tells us to take a prime ideal and assume it is not of the form $\gen(f)$ with an irreducible $f$. Then we will conclude that it is of the form $\gen(x-a,y-b)$ which is the other only non-zero possibility.
%https://math.stackexchange.com/questions/3764847/prime-ideals-of-bbb-cx-y
\begin{ptcbr}
Take a non-principal ideal $\lie p\in\Spec\bC[x,y]$, we begin by wanting to find such $f,g$ with $\gcd(f,g)=1$.\par
If this were not the case, then all polynomials in $\lie p$ would have a common factor. Let $p=\gcd(f)_{f\in\lie p}$, then $p$ is a generator for $\lie p$. As it was the case that $\lie p$ wasn't principal, our assumption that no such $f,g$ exist must be false.\par 
Assume that $g$'s degree in $y$ is lower than $f$'s we may apply the division algorithm on $\bC(x)[y]$ to obtain 
$$f=qg+r,\quad q,r\in \bC(x)[y]\word{and}\deg_y(r)\leq\deg_y(g).$$
We may iterate this process and continue dividing with the residues in order to obtain 
$$g=q_2r+r_2\To r=q_3r_2+r_3\To\dots$$
until we reach a point where the remainder has degree zero in $y$. Retracing the equalities from the last point to the first equation, let us write 
$$f(x,y)=\frac{q_1(x,y)}{q_2(x)}g(x,y)+\frac{r_1(x)}{r_2(x)}$$
where $\frac{r_1}{r_2}$ is the last remainder. Homogenizing we obtain an equation of the form 
$$q_2r_2f=q_1r_2g+r_1q_2\To r_1q_2\in\gen(f,g)$$
and we may also see that $r_1q_2$ is a polynomial depending only on $x$. Thus we may factor it into 
$$r_1q_2(x)=\prod_{i=1}^d(x-a_i)\To \exists j((x-a_j)\in\lie p).$$
The same argument may be repeated but this time we obtain a polynomial $(y-b)\in\lie p$. With this we have 
$$\gen(x-a,y-b)\subseteq\lie p$$
and as $\lie p$ is a proper prime ideal, it must occur that $\lie p$ is this maximal ideal.
\iffalse
%https://math.stackexchange.com/questions/1798660/coprime-polynomials-in-kx-y-are-also-coprime-in-kyx
Now, we know that $f,g$ are coprime in $C[x,y]$, we now want to show that they are coprime in $\bC(x)[y]$. Assume that
 $$h\in\bC(x)[y]\word{with}h\mid f,\ h\mid g,$$
then 
$$h(x,y)=\frac{p_0(x)}{q_0(x)}+\frac{p_1(x)y}{q_1(x)}+\dots+\frac{p_n(x)y^n}{q_n(x)}.$$
If we take $q=q_0q_1\dots q_n$ then $qh\in\bC[x,y]$ and 
$$qh\mid qf,\word{and}qh\mid qg\word{in}\bC(x)[y].$$
\fi
\end{ptcbr}
    
\begin{Ej}[3.2.K Vakil]
    Suppose $S$ is a multiplicative subset of $A$. Describe an order-preserving bijection of the prime ideals of $S^{-1}A$ with the prime ideals of $A$ that don't meet the multiplicative set $S$.
\end{Ej}

\begin{ptcbr}
    %pg110 RobUlloa
   We will describe the bijection 
   $$\set{\lie p\in\Spec A\:\ \lie p\cap S=\emptyset}\to\Spec S^{-1}A.$$
   If $\lie p\in\Spec A$ with $\lie p\cap S=\emptyset$ we will show that $S^{-1}\lie p$ is a prime ideal in $S^{-1}A$. Suppose that $\frac{a_1}{s_1}\frac{a_2}{s_2}\in S^{-1}\lie p$. Then there exist $p\in\lie p$ and $s\in S$ such that 
   $$\frac{a_1}{s_1}\frac{a_2}{s_2}=\frac{p}{s}\To u(a_1a_2s-s_1s_2p)=0,\word{for some}u\in S.$$
   Now 
   $$us_1s_2p\in\lie p\To ua_1a_2s\in\lie p\To a_1a_2\in\lie p\To a_1\in\lie p\lor a_2\in\lie p$$
   from which we conclude that either $\frac{a_1}{s_1}$ or $\frac{a_2}{s_2}$ is in $S^{-1}\lie p$, which means that $S^{-1}\lie p$ is a prime ideal.\par 
   On the other hand if $\lie q\in\Spec S^{-1}A$ we can take its preimage through the mapping:
   $$\phi^{-1}\bonj{\lie q}=\Set{x\in A\:\ \frac x1\in\lie q}$$ 
   and we will show that this ideal doesn't intersect $S$.\par 
   On the contrary, if it did, if there was $s_0\in S\cap\phi^{-1}\bonj{\lie q}$ then $\frac{s_0}{1}\in\lie q$. Then 
   $$\left(\frac{1}{s_0}\right)\left(\frac{s_0}{1}\right)=1\in\lie q\word{because}\lie q\word{is prime}.$$
   This means that $\lie q$ must be the whole ring, but if $\lie q$ were proper, we would have a contradiction. This means that our assumption was wrong and therefore $\phi^{-1}\bonj{\lie q}\cap S=\emptyset$.\par 
   This is the bijection in question. It preserves order because preimages of sets preserve order.
\end{ptcbr}
\
\begin{Ej}[3.2.Q Vakil]
    Consider the map of sets $\pi\:\bA^n_\bZ\to\Spec(\bZ)$
given by the ring map $\bZ\to\bZ[x_1,\dots,x_n]$. If $p\in\bZ$ is prime, describe a bijection between the fiber $\pi^{-1}\left(\bonj
{\gen(p)}\right)$ and $\bA_{\bF_p}^n$. (You won't need to describe either set! Which is good because you can't.) This exercise may give you a sense of how to picture maps (see Figure 3.7), and in particular why you can think of $\bA^n_\bZ$ as an “$\bA^n$-bundle” over $\Spec\bZ$. (Can you interpret the fiber over $\bonj{(0)}$ as $\bA^n_k$ for some field $k$?)
\end{Ej}

\begin{ptcbr}

\end{ptcbr}

\begin{Ej}[3.5.B Vakil]
    
\end{Ej}

\begin{ptcbr}

\end{ptcbr}

\begin{Ej}[3.6.M Vakil]
    Verify that $\bonj{\gen(y-x^2)}\in\bA^2_\bC$
 is a generic point for $V(y-x^2)$. 
\end{Ej}

\begin{ptcbr}
Call $\lie p=\gen(y-x^2)$, the closure of $\set{\lie p}$ is 
$$\ov{\set{\lie p}}=\bigcap_{\substack{F\supseteq\set{\lie p}\\ F\ \text{closed}}}F=\bigcap_{\lie p\in V(S)}V(S).$$
We can see that $\ov{\set{\lie p}}\subseteq V(y-x^2)$ because $\lie p\in V(y-x^2)$.
\end{ptcbr}

\begin{Ej}[3.6.P Vakil]
    Show that $\bA^2_\bC$ is a Noetherian topological space: any decreasing
sequence of closed subsets of $\bA^2_\bC= \Spec\bC[x, y]$ must eventually stabilize. Note that it can take arbitrarily long to stabilize. (The closed subsets of $\bA^2_\bC$ were described in 3.4.5.).\par
Show that $\bC^2$ with the classical topology is not a Noetherian topological space.
\end{Ej}

\begin{ptcbr}
Let $(F_k)$ be a descending chain of closed subsets of $\bA^2_\bC$. Every $F_k$ can be seen to be $V(I_k)$ where $I_k\triangleleft\bC[x,y]$ is an ideal. So by the Nullstellensatz: 
\begin{align*}
    &F_1\supseteq F_2\supseteq\dots\\
    \To&V(I_1)\supseteq V(I_2)\supseteq\dots\\
    \To&I_1\subseteq I_2\subseteq\dots
\end{align*}
Now $(I_k)$ is an ascending chain of ideals in $\bC[x,y]$, and as this ring is Noetherian, ascending chains stabilize. Which means that there exists $r$ with the property that 
$$I_r=I_{r+1}=\dots$$
and therefore $V(I_r)=V(I_{r+1})=\dots$ from which we extract that $\bA^2_\bC$ is Noetherian as a topological space.\par 
On the other hand if we consider the collection of closed balls $\left(\ov B\left(
0,\frac{1}{n}\right)\right)_{n\in\bN}$, we see that this collection doesn't stabilize. If it did, there would exist an $r$ such that 
$$\ov B\left(0,\frac
1r\right)=\ov B\left(0,\frac
{1}{r+1}\right)=\dots$$
but there are no points in $\ov B\left(0,\frac
1r\right)\less\ov B\left(0,\frac
{1}{r+1}\right)$ which are limits of sequences inside $\ov B\left(0,\frac
{1}{r+1}\right)$ because any point in the annulus has positive distance to the smaller ball. 
\end{ptcbr}

\subsection{Missing from last HW}

\begin{Ej}
    Suppose $\phi\:\cF\to\cG$ is a morphism of sheaves of sets on a topological space $X$. Show that the following are equivalent:
\begin{enumerate}
    \item $\phi$ is an epimorphism in the category of sheaves. 
    \item $\phi$ is surjective on the level of stalks: $\phi_p\:\cF_p\to\cG_p$ is surjective for $p\in X$.
\end{enumerate}
\end{Ej}


\end{document} 