\documentclass[12pt]{memoir}

\def\nsemestre {I}
\def\nterm {Spring}
\def\nyear {2023}
\def\nprofesor {Mark Shoemaker}
\def\nsigla {MATH673}
\def\nsiglahead {Algebraic Geometry}
\def\nextra {HW3}
\def\nlang {ENG}
\input{../../headerVarillyDiff}

\begin{document}
%\begin{multicols}{2}

\begin{Ej}[2.3.C. Vakil]
   Suppose $\cF$ is a presheaf and $\cG$ is a sheaf, both of sets, on $X$. Let $\cH\emph{om}(\cF,\cG)$ be the collection of data 
   $$\cH\emph{om}(\cF,\cG)(U)\:=\Mor(\cF|_U,\cG|_U).$$
   Show that this is a sheaf of sets on $X$. 
\end{Ej}
%https://math.stackexchange.com/questions/294802/prove-that-sheaf-hom-is-a-sheaf/294881
\begin{ptcbr}
    We first need to show that $\cH\emph{om}(\cF,\cG)$ is a presheaf, this requires a sensible notion of restriction mapping which satisfies the following:
    \begin{enumerate}[i)]
        \itemsep=-0.4em
        \item $\res_{U,U}=\id_{(\ast)}$ where the identity map is over the object $\cH\emph{om}(\cF,\cG)(U)$.
        \item If $U\subseteq V\subseteq W$ then $\res_{W,U}=\res_{V,U}\circ\res_{W,V}$.
    \end{enumerate}
    Let us consider two objects $\Mor(\cF|_U,\cG|_U)$ and $\Mor(\cF|_V,\cG|_V)$ with $U\subseteq V$. A restriction mapping acts on sections, and sections on these sets are morphisms of sheaves. Our restriction mapping takes $\vf\in\Mor(\cF|_V,\cG|_V)$ to $\res_{V,U}(\vf)\in\Mor(\cF|_U,\cG|_U)$, but recall $\vf$ is a collection of maps of objects of the form 
    $$\vf(W)\:\ \cF(W)\to\cG(W),\word{with}W\subseteq V.$$
    In this sense, it suffices to only consider the open sets contained in $U$. We declare that $\res_{V,U}(\vf)$ is the collection of maps 
    $$\vf(W)\:\ \cF(W)\to\cG(W),\word{with}\un{W\subseteq U}.$$
    \begin{enumerate}[i)]
        \itemsep=-0.4em
        \item The map $\res_{U,U}(\vf)$ acts as follows, every map of the form $\vf(W)$ with $W\subseteq U$ is sent to the map $\vf(W)$ between the same objects because $W\subseteq U$ is still itself.\par 
        This means that $\res_{U,U}$ is the identity map in $\Mor(\cF|_U,\cG|_U)$.
        \item Now suppose $U\subseteq V\subseteq W$ are open sets, then $\res_{V,U}\circ\res_{W,V}$ acts on $\vf$ first by restricting from open sets in $W$ to open sets in $V$ and next by passing from open sets in $V$ to only considering the open sets in $U$.\par
        This is the same as starting with the open sets in $W$ and then only considering the open sets in $U$. The last action is the same as what $\res_{W,U}$ does to $\vf$.
    \end{enumerate}
    This allows us to conclude that the sheaf-Hom is indeed a presheaf. We now have to verify the two sheaf axioms:
    \begin{enumerate}[i)]
        \itemsep=-0.4em
        \item Take $(U_i)$ an cover of $U\subseteq X$ with $\vf,\psi\:\cF|_U\to\cG|_U$ sections which coincide in every covering set. This means that 
        $$\res_{U,U_i}(\vf)=\res_{U,U_i}(\psi)\iff\forall i\bonj{\vf(V)=\psi(V),\ V\subseteq U_i}.$$
        Where $\vf(V),\psi(V)$ are maps of objects from $\cF(V)$ to $\cG(V)$. We wish to show that they coincide on all of $U$, which means that for any $V\subseteq U$ and $f\in\cF(V)$, it holds that
        $$\vf(V)(f)=\psi(V)(f).$$
        Even though we may not talk about these sections directly, we can talk about them after restricting from $V$ to $V\cap U_i$ where $U_i$ is any covering set. To do so, let us introduce the following diagrams:
        \begin{center}
            % https://tikzcd.yichuanshen.de/#N4Igdg9gJgpgziAXAbVABwnAlgFyxMJZABgBpiBdUkANwEMAbAVxiRAB12BjAMQAoAagEoQAX1LpMufIRRkAjFVqMWbTr0Hq6aAAQBVAPpYR4ydjwEi88kvrNWiDtwDigkxJAZzMq6UXU7VUd1VwEtXUNjMQ8vaUsUACYbAJUHEAAzaLM42WQAZmTlezY+Tjo4HHdsi1yk-yKgpwAneAA9YHUeUQNgAVIw7m19I1E+dKrPKRqiAvrAtNL2cpwyipMlGCgAc3giUHSmiABbJDIQHAgkJIa0zhp0t1JONGw3EGoGOgAjGAYABSmPkcDBg6RwWQyhxOiDOFyQ1huanYLTg7U63V6-XCwywoneIE+P3+gPiICaWC2AAtwaZIccrtQ4YgCojgsi2h0XBi+gMuENInjaQd6YgEUyWYTfgDvKTyVTwSlimz7ppBhEjEInuwXlhVXz1VEhVCkAAWRmXRAANiNIst5qQAHYbdCAKz2xBOjzC6Fm84Wl2iCiiIA
\begin{tikzcd}
    \cF(V) \arrow[r, "{\vf(V),\psi(V)}"] \arrow[d, "{\res^{\cF}_{V,V\cap U_i}}"'] & \cG(V) \arrow[d, "{\res^{\cG}_{V,V\cap U_i}}"] & f \arrow[d] \arrow[r]                   & (\ast) \arrow[d] \\
    \cF(V\cap U_i) \arrow[r]                 & \cG(V\cap U_i)                                 & {\res^{\cF}_{V,V\cap U_i}(f)} \arrow[r] & (\ast\ast)      
    \end{tikzcd}
        \end{center}
The lower arrow in the left diagram is either of the two morphisms $\vf(V\cap U_i),\psi(V\cap U_i)$. The right diagram is the same but section-wise:
\begin{itemize}
    \itemsep=-0.4em
    \item The upper right corner is the image of the section $f$ inside $\cG(V)$ through $\vf(V)$ or $\psi(V)$.
    \item The lower right corner can be interpreted in two ways which coincide:
    $$\vf(V\cap U_i)\bonj{\res^{\cF}_{V,V\cap U_i}(f)}=\res^{\cG}_{V,V\cap U_i}(\vf(V)(f))$$
    and the same expression for $\psi$ when that's the case. This equality is due to the fact that $\vf,\psi$ are morphisms of sheaves and therefore commute with restrictions.
\end{itemize}
Recall now that $\vf(V)=\psi(V)$ for $V\subseteq U_i$, in particular we have $\vf(V\cap U_i)=\psi(V\cap U_i)$. So mapping $f$ from the upper left to the lower right gives us
\begin{align*}
\res^{\cG}_{V,V\cap U_i}(\vf(V)(f))= &\vf(V\cap U_i)\bonj{\res^{\cF}_{V,V\cap U_i}(f)}\\
=&\psi(V\cap U_i)\bonj{\res^{\cF}_{V,V\cap U_i}(f)}\\
=&\res^{\cG}_{V,V\cap U_i}(\psi(V)(f))
\end{align*}
where the first and last equalities occur because $\vf$ and $\psi$ are morphisms of sheaves and the middle one because of the hypothesis.\par 
By the identity axiom on $\cG$, as $\cG$ is a sheaf, we can conclude that $\vf(V)(f)=\psi(V)(f)$. This means that $\vf(V)=\psi(V)$, but as $V\subseteq U$ is arbitrary, we conclude that $\vf=\psi$ and therefore we get the identity axiom.
\item Once again let us take $(U_i)$ to be an open cover of $U\subseteq X$ along with $\vf_i\in\cH\emph{om}(\cF,\cG)(U_i)$ for each $i$. These are morphisms of sheaves, which means that for all open subsets $V\subseteq U_i$  they are maps between objects: 
$$\vf_i(V)\:\cF(V)\to\cG(V),\ V\subseteq U_i.$$
Assume now that the condition $\res_{U_i,U_i\cap U_j}(\vf_i)=\res_{U_j,U_i\cap U_j}(\vf_j)$ holds for all $i,j$. We must show that there exists a section $\vf\in\cH\emph{om}(\cF,\cG)(U)$, which is
$$\vf(V)\:\cF(V)\to\cG(V),\ V\subseteq U$$
that satisfies $\res_{U,U_i}(\vf)=\vf_i$. This means that for open sets $V\subseteq U_i$, it must hold that 
$$\res_{U,U_i}(\vf)(V)=\vf_i(V),\ V\subseteq U_i.$$
For this purpose, we will use the gluing axiom on the sheaf $\cG$. Let us now proceed by taking a section $f\in\cF(V)$ with $V\subseteq U$ and map it through the following diagram:
\begin{center}
    % https://tikzcd.yichuanshen.de/#N4Igdg9gJgpgziAXAbVABwnAlgFyxMJZABgBpiBdUkANwEMAbAVxiRAB12BjAMQAoAagEoQAX1LpMufIRQBGUnKq1GLNp16CNdNAAIAqgH0sI8ZOx4CRAEyLl9Zq0QduAcS3cdB46YkgMFjJECtb2qk4umgLaekZYMd4AVr7m0lYotqHUDmrOGu7RnrHGCUbJYsowUADm8ESgAGYAThAAtkhkIDgQSAoqjursTfAAesAaPKKGwAKkhVxecaIg1Ax0AEYwDAAKUpayIE1Y1QAWOGJ+zW291N1ItiBrmzt7Qc4MMA3n2eGDNA3GDwLYomC6NFrtRB9O6IADMPwGeSGo3G3Em03mi2Mc1KJSKSWWqw2W12gXSh2OZzBICukIeMIALAjci5hnAxvkpjNcVgcfi4rjEsszDSIUh4V0eogmY9iS8yQcjqdzqIKKIgA
\begin{tikzcd}
    \cF(V) \arrow[rd, "{\res^{\cF}_{V,V\cap U_i}}"'] &                                                                                                       &                                                                        \\
                                                     & \cF(V\cap U_i) \arrow[r, "\vf_i(V\cap U_i)"] \arrow[d, "{\res^{\cF}_{V\cap U_i,V\cap U_i\cap U_j}}"'] & \cG(V\cap U_i) \arrow[d, "{\res^{\cG}_{V\cap U_i,V\cap U_i\cap U_j}}"] \\
                                                     & \cF(V\cap U_i\cap U_j) \arrow[r]                                                                      & \cG(V\cap U_i\cap U_j)                                                
    \end{tikzcd}
\end{center}
where the lower arrow is the map $\vf_i(V\cap U_i\cap U_j)$. We can construct a similar diagram of $\vf_j$. A section $f\in\cF(V)$ maps through that diagram as follows: 
\begin{center}
    % https://tikzcd.yichuanshen.de/#N4Igdg9gJgpgziAXAbVABwnAlgFyxMJZABgBpiBdUkANwEMAbAVxiRADMQBfU9TXfIRQBGUsKq1GLNgB0ZAJ3gA9YHIDGAMS4B9YADVSe9XTQACAKrasXABTsAlN14gM2PASIAmMRPrNWiCByNOxWNkYyaiYWVvY2copwKupaugYRUWaW1nb2jjx8boJEop6+UgFBCsqqkan6xllWho0xWK2WAFa2CTUpOvotkdHZtg75zq4CHijeZdR+0oHxMnRwOPkSMFAA5vBEoOzyEAC2SGQgOBBIwgUcx2eIopfXiJ53R6c31FdIAMwfB7-H6vAAsgK+bxBSHBFC4QA
\begin{tikzcd}
    f \arrow[rd] &                                                                                   &                                                           \\
                 & {\res^{\cF}_{V,V\cap U_i}(f)} \arrow[r] \arrow[d]                                 & {\vf_i(V\cap U_i)\bonj{\res^{\cF}_{V,V\cap U_i}(f)}} \arrow[d] \\
                 & {\res^{\cF}_{V\cap U_i,V\cap U_i\cap U_j}\bonj{\res^{\cF}_{V,V\cap U_i}(f)}} \arrow[r] & (\ast)                                                   
    \end{tikzcd}
\end{center}
and the lower right corner is either the restriction of the upper right corner, or the image of the lower left which by $\vf_i(V\cap U_i\cap U_j)$. As the $vf_i$ are morphisms of sheaves, both elements are equal. This can be expressed as follows
\begin{align*}
    &\vf_i(V\cap U_i\cap U_j)\left(\res^{\cF}_{V\cap U_i,V\cap U_i\cap U_j}\bonj{\res^{\cF}_{V,V\cap U_i}(f)}\right)\\
=&\res^{\cG}_{V\cap U_i,V\cap U_i\cap U_j}\left(\vf_i(V\cap U_i)\bonj{\res^{\cF}_{V,V\cap U_i}(f)}\right)
\end{align*}
But, let us simplify notation a bit by remembering that the composition of restriction maps is the beginning-to-end restriction map. This means that 
$$\res^{\cF}_{V\cap U_i,V\cap U_i\cap U_j}\bonj{\res^{\cF}_{V,V\cap U_i}(f)}=\res^{\cF}_{V,V\cap U_i\cap U_j}(f).$$
With this in hand, and remembering the hypothesis that our $\vf_i$'s coincide on intersections of the covering sets, we have:
\begin{align*}
    &\res^{\cG}_{V\cap U_i,V\cap U_i\cap U_j}\left(\vf_i(V\cap U_i)\bonj{\res^{\cF}_{V,V\cap U_i}(f)}\right)\\
    =&\vf_i(V\cap U_i\cap U_j)\bonj{\res^{\cF}_{V,V\cap U_i\cap U_j}(f)}\\
    =&\vf_i(V\cap U_i\cap U_j)\bonj{\res^{\cF}_{V,V\cap U_i\cap U_j}(f)}\\
    =&\res^{\cG}_{V\cap U_j,V\cap U_i\cap U_j}\left(\vf_j(V\cap U_j)\bonj{\res^{\cF}_{V,V\cap U_j}(f)}\right).
\end{align*}
So by gluing the maps $\vf_i(V\cap U_i)\bonj{\res^{\cF}_{V,V\cap U_i}(f)}$ in $\cG$ we may construct a map $g\in\cG(V)$ such that 
$$\res_{V,V\cap U_i}^{\cG}(g)=\vf_i(V\cap U_i)\bonj{\res^{\cF}_{V,V\cap U_i}(f)}$$
for each $i$. We finally define the glued map $\vf$ in $\cH\emph{om}$ which takes our original $f$ to this $g$ which we have found. It follows from our construction that 
$$\res_{U,U_i}(\vf)(V)=\vf_i(V),\ V\subseteq U_i.$$
\end{enumerate}
After verifying the axioms, we may conclude that the sheaf Hom is indeed a sheaf. 
\end{ptcbr}

\begin{Ej}[2.3.F Vakil]
Show that the presheaf cokernel satisfies the universal property of cokernels in the category of presheaves.
 \end{Ej}

 \begin{ptcbr}
    Given a map of presheaves $\vf$, we must show that for $\coker_{\text{pre}}\vf$ given the following diagram:
    \begin{center}
        % https://tikzcd.yichuanshen.de/#N4Igdg9gJgpgziAXAbVABwnAlgFyxMJZABgBoBGAXVJADcBDAGwFcYkQAdDgYwDEQAvqXSZc+QinKli1Ok1bsu3AOKDhIDNjwEiUqjQYs2iTjwgBrGACcA+sC44YADxzA0VmAIFdaAMzUiWuJEAEykIbKGCiYAwoKyMFAA5vBEoL5WEAC2SGQgOBBIAMwG8sYgxCA0jPQARjCMAAqi2hIgVlhJABY4VSD1YFBIALRFxELpmTmIUvmFiCX9MIPFeVHl3ADkASAZ2Uhhc8XVWGDlUPRwXYk7e9OzBQelRuzcfTX1TS3BJowwvr0JrsprkaI9EId1uxKtU6g1mkEdCYOt1Aeo7qCjjNntFTH54gIgA
\begin{tikzcd}
    & \cG \arrow[rdd, "c'", bend left] \arrow[d, "c"] &   \\
\cF \arrow[rrd, "0"', bend right] \arrow[r, "0"'] \arrow[ru, "\vf"] & \coker_{\text{pre}}\vf \arrow[rd, dashed]       &   \\
    &                                                 & \cC
\end{tikzcd}
    \end{center}
    that there exists a unique morphism of presheaves $\psi\:\coker_{\text{pre}}\vf\to \cC$. Taking out particular objects for any open set $U$ we have the same diagram but in terms of objects in the underlying category which is an abelian category. Thus, there exists a unique map 
    $$\psi(U)\:\coker_{\text{pre}}\vf(U)\to\cC(U)$$
    and with this we may define the morphism of presheaves $\coker_{\text{pre}}\vf\to \cC$ by taking each of these maps into our collection of data. This immediately gives us unicity by construction and we are left to check that $\psi$ is a morphism of presheaves. Thi means that for $U\subseteq V$, the following diagram commutes
    \begin{center}
        % https://tikzcd.yichuanshen.de/#N4Igdg9gJgpgziAXAbVABwnAlgFyxMJZABgBpiBdUkANwEMAbAVxiRAB12BjCAaxgBOAfWCccMAB45gaATAC+8zjQBmACgBqAShDzS6TLnyEUARnJVajFm05cAwpp16D2PASLnTl+s1aIObkcAVWd9EAw3YyIyb2pfGwC7PkERMUlpWQUldlU1UN1LGCgAc3giUBUBCABbJDIQHAgkAGZ4639AuTgAPVFuFIF5EQ1SYPkQagY6ACMYBgAFQ3cTEAEsEoALHF1wqtr66iakcys-W3Y0bCdJkGm5xeXogIYYFR2XEH26xDbG5sQACZ2uckpdrgUprN5ksoh4Austh89tUfqdjkCQYkuvARmN5H07PYJvIKPIgA
\begin{tikzcd}
    \coker_{\text{pre}}\vf(V) \arrow[d, "{\res^{\coker}_{V,U}}"'] \arrow[r, "\psi(V)"] & \cC(V) \arrow[d, "{\res_{V,U}^{\cC}}"] \\
    \coker_{\text{pre}}\vf(U) \arrow[r, "\psi(U)"']                                    & \cC(U)                                
    \end{tikzcd}
    \end{center}
    Let us take $f\in\coker_{\text{pre}}\vf(V)$ and see how it maps on both sides of the diagram. However, we are not alone in this endeavor; recall that the cokernel isn't only the object, it's the object \emph{and the epic morphism} $c(V)\:\cG(V)\to \coker_{\text{pre}}\vf(V)$. By this, 
    $$\exists g\in\cG(V)(c(V)(g)=f)$$
    and restricting our view to the upper triangle in the cokernel diagram, we have that 
    $$\psi(V)(f)=\psi(V)\bonj{c(V)(g)}=c'(V)(g).$$
    With this fact in hand, let us map $f$
    \begin{align*}
        \res_{V,U}^\cC\bonj{\psi(V)(f)}=&\res_{V,U}^\cC\bonj{c'(V)(g)}=c'(U)\bonj{\res_{V,U}^\cG(g)}\\
        =&\psi(U)\left(c(U)\bonj{\res_{V,U}^\cG(g)}\right)=\psi(U)\left(\res_{V,U}^{\coker}\bonj{c(V)(g)}\right)\\
        =&\psi(U)\left(\res_{V,U}^{\coker}(f)\right)
    \end{align*}
    where we have liberally used the fact that $c,c'$ are maps of sheaves and thus commute with restrictions. From this chain of equalities we conclude that $\psi$ commutes with restrctions and therefore it's a map of sheaves. This uniquely determines the map $\psi\:\coker_{\text{pre}}\vf\to\cC$ and this means that $\coker_{\text{pre}}\vf$ satisfies the universal property of cokernels in the category of presheaves.
 \end{ptcbr}

\begin{Ej}[2.3.H Vakil]
    Show that a sequence of presheaves $0\to\cF_1\to\dots\to\cF_n\to 0$ is exact if and only if $0\to\cF_1(U)\to\dots\to\cF_n(U)\to 0$ is exact for $U\subseteq X$.
\end{Ej}

\red{UNFINISHED}
\begin{Ej}[2.3.I Vakil]
    Suppose $\vf\:\cF\to\cG$ is a morphism of \emph{sheaves}.
    \begin{enumerate}[i)]
        \itemsep=-0.4em
        \item Show that the presheaf kernel $\ker_{\text{pre}}\vf$ is in fact a sheaf. 
        \item Show that it satisfies the universal property of kernels.
    \end{enumerate}
    \hint{The second question follows immediately from the fact that $\ker_{\text{pre}}\vf$ satisfies the universal property in the category of \emph{presheaves}.}
\end{Ej}

\begin{ptcbr}
    \footnote{If time permits we will show that the presheaf kernel is also a presheaf.} We must show that the presheaf kernel satisfies the two sheaf axioms:
    \begin{enumerate}[i)]
        \itemsep=-0.4em
        \item Let $U\subseteq X$ be an open set with $(U_i)$ an open cover of $U$. Suppose $f,g\in\ker_{\text{pre}}\vf(U)$ which coincide in every covering set. The following diagram is used\footnote{In the exercise to show kernel is presheaf.} to define the restriction mapping on the presheaf kernel:
        \begin{center}
        % https://tikzcd.yichuanshen.de/#N4Igdg9gJgpgziAXAbVABwnAlgFyxMJZABgBpiBdUkANwEMAbAVxiRGJAF9T1Nd9CKAIzkqtRizYAdKQGsYAJwD6wGThgAPHMDQKYnTjJoAzABQBVAJRceIDNjwEiIoWPrNWiEDPnLVU9S0dPQMjM3MlLGtuXgcBIgAmUWp3SS8ZAGMAMQto23t+JxQAZmTxD2kpDIBxXJtYwsFkJNcUiU9vKpyIqPq7Pkcm0tbytM6ai0i8hsGiMhHUjo5OMRgoAHN4IlBjBQgAWyQyEBwIJCEYkF2D8+pTpFLRjpl8HDo+68PER-vEABY2hV0lITHVLp8kABWO5nRAANkBYzCk161AYdAARjAGAAFAbxLwKLDrAAWOA+ey+SROsOhT0qrzokRAaMx2LxcSKICJpPJ4MpSAA7DCkAl+TdECIaaK0VgwB0oHQ4CS1iz6cDNFg4Dg4ABCCkSn60xHPKR6OAqcykHqcAB6mSyBq+AOl8JNlXNlutkTtmWqXAonCAA
        \begin{tikzcd}
            0 \arrow[r] & \ker_{\text{pre}}\vf(U) \arrow[r, "\iota"] \arrow[d, "\exists!", dashed] & \cF(U) \arrow[r, "\vf(U)"] \arrow[d, "{\res_{U,U_i}^\cF}"] & \cG(U) \arrow[d, "{\res_{U,U_i}^\cG}"] \\
            0 \arrow[r] & \ker_{\text{pre}}\vf(U_i) \arrow[r, "\iota_i"']                          & \cF(U_i) \arrow[r, "\vf(U_i)"']                            & \cG(U_i)                              
        \end{tikzcd}
        \end{center}
        So let us assume that for all $i$, we have $\res^{\ker}_{U,U_i}(f)=\res^{\ker}_{U,U_i}(g)$. We can include them into $\cF(U_i)$ with $\iota_i$ show that we have 
        $$\iota_i\bonj{\res^{\ker}_{U,U_i}(f)}=\iota_i\bonj{\res^{\ker}_{U,U_i}(g)}$$
        but as we have (assumed) that $\ker_{\text{pre}}\vf$ is a presheaf, the left square commutes. So we have 
        $$\res^{\cF}_{U,U_i}\left(\iota(f)\right)=\res^{\cF}_{U,U_i}\left(\iota(g)\right)$$
        which by the identiy axiom on $\cF$, we have that $\iota(f)=\iota(g)$. As $\iota$ is injective we have that $f=g$, verifying the identity axiom on $\ker_{\text{pre}}\vf$.
        \item Once again consider an open cover $(U_i)$ of $U\subseteq X$ with $f_i\in \ker_{\text{pre}}\vf(U_i)$ for each $i$. Assume that for all $i,j$ we have 
        $$\res^{\ker}_{U_i,U_i\cap U_j}(f_i)=\res^{\ker}_{U_j,U_i\cap U_j}(f_j)$$
        then, using the corresponding inclusion map which $\iota_{ij}\:\ker_{\text{pre}}\vf(U_i\cap U_j)\to\cF(U_i\cap U_j )$ we get 
        $$\iota_{ij}\bonj{\res^{\ker}_{U_i,U_i\cap U_j}(f_i)}=\iota_{ij}\bonj{\res^{\ker}_{U_j,U_i\cap U_j}(f_j)}$$
        which leads us to 
        $$\res^{\cF}_{U_i,U_i\cap U_j}\bonj{\iota_i(f_i)}=\res^{\cF}_{U_i,U_i\cap U_j}\bonj{\iota_j(f_j)}$$
        by commutitativy of the left square of the following diagram (and a similar one for $j$):
        \begin{center}
            % https://tikzcd.yichuanshen.de/#N4Igdg9gJgpgziAXAbVABwnAlgFyxMJZABgBpiBdUkANwEMAbAVxiRGJAF9T1Nd9CKAIzkqtRizYAdKQGsYAJwD6wGThgAPHMDQKYnTjJoAzABQBVJVgCUXHiAzY8BIiKFj6zVohAz5y1Sl1LR09AyMzSywZAGM6NAACSwArW25eJwEiACZRak9JH1iAMQsrNPtHfhcUAGY88S9pKRiAcTKbOwzqwWRc93yJb18W0qjY+KSlVK6HPmde+oHGwpG2jonElIruhaIyZYLhjk4xGCgAc3giUGMFCABbJDIQHAgkIXSQO8eP6jekPUVsMZPgcHQrLMfk9EECAYgACyDJpFKQmDo7b73GEAVn+70QADZkasIhsWpNtiBqAw6AAjGAMAAK8yyPgUWAuAAscFDsUhcq8CXjgc0wRDgFhkpxqSBaQzmayaiAOdzeV9oUgAOz4gUa-mIERCgU0rBgYZwCAMLBQWVHZp6OAqKKkcYUrbTTgAPUC-hl+t+sN1iBF9tRjudVldVk2U2lXpKfMDSONRJJIKkEeALrdcQ98dirS4FE4QA
\begin{tikzcd}
    0 \arrow[r] & \ker_{\text{pre}}\vf(U_i) \arrow[r, "\iota_i"] \arrow[d, "{\res_{U_i,U_i\cap U_j}^{\ker}}"] & \cF(U_i) \arrow[r, "\vf(U_i)"] \arrow[d, "{\res_{U_i,U_i\cap U_j}^\cF}"] & \cG(U_i) \arrow[d, "{\res_{U_i,U_i\cap U_j}^\cG}"] \\
    0 \arrow[r] & \ker_{\text{pre}}\vf(U_i\cap U_j) \arrow[r, "\iota_{ij}"']                                  & \cF(U_i\cap U_j) \arrow[r, "\vf(U_i\cap U_j)"']                          & \cG(U_i\cap U_j)                                  
    \end{tikzcd}
        \end{center}
        Gluing inside $\cF$ we get $\widetilde{f}\in\cF(U)$ such that $\res_{U,U_i}^{\cF}(\widetilde{f})=\iota_i(f_i)$. Mapping $\tilde{f}$ through $\vf(U)$ we can restrict to the covering set to get 
        $$\res_{U,U_i}^{\cG}\bonj{\vf(U)(\tilde{f})}=\vf(U_i)\bonj{\res_{U,U_i}^{\cF}(\tilde{f})}=\vf(U_i)\bonj{\iota_i(f_i)}=0$$
        which means that $\vf(U)(\tilde{f})$ restricts to 0. As $\cG$ is a sheaf, it must occur that $\vf(U)(\tilde{f})=0$ and therefore by exactness of the kernel we find $f\in\ker_{\text{pre}}\vf(U)$ such that $\iota(f)=\tilde{f}$. Such $f$ is the desired element which satisfies the gluing axiom for $\ker_{\text{pre}}\vf$.
    \end{enumerate}
\end{ptcbr}
\begin{Ej}[2.4.C Vakil]
    If $\vf,\psi$ are morphisms from a presheaf of sets $\cF$ to a sheaf of sets $\cG$ that induce the same maps on each stalk, show that $\vf=\psi$. As a hint consider the following diagram:
    \begin{center}
        % https://tikzcd.yichuanshen.de/#N4Igdg9gJgpgziAXAbVABwnAlgFyxMJZABgBpiBdUkANwEMAbAVxiRAB12BjAMQAoAqgEoQAX1LpMufIRQBGclVqMWbTlwDigkeMnY8BImTlL6zVog7s0AJ2gB9YGk5YwAAgGj1Pe2jESQDH0ZIgUTajNVS05bBycXd091DV8xJRgoAHN4IlAAMzsAWyQyEBwIJAAmXRACiGLESupypABmGrqG0pbEOQ6ipAUyisRW5rosBjYACwgIAGsQagY6ACMYBgAFKQNZEAYYPJw00SA
\begin{tikzcd}
    \cF(U) \arrow[d] \arrow[r]    & \cG(U) \arrow[d, hook] \\
    \prod_{p\in U}\cF_p \arrow[r] & \prod_{p\in U}\cG_p   
    \end{tikzcd}
    \end{center}
\end{Ej}

\red{UNFINISHED}

\end{document} 