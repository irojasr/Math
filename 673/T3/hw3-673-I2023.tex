\documentclass[12pt]{memoir}

\def\nsemestre {I}
\def\nterm {Spring}
\def\nyear {2023}
\def\nprofesor {Mark Shoemaker}
\def\nsigla {MATH673}
\def\nsiglahead {Algebraic Geometry}
\def\nextra {HW3}
\def\nlang {ENG}

\makeatletter
\ifx \nauthor\undefined
  \def\nauthor{Ignacio Rojas}
\else
\fi

\ifx \nextra \undefined
\ifx \nlang \undefined
\author{Basado en las clases impartidas por \nprofesor \\\small Notas tomadas por \nauthor}
\else
\author{Based on the lectures by \nprofesor \\\small Notes written by \nauthor}
\fi
\else
\author{\nauthor}
\fi
\date{\nterm\ \nyear}

%%%%%%%%%%%%%
%% 1. Pacotes
%%%%%%%%%%%%%

\usepackage{alltt}
\usepackage{amsfonts}
\usepackage{amsmath}
\usepackage{amssymb}
\usepackage{amsthm}
\usepackage{algorithm}
\usepackage[noend]{algpseudocode}
\usepackage{array}
\newcommand\hmmax{0} % default 3
\newcommand\bmmax{0} % default 4 %%tex.se/3676,219310
%\usepackage{bbold}
\usepackage{bm}
\usepackage{booktabs}
%\usepackage{caption}
%\usepackage{cancel}
%\usepackage{dsfont}
\usepackage{esint}
\usepackage{fancyhdr}
\usepackage{graphicx}
\usepackage[utf8]{inputenc}
\usepackage{listings}
\usepackage{mathabx}
\usepackage[cal=euler]{mathalfa}
%\usepackage[cal=euler,frak=euler]{mathalfa} % mathcal (JIRR) precisabamos correr initexmf --mkmaps en cmd JCVDG
\usepackage{mathdots}
\usepackage{mathrsfs}
%\usepackage{mathtools}
\usepackage{microtype}
\usepackage{multicol}
\usepackage{multirow}
\usepackage[theoremfont,largesc,tighter,osf]{newpxtext} %JCV Diff
\let\widering\undefined
%\usepackage[bigdelims,vvarbb]{newpxmath} %JCVDG
%por alguna razón esto afectaba las tildes en \min, \lim y demás
%\usepackage{pdflscape}
\usepackage{pgfplots}
\usepackage{physics}
\usepackage{siunitx}
\usepackage{slashed}
%\usepackage{stmaryrd}
%\SetSymbolFont{stmry}{bold}{U}{stmry}{m}{n}
%\usepackage{subfigure}
\usepackage{subcaption}
\usepackage{tabularx}
\usepackage[breakable,skins]{tcolorbox}
\usepackage{textcomp} %%JCVDG
\usepackage{tikz}
\usepackage{tkz-euclide}
\usepackage[normalem]{ulem}
\usepackage[all]{xy}
\usepackage{imakeidx}
\ifx \nlang \undefined
\usepackage[spanish]{babel}
\else\fi 
\usepackage{wrapfig}

%%%%%%%%%%%%%%%%%%%%
%% 2. Document Setup
%%%%%%%%%%%%%%%%%%%%

\ifx \nextra \undefined
    \ifx \nlang \undefined
    \makeindex[intoc, title=Índice Analítico] %Título de índice analítico
    %El índice general es aquel en el que se indican los capítulos, títulos y subtítulos del libro.
    %Índice onomástico es donde aparece el nombre de personas mencionadas en el texto, por orden alfabético con el número de las páginas donde aparecen.
    %El índice analítico se refiere a los temas y conceptos que aparecen en el libro
    \indexsetup{othercode={\fancyhead[LE]{\emph{Índice Analítico}}}}
    \else
    \makeindex[intoc, title=Index] 
    \indexsetup{othercode={\fancyhead[LE]{\emph{Index}}}}
    \fi
  \usepackage[pdftex,
    hidelinks,
    pdfauthor={\nauthor},
    pdfsubject={Notas: \nsiglahead\ \nsemestre-\nyear},
    pdftitle={Semestre \nsemestre\ - \nsigla},
  pdfkeywords={UCR Costa Rica Matem\'aticas Mate \nsemestre\ \nterm\ \nyear\ \nsiglahead}]{hyperref}
  \title{\nsigla\ --- \nsiglahead}
\else
  \usepackage[pdftex,
     hidelinks,
    pdfauthor={\nauthor},
    pdfsubject={\nextra \nsiglahead\ \nsemestre-\nyear},
    pdftitle={Semestre \nsemestre\ - \nsigla},
  pdfkeywords={UCR Costa Rica Matem\'aticas Mate \nsemestre\ \nterm\ \nyear\ \nsiglahead\ \nextra}]{hyperref}

  \title{\nsigla\ --- \nsiglahead \\ {\Large \nextra}}
  \renewcommand\printindex{}
\fi

\pgfplotsset{compat=1.12}


\pagestyle{fancy}
\setlength{\headheight}{15.72pt} %preceding warning said make it at least this


\ifx \nsiglahead \undefined
\def\nsiglahead{\nsigla}
\fi

\lhead{} %%%empty lhead
\rfoot{\thepage}

\ifx \nextra \undefined
  \chead{
    \ifnum\thepage=1
    \else
      \ifx \nlang \undefined
      \textbf{Notas \nsiglahead\ \nsemestre-\nyear}
      \else
      \textbf{Notes \nsiglahead\ \nsemestre-\nyear}
      \fi
    \fi}
  \rhead{}%\firstxmark} % Top right header
\else
%    \chead{
%    \ifnum\thepage=1
%    \else
%      \textbf{Notas \nsiglahead\ \nsemestre-\nyear \ (\nextra)}
%    \fi}
     \chead{
       \textbf{\nextra\ \nsigla\ \nsemestre-\nyear}
     }
     \rhead{
       \textbf{\nauthor}
     }
\fi
\lfoot{}%\lastxmark} % Bottom left footer
\cfoot{} % Bottom center footer

\usetikzlibrary{arrows.meta}
\usetikzlibrary{decorations.markings}
\usetikzlibrary{decorations.pathmorphing}
\usetikzlibrary{positioning}
\usetikzlibrary{fadings}
\usetikzlibrary{intersections}
\usetikzlibrary{cd}

\ifx \nhtml \undefined
\else
  \renewcommand\printindex{}
  \DisableLigatures[f]{family = *}
  \let\Contentsline\contentsline
  \renewcommand\contentsline[3]{\Contentsline{#1}{#2}{}}
  \renewcommand{\@dotsep}{10000}
  \newlength\currentparindent
  \setlength\currentparindent\parindent

  \newcommand\@minipagerestore{\setlength{\parindent}{\currentparindent}}
  \usepackage[active,tightpage,pdftex]{preview}
  \renewcommand{\PreviewBorder}{0.1cm}

  \newenvironment{stretchpage}%
  {\begin{preview}\begin{minipage}{\hsize}}%
    {\end{minipage}\end{preview}}
  \AtBeginDocument{\begin{stretchpage}}
  \AtEndDocument{\end{stretchpage}}

  \newcommand{\@@newpage}{\end{stretchpage}\begin{stretchpage}}

  \let\@real@section\section
  \renewcommand{\section}{\@@newpage\@real@section}
  \let\@real@subsection\subsection
  \renewcommand{\subsection}{\@ifstar{\@real@subsection*}{\@@newpage\@real@subsection}}
\fi
\ifx \ntrim \undefined
\usepackage[shortlabels]{enumitem} %mfw package order matters por savetrees
\else
  \usepackage{geometry}
  \geometry{
    papersize={379pt, 699pt},
    textwidth=345pt,
    textheight=596pt,
    left=17pt,
    top=54pt,
    right=17pt
  }
  \headwidth=345pt
 \usepackage[extreme]{savetrees}
\fi

\ifx \darktheme\undefined
\else
\pagecolor[rgb]{0.2,0.231,0.302}%{0.23,0.258,0.321}
\color[rgb]{1,1,1}
\fi

\ifx \nextra \undefined
\let\@real@maketitle\maketitle
\renewcommand{\maketitle}{\@real@maketitle\begin{center}\begin{minipage}[c]{0.9\textwidth}\centering\footnotesize 
  \ifx \nlang \undefined
  Estas notas no están respaldadas por los profesores y han sido modificadas (a menudo de manera significativa) después de las clases. No están lejos de ser representaciones precisas de lo que realmente se dio en clase y en particular todos los errores son casi seguramente míos.
  \else 
  Please note that these notes were not provided or endorsed by the lecturer and have been significantly altered after the class. They may not accurately reflect the content covered in class and any errors are solely my responsibility.
  \fi
\end{minipage}\end{center}}
\else
\fi

\def\moverlay{\mathpalette\mov@rlay}
\def\mov@rlay#1#2{\leavevmode\vtop{%
   \baselineskip\z@skip \lineskiplimit-\maxdimen
   \ialign{\hfil$\m@th#1##$\hfil\cr#2\crcr}}}
\newcommand{\charfusion}[3][\mathord]{
    #1{\ifx#1\mathop\vphantom{#2}\fi
        \mathpalette\mov@rlay{#2\cr#3}
      }
    \ifx#1\mathop\expandafter\displaylimits\fi}

%%%%%%%%%%%%%%%%%%%%%%%%%%%%%%
%% 2.1 Some internal machinery
%%%%%%%%%%%%%%%%%%%%%%%%%%%%%%

\makeatletter
\renewcommand{\section}{\@startsection{section}{1}{\z@}%
							 {-3.25ex \@plus -1ex \@minus -.2ex}%
							 {1.5ex \@plus.2ex}%
							 {\normalfont\large\bfseries}}
\renewcommand{\subsection}{\@startsection{subsection}{2}{\z@}%
							 {-3.25ex \@plus -1ex \@minus -.2ex}%
							 {1.5ex \@plus .2ex}%
               {\normalfont\normalsize\bfseries}}
\newcommand*{\defeq}{\!\mathrel{\rlap{%
             \raisebox{0.3ex}{$\m@th\cdot$}}%
             \raisebox{-0.3ex}{$\m@th\cdot$}}%
                    =\!}
\makeatother
\ifx\ntrim\undefined
\newcommand{\coursetitle}{\nsigla: \nsiglahead}
\ifx\nextra\undefined
\pagestyle{ruled}
\makeoddhead{ruled}{\coursetitle}{}{\rightmark}
\else\fi
\settypeblocksize{49pc}{37pc}{*}
\setlrmargins{*}{*}{1.2}
\setulmargins{*}{*}{0.8}
\setheadfoot{16pt}{30pt}
\setheaderspaces{*}{1.5pc}{1}
\setmarginnotes{1pt}{1pt}{1pt}
\checkandfixthelayout

\setlength{\unitlength}{3pt}
\setlength{\hfuzz}{1pt}

\setlength{\fboxsep}{6pt}

\setlength{\footskip}{17pt}

\linespread{1.1}
\else\fi
\renewcommand{\cftdotsep}{\cftnodots} %%% no dots in ToC
\setpnumwidth{2em}  %%% width of page-number box in ToC


\newcommand{\stophere}{\relax} %% can be changed to `\endinput'
% \newcommand{\stophere}{\endinput} %% can be changed to `\relax'


\DeclareRobustCommand{\qned}{\ifmmode
  \else \leavevmode\unskip\penalty9999 \hbox{}\nobreak\hfill \fi
  \quad\hbox{\qnedsymbol}}
\newcommand{\qnedsymbol}{$\boxminus$} %% No-proofs end with `\qned'

\DeclareRobustCommand{\qef}{\ifmmode
  \else \leavevmode\unskip\penalty9999 \hbox{}\nobreak\hfill \fi
  \quad\hbox{\qefsymbol}}
\newcommand{\qefsymbol}{$\lozenge$} %% Examples end with `\qef'
\def\enddefn{\qef\endtrivlist}      %% `\qef' automático en defns
\def\endejem{\qef\endtrivlist}      %% `\qef' automático en ejemplos

\newcommand{\hideqed}{\renewcommand{\qed}{}} %% to suppress `\qed'
\newcommand{\hideqef}{\renewcommand{\qef}{}} %% to suppress `\qef'

% \newcommand{\ldbrack}{\ensuremath{[\mskip-2.5mu[}} %% corchetes [[
% \newcommand{\rdbrack}{\ensuremath{]\mskip-2.5mu]}} %% corchetes ]]

\newcommand{\stroke}{\mathbin|}     %% (for `\bbraket' and such)

\newcommand{\rtri}{\blacktriangleright} %% (for `\marker' and such)
\newcommand{\tribar}{|\mkern-2mu|\mkern-2mu|} %% norma triple: |||


%% Formatting changes:

\renewcommand{\labelitemi}{$\diamond$} %% instead of bullets

\renewcommand{\theenumi}{\alph{enumi}}  %% use lowercase letters
\renewcommand{\labelenumi}{\textup{(\theenumi)}} %% inside parentheses

%%%%%%%%%%%%%%
%% 2.2. Colors
%%%%%%%%%%%%%%

\definecolor{MATLABgreen}{RGB}{28,172,0} % color values Red, Green, Blue
\definecolor{MATLABlila}{RGB}{170,55,241}
\definecolor{dankBlue}{RGB}{51,60,77} % color values Red, Green, Blue
\definecolor{dankBlueLite}{RGB}{82,97,125} % color values Red, Green, Blue
\definecolor{celesUCR}{RGB}{0,192,243}
\definecolor{azulUCR}{RGB}{0,93,164}
\definecolor{verdeUCR}{RGB}{109,192,103}
\definecolor{yelloUCR}{RGB}{255,224,106}

%%%%%%%%%%%%%%%%%%%%%%%%%%%
%% 3. Theorems and suchlike
%%%%%%%%%%%%%%%%%%%%%%%%%%%

\ifx\nlang\undefined

\theoremstyle{plain}
\ifx \nextra \undefined
\newtheorem{Th}{Teorema}[section]      %%% Theorem 1.1.1
\newtheorem{Tmon}[Th]{Teoremón}
\newtheorem{Prop}[Th]{Proposición}     %%% Proposition 1.1.2
\newtheorem{Lem}[Th]{Lema}             %%% Lemma 1.1.3
\newtheorem{Cor}[Th]{Corolario}        %%% Corollary 1.1.4
\else
\newtheorem{Th}{Teorema}               %%% Theorem 1.1.1
\newtheorem{Tmon}{Teoremón}
\newtheorem{Prop}{Proposición}         %%% Proposition 1.1.2
\newtheorem{Lem}{Lema}                 %%% Lemma 3
\newtheorem{Cor}{Corolario}            %%% Corollary 4
\fi
\newtheorem*{nonum-Th}{Teorema}        %%% No-numbered Theorem
\newtheorem*{nonum-Cor}{Corolario}     %%% No-numbered Corollary

\theoremstyle{definition}
\ifx \nextra \undefined
\newtheorem{Def}[Th]{Definición}       %%% Definition 1.1.5
\newtheorem{Ex}[Th]{Ejemplo}           %%% Example 1.1.6
\newtheorem{Ej}[Th]{Ejercicio}         %%% Ejercicio 1.1.7
\else
\newtheorem{Def}{Definición}           %%% Definition 5
\newtheorem{Ex}{Ejemplo}               %%% Example 6
\newtheorem{Ej}{Ejercicio}             %%% Ejercicio 7
\fi
\newtheorem{Hec}[Th]{Hecho}            %%% Hecho 1.1.8
\newtheorem*{nonum-Def}{Definición}    %%% No number Definition
\newtheorem*{nonum-Ex}{Ejemplo}        %%% No number Example
\newtheorem*{nonum-Ej}{Ejercicio}      %%% No number Ejercicio
\newtheorem*{nonum-Hec}{Hecho}         %%% No number Fact


\theoremstyle{remark}
\newtheorem{Rmk}[Th]{Observación}      %%%Remark 1.1.9
\newtheorem*{nonum-Rmk}{Observación}         %%% No number Fact
\newtheorem*{Notn}{Notaci\'on}        %% Notaciones
\newtheorem*{Warn}{Advertencia}       %% Advertencias
\newtheorem*{Qn}{Pregunta}            %% Pregunta

\else

\theoremstyle{plain}
\ifx \nextra \undefined
\newtheorem{Th}{Theorem}[section]      %%% Theorem 1.1.1
\newtheorem{Tmon}[Th]{Teoremón}
\newtheorem{Prop}[Th]{Proposition}     %%% Proposition 1.1.2
\newtheorem{Lem}[Th]{Lemma}             %%% Lemma 1.1.3
\newtheorem{Cor}[Th]{Corollary}        %%% Corollary 1.1.4
\else
\newtheorem{Th}{Theorem}               %%% Theorem 1.1.1
\newtheorem{Tmon}{Teoremón}
\newtheorem{Prop}{Proposition}         %%% Proposition 1.1.2
\newtheorem{Lem}{Lemma}                 %%% Lemma 3
\newtheorem{Cor}{Corollary}            %%% Corollary 4
\fi
\newtheorem*{nonum-Th}{Theorem}        %%% No-numbered Theorem
\newtheorem*{nonum-Cor}{Corollary}     %%% No-numbered Corollary

\theoremstyle{definition}
\ifx \nextra \undefined
\newtheorem{Def}[Th]{Definition}       %%% Definition 1.1.5
\newtheorem{Ex}[Th]{Example}           %%% Example 1.1.6
\newtheorem{Ej}[Th]{Exercise}         %%% Exercise 1.1.7
\else
\newtheorem{Def}{Definition}           %%% Definition 5
\newtheorem{Ex}{Example}               %%% Example 6
\newtheorem{Ej}{Exercise}             %%% Exercise 7
\fi
\newtheorem{Hec}[Th]{Fact}            %%% Fact 1.1.8
\newtheorem*{nonum-Def}{Definition}    %%% No number Definition
\newtheorem*{nonum-Ex}{Example}        %%% No number Example
\newtheorem*{nonum-Ej}{Exercise}      %%% No number Exercise
\newtheorem*{nonum-Hec}{Fact}         %%% No number Fact


\theoremstyle{remark}
\newtheorem{Rmk}[Th]{Remark}      %%%Remark 1.1.9
\newtheorem*{nonum-Rmk}{Remark}         %%% No number Fact
\newtheorem*{Notn}{Notation}        %% Notaciones
\newtheorem*{Warn}{Warning}       %% Warnings
\newtheorem*{Qn}{Question}            %% Question

\fi 

\numberwithin{equation}{section}

\setlength{\parindent}{3ex}

% \renewcommand{\labelitemi}{--}
% \renewcommand{\labelitemii}{$\circ$}
% \renewcommand{\labelenumi}{(\roman{*})}

%\let\stdsection\section
%\renewcommand\section{\newpage\stdsection}

\newcommand\qedsym{\hfill\ensuremath{\square}}
% Strike through
\def\st{\bgroup \ULdepth=-.55ex \ULset}

%%%%%%%%% === My T Color Box === %%%%%%%%%%%%%%

\ifx\nlang\undefined
\ifx \darktheme\undefined
\newtcolorbox{ptcb}{
colframe = black,
colback = white,
breakable,
enhanced
}
\newtcolorbox{ptcbp}{
colframe = black,
colback = white,
coltitle = black,
colbacktitle = black!40,
title = Prueba,
breakable,
enhanced
}
\newtcolorbox{ptcbr}{
colframe = blue,
colback = white,
coltitle = blue,
colbacktitle = blue!40,
title = Respuesta,
breakable,
enhanced
}
\else
\newtcolorbox{ptcb}{
colframe = white,
colback = dankBlue,
colupper = white,
breakable,
enhanced
}
\newtcolorbox{ptcbp}{
colframe = white,
colback = dankBlue,
colupper = white,
coltitle = white,
colbacktitle = dankBlueLite,
title = Prueba,
breakable,
enhanced
}
\newtcolorbox{ptcbr}{
colframe = white,
colback = white,
coltitle = blue,
colbacktitle = blue!40,
title = Respuesta,
breakable,
enhanced
}
\fi

\else
\ifx \darktheme\undefined
\newtcolorbox{ptcb}{
colframe = black,
colback = white,
breakable,
enhanced
}
\newtcolorbox{ptcbp}{
colframe = black,
colback = white,
coltitle = black,
colbacktitle = black!40,
title = Proof,
breakable,
enhanced
}
\newtcolorbox{ptcbr}{
colframe = blue,
colback = white,
coltitle = blue,
colbacktitle = blue!40,
title = Answer,
breakable,
enhanced
}
\else
\newtcolorbox{ptcb}{
colframe = white,
colback = dankBlue,
colupper = white,
breakable,
enhanced
}
\newtcolorbox{ptcbp}{
colframe = white,
colback = dankBlue,
colupper = white,
coltitle = white,
colbacktitle = dankBlueLite,
title = Proof,
breakable,
enhanced
}
\newtcolorbox{ptcbr}{
colframe = white,
colback = white,
coltitle = blue,
colbacktitle = blue!40,
title = Answer,
breakable,
enhanced
}
\fi
\fi


%%%%%%%%% === Listings === %%%%%%%%%%%%%%
\lstset{basicstyle=\ttfamily,breaklines=true}

\lstset{language=Matlab,%
    %basicstyle=\color{red},
    breaklines=true,%
    morekeywords={matlab2tikz},
    keywordstyle=\color{blue},%
    morekeywords=[2]{1}, keywordstyle=[2]{\color{black}},
    identifierstyle=\color{black},%
    stringstyle=\color{MATLABlila},
    commentstyle=\color{MATLABgreen},%
    showstringspaces=false,%without this there will be a symbol in the places where there is a space
    numbers=left,%
    numberstyle={\tiny \color{black}},% size of the numbers
    numbersep=9pt, % this defines how far the numbers are from the text
   % emph=[1]{for,end,break,function,if,elseif,else},emphstyle=[1]\color{blue}, %some words to emphasise
    %emph=[2]{word1,word2}, emphstyle=[2]{style},
}

%%%%%%%%%%%%%%%%%%%%%%%%%%
%% 4. Simple abbreviations
%%%%%%%%%%%%%%%%%%%%%%%%%%

%%% Operator names:

\DeclareMathOperator{\area}{area}
\DeclareMathOperator{\card}{card}
\DeclareMathOperator{\ccl}{ccl}
\DeclareMathOperator{\ch}{ch}
\DeclareMathOperator{\cl}{cl}
\DeclareMathOperator{\coker}{coker}
\DeclareMathOperator{\Conv}{Conv}   %%Convex hull
\DeclareMathOperator{\cosec}{cosec}
\DeclareMathOperator{\cosech}{cosech}
\DeclareMathOperator{\covol}{covol}
\DeclareDocumentCommand\curl{}{\operatorname{curl}} 
\DeclareMathOperator{\diag}{diag}
\DeclareMathOperator{\diam}{diam}
\DeclareMathOperator{\Diff}{Diff}
\DeclareDocumentCommand\div{}{\operatorname{div}} 
\DeclareMathOperator{\energy}{energy}
\DeclareMathOperator{\erfc}{erfc}
\DeclareMathOperator{\Ext}{Ext}
\DeclareMathOperator{\fst}{fst}
\DeclareMathOperator{\Fit}{Fit}
\DeclareMathOperator{\gr}{gr}
\DeclareMathOperator{\hcf}{hcf}
\DeclareMathOperator{\Hilb}{Hilb} %Hilbert scheme
\DeclareMathOperator{\id}{id}
\DeclareMathOperator{\Ind}{Ind}
\DeclareMathOperator{\Int}{Int}
\DeclareMathOperator{\Isom}{Isom}
\DeclareMathOperator{\lcm}{lcm}
\DeclareMathOperator{\length}{length}
\DeclareMathOperator{\Lie}{Lie}
\DeclareMathOperator{\like}{like}
\DeclareMathOperator{\Lk}{Lk}
\DeclareMathOperator{\Maps}{Maps}
\DeclareMathOperator{\mcd}{mcd}
\DeclareMathOperator{\mcm}{mcm}
\DeclareMathOperator{\Min}{Min}
\DeclareMathOperator{\orb}{orb}
\DeclareMathOperator{\ord}{ord}
\DeclareMathOperator{\otp}{otp}
\DeclareMathOperator{\pr}{pr}       %% proyector
\DeclareMathOperator{\poly}{poly}
\DeclareMathOperator{\rel}{rel}
\DeclareMathOperator{\Rad}{Rad}
\DeclareMathOperator*{\res}{res}
\DeclareMathOperator{\Ric}{Ric}
\DeclareMathOperator{\rk}{rk}
\DeclareMathOperator{\Rees}{Rees}
\DeclareMathOperator{\Root}{Root}
\DeclareMathOperator{\rot}{rot}         %% rotacional
\DeclareMathOperator{\spn}{span}
\DeclareMathOperator{\St}{St}
\DeclareMathOperator{\supp}{supp}
\DeclareMathOperator{\Syl}{Syl}
\DeclareMathOperator{\Sym}{Sym}
\DeclareMathOperator{\vol}{vol}

% not-math
\newcommand{\bolds}[1]{{\bfseries #1}}
\newcommand{\cat}[1]{\mathsf{#1}}
\newcommand{\ph}{\,\cdot\,}
\newcommand{\term}[1]{\un{#1}\index{#1}}
\newcommand{\phantomeq}{\hphantom{{}={}}}
\newcommand{\ttt}{\texttt}
\newcommand{\red}[1]{\textcolor{red}{#1}}
\newcommand{\prp}[1]{\textcolor{purple}{#1}}
\newcommand{\blu}[1]{\textcolor{azulUCR}{#1}}
\newcommand{\green}[1]{\textcolor{verdeUCR}{#1}}
\newcommand{\yelo}[1]{\textcolor{yelloUCR}{#1}}
\newcommand{\cele}[1]{\textcolor{celesUCR}{#1}}

%functions
\DeclareMathOperator{\sgn}{sgn}
\newcommand*{\Cdot}{{\raisebox{-0.25ex}{\scalebox{1.5}{$\cdot$}}}}      %% cdot más grande
\newcommand{\ind}{\mathbf{1}}       %%%indicator function
\newcommand{\mm}{\mathfrak{m}}      %%%metric


% Greek letters:

\newcommand{\al}{\alpha}                %% short for  \alpha
\newcommand{\bt}{\beta}                 %% short for  \beta
\newcommand{\Dl}{\Delta}                %% short for  \Delta
\newcommand{\dl}{\delta}                %% short for  \delta
\newcommand{\eps}{\varepsilon}          %% short for  \varepsilon
\newcommand{\Ga}{\Gamma}                %% short for  \Gamma
\newcommand{\ga}{\gamma}                %% short for  \gamma
\newcommand{\kp}{\kappa}                %% short for  \kappa
\newcommand{\La}{\Lambda}               %% short for  \Lambda
\newcommand{\la}{\lambda}               %% short for  \lambda
\newcommand{\Om}{\Omega}                %% short for  \Omega
\newcommand{\om}{\omega}                %% short for  \omega
\newcommand{\Sg}{\Sigma}                %% short for  \Sigma
\newcommand{\sg}{\sigma}                %% short for  \sigma
\newcommand{\Te}{\Theta}                %% short for  \Theta
\newcommand{\te}{\theta}                %% short for  \theta
\newcommand{\ups}{\upsilon}             %% short for  \upsilon
\newcommand{\vf}{\varphi}               %% short for  \varphi
\newcommand{\ze}{\zeta}                 %% short for  \zeta
\newcommand{\vsg}{\varsigma}            %% short for  \varsigma
\newcommand{\vte}{\vartheta}            %% short for  \vartheta

%Boldface letters

\newcommand{\bA}{\mathbb{A}}        %% antisimetrizador
\newcommand{\bB}{\mathbb{B}}        %% bola unitaria
\newcommand{\bC}{\mathbb{C}}    %%% números complejos
\newcommand{\bCP}{\mathbb{CP}}  %%% espacio proyectivo complejo
\newcommand{\bD}{\mathbb{D}}        %% Poincaré disk
\newcommand{\bE}{\mathbb{E}}
\newcommand{\bF}{\mathbb{F}}        %% un cuerpo
\newcommand{\bH}{\mathbb{H}}        %% cuaterniones
\newcommand{\bI}{\mathbb{I}}        %% ideal de zeros
\newcommand{\bK}{\mathbb{K}}            %% ein korper
\newcommand{\bN}{\mathbb{N}}    %%% números naturales
\newcommand{\bP}{\mathbb{P}}        %% números enteros positivos
\newcommand{\bQ}{\mathbb{Q}}    %%% números racionales
\newcommand{\bR}{\mathbb{R}}    %%% números reales
\newcommand{\bRP}{\mathbb{RP}}  %%% espacio proyectivo real
\newcommand{\bS}{\mathbb{S}}    %%% esfera
\newcommand{\bT}{\mathbb{T}}        %% círculo o toro
\newcommand{\bV}{\mathbb{V}}        %% lugar geométrico de ceros
\newcommand{\bZ}{\mathbb{Z}}    %%% números enteros

%Script letters:

\newcommand{\cA}{\mathcal{A}}           %% formas diferenciales
\newcommand{\cB}{\mathcal{B}}           %% una base vectorial
\newcommand{\cC}{\mathcal{C}}           %% otra base vectorial
\newcommand{\cD}{\mathcal{D}}           %% funciones de prueba
\newcommand{\cE}{\mathcal{E}}           %% un modulo proyectivo
\newcommand{\cF}{\mathcal{F}}           %% espacio de Fock
\newcommand{\cG}{\mathcal{G}}           %% funtor de Gelfand
\newcommand{\cH}{\mathcal{H}}           %% espacio de Hilbert
\newcommand{\cI}{\mathcal{I}}           %% un funtor de inclusion
\newcommand{\cJ}{\mathcal{J}}           %% otro funtor
\newcommand{\cK}{\mathcal{K}}           %% otro espacio de Hilbert
\newcommand{\cL}{\mathcal{L}}           %% operadores lineales
\newcommand{\cM}{\mathcal{M}}           %% multiplicadores
\newcommand{\cN}{\mathcal{N}}           %% funciones nulas
\newcommand{\cO}{\mathcal{O}}           %% funciones de crec-to lento
\newcommand{\cP}{\mathcal{P}}           %% una particion
\newcommand{\cR}{\mathcal{R}}           %% funciones representativas
\newcommand{\cQ}{\mathcal{Q}}           %% otra particion
\newcommand{\cS}{\mathcal{S}}           %% funciones de Schwartz
\newcommand{\cT}{\mathcal{T}}           %% una topologia
\newcommand{\cU}{\mathcal{U}}           %% cubrimiento abierto
\newcommand{\cV}{\mathcal{V}}           %% vecindarioas
\newcommand{\cW}{\mathcal{W}}           %% grupo de Weyl
\newcommand{\cZ}{\mathcal{Z}}           %% topología de Zariski

%%% Fraktur letters:

\newcommand{\gA}{\mathfrak{A}}      %% un atlas
\newcommand{\g}{\mathfrak{g}}       %% un álgebra de Lie
\newcommand{\gB}{\mathfrak{B}}      %% otro atlas
\newcommand{\ggl}{\mathfrak{gl}}    %% álg de Lie general lineal
\newcommand{\gsl}{\mathfrak{sl}}    %% álg de Lie especial lineal
\newcommand{\gso}{\mathfrak{so}}    %% álg de Lie especial ortogonal
\newcommand{\gsu}{\mathfrak{su}}    %% álg de Lie especial unitaria
\newcommand{\gX}{\mathfrak{X}}      %% campos vectoriales

%%% Roman letters:

\newcommand{\dR}{\mathrm{dR}}       %% cohomología de de Rham
\newcommand{\rGL}{\mathrm{GL}}      %% grupo general lineal
\newcommand{\rO}{\mathrm{O}}        %% grupo ortogonal
\newcommand{\rSL}{\mathrm{SL}}      %% grupo especial lineal
\newcommand{\rSO}{\mathrm{SO}}      %% grupo ortogonal especial
\newcommand{\rSp}{\mathrm{Sp}}      %% grupo simpléctico
\newcommand{\rSU}{\mathrm{SU}}      %% grupo unitario especial
\newcommand{\rU}{\mathrm{U}}        %% grupo unitario
\newcommand{\rUH}{\mathrm{UH}}      %% cuaterniones unitarias
\newcommand{\rT}{\mathrm{T}}        %% grupo triangular

% Sanserif letters:

\newcommand{\sA}{\mathsf{A}}            %% algebras de Lie A_n
\newcommand{\sB}{\mathsf{B}}            %% grupo como categoria
\newcommand{\sC}{\mathsf{C}}            %% una categoria
\newcommand{\sD}{\mathsf{D}}            %% otra categoria
\newcommand{\sE}{\mathsf{E}}            %% otra categoria mas
\newcommand{\sF}{\mathsf{F}}            %% algebra de Lie F_4
\newcommand{\sG}{\mathsf{G}}            %% algebra de Lie G_2
\newcommand{\sJ}{\mathsf{J}}            %% un poset
\newcommand{\sK}{\mathsf{K}}            %% un poset
\newcommand{\sL}{\mathcal{L}}           %% derivada de Lie
\newcommand{\sN}{\mathsf{N}}            %% categoría con objetos \bN
\newcommand{\sT}{\mathsf{T}}            %% transpuesta

%%% Boldface letters:

\bmdefine{\CC}{C}                       %% C negrilla
\bmdefine{\cc}{c}
%\bmdefine{\dd}{d}                       %% d negrilla
\bmdefine{\ee}{e}                       %% vector e
\bmdefine{\eeps}{\varepsilon}           %% basic form \eps
\bmdefine{\FF}{F}                       %% vector F
\bmdefine{\ff}{f}                       %% vector f
\bmdefine{\ii}{i}                       %% cuaternion i
\bmdefine{\jj}{j}                       %% cuaternion j
\bmdefine{\kk}{k}                       %% cuaternion k
\bmdefine{\lla}{\lambda}                %% sucesion \la
\bmdefine{\mmu}{\mu}                    %% sucesion \mu
\bmdefine{\pp}{p}                       %% vector p
\bmdefine{\qq}{q}                       %% vector q
\bmdefine{\rr}{r}                       %% vector r
\bmdefine{\ssg}{\sigma}                 %% vector \sg
%\bmdefine{\sss}{s}
%\bmdefine{\ttt}{t}
\bmdefine{\VV}{V}                       %% V negrilla
\bmdefine{\xx}{x}                       %% sucesion x
\bmdefine{\xxi}{\xi}                    %% vector \xi
\bmdefine{\yy}{y}                       %% sucesion y
\bmdefine{\zz}{z}                       %% sucesion z

% Matrix groups
\DeclareMathOperator{\GL}{GL}   %%% grupo general lineal
\DeclareMathOperator{\Or}{O}    %%% grupo ortogonal
\DeclareMathOperator{\PGL}{PGL} %%% grupo proyectivo lineal
\DeclareMathOperator{\PSL}{PSL} %%% grupo proyectivo lineal especial
\DeclareMathOperator{\PSO}{PSO} %%% grupo proyectivo ortogonal
\DeclareMathOperator{\PSU}{PSU} %%% grupo proyectivo unitario
\DeclareMathOperator{\SL}{SL}   %%% grupo especial lineal
\DeclareMathOperator{\SO}{SO}   %%% grupo especial ortogonal
\DeclareMathOperator{\SU}{SU}   %%% grupo especial unitario

% Numericc
\newcommand{\argmin}{\text{argm\'in}}
\DeclareMathOperator{\dof}{dof}

%% Brackets
\newcommand{\conj}[1]{\left\lbrace#1\right\rbrace}
\newcommand{\bonj}[1]{\left\lbrack#1\right\rbrack}
\newcommand{\obonj}[1]{\left\rbrack#1\right\lbrack}
\newcommand{\rbonj}[1]{\left\rbrack#1\right\rbrack}
\newcommand{\lbonj}[1]{\left\lbrack#1\right\lbrack}
\newcommand{\snm}[1]{\|#1\|}           %small norma
\newcommand{\nm}[1]{\left\|#1\right\|} %norma pegadita
\newcommand{\pnm}[1]{\biggl|\biggl|#1\biggr|\biggr|}
\let\oldvec=\vec
\renewcommand{\vec}[1]{\mathbf{#1}}
\newcommand\quot[2]{
        \mathchoice
            {% \displaystyle
                \text{\raise1ex\hbox{$#1$}\Big/\lower1ex\hbox{$#2$}}%
            }
            {% \textstyle
                {^{ #1}/_{ #2}}
            }
            {% \scriptstyle
                {^{ #1}/_{ #2}}
            }
            {% \scriptscriptstyle
                {^{ #1}/_{ #2}}
            }
    }
%\newcommand*\quot[2]{{^{\textstyle #1}\big/_{\textstyle #2}}}
\newcommand*\squot[2]{{^{ #1}/_{ #2}}}%%%small quotient
\newcommand{\multinom}[2]{\ensuremath{\left(\kern-.3em\left(\genfrac{}{}{0pt}{}{#1}{#2}\right)\kern-.3em\right)}}

% Probability
\DeclareMathOperator{\Bernoulli}{Bernoulli}
\DeclareMathOperator{\betaD}{beta}
\DeclareMathOperator{\bias}{bias}
\DeclareMathOperator{\binomial}{binomial}
\DeclareMathOperator{\corr}{corr}
\DeclareMathOperator{\cov}{cov}
\DeclareMathOperator{\gammaD}{gamma}
\DeclareMathOperator{\mse}{mse}
\DeclareMathOperator{\multinomial}{multinomial}
\DeclareMathOperator{\Poisson}{Poisson}
\DeclareMathOperator{\Var}{Var}     %%%variance
\DeclareMathOperator{\Cov}{Cov}     %%%Covariance
\renewcommand{\mid}{\;\ifnum\currentgrouptype=16 \middle\fi|\;}

% Combinatorics
\DeclareMathOperator{\ins}{ins}   % insertion tableaux
\DeclareMathOperator{\asc}{asc}   % ascents
\DeclareMathOperator{\rw}{rw}     % reading word
\DeclareMathOperator{\rev}{rev}     % reading word
\DeclareMathOperator{\rect}{rect} % rectification of young tableau
\DeclareMathOperator{\sh}{sh}     % shape of young tableau
\DeclareMathOperator{\std}{std}   % standarization
\DeclareMathOperator{\Fl}{\mathcal{F}\ell}       %% conjunto de Flags
\DeclareMathOperator{\Frob}{Frob} % Frobenius map

% Algebra
\DeclareMathOperator{\Ad}{Ad}       %% acción adjunta
\DeclareMathOperator{\adj}{adj}
\DeclareMathOperator{\Ann}{Ann}     %% aniquilador o anulador de módulos
\DeclareMathOperator{\Ass}{Ass}     %% ideales asociados
\DeclareMathOperator{\Aut}{Aut}
\DeclareMathOperator{\Bl}{\mathcal{B}\!\ell}       %% blowup de un espacio
\DeclareMathOperator{\Char}{char}
\DeclareMathOperator{\codim}{codim}
\DeclareMathOperator{\disc}{disc}
\DeclareMathOperator{\dom}{dom}
\DeclareMathOperator{\End}{End}     %%%space of endomorphisms
\DeclareMathOperator{\Fix}{Fix}
\DeclareMathOperator{\Frac}{Frac}
\DeclareMathOperator{\Gal}{Gal}
\DeclareMathOperator{\gen}{gen}     %%%set generated by...
\DeclareMathOperator{\Gr}{Gr}       %%%Grassmannian
\DeclareMathOperator{\Hom}{Hom}
\DeclareMathOperator{\Hurw}{Hurw}
\DeclareMathOperator{\image}{image}
\DeclareMathOperator{\Mor}{Mor}
\DeclareMathOperator{\Nil}{Nil}
\DeclareMathOperator{\Orb}{Orb}
\DeclareMathOperator{\Pic}{Pic}     %%% grupo de Picard 
\DeclareMathOperator{\Quot}{Quot}
\DeclareMathOperator{\Spec}{Spec}
\DeclareMathOperator{\Stab}{Stab}
\DeclareMathOperator{\Taut}{Taut}

% Analysis
\DeclareMathOperator*{\esssup}{ess\hspace{0.5mm}sup}
\DeclareMathOperator*{\essinf}{ess\hspace{0.5mm}inf}
%\DeclareMathOperator{\Int}{Int}     %%%interior vacilon funcional

\newcommand{\loc}{\text{loc}}
\newcommand{\LB}{\cL_\cB}           %%%bounded linear operator

% Logic
\newcommand{\cleq}{\preccurlyeq}
\newcommand{\cgeq}{\succcurlyeq}

% Others
\renewcommand{\ev}{\operatorname{ev}}     %%%evalutation previously expectation value physics package
\newcommand{\bigcupdot}{\charfusion[\mathop]{\bigcup}{\Cdot}} %%JCVDG
%\renewcommand{\bigcupdot}{\charfusion[\mathop]{\bigcup}{\Cdot}}
\newcommand{\cupdot}{\charfusion[\mathbin]{\cup}{\Cdot}}
\newcommand{\exterior}{\mathchoice{{\textstyle\bigwedge}}{{\bigwedge}}{{\textstyle\wedge}}{{\scriptstyle\wedge}}}
\newcommand{\hol}{\mathfrak{hol}}
\newcommand{\Id}{\mathrm{Id}}
\newcommand{\lie}[1]{\mathfrak{#1}}
\newcommand{\qeq}{\mathrel{``{=}"}}
\newcommand{\wsto}{\stackrel{\mathrm{w}^*}{\to}}
\newcommand{\wt}{\mathrm{wt}}

%\let\Im\relax
%\let\Re\relax

%%% Shorter symbol names:

\newcommand{\bull}{{\scriptstyle\bullet}}  %% vertice en figuras
\newcommand{\del}{\partial}             %% short for  \partial
\newcommand{\downto}{\downarrow}        %% limite a la derecha
\newcommand{\dsp}{\displaystyle}        %% despliegue en texto
\renewcommand{\geq}{\geqslant}          %% mayor o igual (variante)
\newcommand{\hookto}{\hookrightarrow}     %% inclusion arrow
\newcommand{\isom}{\simeq}              %% isomorfismo
\renewcommand{\l}{\ell}                   %% ele cursiva
\renewcommand{\leq}{\leqslant}          %% menor o igual (variante)
\newcommand{\less}{\setminus}           %% set difference
\newcommand{\otto}{\leftrightarrow}     %% bijection
\newcommand{\ox}{\otimes}               %% producto tensorial
\newcommand{\rt}{\triangleleft}         %% un orden parcial
\newcommand{\rteq}{\trianglelefteq}     %% normal subgroup
\newcommand{\up}{{\mathord{\uparrow}}}  %% espinor `up'
\newcommand{\upto}{\uparrow}            %% left hand limit
\newcommand{\w}{\wedge}                 %% producto exterior
\newcommand{\wto}{\rightharpoonup}      %% convergencia debil
\newcommand{\x}{\times}                 %% producto vectorial
\renewcommand{\.}{\Cdot}                %% producto escalar
\renewcommand{\:}{\mathbin{:}}          %% colon in  f: A -> B
\newcommand{\into}{\rightarrowtail}     %% injection arrow
\newcommand{\lr}{\dashv}                %% adjunction
\newcommand{\lt}{\triangleright}        %% a left action
\newcommand{\lteq}{\trianglerighteq}    %% normal supergroup
\newcommand{\nb}{\nabla}                %% homomorfismo de suma
\newcommand{\nisom}{\not\simeq}         %% negacion de isomorfismo
%\newcommand{\oast}{\circledast}         %% variante de * (ya existe en stmaryrd)
\newcommand{\onto}{\twoheadrightarrow}  %% surjection arrow
\newcommand{\opp}{\circ}                %% objeto opuesto
\newcommand{\ottto}{\longleftrightarrow} %% bijection in display
\newcommand{\pullb}{\lrcorner}          %% simbolo de pullback
\newcommand{\pushf}{\ulcorner}          %% simbolo de pushout
\newcommand{\rx}{\rtimes}               %% producto semidirecto
\newcommand{\To}{\Rightarrow}           %% entre funtores
\newcommand{\tofro}{\rightleftarrows}   %% pair of opposed maps
\newcommand{\toto}{\rightrightarrows}   %% pair of parallel maps

\renewcommand{\2}{\flat}                  %% marcador de sucesiones
\newcommand{\3}{\sharp}                 %% marcador de sucesiones
\newcommand{\4}{\natural}               %% marcador de morfismos
% \newcommand{\5}{\diamond}               %% for roots of trees
% \newcommand{\7}{\dagger}                %% adjunto de operador
\newcommand{\8}{\bullet}                %% anonymous degree

%%% Useful abbreviations:

\newcommand{\Coo}{\cC^\infty}         %% funciones suaves
\newcommand{\ctr}{\mathbin{\lrcorner\,}} %% contraction symbol
\newcommand{\nbf}{{\vec\nabla}}     %% short for  \vec\nabla

\newcommand{\as}{\quad\text{cuando}\enspace} %% `cuando' en límites
\newcommand{\bCoo}{{\bC_\infty}}    %% esfera de Riemann
% \newcommand{\bRoo}{{\bR_\infty}}    %% círculo real extendido

%%% Repeated relations:

\newcommand{\cupycup}{\cup\cdots\cup} %% unión repetida
\newcommand{\capycap}{\cap\cdots\cap} %% intersección repetida
\newcommand{\sys}{\subset\cdots\subset}%% subconjunto propio repetido
\newcommand{\subysub}{\subseteq\cdots\subseteq} %%subconjunto repetido
\newcommand{\oxyox}{\otimes\cdots\otimes} %% prod tensorial repetido
\newcommand{\wyw}{\wedge\cdots\wedge} %% producto exterior repetido
\newcommand{\opyop}{\oplus\cdots\oplus} %% suma directa repetida
\newcommand{\xyx}{\times\cdots\times} %% producto directo repetido

%%% Arrows with riders:

\newcommand{\longto}{\mathop{\longrightarrow}\limits}

%%% Small fractions in displays:

\newcommand{\half}{{\mathchoice{\nhalf}{\thalf}{\shalf}{\shalf}}} %%display text script script^2
\newcommand{\happi}{{\tfrac{\pi}{2}}} %% small fraction  \pi/2
\newcommand{\quarter}{\tfrac{1}{4}} %% small fraction  1/4
\newcommand{\nhalf}{\frac{1}{2}}
\newcommand{\shalf}{{\scriptstyle\frac{1}{2}}} %% tiny fraction 1/2
\newcommand{\thalf}{{\tfrac{1}{2}}} %% small fraction  1/2
\renewcommand{\third}{\tfrac{1}{3}}   %% small fraction  1/3 %Hay que renew porque mathabx toma second y third como x'' y x''' por ejemplo

\newcommand{\ihalf}{{\tfrac{i}{2}}} %% small fraction  i/2

%%%%%%%%%%%%%%%%%%%%%%%%%%%%%
%% 5. Commands with arguments
%%%%%%%%%%%%%%%%%%%%%%%%%%%%%

%%% Accent-like commands, abbreviated:

\newcommand{\ov}{\overline}        %% short for  \overline
\newcommand{\un}{\underline}       %% short for  \underline
\newcommand{\wh}{\widehat}          %% short for  \widehat

%%% Separate words in displays:

\newcommand{\word}[1]{\quad\text{#1}\quad} %% texto intercalado

%%% Webpage locator:

\newcommand{\zelda}[1]{$\langle${\footnotesize\texttt{#1}}$\rangle$}

%% Symbol placement:

\newcommand{\pre}[1]{{}^{#1\!}} %% upper left exponent

%%% Proof-part labels:

\newcommand{\Adiff}[2]{\ensuremath{\Ad\,(\mathrm{#1})\Longleftrightarrow
    (\mathrm{#2})}:\enspace}
\newcommand{\Adimp}[2]{\ensuremath{\Ad\,(\mathrm{#1})\Longrightarrow
    (\mathrm{#2})}:\enspace}
\newcommand{\Adit}[1]{\ensuremath{\Ad\,(\mathrm{#1})}:\enspace}

%%% Enclose one argument with delimiters:

\newcommand{\bool}[1]{\llbracket#1\rrbracket} %% condición booleana
\newcommand{\combo}[1]{\operatorname{co}(#1)} %% convex combo
\newcommand{\lin}[1]{\operatorname{lin}\langle#1\rangle} %% `span'
\newcommand{\set}[1]{\{\,#1\,\}}    %% set notation

\newcommand{\floor}[1]{\lfloor#1\rfloor} %% mayor entero <= x
\newcommand{\Set}[1]{\biggl\{\,#1\,\biggr\}} %% set notation (large)
\newcommand{\roof}[1]{\lceil#1\rceil} %% menor entero >= x
\newcommand{\genr}[1]{\left\langle #1\right\rangle}     %% grupo generado por #1

%%% Asides:

\newcommand{\aside}[1]{$\llbracket$\,#1\,$\rrbracket$} % nota lateral
\ifx \nlang \undefined
\newcommand{\hint}[1]{$\llbracket$\,In\-di\-ca\-ci\'on: #1\,$\rrbracket$}
\else
\newcommand{\hint}[1]{$\llbracket$\,Hint: #1\,$\rrbracket$}
\fi 


%%% Matrices:

\newcommand{\onebytwo}[2]{\begin{pmatrix} %% 1 x 2 matrix
  #1 & #2 \end{pmatrix}}
\newcommand{\onebythree}[3]{\begin{pmatrix} %% 1 x 3 matrix
  #1 & #2 & #3 \end{pmatrix}}
\newcommand{\onebyfour}[4]{\begin{pmatrix} %% 1 x 4 matrix
  #1 & #2 & #3 & #4 \end{pmatrix}}
\newcommand{\twobyone}[2]{\begin{pmatrix} %% 2 x 1 matrix
   #1 \\ #2 \end{pmatrix}}
\newcommand{\twobytwo}[4]{\begin{pmatrix} %% 2 x 2 matrix
   #1 & #2 \\ #3 & #4 \end{pmatrix}}
\newcommand{\twobythree}[6]{\begin{pmatrix} %% 2 x 3 matrix
    #1 & #2 & #3\\ #4 & #5 & #6 \end{pmatrix}}
\newcommand{\threebyone}[3]{\begin{pmatrix} %% 3 x 1 matrix
   #1 \\ #2 \\ #3 \end{pmatrix}}
\newcommand{\threebythree}[9]{\begin{pmatrix} %% 3 x 3 matrix
   #1 & #2 & #3 \\ #4 & #5 & #6 \\ #7 & #8 & #9 \end{pmatrix}}
\newcommand{\fourbyone}[4]{\begin{pmatrix} %% 2 x 1 matrix
   #1 \\ #2 \\ #3 \\ #4 \end{pmatrix}}
%\newcommand{\fourbyfour}[16]{\begin{pmatrix} %% 4 x 4 matrix
%  #1 & #2 & #3 & #4\\ #5 & #6 & #7 & #8 \\ #9 & #10 & #11 & #12 \\ #13 & #14 & #15 & #16 \end{pmatrix}}
\newcommand{\nbyn}[9]{\begin{pmatrix} %% 4 x 4 matrix with prefilled entries
  #1 & #2 & \cdots & #3\\ #4 & #5 & \cdots & #6 \\ \vdots & \vdots & \ddots & \vdots \\ #7 & #8 & \cdots & #9 \end{pmatrix}}

%%%%%%%%%%%%%%%%%%%%%%%%%%%%
%% 6. Hyphenation exceptions
%%%%%%%%%%%%%%%%%%%%%%%%%%%%

\hyphenation{auto-va-lor auto-va-lo-res auto-vec-tor auto-vec-to-res
car-di-na-li-dad ce-rra-da ce-rra-do ce-rra-das ce-rra-dos cons-tan-te
cons-tan-tes cons-truc-ci cons-truir con-ti-nua con-ti-nua-mente
con-ti-nuas con-ti-nui-dad con-ti-nuo con-ti-nuos co-rres-pon-den-cia
co-rres-pon-de co-rres-pon-den co-rres-pon-dien-te
co-rres-pon-dien-tes co-va-rian-te cual-quier cual-quiera
cu-bri-mien-to desa-rro-lla-do desa-rro-llar des-pu dia-go-nal
dia-go-na-les di-fe-ren-cia-ble di-fe-ren-cia-bles di-fe-ren-cial
di-fe-ren-cia-les di-fe-ren-te di-fe-ren-tes dis-cre-ta dis-cre-tas
dis-cre-to dis-cre-tos di-vi-si-bi-li-dad di-vi-si-ble ele-men-tal
ele-men-ta-les ele-men-to ele-men-tos equi-va-len-cia equi-va-lente
equi-va-lentes equi-va-rian-te equi-va-rian-tes eu-cli-dia-na
eu-cli-dia-nas eu-cli-dia-no eu-cli-dia-nos Fi-gu-ra Gal-ois
gal-oi-sia-na ge-ne-rada ge-ne-rado ge-ne-ra-dor ge-ne-ra-do-res
ge-ne-ral ge-ne-ra-les ge-ne-ra-li-dad ge-ne-ra-li-za ge-ne-ra-li-zan
ge-ne-ran ge-ne-rar geo-me-tr geo-me-try Ha-da-mard ho-meo-mor-fis-mo
ho-meo-mor-fo idea-les in-de-pen-dien-te in-de-pen-dien-tes
in-va-rian-cia in-va-rian-te in-va-rian-tes li-ne-a-les
li-ne-al-men-te ma-ne-ra me-dian-te mo-der-no nin-gu-no nues-tra
nues-tro nu-me-ra-ble ope-ra-ci ope-ra-cio-nes ope-ra-dor
ope-ra-do-res or-to-go-nal par-ti-cu-lar pro-ce-di-mien-to pro-duc-to
pro-duc-tos pro-pie-dad pro-pie-da-des pro-po-si-ci re-fe-ren-cia
re-fle-xi-va re-fle-xi-vas re-fle-xi-vo re-fle-xi-vos re-so-lu-ble
res-pec-ti-va-men-te res-pec-ti-vo res-pec-ti-vos res-pec-to
sa-tis-fa-ce sepa-ra-ble sepa-ra-bles si-guien-te si-guien-tes
subes-pa-cio subes-pa-cios te-dra-edro te-tra-edros tri-vial
tri-via-les uti-lidad va-lo-res va-ria-ble va-ria-bles va-rie-dad
va-rie-da-des ve-cin-da-rio ve-cin-da-rios vec-to-rial vec-to-ria-les
vice-versa}


%%% TikZ arrows and such

\pgfarrowsdeclarecombine{twolatex'}{twolatex'}{latex'}{latex'}{latex'}{latex'}
\tikzset{->/.style = {decoration={markings,
                                  mark=at position 1 with {\arrow[scale=2]{latex'}}},
                      postaction={decorate}}}
\tikzset{<-/.style = {decoration={markings,
                                  mark=at position 0 with {\arrowreversed[scale=2]{latex'}}},
                      postaction={decorate}}}
\tikzset{<->/.style = {decoration={markings,
                                   mark=at position 0 with {\arrowreversed[scale=2]{latex'}},
                                   mark=at position 1 with {\arrow[scale=2]{latex'}}},
                       postaction={decorate}}}
\tikzset{->-/.style = {decoration={markings,
                                   mark=at position #1 with {\arrow[scale=2]{latex'}}},
                       postaction={decorate}}}
\tikzset{-<-/.style = {decoration={markings,
                                   mark=at position #1 with {\arrowreversed[scale=2]{latex'}}},
                       postaction={decorate}}}
\tikzset{->>/.style = {decoration={markings,
                                  mark=at position 1 with {\arrow[scale=2]{latex'}}},
                      postaction={decorate}}}
\tikzset{<<-/.style = {decoration={markings,
                                  mark=at position 0 with {\arrowreversed[scale=2]{twolatex'}}},
                      postaction={decorate}}}
\tikzset{<<->>/.style = {decoration={markings,
                                   mark=at position 0 with {\arrowreversed[scale=2]{twolatex'}},
                                   mark=at position 1 with {\arrow[scale=2]{twolatex'}}},
                       postaction={decorate}}}
\tikzset{->>-/.style = {decoration={markings,
                                   mark=at position #1 with {\arrow[scale=2]{twolatex'}}},
                       postaction={decorate}}}
\tikzset{-<<-/.style = {decoration={markings,
                                   mark=at position #1 with {\arrowreversed[scale=2]{twolatex'}}},
                       postaction={decorate}}}

\tikzset{circ/.style = {fill, circle, inner sep = 0, minimum size = 3}}
\tikzset{scirc/.style = {fill, circle, inner sep = 0, minimum size = 1.5}}
\tikzset{mstate/.style={circle, draw, blue, text=black, minimum width=0.7cm}}

\tikzset{eqpic/.style={baseline={([yshift=-.5ex]current bounding box.center)}}}
\tikzset{commutative diagrams/.cd,cdmap/.style={/tikz/column 1/.append style={anchor=base east},/tikz/column 2/.append style={anchor=base west},row sep=tiny}}

\definecolor{mblue}{rgb}{0.2, 0.3, 0.8}
\definecolor{morange}{rgb}{1, 0.5, 0}
\definecolor{mgreen}{rgb}{0.1, 0.4, 0.2}
\definecolor{mred}{rgb}{0.5, 0, 0}

\def\drawcirculararc(#1,#2)(#3,#4)(#5,#6){%
    \pgfmathsetmacro\cA{(#1*#1+#2*#2-#3*#3-#4*#4)/2}%
    \pgfmathsetmacro\cB{(#1*#1+#2*#2-#5*#5-#6*#6)/2}%
    \pgfmathsetmacro\cy{(\cB*(#1-#3)-\cA*(#1-#5))/%
                        ((#2-#6)*(#1-#3)-(#2-#4)*(#1-#5))}%
    \pgfmathsetmacro\cx{(\cA-\cy*(#2-#4))/(#1-#3)}%
    \pgfmathsetmacro\cr{sqrt((#1-\cx)*(#1-\cx)+(#2-\cy)*(#2-\cy))}%
    \pgfmathsetmacro\cA{atan2(#2-\cy,#1-\cx)}%
    \pgfmathsetmacro\cB{atan2(#6-\cy,#5-\cx)}%
    \pgfmathparse{\cB<\cA}%
    \ifnum\pgfmathresult=1
        \pgfmathsetmacro\cB{\cB+360}%
    \fi
    \draw (#1,#2) arc (\cA:\cB:\cr);%
}
\newcommand\getCoord[3]{\newdimen{#1}\newdimen{#2}\pgfextractx{#1}{\pgfpointanchor{#3}{center}}\pgfextracty{#2}{\pgfpointanchor{#3}{center}}}

\newcommand\qedshift{\vspace{-17pt}}
\newcommand\fakeqed{\pushQED{\qed}\qedhere}

\def\Xint#1{\mathchoice
   {\XXint\displaystyle\textstyle{#1}}%
   {\XXint\textstyle\scriptstyle{#1}}%
   {\XXint\scriptstyle\scriptscriptstyle{#1}}%
   {\XXint\scriptscriptstyle\scriptscriptstyle{#1}}%
   \!\int}
\def\XXint#1#2#3{{\setbox0=\hbox{$#1{#2#3}{\int}$}
     \vcenter{\hbox{$#2#3$}}\kern-.5\wd0}}
\def\ddashint{\Xint=}
\def\dashint{\Xint-}

\newcommand\separator{{\centering\rule{2cm}{0.2pt}\vspace{2pt}\par}}

\newenvironment{own}{\color{gray!70!black}}{}

\newcommand\makecenter[1]{\raisebox{-0.5\height}{#1}}

\mathchardef\mdash="2D

\newenvironment{significant}{\begin{center}\begin{minipage}{0.9\textwidth}\centering\em}{\end{minipage}\end{center}}
\DeclareRobustCommand{\rvdots}{%
  \vbox{
    \baselineskip4\p@\lineskiplimit\z@
    \kern-\p@
    \hbox{.}\hbox{.}\hbox{.}
  }}
\DeclareRobustCommand\tph[3]{{\texorpdfstring{#1}{#2}}}
\def\BState{\State\hskip-\ALG@thistlm}

\makeatother 

\begin{document}
%\begin{multicols}{2}

\begin{Ej}[2.3.C. Vakil]
   Suppose $\cF$ is a presheaf and $\cG$ is a sheaf, both of sets, on $X$. Let $\cH\emph{om}(\cF,\cG)$ be the collection of data 
   $$\cH\emph{om}(\cF,\cG)(U)\:=\Mor(\cF|_U,\cG|_U).$$
   Show that this is a sheaf of sets on $X$. 
\end{Ej}
%https://math.stackexchange.com/questions/294802/prove-that-sheaf-hom-is-a-sheaf/294881
\begin{ptcbr}
    We first need to show that $\cH\emph{om}(\cF,\cG)$ is a presheaf, this requires a sensible notion of restriction mapping which satisfies the following:
    \begin{enumerate}[i)]
        \itemsep=-0.4em
        \item $\res_{U,U}=\id_{(\ast)}$ where the identity map is over the object $\cH\emph{om}(\cF,\cG)(U)$.
        \item If $U\subseteq V\subseteq W$ then $\res_{W,U}=\res_{V,U}\circ\res_{W,V}$.
    \end{enumerate}
    Let us consider two objects $\Mor(\cF|_U,\cG|_U)$ and $\Mor(\cF|_V,\cG|_V)$ with $U\subseteq V$. A restriction mapping acts on sections, and sections on these sets are morphisms of sheaves. Our restriction mapping takes $\vf\in\Mor(\cF|_V,\cG|_V)$ to $\res_{V,U}(\vf)\in\Mor(\cF|_U,\cG|_U)$, but recall $\vf$ is a collection of maps of objects of the form 
    $$\vf(W)\:\ \cF(W)\to\cG(W),\word{with}W\subseteq V.$$
    In this sense, it suffices to only consider the open sets contained in $U$. We declare that $\res_{V,U}(\vf)$ is the collection of maps 
    $$\vf(W)\:\ \cF(W)\to\cG(W),\word{with}\un{W\subseteq U}.$$
    \begin{enumerate}[i)]
        \itemsep=-0.4em
        \item The map $\res_{U,U}(\vf)$ acts as follows, every map of the form $\vf(W)$ with $W\subseteq U$ is sent to the map $\vf(W)$ between the same objects because $W\subseteq U$ is still itself.\par 
        This means that $\res_{U,U}$ is the identity map in $\Mor(\cF|_U,\cG|_U)$.
        \item Now suppose $U\subseteq V\subseteq W$ are open sets, then $\res_{V,U}\circ\res_{W,V}$ acts on $\vf$ first by restricting from open sets in $W$ to open sets in $V$ and next by passing from open sets in $V$ to only considering the open sets in $U$.\par
        This is the same as starting with the open sets in $W$ and then only considering the open sets in $U$. The last action is the same as what $\res_{W,U}$ does to $\vf$.
    \end{enumerate}
    This allows us to conclude that the sheaf-Hom is indeed a presheaf. We now have to verify the two sheaf axioms:
    \begin{enumerate}[i)]
        \itemsep=-0.4em
        \item Take $(U_i)$ an cover of $U\subseteq X$ with $\vf,\psi\:\cF|_U\to\cG|_U$ sections which coincide in every covering set. This means that 
        $$\res_{U,U_i}(\vf)=\res_{U,U_i}(\psi)\iff\forall i\bonj{\vf(V)=\psi(V),\ V\subseteq U_i}.$$
        Where $\vf(V),\psi(V)$ are maps of objects from $\cF(V)$ to $\cG(V)$. We wish to show that they coincide on all of $U$, which means that for any $V\subseteq U$ and $f\in\cF(V)$, it holds that
        $$\vf(V)(f)=\psi(V)(f).$$
        Even though we may not talk about these sections directly, we can talk about them after restricting from $V$ to $V\cap U_i$ where $U_i$ is any covering set. To do so, let us introduce the following diagrams:
        \begin{center}
            % https://tikzcd.yichuanshen.de/#N4Igdg9gJgpgziAXAbVABwnAlgFyxMJZABgBpiBdUkANwEMAbAVxiRAB12BjAMQAoAagEoQAX1LpMufIRRkAjFVqMWbTr0Hq6aAAQBVAPpYR4ydjwEi88kvrNWiDtwDigkxJAZzMq6UXU7VUd1VwEtXUNjMQ8vaUsUACYbAJUHEAAzaLM42WQAZmTlezY+Tjo4HHdsi1yk-yKgpwAneAA9YHUeUQNgAVIw7m19I1E+dKrPKRqiAvrAtNL2cpwyipMlGCgAc3giUHSmiABbJDIQHAgkJIa0zhp0t1JONGw3EGoGOgAjGAYABSmPkcDBg6RwWQyhxOiDOFyQ1huanYLTg7U63V6-XCwywoneIE+P3+gPiICaWC2AAtwaZIccrtQ4YgCojgsi2h0XBi+gMuENInjaQd6YgEUyWYTfgDvKTyVTwSlimz7ppBhEjEInuwXlhVXz1VEhVCkAAWRmXRAANiNIst5qQAHYbdCAKz2xBOjzC6Fm84Wl2iCiiIA
\begin{tikzcd}
    \cF(V) \arrow[r, "{\vf(V),\psi(V)}"] \arrow[d, "{\res^{\cF}_{V,V\cap U_i}}"'] & \cG(V) \arrow[d, "{\res^{\cG}_{V,V\cap U_i}}"] & f \arrow[d] \arrow[r]                   & (\ast) \arrow[d] \\
    \cF(V\cap U_i) \arrow[r]                 & \cG(V\cap U_i)                                 & {\res^{\cF}_{V,V\cap U_i}(f)} \arrow[r] & (\ast\ast)      
    \end{tikzcd}
        \end{center}
The lower arrow in the left diagram is either of the two morphisms $\vf(V\cap U_i),\psi(V\cap U_i)$. The right diagram is the same but section-wise:
\begin{itemize}
    \itemsep=-0.4em
    \item The upper right corner is the image of the section $f$ inside $\cG(V)$ through $\vf(V)$ or $\psi(V)$.
    \item The lower right corner can be interpreted in two ways which coincide:
    $$\vf(V\cap U_i)\bonj{\res^{\cF}_{V,V\cap U_i}(f)}=\res^{\cG}_{V,V\cap U_i}(\vf(V)(f))$$
    and the same expression for $\psi$ when that's the case. This equality is due to the fact that $\vf,\psi$ are morphisms of sheaves and therefore commute with restrictions.
\end{itemize}
Recall now that $\vf(V)=\psi(V)$ for $V\subseteq U_i$, in particular we have $\vf(V\cap U_i)=\psi(V\cap U_i)$. So mapping $f$ from the upper left to the lower right gives us
\begin{align*}
\res^{\cG}_{V,V\cap U_i}(\vf(V)(f))= &\vf(V\cap U_i)\bonj{\res^{\cF}_{V,V\cap U_i}(f)}\\
=&\psi(V\cap U_i)\bonj{\res^{\cF}_{V,V\cap U_i}(f)}\\
=&\res^{\cG}_{V,V\cap U_i}(\psi(V)(f))
\end{align*}
where the first and last equalities occur because $\vf$ and $\psi$ are morphisms of sheaves and the middle one because of the hypothesis.\par 
By the identity axiom on $\cG$, as $\cG$ is a sheaf, we can conclude that $\vf(V)(f)=\psi(V)(f)$. This means that $\vf(V)=\psi(V)$, but as $V\subseteq U$ is arbitrary, we conclude that $\vf=\psi$ and therefore we get the identity axiom.
\item Once again let us take $(U_i)$ to be an open cover of $U\subseteq X$ along with $\vf_i\in\cH\emph{om}(\cF,\cG)(U_i)$ for each $i$. These are morphisms of sheaves, which means that for all open subsets $V\subseteq U_i$  they are maps between objects: 
$$\vf_i(V)\:\cF(V)\to\cG(V),\ V\subseteq U_i.$$
Assume now that the condition $\res_{U_i,U_i\cap U_j}(\vf_i)=\res_{U_j,U_i\cap U_j}(\vf_j)$ holds for all $i,j$. We must show that there exists a section $\vf\in\cH\emph{om}(\cF,\cG)(U)$, which is
$$\vf(V)\:\cF(V)\to\cG(V),\ V\subseteq U$$
that satisfies $\res_{U,U_i}(\vf)=\vf_i$. This means that for open sets $V\subseteq U_i$, it must hold that 
$$\res_{U,U_i}(\vf)(V)=\vf_i(V),\ V\subseteq U_i.$$
For this purpose, we will use the gluing axiom on the sheaf $\cG$. Let us now proceed by taking a section $f\in\cF(V)$ with $V\subseteq U$ and map it through the following diagram:
\begin{center}
    % https://tikzcd.yichuanshen.de/#N4Igdg9gJgpgziAXAbVABwnAlgFyxMJZABgBpiBdUkANwEMAbAVxiRAB12BjAMQAoAagEoQAX1LpMufIRQBGUnKq1GLNp16CNdNAAIAqgH0sI8ZOx4CRAEyLl9Zq0QduAcS3cdB46YkgMFjJECtb2qk4umgLaekZYMd4AVr7m0lYotqHUDmrOGu7RnrHGCUbJYsowUADm8ESgAGYAThAAtkhkIDgQSAoqjursTfAAesAaPKKGwAKkhVxecaIg1Ax0AEYwDAAKUpayIE1Y1QAWOGJ+zW291N1ItiBrmzt7Qc4MMA3n2eGDNA3GDwLYomC6NFrtRB9O6IADMPwGeSGo3G3Em03mi2Mc1KJSKSWWqw2W12gXSh2OZzBICukIeMIALAjci5hnAxvkpjNcVgcfi4rjEsszDSIUh4V0eogmY9iS8yQcjqdzqIKKIgA
\begin{tikzcd}
    \cF(V) \arrow[rd, "{\res^{\cF}_{V,V\cap U_i}}"'] &                                                                                                       &                                                                        \\
                                                     & \cF(V\cap U_i) \arrow[r, "\vf_i(V\cap U_i)"] \arrow[d, "{\res^{\cF}_{V\cap U_i,V\cap U_i\cap U_j}}"'] & \cG(V\cap U_i) \arrow[d, "{\res^{\cG}_{V\cap U_i,V\cap U_i\cap U_j}}"] \\
                                                     & \cF(V\cap U_i\cap U_j) \arrow[r]                                                                      & \cG(V\cap U_i\cap U_j)                                                
    \end{tikzcd}
\end{center}
where the lower arrow is the map $\vf_i(V\cap U_i\cap U_j)$. We can construct a similar diagram of $\vf_j$. A section $f\in\cF(V)$ maps through that diagram as follows: 
\begin{center}
    % https://tikzcd.yichuanshen.de/#N4Igdg9gJgpgziAXAbVABwnAlgFyxMJZABgBpiBdUkANwEMAbAVxiRADMQBfU9TXfIRQBGUsKq1GLNgB0ZAJ3gA9YHIDGAMS4B9YADVSe9XTQACAKrasXABTsAlN14gM2PASIAmMRPrNWiCByNOxWNkYyaiYWVvY2copwKupaugYRUWaW1nb2jjx8boJEop6+UgFBCsqqkan6xllWho0xWK2WAFa2CTUpOvotkdHZtg75zq4CHijeZdR+0oHxMnRwOPkSMFAA5vBEoOzyEAC2SGQgOBBIwgUcx2eIopfXiJ53R6c31FdIAMwfB7-H6vAAsgK+bxBSHBFC4QA
\begin{tikzcd}
    f \arrow[rd] &                                                                                   &                                                           \\
                 & {\res^{\cF}_{V,V\cap U_i}(f)} \arrow[r] \arrow[d]                                 & {\vf_i(V\cap U_i)\bonj{\res^{\cF}_{V,V\cap U_i}(f)}} \arrow[d] \\
                 & {\res^{\cF}_{V\cap U_i,V\cap U_i\cap U_j}\bonj{\res^{\cF}_{V,V\cap U_i}(f)}} \arrow[r] & (\ast)                                                   
    \end{tikzcd}
\end{center}
and the lower right corner is either the restriction of the upper right corner, or the image of the lower left which by $\vf_i(V\cap U_i\cap U_j)$. As the $vf_i$ are morphisms of sheaves, both elements are equal. This can be expressed as follows
\begin{align*}
    &\vf_i(V\cap U_i\cap U_j)\left(\res^{\cF}_{V\cap U_i,V\cap U_i\cap U_j}\bonj{\res^{\cF}_{V,V\cap U_i}(f)}\right)\\
=&\res^{\cG}_{V\cap U_i,V\cap U_i\cap U_j}\left(\vf_i(V\cap U_i)\bonj{\res^{\cF}_{V,V\cap U_i}(f)}\right)
\end{align*}
But, let us simplify notation a bit by remembering that the composition of restriction maps is the beginning-to-end restriction map. This means that 
$$\res^{\cF}_{V\cap U_i,V\cap U_i\cap U_j}\bonj{\res^{\cF}_{V,V\cap U_i}(f)}=\res^{\cF}_{V,V\cap U_i\cap U_j}(f).$$
With this in hand, and remembering the hypothesis that our $\vf_i$'s coincide on intersections of the covering sets, we have:
\begin{align*}
    &\res^{\cG}_{V\cap U_i,V\cap U_i\cap U_j}\left(\vf_i(V\cap U_i)\bonj{\res^{\cF}_{V,V\cap U_i}(f)}\right)\\
    =&\vf_i(V\cap U_i\cap U_j)\bonj{\res^{\cF}_{V,V\cap U_i\cap U_j}(f)}\\
    =&\vf_i(V\cap U_i\cap U_j)\bonj{\res^{\cF}_{V,V\cap U_i\cap U_j}(f)}\\
    =&\res^{\cG}_{V\cap U_j,V\cap U_i\cap U_j}\left(\vf_j(V\cap U_j)\bonj{\res^{\cF}_{V,V\cap U_j}(f)}\right).
\end{align*}
So by gluing the maps $\vf_i(V\cap U_i)\bonj{\res^{\cF}_{V,V\cap U_i}(f)}$ in $\cG$ we may construct a map $g\in\cG(V)$ such that 
$$\res_{V,V\cap U_i}^{\cG}(g)=\vf_i(V\cap U_i)\bonj{\res^{\cF}_{V,V\cap U_i}(f)}$$
for each $i$. We finally define the glued map $\vf$ in $\cH\emph{om}$ which takes our original $f$ to this $g$ which we have found. It follows from our construction that 
$$\res_{U,U_i}(\vf)(V)=\vf_i(V),\ V\subseteq U_i.$$
\end{enumerate}
After verifying the axioms, we may conclude that the sheaf Hom is indeed a sheaf. 
\end{ptcbr}

\begin{Ej}[2.3.F Vakil]
Show that the presheaf cokernel satisfies the universal property of cokernels in the category of presheaves.
 \end{Ej}

 \begin{ptcbr}
    Given a map of presheaves $\vf$, we must show that for $\coker_{\text{pre}}\vf$ given the following diagram:
    \begin{center}
        % https://tikzcd.yichuanshen.de/#N4Igdg9gJgpgziAXAbVABwnAlgFyxMJZABgBoBGAXVJADcBDAGwFcYkQAdDgYwDEQAvqXSZc+QinKli1Ok1bsu3AOKDhIDNjwEiUqjQYs2iTjwgBrGACcA+sC44YADxzA0VmAIFdaAMzUiWuJEAEykIbKGCiYAwoKyMFAA5vBEoL5WEAC2SGQgOBBIAMwG8sYgxCA0jPQARjCMAAqi2hIgVlhJABY4VSD1YFBIALRFxELpmTmIUvmFiCX9MIPFeVHl3ADkASAZ2Uhhc8XVWGDlUPRwXYk7e9OzBQelRuzcfTX1TS3BJowwvr0JrsprkaI9EId1uxKtU6g1mkEdCYOt1Aeo7qCjjNntFTH54gIgA
\begin{tikzcd}
    & \cG \arrow[rdd, "c'", bend left] \arrow[d, "c"] &   \\
\cF \arrow[rrd, "0"', bend right] \arrow[r, "0"'] \arrow[ru, "\vf"] & \coker_{\text{pre}}\vf \arrow[rd, dashed]       &   \\
    &                                                 & \cC
\end{tikzcd}
    \end{center}
    that there exists a unique morphism of presheaves $\psi\:\coker_{\text{pre}}\vf\to \cC$. Taking out particular objects for any open set $U$ we have the same diagram but in terms of objects in the underlying category which is an abelian category. Thus, there exists a unique map 
    $$\psi(U)\:\coker_{\text{pre}}\vf(U)\to\cC(U)$$
    and with this we may define the morphism of presheaves $\coker_{\text{pre}}\vf\to \cC$ by taking each of these maps into our collection of data. This immediately gives us unicity by construction and we are left to check that $\psi$ is a morphism of presheaves. Thi means that for $U\subseteq V$, the following diagram commutes
    \begin{center}
        % https://tikzcd.yichuanshen.de/#N4Igdg9gJgpgziAXAbVABwnAlgFyxMJZABgBpiBdUkANwEMAbAVxiRAB12BjCAaxgBOAfWCccMAB45gaATAC+8zjQBmACgBqAShDzS6TLnyEUARnJVajFm05cAwpp16D2PASLnTl+s1aIObkcAVWd9EAw3YyIyb2pfGwC7PkERMUlpWQUldlU1UN1LGCgAc3giUBUBCABbJDIQHAgkAGZ4639AuTgAPVFuFIF5EQ1SYPkQagY6ACMYBgAFQ3cTEAEsEoALHF1wqtr66iakcys-W3Y0bCdJkGm5xeXogIYYFR2XEH26xDbG5sQACZ2uckpdrgUprN5ksoh4Austh89tUfqdjkCQYkuvARmN5H07PYJvIKPIgA
\begin{tikzcd}
    \coker_{\text{pre}}\vf(V) \arrow[d, "{\res^{\coker}_{V,U}}"'] \arrow[r, "\psi(V)"] & \cC(V) \arrow[d, "{\res_{V,U}^{\cC}}"] \\
    \coker_{\text{pre}}\vf(U) \arrow[r, "\psi(U)"']                                    & \cC(U)                                
    \end{tikzcd}
    \end{center}
    Let us take $f\in\coker_{\text{pre}}\vf(V)$ and see how it maps on both sides of the diagram. However, we are not alone in this endeavor; recall that the cokernel isn't only the object, it's the object \emph{and the epic morphism} $c(V)\:\cG(V)\to \coker_{\text{pre}}\vf(V)$. By this, 
    $$\exists g\in\cG(V)(c(V)(g)=f)$$
    and restricting our view to the upper triangle in the cokernel diagram, we have that 
    $$\psi(V)(f)=\psi(V)\bonj{c(V)(g)}=c'(V)(g).$$
    With this fact in hand, let us map $f$
    \begin{align*}
        \res_{V,U}^\cC\bonj{\psi(V)(f)}=&\res_{V,U}^\cC\bonj{c'(V)(g)}=c'(U)\bonj{\res_{V,U}^\cG(g)}\\
        =&\psi(U)\left(c(U)\bonj{\res_{V,U}^\cG(g)}\right)=\psi(U)\left(\res_{V,U}^{\coker}\bonj{c(V)(g)}\right)\\
        =&\psi(U)\left(\res_{V,U}^{\coker}(f)\right)
    \end{align*}
    where we have liberally used the fact that $c,c'$ are maps of sheaves and thus commute with restrictions. From this chain of equalities we conclude that $\psi$ commutes with restrctions and therefore it's a map of sheaves. This uniquely determines the map $\psi\:\coker_{\text{pre}}\vf\to\cC$ and this means that $\coker_{\text{pre}}\vf$ satisfies the universal property of cokernels in the category of presheaves.
 \end{ptcbr}

\begin{Ej}[2.3.H Vakil]
    Show that a sequence of presheaves $0\to\cF_1\to\dots\to\cF_n\to 0$ is exact if and only if $0\to\cF_1(U)\to\dots\to\cF_n(U)\to 0$ is exact for $U\subseteq X$.
\end{Ej}

\red{UNFINISHED}
\begin{Ej}[2.3.I Vakil]
    Suppose $\vf\:\cF\to\cG$ is a morphism of \emph{sheaves}.
    \begin{enumerate}[i)]
        \itemsep=-0.4em
        \item Show that the presheaf kernel $\ker_{\text{pre}}\vf$ is in fact a sheaf. 
        \item Show that it satisfies the universal property of kernels.
    \end{enumerate}
    \hint{The second question follows immediately from the fact that $\ker_{\text{pre}}\vf$ satisfies the universal property in the category of \emph{presheaves}.}
\end{Ej}

\begin{ptcbr}
    \footnote{If time permits we will show that the presheaf kernel is also a presheaf.} We must show that the presheaf kernel satisfies the two sheaf axioms:
    \begin{enumerate}[i)]
        \itemsep=-0.4em
        \item Let $U\subseteq X$ be an open set with $(U_i)$ an open cover of $U$. Suppose $f,g\in\ker_{\text{pre}}\vf(U)$ which coincide in every covering set. The following diagram is used\footnote{In the exercise to show kernel is presheaf.} to define the restriction mapping on the presheaf kernel:
        \begin{center}
        % https://tikzcd.yichuanshen.de/#N4Igdg9gJgpgziAXAbVABwnAlgFyxMJZABgBpiBdUkANwEMAbAVxiRGJAF9T1Nd9CKAIzkqtRizYAdKQGsYAJwD6wGThgAPHMDQKYnTjJoAzABQBVAJRceIDNjwEiIoWPrNWiEDPnLVU9S0dPQMjM3MlLGtuXgcBIgAmUWp3SS8ZAGMAMQto23t+JxQAZmTxD2kpDIBxXJtYwsFkJNcUiU9vKpyIqPq7Pkcm0tbytM6ai0i8hsGiMhHUjo5OMRgoAHN4IlBjBQgAWyQyEBwIJCEYkF2D8+pTpFLRjpl8HDo+68PER-vEABY2hV0lITHVLp8kABWO5nRAANkBYzCk161AYdAARjAGAAFAbxLwKLDrAAWOA+ey+SROsOhT0qrzokRAaMx2LxcSKICJpPJ4MpSAA7DCkAl+TdECIaaK0VgwB0oHQ4CS1iz6cDNFg4Dg4ABCCkSn60xHPKR6OAqcykHqcAB6mSyBq+AOl8JNlXNlutkTtmWqXAonCAA
        \begin{tikzcd}
            0 \arrow[r] & \ker_{\text{pre}}\vf(U) \arrow[r, "\iota"] \arrow[d, "\exists!", dashed] & \cF(U) \arrow[r, "\vf(U)"] \arrow[d, "{\res_{U,U_i}^\cF}"] & \cG(U) \arrow[d, "{\res_{U,U_i}^\cG}"] \\
            0 \arrow[r] & \ker_{\text{pre}}\vf(U_i) \arrow[r, "\iota_i"']                          & \cF(U_i) \arrow[r, "\vf(U_i)"']                            & \cG(U_i)                              
        \end{tikzcd}
        \end{center}
        So let us assume that for all $i$, we have $\res^{\ker}_{U,U_i}(f)=\res^{\ker}_{U,U_i}(g)$. We can include them into $\cF(U_i)$ with $\iota_i$ show that we have 
        $$\iota_i\bonj{\res^{\ker}_{U,U_i}(f)}=\iota_i\bonj{\res^{\ker}_{U,U_i}(g)}$$
        but as we have (assumed) that $\ker_{\text{pre}}\vf$ is a presheaf, the left square commutes. So we have 
        $$\res^{\cF}_{U,U_i}\left(\iota(f)\right)=\res^{\cF}_{U,U_i}\left(\iota(g)\right)$$
        which by the identiy axiom on $\cF$, we have that $\iota(f)=\iota(g)$. As $\iota$ is injective we have that $f=g$, verifying the identity axiom on $\ker_{\text{pre}}\vf$.
        \item Once again consider an open cover $(U_i)$ of $U\subseteq X$ with $f_i\in \ker_{\text{pre}}\vf(U_i)$ for each $i$. Assume that for all $i,j$ we have 
        $$\res^{\ker}_{U_i,U_i\cap U_j}(f_i)=\res^{\ker}_{U_j,U_i\cap U_j}(f_j)$$
        then, using the corresponding inclusion map which $\iota_{ij}\:\ker_{\text{pre}}\vf(U_i\cap U_j)\to\cF(U_i\cap U_j )$ we get 
        $$\iota_{ij}\bonj{\res^{\ker}_{U_i,U_i\cap U_j}(f_i)}=\iota_{ij}\bonj{\res^{\ker}_{U_j,U_i\cap U_j}(f_j)}$$
        which leads us to 
        $$\res^{\cF}_{U_i,U_i\cap U_j}\bonj{\iota_i(f_i)}=\res^{\cF}_{U_i,U_i\cap U_j}\bonj{\iota_j(f_j)}$$
        by commutitativy of the left square of the following diagram (and a similar one for $j$):
        \begin{center}
            % https://tikzcd.yichuanshen.de/#N4Igdg9gJgpgziAXAbVABwnAlgFyxMJZABgBpiBdUkANwEMAbAVxiRGJAF9T1Nd9CKAIzkqtRizYAdKQGsYAJwD6wGThgAPHMDQKYnTjJoAzABQBVJVgCUXHiAzY8BIiKFj6zVohAz5y1Sl1LR09AyMzSywZAGM6NAACSwArW25eJwEiACZRak9JH1iAMQsrNPtHfhcUAGY88S9pKRiAcTKbOwzqwWRc93yJb18W0qjY+KSlVK6HPmde+oHGwpG2jonElIruhaIyZYLhjk4xGCgAc3giUGMFCABbJDIQHAgkIXSQO8eP6jekPUVsMZPgcHQrLMfk9EECAYgACyDJpFKQmDo7b73GEAVn+70QADZkasIhsWpNtiBqAw6AAjGAMAAK8yyPgUWAuAAscFDsUhcq8CXjgc0wRDgFhkpxqSBaQzmayaiAOdzeV9oUgAOz4gUa-mIERCgU0rBgYZwCAMLBQWVHZp6OAqKKkcYUrbTTgAPUC-hl+t+sN1iBF9tRjudVldVk2U2lXpKfMDSONRJJIKkEeALrdcQ98dirS4FE4QA
\begin{tikzcd}
    0 \arrow[r] & \ker_{\text{pre}}\vf(U_i) \arrow[r, "\iota_i"] \arrow[d, "{\res_{U_i,U_i\cap U_j}^{\ker}}"] & \cF(U_i) \arrow[r, "\vf(U_i)"] \arrow[d, "{\res_{U_i,U_i\cap U_j}^\cF}"] & \cG(U_i) \arrow[d, "{\res_{U_i,U_i\cap U_j}^\cG}"] \\
    0 \arrow[r] & \ker_{\text{pre}}\vf(U_i\cap U_j) \arrow[r, "\iota_{ij}"']                                  & \cF(U_i\cap U_j) \arrow[r, "\vf(U_i\cap U_j)"']                          & \cG(U_i\cap U_j)                                  
    \end{tikzcd}
        \end{center}
        Gluing inside $\cF$ we get $\widetilde{f}\in\cF(U)$ such that $\res_{U,U_i}^{\cF}(\widetilde{f})=\iota_i(f_i)$. Mapping $\tilde{f}$ through $\vf(U)$ we can restrict to the covering set to get 
        $$\res_{U,U_i}^{\cG}\bonj{\vf(U)(\tilde{f})}=\vf(U_i)\bonj{\res_{U,U_i}^{\cF}(\tilde{f})}=\vf(U_i)\bonj{\iota_i(f_i)}=0$$
        which means that $\vf(U)(\tilde{f})$ restricts to 0. As $\cG$ is a sheaf, it must occur that $\vf(U)(\tilde{f})=0$ and therefore by exactness of the kernel we find $f\in\ker_{\text{pre}}\vf(U)$ such that $\iota(f)=\tilde{f}$. Such $f$ is the desired element which satisfies the gluing axiom for $\ker_{\text{pre}}\vf$.
    \end{enumerate}
\end{ptcbr}
\begin{Ej}[2.4.C Vakil]
    If $\vf,\psi$ are morphisms from a presheaf of sets $\cF$ to a sheaf of sets $\cG$ that induce the same maps on each stalk, show that $\vf=\psi$. As a hint consider the following diagram:
    \begin{center}
        % https://tikzcd.yichuanshen.de/#N4Igdg9gJgpgziAXAbVABwnAlgFyxMJZABgBpiBdUkANwEMAbAVxiRAB12BjAMQAoAqgEoQAX1LpMufIRQBGclVqMWbTlwDigkeMnY8BImTlL6zVog7s0AJ2gB9YGk5YwAAgGj1Pe2jESQDH0ZIgUTajNVS05bBycXd091DV8xJRgoAHN4IlAAMzsAWyQyEBwIJAAmXRACiGLESupypABmGrqG0pbEOQ6ipAUyisRW5rosBjYACwgIAGsQagY6ACMYBgAFKQNZEAYYPJw00SA
\begin{tikzcd}
    \cF(U) \arrow[d] \arrow[r]    & \cG(U) \arrow[d, hook] \\
    \prod_{p\in U}\cF_p \arrow[r] & \prod_{p\in U}\cG_p   
    \end{tikzcd}
    \end{center}
\end{Ej}

\red{UNFINISHED}

\end{document} 