\documentclass[12pt]{memoir}

\def\nsemestre {I}
\def\nterm {Spring}
\def\nyear {2023}
\def\nprofesor {Mark Shoemaker}
\def\nsigla {MATH673}
\def\nsiglahead {Algebraic Geometry}
\def\nextra {HW6}
\def\nlang {ENG}
\input{../../headerVarillyDiff}
\DeclareMathOperator{\sh}{sh}
\begin{document}
%\begin{multicols}{2}

    \begin{Ej}[4.1.A Vakil]
        Show that the natural map $A_f\to\cO_{\Spec(A)}(D(f))$ is an isomorphism. \hint{Exercise 3.5.E Vakil.}
    \end{Ej}
    %NEUR

    First let us recall that Exercise 3.5.E is the following:

    \begin{Lem}
        The next statements are equivalent:
        \begin{enumerate}[i)]
            \itemsep=-0.4em
            \item $D(f)\subseteq D(g)$.
            \item $\exists n(n\geq 1\To f^n\in\gen(g))$.
            \item $g$ is an invertible element of $A_f$.
        \end{enumerate}
    \end{Lem}

    We have proven this in class so let us make a quick recapitulation.

    \begin{ptcb}
        The first two statements are equivalent because 
        \begin{align*}
            D(f)\subseteq D(g)&\iff V(g)\subseteq V(f)\\
            &\iff \set{\lie p\:\ \gen(g)\subseteq\lie p}\subseteq \set{\lie p\:\ \gen(f)\subseteq\lie p}
        \end{align*}
        The last statement can be rephrased as \emph{if a prime contains $g$, then it also contains $f$}. In particular this equivalent to saying 
        \begin{align*}
        &f\in\bigcap_{g\in\lie p}\lie p=\sqrt{\gen(g)}\\
        \iff &\exists n(n\geq 1\To f^n\in\gen(g)).
        \end{align*}
        For the last two statements, we first assume $g$ is invertible in $A_f$. This means that there exists an $n$ such that
        $$\left(\frac{g}{1}\right)\left(\frac{a}{f^n}\right)=\frac{1}{1}.$$
        Recall that the equality condition in the localization means that there exists and element $f^m$ with $m\geq 1$ which is invertible in $A_f$ such that 
        $$f^m(ag-f^n)=0\To agf^m=f^{m+n}.$$
        This last equation is in $A$ without localizing, and the term on the right, $agf^m$, is in $\gen(g)$. Thus the power we were searching for is $m+n$ and $f^{m+n}\in\gen(g)$.\par 
        On the other direction, if $f^n\in\gen(g)$ for some $n\geq 1$, then there is an $a\in A$ such that 
        $$f^n=ag,$$
        and localizing at $f$ turns this equation into $\frac{1}{g}=\frac{a}{f^n}$. 
    \end{ptcb}
    \begin{ptcbr}
    We begin by recalling the definition of $\cO_{\Spec(A)}(D(f))$, we have 
    $$\cO_{\Spec(A)}(D(f))=S^{-1}A,\word{where}S=\set{g\in A\: D(f)\subseteq D(g)}.$$
    By the lemma we can rewrite $S$ as
    $$S=\set{g\in A\: \exists n(f^n\in\gen(g))}.$$
    Now notice that when localizing at $S$ we are able to invert $f^n$ for some $n$. From this we have that $f$ is also invertible in $S^{-1}A$ because 
    $$f^ng=u\To f(f^{n-1}g)=u\To f\text{ is invertible}.$$
    This means that localizing at $S$ is a further localization of $A$ at $f$ because we have already inverted all powers of $f$.\par 
    Notice however that this isn't adding anything new to $A_f$, because of the last equivalence of the lemma. Every $g$ such that $D(f)\subseteq D(g)$ is already invertible in $A_f$. We conclude that the inclusion is actually an isomorphism.
    \end{ptcbr}

    \begin{Ej}[Restrictions]
        Do the following:
        \begin{enumerate}[i)]
            \itemsep=-0.4em
            \item Explain, using Definition 4.1.1 (and not exercise 4.1.A) what the restriction map is.
            \item Explain, using exercise 4.1.A what the restriction map is.
        \end{enumerate}
    \end{Ej}
    
    \begin{ptcbr}
    \begin{enumerate}[i)]
        \itemsep=-0.4em
        \item Recall that 
    $$\cO_{\Spec(A)}(D(f))=(S^f)^{-1}A,\word{where}S^f=\set{h\in A\: D(f)\subseteq D(h)}$$
    and on the same vein the set associated to $D(g)$ is the localization at $S^g=\set{h\in A\: D(g)\subseteq D(h)}$. So if we take $D(f)\subseteq D(g)$ then the restriction map is a function 
    $$\res_{D(g),D(f)}\cO_{\Spec(A)}(D(g))\to\cO_{\Spec(A)}(D(f)).$$
    \item Using the previous exercise we have the isomorphism between localizing at $S^f$ and localizing at powers of $f$. So once again let us assume that $D(f)\subseteq D(g)$, then the restriction map is a function 
%TO CONTAIN IS TO DIVIDE
    $$\res_{D(g),D(f)}A_f\to A_g.$$
    In this case we have an element $\frac{a}{f^n}$ which is being mapped to 

    \end{enumerate}
    \end{ptcbr}

    \begin{Ej}[4.1.D Vakil]
        Suppose $M$ is an $A$-module. Show
that the following construction describes a sheaf $\widetilde{M}$ on the distinguished base. Define $\widetilde{M}(D(f))$ to be the localization of $M$ at the multiplicative set of all functions that do not vanish outside of $V(f)$.\par
 Define restriction maps $\res_{D(f),D(g)}$ in the analogous way to $\cO_{\Spec(A)}$.\par
  Show that this defines a sheaf on the distinguished base, and hence a sheaf on $\Spec(A)$. Then show that this is an $\cO_{\Spec(A)}$-module.
    \end{Ej}
    
    \begin{ptcbr}
    
    \end{ptcbr}

    \begin{Ej}
        Let $A = \bC[x,y]$ and let $\lie p=\gen(y)$, viewed as a point of $X = \Spec(A)$. What is $\cO_{X,p}$?\par
        Recall that $\cO_{X,p}$ is a local ring, that is, it has a unique maximal ideal, $\lie m_p$.\par 
        What is the residue field $\kp_{\lie p}=\cO_{X,p}/\lie m_p$?
    \end{Ej}
    
    \begin{ptcbr}
    
    \end{ptcbr}

    \begin{Ej}[4.4.A Vakil]
        Show that you can glue an arbitrary collection of schemes together. Suppose we are given:
        \begin{itemize}
            \itemsep=-0.4em
            \item  schemes $X_i$ (as $i$ runs over some index set $I$, not necessarily finite),
            \item open subschemes $X_{ij}\subseteq X_i$ with $X_{ii}=X_i$,
            \item isomorphisms $f_{ij}\:X_{ij}\to X_{ji}$ with $f_{ii}$ the identity
        \end{itemize}
        such that 
        \begin{significant}
            the isomorphisms “agree on triple intersections”,
            i.e.,
            $$f_{ik}\mid_{X_{ij}\cap X_{ik}}= f_{jk}\mid_{X_{ji}\cap X_{jk}}\circ f_{ij}\mid_{X_{ij}\cap X_{ik}}\circ$$
            (so implicitly, to make sense of
            the right side, $f_{ij}(X_{ik}\cap X_{ij})\subseteq X_{jk}$).            
        \end{significant}
        This \emph{cocycle condition} ensures that $f_{ij}$ and $f_{ji}$ are inverses. In fact, the hypothesis that $f_{ii}$ is the identity also follows from the cocycle condition.\par 
        Show that there is a unique scheme $X$ (up to unique isomorphism) along with open subsets isomorphic to the $X_i$ respecting this gluing data in the obvious sense.\hint{what is $X$ as a set? What is the topology on this set? In terms of your description of the open sets of $X$, what are the sections of this sheaf over each open set?}
    \end{Ej}
    
    \begin{ptcbr}
    
    \end{ptcbr}
        

\end{document} 