\documentclass[12pt]{memoir}

\def\nsemestre {I}
\def\nterm {Spring}
\def\nyear {2023}
\def\nprofesor {Mark Shoemaker}
\def\nsigla {MATH673}
\def\nsiglahead {Algebraic Geometry}
\def\nextra {HW6}
\def\nlang {ENG}
\input{../../headerVarillyDiff}
\DeclareMathOperator{\sh}{sh}
\begin{document}
%\begin{multicols}{2}

    \begin{Ej}[4.1.A Vakil]
        Show that the natural map $A_f\to\cO_{\Spec(A)}(D(f))$ is an isomorphism. \hint{Exercise 3.5.E Vakil.}
    \end{Ej}
    %NEUR

    First let us recall that Exercise 3.5.E is the following:

    \begin{Lem}
        The next statements are equivalent:
        \begin{enumerate}[i)]
            \itemsep=-0.4em
            \item $D(f)\subseteq D(g)$.
            \item $\exists n(n\geq 1\To f^n\in\gen(g))$.
            \item $g$ is an invertible element of $A_f$.
        \end{enumerate}
    \end{Lem}

    We have proven this in class so let us make a quick recapitulation.

    \begin{ptcb}
        The first two statements are equivalent because 
        \begin{align*}
            D(f)\subseteq D(g)&\iff V(g)\subseteq V(f)\\
            &\iff \set{\lie p\:\ \gen(g)\subseteq\lie p}\subseteq \set{\lie p\:\ \gen(f)\subseteq\lie p}
        \end{align*}
        The last statement can be rephrased as \emph{if a prime contains $g$, then it also contains $f$}. In particular this equivalent to saying 
        \begin{align*}
        &f\in\bigcap_{g\in\lie p}\lie p=\sqrt{\gen(g)}\\
        \iff &\exists n(n\geq 1\To f^n\in\gen(g)).
        \end{align*}
        For the last two statements, we first assume $g$ is invertible in $A_f$. This means that there exists an $n$ such that
        $$\left(\frac{g}{1}\right)\left(\frac{a}{f^n}\right)=\frac{1}{1}.$$
        Recall that the equality condition in the localization means that there exists and element $f^m$ with $m\geq 1$ which is invertible in $A_f$ such that 
        $$f^m(ag-f^n)=0\To agf^m=f^{m+n}.$$
        This last equation is in $A$ without localizing, and the term on the right, $agf^m$, is in $\gen(g)$. Thus the power we were searching for is $m+n$ and $f^{m+n}\in\gen(g)$.\par 
        On the other direction, if $f^n\in\gen(g)$ for some $n\geq 1$, then there is an $a\in A$ such that 
        $$f^n=ag,$$
        and localizing at $f$ turns this equation into $\frac{1}{g}=\frac{a}{f^n}$. 
    \end{ptcb}
    \begin{ptcbr}
    We begin by recalling the definition of $\cO_{\Spec(A)}(D(f))$, we have 
    $$\cO_{\Spec(A)}(D(f))=S^{-1}A,\word{where}S=\set{g\in A\: D(f)\subseteq D(g)}.$$
    By the lemma we can rewrite $S$ as
    $$S=\set{g\in A\: \exists n(f^n\in\gen(g))}.$$
    Now notice that when localizing at $S$ we are able to invert $f^n$ for some $n$. From this we have that $f$ is also invertible in $S^{-1}A$ because 
    $$f^ng=u\To f(f^{n-1}g)=u\To f\text{ is invertible}.$$
    This means that localizing at $S$ is a further localization of $A$ at $f$ because we have already inverted all powers of $f$.\par 
    Notice however that this isn't adding anything new to $A_f$, because of the last equivalence of the lemma. Every $g$ such that $D(f)\subseteq D(g)$ is already invertible in $A_f$. We conclude that the inclusion is actually an isomorphism.
    \end{ptcbr}

    \begin{Ej}[Restrictions]
        Do the following:
        \begin{enumerate}[i)]
            \itemsep=-0.4em
            \item Explain, using Definition 4.1.1 (and not exercise 4.1.A) what the restriction map is.
            \item Explain, using exercise 4.1.A what the restriction map is.
        \end{enumerate}
    \end{Ej}
    
    \begin{ptcbr}
        \iffalse
Let us begin by calling 
$$S^f=\set{h\in A\:\ D(f)\subseteq D(h)}$$
and likewise for $S^g$. Using this set, our definition for the structure sheaf is $\cO_{\Spec(A)}(D(f))=(S^f)^{-1}A$.\par 
Recall that from the previous exercise we have that $D(f)\subseteq D(h)$ is equivalent to saying that $h$ is invertible in $A_f$:
$$\frac{1}{h}=\frac{a}{f^n},\word{for}n\in\bN\word{and}a\in A.$$
So in terms of a relation between $S^f$ and $S^g$, given that $D(f)\subseteq D(g)$, we can see that $S^g\subseteq S^f$:
$$h\in S^g\To \frac1h=\frac{a}{g^n}\To \frac1h=a\left(\frac{b}{f^m}\right)^n=\frac{ab^n}{f^{mn}}\To h\in S^f.$$
Thus when localizing we have $(S^g)^{-1}A\supseteq (S^f)^{-1}A$, which means that it does make sense to restrict from $\cO_{\Spec(A)}(D(g))$ to $\cO_{\Spec(A)}(D(f))$.
\fi
    \begin{enumerate}[i)]
        \itemsep=-0.4em
        \item Recall that 
    $$\cO_{\Spec(A)}(D(f))=(S^f)^{-1}A,\word{where}S^f=\set{h\in A\: D(f)\subseteq D(h)}$$
    and on the same vein the set associated to $D(g)$ is the localization at $S^g=\set{h\in A\: D(g)\subseteq D(h)}$. So if we take $D(f)\subseteq D(g)$, we can inject $A$ into both localizations via $a\mapsto\frac{a}{1}$ as follows:
    \begin{center}
        % https://tikzcd.yichuanshen.de/#N4Igdg9gJgpgziAXAbVABwnAlgFyxMJZABgBpiBdUkANwEMAbAVxiRAEEQBfU9TXfIRQBGclVqMWbABQBlAHoAzAJTzgAWmFdOPPtjwEio4ePrNWiEHPkBzVRq07xMKDfhFQigE4QAtkjIQHAgkUQlzNgAdSJpFAH1Fbl4Qbz8A6mCkACZqMylLaNi4mxBqBjoAIxgGAAV+AyEQLywbAAscbgouIA
\begin{tikzcd}
    A \arrow[r, "\vf_f"] \arrow[rd, "\vf_g"'] & (S^f)^{-1}A \\
                                              & (S^g)^{-1}A
    \end{tikzcd}
    \end{center}
    Now for elements $h\in S^g$, $\vf^f(h)$ is an invertible element in $(S^f)^{-1}A$ because $D(f)\subseteq D(g)$. So, by universality of the localization, we have that there exists a unique map 
    $$(S^g)^{-1}A\to(S^f)^{-1}A$$
    and such map is the desired restriction map.
    \item Using the previous exercise we have the isomorphism between localizing at $S^f$ and localizing at powers of $f$. So once again let us assume that $D(f)\subseteq D(g)$, then the restriction map is a function 
%TO CONTAIN IS TO DIVIDE
    $$\res_{D(g),D(f)}A_g\to A_f.$$
    In this case we have an element $\frac{a}{h^n}$ with $h\in S^g$. Recall that this means that $h$ is invertible in $A_g$ so we may write 
    $$\frac{1}{h}=\frac{b}{g^m},\word{where}b\in A,\ m\in\bN.$$
    But now, as $D(f)\subseteq D(g)$, $g$ is invertible in $A_f$ so once again we have 
    $$\frac{1}{g}=\frac{c}{f^r},\word{where}c\in A,\ r\in\bN.$$
    Combining these facts we have 
    $$\frac{a}{h^n}=a\frac{b^n}{g^{mn}}=ab^n\frac{c^{mn}}{f^{mnr}},$$
    and so this element inside $A_f$ is where we map our original element to.
    \end{enumerate}
    \end{ptcbr}

    \begin{Ej}[4.1.D Vakil]
        Suppose $M$ is an $A$-module. Show
that the following construction describes a sheaf $\widetilde{M}$ on the distinguished base. Define $\widetilde{M}(D(f))$ to be the localization of $M$ at the multiplicative set of all functions that do not vanish outside of $V(f)$.\par
 Define restriction maps $\res_{D(f),D(g)}$ in the analogous way to $\cO_{\Spec(A)}$.\par
  Show that this defines a sheaf on the distinguished base, and hence a sheaf on $\Spec(A)$. Then show that this is an $\cO_{\Spec(A)}$-module.
    \end{Ej}
    
    \begin{ptcbr}
    We are now considering the following space $(\Spec(A),\widetilde{M})$, in other words we are endowing $\Spec(A)$ with sheaf of $A$-modules.\par 
    We first verify that $\widetilde{M}$ is a presheaf. In the same way we defined the sheaf on the basis elements we have 
    $$\widetilde{M}(D(f))=(S^f)^{-1}M,\word{where}S^f=\set{h\in A\: D(f)\subseteq D(h)}.$$
    The restriction maps are defined in the same fashion by universality, so if we have the diagram 
    $$% https://tikzcd.yichuanshen.de/#N4Igdg9gJgpgziAXAbVABwnAlgFyxMJZABgBpiBdUkANwEMAbAVxiRAFkQBfU9TXfIRQBGclVqMWbABQBlAHoAzAJTzgAWmFdOPPtjwEio4ePrNWiEHKWqNWneJhQA5vCKhFAJwgBbJGRAcCCRRCXM2AB0ImkUAfUVuXhAvX39qIKQAJmozKUsomPiQagY6ACMYBgAFfgMhEE8sZwALHG4KLiA
    \begin{tikzcd}
    M \arrow[r, "\vf_f"] \arrow[rd, "\vf_f"'] & (S^f)^{-1}M \\
                                              & (S^f)^{-1}M
    \end{tikzcd}$$
    then there's a restriction map from each localization to another. By universality they must be the same arrow, and as the only arrow which goes from $(S^f)^{-1}M$ to itself is the identity we have 
    $$\res_{D(f),D(f)}=\id_{(S^f)^{-1}M}.$$
    Now, we have the composition of restriction maps is the longer restriction map. Consider the following diagram:
    \begin{center}
        % https://tikzcd.yichuanshen.de/#N4Igdg9gJgpgziAXAbVABwnAlgFyxMJZABgBoBGAXVJADcBDAGwFcYkQBZEAX1PU1z5CKchWp0mrdgAoAygD0A5gEp5wALTluXXv2x4CRUQCZxDFm0Qg58gBaqNWnXxAZ9Qo6WJnJl6woAzB01tHnEYKEV4IlAAgCcIAFskMhAcCCRRCQt2AB1c2gCAfUUeF3iklJp0pGMacykrfMKi2xAaRnoAIxhGAAUBA2EQOKxFWxwy2ITkxFSaxABmet88guKAqZAK2bq0jMQtcpnM6oPlkEYsMD8oejhbCK2d2rPTy+vb+8eoZ5PEPYLC49MC-RDqAAsAE4Op92HcHk9uJRuEA
\begin{tikzcd}
    & (S^f)^{-1}M                                                               \\
M \arrow[r, "\vf_g"] \arrow[rd, "\vf_h"'] \arrow[ru, "\vf_f"] & (S^g)^{-1}M \arrow[u, dashed]                                             \\
    & (S^h)^{-1}M \arrow[u] \arrow[u, dashed] \arrow[uu, dashed, bend right=49]
\end{tikzcd}
    \end{center}
We have that the composition exists by universality as do the smaller 
arrow. Universality guarantees that they are the same arrow and thus 
$$\res_{D(h),D(f)}=\res_{D(g),D(f)}\circ\res_{D(h),D(g)}.$$
Let us now verify the sheaf axioms. We begin by considering an open cover of $\Spec(A)$ which we can reduce to a finite sub-cover by quasi-compactness, this is 
$$\Spec(A)=\bigcup_{i\in I}D(f_i)=\bigcup_{i=1}^{n}D(f_i).$$
\begin{itemize}
    \item Recall that our sheaf sets are modules, so injectivity is equivalent to kernels being trivial. Thus to verify the identity axiom, we consider $s\in\widetilde{M}(\Spec(A))$ with the assumption that $\res_{D(f_i)}(s)=0$. We wish to show $s=0$.\par 
    The restriction lives in one our localization, this means that 
    $$\res_{D(f_i)}(s)\in\widetilde{M}(D(f_i))=(S^{f_i})^{-1}M\To\res_{D(f_i)}(s)=\frac{m}{g},\word{where}g\in S^{f_i}.$$
    Now this fraction is zero, which means that there's an invertible element $u\in S^{f_i}$, which is $\frac{1}{u}=\frac{a}{f_i^{s}}$, that satisfies
    $$u(1\.m-0.g)=0\To aum=a\.0=0\To f_i^{s}m=0.$$
    This last proposition holds for all $i$, and as we have $\genr{f_1^s,\dots,f_n^s}=A$ we have that $1=\sum c_if_i^s$. As $M$ is an $A$-module, the next equation holds in $M$:
    $$m=\left(\sum_{i=1}^{n}c_if_i^{s}\right)m=\sum_{i=1}^{n}c_i(f_i^{s}m)=\sum0=0.$$
    Thus $m=0$ and we have the identity axiom\footnote{As 4.1.B mentions that it is possible to replace $D(f_i)$ by $D(f)$, thus generalizing the argument, I don't see how it is possible in the proof of the theorem nor in this exercise. This is because if don't have the finite number of $f_i$'s then we can't say that $1$ is the linear combination that we have recovered.}.
    \item We proceed as in the proof of the gluing axiom for the structure sheaf. First by taking a finite set of indices and then generalizing to an infinite set of indices.\par 
    We take sections $(s_i)$ such that 
    $$\res_{D(f_i),D(f_i)\cap D(f_j)}(s_i)=\res_{D(f_j),D(f_i)\cap D(f_j)}(s_j)\word{for all}i,j.$$
    The set $\tilde{M}(D(f_i)\cap D(f_j))$ is the localization of $M$ at $(S^{f_i}\cap S^{f_j})$. (\red{Couldn't finish this one, wrapping my head around this localization was a bit jarring. Would it be possible to discuss later?})
\end{itemize}
    \end{ptcbr}

    \begin{Ej}
        Let $A = \bC[x,y]$ and let $\lie p=\gen(y)$, viewed as a point of $X = \Spec(A)$. What is $\cO_{X,p}$?\par
        Recall that $\cO_{X,p}$ is a local ring, that is, it has a unique maximal ideal, $\lie m_p$.\par 
        What is the residue field $\kp_{\lie p}=\cO_{X,p}/\lie m_p$?
    \end{Ej}
    
    \begin{ptcbr}
    The set $\cO_{X,\lie p}$ is the stalk of the structure sheaf at the point $\lie p\in \Spec A$. Germs inside $\cO_{X,\lie p}$ are equivalence classes of pairs $(f,D(g))$ where $\lie p\in D(g)$ and $f\in\cO_X(D(g))$.\par 
    Recall that in our case 
    $$\cO_X(D(g))\isom A_g=\bC[x,y]_g$$
    so the germs are equivalence classes of rational functions with denominators $g^r$ about $\gen(y)$ which don't vanish.\par 
    The maximal ideal of the stalk is the germs which vanish at $\gen(y)$. Recall that $f$ vanishes at $\gen(y)$ when $f\bmod\gen(y)=0$. So $f$ musn't have any multiples of $y$. The only thing we are left with is rational functions on $x$ so it must hold that $\kp_{\lie p}\isom \bC(x)$.
    \end{ptcbr}

    \begin{Ej}[4.4.A Vakil]
        Show that you can glue an arbitrary collection of schemes together. Suppose we are given:
        \begin{itemize}
            \itemsep=-0.4em
            \item  schemes $X_i$ (as $i$ runs over some index set $I$, not necessarily finite),
            \item open subschemes $X_{ij}\subseteq X_i$ with $X_{ii}=X_i$,
            \item isomorphisms $f_{ij}\:X_{ij}\to X_{ji}$ with $f_{ii}$ the identity
        \end{itemize}
        such that 
        \begin{significant}
            the isomorphisms “agree on triple intersections”,
            i.e.,
            $$f_{ik}\mid_{X_{ij}\cap X_{ik}}= f_{jk}\mid_{X_{ji}\cap X_{jk}}\circ f_{ij}\mid_{X_{ij}\cap X_{ik}}\circ$$
            (so implicitly, to make sense of
            the right side, $f_{ij}(X_{ik}\cap X_{ij})\subseteq X_{jk}$).            
        \end{significant}
        This \emph{cocycle condition} ensures that $f_{ij}$ and $f_{ji}$ are inverses. In fact, the hypothesis that $f_{ii}$ is the identity also follows from the cocycle condition.\par 
        Show that there is a unique scheme $X$ (up to unique isomorphism) along with open subsets isomorphic to the $X_i$ respecting this gluing data in the obvious sense.\hint{what is $X$ as a set? What is the topology on this set? In terms of your description of the open sets of $X$, what are the sections of this sheaf over each open set?}
    \end{Ej}
    \iffalse
    \begin{ptcbr}
    
    \end{ptcbr}
        \fi

\end{document} 