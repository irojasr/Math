\documentclass[12pt]{memoir}

\def\nsemestre {I}
\def\nterm {Spring}
\def\nyear {2023}
\def\nprofesor {Mark Shoemaker}
\def\nsigla {MATH673}
\def\nsiglahead {Algebraic Geometry}
\def\nextra {HW6}
\def\nlang {ENG}
\input{../../headerVarillyDiff}
\DeclareMathOperator{\sh}{sh}
\begin{document}
%\begin{multicols}{2}

    \begin{Ej}[4.1.A Vakil]
        Show that the natural map $A_f\to\cO_{\Spec(A)}(D(f))$ is an isomorphism. \hint{Exercise 3.5.E Vakil.}
    \end{Ej}
    %NEUR
    \begin{ptcbr}
    
    \end{ptcbr}

    \begin{Ej}[Restrictions]
        Do the following:
        \begin{enumerate}[i)]
            \itemsep=-0.4em
            \item Explain, using Definition 4.1.1 (and not exercise 4.1.A) what the restriction map is.
            \item Explain, using exercise 4.1.A what the restriction map is.
        \end{enumerate}
    \end{Ej}
    
    \begin{ptcbr}
    
    \end{ptcbr}

    \begin{Ej}[4.1.D Vakil]
        Suppose $M$ is an $A$-module. Show
that the following construction describes a sheaf $\widetilde{M}$ on the distinguished base. Define $\widetilde{M}(D(f))$ to be the localization of $M$ at the multiplicative set of all functions that do not vanish outside of $V(f)$.\par
 Define restriction maps $\res_{D(f),D(g)}$ in the analogous way to $\cO_{\Spec(A)}$.\par
  Show that this defines a sheaf on the distinguished base, and hence a sheaf on $\Spec(A)$. Then show that this is an $\cO_{\Spec(A)}$-module.
    \end{Ej}
    
    \begin{ptcbr}
    
    \end{ptcbr}

    \begin{Ej}
        Let $A = \bC[x,y]$ and let $\lie p=\gen(y)$, viewed as a point of $X = \Spec(A)$. What is $\cO_{X,p}$?\par
        Recall that $\cO_{X,p}$ is a local ring, that is, it has a unique maximal ideal, $\lie m_p$.\par 
        What is the residue field $\kp_{\lie p}=\cO_{X,p}/\lie m_p$?
    \end{Ej}
    
    \begin{ptcbr}
    
    \end{ptcbr}

    \begin{Ej}[4.4.A Vakil]
        Show that you can glue an arbitrary collection of schemes together. Suppose we are given:
        \begin{itemize}
            \itemsep=-0.4em
            \item  schemes $X_i$ (as $i$ runs over some index set $I$, not necessarily finite),
            \item open subschemes $X_{ij}\subseteq X_i$ with $X_{ii}=X_i$,
            \item isomorphisms $f_{ij}\:X_{ij}\to X_{ji}$ with $f_{ii}$ the identity
        \end{itemize}
        such that 
        \begin{significant}
            the isomorphisms “agree on triple intersections”,
            i.e.,
            $$f_{ik}\mid_{X_{ij}\cap X_{ik}}= f_{jk}\mid_{X_{ji}\cap X_{jk}}\circ f_{ij}\mid_{X_{ij}\cap X_{ik}}\circ$$
            (so implicitly, to make sense of
            the right side, $f_{ij}(X_{ik}\cap X_{ij})\subseteq X_{jk}$).            
        \end{significant}
        This \emph{cocycle condition} ensures that $f_{ij}$ and $f_{ji}$ are inverses. In fact, the hypothesis that $f_{ii}$ is the identity also follows from the cocycle condition.\par 
        Show that there is a unique scheme $X$ (up to unique isomorphism) along with open subsets isomorphic to the $X_i$ respecting this gluing data in the obvious sense.\hint{what is $X$ as a set? What is the topology on this set? In terms of your description of the open sets of $X$, what are the sections of this sheaf over each open set?}
    \end{Ej}
    
    \begin{ptcbr}
    
    \end{ptcbr}
        

\end{document} 