\documentclass[12pt]{memoir}

\def\nsemestre {I}
\def\nterm {Spring}
\def\nyear {2023}
\def\nprofesor {Mark Shoemaker}
\def\nsigla {MATH673}
\def\nsiglahead {Algebraic Geometry}
\def\nextra {HW1}
\def\nlang {ENG}
\input{../../headerVarillyDiff}

\begin{document}
%\begin{multicols}{2}
    \begin{Ej}[1.3.C]
        Show that $A\to S^{-1}A$ is injective if and only if $S$ contains no
        zero divisors
    \end{Ej}
    
    \begin{Lem}
    A map $f$ in $\cat{Ring}$ is injective if an only if $\ker(f)=\set{0}$.
    \end{Lem}
    
    \begin{ptcbp}
    Suppose $f$ is injective, then if $x\in\ker(f)$ it holds that $f(x)=0$. As $f$ is a morphism of rings $0=f(0)$ which means that $f(x)=f(0)$ and as $f$ is injective, we can conclude that $x=0$ meaning that the kernel is trivial.\par 
    On the flip-side, take $f(x)=f(y)$. As $f$ is a morphism, $f(x-y)=0$. But then $x-y\in\ker(f)$ which means that $x-y=0$, letting us conclude that $x=y$. Thus $f$ is injective.
    \end{ptcbp}
    
    \begin{ptcbr}
    Call $\pi(a)=\frac{a}{1}$ the canonical map from $A$ to $S^{-1}A$. We will prove that $\ker(\pi)=\set{0}$ if and only if $S$ has no zero divisors. \par 
    To that effect suppose $s\in S$ is a zero divisor. This means that 
    $$\exists t (t\neq 0\land st=0).$$
    Now $\pi(t)=\frac{t}{1}$. But inside $S^{-1}A$ we have $\frac{t}{1}=\frac{0}{1}$ because 
    $$\frac{t}{1}=\frac{0}{1}\iff \exists t'\in S(t'(t\.1-0\.1)=0).$$
    Namely, such a $t'$ would be $s$. So $\pi(t)=0$ and, as $t\neq 0$, we have that $\ker(\pi)$ is not trivial.\par 
    On the other hand suppose $s,t\in S$ are elements that satisfy $st=0$. For the sake of argument suppose $t\neq 0$, we are set to prove that $s=0$ and to do this, we'll show that $s\in\ker(\pi)$. Notice that 
    $$\pi(s)=\frac{s}{1}=\frac{0}{1}\iff \exists s'\in S(s'(s\.1-1\.0)=0).$$
    The $s'$ we are looking for is $s'=t$. So, it follows that $s\in\ker(\pi)$. But as $\pi$ has trivial kernel, $s=0$ which is what we wanted. 
    \end{ptcbr}

    \begin{Ej}[1.3.Q]
        Describe the colimit of the diagram $F\: J\to\cat{Set}$ given by $\ast\leftarrow\ast\to\ast$.If the two squares in the following commutative diagram are Cartesian
  diagrams, show that the ``outside rectangle'' (involving $U, V, Y$, and $Z$) is also a Cartesian diagram.
  \end{Ej}
  %https://www.math3ma.com/blog/limits-and-colimits-part-3
  \begin{figure}[h]
      \centering
      % https://tikzcd.yichuanshen.de/#N4Igdg9gJgpgziAXAbVABwnAlgFyxMJZABgBpiBdUkANwEMAbAVxiRAFUQBfU9TXfIRQBGclVqMWbAGrdeIDNjwEiZYePrNWiEAHU5fJYKKj11TVJ0ANAwv7KhJUgCYNk7SACatxQJUiXNy02AC1ucRgoAHN4IlAAMwAnCABbJDIQHAgkZx4E5LTEZ2ospABmPJAk1PSS7MRhSurC0Uz6ivlmpAAWOqQAViaCnL7EbqGaxDLRwYouIA
  \begin{tikzcd}
      U \arrow[d] \arrow[r] & V \arrow[d] \\
      W \arrow[r] \arrow[d] & X \arrow[d] \\
      Y \arrow[r]           & Z          
      \end{tikzcd}
  \end{figure}
  
  \begin{ptcbr}
      Suppose that there's an object $T$ with maps to $U$ and $Y$ as in the following diagram:
      \begin{center}
          % https://tikzcd.yichuanshen.de/#N4Igdg9gJgpgziAXAbVABwnAlgFyxMJZARgBpiBdUkANwEMAbAVxiRAFUQBfU9TXfIRQAmclVqMWbAGrdeIDNjwEiZYePrNWiEAHU5fJYKKj11TVJ0ANAwv7KhJUgGYNk7SACatxQJUiXNy02AC0fe2MUAAZSKKDLEAAVbnEYKABzeCJQADMAJwgAWyQYkBwIJGEeXILixFEyisRnapB8opLqcqRiVva6skakFvl+pAAWLqaAVj7ayqmJuY7mxcRZ0fnEADY1yZAAIxgwKCQAWmco5brdocRBo5Phq82V2+7EUoYsMA8oOjgAAs0ikuEA
  \begin{tikzcd}
      T \arrow[rddd, bend right] \arrow[rrd, bend left] \arrow[rd, dashed] &                       &             \\
                                                                           & U \arrow[d] \arrow[r] & V \arrow[d] \\
                                                                           & W \arrow[r] \arrow[d] & X \arrow[d] \\
                                                                           & Y \arrow[r]           & Z          
      \end{tikzcd}
      \end{center}
      We wish to show that there's a unique morphism $T\to U$ such that $T\to V$ and $T\to Y$ factor through $T\to U$.\par 
      Our first step is to construct a unique morphism from $T\to W$. This is done because we have a morphism $T\to X$ (the composition of $T\to V$, $V\to X$) and a morphism $T\to Y$ whose compositions to $Z$ agree. By universal property of the pullback ($W$ as pullback) we have that there's a unique morphism $T\to W$ through which the respective morphisms factor.
      \begin{center}
          % https://tikzcd.yichuanshen.de/#N4Igdg9gJgpgziAXAbVABwnAlgFyxMJZARgBpiBdUkANwEMAbAVxiRAFUQBfU9TXfIRQAmclVqMWbAGrdeIDNjwEiZYePrNWiEAHU5fJYKKj11TVJ0ANAwv7KhyAAyknGydpAAVbuJhQAc3giUAAzACcIAFskFxAcCCRhHjDImMRReMTEAGYUkAjo2OoEpGJ8wvSyLKQ8+UqkABYS7OqAIxgwKFqnCrSmluKQBiwwTyg6OAALf1sGxGaajOoOrqQAWhy4kbG2CenZ8w82AB0TmAAPLDgcOABCXy4gA
  \begin{tikzcd}
      T \arrow[rrd, bend left] \arrow[rd, dashed] \arrow[rdd, "\exists!", dashed, bend right] &                       &             \\
                                                                                              & U \arrow[d] \arrow[r] & V \arrow[d] \\
                                                                                              & W \arrow[r]           & X          
      \end{tikzcd} 
      \end{center}
      With this in hand, we now have $T\to V$ and $T\to W$ whose compositions to $X$ agree. So by universality of $U$, there exists a unique morphism $T\to U$ through which the corresponding morphisms factor.   
  \end{ptcbr}

\begin{Ej}[1.3.T]
Show that coproduct for $\cat{Set}$ is disjoint union. 
\end{Ej}

\begin{ptcbr}
    Recall that the disjoint union of $A_1$ and $A_2$ is defined as a set as
    $$A_1\cupdot A_2\set{(a_i,i)\:\ a_i\in A_i,\ I=1,2}.$$
    This allows us to define maps $\iota_i\: A_i\to A_1\cupdot A_2,\ x\mapsto (x,1)$ which are morphisms in $\cat{Set}$ because they're defined everywhere. We are to show that this set satisfies the universal property of coproducts.\par
    Suppose $B$ is a set such that $f_i\:\ A_i\to B$ are well defined. We must define a unique function $g:\ A_1\cupdot A_2\to B$ such that $f_i=g\iota_i$, this is done as follows: 
    $$g(a,i)=\begin{cases}
        f_1(a),\ i=1,\\
        f_2(a),\ i=2.
    \end{cases}$$ 
    We verify the factoring property:
    $$g\circ\iota_1(a_1)=g(a_1,1)=f_1(a_1),\quad g\circ\iota_2(a_2)=g(a_2,2)=f_2(a_2).$$
    By construction, we have defined $g$ uniquely.
\end{ptcbr}

\begin{Ej}[1.3.U]
    Suppose $A\to B$ and $A\to C$ are two ring morphisms, so in particular $B$ and $C$ are $A$-modules. Recall that $B\ox_AC$ has a ring structure.
    \begin{enumerate}[i)]
        \itemsep=-0.4em
        \item Show that there is a natural morphism $\iota_B\: B\to B\ox_AC,\ b\mapsto b\ox 1$. Similarly for $C$.
        \item Show that this gives a pushout on rings. In other words, the following diagram satisfies the universal property of the pushout.   
    \end{enumerate}
\end{Ej}

\begin{figure}[h]
    \centering
% https://tikzcd.yichuanshen.de/#N4Igdg9gJgpgziAXAbVABwnAlgFyxMJZABgBpiBdUkANwEMAbAVxiRACEAdTiADwH0AggGEQAX1LpMufIRRkAjFVqMWbduMkgM2PASILSS6vWatEIQZqm7ZB8stNqLoscphQA5vCKgAZgBOEAC2SABM1DgQSIYgDHQARjAMAArSenJxMH44ICaq5iDcCbkS-kGhiBEgUUgAzNTxSanpdhYBWJ4AFrn5ZmzcjNYggSH1kdGIZHGJyWm2+u2dPXkq-Rbc+Dh0-K5ao5WxtVN9zkWcWzsabmJAA
\begin{tikzcd}
    B\ox_AC                & C \arrow[l, "\iota_C"']              \\
    B \arrow[u, "\iota_B"] & A \arrow[l, "\bt"] \arrow[u, "\al"']
    \end{tikzcd}
\end{figure}

\begin{ptcbr}
    \begin{enumerate}[i)]
        \itemsep=-0.4em
        \item The map $\iota_B(b)=b\ox 1$ is a homomorphism in virtue that $B\ox_AC$ is a tensor product. By construction, all bilinear maps factor through the tensor product as linear maps. This map is one of the factors which should be linear. The same holds for $C$. 
        \item Let us now take $M$ an $A$-module with morphisms $f_B\:B\to M$ and $f_C\: C\to M$. This can described by the following diagram:
        \begin{center}
            % https://tikzcd.yichuanshen.de/#N4Igdg9gJgpgziAXAbVABwnAlgFyxMJZARgBpiBdUkANwEMAbAVxiRACEAdTiADwH0AggGEQAX1LpMufIRRkATFVqMWbduMkgM2PASILSS6vWatEIQZqm7ZB8stNqLoiTZn6UABlJfHq8xAAWXFlGCgAc3giUAAzACcIAFskQxAcCCQyEAY6ACMYBgAFaT05HJhYnBATALZuPOq3EATk1OoMpABmalyC4tK7C3isCIALatqzes5GaxbElMQe9MzEHxz8wpLbTxAR8cmVaYtufBw6flctVqXszvWp5xAziAv+DWbbrI61gBZqAUwFBuhsnIFYh95t9lr8kACQECQYgALRdMF1CyQ0S9LYDXblA4TUJiIA
\begin{tikzcd}
    M &                                                      &                                                         \\
      & B\ox_AC                                              & C \arrow[l, "\iota_C"'] \arrow[llu, "f_C"', bend right] \\
      & B \arrow[u, "\iota_B"] \arrow[luu, "f_B", bend left] & A \arrow[l, "\bt"] \arrow[u, "\al"']                   
    \end{tikzcd}
        \end{center}
        However, let us take advantage of the tensor product, \emph{gatekeeper of bilinear maps}. This morphisms can be combined into a bilinear map from $B\x C\to M$. We define 
        $$f\: B\x C\to M,\ (b,c)\mapsto f_B(b)f_C(c)$$
        and by universal property of the tensor product, there exists a unique map $\tilde{f}\: B\ox_AC\to M$ through which $f$ factors. Finally $f_B$ and $f_C$ factor through $\tilde{f}$ by diagram chasing and thus by universality of the tensor product we have that it satisfies the pushout universal property in this case.
    \end{enumerate}


\end{ptcbr}

\begin{Ej}
    Describe the colimit of the diagram $F\: J\to\cat{Set}$ given by $\ast\leftarrow\ast\to\ast$.
\end{Ej}
%https://www.math3ma.com/blog/limits-and-colimits-part-3
\begin{ptcbr}
    Recall that the colimit of a diagram $F\: J\to\cat{C}$ is an object $\text{colim} A_i\in\text{Obj}\cat{C}$ with morphisms $f_j\: A_j\to\text{colim} A_i$ such that if $m\: k\to j$ is a morphism in $J$, then the following diagram commutes
\begin{center}
    % https://tikzcd.yichuanshen.de/#N4Igdg9gJgpgziAXAbVABwnAlgFyxMJZABgBpiBdUkANwEMAbAVxiRAB12BjCBrAWwAEAQQD6WEAF9S6TLnyEUZAIxVajFmzEArKTJAZseAkWWlV1es1aIQYgNZS1MKAHN4RUADMAThH5IAEzUOBBIZurWbABiABT8AJR63n4BiBGhSGSRmrZeorrSKf5BIWGI2Va5IPmO1Ax0AEYwDAAKcsaKID5YrgAWOE6SQA
\begin{tikzcd}
    \text{colim} A_i           &                                          \\
    A_j \arrow[u, "f_j"] & A_k \arrow[l, "F(m)"] \arrow[lu, "f_k"']
    \end{tikzcd}
\end{center}
In our case, since we only have three objects the diagram looks like this
\begin{center}
    % https://tikzcd.yichuanshen.de/#N4Igdg9gJgpgziAXAbVABwnAlgFyxMJZABgBpiBdUkANwEMAbAVxiRAGEAZEAX1PUy58hFGQCMVWoxZsAQr34gM2PASJjSE6vWatEIAIIKBK4evKSdM-e16SYUAObwioAGYAnCAFskAJmocCCQxPncvX0QNECCkYjCQTx9-QODEAGYEpMj01LieCh4gA
\begin{tikzcd}
    CL          & C \arrow[l]           \\
    B \arrow[u] & A \arrow[l] \arrow[u]
    \end{tikzcd}
\end{center}
where $CL$ is the colimit object. In this particular case the colimit coincides with the pushout by universality.
\end{ptcbr}

\begin{Ej}[1.4.F]
    Verify that the $A$-module described above is indeed the colimit.
\end{Ej}

The $A$-module in question is $\quot{\coprod A_i}{\sim}$ where $\sim$ is the relation 
$$(a_i,i)\sim (a_j,j)\iff \exists (f\: A_i\to A_k,\ g\: A_j\to A_k)(f(a_i)=g(a_j)),$$
addition of $m_i\in M_i,\ m_j\in M_j$ is defined as 
$$m_i+m_j\:= F(u)(m_i)+F(v)(m_j)$$
where $u,v$ are arrows from $i,j$ to $\l$. The sum lies in $M_\l$. Multiplication is defined in an obvious\footnote{It's not obvious to me.} way and the zero element is $m_i$ such that there is an arrow $u\: i\to k$ for which $F(u)(m_i)=0$.\par 
I must admit I wasn't able to tackle this problem. 
\end{document} 