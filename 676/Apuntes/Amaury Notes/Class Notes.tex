\documentclass[11pt]{article}
\usepackage[utf8]{inputenc}	% Para caracteres en español
\usepackage{amsmath,amsthm,amsfonts,amssymb,amscd,mathtools}
%-------------------------------------
%for problems and solutions
\usepackage{enumitem, fancyhdr, comment, graphicx, environ,tikzsymbols, euscript, wasysym}
\newenvironment{problem}[2][Problem]{\begin{trivlist}
		\item[\hskip \labelsep {\bfseries #1}\hskip \labelsep {\bfseries #2.}]}{\end{trivlist}}
\newenvironment{sol}
{\emph{Solution:}
}
{
	%\copyright
	%\Strichmaxerl[1.2][54][28]
	\Tribar[-1][blue][red][green]
	%\capricornus3w
	%\Walley[1.2]
	%\mathrightghost
	%q.e.d.
}
%-------------------------------------
\usepackage{multirow,booktabs}
%\usepackage[table]{xcolor}
\usepackage{tikz}
\usepackage{tikz-cd} 
\usepackage{tikzsymbols, euscript, wasysym}
\usepackage{tkz-graph}
\usepackage{xcolor}
\usepackage{fullpage}
\usepackage{lastpage}
\usepackage{enumitem}
\usepackage{fancyhdr}
\usepackage{mathrsfs}
\usepackage{wrapfig}
\usepackage{setspace}
\usepackage{calc}
\usepackage{multicol}
\usepackage{cancel}
\usepackage[retainorgcmds]{IEEEtrantools}
\usepackage[margin=3cm]{geometry}
\newlength{\tabcont}
\setlength{\parindent}{0.0in}
\setlength{\parskip}{0.05in}
\usepackage{empheq}
\usepackage{framed}
\usepackage[most]{tcolorbox}
\colorlet{shadecolor}{orange!15}
\parindent 0in
\parskip 12pt
\geometry{margin=1in, headsep=0.25in}

%--------------------
%--------------------
%Theorem
\theoremstyle{definition}
%\newtheorem{defn}{Definition}
\newtheorem{reg}{Rule}
\newtheorem{exer}{Exercise}
%\newtheorem{ex}{Example}
\newtheorem{note}{Note}
\newtheorem{remark}{Remark}
\newtheorem{axiom}{Axiom}[section] 
%\newtheorem{theorem}{Theorem}[section] 
\newtheorem{proposition}{Proposition}[section] 
\newtheorem{conjecture}{Conjecture}[section]
%\newtheorem{lemma}{Lemma}[section] 
%\newtheorem{corollary}{Corollary}[section] 

%The following is to color corollary, theorems, etc.
\usepackage{xcolor}
\usepackage{amsthm}
\usepackage{framed}
\theoremstyle{plain}% default
\newtheorem{prototheorem}{Theorem}[section]

\newenvironment{theorem}
   {\colorlet{shadecolor}{orange!15}\begin{shaded} \begin{prototheorem}}
   {\end{prototheorem}\end{shaded}}

\newtheorem{protolemma}[prototheorem]{Lemma}
\newenvironment{lemma}
   {\colorlet{shadecolor}{blue!15} \begin{shaded}\begin{protolemma}}
   {\end{protolemma}\end{shaded}}

\newtheorem{protocorollary}[prototheorem]{Corollary}
\newenvironment{corollary}
   {\colorlet{shadecolor}{pink!15}\begin{shaded} \begin{protocorollary}}
   {\end{protocorollary}\end{shaded}}

\theoremstyle{definition}
\newtheorem{protonotation}{Notation}[section]
\newenvironment{notation}
   {\colorlet{shadecolor}{green!15}\begin{shaded}\begin{protonotation}}
   {\end{protonotation}\end{shaded}}

\newtheorem{protoexample}{Example}[section]
\newenvironment{ex}
   {\colorlet{shadecolor}{red!15}\begin{shaded}\begin{protoexample}}
   {\end{protoexample}\end{shaded}}

\newtheorem{protodefinition}{Definition}[section]
\newenvironment{defn}
   {\colorlet{shadecolor}{black!15}\begin{shaded}\begin{protodefinition}}
   {\end{protodefinition}\end{shaded}}


%--------------------
%Commands
\def\a{\alpha}
\def\l{\lambda}
\def\r{\rho}
\def\ZZ{{\mathbb Z}}
\def\QQ{{\mathbb Q}}
\def\RR{{\mathbb R}}
\def\FF{{\mathbb F}}
\def\KK{{\mathbb K}}
\def\CC{{\mathbb C}}
\def\NN{{\mathbb N}}
\def\XX{{\mathbb X}}
\def\YY{{\mathbb Y}}
\def\TT{{\mathbb T}}
\def\PP{{\mathbb P}}
\def\subgroup{{ \; \leqslant \;}}
\def\normalsubgroup{{ \; \trianglelefteq \;}}
\allowdisplaybreaks
\def\B{\mathcal{B}}
\def\C{\mathcal{C}}
\def\A{\mathcal{A}}
\def\D{\mathcal{D}}
\def\a{\alpha}
\def\w{\varpi}
\def\al{\alpha}
\def\fg{\mathfrak{g}}
\def\fh{\mathfrak{h}}
\def\la{\lambda}
\def\be{\beta}
\def\P{\mathscr{P}}
\def\so{\mathfrak{so}}
\def\sp{\mathfrak{sp}}

\DeclareMathOperator{\Res}{Res}

\DeclareMathOperator{\val}{val}
\DeclareMathOperator{\trop}{trop}

%For table of contents
\usepackage{color}   %May be necessary if you want to color links
\usepackage{hyperref}
\hypersetup{
    colorlinks=true, %set true if you want colored links
    linktoc=all,     %set to all if you want both sections and subsections linked
    linkcolor=blue,  %choose some color if you want links to stand out
}
%-------------


%Gaussian Coefficients
\def\gausscoeff#1#2#3{\left[\begin{array}{c}#1\\#2\end{array}\right]_{#3}}






%-------------------------

%For displaystyle everywhere
%\everymath{\displaystyle}
%------------------------------

\begin{document}
%\setcounter{section}{8}
\title{Combinatorics Notes}


%\tableofcontents


\thispagestyle{empty}


\begin{center}
{\LARGE \bf Tropical Geometry}\\
{\large MATH 676}\\
Fall 2023
\end{center}

These notes arose from Tropical Geometry with Dr. Renzo Cavaleri during the Fall of 2023 at CSU. They come from his lectures.

\tableofcontents

\newpage


\section{Intro}

\subsection{Current State of Literature}

There are current books for Tropical Geometry. This includes
\begin{itemize}
    \item \emph{Tropical Geometry} by Maclagan-Sturmfels, which has a very algebraic take on the subject.
    \item \emph{Tropical Geometry}, Which is in progress, being written by Mikhalkin-Rau, whcih has a geometric and intersection theoretic take.
    \item There are also various expository articles for Tropical Geometry
\end{itemize}

\subsection{Tropical Geometry}

Tropical Geometry is sometimes called a `combinatorial'' shadow of algebraic geometry. We take a inputs algebraic varieties, and receive a an output a piecewise linear object.

\begin{ex}
    THe input can be a line in the plane $\CC^2$, i.e. $az+bw=d$. Then the output of the construction can be a tripod/tropical $Y$, i.e.e three lines connecting a vertex.


    The input could also be an elliptic curve in $\CC^2$, and hte output can be another more complicated connection of verticles and lines (a tropical cubic)


    FInally, we can consider an abstract  nodal curve (a sphere and tori connected at verticies), with the corresponding piecewise lienar object being a dual graph, which has a vertex at ease component, an edge for each node, and a label for each part.
\end{ex}



The questions that naturally arise are
\begin{enumerate}
    \item What algebraic information about the initial object is carried over in the simplified object?
    \item How do we extract the information carried by the simplified object?
    \item Does the lifting problem have a solution?
\end{enumerate}


THere are four ways to tropicalize the algebraic varity.

\subsubsection{Tropical Smei-Field}
We do "algebraic geometry" over the tropical semi-field. THe tropical semi-field is $(\mathbb{T}, \oplus, \odot ) = ( \RR \cup \{- \infty\}, \max, + )$. We can take a polynomial $p(x_1, \dots, x_n)$. Our variety is the roots of the polynomial.If we consider the tropial $p(x_1, \dots, x_n) : (\RR \cup \{- \infty\})^n \rightarrow \RR \cup \{- \infty\}$, this is an affine piecewise linear function. For example, We can take $p(x_1, x_2)$.



The tropical hypersurface is the locus of non-linearlity.

A tropical polynomial would be something of the form $p(x,y) : x \oplus y \oplus 0= max(x,y,0)$


By th ecorner locus, we take the graph of our function.





Why does this work to make piecewise linear? We get expressions such as $3 \odot x^2 = 3 + 2x$.


\subsubsection{Valued Fields}

our second perspective will be of valued fields. We let $K$ be a field with a valuation, i.e. let $K= \CC(t)$ be the field of rational functions in 1 variables. A valuation $val_0$ is a function $val_0: \CC(t) \rightarrow \RR \cup \{ \infty\}$, where $0 \mapsto \infty$, else $f \mapsto$ order of vanishing (or pole) as you apporach $t \rightarrow 0$.


We can ask what happens to the order of vanishing when you add 2 functions. In this case the order of vanishing is the minimum of the two orders. i.e. if $f_1=t^2$, $f_2=t^3$, then $val_0 f_1 = 2$, $val_0 f_2 = 3$, and then $va_0(f_1+f_2)=2$.  So in this context, $val_0(f_1+f_2) \geq min(val_0 f_1, val_0f_2)$, and $val_0(f_1f_2) = val_0 f_1 + val_0 f_2$. Let $p(x_1, \dots, x_n) \in K[x_1, \dots, x_n]$. Take the variety $X = \{x_1, \dots, x_n) \; |\; p(x_1, \dots, x_n=0) \} \subset K^n$. We can take the map $K^n \rightarrow (\RR \cup \{ \infty\})^n$ by taking the valuation at every coordinate, i.e. apply  $val_{0}$ to the result. Each point of $K$ is some rational function of $t$, so looking at the order of vanishing makes sense. So the tropicalization of X is the image via this map.


For example, we can take $p(x,y) = tx + y + t^2$. Then $X = \{ (x,y) \; |\; tx+y+t^2 = 0\}= \{ (x,y) \; |\; y = -tx -t^2\} $. So $(0,-t^2)$ is an option. So the point $(\infty, 2)$ is a point in $Trop(X)$.



\subsubsection{Amoebas}

We consider $p(x_1, \dots, x_n) \in \CC[x_1, \dots, x_n]$. It defines $ X= \{(z_1, \dots, z_n) \; |\; p(z_1, \dots, z_n) = 0 \} \subset \CC^n$. We can consider the map  from $\CC^n$ to the image $\CC^n  \rightarrow  (\RR \cup \{-\infty\})^n$ via $(z_1, \dots, z_n) \mapsto (\log_t|z_1|, \dots, \log_t |z_n|) $. The image of $X$ via this map $Im(X)$ is the $t$-Amoeba of $X$. 


\begin{ex}
    $p(x,y) = x+y+1$ We can take the variety, points such as $x=i$, $y=-1-i$, and $p(x,y) \in X$. We then consider $(log_t 1, \log_2 \sqrt{2})$
\end{ex}


We can take $\lim\limits_{t \rightarrow \infty} (t-Amoeba)$, called the spine of the amoeba or $Trop(X)$.


\subsubsection{Degenerations of Algebraic Varieties}



This is a complicated way to get tropics



\section{Motivation of Tropical Geometry}

\subsection{Tropical Arithmetics}

\subsubsection{Minimization Problems}

An important question is where tropical numbers come from. One example is toll minimization problems. We let every city be a vertex, and every directed edge is a tollway. Let $A$, $B$, and $C$ be the three cities/vertices. The incidence matrix would be

\begin{align*}
    M= \begin{bmatrix}
        0 & \infty & 2
        \\
        x & 0 & y
        \\
        \infty & 1 & 0
    \end{bmatrix}
\end{align*}
where the rows represent teh starting point and the columns represent the ending location, so there is a toll road from A to B which costs 2. There is no road from A to B. In other words, $M_{i,j}$ records the (minimum) price of going from city $i$ to city $j$ in at most one trip (one trip being moving across one toll road, i.e. traversing one edge). We can now ask how to compute the best strategy of going from $i$ to $j$ in at most two trips.

\begin{ex}
    If I want to go from $A$ to $B$ in 2 steps, We can do $A \rightarrow A \rightarrow B$, $A \rightarrow B \rightarrow B$, or $A \rightarrow C \rightarrow B$. THe associated costs are $0+\infty$, $\infty + 0$, and $2 + 1$, respectively. THe best strategy is the minimum of these three, which is $A \rightarrow C \rightarrow B$, which gives a cost of $3$.However, these sums can be thought of as coming from entries of the matrix. THe associated sums are $a_{11}+a_{12}$, $a_{12}+a_{22}$, and $a_{13}+a_{32}$. 


    WHere do these valeus come from? The $b_{12}$ entry of $M^2$ is $\sum\limits_{j=1}^n a_{1j}a_{j2}$. So the best strategy is equal to the $12$ entry of $M^2$, so long as you interpret $+$ as the minimum, and $*$ as $+$. In that sense, $b_{12}$ entry is $M^2$ is $min(a_{11}+a_{12}, a_{12} +a_{22}, a_{13}+a_{32})$. 



    If we make the assumption that there are no negative tolls, this arithmetic for higher powers of $n$ gives the best solution for up to $n$ trips. Eventually we stablize, as for a high enough $N$ the solutions will say to stop moving for a certain number of steps. (Eventually we get something like idempotence).
\end{ex}


The minimization problem eventually becomes a linear algebra problem over $(\mathbb{T}, \oplus, \odot) = ( \RR \cup \{\infty\}, \min, +)$.


\subsubsection{Forget the Phase}

Another context is in physics/electromagnetism. If we take a complex number, we can write it in polar coordinates, i.e. $z \in \CC$ can be written as $z = rho e^{i \theta}$. Maybe we dont particularly care about the phase ($\theta$), or our work works on a logarithmic scale. So our function would be $T_t: \CC \rightarrow \RR\cup \{-\infty\}$, where $z \mapsto \log_t|z|$. We add the point at negative infinity to be able to define $T_t(0)$.


We have that $\CC$ has it's own natural operations, and we can ask if we can induce operations on $\RR\cup \{-\infty\}$ utilizing the map $T_t$. If we want to define addition on $\RR\cup\{-\infty\}$, we could define it as $x \boxplus y$. We could try $ T_t( T_t^{-1}(x) + T_t^{-1}(t))$. The problem this is not well defined, as different inverse images can have different absolute values. This is a hyper operations, This is a function$ X \times X \rightarrow \mathcal{P}(X)$, to the power set of $X$. So this function outputs an interval.

We first wish to understand $T_t^{-1}(y)$. This is $T_t^{-1}(y) = \{t^ye^{-\theta}\}$. adding and subtracting is thus given by the interval of the possible extremes, so we get the interval.

\begin{align*}
    T_t( T_t^{-1}(x) + T_t^{-1}(t)) = [ \log_{t} |t^x-t^y|, \log_t(t^x+t^y)]
\end{align*}

But we don't want a hyperoperation, we want an opeartion. TO pick something that always works, we can either pick the max or the minimum. We can also see if in a limiting process, we get one answer.  We Start with $x \boxplus_t y \rightarrow x+_t y$, with some consistent choice. we can also ask for $\lim\limits_{t \rightarrow \infty}$ to get a new operation. We define $x \boxplus y = \lim\limits_{t \rightarrow \infty} x \boxplus_t y$. Why is this better? If we take $t^2e^{i\theta_1} + t^4 e^{i\theta_2}$. So what really matters at the end is that our expression is equal to $t^4(t^{-2}e^{i\theta} +e^{i\theta})$. Taking $\log_t$, we have $4+ \log_2|t^{-2}e^{i\theta} + e^{i\theta})$ Taking the limit as $t \rightarrow \infty$, we have $4+0=4$. So this is not a hperoperation if $x=2$ $y=4$. But, if we have $x=2$ and $y=2$, we get issues. We have
\begin{align*}
    x\boxplus y = \begin{cases}
        \max(x,y) & x \neq y
        \\
        (\infty, \max(x,y) ) & x =y
    \end{cases}
\end{align*}

This suggests that the nice consistent choice to define $x\oplus y$ is $\max$. This is fulyl consistent when $x \neq y$ regardless of any decision, and it makes the interval $(\infty, \max(x,y))$ work nicer. To define multiplication, we would say $x*y = T_t(T_t^{-1}(x) T_t^{-1}(y))= T_t(t^xe^{i\theta_1}t^ye^{i\theta_2}) = x+y$. This leads to the tropical numbers, where
$(\mathbb{T}, \oplus, \odot) = \RR\cup \{-\infty\}, \max, +)$.







\subsubsection{Puiseux  Series}


In cacl 1, we find ways to approximate. In approximation, we lose some information. we learn that $\lim\limits_{t\rightarrow 0} \frac{\sin(t)}{t} =1$, which essenitailly says $\sin(t) = t+\mathscr{o}(t)$, and $\frac{1}{t} = t^{-1} + 0$. In otherwords, $\lim\limits_{t \rightarrow 0} (t+\mathscr{o}(t))t^{-1} = 1 + \mathscr{o}(1)$, where $ \mathscr{o}(t)$ goes to zero as $t$ goes to zero. All of this is to say that we have a concept of order of vanishing of functions.




A Prototypical example of a field with a valuation is the field of Puiseaux series. Dentoed $\CC\{\{t\}\}$, this is the Laurent series in $t$ with rational exponenets, and all exponents of terms with nonzero coefficients have a common denominator. For example, $\sum\limits_{i=-37}^{\infty} t^{i/42}$ works, but $\sum\limits_{i=1}^\infty t^{1/i}$ does not work. In otherwords, $\CC\{\{t\}\} = \bigcup\limits_{n \in \NN} Laur(t^{1/n})$, that is to say it is th eunion of all Lauraent series in the variable $t^{1/n}$ This has a valuation
\begin{align}
    val: \CC\{\{t\}\} &\rightarrow \RR \cup \{\infty\}
    \\
    (a_N\neq 0)\; \sum\limits{i=N}^\infty a_it^{q_i} &\mapsto q_N
    \\
    0 &\mapsto \infty
\end{align}

This map tehcincially has as it's image $\QQ \cup \{\infty\}$.


Once again, the properties/axioms of a valuation field is that
\begin{itemize}
    \item $val(\alpha + \beta) \geq \min \{val(\alpha) + val(\beta)\}$,
    \item $val(\alpha * \beta) = val(\alpha + val(\beta)$.
\end{itemize}


If we want to induce a sum on the image, we would have a hyper operation. We define $x \boxplus y := val( val^{-1}(x) + val^{-1}(y) ) =\begin{cases} [min(x,y), \infty]  &x = y
\\
\min(x,y) & x \neq y\end{cases}$. For example, we would have $0 \boxplus 0 := val( a_0 +t^{q_1}a_1 + \cdots) + (-a_0 + t^{r_1}b_1 + \cdots ))$. THen we can let $q_1$ and $r_1$ equal whatever we want.

To avoid issues, we say $x+y := \min(x,y)$. This takes us to $(\RR \cup \infty, \min, +)$.

\section{The Tropical Semifield}

\begin{defn}
    The \emph{tropical semifield} is one of $(\{- \infty\} \cup \RR, \max, +)$, or $(\RR \cup \{\infty\}, \min, +)$, either one of which is denoted by $(\mathbb{T}, \oplus, \odot)$. The operations are associative, and the distributive law holds.
\end{defn}

The two definitions are isomorphic via $x \mapsto -x$. In the following writing, we will tend to utilize max more often than min. We have he following properties of this semifield

\begin{enumerate}
    \item $\{-\infty\}$ is the additive identity,
    \item $\{-\infty\}$ is the only element that ha an additive inverse.
    \item $\oplus$ is an idempotent operation $x \oplus x = x$.`
    \item We cannot add inverses, not even formally.
    \item $0$ is the multiplicative identity.
    \item every eleemnt $x \neq - \infty$ has a multiplicative inverse (namely $-x$).
\end{enumerate}

\begin{ex}
    Let us solve $x \oplus y = - \infty$. This is to say $\max(x,y) = - \infty$, which necessitates $x=y = -\infty$. To see that inverses are not definable, We take $x \in \mathbb{T}$. Let us attempt to construct a formal inverse $y$, defined by $x \oplus y = -\infty$. This will  force $x = -\infty$. We see this by considering $x \oplus x \oplus y= (x\oplus x) \oplus (y) = -\infty$. As this operation is associative, we also have $x \oplus (x \oplus y) = x \oplus - \infty = x$, and so $x= -\infty$.
\end{ex}


\subsection{Weird/fun facts}


Pascal's triangle in tropical arithmetic looks like all zeros. Furthermore, the fresh,ans dream holds $(x\oplus y)^n = x^n \oplus y^n$. This is beacuse $(x \oplus y)^n = n *(\max(x,y)) = \max(nx,ny)$. However, the fact taht $x^2 \oplus (x \odot ) \oplus y^2 = x^2 \oplus y^2$ does not imply cancelation, i.e.e we do not autoamtically have that $x \odot y = -\infty$.


The Trpical determinat of a matrix (call it permanent) gives a solution to the assignment problem: We have $n$-jobs to $n$-workers. $x_{ij}$ is the profitablility of worker $i$ in job $j$, and the goal is to find the best assignment to maximize profits. This is the tropical determinant of the matrix $X$. We define $tropdet(X) = \sum_{\sigma \in S_d}  \prod_{\sigma(i)}x$. where the sum is the tropical sum and the product is the tropical product. We do not get a signed determienant as tropical geometry does not have subtraction.




\subsection{Algebraic Geometry: Tropical Univeraitate polynomials and their roots}

\begin{defn}
    A tropical univariate (Laurent) monomial is an expresion of the form $a \odot x^{\odot m}$, where $a \in \TT$ and $m \in \ZZ$.
\end{defn}

In ordinary algebra, a tropical monomial corresponds to an affine linear function with integer slope.

\begin{ex}
    $\sqrt{5}\odot x^{\odot 3} \sqrt{5} + 3x = y$. This is an affine linear transformation. It is affine because we can shift via $a$, and it has integer slope as the slope is $m$. Furthermore, $\{-\infty\}\odot x^{\odot m} = -\infty + mx = - \infty$, so we maintain that multiplying by the additive identity gives the additive identity.
\end{ex}


\begin{defn}
    A tropical univariate (Laurent) polynomial is the finite sum of monomials.
\end{defn}

The tropical univeraite polynomial corresponds to a continiuous, piecewise affine linear function with $\ZZ$-slopes


\begin{ex}
    $p(x) =  -5\odot x^{\odot 2} \oplus (-2)\odot x^{\odot -3} \oplus 0= \max (-5+2x, -2-3x, 0)$
\end{ex}


By construction, tropical polynomials give rise to convex functions. In the univeraitae case, the map from tropical $L$ polynomials to convex $\ZZ$ affine piecewise linear (with finitely many distinct regions of linearity) functions is a surjective map. However, this map is not surjective.

\begin{ex}
    Consider $p_1(x) = x^{-1} \oplus x$, i.e.e $y = |x|$. THen consider $p_2(x) = x^{-1} \oplus x \oplus 0$. Then we once again get $y=|x|$. Furthermore, we can take $p_{3}=x^{-1} \oplus x \oplus -8$, this works for any negative number.


    In general, $p(x) = p(x) \oplus \{-\infty\}$. However, $p(x) \oplus$ any function which is smaller than the minimum value attained by $p(x)$ does nto change the output of $p(x)$, i.e.  $p(x) =  -5\odot x^{\odot 2} \oplus (-2)\odot x^{\odot -3} \oplus 0= p(x) =  -5\odot x^{\odot 2} \oplus (-2)\odot x^{\odot -3} \oplus 0 \oplus (-4)\odot x$.
\end{ex}

Now that we have defined polynomials, we wish to make an interpretaion of roots. It does not make sense to say "values of $x$ for which $p(x)= - \infty$, as that typically wont happen (We deal with max). Solving for $0$ also doesn't help, as in this context $0$ is just anotehr number/function.



\begin{defn}
    Let $p(x) \in \TT[x]$ (an honest polynomial, no negative exponents, i.e. only positive slopes). Then
    \begin{itemize}
        \item $(-\infty)$ is a root of $p$ if the slope of the corresponding affine piecewise linear function is $\neq 0$ for $x <<0$.
        \item We allow $r \in \RR$ to be a root of $p$ if $f_p'(r)$ (The piecewise lienar function arrising from $p$, and the derivative of THAT function). is not defined.
    \end{itemize}
\end{defn}

In other words, roots will be where the function changes slopes. Now we can discuss multiplicities.


\begin{defn}
    If $- \infty$ is a root, its multiplicity is equal to the slope of $f_p(x)$ for $x<<0$. If $r\in R$ is a root, its multiplicity is the difference in slopes across $r$.
\end{defn}

\begin{ex}
    Take $p(x) = x^2 \oplus 1\odot x \oplus 0= \max(2x,1+x, 0)$  has two simple roots, one at $x=0$ and one at $x=1$.On the other hand, for $p(x) = x^2\oplus 1 \odot x$, we get two simple roots, one at $x=1$ and one at $- \infty$.


    Now, take $p(x) \oplus (-1)\otimes x \oplus 0$. Then there is a double root at $x=0$, as we go from slope $0$ to slope $2$.
\end{ex}



\begin{note}
    If we accept Laurent polynomials, then the multiplicity of $-\infty$ is equal to the slope towards $- \infty$.
\end{note}


\begin{ex}
    Take $q(x) = (-5)\otimes x^2 \oplus (-2) \otimes x^{-3} \oplus 0$, Then $-\infty$ has a pole of order $3$, the re is a root of order 3 at $x= -5/2$, and a root of order 2 at $x=2$
\end{ex}

\begin{lemma}
    Only $-\infty$ can have a pole (root with negative multiplicity) due to concavity.
\end{lemma}


\begin{lemma}
    $r \neq -\infty$ is a root for $p(x)$ iff when you write down $f_p(x) = \max( f_{m_0}(x), \dots, f_{m_d}(x) )$ at $r$ the maximum value is obtained at least twice.

    Furthermore, the multiplicty of the root is equal to the difference in the two extremal positions where the max is attained for $r$.
\end{lemma}




%September 1st written notes



\section{Root finding}

In previous lectures, we had the following theorem

\begin{theorem}
    Let $q(X) = \sum A_iX^i$, $A_i \in \CC \{\{t\}\}$, $a_i = val(Ai)$, and consider $p(x)  trop (q(X)) = \sum a_i \odot x^{\odot i}$. Then $R$ is a root of $q$ implies $r=val(R)$ is a root of $p$. FUrhtermore, if $r$ is a root of $p$, then there exists a root $R$ of $q$ where $r = val(R)$.
\end{theorem}
We prove this inductively. 


\subsection{Combinatorilization of tropical root finding}

We momentarily convert to the max convention. We take $p(x) = \sum\limits_{i=0}^d a_i \odot x^{\odot i}$. Indeed, $p(x) = \max_i \{a_i + ix\}$. There are $d+1$ lines $y= a_i + ix$. It is a finite process to intersect every pair of these lines, and then to compare the corresponding heights of each function to find the max. To make root finding more efficient, we start from left to right. To find the left most root, consider the left most intersection. It is thus the minimum of the x-value of intersection of the line $y=a_0$ and $y=a_i + ix$, when there exist a horizontal line (otherwise start with the line with the smallest slope, or factor out powers of $x$). THis is the $\min\{ x= (a_0-a_i)/i\}=-\max \{\frac{a_i-a_0}{i}\}$. $\frac{a_i-a_0}{i}$ is now reminiscent of slopes. wWe view this as rise over run, we look at all the lines through $(0,a_0)$ connecting to $(i,a_i)$, and we are lookign for the largest such slope. This gives us $a_j$ for our first root. Our next root is to the right of $a_j$,. This is because the corresponding line has a larger slope than the preceeeding lines, and the preeceding lines have a alter intersection, and are thus no longer considered in our root finding. So we use that as our next starting point, and we continue on.

We get a description of how to find the roots utilizing these points. This results in the following alforithm.

\begin{enumerate}
    \item The segment $[0,d]\subset \RR$ (aka the Newton polytope of the polynomial $p(x)$) is the convex hull of the $i$ such that $a_i \neq - \infty$ (i.e. assume degree $d$ polynomial with constant term)s.
    \item Find the convex hull of the points $(i,a_i)\in [0,d] \times \TT$ for $i \in \NN \cap [0,d]$.
    \item We construct the line from $(0,a_0)$ to $(d,a_d)$, and we disregard the section of the convex hull below this line. We call the section above this line $\Sigma^+$
    \item Project the vertices of $\Sigma^+$ back onto $[0,d]$. This gives a regular subdivision of the Newton polytope.
\end{enumerate}
Following this algorithm, we have that
\begin{itemize}
    \item [A)] The roots of $p(x)$ are in bijection with the complement of the projections, i.e. the subdivisions of the Newton polytope.
    \item [B)] The value of the root corresponding to a given segment $(i,j)$ is found by solving the equation $a_i + ix = a_j  +jx$.
    \item [C)] The multiplicity of the root is equal to the length of the segment.
\end{itemize}



\begin{ex}
    Let $p(x) = 0 \oplus (1 \odot x) \oplus (1\odot x^2) \oplus x^3 \oplus (2\odot x^4) \oplus (1 \odot x^5)$. When graphing the convex hull, the points above the line connecting $a_0$ and $a_5$ are $a_1$, $a_2$, $a_4$. The vertices are $a_0$, $a_1$, $a_4$, and $a_5$. Thus, we expect to have two simple roots $r_1$ and $r_3$, and one root of multiplicity of $3$ $r_2$. To find the root $r_1$, we solve $0=1+x$, $x=-1$. To find $r_2$, we solve $1+x = 2 + 4x$, which is $x=\frac{-1}{3}$. To find $r_3$, we solve $2 + 4x=1+5x$, which has as solution $x=1$.
\end{ex}





\subsection{Grobner}
%Gr\"ober
Let $\KK$ be a fie.d witha  valuation, such as $\CC\{\{t\}\}$. Then let $R_\KK$ be the set of all elements with non-negative valaution, i.e. $\bigcup \limits_{n >0}\CC[[t^{1/n}]]$. In partcular, we can consider ideals and maximal ideals. and we define $M_{R_\KK}$ to be the maximum ideal, the set of all elements with positive valuation. and thus $\bigcup\limits_{n>0} t^{1/n}\CC[[t^{1/n}]]$. $k$ is the residue field $R_\KK/ M_{R_\KK} = \CC$. The subset relation is $M_{R_\KK} \subset R_\KK \subset \KK$.

\begin{ex}
    Suppose we had $q(x) = t^{-4} \sqrt{2} x + 3t^2x^2$. Suppose we picted a valuation $val(x) = -3$. We can thus ask for the valaution of the individual terms. Then $t^{-4} \rightarrow -4$, $\sqrt{2}x \rightarrow 0+(-3)$, and $3t^2 x^2 \rightarrow 2-6= -4$. Now, we dcide that the lowest order terms are from $t^{-4}$ and $3t^2x^2$. Now, we no longer care about the value of $t$, so we only keep the coefficients. We say we ahve an inital form of $q$ for a valuation of $x$. $In_{w = -3} (q(x)) = 1+ 3x^2$. With a polynomial in a value field, we have an equivalence relationbased on the initial forms. Breaking up $\RR$ into initial forms gives roots of $q(x)$. The roots are where the valautions give initial forms which are not monomilas.
\end{ex}


We say $\KK$ is our valued field, such as $\CC\{\{t\}\}$. $R_\KK$ is the set of all series that start with $t \geq 0$, while $M_\KK$ is the set of all series that start with $t \geq 0$. The residue field $R/M = \CC$. If we start with a polynomial $q(X) \in \KK[x]$ and $w= val(x) \in \RR$, we get an initial form $In_w q$ to then get a polynomial with coefficiens in the residue field $\CC$. We look at teh valuation of each monomial assuming $val(x)=w$, then we sabe only the monomials with the smallest valuation, and we only keep the coefficient in front of $t^*$ (where $t^*$ is the smallest term).

\begin{ex}
    COnsider $q(X) = (t^{-4} + t^2) + \sqrt{2} x) + 2t^2x^2 $. THe individual valautions are $-4$, $-3$, and $(2-6)=-4$. Since $-4$ is the lowest valuation, we only consider the first and third form. SO we only consider $t^{-4}+t^2$, and $3t^2x^2$. Then the limit with the limit of $t$,  we get $In_{-3} q = 1 + 3x^2$. We define $W:=trop q(w)$. THen $In_w q [t^{-W} q(t^wx)] |_{t=0}$
\end{ex}


($\sqrt{2}$ is ignored because valuations of roducts are added, and $\sqrt{2}=\sqrt{2}t^0$, which has valuation $0$)


We now consider $(\RR,w)$, with some fixed $q(X)$. We can define an equivalnce relation by $w_1 \equiv w_2 \iff In_{w_1}q = In_{w_2}q$. This equivalence relation decomposes $\RR$ into equivalence classes of two types. THe equivalence classes are either single points or open intervals. The open intervals correspond to when the initial form is a monomial. The single points are otherwise.


\begin{ex}
    Consider $q(X) = t^2 + \sqrt{2}x + 3x^2x^2$. At 
\end{ex}




\begin{defn}
    The complement of the locus of $w$ such that $In_W q$ is a monomial is called the Grobner complex of $q(X)$.
\end{defn}



THe Grobner complex of $q(X)$ is equal to the roots of $trop(q(x))$.



\section{More Variables}


Let $p(x,y)= \sum a_{ij} \odot x^i\odot y^j$ be a tropical polynomial in two variables. 

\begin{defn}
    Define a tropical curve $V(p)$ to be either
    \begin{enumerate}
        \item THe locus in the domain of piecewise linear $p$ where $p$ is not linear, or
        $(x,y)\; |\; \max(a_{ij} +ix+jy)$ is attained $>1$.
    \end{enumerate}
\end{defn}

We have a corrsepondence theorem.

\begin{theorem}
    If $q(x,y)$ is a polynomial with coefficients over a valued field $\CC\{\{t\}\}$, and $trop (q) = p$, then the tropical curve $V(p)$ is equal to the closure of the valuation of th epoints $\{(val(x), val(y)) \:\; |\; (x,y) \in V(q)\}$.
\end{theorem}



We can then study structural properties of tropical curves. We get correspondece statemnt with subdivisions of Newton polygon, and we get balancing and edge weights.  We will also see the tropical versions of classical plane curve theorems. In particular, we get a tropical Bezout theorem (two projective curves of degree d and e intersect in $d*e$ points) and a tropical deg/genus formula.

%9/15

To recap $p(x,y)$ a tropical polynomial, we can define the variety of p $V(p)$, whic is either the locus of non-linearity of $p$, or the locus where the maximum is attained more than once. Then for $q(X,Y)$ a polynomial, we can tropicalize it.




\subsection{Lines}
Lines are $V(p)$ such that $deg(p)=1$.
Now, to see what happens with tropical lines, conside $p(x,y) = (a \odot x) \oplus (b \odot y) \oplus c$. Assume $-\infty<a<b<c$. Then $p(x,y) = max\{ a+x,b+y,c\}$.

Setting any two equal to eachother, we get $a+x=b+y$, $a+x=c$, and $b+y=c$. We get three lines $y=x+(a-b)$, $x=c-a$, and $y=c-b$.

Every tropical line is of the form of a tripod. Even if we only keep $-\infty< a,b,c$, we still keep the tripod, and the corresponding regions of maxima are maintained. THis is because the locus are found by setting (constant plus variable) = constant which gives a vertical or horizontal line, or (constant lus variable)=(constant plus variable), which givesa  line of slope one


\begin{ex}
    What happens when some of the coefficients are $-\infty$.
\end{ex}



As a second perspective, lets let $q(X,Y)$ be a degree $1$ polynomial with coefficients from the Puiseux  series. We thus have $q(X,Y)=t^aX+t^bY+t^c$. If we tropicalize $q$, we get $trop(q)=(a \odot x) \oplus (b \odot y) \oplus c$. If we take $V(trop (q))$ we get the thing we had before modulo adjusting for switch between min and max conventions. In this context, $V(q) = \{(X,Y)\; |\; q(X,Y)=0\} \subset (\KK^*)^2$


The great thing about lines is that they can be parameterized. We can write $V(q)= \{(\alpha, -t^{a-b}\alpha -t^{c-b})\; |\; \alpha \in \KK\}$. Now, for any point in $V(q)$, we want to take the valuation $\{(val(\alpha),val( -t^{a-b}\alpha -t^{c-b}))\; |\; \alpha \in \KK^*\} \subset \RR^2$. We now study the valaution of $-t^{a-b}\alpha -t^{c-b}$, which is done by studying the valaution of the individual terms. THe first term has valaution $a-b + val(\alpha)$, and the valaution on the right equals $c-b$. Interesting things happen whenthe valautiaons are equal, when$ val(\alpha)= c-a$.


Finally, we consider $val(\alpha) = c-a$ our claim is that we can obtain any value for $Y$, but it has to be $\geq c-b$

\begin{proof}
    Let $\gamma \geq 0$, and let $\alpha =-t^{c-a}(1+t^\gamma)$. We need $\gamma \geq 0$, so that hte vlautation of $1+t^{\gamma}$ equals zero, and so the valuation of $\gamma$ remains $c-a$. Then $val(Y(\alpha)) =val(-t^{a-b}(-t^{c-a}(1+t^\gamma)) - t^{c-b} )$. This equals $val(t^{c-b(1+t^\gamma}) -t^{c-b}) = val(t^{\gamma +c-b}) = \gamma+c-b$, and so we can make $y$ take valuation any value greater than or equal to $c-b$.
\end{proof}


If we were to send $a \mapsto -a$, $b \mapsto -b$, $c \mapsto -c$, $X \mapsto -X$, and $Y \mapsto -Y$.


This process can be repeated for the Amoeba perspective. Take a family of polynomials $q_t(X,Y)$. The coefficients are functions of $t$, but we specifiy $q_t(X,Y) = t^aX + t^bY+t^c$. We now want to consdier $q_t = 0 \subset \CC^2$. For every $(X,Y) \in L$, we consider $(\log_t|X|,\log_t|Y|)$. We first study the real trace of this object, i.e. when$ X,Y \in \RR$. We now have three cases to consider. 


We can consider real image, where $X,Y \in |RR$, and further when $0< X,Y$. THen we can take $(\log_tX,\log_tY)$. We can once again parameterize to get $X= t^\alpha$, $Y = -t^{a-b+\alpha} -t^{c-b}$.  For each of the three cases, we pick up asymptotes.



%Sept 18 9/18




To recap, we start with a famil of lines indexed by $t \in \RR_{>1}$, dentoed $L_t= \{t^aX + t^bY -t^c=0\} \subset \CC^2$. We denote We have the function$ T_t$, which makes $x= \log_t|X|$, and $y=\log_t|Y|$. We can solve for thi line, and get $Y= -t^{a-b}X + t^{c-b}$.


We focus momentarily at when $X,Y \in \RR$, so we focus on $\RR^2$. In the case where $X,Y$ are both positive is our first case. In particular, $0<X<t^{c-a}$. In particular, $-\infty<x<c-a$.


We define a path $X_s :=e^{i\pi s}X_0$, where $s \in [0,1]$. Then for each $x_s$, we have a corresponding $Y_s$ whcih is the correspodning equation of $L$ for $X_s$. Now, we can ask about what happens to $T_t(|X_s|, |Y_s|$. In this case, phase changes are irrelevant to the $X$ corredenate., so $T_t(|X_s|,|Y_s|)= (\log_t(X_0), f(s))$, where $f(s)$ is continuous. This traces a full interval. which allows us to take advantage of the other cases where $X$ and $Y$ can be complex numbers.





We now take $q(X,Y)\in \KK[X,Y]$ (Consider Puiseaux series for $\KK$). We then consider the variety $V(q) = \{(X,Y) \; |\; q(X,Y) = 0 \} \subset (\KK^*)^2$. We can also take the tropicalization $p(x,y) = trop(q(X,Y))$. From here we can define the variety of $p$ to be $V(p)$< which is the locus where $p$ fails to be lienar. $V(p) \subset \RR^2$. We also have the function $(\KK^*)^2 \rightarrow \RR^2$ defiend by $(val, val)$, whcih takes the valuation fo the coordinates. We hope to call $(val, val)$ $trop$. 

\begin{theorem}[Kapranou's]
$\overline{trop(V(q)} = V(trop(q))$, where the closure is with respect to the euclidean topology of $\RR^2$.
\end{theorem}
\begin{proof}
    $\subset$ still the same idea. If $(x_0,y_0)\in trop(V(q))$, that means that there exists $(X_0,Y_0)\in (\KK^*)^2$ such that $val(X_0)= x_0$, $val(Y_0)=y_0$, and $q(X_0,Y_0)=0$. Let $q = \sum a_{ij}X^iY^j$, let $m_{ij}=a_ijX^iY^j$ be the monomial. We can then consider the collection $\{m_{ij}(X_0,Y_0)\}_{ij}$ whcih is a collection of elements of $\KK^*$ with the property that their sum $=0$. We let $\mu= \min  val\{m_{ij}(X_0,Y_0)\}_{ij}$. The claim is that their are at least two monomials whose valuaton is $\mu$. This implies that $(x_0,y_0) \in V(p)$. We have shown $trop(V(q)) \subset V(trop(q))$. However, $V(trop(q))$ is closed in the Euclidean topology (since the variety comes from equalities and inequalities).



    $\supset$ This direction is a bit tougher. We will prove the claim in dimension $0$ and proceed by induction. We first want to show that $V(trop(q)) \cap \QQ^2$ is dense in $V(trop(q))$. This is true because all monomials $m_{ij}$ correspond to affine linear functions with integer slopes and rational coefficients. We had $trop(m_{ij})=\val(a_{ij}) \odot x^i \odot y^j= \val(a_{ij}) + ix+jy$, where $\val(a_{ij})\in \QQ$, and $ix+iy \in \NN$.


    We can thus focus on rational points, checking that $V(trop(q)) \cap \QQ^2$ lives in $trop (V(q))$, and then it will follow that the closures gives $V(trop(q)) = \overline{V(trop(q)) \cap \QQ^2} \subset \overline{trop(V(q))}$.


    We proceed by the following assumption. If we have a polynomial $q(X,Y)$, we can consider it a polynomial in $Y$ with coefficients in $Y$, i.e. $q(X,Y) = r_0(X) + r_1(X)Y + \cdots r_d(X)Y^d$, with $r_i(X) \in \KK[X]$. We assume that $r_i(X)$ is a monomial for every $i$. We have $(x_0,y_0) \in V(trop(q))$, and we want to find corresponding $(X_0,Y_0)\in (\KK^*)^2$ such that 
    \begin{enumerate}
        \item $q(X_0,Y_0)=0$, and
        \item $\val(X_0=x_0$, and $\val(Y_0)=y_0$
    \end{enumerate}

    We choose $X_0$ arbitrarily, so long as $\val(X_0)=x_0$. Because we have made our assumption of $r_i(X)$ being a monomial, no matter how we choose $X_0$, $X_0$ is not a root of these monomials, and so this implies that $r_i(X_0) \neq 0$ for all $i$. Let us now consider the polynomial $q(X_0,Y_0)= \sum r_i(X_0)Y^i \in \KK[Y]$. Let $\tilde{p}(y) = trop(q(X,Y)) = \bigoplus\limits_{i=1}^d\val(r_i(X_0))\odot y^i$. Furthermore, we claim$y_0$ is a root of $\tilde{p}(y)= \min(\val(r_i(X_0) + iy) = \min(\val(a_{ij} + jx_0 + iy)$ recall that $a_{ij}$ is a Puiseux series. Then

    \begin{align*}
        \tilde{p}(y) = \min(\val(a_{ij} + jx_0 + iy)= trop(q(x,y)|_{x=x_0})
    \end{align*}
    As we started with $(x_0,y_0)$ is in $V(trop(q))$, then clearly $y_0$ is a root of our $\tilde{p}(y)$.


    By this univaraite case, there exist $Y_0 \in \KK^*$ such that $Y_0$ is a root of $q(X_0,Y)$, and $\val(Y_0)=y_0$.

    We now need to show that the polnomial case proves the general case. TO see why the assumption that $r_i(x)$ is not monomial in $X$ isn't too restrictive, consider $q(X,Y) = XY +X^2Y=(X+X^2)Y$. THis is not o the form $\sum r_i(X)Y^i$. However, we can consider $\tilde{q}(X,)=q(XY,Y)=XY^2+X^2Y^3$. This satisfies the assumption of monomials. If $(\tilde{X_0},\tilde{Y_0})$ is a solution for $\tilde{q}=0$, then  $\left(\frac{\tilde{X_0}}{\tilde{Y_0} },\tilde{Y_0}  \right)$ Is a olution for $q=0$. The key point is that $\tilde{q}$ is obtained from $q$ by an invertibe transformation in $(\KK^*)^2$.


    %If you have to divide by zero, call Chuck Norris -Renzo


    GIven $q(X,Y)$ of degree $d$, define $\tilde{q}(X,Y) = q(XY, Y^{d-1})$, this satisfies teh assumptions we have made. If we have $q(X,Y) = \sum r_{ij} x^iY^j$, then $\tilde{q}(X,Y) = \sum r_{ij}X^iY^{(d+1)j+1}$. Now, if we ask if its possible to have conflicitng $i,j$, ie.e $(d+1)j_1+i_1=(d+1)j_2+i_2$? This says we need $(d+1)j_1-j_2=i_2-i_1$. However, $0\leq i_1,i_2 \leq d$, so there differnece is $\leq d$. Furthermore, $j_1-j_2 \geq d+1$ when $j_1=j_2$. (Easier proof, sifting powers of $Y$ first, then assure powers of $X$ cannot cause overlap).



    
\end{proof}

We can ask which polynomial with Puiseaux valued coefficienst has as it's tropical polynomial $p(x,y)=xy\oplus x \oplus \oplus y 0$. valuations of zero only occurs with the constant complex numbers. We set the coefficients associated with $X$ and $Y's$ to be $1$, so we let $q(X,Y)= XY+X+Y+C$, where $C \in \CC$. $q$ is a polynomial in $\KK[X,Y]$ with the property that $trop(q) = p$. Now, for articular $q$, we can ask for $V(q)$. For $q=(X+1)(Y+1)+(C-1)=0$, we can ask for hte set $\{(X+1)(Y+1) = C\}$. This is a translation of the $\CC^2$ hyperbola where the asymptoes are the axis $X=-1$ and $Y=-1$.


We can ask to describe this curve in the projectiv plane. We homogonize with $Z$, to then describe ht epoints at infinity. It is odd that The curve hits infinity twice, considering thats two of the three special points of the projective plane of $\CC^2$, the other being the origin. Instead, we can compactify via $\PP^1\times \PP^1$, the product of $\CC^2\cup \{\infty\}$. We do this by aking the polynomial bi-homogeneous (homogenous on $X$, and Homogenous on $Y$). So we have $\tilde{q} = X_1Y_1 +X_1Y_0+Y_1X_0 + X_0Y_0C = 0$. THis is done by treating $Y$ as a constant, then we get homogenous in $X$, and vice versa. THis is bihomogenous of degree $1$. Now we do not intersect the four special points at all, and we intersect each of the special lines once. This gives us general behavior (transversal intersection with the boundary).

Somehow, the shape of the tropical curve tells us that it is tropicalization of some plane curve, but it should be compactified in $\PP^1\times \PP^1$, not $\PP^2$.





\begin{theorem}{Bummer}
    Let $q(X,Y)= \sum\limits_{i+j\leq d}a_{ij}X^iY^j$ be a polynomial of degree $d$ in $\CC[X,Y] \subset \CC\{\{t\}\} [X,Y]$, and all coeffiencs are $\neq 0$, i.e.e $a_{ij} \neq 0$ for all $i,j$. Then $\overline{\trop(v(q))}$ looks like a tropical line with vertex at $(0,0)$. 
\end{theorem}

This occurs for trivially valued field, where $0 \mapsto \infty$, and everything else maps to $0$.


\begin{proof}
    The tropicalization of $q$ is $\bigoplus x^i\odot y^j$ (the $a_{ij}$ are all nonzero $\CC$, o their valuation is $0$). Then $\trop(q) = \min\{ix+jy\}_{i+j\leq d}$. Then the minimum is always obtained by $0$ in the first quadrant when $i=j=0$, $dx$ in the section containing the second quadrant, and $dy$ in the section containing the fourth quadrant.
\end{proof}



\begin{defn}
    Let $p(x,y) = \bigoplus a_{ij} \odot x^i\odot y^j$ be a tropical polynomial. The \emph{Newton polygon} of $p$ is the convex hull of $(i,j)$ such tha t$a_{ij} \neq - \infty$.
\end{defn}


\begin{defn}
Let $\Sigma$ be the convx hull of the points $(i,j,a_{ij})\subset vert(NP)\times \RR$, $\Sigma$ (vert(NP) is the vertex set of the newton polygon) is a convec polytope in $\RR^2 \times \RR$. We consdier $\pi_z: \RR^2 \times \RR \rightarrow \RR^2$ defined by $(x,y), z \mapsto (x,y)$, and let $\tilde{N}$ be the subdivision of the newton polytop by projectin the corerns of $\Sigma$ you can see from above (positive $z$ coordinate). Then the tropical curve $V(p)$ is \emph{dual} to such a subdivision, i.e.

\begin{enumerate}
    \item There is a bijection between vertices/edge of a tropical curve $\leftrightarrow$ faces/edges of $\tilde{N}$,
    \item reversing poset structure given by inclsuion into the closure,
    \item every edge of $V(p)$ is $\perp$ to the edge of $\tilde{N}$ it corresponds to, and
    \item The coordinates of a vertex are found by solving the linear system of equations obtained by setting equal the linear functions corresponding to monomials corresponding to vertices of the face of the $\tilde{N}$ dual to $V$.
\end{enumerate}

\end{defn}


\begin{ex}
  The Newton polynomial of $p(x,y) = 0 \oplus x^2 \oplus y^2 \oplus 1x \oplus 1y \oplus (1+xy)$ has six points in a triangle, the three vertices and the three midpoints of the line segments. Each point gets a height corresponding to the value of the coefficent of the monomial ($y^2$ has coefficent 0, and thus weight 0, while $1xy$ has coefficient $1$, and thus weight 1). We then get the three dimensional polytope $(a,b, wt(a,b))$, drape  acurtain over the top, and we get a distinguished top correspondign to the triangle with vertices (the midpoints of the line segments whcih had weight 1).
\end{ex}

\begin{theorem}
    If $q(X,Y)\in \KK[X,Y]$ such that $trop(q) = p $ (modulo adapting for min/max). Then the subdivision of the Newton polytope keep track of the initial forms of $q$ , in the sense that for any cell in the Newton polygon subdivision the initial form is given by the monomials corresponding to the lattice points in the cell.
\end{theorem}


\begin{ex}
    Consider $q(X,Y) = 7+3X^2+Y^2+t^{-1}X + 2t^{-1}Y + t^{-1}XY$.
\end{ex}


The silly but crucial observation to prove this theorem is

\begin{lemma}
    Evaluating a tropical monomial at a point $(x_0,y_0)$ can be done as a dot product
\end{lemma}
\begin{proof}
    A tropical monomial is of the form $m=a\odot x^i \odot y^j$. Then $m(x_0,y_0) = a + ix_0 +j_0 = (i,j,a)\cdot(x_0,y_0,1)$.
\end{proof}


\begin{proof}

When we construct the subdivision of the Newton polygon, we consider all points wit coordinates $(i,j,a_{ij})$ as $i,j$ range where $a_{ij} \neq -\infty$. So evalauting at $(x_0,y_0)$ ammounts to searching for the maximum of the dot product of the vector $(x_0,y_0,1)$ with all points $(i,j,a_{ij})$.


Evaluating the tropical polynomial at the normal vector to the plane at the top of the poltope for vectors on the edges of that face, we get zero, so the evaluation at the vertices of the vector are equal, so the evalaution of the point at the two monomials is equal. We want the vertex to be on the face of the tropical curve. For any other dot products, the evalautions are negative (the other vertices are below the plane, so the evalautions of the dot product will be negative). We construc thte vectors from the plane down, so the vertex on the face is larger than the ones below.  So we have $m_{ij}(n_x,y) <\tilde{m_{ij}}(n_x,n_y)$. When $m_{ij}$ correspond to vertex not in face, and $\tilde{m_{ij}}$ correpsond to vertex in face.



This si doen for every face of the Newton Polygon. then $(n_x,n_y) \in \RR^2$ is the vertex of the tropical curve dual to the particular face we were considering.




Precisely, we consdier the faces of the convex whose outward pointing normal has psotiive $z$ coordinate. In higher dimenion, we say the last coordinate of the normal vector of the face has positive value.

    
For the edges, we focus on a particular edge $e$ of the Newton polygon. $e$ bounds two face $F_1$ and $F_2$. $F_1$ and $F_2$ have hteir respective normal vectors $n_1$ and $n_2$. $e$ dots to zero with $n_1$ and $n_2$. THe same is true for all linear combinations of $n_1$ and $n_2$. In particular, it is true on the segement connecting $n_1$ and $n_2$. So every point in the segment in $\RR^2$ joining $(n_x,n_y)_1$ and $(n_x,n_y)_2$ has the property that the vector $(n_x,n_y,1)_i \dot (m_1)=(n_x,n_y,1)_i \dot (m_2) > (n_x,n_y,1)_i m_{other}$. As the maximum is obtained twice, those points belong to the tropical curve.

\end{proof}


\section{Sufficient conditions for tropical curves}
%9/27 Sept 27

\begin{defn}
    Any edge of a tropical plane curve $V(p)$ is given weight $\omega_e$ equal to the lattivce length of the segment of the Newton Polytope Subdivision dual to the edge.
\end{defn}

\begin{defn}
    A \emph{primitive vector} of teh direction vector $p$ is the first integral vector (vector with integer coordinates) after the origin whic lies on the line defiend by $p$ in the directon of p.
\end{defn}


\begin{theorem}
    Tropical plane curves are balanced, i.e. at every vertex, the summation $\sum\limits_{v \in e}\omega_e \overrightarrow{p}_e=0$, where $\omega_e$ is the weight of the edge, and $\overrightarrow{p}_e$ is the primitive vector in the direction of $e$.
\end{theorem}

\begin{proof}
    Any vertex $v$ is dual to a face $F_v$ of the Newton polygon subdivision. For every edge bounding $F_v$, the vector $\omega_e p_e$ is obtained by the vector tracing the dual edge via the lienar transformation $(x,y)\mapsto (y,-x)$. Now,  $\sum\limits_{v \in e}\omega_e \overrightarrow{p}_e=0$ is equivalent to the poylgon being a closed polygon.
\end{proof}


Now that we have defined weights, we want to ask what they represent.


If we are just looking for solutions of $x^2yP(x^3y)=0$ in the torus $(\CC^*)^2$ or asymptotically ($|x|,|y|>>0$), then (A) the monomial part $x^2y$ is irrelevant (Only vanish at zero or infinity), and (B) We have $\deg(P) =$ lattice length of the segmentOn th etorus, we have 1-parameter subgroup orbits of the form $x^3y=r_i$, where $r_i$ is a root of $p$ and is counted with multiplicity.


So if we take a point in the tropical curve $w$. We Think of the Puiseaux plane as many planes assembled by valaution. Now that we have fixed valaitions, we single out a aprticular $\CC^2$. When $x,y$ are very large, the polynoial is approximated by the initial form, sow e get (eight of edge) number of torus orbits. THe direction of the edge in the subdivision tells us what we get for our 1-parameter subgroup. The primtive vector is $(3,1)$, so we get $x^3y$.


The following are things to consider about tropical curves. We can draw all topological types of tropical plane conics and cubics. We can experiment wit various tropical plane curves and seek a conjecture to compute their $b_1$. FInally, we can study a pencil of tropical conics,i.e. draw a conic, pick four points on it in general position, then find all conics through those four points.

When followning four parameter conics throught the four points, we find interesting points between leaves and edges.

%oct 2, 10/2




\section{Intersection of Tropical Curves}

\begin{defn}
Two tropical curves intersect \emph{transversally} if they intersect in finitely many points which are not vertices of either curve.
\end{defn}


Our second kind of intersection will be stable intersection.
Let $v \in \RR^2 -\{(0,0)\}$, and define $\Gamma_1 \cap_v \Gamma_2= \lim\limits_{t \rightarrow 0} \Gamma_1 \cap (\Gamma_2 +tv)$, i.e. translate $\Gamma_2$ by the vector $v $. If we were to pick another $v$, our translations would be different. We are forced to ask ourselves if distinct choices of $v$ may lead to different limits.


\begin{note}
    Points of intersections should be weighted by multiplicites of the edged they belong to.
\end{note}



\begin{ex}
    How do the complex curves $C_1=\{x^a=y^b\}$ and $C_2=\{x^c=y^d\}$ intersect in $\CC^2$?
    \begin{sol}
        We can parameterize $x=t^b$, $y=t^a$, then we get $t^{bc}(t^{ad-bc}-1)=0$
    \end{sol}
\end{ex}


\begin{defn}
    Let $p$ be a point of transversal intersection of $\Gamma_1, \Gamma_2$. We let the multiplciity of the point of intersection $p$ to be 
    \begin{align*}
        M_p(\Gamma_1, \Gamma_2):=w_{e_1}w_{e_2} \left| \det \begin{bmatrix}
            P_{1,x} &P_{2,x} \\
            P_{1,y} &P_{2,y}
        \end{bmatrix}  \right| =[\ZZ^2\;|\; p_1 \ZZ + p_2 \ZZ]
    \end{align*}
    Where $p_i$ are the direction vectors of the edges for the interection, $p_{1,x}$ is the $x$ coordinate of the direction vector for edge $1$, and $w_{ei}$ is the weight of the corresponding edge
\end{defn}


%Oct 16 10-16


Last week: Fans $\Sigma$ lead to toric varieties $X_\Sigma$. Maximal cones lead to affine charts, and faces lead to transition functions. $\Sigma \subset N_\RR$. 


we get two neat consequences.
\begin{enumerate}
    \item THere is an incluision into closure reversing bijection between cones of $\sigma$ and the torus orbits of $X_\Sigma$ (also,e xhcanging dimension with co dimension).
    \item $T$-equivariant maps of toric varieties correspond to maps of fans.
\end{enumerate}

\begin{defn}
    givne $\Sigma_1 \subset N_{1\RR}$ and $\Sigma_2 \subset N_{2\RR}$ a \emph{map of fans} is a $\ZZ$-linear map $L: N_{1\RR}\rightarrow N_{2\RR}$ such that for every cone $\tau$ of $\Sigma_1$, $L(\tau) \subset $ a cone of $\Sigma_2$
\end{defn}
with the embedding, we do not necessarily want to subdivide fans,a s that introduces non trivial orbits.

If a toric varitey is made of toric variteies, torus are homotopic to circles, whch have genus 0, os they contribute trivially.
TO see how this bijection works

\begin{enumerate}
    \item FOr every cone $\tau$ of $\Sigma$, look at the limits a $t \rightarrow 0$ of  torus orbits of $1$-parameter subgroups $|gamma$< with $\gamma \in \tau^0$ (in the interior of the cone $\tau$ ).
    \item For every affine patch dual to a cone, set all the coordinates taht you can set to zero to zero
\end{enumerate}



Toric varieties can be viewed as a genralization fo projective space (we basically have homogenous coordinates). We consider the Quotient Construction. Given a fan $\Sigma$, the toric variety $X_\Sigma$ can be obtianed a a quotient space of the form $\CC^{N}- \{Irrelevalnt\} / G$, where $N$ is the number of rays in the fan $\Sigma$ (for every ray we get a homorgenous coordinate in the toric variety), the irrelevant stuff is the locus determined by sets of rays that do not span cones of the fan, and $G$ is given by linear relations among rays.



%Oct 18, 10-18


Toric varieties have an orbit cone corespondence. THis a bijection between the cones of a toric variety, and the toric orbits.  We have that good maps of toric varieties correpsond to maps of fans. We also hav a quotient contruction $X_\Sigma = \CC^N - \{stuff\}$, where $N$ is the number of rays, the stuff is the collection of rays not spanning a cone, and $G$ is a linear relation among rays.

\begin{ex}
    $\PP^2$ has three rays, denoted by $P_Y$, $P_X$, and $P_Z$. THe only subset of rays NOT spanning a cone is the subset $\{P_X, P_Y, P_Z\}$ (any two span a cone). so we throw away the locus $\{X=Y=Z=0\}$, i.e.e the origin. The relation we have between the three is that $1P_X + 1 P_Y + 1 P_Z=0$ (the vectors are $P_X=(1,0)$, $P_Y=(0,1)$, and $P_Z=(-1,-1)$). all of the 1's tell us we have a one dimensiaonl torus, and our action will be $t(X,Y,Z) = (t^1 X,t^1,Y,t^1Z)$ With these relations, we get $\CC^*$

    Now,w e can take the fan, and declare that any ray of the fan is generated bya  basis vector of a new vector space. so $P_X=(1,0,0)$, $P_Y=(0,1,0)$, and $P_Z=(0,0,1)$. Then our three corrsponding rays in $\RR^2$ become the three positive half lines of the X, Y, Z axis. Now we want to lift the cones in our toric variety, so we get octant planes Now, everything we make is contained in the first octant of $|RR^3$, so ths is a subset of all the cones of the first octant, but the first octant with removal Bu the first octant is $\CC^n$, and we removed something. We remove orbits corresponding to not being centered in th eoctan, i.e.e we remove the three dimensional cone, so teh toric variety is $\CC^3$ minus the orbit correpsonding to the cetner cone, whcih is $(0,0,0)$.
\end{ex}





In tropica geoemtry, $k $ is identitfied with $\TT= \RR \cup \{\infty\}$, then $k^*$ corresponds to $\RR$> THen $T= (k^*)^n$ correpsonds to $\RR^n$, and the action $*$ corresponds to $+$. 



How to make tropical $\PP^2$


%Oct 20 10-20

\section{Tropical Toric Varieties}

From th efan $\Sigma$ we use tropical numbers and tropical operations for transition functions, i.e. $\TT \PP^2$ provides us with $\TT^2, x_1,y_i$, for $i=1,2,3$. We get relations from $i=1$ and $i=2$ via $x_i=-x_2$ \& $y_1=y_2-x_2$, from $2$ to $3$ via $x_2 = y_3-x_3$ \& $y_2 = -x_3$, and finally from 1 and 3 via $x_1 = x_3-y_3$ \& $y_1 = -y_3$.

We get invertible in GLZ, which is our notion of automorphism of the torus. We can always do this translation. We observe that the tropical toric variety $\TT X_\Sigma$ has a stratification into ``$\RR^k$''strata, whcih has a natural poset isomorphism with the stratification of the complex toric variety toric variety $X_\Sigma$. 
Furthermore, $\TT X_\Sigma$ has the structure of an $``\infty''$ polytope, i.e. scale the polytope so that lengths go to infinity, which is the normal/dual polytope to the fan of $X_\Sigma$.

A stratification is a disjoint union into locally closed spaces. We get 0-dim, 1-dim, and 2-dim spaces, each is a tropical tori. THis is similar as how $\CC\PP^2$ has a startification as points, lines, and the surface of the triangle. 



Now, the shortcut to get the tropical troic variety via the fan of $\PP^2$, we make rays orthogonal to the rays of $\PP^2$, introduce fans correspodnign to each of the 2-dim pieces of the fan, and set the lines to infinty.

The role taken by $\lim\limits-{t\rightarrow 0} t*p$ in $\CC$-toric land is replaced by $\lim\limits-{T \rightarrow \infty} T+p$ in tropical rotic land. So if we tak e a one parameter subgroup, say $(1,1)$, and we ask for hte orbits of this subgroup, the orbits are found fvia $T(x_1,y_1) = (T+x_1,T+y_1)$. SO regardles of the starting point, the orbit gets to the point at infinity. FOr the one parameter subgroup $(2,1)$, we have $T(x_1,y_1) = (2T+x_1, T+y_1)$.



We can also show how the quotient construction works tropically. We being at $\PP^1$. IN our quptient construction, this is $\CC^2-(0,0)/\CC^*(t(x,y)=(tx,ty))$, where $t*(x,y)=(tx,ty)$. THe one parameter subgroup comes from the linear relation on the fans of $\PP^1$, the relation being $p_1+p_2=0$. Now we do the same from the fan to get us into tropical space. $\TT\PP^1 = \TT^2 -(\infty, \infty)/\RR (T(X,Y)=(T+X,T+Y))$, where $\RR$ is the torus on $\TT$. We expect $\TT\PP^1$ to be a line segment, as we have $\PP^1$ is two rays connected. We get a point for the first ray, a point for the second, and a line connecting the two. THis is related to initial forms (See other texts).


We have parallel lines connecting to the singular point at infinity which we remove. This means that each parallel ray hits a different point at infinity. We can create an orthoognal line to these rays, and we say this ray hits two other points at infinity (at the limtis of identifying the x and y axis).



We get some kind of orthogonal line to these parallel rays, and this line intersects each orbit exactly once. This line connects at poitns which represent $(\infty,0)$ and $(0, \infty)$. 


We have our original consturction of tropical curves. Now, assume we have $Y \subset Torus \subset X_\Sigma$, so a curve in a torus in a toric variety. Now e have $Trop(Y) \subset \RR^n \subset \TT X_\Sigma$. 

\begin{defn}
    The \emph{extended Tropicalization} of $Y$ inside of $X_\Sigma$ is the closure of $Trop(Y)$ inside of $\TT X_\Sigma$.
\end{defn}


So we start with a line $L \subset (\CC^*)^2 \subset \PP^2$. This gives rise to a tropical line $Trop(L) \subset \RR^2$. THe closure of $Trop(L)$ gives three aditional points, one for each section of the tropical $L$ which hits lines at infinities. 


Next time, we will compare the combinatorics of troical liens in varieiteis to toric varieities. 



%Oct 27 10/27


\begin{theorem}
     $\overline{Y} \cap O_\sigma \neq \emptyset$ and dimensional transversality $codim_{\overline{Y}}(\overline{Y} \cap O_\sigma) = codim_{X_\Sigma}(O_\sigma)$ iff $Trop Y= |\Sigma|$
\end{theorem}

\begin{proof}
    $\implies$ we know that $trop(Y) \subset |\Sigma|$. We want to prove $\supset$. First, we have that any $\sigma \in \Sigma$ must intersect $trop Y$, as $\overline{Y} \cap O_\sigma \neq \emptyset$. Second, we have that $\dim(\Sigma)$ is at most $\dim(Y)$, by the dimensional transversality. Third, we have that a top dimensional cone of $\Sigma$ cannot intersect $trop(Y)$ in a positive codimension locus, otherwise $Trop(Y)$ would intersect $(d+1)$ dimensional cones of $\Sigma$ (which don't exist). Finally, a top dimenional cone of $\sigma$ cannot be partially covered by $trop(Y)$, as either this would violate the balacing condition for $Trop (Y)$ (balancing condition in hgiher dimensions is done by quotienting), or trop Y has to intersect at a face of $\Sigma$, not the of a face. 
    
    
    With these four conditions, we get that every $\sigma \in \Sigma$ must interect $trop(Y)$, and the itnersections are all showing $trop()$ is covering $\sigma$, so we must have that $|\Sigma| \subset trop(Y)$.


    $\impliedby$ Now we are assuming that $trop(Y) = |\Sigma|$. We need some algebraic geometry black boxes.

    \begin{enumerate}
        \item There are no positive dimensional compact subvarieties of tori.
        \item For toric varieties, an orbit $O_\sigma$ of codimension $k$ is (locally) cut out be exactly $k$ equations
        \item FOr any subvariety $Z \subset X_\Sigma$ and any hypersurface $HS \subset X_\Sigma$, $Z \cap HS$ either rmains the same dimension of $Z$, the dimension goes down by exactly one, or the intersection is empty.
    \end{enumerate}

    WIth these three facts we can beign to show the other implication. We begin with $\sigma_d$ being a top dimensional cone in $\Sigma$. Then$ O_{\sigma_d}$ is a codimension $d$ orbit. FUrthermore, we have $O_{\sigma_d} \cap \overline{Y} \neq \emptyset$, and that $O_{\sigma_D}$ is isomorphic to some torus. Because $\overline{Y}$ is complete, the intersection $O_{\sigma_d} \cap \overline{Y}$ is complete, and so it has a finite number of points.


    All together, this says we can obtain the orbit $O_\sigma$ by cutting down by $k$ hypersurfaces. In particular, if we have our cone $\sigma$, we can take a chain of subsequent surfaces $r_1$, $span(r_1,r_2)$, $span(r_1,_2,r_3)$, and correpsonding to $r_1$ we have a hypersurface $H_1$ (orbit of $r_1$, $H_1 \cap H_2$ is orbit of $\sigma_2$, and so on), we can take $\overline{Y} \cap H_1$, then$ \overline{Y}\cap H_1 \cap H_2$, and then $\overline{Y}\cap H_1 \cap H_2 \cap H_2$. We know $\overline{Y}$ has dimension 3, at the end we have just points at the end, and we drop dimensions by exactly one at every step.



    


    %Oct 30 10/30

 


    We now reiterate the previously stated algebraic geometry black boxes.
    \begin{enumerate}
        \item The only complete/compact subvariety of a torus are 0-dimensional
        \item In a smooth toric variety, every orbit of sigma $O_\sigma$ of codimension $k$ i locally cut out by $K$-equations.
        \item If $Y \subset X_\Sigma$, and $Y \cap Hypersurface \neq 0$, then the dimension of the intersection goes down at most by $1$.
    \end{enumerate}

Now we assume $Trop(Y)=|\Sigma|$   We can consider some gace $\sigma\in \Sigma$, and let $\tilde{\sigma}$ to be a top dimenisional cone of $\Sigma$ containign $\sigma$ as a face. We choose an ordeing o the rays of $\tilde{\sigma}$ such that the first $k$ rays belong to $\sigma$. (Each ray corresponds to a codimension 1 orbit, after the $k$ steps we get to an intersection with $O_\sigma \cap Y$). We let $H_i$ be the closure $\overline{O_{\rho_i}}$, i.e. the hupersurface in $X _\sigma$ corresponding to the ray $\rho_i$. Now, $dim(\overline{Y} \cap H_1) \geq dim(\overline{Y}) -1$. Then, $\dim ((\overline{Y} \cap H_1)\cap H_2) \geq \dim(\overline{Y})-2$. Eventually we get to $dim(\overline{Y} \cap H_1 \cap \cdots \cap H_k) \geq dim(\overline{Y})-k$. To check that we get minimal dimensions,w e terminate at $dim(\overline{Y} \cap \cdots \cap H_d) = dim(\overline{Y} \cap O_{\tilde{\sigma}})$. Note $\overline{Y} \cap \overline{O}_{\tilde{\sigma}} = \overline{Y} \cap O_{\tilde{\sigma}}$ (inherently the closure of $Y$ misse s any missing limit points of the orbit). So we know that $\overline{Y} \cap \overline{O_{\tilde{\sigma}}}$ is compact, comapct living in the torus $O_{\tilde{\sigma}}$, but by our black box we get that subvariety of a torus is 0 dimension, at every step we needed to reduce our dimension by exactly once, so at no point would our dimenion ever stay the same.




\end{proof}

\begin{defn}
$\overline{Y} \subset T \subset X_\Sigma$ is a \emph{tropical compactification} when $trop(Y) = |\Sigma|$
\end{defn}


This definition says that $Y$ thinks $X_\Sigma$ is a good place to be comapctified in. This is because $X_\Sigma$ doesnt waste any orbits, and all orbit of $X_\Sigma$ are dimensionalyl transverse to $\overline{Y}$.


In particular, the nice properiesof the left hand side of the theorem hold. So far we have not talekd about the singularity of $\overline{Y}$ when we compactify it. It is possibly that $\overline{Y}$ could have hotten singular in this process. Via a technical process and analysis with a toolkit from a third course in algebraic geometry, we notice that he statement of tropical compactificatioojon is discussing the support of $\Sigma$, not the fan itself.


In torc geoemtry if we take a fan and subdivide into cones, this corresponds to blow ups in strata of troic varieties (whcih are used to resolve singularities). So maybe we had singlaurities, but we use blow ups to reslve singularities.

\begin{enumerate}
    \item One can always refine $\Sigma$ so that $\overline{Y}$ is Cohen-Macaulay (read ``not too badly singular'')
    \item In Char 0, $X_\Sigma$ projective, then we can find an open subset such that $\overline{Y} \cap O_\sigma$ is smooth for all $\sigma$.
\end{enumerate}



%Nov 1, 11/1



Now, $K$-trivially valued, tropical compactification tells us that if $Y \subset T$, we get that $trop(Y)$ determins a toric variety inside whcih $Y$ compactifies nicely!

\section{Geometric Tropicalization}

If $Y\subset X_\Sigma$ is  a subset of a toric variety, and $Y$ sits ``nicely'' in $X_\Sigma$, then the toric variety allows us to know $trop(Y)$. We illustrate this in an example, and then generalize.

\begin{ex}
    We take the line in $\PP^2$, $\{X+Y+Z=0\} \subset \PP^2$. THe strate of the line is defiend via a boundary complex, where smaller stratifications lead to larger structures in the boundary complex. In the case of $\PP^2$ and the boundary complex of $L$, we get three points $\{a,b,c\}$. We can construct the cone over the boundary complex $cx$ by takinga  point in an independent dimension from $a,b,c$, and joinign th epoints with half lines. The toric variety has 3 divisors (codimension 1 subvarieties, in $\PP^2$ it is the lines, not lying within the torus) on $\PP^2$ that induce (via intersection) 3 divisors on $L$ (the points of intersection of $L$ with the toric invariant lines of $\PP^2$). DIvisors give rise to ``divisorial valuation'' $val_D:K(L) \rightarrow \ZZ \cup \infty$, where for every function $f\in K(L)$ it defines the order of vansihing or pole of $f$ along $D$. so as longa s the space is tnot too singluar, locally and divisor has one local equation, take the rational function where on this local set is valid, and see how many times we can factor the equation with our function.


    Notice that $val_a:K(L) \rightarrow \ZZ\cup \infty$, where we get a rational function on $\PP^2$, restrict to $L$, then do the evaluation, but we really like monomials as rational functions. So we actually take $val_a:M _T\rightarrow K(L) \rightarrow \ZZ\cup \infty$, where $M_T \rightarrow K(L)$ is via restricution, and $K(L) \rightarrow \ZZ \cup \infty$ is the order of vanishing. So we asign an elmeent of $M$ a valuation, which is dual to M, so $image(val_a) \subset N_T$, which is connected to the co-character lattice.

    $M_T$ is the character space of the torus i.e. monomials, and $N_T$ is the co-character 1-parameter subgroups.

    For our recipe, we use $val_a$, $val_b$, and $val_c$ to get three points in $N_T$. But we think of each of these points as points of height one in the cone of the boundary complex, and we take hte cone over these point in $N^T$. We then get $Trop(Y)$. 



    Now we do comupations, we take the point $a=(0:1:1)$. We can parameterize ht elien $L$ aroundhte neighborhood of $a$ via $t$ by defineing affine coordiantes in $\PP^2$, so we take $x = \frac{X}{Z}$, $y= \frac{Y}{Z}$, asn the line we paraterize as $x(t)=t$, and $y(t) =1-t$. We now need to compute $val_a$. In thise case, $M_T$ is $\ZZ_2$, so we can test on the generating set $x$ and $y$. So $val_a(x)=1$, as the ordering of vanishing of $t$ at $0$ is $1$. $val_a()=0$, as the order of vanishing of $1-t$ at $t=0$ is $0$. Similarly, $b=(1:0:1)$, this swithces the role of $x$ and $y$, so $xx(t) = 1-t$ ad $y(t)=t$, so $val_b(x)=0$ and $val_a(y)=1$. THe trickier point is $c$, whcih is the point at infinity. We can choose a parameteriation of $L$ near this point centered at $c$, which becomes $x(t)=\frac{1}{t}$ and $y(t) = 1-\frac{1}{t}$. Now $val_c(x)=-1$ and $val_c(y)=-1$. We thus define $val_a=(1,0)$, $val_b=(0,1)$, and $val_c=(-1,-1)$. THis gives us Trop(L). 


\end{ex}



In general, we start witha  avery affine variety with divisorial boundary. A divisorial boundary, we start with a space $Y$, with codimension $1$ subvarietiies, verry affine means that $Y- \partial Y$ is isomorphic to a closed subvariety of a torus (closed inside the torus, typically the complement lies in a torus).

\begin{defn}
    We say $\partial Y$ has combinatorial normal crossings if every time $n$ divisors intersect hte intersect in codimension $n$. If the intersection is transversal, we say this is a simple normal crossing
\end{defn}

Simple normal crossing means that the boundary locally looks like hyperplane coordinates in our space.

\begin{defn}
A noraml crossing is locally SNC
\end{defn}






We can ask for divisorial evaluations. We embedd the curvve minus its boundary intoa  torus. In practice we can embedd tours boundary into any toric variety we choose. When $Y- \partial Y \subset T$, and toric variety containing $T$ as their dense torus gives rise toa  divisorial valuations.

\begin{lemma}
    \begin{enumerate}
        \item $Trop(Y) = \{c *val_D \; |\; c \in \RR_{\geq 0}, \ \text{D is any divisor comiong from and }\; X_\Sigma \supset T\} \subset N_T$
        \item If $\partial Y$ is combinatorial normal crossing, then there xists a map $\pi: C \triangle_{\partial Y} \rightarrow N_T$, $Y$ was a variety with some divisor, sow e use those varieties, so that $\pi(D_i,1) \mapsto val_{D_i}$, and we extend by linearlity.With combinatorial normal crossings, we can guarantee that $Trop(Y) \subset Im(\pi)$. IF we want equality, we ensure simple normal crossing boundaries, or (if char 0 is fine), comibinatorial normal crossing with characterisitc 0 is sufficient.
    \end{enumerate}
\end{lemma}



%Nov 6 11/6

We get  a ma from the torus $T^2$ to $\PP_\KK^3$ via $(s:u) \mapsto (1::u:tsu)$. We get the conic $x_0x_3=tx_1x_2$. For $t\neq 0$ we get he smooth quadratic, for $t=1$ we get the itnersection of planes.
The two copies of $\PP^2$ which have the line is also given by the lines of Puiseaux valued points correspodnign to the vfact taht the verticeis of the curve are $(0,0)$ and $(1,1)$. The tropical curve tells us that the central fiber of the family of curves corresponds to $x+y=0$, when $x_0$ and $x_3$ both equal $0$.


When the valuation of $x$ and $y$ are both zero, then the valaution fo $txy$ is irrelevant.


We get a fan in $N_T \times \RR$ and a ray dealing with $t$ depending on expandin g on $t$ from $1+x+y+txy=0$. We have the inverse image of a fan $\PP^1 \times \PP^1$, under the special./generic fiber, we have the actual tropical curve.









\section{Student Presentations}

\subsection{Joel, Gfan }

Gfan is a software package for computing Grobner fans and tropical varieties. Overview of installation. Gfan runs on the kernel.Gfan consists of various programs. Gfan can be run by creating scripts to pass to teh Gfan library. Gfan natively supports $\QQ$ and finite fields. Integers can be used with some work. For example, we can begin with an ideal, such as $\QQ[x,y,z]$ and the ideal $\{xxy-z,yyz-x,zzx-y\}$. THe library ca compute a Grobner basis for any such ideal.

There is a relationship between Grobner basis and tropical varieities. A Grobner basis can give a Grobner cone. A collection of these Grobner cones construct a Grobner fan (distinct from tropical fan). A tropical variet is a union of Grobner cones, and is thus a subfan of a Grobner fan. Gfan does not have the most advanced visualization techniques. Gfan can visualize  figure files (only with xfig). We can imagine the Grobner fan exists in $\RR^3$ in the positive quadrant. The Grobner cone is 2-D cones. THe unit vectors in $\RR^3$ form a triangle, and hte imge is hte intersection of these cones with that triangle. Given an ideal and a permutation in a permutation group, it can compute a cone in an orbit of tha tpermutaiton group.

If we want to consider troical varieties, we can give Gfan a principle ideal $(x+y+z+w)$ within $\QQ[x,y,z,w]$. If we wante dto ask more of the tropical verities, we would run tropical interection to compute the rays, the cones generated by teh rays. This is a tropical hyperplane in $4$-dim.

Genericalyl, Gfan does trivially valued fields. It is possible to add nontrivial valuations, butt hat takes longer to compute.


%Nov 10, 11/10
\subsection{Kristina, Tropical Geometry of Deep Neural Networks}

There is an equivalence between feedforward neural networks with ReLU activiation and tropical rational functions.

(Disucssion of the cat vs dog pictures for neural networks) Orientation, identifiable features, replicate human brain processing. Neural networks have hidden layers. Each layer is a matrix product (weight assignment) and vector addition (bias). We then introduce cost functions to compare results, back propogation (gradient descent) to adjust weights.

We have activation functions. THe first is the sigmoid $\sigma(Ax+b)= \frac{1}{1+e^{-(Ax+b)}}$ we also have the rectified linear unit ReLU $\sigma(Ax+b)= \max(0,Ax+b)$. ReLU makes data more sparse, and looks like tropical geometry. First we make assumptions about our $L$-layer network. We assume that the weight matricies $A^(1), \dots, A^{(l)}$ are integer values, the biasvectors $b^{(1)}, \dots, b^{(l)}$ are real valued, and the activation functions take the form  $\sigma(Ax+b)= \max(0,Ax+b)$.


To build our equivalence, we first consider the output from teh first layer in the neural netwrok $\nu(x) = \max\{Ax+b,t\}$, where $t \in (\RR\cup \infty)^l$. So we can rewrite $\max\{Ax+b,t\} = \max\{A_+x+b, A_-x + t\} - A_-x$. SO every coordinat eof a one layer network si the difference of two tropical polynomials. FOr networks with uktuple layers, aplky this decomposition recursively.

\begin{theorem}
    A feedworard neural network under the assumptions is a function $\nu: \RR^d \rightarrow \RR^p$ whose coordinates are tropical rational fuinctions of the input, i.e. $\nu(x) F(x) \oslash G(x) = F(x) -G(x)$, where $F$ and $G$ are tropical rational functions.
\end{theorem}


We can use this equivalence to consider decision boundaries of a neural network. The input space of a neural network is partitionined into disjoint subsets, where each subset determines a final deciison (what is a dog, what is a cat). SO our input space might be a tropical curve, and the 2-cells give deicison boundaries. We can bound the number o linear reions of a NN byy bounding vertiices in the dual subdivision of the Newton polyon. THis number of linear reions measures complexity of a neural network. THese dont make better bounds, but it shows that tropical geometries can do the same work.

\subsection{Jacob, The Joswig Algorithm}


We let $\TT:=(\RR\cup\{\infty\}, \min, +)$. We begin by saying tha tnetworks can be modeled using graphs. We have Dijkstra's Algorithm which gives us shortest path (including weights). THis is relevant to transversing cities with wieghts on roads. Sometimes, fixed edge weights are too limiting. We let $x,y \in \TT$ be parameters. We have separated graphs, only one copy of each varaible associated to each edge (no two edges have ht esame varaible). We then have  a parametric shortest path. We let $\odot$ be a concatination of edges/paths, and $\oplus$ a comparison. We interpret $\infty$ as beign an edge that doesnt exist. We can consider the two parametric equations to be $4+y \leq 5+x$ to go down the left path, or $4+y \geq 5+x$ to go down the right path, so we have $\min\{4+y,5+x\} = (4\odot y) \oplus (5\odot x)$.

The key observation is hat the reions of optimal slutions are separated by tropical varieties. Now, what if our graph has mutliple tropical polyomials. Each tropical polynomial corresponds to a different destination node. We can instead considere $(A\rightarrow B,2)$, $(A\rightarrow B,y)$, $(B\rightarrow D,1)$, $(A\rightarrow D,2 )$, $(A \rightarrow C, x)$, $(A \rightarrow C,2)$, and $(C \rightarrow D,1)$. We then have polynomials with redunces when how we end at $D$. The Joswig Algorithm is reducing these polynomials. A selection of a solution is selecting a path for each destination node. In each region we select a path for that destination. 

We decompose our parameter space into cells where within each cell we have an optimatl solution. If we have a path through our x-y plane, we can segment that path into optiaml segments.


This decomposition of aprameter space came from three tropical polynomaisl. We can ask if we can describe the decompsoition as the tropical vareites of some polynomials. We cannot! We have a proof by pictorial contradiciton. We have verticies on the corners, which have oefficients $\infty$, but that would cause kinds and additional cells (new verticies).


You are guaranteed convex cells (for any two solutions in a cell, any solutionin between is a solution). The computaiton are also doable and efficient. In sumamry, the Joswig algorithm produces a decomp of param space into comvex cells via tropical varitieis. THe algorithm works on the order of 10's parameters, as the parametric shortest path problem is (probably) NP-complete (or hard).




\subsection{Natalie, Group THeory and Tropical Geometry}

We let $\xi$ be a nonzero real number, and denote $G_\xi$ by the group generated by $A = \begin{bmatrix}
    1 & 1 \\ 0 & 1
\end{bmatrix}$ and $X = \begin{bmatrix}
    1 & \xi \\ 0 & 1
\end{bmatrix}$. We then ask if $G_\xi$ is finitely presented. If $\xi$ is transcedental, the answer is no. However, let $\xi$ be the root of some irreducible polynomial $f(x)\in \ZZ[x]$. Then $G_\xi$ is finitely presented iff $\xi$ or $\frac{1}{\xi}$ is an algebraic integer over $\QQ$.

The statement that $\xi$ or $\frac{1}{\xi}$ is an algebraic integer over $\QQ$ implies that highest order term or lowest order term of $f(x)$ is $\pm 1$.


Consider the Laurent Polynomial ring $S = \ZZ[x_1^{\pm 1}, \dots, x_n^{\pm 1}]$. Units are monomials $\pm x_1^{a_1}\cdots x_n^{a_n}$.

\begin{defn}
The \emph{initial form} of $f(\overrightarrow{x})$ wrt $\overrightarrow{w}$ is a polynomial that records the terms that admit the minimym when tropicalized with $\overrightarrow{w} =val(\overrightarrow{x})$.
\end{defn}
\begin{ex}
    Consider $f(x) = 4x^3+3x+2$. Then $Trop(f) = \min(3val(x), val(x), 0)$. Now, let $w=1=val(x)$. Then $trop(f)=\min\{3,1,0\}$, the initial form $\in_{w=1}(f) = 2$, as $2$ is where the valaution of zero comes from. Now, let $w=0$. Then the minimum is hte same, so $in_{w=0}(f) = 4x^3+3x+2$, as everything hits the minimum.
\end{ex}


\begin{defn}
    Let $I$ be a proper ideal of $S$.  Then the \emph{initial ideal} $in_w(I)$ is hte ideal generated by all initial forms $in_w(f)$ where $f$ runs through $I$.
\end{defn}


\begin{defn}
The \emph{Tropical variety} of $I$ is $V_{trop_\ZZ}(I) = \{w \in \RR^n\; |\; in_w(I) \neq S\}$, or $V_{trop_\ZZ}(I) = \{ w \in \RR^n \; |\; \in_w(I) \; \text{does not contain a unit of S}\}$.
\end{defn}



\begin{ex}
Let $I$ be the principal ideal $I = \langle x_1 +x_2 + 3\rangle= \{sx_1+sx_2+3s\; |\; s\in S\}$. We want to find $V_{trop_\ZZ}(I)$. First we find $in_w(f)$ for possible values of $w\in \RR^2$. Note that $trop(f)= \min\{val(s) + \val(x_1), \val(s) + \val(x_2), \val(s)\}$. Now we let $w=(a,b)$, and so $val(x_1)=a$ and $\val(x_2)=b$. Now, if $a<0,b$, then $\in_w(f) = sx_1$ If we let $x_1$ have valuation $1$, we would get a unit, so it is not in the variety. If $0<a,b$, then $\in_w(f)=3s$, not a unit, so $w$ is in the variet.
\end{ex}


\begin{ex}
Let $\xi$ be a root of an irreducile polynomial $p(X) =a_2x^2+a_1x+a_0\in |ZZ[x^{\pm1}]$. Then $trop(p(x)) = \min\{2\val(x), \val(x), 0\}$. We have $\in_{w<0}=a_2x^2$, $in_{w =0} = a_2x^2+a_1x+a_0$, and $\in_{w>0} =a_0$.
\end{ex}




\subsection{Kylie, Cryptography}

Cryptography is a format for sending secret methods that can't be read by anyone besides the recipient. THe two major schools are public and private key cryptography. Within Public key cryptography, the re are two keys: one public and one private. The public key is used to encrypt information, while the private key is used to decrypt information. In Private Key cryptography, th ere is a single private key used for both encryption and decryption.

Public key is mroe secure, while private key is faster. Int eh discussed paper, a key exhcange protocol is used. This is a scure way to exchange a key using public methods between parties.


We will take the min convention. So $\begin{bmatrix}
    1 & 2 \\ \infty & -1
\end{bmatrix}\oplus \begin{bmatrix}
    0 & 3 \\ 4 & 1
\end{bmatrix}= \begin{bmatrix}
    0 & 2 \\ 4 & -1
\end{bmatrix}$, $\begin{bmatrix}
    1 & 2 \\ \infty & -1
\end{bmatrix}\otimes \begin{bmatrix}
    0 & 3 \\ 4 & 1
\end{bmatrix}= \begin{bmatrix}
    1 & 3 \\ 3 & 0
\end{bmatrix}$ $\otimes$ is matrix multiplication (and then using plus and min) is not commutative, and the tropical identtiy matrix is the matrix with zero on the diagonal, $|infty$ elsewhere. DIagoanl matricies have something on the diagonal.


We want our public key to be $n \times n$ matrices $A$ and $B$ with tropical entiries, witht eh requirement that $A \otimes B \neq B \otimes A$.


Alice has a secret: two tropical poynomials $p_1$ and $p_2$ with integer coefficients. Bob  does the same with $q_1$ and $q_2$. Alice computes $p_1(A) \otimes p_2(B)=P_{Alice}$, while Bob computes $q_1(A)\otimes q_2(B)=Q_{Bob}$. Then alice sends bob her $P_{ALice}$, Bob sends Alice $Q_{Bob}$. Alice knows $P_1(A)$ and $P_2(B)$. Then Alice computes $p_1(A) \otimes Q_{Bob}\otimes p_2(B)=K_{Alice}$, while Bob computes $q_1(A) \otimes P_{Alice} \otimes q_2(B)=K_{Bob}$. It turns outu that 

\begin{lemma}
    $K_{Alice}=K_{Bob}$
\end{lemma}

\begin{proof}
    Tropical matrices commute with themselves, so $q_1(A)\otimes p_1(A)= p_1(A) \otimes q_1(A)$.
\end{proof}



The key will then be the matrix $K_a=K_b=K$. We can then use these matrices to code and decode matrices via vector multiplication. PROBLEM! Tropical invertible matrices are rare.

\begin{lemma}
    The only tropical invertable matrices are permutations of a diagonal matrix.
\end{lemma}

\begin{proof}
    That permutations of a diaognla matrix is easy. TO see that invertibles must be permutation of a diaognal, we note that $A \otimes B = I$, we have $\oplus a_{ik} \otimes b_{kj}$ whcih is $0$ if $i =j$ and $\infty$ if $i \neq j$. Keep going to see issues.
\end{proof}



\subsection{Ian, A walk through the Tropics of eigenvalues}

Max convention
As an example, we have instructions for making mac and cheese
s1)  boil whatever
s2) cook pasta
s3) grate cheese
s4) make bachemel
s4.5) lightly cook tomatoes
s5) melt cheese
s6) combine. But there are steps that can happen at the same time. step 2 depends on step 1, but grating cheese and making bachemel can happen at the same time. But melting cheese requires rating cheese. Step six is a final step that everything else needs to happen first. This provides a flow of operations. We want to ask when is the soonest we can start a step.

We create a vector $\overrightarrow{x}$ where $x_j$ says when we can start task $j$. For example, step 6 can start some time after starting step $2$. We get a weighted graph where the weights say when can subsequent steps can occur. The max of all paths leading to step $6$ gives us $x_6$. Thus $x_i = \max\{A_{ij} +x_j\}$, where $A_{ij}$ is how long it takes to go from $j$ to $i$. This correpsodns to the matrix equation $A \odot x = x$



\begin{defn}
$\lambda \odot x = A \dot x$, where $\lambda$ is an eigenvalue with eigenvector $x$
\end{defn}


We let $A$ be a equare matrix, and we consider it as an adjacency matrix, where $-\infty$ denotes no edge. existing, and zero and negative edges are allowed, $A_{ij}$ is $j \rightarrow i$.


We can think of eigenvalues with respect to this graph.

\begin{lemma}
    If $A\odot x = x \odot \lambda$, there exists a normalized cycle of averaged wieght $\lambda$
\end{lemma}

A normalized cycle is the sum of weights devided by the number of edges
\begin{ex}
Let $A = \begin{bmatrix}
    -\infty & 2 & - \infty\\ -\infty &4 & - \infty \\ -\infty & - \infty &5
\end{bmatrix} \begin{bmatrix}
    0\\2\\-\infty
\end{bmatrix}= \begin{bmatrix}
    4 \\ 6 \\-\infty
\end{bmatrix}.$ Here $|lambda = 4$, and we have another igenvector $(-\infty, -\infty, 5)$ with eigenvalue $5$.
\end{ex}



\begin{lemma}
    Let $\sigma$ be any cycle in our graph. Then the maximum normalized cyucle $\max\limits_{\sigma} w(\sigma)/|\sigma| (\neq \pm \infty)$ is an eigenvalue for the associated matrix $A$.
\end{lemma}


\begin{theorem}
    If the graph  associated with $A$ is strongly connected then there is exactly one eigenvalue, $\lambda=\max\limits_{\sigma} w(\sigma)/|\sigma|$.
\end{theorem}


In terms of our computation, we have $A \odot x = \lambda \odot x$, which rewrites as $\max\limits_{j} (A_{ij} + x_j) = \lambda + c_i$, sow  get an inequality $a_{ij} x_j \leq \lambda + x_i$, so we have $A_{ij}+x_j-x_i\leq\lambda$.




\subsection{Seth, Tropical varieties of Higher codimension}

We begin by discussing classic algebraic geometry and the varieites there. In this setting

\begin{defn}
A \emph{hypersurface} is a vanishing set of a single polynmial $f \in K[x_1, \dots, x_n]$, denoted $V(f) = \{(x_1, \dots, x_n) \; |\; f(x_1, \dots, x_n)=0\}$
\end{defn}

\begin{ex}
    For $f(x+y+1)$, then $V(f)$ is the line $y=-x-1$.
\end{ex}



In higer cosien sion varieties are vanishing sets of ideals $V(I)$, whre $I=(f,g)$. Then we get $V(I)= \bigcap\limits-{f \in I} V(f)$. THis is nice because $I$ is finitely generated (Hilbert Basis THeorem), and $V(I) = \bigcap\limits_{f \in gen(I)} V(f)$. In the tropical setting, $V(I) = \bigcap V(f)$ for $f$ a genratr does not work.


To translate into tropical geometry, we let $f \in K[x_1^{\pm1}, \dots, x_n^{\pm1}]$, then$ trop(f):\RR^n \rightarrow \RR$, where $w=(w_1, \dots, w_n) \mapsto$ max of hte valuation of terms. A tropical hypersurface is the set of ties, $trop(v(f)) = \{w \in \RR^n \; |\; \text{trop(f) attains max at least twice}\}$.




\begin{ex}
Let $f=x++1 \in K[X^{\pm1}]$. Then $trop(f) :\RR^2\rightarrow \RR$, where $(w_1,w_2) \mapsto \max(w_1,w_2, 0)$. Then $trop(V(f))= _1=w_2\geq 0$, $w_1=0\geq w_2$, or $w_2 = 0 \geq w_1$, so we get a tropical curve. 
\end{ex}

By Kapranov's THeorem, $\trop(V(f)) = \overline{\val(V(f))}$.

\begin{defn}
A \emph{tropical variety} is $trop(V(I))= \bigcap\limits_{f \in I} \trop(V(f))$
\end{defn}


\begin{theorem}{Fundamental Theorem of tropical algebraic geometry}
    $trop(V(I))= \overline{\val(V(f))}$.
\end{theorem}

However, $\bigcap\limits_{f \in I} \trop(V(f)) \neq \bigcap\limits_{g \in gens(I) \in I} \trop(V(g))$. Similarly to Grobner basis, we define a basis whcih works to work In

\begin{defn}
    A \emph{Tropical basis} T for an ideal $I$ is any basis for which $\bigcap\limits_{f \in I} \trop(V(f)) = \bigcap\limits_{g \in T \in I} \trop(V(g))$
\end{defn}



So when we consider $f=x+y+1$ and $ g= x+2y$ (the line is $w_1=w_2$), we consider $\max(w_1,w_2)$. When we intersect the two we get a ray. This si not balnced. So $I=(x+y+1,x+2y)$ generates an ideal, but it is not a tropical basis. If we add in $y-1$ (coming from $x+2y-(x+y+1)$), we then get a tropical basis. This adds $h=y-1$ whcih is a horizontal line, which the additional intersection gives us a point.



%Nov 27 11/27

\subsection{Sam Line Bundles over CP1 and TP1}

\center{Manifold Structure on $\CC\PP^1$}

. We define $\CC\PP^1$ as $(\CC^2-\{(0,0)\})/ ( (X,Y) ~ (\lambda X,\lambda Y))$ for all $\lambda \in \CC$. We take open covers $U_1 = \{[X:Y] \in \CC\PP^1 \; |\; Y \neq 0\}$ and $U_2 = \{[X:Y] \in \CC\PP^1 \; |\; X \neq 0\}$, an maps $\varphi_1,\varphi_2:\CC\PP^2 \rightarrow \CC$ defined by $\varphi_1:[X:Y]\mapsto \frac{X}{Y}=: x$ and $\varphi_2:[X:Y]\mapsto \frac{Y}{X} =: y$. 


We have an equivalent contruction $\CC\PP^1:= ( (\CC,x) \sqcup (\CC,y))/(x=\frac{1}{y})$.



\center{line bundles}

\begin{defn}
    A \emph{vector bundle} of rank$ n$ over a space $X$ is itself a space $L$ together with a projection $\pi:L \rightarrow X$ such that
    \begin{enumerate}
        \item There exists an opern cover of $X$ $U:= \bigcup\limits_i U_i$ satisfying $\pi^{-1} (U_i) \simeq u_ix\CC^n$, and if $\phi_i$ is such a morphism, $\pi:(\pi^{-1}(U_i))\rightarrow U_i$ is equal to $p_1\circ \phi_i$, where $p_1$ is the projection onto the first coordinate.
        \item The map $\phi_j \circ \phi_i^{-1}:(U_i \cap U_j) \times \CC^n \rightarrow (u_i \cap u_j)\times \CC^n$ is a lienar map on $\CC$ and the identity on the intersection.
    \end{enumerate}
\end{defn}

\begin{defn}
    the $\phi_i$ are called \emph{trivializations}. 
\end{defn}

\begin{defn}
    A \emph{line bundle} is a rank 1 vector bundle
\end{defn}



\begin{defn}
    Let $\pi:L \rightarrow X$ be a vector bundle (line bundle). Let $U\subset X$ be open. A \emph{section of the bunle} over $U$ is a morphism $s:U \rightarrow L$ such that $\pi \circ s =id_U$.
\end{defn}

Sections choose a vector space element for each point of $X$.

\begin{ex}
    We can take $L= \CC\PP^1 \times \CC$ which is the trivial line bundle over compelx projective space. Sections have th eform $s:x \mapsto (x,f(x))$, where $f(x)$ is a compelx number associated to $x$. So sections of $\pi:L \rightarrow \PP^1$ define functions on complex projective space.
\end{ex}


\begin{ex}
    The Tautological Line bundle Is
    \begin{align*}
        L=\{([X:Y],(x,y)) \in \CC\PP^1 \times \CC^2 \; |\; (x,y) = (\lambda X, \lambda Y)\}
    \end{align*}

    So for each line in $\CC\PP^1$ we pick a representative point on that line.
\end{ex}




\center{Tropical Line Bundles}

We have a similar defintiion when defining line bundles over a tropical space $X$.


\begin{defn}
    A \emph{line bundle} over a tropical space $X$ is a space $L$ and a projection $\pi:L rightarrow X$ so that
    \begin{enumerate}
        \item There exists an open cover $U = \bigcup \limits_{i} U_i$, and
        \item there exist $\phi_i:\pi^{-1}(U_i) \rightarrow U_i \times \TT$, which are tropical isomorphisms, and such that $p_1 \circ \phi_i = \pi$ on $\pi^{-1}(U_i)$.
    \end{enumerate}
\end{defn}


  These maps induce automorphisms on the tropical semiring $\TT$. We define $\tilde{\varphi_{ij}} := \phi_j \circ \phi_i^{-1}: (U_i \cap U_j) \times \TT \rightarrow (u_i \cap u_j)\times \TT$. In particular,$ \tilde{\phi_{ij}}$ is an automrophism on each $\{x\} \times \TT$.

  \begin{note}
    $Aut(\TT) \simeq \RR$.
  \end{note}



  This lets us define the map $\phi_{ij} : U _i \cap U_j \rightarrow \RR$ via
  $\tilde{\phi_{ij}}(x,t) = (x,\phi_{ij}(x) \odot t)$.


  \begin{ex}
    Take $\TT\PP^1 := (\TT,x) \sqcup (\TT,y)/(x=-y)=[-\infty,\infty]$. We take an open cover $U_1:= [-\infty, \infty)$ and $U_2:=(-\infty, \infty]$. We can take the trivial line bundle $L:=\TT\PP^1 \times \TT$. Then the morphisms are $\phi_1=\phi_2=0$, and so the induced map we get is $\phi_{12}:(\infty,\infty) \rightarrow \RR$, whcih is the constant $0$ function.
  \end{ex}







\end{document}