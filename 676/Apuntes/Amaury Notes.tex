\documentclass[11pt]{article}

\usepackage[utf8]{inputenc}	% Para caracteres en español
\usepackage{amsmath,amsthm,amsfonts,amssymb,amscd,mathtools}
%-------------------------------------

%-------------------------------------

\usepackage{fullpage}
\usepackage{lastpage}
\usepackage{enumitem}
\usepackage{fancyhdr}
\usepackage{mathrsfs}
\usepackage{wrapfig}
\usepackage{setspace}
\usepackage{calc}
\usepackage{multicol}
\usepackage{cancel}
\usepackage[retainorgcmds]{IEEEtrantools}

\usepackage{empheq}
\usepackage{framed}


%--------------------
%--------------------
%Theorem
%\theoremstyle{definition}
%\newtheorem{defn}{Definition}
\newtheorem{reg}{Rule}
\newtheorem{exer}{Exercise}
%\newtheorem{ex}{Example}
\newtheorem{note}{Note}
\newtheorem{remark}{Remark}
\newtheorem{axiom}{Axiom}[section] 
%\newtheorem{theorem}{Theorem}[section] 
\newtheorem{proposition}{Proposition}[section] 
\newtheorem{conjecture}{Conjecture}[section]
%\newtheorem{lemma}{Lemma}[section] 
%\newtheorem{corollary}{Corollary}[section] 

%The following is to color corollary, theorems, etc.
\usepackage{xcolor}
\usepackage{amsthm}
\usepackage{framed}
%\theoremstyle{plain}% default
\newtheorem{prototheorem}{Theorem}[section]

\newenvironment{theorem}
   {\begin{prototheorem}}
   {\end{prototheorem}}

\newtheorem{protolemma}[prototheorem]{Lemma}
\newenvironment{lemma}
   {\begin{protolemma}}
   {\end{protolemma}}

\newtheorem{protocorollary}[prototheorem]{Corollary}
\newenvironment{corollary}
   {\begin{protocorollary}}
   {\end{protocorollary}}

\theoremstyle{definition}
\newtheorem{protonotation}{Notation}[section]
\newenvironment{notation}
   {\begin{protonotation}}
   {\end{protonotation}}

\newtheorem{protoexample}{Example}[section]
\newenvironment{ex}
   {\begin{protoexample}}
   {\end{protoexample}}

\newtheorem{protodefinition}{Definition}[section]
\newenvironment{define}
   {\begin{protodefinition}}
   {\end{protodefinition}}


%--------------------
%Commands
\def\a{\alpha}
\def\l{\lambda}
\def\r{\rho}
\def\ZZ{{\mathbb Z}}
\def\QQ{{\mathbb Q}}
\def\RR{{\mathbb R}}
\def\FF{{\mathbb F}}
\def\KK{{\mathbb K}}
\def\CC{{\mathbb C}}
\def\NN{{\mathbb N}}
\def\XX{{\mathbb X}}
\def\YY{{\mathbb Y}}
\def\TT{{\mathbb T}}
\def\PP{{\mathbb P}}
\def\AA{{\mathbb A}}
\def\subgroup{{ \; \leqslant \;}}
\def\normalsubgroup{{ \; \trianglelefteq \;}}
\allowdisplaybreaks
\def\B{\mathcal{B}}
\def\C{\mathcal{C}}
\def\A{\mathcal{A}}
\def\D{\mathcal{D}}
\def\a{\alpha}
\def\w{\varpi}
\def\al{\alpha}
\def\fg{\mathfrak{g}}
\def\fh{\mathfrak{h}}
\def\la{\lambda}
\def\be{\beta}
\def\P{\mathscr{P}}
\def\so{\mathfrak{so}}
\def\sp{\mathfrak{sp}}

\DeclareMathOperator{\val}{val}
\DeclareMathOperator{\trop}{trop}




%-------------------------

%For displaystyle everywhere
%\everymath{\displaystyle}
%------------------------------

\begin{document}
%\setcounter{section}{8}
\title{Combinatorics Notes}


%\tableofcontents


\thispagestyle{empty}


\begin{center}
{\LARGE \bf Tropical Geometry}\\
{\large MATH 676}\\
Fall 2023
\end{center}

These notes arose from Tropical Geometry with Dr. Renzo Cavaleri during the Fall of 2023 at CSU. They come from his lectures.



\newpage



%September 1st written notes



%9/15

%Sept 18 9/18





%oct 2, 10/2




%Oct 16 10-16




%Oct 18, 10-18



%Oct 20 10-20


%Oct 27 10/27



    


    %Oct 30 10/30

 



%Nov 1, 11/1





%Nov 6 11/6



\section{Student Presentations}

\subsection{Joel, Gfan }

Gfan is a software package for computing Grobner fans and tropical varieties. Overview of installation. Gfan runs on the kernel.Gfan consists of various programs. Gfan can be run by creating scripts to pass to teh Gfan library. Gfan natively supports $\QQ$ and finite fields. Integers can be used with some work. For example, we can begin with an ideal, such as $\QQ[x,y,z]$ and the ideal $\{xxy-z,yyz-x,zzx-y\}$. THe library ca compute a Grobner basis for any such ideal.

There is a relationship between Grobner basis and tropical varieities. A Grobner basis can give a Grobner cone. A collection of these Grobner cones construct a Grobner fan (distinct from tropical fan). A tropical variet is a union of Grobner cones, and is thus a subfan of a Grobner fan. Gfan does not have the most advanced visualization techniques. Gfan can visualize  figure files (only with xfig). We can imagine the Grobner fan exists in $\RR^3$ in the positive quadrant. The Grobner cone is 2-D cones. THe unit vectors in $\RR^3$ form a triangle, and hte imge is hte intersection of these cones with that triangle. Given an ideal and a permutation in a permutation group, it can compute a cone in an orbit of tha tpermutaiton group.

If we want to consider troical varieties, we can give Gfan a principle ideal $(x+y+z+w)$ within $\QQ[x,y,z,w]$. If we wante dto ask more of the tropical verities, we would run tropical interection to compute the rays, the cones generated by teh rays. This is a tropical hyperplane in $4$-dim.

Genericalyl, Gfan does trivially valued fields. It is possible to add nontrivial valuations, butt hat takes longer to compute.


%Nov 10, 11/10
\subsection{Kristina, Tropical Geometry of Deep Neural Networks}

There is an equivalence between feedforward neural networks with ReLU activiation and tropical rational functions.

(Disucssion of the cat vs dog pictures for neural networks) Orientation, identifiable features, replicate human brain processing. Neural networks have hidden layers. Each layer is a matrix product (weight assignment) and vector addition (bias). We then introduce cost functions to compare results, back propogation (gradient descent) to adjust weights.

We have activation functions. THe first is the sigmoid $\sigma(Ax+b)= \frac{1}{1+e^{-(Ax+b)}}$ we also have the rectified linear unit ReLU $\sigma(Ax+b)= \max(0,Ax+b)$. ReLU makes data more sparse, and looks like tropical geometry. First we make assumptions about our $L$-layer network. We assume that the weight matricies $A^(1), \dots, A^{(l)}$ are integer values, the biasvectors $b^{(1)}, \dots, b^{(l)}$ are real valued, and the activation functions take the form  $\sigma(Ax+b)= \max(0,Ax+b)$.


To build our equivalence, we first consider the output from teh first layer in the neural netwrok $\nu(x) = \max\{Ax+b,t\}$, where $t \in (\RR\cup \infty)^l$. So we can rewrite $\max\{Ax+b,t\} = \max\{A_+x+b, A_-x + t\} - A_-x$. SO every coordinat eof a one layer network si the difference of two tropical polynomials. FOr networks with uktuple layers, aplky this decomposition recursively.

\begin{theorem}
    A feedworard neural network under the assumptions is a function $\nu: \RR^d \rightarrow \RR^p$ whose coordinates are tropical rational fuinctions of the input, i.e. $\nu(x) F(x) \oslash G(x) = F(x) -G(x)$, where $F$ and $G$ are tropical rational functions.
\end{theorem}


We can use this equivalence to consider decision boundaries of a neural network. The input space of a neural network is partitionined into disjoint subsets, where each subset determines a final deciison (what is a dog, what is a cat). SO our input space might be a tropical curve, and the 2-cells give deicison boundaries. We can bound the number o linear reions of a NN byy bounding vertiices in the dual subdivision of the Newton polyon. THis number of linear reions measures complexity of a neural network. THese dont make better bounds, but it shows that tropical geometries can do the same work.

\subsection{Jacob, The Joswig Algorithm}


We let $\TT:=(\RR\cup\{\infty\}, \min, +)$. We begin by saying tha tnetworks can be modeled using graphs. We have Dijkstra's Algorithm which gives us shortest path (including weights). THis is relevant to transversing cities with wieghts on roads. Sometimes, fixed edge weights are too limiting. We let $x,y \in \TT$ be parameters. We have separated graphs, only one copy of each varaible associated to each edge (no two edges have ht esame varaible). We then have  a parametric shortest path. We let $\odot$ be a concatination of edges/paths, and $\oplus$ a comparison. We interpret $\infty$ as beign an edge that doesnt exist. We can consider the two parametric equations to be $4+y \leq 5+x$ to go down the left path, or $4+y \geq 5+x$ to go down the right path, so we have $\min\{4+y,5+x\} = (4\odot y) \oplus (5\odot x)$.

The key observation is hat the reions of optimal slutions are separated by tropical varieties. Now, what if our graph has mutliple tropical polyomials. Each tropical polynomial corresponds to a different destination node. We can instead considere $(A\rightarrow B,2)$, $(A\rightarrow B,y)$, $(B\rightarrow D,1)$, $(A\rightarrow D,2 )$, $(A \rightarrow C, x)$, $(A \rightarrow C,2)$, and $(C \rightarrow D,1)$. We then have polynomials with redunces when how we end at $D$. The Joswig Algorithm is reducing these polynomials. A selection of a solution is selecting a path for each destination node. In each region we select a path for that destination. 

We decompose our parameter space into cells where within each cell we have an optimatl solution. If we have a path through our x-y plane, we can segment that path into optiaml segments.


This decomposition of aprameter space came from three tropical polynomaisl. We can ask if we can describe the decompsoition as the tropical vareites of some polynomials. We cannot! We have a proof by pictorial contradiciton. We have verticies on the corners, which have oefficients $\infty$, but that would cause kinds and additional cells (new verticies).


You are guaranteed convex cells (for any two solutions in a cell, any solutionin between is a solution). The computaiton are also doable and efficient. In sumamry, the Joswig algorithm produces a decomp of param space into comvex cells via tropical varitieis. THe algorithm works on the order of 10's parameters, as the parametric shortest path problem is (probably) NP-complete (or hard).




\subsection{Natalie, Group THeory and Tropical Geometry}

We let $\xi$ be a nonzero real number, and denote $G_\xi$ by the group generated by $A = \begin{bmatrix}
    1 & 1 \\ 0 & 1
\end{bmatrix}$ and $X = \begin{bmatrix}
    1 & \xi \\ 0 & 1
\end{bmatrix}$. We then ask if $G_\xi$ is finitely presented. If $\xi$ is transcedental, the answer is no. However, let $\xi$ be the root of some irreducible polynomial $f(x)\in \ZZ[x]$. Then $G_\xi$ is finitely presented iff $\xi$ or $\frac{1}{\xi}$ is an algebraic integer over $\QQ$.

The statement that $\xi$ or $\frac{1}{\xi}$ is an algebraic integer over $\QQ$ implies that highest order term or lowest order term of $f(x)$ is $\pm 1$.


Consider the Laurent Polynomial ring $S = \ZZ[x_1^{\pm 1}, \dots, x_n^{\pm 1}]$. Units are monomials $\pm x_1^{a_1}\cdots x_n^{a_n}$.

\begin{define}
The \emph{initial form} of $f(\overrightarrow{x})$ wrt $\overrightarrow{w}$ is a polynomial that records the terms that admit the minimym when tropicalized with $\overrightarrow{w} =val(\overrightarrow{x})$.
\end{define}
\begin{ex}
    Consider $f(x) = 4x^3+3x+2$. Then $Trop(f) = \min(3val(x), val(x), 0)$. Now, let $w=1=val(x)$. Then $trop(f)=\min\{3,1,0\}$, the initial form $\in_{w=1}(f) = 2$, as $2$ is where the valaution of zero comes from. Now, let $w=0$. Then the minimum is hte same, so $in_{w=0}(f) = 4x^3+3x+2$, as everything hits the minimum.
\end{ex}


\begin{define}
    Let $I$ be a proper ideal of $S$.  Then the \emph{initial ideal} $in_w(I)$ is hte ideal generated by all initial forms $in_w(f)$ where $f$ runs through $I$.
\end{define}


\begin{define}
The \emph{Tropical variety} of $I$ is $V_{trop_\ZZ}(I) = \{w \in \RR^n\; |\; in_w(I) \neq S\}$, or $V_{trop_\ZZ}(I) = \{ w \in \RR^n \; |\; \in_w(I) \; \text{does not contain a unit of S}\}$.
\end{define}



\begin{ex}
Let $I$ be the principal ideal $I = \langle x_1 +x_2 + 3\rangle= \{sx_1+sx_2+3s\; |\; s\in S\}$. We want to find $V_{trop_\ZZ}(I)$. First we find $in_w(f)$ for possible values of $w\in \RR^2$. Note that $trop(f)= \min\{val(s) + \val(x_1), \val(s) + \val(x_2), \val(s)\}$. Now we let $w=(a,b)$, and so $val(x_1)=a$ and $\val(x_2)=b$. Now, if $a<0,b$, then $\in_w(f) = sx_1$ If we let $x_1$ have valuation $1$, we would get a unit, so it is not in the variety. If $0<a,b$, then $\in_w(f)=3s$, not a unit, so $w$ is in the variet.
\end{ex}


\begin{ex}
Let $\xi$ be a root of an irreducile polynomial $p(X) =a_2x^2+a_1x+a_0\in |ZZ[x^{\pm1}]$. Then $trop(p(x)) = \min\{2\val(x), \val(x), 0\}$. We have $\in_{w<0}=a_2x^2$, $in_{w =0} = a_2x^2+a_1x+a_0$, and $\in_{w>0} =a_0$.
\end{ex}




\subsection{Kylie, Cryptography}

Cryptography is a format for sending secret methods that can't be read by anyone besides the recipient. THe two major schools are public and private key cryptography. Within Public key cryptography, the re are two keys: one public and one private. The public key is used to encrypt information, while the private key is used to decrypt information. In Private Key cryptography, th ere is a single private key used for both encryption and decryption.

Public key is mroe secure, while private key is faster. Int eh discussed paper, a key exhcange protocol is used. This is a scure way to exchange a key using public methods between parties.


We will take the min convention. So $\begin{bmatrix}
    1 & 2 \\ \infty & -1
\end{bmatrix}\oplus \begin{bmatrix}
    0 & 3 \\ 4 & 1
\end{bmatrix}= \begin{bmatrix}
    0 & 2 \\ 4 & -1
\end{bmatrix}$, $\begin{bmatrix}
    1 & 2 \\ \infty & -1
\end{bmatrix}\otimes \begin{bmatrix}
    0 & 3 \\ 4 & 1
\end{bmatrix}= \begin{bmatrix}
    1 & 3 \\ 3 & 0
\end{bmatrix}$ $\otimes$ is matrix multiplication (and then using plus and min) is not commutative, and the tropical identtiy matrix is the matrix with zero on the diagonal, $|infty$ elsewhere. DIagoanl matricies have something on the diagonal.


We want our public key to be $n \times n$ matrices $A$ and $B$ with tropical entiries, witht eh requirement that $A \otimes B \neq B \otimes A$.


Alice has a secret: two tropical poynomials $p_1$ and $p_2$ with integer coefficients. Bob  does the same with $q_1$ and $q_2$. Alice computes $p_1(A) \otimes p_2(B)=P_{Alice}$, while Bob computes $q_1(A)\otimes q_2(B)=Q_{Bob}$. Then alice sends bob her $P_{ALice}$, Bob sends Alice $Q_{Bob}$. Alice knows $P_1(A)$ and $P_2(B)$. Then Alice computes $p_1(A) \otimes Q_{Bob}\otimes p_2(B)=K_{Alice}$, while Bob computes $q_1(A) \otimes P_{Alice} \otimes q_2(B)=K_{Bob}$. It turns outu that 

\begin{lemma}
    $K_{Alice}=K_{Bob}$
\end{lemma}

\begin{proof}
    Tropical matrices commute with themselves, so $q_1(A)\otimes p_1(A)= p_1(A) \otimes q_1(A)$.
\end{proof}



The key will then be the matrix $K_a=K_b=K$. We can then use these matrices to code and decode matrices via vector multiplication. PROBLEM! Tropical invertible matrices are rare.

\begin{lemma}
    The only tropical invertable matrices are permutations of a diagonal matrix.
\end{lemma}

\begin{proof}
    That permutations of a diaognla matrix is easy. TO see that invertibles must be permutation of a diaognal, we note that $A \otimes B = I$, we have $\oplus a_{ik} \otimes b_{kj}$ whcih is $0$ if $i =j$ and $\infty$ if $i \neq j$. Keep going to see issues.
\end{proof}



\subsection{Ian, A walk through the Tropics of eigenvalues}

Max convention
As an example, we have instructions for making mac and cheese
s1)  boil whatever
s2) cook pasta
s3) grate cheese
s4) make bachemel
s4.5) lightly cook tomatoes
s5) melt cheese
s6) combine. But there are steps that can happen at the same time. step 2 depends on step 1, but grating cheese and making bachemel can happen at the same time. But melting cheese requires rating cheese. Step six is a final step that everything else needs to happen first. This provides a flow of operations. We want to ask when is the soonest we can start a step.

We create a vector $\overrightarrow{x}$ where $x_j$ says when we can start task $j$. For example, step 6 can start some time after starting step $2$. We get a weighted graph where the weights say when can subsequent steps can occur. The max of all paths leading to step $6$ gives us $x_6$. Thus $x_i = \max\{A_{ij} +x_j\}$, where $A_{ij}$ is how long it takes to go from $j$ to $i$. This correpsodns to the matrix equation $A \odot x = x$



\begin{define}
$\lambda \odot x = A \dot x$, where $\lambda$ is an eigenvalue with eigenvector $x$
\end{define}


We let $A$ be a equare matrix, and we consider it as an adjacency matrix, where $-\infty$ denotes no edge. existing, and zero and negative edges are allowed, $A_{ij}$ is $j \rightarrow i$.


We can think of eigenvalues with respect to this graph.

\begin{lemma}
    If $A\odot x = x \odot \lambda$, there exists a normalized cycle of averaged wieght $\lambda$
\end{lemma}

A normalized cycle is the sum of weights devided by the number of edges
\begin{ex}
Let $A = \begin{bmatrix}
    -\infty & 2 & - \infty\\ -\infty &4 & - \infty \\ -\infty & - \infty &5
\end{bmatrix} \begin{bmatrix}
    0\\2\\-\infty
\end{bmatrix}= \begin{bmatrix}
    4 \\ 6 \\-\infty
\end{bmatrix}.$ Here $|lambda = 4$, and we have another igenvector $(-\infty, -\infty, 5)$ with eigenvalue $5$.
\end{ex}



\begin{lemma}
    Let $\sigma$ be any cycle in our graph. Then the maximum normalized cyucle $\max\limits_{\sigma} w(\sigma)/|\sigma| (\neq \pm \infty)$ is an eigenvalue for the associated matrix $A$.
\end{lemma}


\begin{theorem}
    If the graph  associated with $A$ is strongly connected then there is exactly one eigenvalue, $\lambda=\max\limits_{\sigma} w(\sigma)/|\sigma|$.
\end{theorem}


In terms of our computation, we have $A \odot x = \lambda \odot x$, which rewrites as $\max\limits_{j} (A_{ij} + x_j) = \lambda + c_i$, sow  get an inequality $a_{ij} x_j \leq \lambda + x_i$, so we have $A_{ij}+x_j-x_i\leq\lambda$.




\subsection{Seth, Tropical varieties of Higher codimension}

We begin by discussing classic algebraic geometry and the varieites there. In this setting

\begin{define}
A \emph{hypersurface} is a vanishing set of a single polynmial $f \in K[x_1, \dots, x_n]$, denoted $V(f) = \{(x_1, \dots, x_n) \; |\; f(x_1, \dots, x_n)=0\}$
\end{define}

\begin{ex}
    For $f(x+y+1)$, then $V(f)$ is the line $y=-x-1$.
\end{ex}



In higer cosien sion varieties are vanishing sets of ideals $V(I)$, whre $I=(f,g)$. Then we get $V(I)= \bigcap\limits-{f \in I} V(f)$. THis is nice because $I$ is finitely generated (Hilbert Basis THeorem), and $V(I) = \bigcap\limits_{f \in gen(I)} V(f)$. In the tropical setting, $V(I) = \bigcap V(f)$ for $f$ a genratr does not work.


To translate into tropical geometry, we let $f \in K[x_1^{\pm1}, \dots, x_n^{\pm1}]$, then$ trop(f):\RR^n \rightarrow \RR$, where $w=(w_1, \dots, w_n) \mapsto$ max of hte valuation of terms. A tropical hypersurface is the set of ties, $trop(v(f)) = \{w \in \RR^n \; |\; \text{trop(f) attains max at least twice}\}$.




\begin{ex}
Let $f=x++1 \in K[X^{\pm1}]$. Then $trop(f) :\RR^2\rightarrow \RR$, where $(w_1,w_2) \mapsto \max(w_1,w_2, 0)$. Then $trop(V(f))= _1=w_2\geq 0$, $w_1=0\geq w_2$, or $w_2 = 0 \geq w_1$, so we get a tropical curve. 
\end{ex}

By Kapranov's THeorem, $\trop(V(f)) = \overline{\val(V(f))}$.

\begin{define}
A \emph{tropical variety} is $trop(V(I))= \bigcap\limits_{f \in I} \trop(V(f))$
\end{define}


\begin{theorem}{Fundamental Theorem of tropical algebraic geometry}
    $trop(V(I))= \overline{\val(V(f))}$.
\end{theorem}

However, $\bigcap\limits_{f \in I} \trop(V(f)) \neq \bigcap\limits_{g \in gens(I) \in I} \trop(V(g))$. Similarly to Grobner basis, we define a basis whcih works to work In

\begin{define}
    A \emph{Tropical basis} T for an ideal $I$ is any basis for which $\bigcap\limits_{f \in I} \trop(V(f)) = \bigcap\limits_{g \in T \in I} \trop(V(g))$
\end{define}



So when we consider $f=x+y+1$ and $ g= x+2y$ (the line is $w_1=w_2$), we consider $\max(w_1,w_2)$. When we intersect the two we get a ray. This si not balnced. So $I=(x+y+1,x+2y)$ generates an ideal, but it is not a tropical basis. If we add in $y-1$ (coming from $x+2y-(x+y+1)$), we then get a tropical basis. This adds $h=y-1$ whcih is a horizontal line, which the additional intersection gives us a point.



%Nov 27 11/27

\subsection{Sam Line Bundles over CP1 and TP1}

\center{Manifold Structure on $\CC\PP^1$}

. We define $\CC\PP^1$ as $(\CC^2-\{(0,0)\})/ ( (X,Y) ~ (\lambda X,\lambda Y))$ for all $\lambda \in \CC$. We take open covers $U_1 = \{[X:Y] \in \CC\PP^1 \; |\; Y \neq 0\}$ and $U_2 = \{[X:Y] \in \CC\PP^1 \; |\; X \neq 0\}$, an maps $\varphi_1,\varphi_2:\CC\PP^2 \rightarrow \CC$ defined by $\varphi_1:[X:Y]\mapsto \frac{X}{Y}=: x$ and $\varphi_2:[X:Y]\mapsto \frac{Y}{X} =: y$. 


We have an equivalent contruction $\CC\PP^1:= ( (\CC,x) \sqcup (\CC,y))/(x=\frac{1}{y})$.



\center{line bundles}

\begin{define}
    A \emph{vector bundle} of rank$ n$ over a space $X$ is itself a space $L$ together with a projection $\pi:L \rightarrow X$ such that
    \begin{enumerate}
        \item There exists an opern cover of $X$ $U:= \bigcup\limits_i U_i$ satisfying $\pi^{-1} (U_i) \simeq u_ix\CC^n$, and if $\phi_i$ is such a morphism, $\pi:(\pi^{-1}(U_i))\rightarrow U_i$ is equal to $p_1\circ \phi_i$, where $p_1$ is the projection onto the first coordinate.
        \item The map $\phi_j \circ \phi_i^{-1}:(U_i \cap U_j) \times \CC^n \rightarrow (u_i \cap u_j)\times \CC^n$ is a lienar map on $\CC$ and the identity on the intersection.
    \end{enumerate}
\end{define}

\begin{define}
    the $\phi_i$ are called \emph{trivializations}. 
\end{define}

\begin{define}
    A \emph{line bundle} is a rank 1 vector bundle
\end{define}



\begin{define}
    Let $\pi:L \rightarrow X$ be a vector bundle (line bundle). Let $U\subset X$ be open. A \emph{section of the bunle} over $U$ is a morphism $s:U \rightarrow L$ such that $\pi \circ s =id_U$.
\end{define}

Sections choose a vector space element for each point of $X$.

\begin{ex}
    We can take $L= \CC\PP^1 \times \CC$ which is the trivial line bundle over compelx projective space. Sections have th eform $s:x \mapsto (x,f(x))$, where $f(x)$ is a compelx number associated to $x$. So sections of $\pi:L \rightarrow \PP^1$ define functions on complex projective space.
\end{ex}


\begin{ex}
    The Tautological Line bundle Is
    \begin{align*}
        L=\{([X:Y],(x,y)) \in \CC\PP^1 \times \CC^2 \; |\; (x,y) = (\lambda X, \lambda Y)\}
    \end{align*}

    So for each line in $\CC\PP^1$ we pick a representative point on that line.
\end{ex}




\center{Tropical Line Bundles}

We have a similar defintiion when defining line bundles over a tropical space $X$.


\begin{define}
    A \emph{line bundle} over a tropical space $X$ is a space $L$ and a projection $\pi:L rightarrow X$ so that
    \begin{enumerate}
        \item There exists an open cover $U = \bigcup \limits_{i} U_i$, and
        \item there exist $\phi_i:\pi^{-1}(U_i) \rightarrow U_i \times \TT$, which are tropical isomorphisms, and such that $p_1 \circ \phi_i = \pi$ on $\pi^{-1}(U_i)$.
    \end{enumerate}
\end{define}


  These maps induce automorphisms on the tropical semiring $\TT$. We define $\tilde{\varphi_{ij}} := \phi_j \circ \phi_i^{-1}: (U_i \cap U_j) \times \TT \rightarrow (u_i \cap u_j)\times \TT$. In particular,$ \tilde{\phi_{ij}}$ is an automrophism on each $\{x\} \times \TT$.

  \begin{note}
    $Aut(\TT) \simeq \RR$.
  \end{note}



  This lets us define the map $\phi_{ij} : U _i \cap U_j \rightarrow \RR$ via
  $\tilde{\phi_{ij}}(x,t) = (x,\phi_{ij}(x) \odot t)$.


  \begin{ex}
    Take $\TT\PP^1 := (\TT,x) \sqcup (\TT,y)/(x=-y)=[-\infty,\infty]$. We take an open cover $U_1:= [-\infty, \infty)$ and $U_2:=(-\infty, \infty]$. We can take the trivial line bundle $L:=\TT\PP^1 \times \TT$. Then the morphisms are $\phi_1=\phi_2=0$, and so the induced map we get is $\phi_{12}:(\infty,\infty) \rightarrow \RR$, whcih is the constant $0$ function.
  \end{ex}



  \subsection{Ross, Counting Curves Like I Count Stacks}

\begin{ex}
    How many circles are tangent to three circles in a plane? THe answer is three
\end{ex}

\begin{ex}
    How many lines go through two points? It's one.
\end{ex}

\begin{ex}
    How many conics go through five points? It's one.
\end{ex}

We want to tropicalize thee questions so we can more efficiently answer these questions.

We typically modulize and consider $f:\PP^1 \rightarrow \PP^2$. Each condition brings down the dime nsion of the solution set, until we get to zero diemsion, which is a finite numebr of points. THe general dimensionality of our solution is 3d+g-1

We have a related notion in tropical geometry. GIven two points, we can find a single tropical line passing through the two, and given five points we have a single tropical conic goign through the five points (the conic is degree two). 


TO be careful, we say that a tropical curve has degree $d$ if there are exactly d rays going in each direction (for whcih rays go).


We have lengths of a graph , and we define $\gamma:\Gamma \rightarrow \RR^2$ via $\{w_l,P_l\}$, where we have eights of each edge $l$, and some $P$.


We can tropicalize a mdouli space to get the tropical modli space.

THere are many determinants, and multiplicities. $N(d,g)$ is the number of curves satisdying $n$ point conditions, $g$ is genus

Take some complex polygon, project down some vector, to make sure the points in the polygon are separated nicely. We call teh vector we project along $\lambda$ we call a path $\lambda$ incrasing if a path along vertices goes along an increasing set of points when considering the projections. 

Now ony do we have $N(d,g)=N^{trop}(d,g)$, we have $N(d,g)=N^{path}(d,g)$, the number of $\lambda$ increasing paths counting multiplicity.


We now give a recursive defintiion. $\mu(\gamma)=\mu_+(\gamma)\mu_-(\gamma)$, where $\mu_\pm(\gamma) = 2 *Area(T) \mu_\pm(\gamma_\pm')\pm \mu(\gamma_\pm'')$. There are two kinds of paths. $\gamma_+$ looks for when the path turns left, the $\gamma'$ is forming triangle, $\gamma''$ is making a parallolgram, $\gamma_-$ is when you turn right. This makes subdivisions, base case defintiions are the side legnths, wehre $\mu(\delta_\pm)=1$.




\subsection{Sandra, Intro to Berkovich spaces}


WHat if insstead of alg closed cahr 0 field, we had a non archmedian field. We want to build an analog of complex geometry over a non archemdian field

\begin{define}
    A \emph{valued field} is a pari $(K,|\cdot|:k \rightarrow \RR_{\geq0})$, where 
    \begin{enumerate}
        \item $|a|= 0 \iff a=0$,
        \item $|a-b| \leq |a| + |b|$, and 
        \item $|ab|=|a|*|b|$
    \end{enumerate}
\end{define}


\begin{ex}
    $\CC$ with the infinity norm.
\end{ex}


A valued field might be complete. For notation $(\hat{k}, |\cdot|)$ denotes the completion of $K$ with respec tot eh norm $|\cdot|$. If we take $K=\QQ$, and the normal Euclidean norm, $\hat{\QQ} = \RR$. If instead we had the p-adic norm, $\hat{\QQ} = \QQ_p$.

\begin{define}
    $K$ is \emph{complete} if $k=\hat{k}$
\end{define}


We now considered Laurent series. Take $K =\tilde{k}((T)) = \{ f = \sum\limits_{i=-\infty}^\infty a_iT^i \; |\; a_i \in k, \;(i<<0, \; a_i=0) \}$. Then the order of the zero of $f$ is defined to be $\min\{i \; |\; ia_i \neq 0\}$.  Then$ |f|_* :=e^{ord_0(f)}$. We then claim that $(K,|\cdot|_*)$.



The Archemdian principle is $|a+b| > max\{|a|, |b|\}$. $(k, |\cdot|)$ is non-Archmedian if $|a+b| \leq max\{|a|, |b|\}$ for all $a,b\in k$. 


\begin{ex}
    $(\QQ_,|\cdot|_p)$ for all primes $p$.
\end{ex}



\begin{note}
    If $(K, |\cdot|)$ is NA, then the following are true:

    \begin{enumerate}
        \item If $|a|>|b|$, then $|a+b|=|a|$.
        \item All open balls over $k$ are clopen.
        \item $k$ is totally disconected (each connected compnent is a singleton)
    \end{enumerate}
\end{note}

These proeprties make building a manifold theory complicated. 

When we build a manifold, we have certain desired proeprties of a space.

Our first step is to defien a space. Some essential ingredients:
\begin{enumerate}
    \item We need a set $X$
    \item We need a topology on $X$
    \item We need a structure sheaf $O_X$ such that if we pick a section$ f\in O_X(U)$ (U open) is analytic.
\end{enumerate}



%Dec 1, 12/1

\subsection{Jake, Tropical INtersection Theory}


Consider a spcae $X$. We have the Chow ring $A^*(X)$, which is a graded ring (via the codiemnsion). elements are subvarieties of $X$. We have the translation that $+$ is the union, and $\times$ is the intesection. If we consider $X=\PP^2$, then $A^0(\PP^2)=\ZZ$, and so on. THe collection of them is $\ZZ \oplus \ZZ \oplus \ZZ \oplus \cdots$. 



If we cosnider two polynomials $F:ax+by=0$ and $G:cx+dy=0$, we can consider $\lambda F+\mu G=0$. We get that teh equations come from $[0:1]$ and $[1:0]$. This gives polynomials $x^2+y^2=-$, which comes from $(ax+by)(cx+dy)=0$.



Now, if $X = \AA^2$, then $F:V(f(x,y))$. Then $\lambda F + \mu(1) =0$ gives $A^*(\AA^2) = \ZZ$.



\subsection{Conegor, Minkowski Weights}



THis describes classes (pts, lines, concis, etc.) by givinf their intersection number siwth all toric invarient subvarieties (boundaries. We define $X = B|_p \PP^2$. Looking at the fans of $X$, we can ask what happens when $L$, line, intersects, a vertex of $X$.)



\subsection{Daniel, Tropical Covers \& Tropical Riemann Hurwitz formula}

We are trying to emmulate Riemann surfaces.


\begin{define}
    A \emph{Riemann surface} is a complex anaytic maniforld of dimension 1.
\end{define}



\begin{note}
    A space is a compact riemann surface if and only if it is  asmooth projective curve over $\CC$
\end{note}

Compact Riemann surfaces are classified by their genus. We say $X$ has genus $n$ if and only if $x \hookrightarrow \RR^3$ is an $n-$holed torus. Furthermore, holomorphic maps btween compact Riemann SUrfaces $f:X \rightarrow Y$ are branch coverings, i.e. almost everywhere $d$-to-one, where cutting out finitely many problematic points and their pre-images, we get $d$-to-one. Where the map is not $d$-to-one we call that remification. i.e. we say $z^2$ is ramified over $0$ and $\infty$


We have flobal information, which is the genu of $X$ and $Y$, and local information, which is the ramification profile. This information is mro combinatorial. THe Riemann-Hurwitz formula is how we relate the two. If we have $f:X \rightarrow Y$ is nonconstant, holomorphic, and degree $d$, we get the equality $2g(X)-2=d(2g(Y)-2) + \sum\limits_{x \in X} (e_x -1)$, where $d$ si the degree of the map, $2g(X)-2$ is the Euler characterisitc of the space, and $e_x$ is the rammification index of the the point i.e looking at the pre-images and how many branches intersect. This is a realizability condition.


\begin{define}
    An abstract Tropical ucurve is a connected metric graph $\Gamma$ with unbounded rays called ends and a genus function $g:V \rightarrow \NN$ whcih assings a genus to every vertex of our graph
\end{define}


We hav eht egenus fuction so that we can think of the curves as being dual to deformations of Riemann surfaces.
    We can think of the single vertex of degree two as being dual to the genus 2 torus. The deformation splits the genus two surface into two surfaces conncted at a single point, so we get two vertiexs of deree 1 with an edge connecting them. Vertices of degree 1 correspond to the ``irreducible'' surfaces of genus one (the torus), and the vertice connecting them correpsond to the surfaces intersecting at a point.
    \begin{define}
        The local degree at a vertex is the sum of the surrounding weights
    \end{define}
\begin{define}
    A tropical cover $\Pi: Gamma_1 \rightarrow \Gamma_2$ is a surjective map satisfying
    \begin{enumerate}
        \item localy integer affine linear ($\Pi$ scales length by an integer factor). This factor is called the weight of the edge $w(e)$.
        \item Harmonic/Balancing condition
    \end{enumerate}
\end{define}

The first condition allos us to label edges by weights. FOr the seocnd condition, returning to earlier in these notes. If we have a tropical curve, we label with weights so that we get a ``balancing'' of the weights on either sides of verticices.



\begin{define}
    The local Riemann-Hurwitz condition states that for $v \in \Gamma_1$, $v \mapsto v'$ with local deree $d_v$, we have $2g(v) - 2 \geq d_v(2g(v')-2) + \sum\limits_{e \in E, \; v \in e} (w(e) -1)$.
\end{define}


When w conside ra cover of a tropical curve $\Pi$, we get a duality too a cover of Riemann surfaces $\hat{\Pi}$. So satisfying local condition at verticies allows us to satisgy global condition of covers on surfaces.



\subsection{Eve, (Tropical) Hurwitz numbers}


If we have two maps $f:X \rightarrow Y$ and $g: \tilde{X} \rightarrow Y$ between Riemann surfaces we say $f$ and $g$ ar eisomorphic if there exists $\varphi:X \rightarrow \tilde{X}$ such that $f= \varphi \circ g$.. We can allow $X=\tilde{X}$.



Let $Y$ by a connected compact RIemann surface with genus $g$. Fix $b_1, \dots, b_h \in Y$, and $\lambda_1, \dots, \lambda_h$ be a partition of $d \geq 1$. THen $f:X \rightarrow Y$ is a Hurwizst cover if 

\begin{enumerate}
    \item $f$ is holomorphic, 
    \item $X$ is a connected compact genus $h$,
    \item Branch locus of $f$ is $\{b_1, \dots, b_h\}$,
    \item $\lambda_i$ is the ramification produle of $b_i$.
\end{enumerate}



From here, we can say the Hurwitz Number is $H_{d:h \rightarrow g}(\lambda_1, \dots, \lambda_n):= \sum\limits-{[f]} \frac{1}{|Aut(f)|}$.



Now, a tropical cover of $\RR \cup \{\pm\infty\}$. We need to send one valent vertices (ends) to $\pm \infty$. We can count these tropical covers via more discrete data. We fix $\mu$ and $\nu$ as partitions of $d \in \ZZ_{\geq 1}$. If our degree of h tetropical curve is $4$, we can get the left end has parititon $4$, our right end has partition $(2,2)$, the genus of our raph is $1$. We let $r:=2g-2+l(\mu) + l(\nu) >0$ (lenght of partition $\mu$ and $|\nu$), and we then fix $p_1, \dots, p_r \in \RR$. Then the tropical Hurwitz number is $H^{trop}_{d: g \rightarrow 0}(\mu,\nu): \sum\limits_{f} m(f)$, where $M(f)$ is the multiplicity of $f$, defined as $\frac{1}{|Aut(f)|}\prod_{\text{e a bounded edge}}w(e)$


The automorphisms of our graph are either swap two vertices or swap to edges, so the automorphism group has size 4.

\begin{ex}
    \begin{align*}
        H^{trop}_{4:1\rightarrow 0} ((4),(2,2))=\left( \frac{1}{4} 2*2*4\right) +6+3+1 =14
    \end{align*}
    
\end{ex}

We get four total graph coverings.



We have the following correspondence: $H^{trop}_{d: g \rightarrow 0}(\mu,\nu)=H_{d: g \rightarrow 0}(\mu,\nu)$








\end{document}