\documentclass[12pt]{memoir}

\def\nsemestre {II}
\def\nterm {Fall}
\def\nyear {2023}
\def\nprofesor {Renzo Cavalieri}
\def\nsigla {MATH676}
\def\nsiglahead {Tropical Geometry}
\def\nlang {ENG}
%\def\darktheme{}
\input{../../headerVarillyDiff}
\usepackage[enableskew]{youngtab}

\DeclareMathOperator{\val}{val}
\DeclareMathOperator{\Trop}{Trop}
\newcommand{\diamondplus}{\mathbin{\rotatebox[origin=c]{45}{$\boxtimes$}}}

\begin{document}
%\clearpage
\maketitle
%\thispagestyle{empty}
{\small 
\setlength{\parindent}{0em}
\setlength{\parskip}{1em}

This is a topics course on this stuff

\subsubsection*{Requirements}
Knowledge on stuff\par 

\textbf{TO DO:}
\begin{itemize}
    \item Write info on course description and requirements.
    \item Polish info from day 1
    \item Polish last part of day 2
    \item Continue adding info from Renzo's digital notes
\end{itemize}
}
\newpage
\tableofcontents
%\begin{multicols}{2}
\chapter{Combinatorial Shadow of Algebraic Geometry}

\section{Day 1|20230821}

Think of an algorithm where the input is an algebraic variety and the output is a combinatorial object, a piecewise linear object.

\begin{Ex}
    Consider as an input a line in the plane. Say $V(x+y-1)$, then an output would be a tropical line. If we remain in the plane and consider a higher degree polynomial, say an elliptic curve, as an output we obtain a tropical cubic.\par 
    Leaving the plane behind and thinking of abstract nodal curves, we can think of a sphere attached to a torus which is attached to a genus 2 torus, then the corresponding object is what we call the \term{dual graph}.
\end{Ex}

Right now we do not know the specific algorithm, but we can observe that the outputs are \emph{more simple} than the inputs. So the important question is:
\begin{significant}
What algebraic information does the simplified object remember? How do we extract the information the object remembers? And once we know how to work with this objects, can we return to algebraic geometry from any kind of these objects?
\end{significant}

Observe that the number of ends which go to infinity corresponds with the degree. 

\section{Day 2|20230823}
%%Based on 676-Intro
\subsection{Algebraic Geometry on $\bT$}
Let us talk about ways to get into tropical geometry. We will first define the tropical semifield which the base set over which we will do algebraic geometry.

\begin{Def}
    The \term{tropical semifield} is the set $(\bR\cup\set{-\infty})$ equipped with tropical addition and multiplication:
    $$
    \begin{cases}
        x\oplus y=\max(x,y)\\
        x\odot y=x+y
    \end{cases}
    $$
\end{Def}

With this set we can make multivariable polynomials 
$$p(\un x)\:\left(\bR\cup\set{-\infty}\right)^n\to\bR\cup\set{-\infty}$$
which gives rise to their \emph{tropicalization}, a piecewise linear function $\Trop(p)\:\bR^n\to\bR$.

\begin{Ex}
    Consider the polynomial 
    $$p(x,y)=x\oplus y\oplus 0,$$
    its tropicalization is $\Trop(p)(x,y)=\max(x,y,0)$ which indeed is a piecewise linear function from $\bR^2$ to $\bR$.
    \begin{figure}[h!]
        \centering
        \subcaptionbox{$x\oplus y\oplus 0$\label{fig:2.1-LinearTropicalPolynomial}}{\includegraphics[width=0.3\textwidth]{figs/fig2.1-LinearTropicalPolynomial.pdf}}\quad
        \subcaptionbox{Projection onto $xy$-plane\label{fig:2.2-ProjectionLinearTropicalPolynomial}}{\includegraphics[width=0.25\textwidth]{figs/fig2.2-LinearTropicalPolynomialProjected.pdf}}\quad
        \subcaptionbox{Corner locus\label{fig:2.3-CornerLocus}}{\includegraphics[width=0.25\textwidth]{figs/fig2.3-CornerLocus.pdf}}
        %\caption{This is the caption.}
        \label{fig:2.1-and-2.2-and-2.3}
    \end{figure}
    %https://mathematica.stackexchange.com/questions/169777/listplot3d-with-contours-projected-onto-the-xy-plane
    Observe that the surface is not smooth where the planes meet, this is what we will call the \emph{corner locus} or \emph{tropical hypersurface}.
    %In two variables we have a PICTURE. The polynomial $x\oplus y\oplus 0$ is actually $\max(x,y,0)$. This picture is actually the projection of the corner locus. In 3D we can visualize this better.
\end{Ex}

\begin{Def}
    The \term{tropical hypersurface} $V(\Trop(p))$ is the codimension 1 locus in $\bR^n$ where the function is non-linear (corner locus).
\end{Def}

\begin{Ex}
If we consider higher degree tropical polynomials, they will become linear in the usual sense. Consider 
$$p(x)=3x^2=3\odot x\odot x=3+x+x=3+2x$$
which is indeed linear.
\end{Ex}

\subsection{Valued fields}

\begin{Def}
The field of \term{Puiseux series} or rational functions over $\bC$ is $\bC(t)$ where the elements are of the form 
$$f(t)=\sum_{i=k_0}^\infty a_it^{i/n}.$$
The lower bound $k_0$ could be negative and the exponents, are rational with bounded denominators. 
\end{Def}

Consider the valuation 
$$\val_0\:\bC(t)\to\bR\cup\set{\infty},\begin{cases}
    0\mapsto \infty\\
    f\mapsto\text{order of vanishing at }0.
\end{cases}$$
This order of vanishing is the value $\al$ such that $f/t^\al$ approaches a finite non-zero value. The corresponding coefficient in the series expansion of $f$ for this value is called the valuation coefficient.

\begin{Ex}
    What happens to the order of vanishing when you add two functions? Consider $f=t^2,\ g=t^3$, then $f+g=t^2+t^3$ which has order of vanishing $2$. Observe that $2=\min(2,3)$.
\end{Ex}
In general what happens is that
$$\val_0(f_1+f_2)\geq\min(\val_0 f_1,\val_0 f_2),\word{and}\val_0(f_1f_2)=\val_0(f_1)+\val_0(f_2).$$

We can do algebraic geometry over this field! Let $K$ be the field of rational functions, if $p(\un x)\in K\bonj{\un x}$ then we consider the algebraic variety
$$X=V(p)=\set{\vec{x}\:p(\vec{x})=0}\subseteq K^n.$$
Taking the image through the $n$-fold valuation, we will obtain a set in $\left(\bR\cup\set{\infty}\right)^n$. The tropicalization of $X$ is the image via this map:
\begin{figure}[h!]
    
\centering
\tikzset{every picture/.style={line width=0.75pt}} %set default line width to 0.75pt        

\begin{tikzpicture}[x=0.75pt,y=0.75pt,yscale=-1,xscale=1]
%uncomment if require: \path (0,300); %set diagram left start at 0, and has height of 300

%Straight Lines [id:da2454102694181276] 
\draw    (235,63) -- (299,63) ;
\draw [shift={(301,63)}, rotate = 180] [color={rgb, 255:red, 0; green, 0; blue, 0 }  ][line width=0.75]    (10.93,-3.29) .. controls (6.95,-1.4) and (3.31,-0.3) .. (0,0) .. controls (3.31,0.3) and (6.95,1.4) .. (10.93,3.29)   ;
%Straight Lines [id:da5779705315012776] 
\draw    (241,113) -- (298,113) ;
\draw [shift={(300,113)}, rotate = 180] [color={rgb, 255:red, 0; green, 0; blue, 0 }  ][line width=0.75]    (10.93,-3.29) .. controls (6.95,-1.4) and (3.31,-0.3) .. (0,0) .. controls (3.31,0.3) and (6.95,1.4) .. (10.93,3.29)   ;
\draw [shift={(241,113)}, rotate = 180] [color={rgb, 255:red, 0; green, 0; blue, 0 }  ][line width=0.75]    (0,5.59) -- (0,-5.59)   ;

% Text Node
\draw (208,53.4) node [anchor=north west][inner sep=0.75pt]    {$K^{n}$};
% Text Node
\draw (303,53.4) node [anchor=north west][inner sep=0.75pt]    {$(\bR\cup\set{\infty})^{n}$};
% Text Node
\draw (202,103.4) node [anchor=north west][inner sep=0.75pt]    {$V( p)$};
% Text Node
\draw (304,99.4) node [anchor=north west][inner sep=0.75pt]    {$\overline{\operatorname{val}_0( V( p))}$};
% Text Node
\draw (172,153.4) node [anchor=north west][inner sep=0.75pt]    {$\{\vec{x} :p(\vec{x}) =0\}$};
% Text Node
\draw (302,153.4) node [anchor=north west][inner sep=0.75pt]    {$\operatorname{Trop}( V( p))$};
% Text Node
\draw (210.4,98) node [anchor=north west][inner sep=0.75pt]  [rotate=-270]  {$\subseteq $};
% Text Node
\draw (328.4,98) node [anchor=north west][inner sep=0.75pt]  [rotate=-270]  {$\subseteq $};
% Text Node
\draw (229.6,127) node [anchor=north west][inner sep=0.75pt]  [rotate=-90]  {$=$};
% Text Node
\draw (346.6,127) node [anchor=north west][inner sep=0.75pt]  [rotate=-90]  {$=$};
% Text Node
\draw (172,53.4) node [anchor=north west][inner sep=0.75pt]    {$\operatorname{val}_0\:$};


\end{tikzpicture}

\end{figure}\\
and here $\Trop(V(p))$ is the tropical hypersurface for $p$.
\begin{Ex}
    Consider the polynomial in $K\bonj{x,y}$ 
    $$p(x,y)=tx+y+t^2,$$
    then the variety is $X=\set{(x,y)\: tx+y+t^2=0}$ which we can solve to $y=-tx-t^2$.\par 
    If we choose $x=0$ then $y$ becomes $-t^2$. Now we take the valuation of $(0,-t^2)$ and so $(\infty,2)\in\Trop(X)$.
\end{Ex}

\subsection{Amoebas}

Let us return to the usual stage and consider $p\in\bC[\un x]$ which defines an algebraic variety $X=V(p)\subseteq\bC^n$. Now consider the map which sends every coordinate's modulus to its logarithm in base $t$: 
$$\bC^n\to\left(\bR\cup\set{-\infty}\right)^n,\quad (z_1,\dots,z_n)\to(\log_t|z_1|,\dots,\log_t|z_n|).$$


The image of $X$ under this map, $\log_t(X)$, is the $t$-amoeba of $X$. If we take the limit as $t\to\infty$ then we get the \emph{spine} of the amoeba. 

\begin{Ex}
    When $p(x,y)=x+y-1$ then we can describe $V(p)$ via the parametrization $(x,1-x)$. So the corresponding $t$-amoeba in the real case is 
    $$\set{(\log_t|x|,\log_t|1-x|)\: x\in\bR}$$
    and we ordinarily take the limit, we see that the functions converge to zero point-by-point. But the set is actually approaching the spine!
    \begin{figure}[h!]
        \centering
        \subcaptionbox{$X=V(x+y-1)$\label{fig:2.4-Variety}}{\includegraphics[width=0.25\textwidth]{figs/fig2.4-V(x+y-1).pdf}}\quad
        \subcaptionbox{$2$-amoeba of $X$\label{fig:2.5-2Amoeba}}{\includegraphics[width=0.25\textwidth]{figs/fig2.5-2amoeba.pdf}}\quad
        \subcaptionbox{Sequence of amoebas as $t\to\infty$\label{fig:2.6-ApproxAmoebas}}{\includegraphics[width=0.25\textwidth]{figs/fig2.6-ApproximatingAmoebas.pdf}}
        %\caption{This is the caption.}
        \label{fig:2.4-thru-2.6}
    \end{figure}
\end{Ex}

Observe that the spine approaches the tropical hypersurface associated to $p$. In other words we have that the tropical hypersurface is $\lim_{t\to\infty}\log_t(V(p))$.

\subsection{Degenerations}
We may parametrize any algebraic variety with a time variable, then converting the information to a graph, edges code the information about how fast the node forms related to the length.\par
Consider a family of \red{of what, what is this family of?! Stuff? Curve in P1xP1 which eventually becomes P2?}
\begin{figure}[h!]
    \centering
    \includegraphics[width=0.5\textwidth]{figs/fig1.1.png}
\end{figure}

It is too early to understand this point of view. We will set everything up to get to it.\par 
In general, the big idea will be to explore and understand these perspectives in the case of plane curves. We want to show how they are equivalent and then recover classical algebraic geometry results in terms of tropical geometry.

\section{Day 3|20230825}

Recall that the last time we discussed the classical (25 to 30 years old) ways to get to tropical geoemtry. We now would like to answer the question
\begin{significant}
    Where do tropical numbers come from?
\end{significant}
So let us begin with an applications problem and see how the tropical numbers arise from the context of the problem.

\subsection{Tropical Arithmetics}%%%Based on TropicalNumbers.pdf

\subsubsection{Minimizing Tolls}

Consider a set of cities connected by a network of toll-ways:
\begin{figure}[h!]
    \centering
    \includegraphics[width=0.5\textwidth]{figs/fig1.2.png}
\end{figure}
If we only care about minimizing toll expenses when traveling, what would be the cheapest way to go from one given city to another? Let us record the information as an incidence matrix:
$$M_{ij}=\text{price of going from city }i\text{ to city }j\text{ in at most one trip}\To M=\threebythree{0}{\infty}{2}{x}{0}{y}{\infty}{1}{0}$$
In this matrix, the rows determine the outbound city, while the columns are the destination. Each entry records the cost of a toll and tolls are considered to be infinite when the road does not exist. We can also think of $M$ as recording the cheapest toll to go from one city to another with at most one move.\par 
How would we compute the best strategy of going from city $i$ to $j$ in \emph{at most two trips}? If for example we want to find trips from $A$ to $B$ in two steps then we have three choices:
$$AAB,\quad ABB,\quad ACB.$$
The costs of each one are 
$$(0,\infty),\quad (\infty,0),\quad (2,1)$$
so we sum them and take the minimum. That will be the optimal route from $A$ to $B$ in two steps. In fact, if we relate this to the entries of the matrix $M$, we could use $M^2$. However we must redefine our basic operations as follows: 
$$+=\min,\quad\.=+$$
So we have the identification 
$$(1,2)\text{ entry of }M^2=\sum_{j=1}^{3}M_{1k}M_{k2}=\min(M_{11}+M_{12},M_{12}+M_{22},M_{13}+M_{32}).$$
In general:
\begin{align*}
    \threebythree{0}{\infty}{2}{x}{0}{y}{\infty}{1}{0}^2&=\threebythree{\min\threebyone{0+0}{\infty+x}{2+\infty}}{\min\threebyone{0+\infty}{\infty+0}{2+1}}{\min\threebyone{0+2}{\infty+y}{2+0}}{\min\threebyone{x+0}{0+x}{y+\infty}}{\min\threebyone{x+\infty}{0+0}{y+1}}{\min\threebyone{x+2}{0+y}{y+0}}{\min\threebyone{\infty+0}{1+x}{0+\infty}}{\min\threebyone{\infty+\infty}{1+0}{0+1}}{\min\threebyone{\infty+2}{1+y}{0+0}}\\
    &=\threebythree{0}{3}{2}{x}{\min(0,y+1)}{\min(x+2,y)}{1+x}{1}{\min(0,1+y)}.
\end{align*}
Observe that $1+y$ can be the minimum in the diagonal when we allow \emph{negative tolls}.
\begin{Rmk}
If we disallow negative tolls, the products $M^n$ eventually stabilize to a matrix whose entries record the cheapest way to get from one city to another in $n$ steps.
\end{Rmk}
This gives us an intuition that minimization problems correspond to linear algebra problems over $(\bT,+,\.)$ which is precisely $(\bR\cup\set{\infty},\min,+)$.

\subsubsection{Forgetting phases}

Recall that any complex number can be written as $z=re^{i\te}$ where $r\geq 0$. Consider the map $T_t\:\bC\to\set{-\infty}\cup\bR,\quad z\mapsto\log_t(r)$.
\begin{figure}[h!]
    \centering
    \includegraphics[width=0.5\textwidth]{figs/fig1.3.png}
\end{figure}
This map is surjective, and this we can see by checking it is right-invertible. Observe that:
$$
\left\lbrace
\begin{aligned}
    &T_t^{-1}(x)=\set{t^xe^{i\te}}\subseteq\bC,\word{for}x\in\bR,\\
    &T_t^{-1}(-\infty)=0.
\end{aligned}
\right.
$$
With this in hand, we wish to define an exotic addition and multiplication on $\set{-\infty}\cup\bR$ using $T_t$. We will dequantize!\par 
We begin with \textbf{hyper-addition}, the output will be a subset of $\set{-\infty}$ so it's not a binary operation by itself. 
$$x\diamondplus_t y\:= T_t(T_t^{-1}(x)+T_t^{-1}(y))=\bonj{\log_t(|t^x-t^y|),\log_t(t^x+t^y)}.$$
This is an interval in $\set{-\infty}\cup\bR$, in order to make $\diamondplus_t$ into an operation we take a limit:
\begin{figure}[h!] 
    \centering
\begin{tikzpicture}[x=0.75pt,y=0.75pt,yscale=-1,xscale=1]
%uncomment if require: \path (0,300); %set diagram left start at 0, and has height of 300

%Straight Lines [id:da5156276968518897] 
\draw    (85,62.6) -- (142,62.99) ;
\draw [shift={(144,63)}, rotate = 180.39] [color={rgb, 255:red, 0; green, 0; blue, 0 }  ][line width=0.75]    (10.93,-3.29) .. controls (6.95,-1.4) and (3.31,-0.3) .. (0,0) .. controls (3.31,0.3) and (6.95,1.4) .. (10.93,3.29)   ;

%Straight Lines [id:da8224691621679702] 
\draw    (96,133) -- (144,133.38) ;
\draw [shift={(146,133.4)}, rotate = 180.46] [color={rgb, 255:red, 0; green, 0; blue, 0 }  ][line width=0.75]    (10.93,-3.29) .. controls (6.95,-1.4) and (3.31,-0.3) .. (0,0) .. controls (3.31,0.3) and (6.95,1.4) .. (10.93,3.29)   ;
%Straight Lines [id:da27001319663870027] 
\draw    (57,77) -- (57,118) ;
\draw [shift={(57,120)}, rotate = 270] [color={rgb, 255:red, 0; green, 0; blue, 0 }  ][line width=0.75]    (10.93,-3.29) .. controls (6.95,-1.4) and (3.31,-0.3) .. (0,0) .. controls (3.31,0.3) and (6.95,1.4) .. (10.93,3.29)   ;
%Straight Lines [id:da8909494911629017] 
\draw    (195,83) -- (195,120) ;
\draw [shift={(195,122)}, rotate = 270] [color={rgb, 255:red, 0; green, 0; blue, 0 }  ][line width=0.75]    (10.93,-3.29) .. controls (6.95,-1.4) and (3.31,-0.3) .. (0,0) .. controls (3.31,0.3) and (6.95,1.4) .. (10.93,3.29)   ;

% Text Node
\draw (34,53.4) node [anchor=north west][inner sep=0.75pt]    {$x\diamondplus_t y$};
% Text Node
\draw (34,123.4) node [anchor=north west][inner sep=0.75pt]    {$x\ +_{t} \ y$};
% Text Node
\draw (152,53.4) node [anchor=north west][inner sep=0.75pt]    {$x\diamondplus y=\lim _{t\rightarrow \infty } x\diamondplus_t y$};
% Text Node
\draw (152,123.4) node [anchor=north west][inner sep=0.75pt]    {$x+y=\max( x,y)$};
% Text Node
\draw (60,98.5) node [anchor=west] [inner sep=0.75pt]  [font=\scriptsize]  {$\max$};
% Text Node
\draw (114.5,59.4) node [anchor=south] [inner sep=0.75pt]  [font=\scriptsize]  {$\displaystyle\lim _{t\rightarrow \infty }$};
% Text Node
\draw (121,129.8) node [anchor=south] [inner sep=0.75pt]  [font=\scriptsize]  {$\displaystyle\lim _{t\rightarrow \infty }$};
% Text Node
\draw (197,102.5) node [anchor=west] [inner sep=0.75pt]  [font=\scriptsize]  {$\max$};


\end{tikzpicture}

\end{figure}

\begin{Rmk}
Note that $\diamondplus$ is still a hyperoperation. Its output is not a singleton \emph{only} when a dding a number to itself:
$$x\diamondplus y=\begin{cases}
    \max(x,y),\quad x\neq y\\
    \bonj{-\infty,x},\quad x=y
\end{cases}$$
\end{Rmk}

Formally this process, taking a limit of a family of operations, is known as \emph{dequantization}.\par

In the case of multiplication, things go a lot smoother when defining it:

$$x\.y =T_t\bonj{T^{-1}(x)\.T^{-1}(y)}=\log_t\bonj{(t^xe^{i\te})(t^ye^{i\vf})}=\log_t\left(t^{x+y}e^{i(\te+\vf)}\right)$$

Separating the logarithm we get $(x+y)+\log(e^{i(\te+\vf)})/\log(t)$, then letting $t$ grow without bound we see that the operation converges to $x+y$. 

\begin{Ex}
    Let us consider a small example like summing $2$ and $4$. Observe that 
    $$4\diamondplus_t 2=T_t(T_t^{-1}(4)+T_t^{-1}(2))=T_t\left(t^4e^{i\te}+t^2e^{i\vf}\right)$$
    and the term on the inside can be simplified to $t^{4}(e^{i\te}+t^{-2}e^{i\vf})$. $T_t$ takes that expression to
    $$4+\log_t(e^{i\te}+t^{-2}e^{i\vf})=4+\frac{\log(e^{i\te}+t^{-2}e^{i\vf})}{\log t}.$$
    What happens if we take the limit as $t\to\infty$? We get an independent from $t$ result! 
    The term on the right vanishes and we are left with $4=\max(4,2)$. So it got a tad bit better, but it's still a hyperoperation!
\end{Ex}

\begin{Ej}
Check how the definition of $+$ and $\.$ extend to the \emph{number} $-\infty$.
\end{Ej}

\begin{ptcb}
The point of this exercise is to operate $-\infty$ with finite numbers and itself.\par
For a finite $x$ we will find $x+(-\infty)$. This is the limit of the previous hyperoperation:
$$x\diamondplus_t(-\infty)=T_t(T_t^{-1}(x)+T_t^{-1}(-\infty))=T_t(T_t^{-1}(x)+0)=T_t(T_t^{-1}(x))=x.$$
If we let $t$ grow, the result doesn't change and so this goes according to $\max(x,-\infty)=x$.\par 
On the other hand when taking the product:
$$x\.(-\infty)=T_t\bonj{T^{-1}(x)\.T^{-1}(-\infty)}=T_t\bonj{T^{-1}(x)\.0}=T_t(0)=\log_t(0)\to-\infty$$
which is also similar to the notion of $x+(-\infty)=-\infty$.\par 
We can now proceed to operate $-\infty$ with itself:
$$(-\infty)\diamondplus_t(-\infty)=T_t(0)=\log_t(0)=-\infty=\max(-\infty,-\infty),$$
and when taking the product:
$$(-\infty)\.(-\infty)=T_t(0)\log_t(0)=-\infty=(-\infty)+(-\infty)$$
where the last sum is a sum in the usual sense.
\end{ptcb}

So, summarizing this process:
\begin{itemize}
    \item We forgot about the phase of the complex numbers and only looked at them radially. 
    \item The modulus of these numbers was scaled logarithmically.
    \item Finally we took the limit of these operations and obtained the desired (somewhat) result.
\end{itemize}
This is known as Maslov\footnote{Viktor Pavlovich Maslov (1930615-20230803)} dequantization and with this we can see $(\bT,+,\.)$ as $(\set{-\infty}\cup\bR,\max,+)$. Also, we will abbreviate $\lim_{t\to\infty}T_t$ with $T_{t\to\infty}$

\section{Interim}
%%valued fields

\subsection{Tropical Polynomials and Roots}

An univariate,tropical, (Laurent) monomial is equivalent to an affine linear function with integer coefficients.

\begin{Ex}
    We have for example:
    $$5x^2\otto 5+2x,\quad 2x^{-3}\otto 2-3x\ (\text{Laurent}).$$
\end{Ex}

An univariate tropical (Laurent) polynomial is a finite sum of monomials which give rise to a \emph{convex}, continuous, piecewise affine, linear function with integer slopes.

\begin{Ex}
    Consider the function $-5x^2+(-2)x^{-3}+0$ which corresponds to 
    $$\max(-5+2x,-2-3x,0).$$
    If we graph this functions we obtain
    \begin{figure}[h!]
        \centering
        \includegraphics[width=0.5\textwidth]{figs/fig3.1RenzoNotes3.png}
        %\caption{This is the caption.}
        \label{fig:3.1-ConvPLFunc}
    \end{figure}
    Observe that this function is indeed convex, and fulfills all of the previous properties from before.
\end{Ex}

A small measure of care should be taken because any convex-PL-(etc) function corresponds to a tropical polynomial, but there are multiple tropical polynomials which map to the same function.

\begin{Ex}
    Consider the functions 
    $$p_1=x+\frac{1}{x}+0,\quad p_2=x+\frac1x-2.$$
    When converting we get 
    $$\max(x,-x,0),\quad\max(x,-x,-2)$$
    which produce $|x|$ in both cases.
    \begin{figure}[h!]
        \centering
        \includegraphics[width=0.5\textwidth]{figs/fig3.2RenzoNotes3.png}
        \caption{Failure of injectivity as both functions map to $|x|$ with $y=0$ and $y=-2$ shown.}
        \label{fig:3.2-InjectivityFailure}
    \end{figure}
\end{Ex}

To talk about the roots, we will start with a purely combinatorial definition.

\begin{Def}
    Given a polynomial $p\in\bT[x]$ of degree $d$ we say the following:
    \begin{itemize}
        \item $-\infty$ is a root of $p$ if the slope of the piecewise linear function is non-zero for $x\ll 0$.
        \item $x_0\in\bR$ is a root of $p$ if $p'(x_0)$ is undefined.
    \end{itemize}
    We say that the \term{multiplicity} is the change of slopes across the root. And of $p'(x)$, $x\ll 0$ for when the root is $-\infty$.
\end{Def}

\begin{Ex}
    Consider the polynomial $x^2+1\.x^1+0=\max(2x,x,0)$.
    \begin{figure}[h!]
        \centering
        \includegraphics[width=0.5\textwidth]{figs/fig3.3SimpleFiniteRootsTropicalPolynomial.png}
        %\caption{}
        \label{fig:3.3-SimpleFiniteRoots}
    \end{figure}
    We can see that there are changes in slope at $x_1=-1$ and $x_2=1$. The number of roots coincides with the degree of the polynomial as in the usual sense.
\end{Ex}

\begin{Ex}
    Let's remove the zero, recall zero isn't the additive identity, so the polynomial we have is $x^2+1\.x^1=\max(2x,x)$.
    \begin{figure}[h!]
        \centering
        \includegraphics[width=0.5\textwidth]{figs/fig3.4SimpleRootsTropicalPolynomial.png}
        %\caption{}
        \label{fig:3.4-OneFiniteRootOneInfiniteRoot}
    \end{figure}
    Now one of the roots is still $x=1$, but remember that if the slope is non-zero when $x\ll 0$, then $-\infty$ is a root of $p$. This is the case here because the slope is $1$ as $x\to-\infty$. Once again there's two roots $x_1=-\infty$ and $x_2=1$.
\end{Ex}

\begin{Ex}
    Let us change a sign in a coefficient, take $x^2-1\.x^1+0$. But what is tropical subtraction? It's not that, let's convert this slowly into what it's supposed to be:
    $$x^2-1\.x^1+0=(x\.x)+(-1)\.x+0=(2x)+(x+(-1))+0=\max(2x.x-1,0).$$
    \begin{figure}[h!]
        \centering
        \includegraphics[width=0.5\textwidth]{figs/fig3.5DoubleRootTropicalPolynomial1.png}
        %\caption{}
        \label{fig:3.5-DoubleRoot1}
    \end{figure}
    Observe that because the line $y=x-1$ is below our graphs, it doesn't interfere with the calculation of zeroes. So the only place where there occurs a change in sign is $x=0$. This is our unique root with multiplicity $2$.
\end{Ex}

\begin{Ex}
    In a similar fashion, $x^2+0$ also has a double root at $x=0$.
    \begin{figure}[h!]
        \centering
        \includegraphics[width=0.5\textwidth]{figs/fig3.6DoubleRootTropicalPolynomial2.png}
        %\caption{}
        \label{fig:3.6-DoubleRoot6}
    \end{figure}
    There is only one change in slope once again at $x=0$.
\end{Ex}

Questions arise:
\begin{significant}
    Which functions have only one simple zero at $-\infty$? What would a function with an order 2 zero at $-\infty$ look like?
\end{significant}

\begin{Ej}
    Do the following:
    \begin{itemize}
        \item[(5)] Is it possible for a function to have only a simple zero at $-\infty$? Provide an example of function with one simple zero at $-\infty$ or prove that such function cannot exist. 
        \item[(5)] Do functions with zeroes at $-\infty$ have infinite order at such zero or is it arbitrarily high? If a function has a finite order zero at $-\infty$ provide an example of one with a double zero at $-\infty$. Else, prove that such functions have infinite order at that zero.
    \end{itemize}
\end{Ej}

\section{Day 4|20230828}

We have seen where our ideas come from. Certain kinds of minimization problems give rise to our tropical numbers. Also by expressing complex numbers in a logarithmic scale without phase then when inducing a sum we actually get a hypersum. The way we converted into an operation is by taking a limit. Then the algebraic structure we obtained was once again the tropical numbers. Let us talk about the perspective of valued fields.
\subsubsection{Puiseux series}
Recall from our times in Calculus 1 that when resolving indeterminate limits, the relevant information is contained in the order of vanishing of the function.

\begin{Ex}
    Consider the limit $\lim_{t\to 0}\frac{\sin(x)}{x}=1$. Near $t=0$ we have 
    $$\sin(t)=t+o(t)\sim t^1\word{and}\frac{1}{t}=t^{-1}\word{so}t^1t^{-1}=t^0=1.$$
\end{Ex}

From this, we care to study the orders of zeroes and poles of Laurent series. In order to extend the class of functions to an algebraically closed field, we consider Puiseux series, or rational functions. We can identify Puiseux series as 
$$\bC\set{\set{t}}=\bigcup_{n\in\bN}\bC(t^{1/n}).$$
Concretely, elements here are Laurent series with rational exponents and the exponents of terms with non-zero coefficients have a common denominator. 
\begin{Ex}
    The series $\sum_{k=-37}^{\infty}t^{k/42}$ is a Puiseux series while $\sum_{k=1}^\infty t^{1/k}$ is not because the exponents keep getting smaller and smaller.
\end{Ex}
This is the most natural algebraically closed field with a \emph{canonical} valuation. This is the function:
$$\val: \bC\set{\set{t}}\to\bR\cup\set{\infty},\begin{cases}
    0\mapsto\infty\\
    t^{p/q}+\text{higher order}\mapsto p/q
\end{cases}$$

In other words the valuation sends $\sum_{k=k_0}^\infty a_kt^{q_k}$ to $q_{k_0}$.
\begin{Prop}
The previous valuation enjoys the following properties:
\begin{enumerate}[i)]
    \item $\val(\al\.\bt)=\val(\al)+\val(\bt)$.
    \item $\val(\al+\bt)\geq\min(\val(\al),\val(\bt))$.
\end{enumerate}
Equality holds when $\val(\al)\neq\val(\bt)$.
\end{Prop}
So if we decide to define operations on $\bR\cup\set{\infty}$ by inducing them from the operations on $\bC\set{\set{t}}$, then we obtain
\begin{align*}
    &x\diamondplus y=\val\left(\val^{-1}(x)+\val^{-1}(y)\right),
    &x\. y=\val\left(\val^{-1}(x)\.\val^{-1}(y)\right).
\end{align*}
Now $\.$ coincides with usual addition and $+$ is the hyperoperation
$$x\diamondplus y=\begin{cases}
    \min(x,y)\word{when}x\neq y,\\
    [\min(x,y),\infty]\word{when}x=y.
\end{cases}$$
\begin{Ex}
    If we try to sum $0$ with itself, we get 
    $$0\diamondplus 0=\val\left((a_0+a_1t^{q_1}+\dots)+(-a_0+b_1t^{r_1}+\dots)\right)$$
\end{Ex}
The only natural way to turn this into an operation is to define $x+y=\min(x,y)$. In conclusion, the field of Puiseux series with the order of vanishing and poles is congruent to $(\bT,+,\.)$ which in this case is $\left(\bR\cup\set{\infty},\min,+\right)$.

\subsection{The Tropical Semifield}

\begin{Def}
    The \term{tropical semifield} is $(\bT,+,\.)$ where we can choose:
    \begin{itemize}
        \item $\bT=\bR\cup\infty$, $+=\min$ and $\.=+$, the min convention.
        \item $\bT=\set{-\infty}\cup\bR$, $+=\max$ and $\.=+$, the max convention.
    \end{itemize}
\end{Def}

There is a natural isomorphism between the two choices given by $x\mapsto -x$. As we have mentioned, different contexts may be more natural than the other when using certain conventions. We will tipically use the $\max$ convention. 

\begin{Prop}
The following algebraic properties hold for $(\bT,+,\.)$:
\begin{enumerate}[i)]
    \item $0_\bT=-\infty$.
    \item $1_{\bT}=0$.
    \item $x+y=0_\bT$ only has the solution $x=y=0_\bT$. This means that only $-\infty$ has an additive inverse.
    \item Addition is idempotent: $x+x=x$.
    \item Every non-zero element has a multiplicative inverse: $1/x=-x$.
\end{enumerate}
\end{Prop}

\begin{ptcbp}
    \begin{enumerate}[i)]
    \item Observe that $x+0_\bT=\max(x,-\infty)=x$.
    \item $x\.1_\bT=x+0=x$.
    \item $x+y=0_\bT\iff \max(x,y)=-\infty\To x=y=-\infty$.
    \item $x+x=\max(x,x)=x$.
    \item $x\.(1/x)=x+(-x)=0=1_\bT$.
\end{enumerate}
\end{ptcbp}
Observe that it is not possible to adjoin formal additive inverses. Suppose that for $x\in\bT$ there exists a $y$ such that $x+y=0_\bT$, then 
$$(x+x)+y=x+y=0_\bT\word{and}x+(x+y)=x+0_\bT=x\word{but}x\neq 0\_\bT.$$
This means that any invertible element necessarily has to be $-\infty$.

\begin{Ej}[2-]
Which other algebraic properties do these operations enjoy? We have claimed for example that $+$ is associative. Prove this.\par 
Are the operations commutative? Do they distribute with respect to each other?
\end{Ej}

\begin{Prop}[Weird Fun Facts]
Recall that the usual Pascal Triangle is built by adding the \emph{previous two} elements to get the next one. In the tropical case we have 
$$
\begin{tikzpicture}
    \foreach \n in {0,...,2} {
      \foreach \k in {0,...,\n} {
        \node at (\k-\n/2,-\n) {$1_\bT$};
      }
    }
    \end{tikzpicture}
    \word{\raisebox{2.5em}{=}}%tex.se/47016
    \begin{tikzpicture}
        \foreach \n in {0,...,2} {
          \foreach \k in {0,...,\n} {
            \node at (\k-\n/2,-\n) {$0$};
          }
        }
        \end{tikzpicture}
    $$
    and this extends downwards with the same pattern.\par 
    In the case of the tropical binomial theorem, the identity is
    $$(x+y)^n=x^n+y^n\iff n\max(x,y)=\max(nx,ny).$$
\end{Prop}

\begin{Ej}[2]
Recall that the coefficients in the expansion for the binomial theorem are the corresponding elements in the rows of the Pascal Triangle. Verify if the coefficients agree in the tropical case for the binomial theorem.
\end{Ej}


\subsubsection{The Optimal Assignment Problem}

Suppose we have $n$ jobs for $n$ workers. Each worker can only work one job and once the job is taken, no one else can do it. We wish to assign a job to each worker in order to maximize our company's profit.

\begin{Ex}
    As a little example consider Alice and Bob's hydroponics farm. When working with the weeds Alice produces $5$ credits while working with the water she produces $6$. On the other hand Bob produces $3$ and $5$ respectively.\par 
    It is easy to see that Alice should be assigned to to the weed and Bob to the water in order to maximize. But let us apply what we know with tropical arithmetics.\par 
    Call 
    $$M_{ij}=\text{amount of credits work }i\text{ produces when doing job }j.$$
    Then we can summarize the previous information in a matrix 
    $$M=\twobytwo{5}{6}{3}{5}$$
    and if we take the tropical determinant (which is really a permanent since we lack subtraction) we get
    $$\Trop\det M=5\.5+6\.3=\max(5+5,6+3)=10$$
    which is the maximal profit we can make by assigning our workers.
\end{Ex}

\begin{Ej}
    Do the following:
    \begin{itemize}
        \item[(1-)] Construct a $3\x3$ matrix with non-permuted entries such that there's more than one possible assignment for the optimal jobs.
        \item[(1)] Use the combinatorial definition of permanent to show that the tropical determinant of $M$ is indeed the maximal profit. \hint{The definition of permanent is the same as the determinant but without the $(-1)^{\sgn\sg}$.}
        \item[(5)] Assuming you know the tropical determinant of a matrix, devise a way to identify one job combination which reaches the optimum value. 
    \end{itemize}
\end{Ej}
%%%%%%%%%%%% Contents end %%%%%%%%%%%%%%%%

%%%%%%% testing! testing testing testing testing!
\ifx\nextra\undefined
\printindex
\else\fi
\nocite{*}
\bibliographystyle{plain}
\bibliography{bibiTropiGeo.bib}
\end{document} 

