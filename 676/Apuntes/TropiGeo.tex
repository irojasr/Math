\documentclass[12pt]{memoir}

\def\nsemestre {II}
\def\nterm {Fall}
\def\nyear {2023}
\def\nprofesor {Renzo Cavalieri}
\def\nsigla {MATH676}
\def\nsiglahead {Tropical Geometry}
\def\nlang {ENG}
%\def\darktheme{}
\let\footruleskip\relax %%FADIR
\input{../../headerVarillyDiff}
\usepackage[enableskew]{youngtab}

\DeclareMathOperator{\val}{val}
\DeclareMathOperator{\Trop}{Trop}
\newcommand{\diamondplus}{\mathbin{\rotatebox[origin=c]{45}{$\boxtimes$}}} %https://tex.stackexchange.com/questions/73275/rotated-ltimes-symbol

\begin{document}
%\clearpage
\maketitle
%\thispagestyle{empty}
{\small 
\setlength{\parindent}{0em}
\setlength{\parskip}{1em}

This is a topics course on this stuff

\subsubsection*{Requirements}
Knowledge on stuff\par 

\textbf{TO DO:}
\begin{itemize}
    \item Write info on course description and requirements.
    \item Polish info from day 1
    \item Polish last part of day 2
    \item Write Interim about valued fields specifically Puiseux series and notation and on grobner complexes
\end{itemize}
}
\newpage
\tableofcontents
%\begin{multicols}{2}
\chapter{Combinatorial Shadow of Algebraic Geometry}

\section{Day 1|20230821}

Think of an algorithm where the input is an algebraic variety and the output is a combinatorial object, a piecewise linear object.

\begin{Ex}
    Consider as an input a line in the plane. Say $V(x+y-1)$, then an output would be a tropical line. If we remain in the plane and consider a higher degree polynomial, say an elliptic curve, as an output we obtain a tropical cubic.\par 
    Leaving the plane behind and thinking of abstract nodal curves, we can think of a sphere attached to a torus which is attached to a genus 2 torus, then the corresponding object is what we call the \term{dual graph}.
\end{Ex}

Right now we do not know the specific algorithm, but we can observe that the outputs are \emph{more simple} than the inputs. So the important question is:
\begin{significant}
What algebraic information does the simplified object remember? How do we extract the information the object remembers? And once we know how to work with this objects, can we return to algebraic geometry from any kind of these objects?
\end{significant}

Observe that the number of ends which go to infinity corresponds with the degree. 

\section{Day 2|20230823}
%%Based on 676-Intro
\subsection{Algebraic Geometry on $\bT$}
Let us talk about ways to get into tropical geometry. We will first define the tropical semifield which the base set over which we will do algebraic geometry.

\begin{Def}
    The \term{tropical semifield} is the set $(\bR\cup\set{-\infty})$ equipped with tropical addition and multiplication:
    $$
    \begin{cases}
        x\oplus y=\max(x,y)\\
        x\odot y=x+y
    \end{cases}
    $$
\end{Def}

With this set we can make multivariable polynomials 
$$p(\un x)\:\left(\bR\cup\set{-\infty}\right)^n\to\bR\cup\set{-\infty}$$
which gives rise to their \emph{tropicalization}, a piecewise linear function $\Trop(p)\:\bR^n\to\bR$.

\begin{Ex}
    Consider the polynomial 
    $$p(x,y)=x\oplus y\oplus 0,$$
    its tropicalization is $\Trop(p)(x,y)=\max(x,y,0)$ which indeed is a piecewise linear function from $\bR^2$ to $\bR$.
    \begin{figure}[h!]
        \centering
        \subcaptionbox{$x\oplus y\oplus 0$\label{fig:2.1-LinearTropicalPolynomial}}{\includegraphics[width=0.3\textwidth]{figs/fig2-1-LinearTropicalPolynomial.pdf}}\quad
        \subcaptionbox{Projection onto $xy$-plane\label{fig:2.2-ProjectionLinearTropicalPolynomial}}{\includegraphics[width=0.25\textwidth]{figs/fig2-2-LinearTropicalPolynomialProjected.pdf}}\quad
        \subcaptionbox{Corner locus\label{fig:2.3-CornerLocus}}{\includegraphics[width=0.25\textwidth]{figs/fig2-3-CornerLocus.pdf}}
        %\caption{This is the caption.}
        \label{fig:2.1-and-2.2-and-2.3}
    \end{figure}
    %https://mathematica.stackexchange.com/questions/169777/listplot3d-with-contours-projected-onto-the-xy-plane
    Observe that the surface is not smooth where the planes meet, this is what we will call the \emph{corner locus} or \emph{tropical hypersurface}.
    %In two variables we have a PICTURE. The polynomial $x\oplus y\oplus 0$ is actually $\max(x,y,0)$. This picture is actually the projection of the corner locus. In 3D we can visualize this better.
\end{Ex}

\begin{Def}
    The \term{tropical hypersurface} $V(\Trop(p))$ is the codimension 1 locus in $\bR^n$ where the function is non-linear (corner locus).
\end{Def}

\begin{Ex}
If we consider higher degree tropical polynomials, they will become linear in the usual sense. Consider 
$$p(x)=3x^2=3\odot x\odot x=3+x+x=3+2x$$
which is indeed linear.
\end{Ex}

\subsection{Valued fields}

\begin{Def}
The field of \term{Puiseux series} or rational functions over $\bC$ is $\bC(t)$ where the elements are of the form 
$$f(t)=\sum_{i=k_0}^\infty a_it^{i/n}.$$
The lower bound $k_0$ could be negative and the exponents, are rational with bounded denominators. 
\end{Def}

Consider the valuation 
$$\val_0\:\bC(t)\to\bR\cup\set{\infty},\begin{cases}
    0\mapsto \infty\\
    f\mapsto\text{order of vanishing at }0.
\end{cases}$$
This order of vanishing is the value $\al$ such that $f/t^\al$ approaches a finite non-zero value. The corresponding coefficient in the series expansion of $f$ for this value is called the valuation coefficient.

\begin{Ex}
    What happens to the order of vanishing when you add two functions? Consider $f=t^2,\ g=t^3$, then $f+g=t^2+t^3$ which has order of vanishing $2$. Observe that $2=\min(2,3)$.
\end{Ex}
In general what happens is that
$$\val_0(f_1+f_2)\geq\min(\val_0 f_1,\val_0 f_2),\word{and}\val_0(f_1f_2)=\val_0(f_1)+\val_0(f_2).$$

We can do algebraic geometry over this field! Let $K$ be the field of rational functions, if $p(\un x)\in K\bonj{\un x}$ then we consider the algebraic variety
$$X=V(p)=\set{\vec{x}\:p(\vec{x})=0}\subseteq K^n.$$
Taking the image through the $n$-fold valuation, we will obtain a set in $\left(\bR\cup\set{\infty}\right)^n$. The tropicalization of $X$ is the image via this map:
\begin{figure}[h!]
    
\centering
\tikzset{every picture/.style={line width=0.75pt}} %set default line width to 0.75pt        

\begin{tikzpicture}[x=0.75pt,y=0.75pt,yscale=-1,xscale=1]
%uncomment if require: \path (0,300); %set diagram left start at 0, and has height of 300

%Straight Lines [id:da2454102694181276] 
\draw    (235,63) -- (299,63) ;
\draw [shift={(301,63)}, rotate = 180] [color={rgb, 255:red, 0; green, 0; blue, 0 }  ][line width=0.75]    (10.93,-3.29) .. controls (6.95,-1.4) and (3.31,-0.3) .. (0,0) .. controls (3.31,0.3) and (6.95,1.4) .. (10.93,3.29)   ;
%Straight Lines [id:da5779705315012776] 
\draw    (241,113) -- (298,113) ;
\draw [shift={(300,113)}, rotate = 180] [color={rgb, 255:red, 0; green, 0; blue, 0 }  ][line width=0.75]    (10.93,-3.29) .. controls (6.95,-1.4) and (3.31,-0.3) .. (0,0) .. controls (3.31,0.3) and (6.95,1.4) .. (10.93,3.29)   ;
\draw [shift={(241,113)}, rotate = 180] [color={rgb, 255:red, 0; green, 0; blue, 0 }  ][line width=0.75]    (0,5.59) -- (0,-5.59)   ;

% Text Node
\draw (208,53.4) node [anchor=north west][inner sep=0.75pt]    {$K^{n}$};
% Text Node
\draw (303,53.4) node [anchor=north west][inner sep=0.75pt]    {$(\bR\cup\set{\infty})^{n}$};
% Text Node
\draw (202,103.4) node [anchor=north west][inner sep=0.75pt]    {$V( p)$};
% Text Node
\draw (304,99.4) node [anchor=north west][inner sep=0.75pt]    {$\overline{\operatorname{val}_0( V( p))}$};
% Text Node
\draw (172,153.4) node [anchor=north west][inner sep=0.75pt]    {$\{\vec{x} :p(\vec{x}) =0\}$};
% Text Node
\draw (302,153.4) node [anchor=north west][inner sep=0.75pt]    {$\operatorname{Trop}( V( p))$};
% Text Node
\draw (210.4,98) node [anchor=north west][inner sep=0.75pt]  [rotate=-270]  {$\subseteq $};
% Text Node
\draw (328.4,98) node [anchor=north west][inner sep=0.75pt]  [rotate=-270]  {$\subseteq $};
% Text Node
\draw (229.6,127) node [anchor=north west][inner sep=0.75pt]  [rotate=-90]  {$=$};
% Text Node
\draw (346.6,127) node [anchor=north west][inner sep=0.75pt]  [rotate=-90]  {$=$};
% Text Node
\draw (172,53.4) node [anchor=north west][inner sep=0.75pt]    {$\operatorname{val}_0\:$};


\end{tikzpicture}

\end{figure}\\
and here $\Trop(V(p))$ is the tropical hypersurface for $p$.
\begin{Ex}
    Consider the polynomial in $K\bonj{x,y}$ 
    $$p(x,y)=tx+y+t^2,$$
    then the variety is $X=\set{(x,y)\: tx+y+t^2=0}$ which we can solve to $y=-tx-t^2$.\par 
    If we choose $x=0$ then $y$ becomes $-t^2$. Now we take the valuation of $(0,-t^2)$ and so $(\infty,2)\in\Trop(X)$.
\end{Ex}

\subsection{Amoebas}

Let us return to the usual stage and consider $p\in\bC[\un x]$ which defines an algebraic variety $X=V(p)\subseteq\bC^n$. Now consider the map which sends every coordinate's modulus to its logarithm in base $t$: 
$$\bC^n\to\left(\bR\cup\set{-\infty}\right)^n,\quad (z_1,\dots,z_n)\to(\log_t|z_1|,\dots,\log_t|z_n|).$$


The image of $X$ under this map, $\log_t(X)$, is the $t$-amoeba of $X$. If we take the limit as $t\to\infty$ then we get the \emph{spine} of the amoeba. 

\begin{Ex}
    When $p(x,y)=x+y-1$ then we can describe $V(p)$ via the parametrization $(x,1-x)$. So the corresponding $t$-amoeba in the real case is 
    $$\set{(\log_t|x|,\log_t|1-x|)\: x\in\bR}$$
    and we ordinarily take the limit, we see that the functions converge to zero point-by-point. But the set is actually approaching the spine!
    \begin{figure}[h!]
        \centering
        \subcaptionbox{$X=V(x+y-1)$\label{fig:2.4-Variety}}{\includegraphics[width=0.25\textwidth]{figs/fig2-4-V(x+y-1).pdf}}\quad
        \subcaptionbox{$2$-amoeba of $X$\label{fig:2.5-2Amoeba}}{\includegraphics[width=0.25\textwidth]{figs/fig2-5-2amoeba.pdf}}\quad
        \subcaptionbox{Sequence of amoebas as $t\to\infty$\label{fig:2.6-ApproxAmoebas}}{\includegraphics[width=0.25\textwidth]{figs/fig2-6-ApproximatingAmoebas.pdf}}
        %\caption{This is the caption.}
        \label{fig:2.4-thru-2.6}
    \end{figure}
\end{Ex}

Observe that the spine approaches the tropical hypersurface associated to $p$. In other words we have that the tropical hypersurface is $\lim_{t\to\infty}\log_t(V(p))$.

\subsection{Degenerations}
We may parametrize any algebraic variety with a time variable, then converting the information to a graph, edges code the information about how fast the node forms related to the length.\par
Consider a family of \red{of what, what is this family of?! Stuff? Curve in P1xP1 which eventually becomes P2?}
\begin{figure}[h!]
    \centering
    \includegraphics[width=0.5\textwidth]{figs/fig1-1.png}
\end{figure}

It is too early to understand this point of view. We will set everything up to get to it.\par 
In general, the big idea will be to explore and understand these perspectives in the case of plane curves. We want to show how they are equivalent and then recover classical algebraic geometry results in terms of tropical geometry.

\section{Day 3|20230825}

Recall that the last time we discussed the classical (25 to 30 years old) ways to get to tropical geoemtry. We now would like to answer the question
\begin{significant}
    Where do tropical numbers come from?
\end{significant}
So let us begin with an applications problem and see how the tropical numbers arise from the context of the problem.

\subsection{Tropical Arithmetics}%%%Based on TropicalNumbers.pdf

\subsubsection{Minimizing Tolls}

Consider a set of cities connected by a network of toll-ways:
\begin{figure}[h!]
    \centering
    \includegraphics[width=0.5\textwidth]{figs/fig1-2.png}
\end{figure}
If we only care about minimizing toll expenses when traveling, what would be the cheapest way to go from one given city to another? Let us record the information as an incidence matrix:
$$M_{ij}=\text{price of going from city }i\text{ to city }j\text{ in at most one trip}\To M=\threebythree{0}{\infty}{2}{x}{0}{y}{\infty}{1}{0}$$
In this matrix, the rows determine the outbound city, while the columns are the destination. Each entry records the cost of a toll and tolls are considered to be infinite when the road does not exist. We can also think of $M$ as recording the cheapest toll to go from one city to another with at most one move.\par 
How would we compute the best strategy of going from city $i$ to $j$ in \emph{at most two trips}? If for example we want to find trips from $A$ to $B$ in two steps then we have three choices:
$$AAB,\quad ABB,\quad ACB.$$
The costs of each one are 
$$(0,\infty),\quad (\infty,0),\quad (2,1)$$
so we sum them and take the minimum. That will be the optimal route from $A$ to $B$ in two steps. In fact, if we relate this to the entries of the matrix $M$, we could use $M^2$. However we must redefine our basic operations as follows: 
$$+=\min,\quad\.=+$$
So we have the identification 
$$(1,2)\text{ entry of }M^2=\sum_{j=1}^{3}M_{1k}M_{k2}=\min(M_{11}+M_{12},M_{12}+M_{22},M_{13}+M_{32}).$$
In general:
\begin{align*}
    \threebythree{0}{\infty}{2}{x}{0}{y}{\infty}{1}{0}^2&=\threebythree{\min\threebyone{0+0}{\infty+x}{2+\infty}}{\min\threebyone{0+\infty}{\infty+0}{2+1}}{\min\threebyone{0+2}{\infty+y}{2+0}}{\min\threebyone{x+0}{0+x}{y+\infty}}{\min\threebyone{x+\infty}{0+0}{y+1}}{\min\threebyone{x+2}{0+y}{y+0}}{\min\threebyone{\infty+0}{1+x}{0+\infty}}{\min\threebyone{\infty+\infty}{1+0}{0+1}}{\min\threebyone{\infty+2}{1+y}{0+0}}\\
    &=\threebythree{0}{3}{2}{x}{\min(0,y+1)}{\min(x+2,y)}{1+x}{1}{\min(0,1+y)}.
\end{align*}
Observe that $1+y$ can be the minimum in the diagonal when we allow \emph{negative tolls}.
\begin{Rmk}
If we disallow negative tolls, the products $M^n$ eventually stabilize to a matrix whose entries record the cheapest way to get from one city to another in $n$ steps.
\end{Rmk}
This gives us an intuition that minimization problems correspond to linear algebra problems over $(\bT,+,\.)$ which is precisely $(\bR\cup\set{\infty},\min,+)$.

\subsubsection{Forgetting phases}

Recall that any complex number can be written as $z=re^{i\te}$ where $r\geq 0$. Consider the map $T_t\:\bC\to\set{-\infty}\cup\bR,\quad z\mapsto\log_t(r)$.
\begin{figure}[h!]
    \centering
    \includegraphics[width=0.5\textwidth]{figs/fig1-3.png}
\end{figure}
This map is surjective, and this we can see by checking it is right-invertible. Observe that:
$$
\left\lbrace
\begin{aligned}
    &T_t^{-1}(x)=\set{t^xe^{i\te}}\subseteq\bC,\word{for}x\in\bR,\\
    &T_t^{-1}(-\infty)=0.
\end{aligned}
\right.
$$
With this in hand, we wish to define an exotic addition and multiplication on $\set{-\infty}\cup\bR$ using $T_t$. We will dequantize!\par 
We begin with \textbf{hyper-addition}, the output will be a subset of $\set{-\infty}$ so it's not a binary operation by itself. 
$$x\diamondplus_t y\:= T_t(T_t^{-1}(x)+T_t^{-1}(y))=\bonj{\log_t(|t^x-t^y|),\log_t(t^x+t^y)}.$$
This is an interval in $\set{-\infty}\cup\bR$, in order to make $\diamondplus_t$ into an operation we take a limit:
\begin{figure}[h!] 
    \centering
\begin{tikzpicture}[x=0.75pt,y=0.75pt,yscale=-1,xscale=1]
%uncomment if require: \path (0,300); %set diagram left start at 0, and has height of 300

%Straight Lines [id:da5156276968518897] 
\draw    (85,62.6) -- (142,62.99) ;
\draw [shift={(144,63)}, rotate = 180.39] [color={rgb, 255:red, 0; green, 0; blue, 0 }  ][line width=0.75]    (10.93,-3.29) .. controls (6.95,-1.4) and (3.31,-0.3) .. (0,0) .. controls (3.31,0.3) and (6.95,1.4) .. (10.93,3.29)   ;

%Straight Lines [id:da8224691621679702] 
\draw    (96,133) -- (144,133.38) ;
\draw [shift={(146,133.4)}, rotate = 180.46] [color={rgb, 255:red, 0; green, 0; blue, 0 }  ][line width=0.75]    (10.93,-3.29) .. controls (6.95,-1.4) and (3.31,-0.3) .. (0,0) .. controls (3.31,0.3) and (6.95,1.4) .. (10.93,3.29)   ;
%Straight Lines [id:da27001319663870027] 
\draw    (57,77) -- (57,118) ;
\draw [shift={(57,120)}, rotate = 270] [color={rgb, 255:red, 0; green, 0; blue, 0 }  ][line width=0.75]    (10.93,-3.29) .. controls (6.95,-1.4) and (3.31,-0.3) .. (0,0) .. controls (3.31,0.3) and (6.95,1.4) .. (10.93,3.29)   ;
%Straight Lines [id:da8909494911629017] 
\draw    (195,83) -- (195,120) ;
\draw [shift={(195,122)}, rotate = 270] [color={rgb, 255:red, 0; green, 0; blue, 0 }  ][line width=0.75]    (10.93,-3.29) .. controls (6.95,-1.4) and (3.31,-0.3) .. (0,0) .. controls (3.31,0.3) and (6.95,1.4) .. (10.93,3.29)   ;

% Text Node
\draw (34,53.4) node [anchor=north west][inner sep=0.75pt]    {$x\diamondplus_t y$};
% Text Node
\draw (34,123.4) node [anchor=north west][inner sep=0.75pt]    {$x\ +_{t} \ y$};
% Text Node
\draw (152,53.4) node [anchor=north west][inner sep=0.75pt]    {$x\diamondplus y=\lim _{t\rightarrow \infty } x\diamondplus_t y$};
% Text Node
\draw (152,123.4) node [anchor=north west][inner sep=0.75pt]    {$x+y=\max( x,y)$};
% Text Node
\draw (60,98.5) node [anchor=west] [inner sep=0.75pt]  [font=\scriptsize]  {$\max$};
% Text Node
\draw (114.5,59.4) node [anchor=south] [inner sep=0.75pt]  [font=\scriptsize]  {$\displaystyle\lim _{t\rightarrow \infty }$};
% Text Node
\draw (121,129.8) node [anchor=south] [inner sep=0.75pt]  [font=\scriptsize]  {$\displaystyle\lim _{t\rightarrow \infty }$};
% Text Node
\draw (197,102.5) node [anchor=west] [inner sep=0.75pt]  [font=\scriptsize]  {$\max$};


\end{tikzpicture}

\end{figure}

\begin{Rmk}
Note that $\diamondplus$ is still a hyperoperation. Its output is not a singleton \emph{only} when a dding a number to itself:
$$x\diamondplus y=\begin{cases}
    \max(x,y),\quad x\neq y\\
    \bonj{-\infty,x},\quad x=y
\end{cases}$$
\end{Rmk}

Formally this process, taking a limit of a family of operations, is known as \emph{dequantization}.\par

In the case of multiplication, things go a lot smoother when defining it:

$$x\.y =T_t\bonj{T^{-1}(x)\.T^{-1}(y)}=\log_t\bonj{(t^xe^{i\te})(t^ye^{i\vf})}=\log_t\left(t^{x+y}e^{i(\te+\vf)}\right)$$

Separating the logarithm we get $(x+y)+\log(e^{i(\te+\vf)})/\log(t)$, then letting $t$ grow without bound we see that the operation converges to $x+y$. 

\begin{Ex}
    Let us consider a small example like summing $2$ and $4$. Observe that 
    $$4\diamondplus_t 2=T_t(T_t^{-1}(4)+T_t^{-1}(2))=T_t\left(t^4e^{i\te}+t^2e^{i\vf}\right)$$
    and the term on the inside can be simplified to $t^{4}(e^{i\te}+t^{-2}e^{i\vf})$. $T_t$ takes that expression to
    $$4+\log_t(e^{i\te}+t^{-2}e^{i\vf})=4+\frac{\log(e^{i\te}+t^{-2}e^{i\vf})}{\log t}.$$
    What happens if we take the limit as $t\to\infty$? We get an independent from $t$ result! 
    The term on the right vanishes and we are left with $4=\max(4,2)$. So it got a tad bit better, but it's still a hyperoperation!
\end{Ex}

\begin{Ej}
Check how the definition of $+$ and $\.$ extend to the \emph{number} $-\infty$.
\end{Ej}

\begin{ptcb}
The point of this exercise is to operate $-\infty$ with finite numbers and itself.\par
For a finite $x$ we will find $x+(-\infty)$. This is the limit of the previous hyperoperation:
$$x\diamondplus_t(-\infty)=T_t(T_t^{-1}(x)+T_t^{-1}(-\infty))=T_t(T_t^{-1}(x)+0)=T_t(T_t^{-1}(x))=x.$$
If we let $t$ grow, the result doesn't change and so this goes according to $\max(x,-\infty)=x$.\par 
On the other hand when taking the product:
$$x\.(-\infty)=T_t\bonj{T^{-1}(x)\.T^{-1}(-\infty)}=T_t\bonj{T^{-1}(x)\.0}=T_t(0)=\log_t(0)\to-\infty$$
which is also similar to the notion of $x+(-\infty)=-\infty$.\par 
We can now proceed to operate $-\infty$ with itself:
$$(-\infty)\diamondplus_t(-\infty)=T_t(0)=\log_t(0)=-\infty=\max(-\infty,-\infty),$$
and when taking the product:
$$(-\infty)\.(-\infty)=T_t(0)\log_t(0)=-\infty=(-\infty)+(-\infty)$$
where the last sum is a sum in the usual sense.
\end{ptcb}

So, summarizing this process:
\begin{itemize}
    \item We forgot about the phase of the complex numbers and only looked at them radially. 
    \item The modulus of these numbers was scaled logarithmically.
    \item Finally we took the limit of these operations and obtained the desired (somewhat) result.
\end{itemize}
This is known as Maslov\footnote{Viktor Pavlovich Maslov (1930615-20230803)} dequantization and with this we can see $(\bT,+,\.)$ as $(\set{-\infty}\cup\bR,\max,+)$. Also, we will abbreviate $\lim_{t\to\infty}T_t$ with $T_{t\to\infty}$

\section{Interim}
%%valued fields and puiseux seires

\section{Day 4|20230828}

We have seen where our ideas come from. Certain kinds of minimization problems give rise to our tropical numbers. Also by expressing complex numbers in a logarithmic scale without phase then when inducing a sum we actually get a hypersum. The way we converted into an operation is by taking a limit. Then the algebraic structure we obtained was once again the tropical numbers. Let us talk about the perspective of valued fields.
\subsection{Puiseux series}
Recall from our times in Calculus 1 that when resolving indeterminate limits, the relevant information is contained in the order of vanishing of the function.

\begin{Ex}
    Consider the limit $\lim_{t\to 0}\frac{\sin(x)}{x}=1$. Near $t=0$ we have 
    $$\sin(t)=t+o(t)\sim t^1\word{and}\frac{1}{t}=t^{-1}\word{so}t^1t^{-1}=t^0=1.$$
\end{Ex}

From this, we care to study the orders of zeroes and poles of Laurent series. In order to extend the class of functions to an algebraically closed field, we consider Puiseux series, or rational functions. We can identify Puiseux series as 
$$\bC\set{\set{t}}=\bigcup_{n\in\bN}\bC(t^{1/n}).$$
Concretely, elements here are Laurent series with rational exponents and the exponents of terms with non-zero coefficients have a common denominator. 
\begin{Ex}
    The series $\sum_{k=-37}^{\infty}t^{k/42}$ is a Puiseux series while $\sum_{k=1}^\infty t^{1/k}$ is not because the exponents keep getting smaller and smaller.
\end{Ex}
This is the most natural algebraically closed field with a \emph{canonical} valuation. This is the function:
$$\val: \bC\set{\set{t}}\to\bR\cup\set{\infty},\begin{cases}
    0\mapsto\infty\\
    t^{p/q}+\text{higher order}\mapsto p/q
\end{cases}$$

In other words the valuation sends $\sum_{k=k_0}^\infty a_kt^{q_k}$ to $q_{k_0}$.
\begin{Prop}\label{prop:PropertiesOfValuation}
For $\al,\bt\in\bC\set{\set{t}}$, the valuation enjoys the following properties:
\begin{enumerate}[i.]
    \item $\val(\al\.\bt)=\val(\al)+\val(\bt)$.
    \item $\val(\al+\bt)\geq\min(\val(\al),\val(\bt))$.
\end{enumerate}
Equality holds when $\val(\al)\neq\val(\bt)$.
\end{Prop}
So if we decide to define operations on $\bR\cup\set{\infty}$ by inducing them from the operations on $\bC\set{\set{t}}$, then we obtain
\begin{align*}
    &x\diamondplus y=\val\left(\val^{-1}(x)+\val^{-1}(y)\right),
    &x\. y=\val\left(\val^{-1}(x)\.\val^{-1}(y)\right).
\end{align*}
Now $\.$ coincides with usual addition and $+$ is the hyperoperation
$$x\diamondplus y=\begin{cases}
    \min(x,y)\word{when}x\neq y,\\
    [\min(x,y),\infty]\word{when}x=y.
\end{cases}$$
\begin{Ex}
    If we try to sum $0$ with itself, we get 
    $$0\diamondplus 0=\val\left((a_0+a_1t^{q_1}+\dots)+(-a_0+b_1t^{r_1}+\dots)\right)$$
    and this could be either $q_1$ or $r_1$ because the constant terms cancel! 
\end{Ex}
The only natural way to turn this into an operation is to define $x+y=\min(x,y)$. In conclusion, the field of Puiseux series with the order of vanishing and poles is congruent to $(\bT,+,\.)$ which in this case is $\left(\bR\cup\set{\infty},\min,+\right)$.

\subsection{The Tropical Semifield}

\begin{Def}
    The \term{tropical semifield} is $(\bT,\oplus,\odot)$ where we can choose:
    \begin{itemize}
        \item $\bT=\bR\cup\infty$, $\oplus$ to be $\min$ and $\odot$ is $+$, the min convention.
        \item $\bT=\set{-\infty}\cup\bR$, $\oplus=\max$ and $\odot=+$, the max convention.
    \end{itemize}
\end{Def}

There is a natural isomorphism between the two choices given by $x\mapsto -x$. As we have mentioned, different contexts may be more natural than the other when using certain conventions. We will tipically use the $\max$ convention. 

\begin{Prop}
The following algebraic properties hold for $(\bT,+,\.)$:
\begin{enumerate}[i.]
    \item $0_\bT=-\infty$.
    \item $1_{\bT}=0$.
    \item $x+y=0_\bT$ only has the solution $x=y=0_\bT$. This means that only $-\infty$ has an additive inverse.
    \item Addition is idempotent: $x+x=x$.
    \item Every non-zero element has a multiplicative inverse: $1/x=-x$.
\end{enumerate}
\end{Prop}

\begin{ptcbp}
    \begin{enumerate}[i.]
    \item Observe that $x+0_\bT=\max(x,-\infty)=x$.
    \item $x\.1_\bT=x+0=x$.
    \item $x+y=0_\bT\iff \max(x,y)=-\infty\To x=y=-\infty$.
    \item $x+x=\max(x,x)=x$.
    \item $x\.(1/x)=x+(-x)=0=1_\bT$.
\end{enumerate}
\end{ptcbp}
Observe that it is not possible to adjoin formal additive inverses. Suppose that for $x\in\bT$ there exists a $y$ such that $x+y=0_\bT$, then 
$$(x+x)+y=x+y=0_\bT\word{and}x+(x+y)=x+0_\bT=x\word{but}x\neq 0\_\bT.$$
This means that any invertible element necessarily has to be $-\infty$.

\begin{Ej}[2-]
Which other algebraic properties do these operations enjoy? We have claimed for example that $+$ is associative. Prove this.\par 
Are the operations commutative? Do they distribute with respect to each other?
\end{Ej}

\begin{Prop}[Weird Fun Facts]
Recall that the usual Pascal Triangle is built by adding the \emph{previous two} elements to get the next one. In the tropical case we have 
$$
%https://tex.stackexchange.com/questions/17522/pascals-triangle-in-tikz
\begin{tikzpicture}
    \foreach \n in {0,...,2} {
      \foreach \k in {0,...,\n} {
        \node at (\k-\n/2,-\n) {$1_\bT$};
      }
    }
    \end{tikzpicture}
    \word{\raisebox{2.5em}{=}}%tex.se/47016
    \begin{tikzpicture}
        \foreach \n in {0,...,2} {
          \foreach \k in {0,...,\n} {
            \node at (\k-\n/2,-\n) {$0$};
          }
        }
        \end{tikzpicture}
    $$
    and this extends downwards with the same pattern.\par 
    In the case of the tropical binomial theorem, the identity is
    $$(x+y)^n=x^n+y^n\iff n\max(x,y)=\max(nx,ny).$$
\end{Prop}

\begin{Ej}[2]
Recall that the coefficients in the expansion for the binomial theorem are the corresponding elements in the rows of the Pascal Triangle. Verify if the coefficients agree in the tropical case for the binomial theorem.
\end{Ej}


\subsubsection{The Optimal Assignment Problem}

Suppose we have $n$ jobs for $n$ workers. Each worker can only work one job and once the job is taken, no one else can do it. We wish to assign a job to each worker in order to maximize our company's profit.

\begin{Ex}
    As a little example consider Alice and Bob's hydroponics farm. When working with the weeds Alice produces $5$ credits while working with the water she produces $6$. On the other hand Bob produces $3$ and $5$ respectively.\par 
    It is easy to see that Alice should be assigned to to the weed and Bob to the water in order to maximize. But let us apply what we know with tropical arithmetics.\par 
    Call 
    $$M_{ij}=\text{amount of credits work }i\text{ produces when doing job }j.$$
    Then we can summarize the previous information in a matrix 
    $$M=\twobytwo{5}{6}{3}{5}$$
    and if we take the tropical determinant (which is really a permanent since we lack subtraction) we get
    $$\Trop\det M=5\.5+6\.3=\max(5+5,6+3)=10$$
    which is the maximal profit we can make by assigning our workers.
\end{Ex}

\begin{Ej}
    Do the following:
    \begin{itemize}
        \item[(1-)] Construct a $3\x3$ matrix with non-permuted entries such that there's more than one possible assignment for the optimal jobs.
        \item[(1)] Use the combinatorial definition of permanent to show that the tropical determinant of $M$ is indeed the maximal profit. \hint{The definition of permanent is the same as the determinant but without the $(-1)^{\sgn\sg}$.}
        \item[(5)] Assuming you know the tropical determinant of a matrix, devise a way to identify one job combination which reaches the optimum value. 
    \end{itemize}
\end{Ej}

\section{Day 5|20230830}

The last time we talked about the algebraic structure of the value group of the Puiseux series. We now have plenty of motivation of why would we define the tropical numbers. 

\subsection{Tropical Polynomials and Roots}

An univariate,tropical, (Laurent) monomial is equivalent to an affine linear function with integer coefficients. Such a monomial is an expression of the form 
$$a\odot x^{\odot m},\quad a\in\bT,\quad m\in\bZ.$$

\begin{Ex}
    We have for example:
    $$5x^2\otto 5+2x,\quad 2x^{-3}\otto 2-3x\ (\text{Laurent}).$$
    Also consider $\sqrt 5\odot x^{\odot 3}$ which corresponds to $y=\sqrt{5}+3y$. Notice how the slope is always an integer, meanwhile the translation can be any number.
\end{Ex}

An univariate tropical (Laurent) polynomial is a finite sum of monomials which give rise to a \emph{convex}, continuous, piecewise, affine linear function with integer slopes. 

\begin{Ex}
    Consider the function $-5\odot x^{\odot2}\oplus(-2)\odot x^{\odot-3}\oplus 0$ which corresponds to 
    $$\max(-5+2x,-2-3x,0).$$
    If we graph this functions we obtain
    \begin{figure}[h!]
        \centering
        \includegraphics[width=0.5\textwidth]{figs/fig3-1RenzoNotes3.png}
        %\caption{This is the caption.}
        \label{fig:3.1-ConvPLFunc}
    \end{figure}
    Observe that this function is indeed convex, and fulfills all of the previous properties from before. 
\end{Ex}

In fact the map from $\bT[x]$ to convex, affine piecewise linear functions with \emph{finitely} many distinct regions of linearity is surjective. If we don't want to take the finiteness condition into consideration, we have to amplify the domain to tropical Laurent series.\par 
A small measure of care should be taken because there are multiple tropical polynomials which map to the same function.

\begin{Ex}
    Consider the functions 
    $$p_1=x+\frac{1}{x}+0,\quad p_2=x+\frac1x-2.$$
    When converting we get 
    $$\max(x,-x,0),\quad\max(x,-x,-2)$$
    which produce $|x|$ in both cases.
    \begin{figure}[h!]
        \centering
        \includegraphics[width=0.5\textwidth]{figs/fig3-2RenzoNotes3.png}
        \caption{Failure of injectivity as both functions map to $|x|$ with $y=0$ and $y=-2$ shown.}
        \label{fig:3.2-InjectivityFailure}
    \end{figure}
    Adding something which is smaller than the minimum value of the function doesn't change it in general. It also doesn't have to be a constant in general. In the previous example, the the monomial $(-4)\odot x^{\odot 1}$ is smaller than any of the linear functions, so adding it changes nothing.
\end{Ex}

To talk about the roots, we will start with a purely combinatorial definition. 

\begin{Def}
    Given a polynomial $p\in\bT[x]$ of degree $d$ we say the following:
    \begin{itemize}
        \item $-\infty$ is a root of $p$ if the slope of the piecewise linear function is non-zero for $x\ll 0$.
        \item $x_0\in\bR$ is a root of $p$ if $p'(x_0)$ is undefined. Observe that the derivative is undefined only when there's a change in slope.
    \end{itemize}
    We say that the \term{multiplicity} of $x_0$ is the difference between slopes across $x_0$. If $-\infty$ is a root, then its multiplicity is equal to the slope of the associated function for $x\ll 0$.
\end{Def}

\begin{Ex}
    Consider the polynomial $x^{\odot2}\oplus1\odot x^1\oplus 0=\max(2x,x,0)$.
    \begin{figure}[h!]
        \centering
        \includegraphics[width=0.5\textwidth]{figs/fig3-3SimpleFiniteRootsTropicalPolynomial.png}
        %\caption{}
        \label{fig:3.3-SimpleFiniteRoots}
    \end{figure}
    We can see that there are changes in slope at $x_1=-1$ and $x_2=1$. The number of roots coincides with the degree of the polynomial as in the usual sense.
\end{Ex}

\begin{Ex}
    Let's remove the zero, recall zero isn't the additive identity, so the polynomial we have is $x^{\odot2}\oplus1\odot x^1=\max(2x,x)$.
    \begin{figure}[h!]
        \centering
        \includegraphics[width=0.5\textwidth]{figs/fig3-4SimpleRootsTropicalPolynomial.png}
        %\caption{}
        \label{fig:3.4-OneFiniteRootOneInfiniteRoot}
    \end{figure}
    Now one of the roots is still $x=1$, but remember that if the slope is non-zero when $x\ll 0$, then $-\infty$ is a root of $p$. This is the case here because the slope is $1$ as $x\to-\infty$. Once again there's two roots $x_1=-\infty$ and $x_2=1$.
\end{Ex}

\begin{Ex}
    Let us change a sign in a coefficient, take $x^2-1\.x^1+0$. But what is tropical subtraction? It's not that, let's convert this slowly into what it's supposed to be:
    $$x^2-1\.x^1+0=(x\.x)+(-1)\.x+0=(2x)+(x+(-1))+0=\max(2x.x-1,0).$$
    \begin{figure}[h!]
        \centering
        \includegraphics[width=0.5\textwidth]{figs/fig3-5DoubleRootTropicalPolynomial1.png}
        %\caption{}
        \label{fig:3.5-DoubleRoot1}
    \end{figure}
    Observe that because the line $y=x-1$ is below our graphs, it doesn't interfere with the calculation of zeroes. So the only place where there occurs a change in sign is $x=0$. The slope on the right is $2$ and on the left is $0$ so the multiplicity is $2-0=2$.
\end{Ex}

\begin{Ex}
    In a similar fashion, $x^2+0$ also has a double root at $x=0$.
    \begin{figure}[h!]
        \centering
        \includegraphics[width=0.5\textwidth]{figs/fig3-6DoubleRootTropicalPolynomial2.png}
        %\caption{}
        \label{fig:3.6-DoubleRoot6}
    \end{figure}
    There is only one change in slope once again at $x=0$ and the difference in slopes is $2$.
\end{Ex}

\begin{Lem}
For a tropical polynomial $p$, a finite $x_0$ is a root of $f$ if and only if when we write the function as a $\max$ of linear functions, at $x_0$ the maximum value is obtained at least twice.\par 
The multiplicity of the root is equal to the difference in the two extremal positions where the $\max$ is attained.
\end{Lem}

This should be more or less obvious. Being a root means that we are an intersection of two lines which are above all the others. It's pretty useful to have this notions around.

Questions arise:
\begin{significant}
    Which functions have only one simple zero at $-\infty$? What would a function with an order 2 zero at $-\infty$ look like?
\end{significant}

\begin{Ej}
    Do the following:
    \begin{itemize}
        \item[(5)] Is it possible for a function to have only a simple zero at $-\infty$? Provide an example of function with one simple zero at $-\infty$ or prove that such function cannot exist. 
        \item[(5)] Do functions with zeroes at $-\infty$ have infinite order at such zero or is it arbitrarily high? If a function has a finite order zero at $-\infty$ provide an example of one with a double zero at $-\infty$. Else, prove that such functions have infinite order at that zero.
    \end{itemize}
\end{Ej}

\section{Day 6|20230901}

How do we know that the notions of roots are natural or useful?

\subsection{Factorization of Tropical Polynomials}

Suppose a polynomial $p\in\bT[x]$ has roots $a_k$ with multiplicity $m_k$. Then we may factor $p$ as a product of linear polynomials 
$$p(x)=c_0\bigodot(x\oplus a_k)^{m_k}.$$
This $p$ is the affine piecewise-linear function, not the formal object. And so, in a sense, $\bT$ is algebraically closed. But instead of proving this, we will sketch the proof to get an idea of how things \emph{work} with a couple of examples.\par 
The idea of the proof is that we check that product does define a P.L. function with the right slopes and then $c_0$ gives the translation factor.
\begin{Ex}
First lets deal with the case where $-\infty$ is not a root. Consider the polynomial 
$$p(x)=(-1)\oplus(-1)\odot x\oplus(-4)\odot x^4=\max(-1,x-1,4x-4).$$
Remember, as in the case of real polynomials, the square and cube terms are still there. The coefficient that foes along them is just $-\infty$. We can graph the polynomial in order to see the roots:
\begin{figure}[h!]
    \centering
    \includegraphics[width=0.5\textwidth]{figs/fig4-1-InfinityNotRoot.png}
    \caption{Graph of $p(x)$ with roots shown}
    \label{fig:4.1-InfinityNotRoot}
\end{figure}
The points where there is a change in slope are $a_1=0$ and $a_2=1$. Then their multiplicities are $1-0=1$ and $4-1=3$ respectively. We may write $p$ as 
$$p(x)=c_0\odot(x\oplus 0)\odot(x\oplus 1)^{3}=c_0+\max(x,0)+\max(3x,3).$$
Whatever function we have, we can write as the sum of three terms. So let us subdivide the tropical line in order to see which terms goes where.
\begin{table*}[h!]
    \centering
    %\arraystretch{1.3}
    \begin{tabular}{rrrr}\toprule
        $x\leq 0$ & $0\leq x\leq 1$ & $1\leq x$\\ \midrule
        $c_0$& $c_0$&$c_0$\\
        $0$&$x$ & $x$\\
        $3$& $3$ & $3x$\\ \midrule
        $c_0+3$&$c_0+3+x$&$c_0+4x$\\
   \bottomrule
    \end{tabular}
    \legend{Behavior of $p(x)$ across $\bT$}
    \end{table*}
    The constant can be determined by plugging in $x=-\infty$. We can see that 
    \begin{align*}
        p(-\infty)&=(-1)\oplus(-1)\odot (-\infty)\oplus(-4)\odot (-\infty)^4=-1\\
        &=c_0\odot(-\infty\oplus 0)\odot(-\infty\oplus 1)^3=c_0\odot0\odot 1^{\odot 3}.
    \end{align*}
    This gives us the equation $c_0+0+3=-1$ which leads us to $c_0=-4$. With this we verify that 
    $$p(x)=\begin{cases}
        -1&x\leq 0\\
        x-1&0\leq x\leq 1\\
        4x-4&1\leq x
    \end{cases}$$
    So in this case $c_0=p(-\infty)-\sum m_ka_k\in\bR$.
\end{Ex}

\begin{Ex}
    We now explore the case where $-\infty$ is a root or a pole. The argument will essentially be the same with a small modification.\par 
    Consider the function $\frac{1}{x}\oplus x$.
    \begin{figure}[h!]
        \centering
        \includegraphics[width=0.5\textwidth]{figs/fig4-2RootAndPoleProof.png}
        %\caption{}
        \label{fig:4.2-RootAndPoleProof}
    \end{figure}
    We have $-\infty$ as a pole of order $1$ and $0$ is a root of order $1-(-1)=2$. So this can be factored as 
    $$p(x)=c_0\odot(x^{-1})\odot(x+\oplus 1)^2$$
    and even if $-\infty$ doesn't give us a particular value for the function, we can still find $c_0=0$ from the equation $p(0)=0$.\par 
    If on the other hand we have a negative slope then we have a zero at $-\infty$. Consider the function $p(x)=x+x^2$:
    \begin{figure}[h!]
        \centering
        \includegraphics[width=0.5\textwidth]{figs/fig4-3RootsForProof.png}
        %\caption{}
        \label{fig:4.3-RootsForProof}
    \end{figure}
    This function has two simple roots at $-\infty$ and $0$. We may factor it as 
    $$p(x)=c_0\odot(x\oplus-\infty)\odot(x\oplus 0)$$
    and even if $p(-\infty)=-\infty$ we can plug in $0$ to get $0$ back in order to get $c_0=1$.
\end{Ex}

\subsection{Correspondence Theorems}
Recall the maps
$$
\left\lbrace 
\begin{aligned}
    &T_t\: \bC\to\bT\quad(\text{with }\max),\\
    &\val\: \bC\set{\set{t}}\to\bT\quad(\text{with }\min).
\end{aligned}
\right.
$$
If we consider a polynomial 
$$p(X)\in\bC[X]\word{or}p(x)\in\bC\set{\set{t}}[X]$$
then we can produce a tropical polynomial as follows:

\begin{enumerate}[i.]
    \item Apply $T_t$ or $\val$ to the coefficients, and
    \item Perform tropical operations.
\end{enumerate}

We expect that if $r\in\bC$ or $r\in\bC\set{\set{t}}$ is a root of $p$, then $\lim_{t\to\infty}T_t(r)$ will be a root of the new polynomial.\par
Or the other way around, given $p\in\bT[x]$, we can lift the coefficients to $\bC$ or the Puiseux series via the above maps. We can find the roots of the corresponding polynomials in $\bC[x]$ or $\bC\set{\set{t}}[x]$ and then the image of those roots via $T_t$ or $\val$ are the tropical roots of $p(x)$.

\begin{Ex}
    Consider the polynomial $p(x)=2\odot x\oplus3\in\bT[x]$. We wish to construct a polynomial in $\bC[x]$ which tropicalizes to $p$. Take the polynomial 
    $$q(x)=t^2X+t^3\in\bC[x],\quad t>0$$ 
    We could certainly add phase as $e^{i\te}$ to the $t^k$'s, but that won't change anything. Taking the logarithm of the coefficients we get 
    $$t^2\mapsto 2\word{and}t^3\mapsto 3.$$ 
    Then switching the operations to tropical operations we have
    $$t^2X+t^3\xrightarrow[]{\Trop}2\odot X\oplus 3$$ 
    which was our original polynomial $p$.\par 
    Additionally if we solve the equation $q=0$ we obtain the root $X=-t^3/t^2=-t$. Now $\log_t|-t|=1$. Lo and behold, this is the same root of $p(x)$. 
    \begin{figure}[h!]
        \centering
        \includegraphics[width=0.5\textwidth]{figs/fig4-4CorrespondenceRoots1Example.png}
        \caption{Root of $p(x)$ in correspondence with $-t$ of $q(x)$}
        \label{fig:4.4-CorrespondenceRoots1Example}
    \end{figure}
\end{Ex}

We should be skeptical because this was only an example of a linear polynomial. Lets increase the degree and see what happens. Eventually this correspondence must be shown to hold in its entirety.

\begin{Ex}
    Consider the polynomial
    $$q(X)=X^2+t^2X+1\in\bC[X]\xrightarrow[]{\Trop}p(x)=x^2\oplus 2\odot x\oplus 0.$$
    We can identity the roots of $p$ as $-2$ and $2$. However, we may find it difficult to interpret the roots of $q$ as roots of $p$. Observe that using the quadratic formula we may derive those to be:
    $$X_{1,2}=\frac{-t^2}{2}\pm\frac{\sqrt{t^4-4}}{2}=\frac{-t^2}{2}\left(1\pm\sqrt{1-\frac{4}{t^4}}\right).$$
    Even if taking the logarithm seems hard, we are not interested in the logarithm itself, just the limit! Observe that 
    $$\lim_{t\to\infty}\log_t\left|\frac{-t^2}{2}\left(1+\sqrt{1-\frac{4}{t^4}}\right)\right|=2+\lim_{t\to\infty}\frac{1}{\log(t)}\log\left|\frac{1}{2}\left(1+\sqrt{1-\frac{4}{t^4}}\right)\right|$$
    and the quantity on the right tends to $1/\infty$ which collapses to zero and then the logarithm only has $1$ as its argument. So overall we find one our original roots: $2$! The next limit has a different sign so it is not as direct. We may calculate that limit as follows:
    $$\lim_{t\to\infty}\log_t\left|\frac{1}{2}\left(1-\sqrt{1-\frac{4}{t^4}}\right)\right|\approx\lim_{t\to\infty}\log_t\left|\frac{1}{2}\left(1-\left(1-\frac{4}{2t^4}\right)\right)\right|=\lim_{t\to\infty}\log_t\frac{1}{t^4}=-4.$$
    So for the negative root we would actually obtain $2-4=-2$ which is the other root of our polynomial.
    \begin{figure}[h!]
        \centering
        \includegraphics[width=0.5\textwidth]{figs/fig4-5CorrespondenceRoots2Example.png}
        \caption{Indeed the roots of $q$ correspond with $p$'s}
        \label{fig:4.5-CorrespondenceRoots2Example}
    \end{figure}
\end{Ex}

\section{Interim 2}

\begin{Def}
    If $q(x)=\sum a_kx^k\in\bC[X]$ or $\bC\set{\set{t}}[x]$, then the \term{tropicalization} of $q$ is 
    $$\Trop(q)=\sum T_{t\to\infty}(a_k)x^k$$
    or respectively with the valuation. In this case we omit the notation for tropical operations but the sum and product are tropical.
\end{Def}

\begin{Th}
For a polynomial $q$, $r_k$ is a root of $q(x)$ with multiplicity $m_k$ if and only if $T_{t\to\infty}(r_k)$ is a root of $\Trop(q)$ of multiplicity $m_k$.
\end{Th}

In the univariate case, we may prove the theorem using the following lemmas.

\begin{Lem}
$\Trop$ is a multiplicative function on polynomials. That is
$$\Trop(pq)=\Trop(p)\Trop(q)\word{for}p,q\in\bC[x].$$
\end{Lem}

\begin{Lem}
The roots of $\Trop(p)\Trop(q)$ are the union of the roots of the factors. If a root is repeated then the multiplicities are added.
\end{Lem}

\begin{Ej}[2]
Prove the preceding lemmas and then conclude the theorem as a result.
\end{Ej}

Otherwise, we may prove the correspondence theorem in a different way. This is more conducive to a higher number of variables. This is helpful, as in higher dimensions we don't have a fundamental theorem of algebra. But, in this case, the most convenient perspective is the valued field perspective. So let us swtich to that point of view and interpret 
$$x\oplus y=\min(x,y).$$

\begin{Th}
Let $q\in\bC\set{\set{t}}[x]$, then $r\in\bC\set{\set{t}}$ is a root of $q$ if and only if $\val(r)\in\bT\cap\bQ$ is a root of $\Trop(q)$.
\end{Th}

\begin{ptcbp}
Let us begin by considering a root $r$ of $q$, then $q(r)=0$ which means that 
$$a_0+a_1r+\dots+a_dr=0.$$
This is formal sum of monomials which in order to vanish, at least two of the monomials must reach a minimum order of vanishing to cancel. This is equivalent to $\val(r)$ being a root of $\Trop(q)$. \par %%REVIEW
The other directior is substantially more difficult. This will be an instance of a realizability question. We have two cases, $r$ is a finite root or $r=\infty$. We will assume that $r$ is finite and do a proof by example. 
\end{ptcbp}

\begin{Ex}
    Consider the polynomial 
    $$q(x)=tx^3+x^2+x+t\To\Trop q(x)=1\.x^3+x^2+x+1$$
    \begin{figure}[h!]
        \centering
        \includegraphics[width=0.85\textwidth]{figs/fig5-1RealizabilityExampleProof.png}
        \caption{Tropicalization of $q$ in $\min$ convention}
        \label{fig:5.1-RealizabilityExampleProof}
    \end{figure}
    The roots of this polynomial are $-1,0$, and $1$. We will now find a root $r_1\in\bC\set{\set{t}}$ of $q$ with $\val(r_1)=r_1$. For this to happen we requiere 
    $$r_1=yt^{-1}+z\word{where}y\in\bC\word{and}z\in\bC\set{\set{t}},\ \val z>r_1.$$
    We now plug in $r_1$ into $q$ and we obtain
    \begin{align*}
        q(r_1)&=t(yt^{-1}+z)^3+(yt^{-1}+z)^2+(yt^{-1}+z)+t\\
        &=\un{y^3t^{-2}}+3y^2zt^{-1}+3yz^2+z^3t+\un{y^2t^{-2}}+2yzt^{-1}+z^2+yt^{-1}+z+t
    \end{align*}
    Extracting the coefficients we get $y^3+y^2=0$ which means that $y=-1$. Plugging this back into our expression as $y$ we get 
    $$3zt^{-1}-3z^2+z^3t-2zt^{-1}+z^2-t^{-1}+z+t=tz^3-2z^2+(t^{-1}+1)z+(-t^{-1}+t).$$
    Tropicalizing (\red{is it actually or is it the reverse operation?}) we get 
    $$1\.z^3+z^2+(-1)z+(-1)$$
    which has as a root $1>-1$ So 
    $$z=y+z_1\word{with}y\in\bC,\quad z_1\in\bC\set{\set{t}}.$$  
    \begin{figure}[h!]
        \centering
        \includegraphics[width=0.5\textwidth]{figs/fig5-2EndOfProofFiniteCase.png}
        \caption{I don't know what this is}
        \label{fig:5.2-EndOfProofFiniteCase}
    \end{figure}
    \red{ASK MAPLE CODE}
\end{Ex} 

The question now is: how do we turn this idea into a formal proof?
\begin{enumerate}[i.]
    \item We do one root at a time, starting with the rightmost one.
    \item Observe that if $r$ is a tropical root and $\al=yt^r$ with $y$ chosen so cancellation happens, then denoting $\tilde{q}$, $q$ without the $x^0$ term:
    $$\Trop(q(x+\al))>\Trop(\tilde{q})\oplus\Trop(q(\al)).$$
    \item Finally we iterate and check that the sequence of $r_i$'s goes to $\infty$.
\end{enumerate}

\subsection{Combinatorialization of Root Finding}

We will be using the $\max$ convention now. So let us consider $p(x)=\sum_{k=0}^da_kx^k$. Can we a systematic and simple way to say how many roots, with what multiplicity, and what equations to solve?\par 
The left-most root can be found via
$$\min\left(\frac{a_0-a_k}{k}\right)=\text{achieved by }k\text{ such that }\frac{a_0-a_k}{k}\text{ is maximized}.$$
In other words we are looking for the largest slope:
\begin{figure}[h!]
    \centering
    \includegraphics[width=0.5\textwidth]{figs/fig6-1BiggestSlope.png}
    \caption{Difference of coefficients as slopes}
    \label{fig:6.1-BiggestSlope}
\end{figure}
We may repeat this argument for the following roots to get the following algorithm:
\begin{enumerate}[i.]
    \item Let $p_k=(k,a_k)\in\bonj{0,d}\x\set{-\infty}\cup\bR$.
    \item Now $\Sg$ is the convex hull of the points $\set{p_k\:\ k\in[d]}$. We may divide the region into $\Sg^+$ and $\Sg^-$.
    \item Call $q_i=\pi(p_i)$ for $p_i$'s that for the vertices of $\Sg^+$.
\end{enumerate}
The roots will be in bijection with the connected components of $\bonj{0,d}\less\set{q_i}_{i\in I}$ and the multiplicity is the length of the segment.

\begin{Ex}
    Take for example the polynomial 
    $$p(x)=0+1\.x+1\.x^2+x^3+2\.x^4+1\.x^5.$$
    We now place the points in our diagram and project:
    \begin{figure}[h!]
        \centering
        \includegraphics[width=0.5\textwidth]{figs/fig6-2CombinatorializationExample.png}
        \caption{Root finding for $p(x)$}
        \label{fig:6.2-CombinatorializationExample}
    \end{figure}
    From this we deduce that there are $2$ simple roots and $1$ triple root. This come from the equations
    $$
    \left\lbrace
    \begin{aligned}
        &0=x+1&\To x=-1\\
        &x+1=2+4x&\To x=-1/3\\
        &2+4x=1+5x&\To x=1
    \end{aligned}
    \right.
    $$
\end{Ex}

\subsection{Gr\"obner Complexes}

If $K$ is a field with a valuation, then call
$$
\left\lbrace
\begin{aligned}
    &R_K\subseteq K=\text{ elements with non-negative valuation}\\
    &\lie{m}\subseteq R_K=\text{ elements with positive valuation}
\end{aligned}
\right.
$$
so $\quot{R_K}{\lie{m}}$ is a residue field. In the case of tropical polynomials, they form a Gr\"obner complex\footnote{What are Gr\"obner comlpexes? To see in interim.}.
$$\lie{m}=\bigcup_nt^{1/n}\bC\bonj{\bonj{t^{1/n}}}\subseteq R_K=\bigcup_n\bC\bonj{\bonj{t^{1/n}}}\subseteq\bC\set{\set{t}},\word{and}\quot{R_K}{\lie{m}}=\bC.$$

\begin{Def}
    Given $q\in K[x]$ and $w\in\bT$, the \term{initial form} of $q(x)$ with respect to $w$ is a polynomial in $k[x]$that records the part of $q$ that has lowest order when $\val(x)=w$.
\end{Def}

\begin{Ex}
    Let us consider the polynomial 
    $$q(x)=t^{-4}+\sqrt{2}x+3t^2x^2,$$
    Here\footnote{What does this mean?}
    $$t^{-4}\to -4,\quad \sqrt{2}x\to-3,\quad 3t^2x^2\to -4,\word{so}w=-3\footnote{I srsly don't understand}.$$
    We may construct the initial form as $I_wq=1+3x^2$ but formally this is $\bonj{t^4(q(t^{-3}x))}_{t=0}$ and in general if $W=\Trop q(w)$ then 
    $$I_wq=\bonj{t^{-W}(q(t^{w}x))}_{t=0}.$$
\end{Ex}

\subsubsection{Gr\"obner Complex of $q(x)$}

Polyhedral decomposition of $\bR$ (in the case of a valuation space if we want, we can also add in $\infty$ but it usually is left out.) induced by the equivalence relation
$$w_1\sim w_2\iff In_{w_1}q=In_{w_2}q\footnote{Does this refer to initial form?}$$

\begin{Ex}
    Consider the polynomial 
    $$t^2+\sqrt2x+3t^2x^2$$
    and each monomial maps\footnote{Through what? The valuation?} to $2,w$ and $2+2w$ respectively.
    \begin{figure}[h!]
        \centering
        \includegraphics[width=0.5\textwidth]{figs/fig6-3-InitialFormExample.png}
        \caption{Initial form determination and roots}
        \label{fig:6.3-InitialFormExample}
    \end{figure}
    So the tropical roots are the locus where the initial form is not a monomial.
\end{Ex}
\section{Day 7|20230906}

\subsection{Our First Correspondence Theorem}

\begin{Def}
    Given a family of polynomials 
    $$q_t=\sum A_k(t)x^k\in\bC[x]\word{with}t>1$$
    then the \term{tropicalization} of $q_t$ is 
    $$\Trop(q_t)(x)=\sum a_k\odot x^{\odot k},\word{where}a_k=\lim_{t\to\infty}T_t(A_k).$$
    We may also use the $\min$ convention by exchanging the field to Puiseux series and $T_t$ by the valuation.
\end{Def}

\begin{Th}[Correspondence]
For a polynomial $q_t$, $R_t$ is a root of $q_t$ if and only if $\Trop(R_t)=\lim_{t\to\infty}T_t(R_t)$ is a root of $\Trop(q_t)$.
\end{Th}

This is saying that we have an object in algebraic geometry, a polynomial. Tropical geometry will somehow knowing about its roots by degenerating it. Then its easy to find the tropical roots and then there must be certain algebraic roots which should map to them. It may not be easy to understand this last map but at least we have some qualitative information.\par 
We will use the fundamental theorem of algebra to reduce to the linear case. So the first step is to prove the theorem for the case of linear polynomials. We have a couple of lemmas to finish the proof and expand it to the general case:

\begin{Lem}
    $\Trop$ is a multiplicative function on polynomials. That is
    $$\Trop(pq)=\Trop(p)\odot\Trop(q)\word{for}p,q\in\bC[x].$$
    \end{Lem}
    
This first lemma doesn't add anything weird because the tropical product is just the usual addition.

    \begin{Lem}
    The roots of $\Trop(p)\odot\Trop(q)$ are the union of the roots of the factors. If a root is repeated then the multiplicities are added.
    \end{Lem}

    Essentially what this is saying is that if we have two piecewise linear functions which change slope at the same place, then the sum will also change slope at the same place. As the functions are convex, a root can never be cancelled. Except possibly $-\infty$.

    \subsection{Higher Dimension}

    We will go back to the Puiseux series convention now:
    $$P(X)\in\bC\set{\set{t}}\bonj{X},\ P(R)=0\iff \Trop(P)(\val(R))=0.$$
    The easy direction is to begin with a root of our Puiseux polynomial. Let 
    $$P(X)=\sum A_i(t)X^i,\word{and}\Trop(P)(X)\sum a_i\odot x^i$$
    where $a_i=\val(A_i)$. Let $R=R(t)$ be a root of $P(X)$.\par 
    We know $\val(P(R))=\infty$ because $P(R)=0$. Formally $\val(P(R))$ should greater or equal than the minimum of the valuation of each of the monomials evaluated at $R$. In other words 
    $$\min(\val(A_i(t)R^i))=\min_i(a_i+i\val(R))=Trop(P)(R).$$
    Since we know that strict inequality holds, the terms in the formal evaluation with lowest order must cancel, in other words, the minimum is attained at least twice by two different monomials.\par 
    Last week we mentioned attaining the minimum twice is the same as being a root.

    \begin{Ex}
        Consider the polynomial $(t^2+7t^3)X+(t^5+t^{27})=Q(X)$. The root here is $R=-\frac{t^5+t^{27}}{t^2+7t^3}$ and its valuation is $5-2=3$. If we plug in something of this form instead of $X$ we get 
        $$Q(-t^3+O(t^4))=(t^2+7t^3)(-t^3+O(t^4))+(t^5+t^{27})=(-t^5)+t^5+O(t^6)$$
        In particular the first thing that will cancel is the lowest order term: $t^5$. So \emph{two} monomials must have lower order term.
    \end{Ex}

\section{Day 10|20230913}

The next question is if this process makes sense if we instead begin with a Puiseux series polynomial. If the process ends up being the same, does this mean that tropical geometry over a trivial valued field is uninteresting? That's not the case, it's only because we are in dimension zero. 

\subsection{Gr\"obner Complexes}

These types of complexes arise in commutative algebra. The setup begins with a valuated field, in our case Puiseux series $\bC\set{\set{t}}$. We can find the ring of integers, the positive valuated elements, in our field. These types of functions are regular at $t=0$. Inside this ring we have the maximal ideal of functions which vanish at zero. If we wish we can take a quotient to find the residue field which is a copy of $\bC$.\par 
Everytime we are given the data of polynomial $q$ in $\bC\set{\set{t}}\bonj{x}$ plus a choice of a valuation, we can recover the initial form of $q$ which is a polynomial with coefficients in the residue field.\par 
The way to find it is to look at the valuation of each monomial assuming $\val(x)=w$ and then save only the monomials with the smallest valuation and only keep the coefficient in front of the smallest term.

\begin{Ex}
    Consider the polynomial 
    $$q(x)=t^{-4}+t^{2}+\sqrt{2}x+3t^2x^2$$
    and take $w=-3$. This means that $\val(x)=-3$. Let us now consider the valuation monomial by monomial:\par 
    The term $(t^{-4}+t^{2})$ has valuation $-4$ because there's no $x$, next for $\sqrt{2}x$ we have
    $$\val(\sqrt{2}x)=\val(\sqrt{2})+\val(x)=0+(-3)=-3$$
    so it has valuation $-3$ and $3t^2x^2$ has valuation $2-6=-4$. We now consider only the first and last terms as they have the smallest valuation and extract the coefficients of the smallest terms. In the case of $t^{-4}+t^{2}$ its the $1$ accompanying the $t^{-4}$ and a $3$ accompanying the last term. So the initial form is 
    $$\operatorname{In}_{-3}(q)=1+3x^2.$$
\end{Ex}

\red{FORMULA for initial form}\par 
We now define an equivalence relation over $(\bR,w)$: 
$$w_1\sim w_2\iff In_{w_1}q=In_{w_2}q$$ 
which separates $\bR$ into two types of equivalence classes:
\begin{itemize}
    \item Single points in which the initial form is not a monomial.
    \item Open intervals where the initial is a monomial.
\end{itemize}

\begin{Ex}
    Consider the polynomial 
    $$t^2+\sqrt2x+3t^2x^2$$
    and each monomial maps\footnote{Through what? The valuation?} to $2,w$ and $2+2w$ respectively.
    \begin{figure}[h!]
        \centering
        \includegraphics[width=0.5\textwidth]{figs/fig6-3-InitialFormExample.png}
        \caption{Initial form determination and roots}
        \label{fig:6.3-InitialFormExample}
    \end{figure}
    So the tropical roots are the locus where the initial form is not a monomial.
\end{Ex}

\begin{Def}
    The complement of the locus where the initial form is a monomial is called the \term{Gr\"obner complex} of $q(x)$.
\end{Def}

The Gr\"obner complex of $q(x)$ is equal to the roots of $\Trop(q)(x)$. This is indeed in correspondence with Gr\"obner basis, which is very interesting in higher dimension. 

\subsection{1-dimensional Tropical Geometry}

If we have $p(x,y)$ a tropical polynomial in two variables, then we can define its tropical variety to be $V(p)$:
\begin{itemize}
    \item The locus in the domain where the piecewise linear function where $p$ is not linear.
    \item The locus of points $(x,y)$ where the $\max$ associated to each monomial is obtained more than once.
\end{itemize}

We will have a correspondence theorem which says that if $q(x,y)$ is a polynomial with coefficients over a valued field and the tropicalization of $q$ is $p$, then 
$$V(p)=\ov{\set{(\val(x),\val(y))\: (x,y)\in V(q)}}.$$
\begin{Ej}
Show that pairs of rational numbers are dense here. \aside{It has to do with the valuation only taking rational numbers.}
\end{Ej}
In two dimensions we have way more features, the study of tropical curves will enclose the correspondence statement with subdivisions of Newton Polygon and balancing edge weights. Our objective now is to see the tropical versions of tropical curve theorems. For example, the tropical Bezout and tropical degree/genus formula.

\section{Day 11|20230915}

\subsection{Tropical Lines}

If we have a tropical polynomial of degree 1, 
$$p(x,y)=a\odot x\oplus b\odot y\oplus c$$
and assume for the sake of drawing pictures, that $-\infty,a,b,c$. This corresponds to the piecewise linear function 
$$\max(a+x,b+y,c)$$
and if we set any of these two equations equal to each other, we can see that there are three lines that play a role:
\begin{align*}
    &a+x=b+y\To y=x+(a-b)\\
    &a+x=c\To x=c-a\\
    &b+y=c\To y=c-b
\end{align*}
So this is the locus where two functions are equal to each other. In each of the regions the maximum is attained by a particular linear function, the boundary between them is the locus of non-linearity. The point in the middle is $(c-a,c-b)$. 

\begin{figure}[h!]
    \centering
    \includegraphics[width=0.5\textwidth]{figs/fig7-1-TropicalLineExample.png}
    \caption{Graph of $p(x,y)=0$ in $\bR^2$}
    \label{fig:7.1-TropicalLineExample}
\end{figure}

So in general, tropical lines look like this ``tripod'' and changing the $a,b,c$ shifts the graph.

\begin{Ej}[2]
Figure out what happens when a coefficient is $-\infty$.
\end{Ej}

\subsection{The Case of Puiseux Series}

In this case, lines will we the zero loci of polynomials of the form 
$$p(X,Y)=A(t)X+B(t)Y+C(t)$$
with 
$$a=\val(A),\quad b=\val(B)\word{and}c=\val(C).$$
We let $L=\set{(X,Y)\in\bC\set{\set{y}}^2\:\ p(X,Y)=0}$ be the zero locus and then define
$$\Trop(L)=\ov{\set{(\val(X),\val(Y))\:\ (X,Y)\in L}}\subseteq\bT^2.$$
We may parametrize $p$ in the following way, we let $X=\ga(t)$ with an arbitrary valuation and then we solve for $Y$:
$$Y=\underbrace{\frac{-A(t)}{B(t)}\ga(t)}_{a-b+\val\ga}-\underbrace{\frac{C(t)}{B(t)}}_{c-b}$$ 
\begin{Ex}
    Consider the same polynomial in terms of Puiseux series:
    $$q(x,y)=t^ax+t^by+t^c$$
    whose tropicalization is $p(x,y)$. \aside{Observe that it doesn't necessarily have to be $t^a$ as a coefficient, it can be any Puiseux series with valuation $a$.}\par 
    We have that $V(q)=\set{q=0}$ and we can parametrize it as 
    $$x=\al,\quad y=-t^{a-b}\al-t^{c-b},\word{where}\al\in\bK^\ast.$$
    Now for any of these points we are going to take their valuation. Specifically we are looking at the set 
    $$\set{\val(\al,-t^{a-b}\al-t^{c-b})}$$
    $\al$ can have any valuation we want and depending on that, we decide the valuation of the binomial. The valuations are 
    $$a-b+\val(\al)\word{and}c-b$$
    which are equal when $\val(\al)=c-a$.
    \begin{itemize}
        \item What happens if $\val(\al)<c-a$ then $-t^{a-b}$
        \item Something I fell asleep 
    \end{itemize}
    \textbf{Claim: We can obtain any value for $y$ but is to be greater than $c-b$.}
    Let $r\geq 0$ and $\al=-t^{c-a}(1+t^r)$
\end{Ex}

\subsection{The Amoeba Perspective}

Recall that what matters the most is the logarithm base $t$ of our function. So let us continue with 
$$q(x,y)=t^ax+t^by+t^c$$
and play the same as before. Look for solutions to the equation $q_t=0$ in $\bC^2$ which is a line intersecting the $x$ axis at $-t^{c-a}$ and the $y$ at $-t^{c-b}$. Every pair of points $(x,y)$, gives us a pair $(\log_t|x|,\log_t|y|)$. The real trace of this, when $x,y\in\bR$ can be parametrized with $x=t^\al$ and $y=-t^{a-b+\al}$. We analyze the trace in three intervals, 

\section{Day 12|20230918}

Our goal is to understand the image of the line 
$$L_t=\set{t^ax+t^by-t^c=0}\subseteq\bC^2$$
via the map $(x,y)\mapsto (\log_t|x|,\log_t|y|)$. The line $L_t$ has three sections, where both $x,y$ are positive and one section corresponding to each $x$ and $y$ being negative. For ease of calculation we may solve the equation as $y=-t^{a-b}x+t^{c-b}$\par 
Let us consider the case where $x,y$ are both positive. We can see that $0<x<t^{c-a}$, this traces an $x$ in the parameter space such that $-\infty<x<c-a$. Via the solution for $y$ we may write 
$$\log_t|y|=\log_t(t^{c-b}-t^{a-b+x})$$
where we have solved the equation for $x$ which is why we have an $x$ exponent and $y$ is positive as we have assumed. We can simplify this as 
$$\log_t\bonj{t^{c-b}\left(1-t^{a-c+x}\right)}=(c-b)+\log_t(1-t^{a-c+x}).$$
This can be traced as a function of $t$ and in particular
$$\lim_{x\to-\infty}(c-b)+\log_t(1-t^{a-c+x})=c-b\word{and}\lim_{x\to(c-a)^{-}}(c-b)+\log_t(1-t^{a-c+x})=-\infty.$$
With this information we see two asymptotes for our function, $y=c-b$ and $x=c-a$.

\subsection{Arbitrary Degree $d$}

Recall that for a polynomial $q\in\bK[x,y]$, we may describe its algebraic variety in $\bK^2$. We may think of that field as Puiseux series. Along it, we may tropicalize it to $p$ and we get its tropical hypersurface, the set of non-linearity.\par
Kapranov's theorem allows us to see a correspondence as follows:
$$\ov{\Trop(V(q))}=V(\Trop(q)).$$
Left-to-right is still the same idea as the correspondence theorem. If $(x_0,y_0)\in\Trop(V(q))$ then, there exists $(X_0,Y_0)\in\bK^2$ such that $\val(X_0)=x_0,\ \val(Y_0)=y_0$ and $q(X_0,Y_0)=0$. Let 
$$q=\sum a_{ij}X^iY^j$$
If we call $m_{ij}$ each monomial, then $\set{m_{ij}(X_0,Y_0)}_{i,j}$ is a set of elements of $\bK^\ast$ with the property that their sum is zero. Now call 
$$\mu=\min\set{m_{ij}(X_0,Y_0)}_{i,j}$$
we claim that there are at least two monomials whose valuation is $\mu$. If there was only one monomial with valuation $\mu$, then that power of $\mu$ \emph{cannot} be cancelled. This means that $(x_0,y_0) $ is in $V(p)$. Now we use minimality of closure and we are done.\par 
Now the harder direction will use the fact that we have proven this in dimension zero and proceed by induction. First we want to show that $V(\Trop(q))\cap\bQ^2$ is dense in $V(\Trop(q))$. This is true because all monomials $m_{ij}$ correspond to all linear functions with integer slopes of rational coefficients.
$$a_{ij}X^iY^j=\Trop(m_{ij})=\val(a_{ij}\odot x^{i}\odot y^j)=\val(a_{ij})+ix+jy$$
It suffices to check that \red{ERASED TOO QUICK}\par 
Now we wish to proceed by induction. For example a polynomial $q(X,Y)$ can be seen as 
$$q(X,Y)=r_0(X)+r_1(X)Y+\dots+r_d(X)Y^d\word{with}r_i(X)\in\bK[X].$$
We do not lose generality when assuming that all $r_i$'s are monomials\footnote{to see next time}. So we have $(x_0,y_0)\in\in V(\Trop(q))$ and we want to find $(X_0,Y_0)\in(\bK^\ast)^2$ such that 
$$q(X_0,Y_0)=0\word{and}(\val(X_0),\val(Y_0))=(x_0,y_0).$$
Choose $X_0$, however we want as long as we have the valuation condition. Given our assumption, this implies that $r_i(X_0)$ is non-zero for all $i$. Now consider the polynomial 
$$q(X_0,Y)=\sum r_i(X_0)Y^i\in\bK[Y]$$
and its tropicalization
$\tilde{p}(y)=\Trop(q(X_0,Y))=\sum\val(r_i(X_0))y^i=\min(\val R_i(X_0)+iy)=\min$
They are hidden in terms of unknown,

\section{Day 13|20230920}

\begin{Th}
$V(\Trop(q))=\ov{\Trop(V(q))}$
\end{Th}

If we start with a polynomial in valued field, we can tropicalize or look at ... and then take the image of coordinates of points and back in $\bR^2$ then take closure, we end in the same place. The fact that every point of the algebraic curve  lies somewhere is an argument of cancellation of Lowest Order Terms. In particular when plugging the value for the Puiseux solution two terms must cancel. 
\begin{ptcbp}
We have shown that right-to-left is easy, cancellation of L.O.T.\par 
The other direction is trickier, it's a lifting problem. Given $(x_0,y_0)\in V(\Trop(q))\subseteq\bR^2$, then we must find 
$$(X_0,Y_0)\in V(q)\subseteq \bK^{\ast2},\quad\val(X_0)=x_0\quad\val(Y_0)=y_0$$
If we write $q$ then we will assume that we can write 
$$q(X,Y)=\sum r_i(X)Y^i,\quad r_i(X)\ \text{monomials}$$
If we first plug in $X=X_0$ (which is any Puiseux series we want with valuation $x_0$ [We have picked such $X_0$]), 
$$q(X_0,Y)=\sum r_i(X_0)Y^i$$
is a polynomial in $Y$ with Puiseux series coefficients, now tropicalize this $q$ we get 
$$\tilde{p}(y)=\sum\val r_i(X_0)y^i\quad\text{(tropical sum and product now)}.$$
We claim that $y_0$ is a root of $\tilde{p}(y)$. HEre's where we are using the monomial assumption.\par 
What is the linear function associated to $\tilde{p}(y)$:
$$\tilde{p}(y)=\min(\val r_i(X_0)+iy)$$
and as $r_i$ is monomial, call it $r_i(X_0)=A_{ij}X^j$ where $A_{ij}$ is a Puiseux series. So this $\tilde{p}$ becomes:
$$\tilde{p}(y)=\min(\val(A_{ij})+jx_0+iy)$$
which is exactly the tropicalization of $q(x,y)$ and plug in $x_0$. This is a univariate polynomial, which allows to apply the univariate case. So there exists a $Y_0$, Puiseux series, such that $Y_0$ is a root of $q(X_0,Y)$ with $\val(Y_0)=y_0$.\par 
It remains to see that our monomial condition is not a restriction. 
\end{ptcbp}

This allows us not only to lift, but to pick one coordinate freely and then the other one is determined!

\begin{Ex}
    Consider the polynomial 
    $$q(X,Y)=XY+X^2Y=(X+X^2)Y,\word{and}\tilde{q}(X,Y)=q(XY,Y)=XY^2+X^2Y^3$$
    and $\tilde{q}$ does satisfy the previous assumption. If $(\tilde{X}_0,\tilde Y_0)$ is a solution to our problem for $\tilde{q}=0$, then $\left(\frac{\tilde{X}_0}{\tilde Y_0},\tilde Y_0\right)$ is a solution for $q=0$.\par 
    The key point is that $\tilde q$ is obtained $q$ by an invertible transformation in the torus $(\bK^\ast)^2$.
\end{Ex}

\begin{ptcb}
Given $q(X,Y)$ of degree $d$, then picking 
$$\tilde{q}(X,Y)=q(XY,Y^{d+1})$$
satisfies the monomials assumption. This is because \emph{we are giving enough space}.
$$q(X,Y)=\sum r_{ij}X^iY^j\To \tilde{q}(X,Y)=\sum r_{ij}X^iY^{(d+1)j+i}$$
where if we wished to find \dots then 
$$(d+1)j_1+i_1=(d+1)j_2+i_2\To (d+1)(j_1-j_2)=i_2-i_1$$
$j_1-j_2\geq d+1$ when $j_1=j_2$ and $i_2-i_1\leq d$.
\end{ptcb}

\begin{Ex}
    Compute $V(p)$ for the following polynomials
    \begin{itemize}
        \item $p_1=0+x+y+xy$
        \item $p_2=0+x+y-xy$
        \item $p_3=0-x-y+xy$
    \end{itemize}
    For each of this polynomials there are $\binom{4}{2}=6$ line possibilities so we check each one.
\end{Ex}


%%%%%%%%%%%% Contents end %%%%%%%%%%%%%%%%

\ifx\nextra\undefined
\printindex
\else\fi
\nocite{*}
\bibliographystyle{plain}
\bibliography{bibiTropiGeo.bib}
\end{document} 

