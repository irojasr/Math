\documentclass[12pt]{memoir}

\def\nsemestre {II}
\def\nterm {Fall}
\def\nyear {2023}
\def\nprofesor {Renzo Cavalieri}
\def\nsigla {MATH676}
\def\nsiglahead {Tropical Geometry}
\def\nlang {ENG}
%\def\darktheme{}
\input{../../headerVarillyDiff}
\usepackage[enableskew]{youngtab}
\DeclareMathOperator{\ins}{ins}
\DeclareMathOperator{\rw}{rw}
\DeclareMathOperator{\rect}{rect}
\DeclareMathOperator{\sh}{sh}
\DeclareMathOperator{\std}{std}
\DeclareMathOperator{\Frob}{Frob}
\DeclareMathOperator{\val}{val}
\DeclareMathOperator{\Trop}{Trop}
\newcommand{\diamondplus}{\mathbin{\rotatebox[origin=c]{45}{$\boxtimes$}}}
%\newcommand{\rplus}{\rotatebox[origin=c]{45}{\boxplus}}
\begin{document}
%\clearpage
\maketitle
%\thispagestyle{empty}
{\small 
\setlength{\parindent}{0em}
\setlength{\parskip}{1em}

This is a topics course on this stuff

\subsubsection*{Requirements}
Knowledge on stuff
}
\newpage
\tableofcontents
%\begin{multicols}{2}
\chapter{Combinatorial Shadow of Algebraic Geometry}

\section{Day 1|20230821}

Think of an algorithm where the input is an algebraic variety and the output is a combinatorial object, a piecewise linear object.

\begin{Ex}
    Consider as an input a line in the plane. Say $V(x+y-1)$, then an output would be a tropical line. If we remain in the plane and consider a higher degree polynomial, say an elliptic curve, as an output we obtain a tropical cubic.\par 
    Leaving the plane behind and thinking of abstract nodal curves, we can think of a sphere attached to a torus which is attached to a genus 2 torus, then the corresponding object is what we call the \term{dual graph}.
\end{Ex}

Right now we do not know the specific algorithm, but we can observe that the outputs are \emph{more simple} than the inputs. So the important question is:
\begin{significant}
What algebraic information does the simplified object remember? How do we extract the information the object remembers? And once we know how to work with this objects, can we return to algebraic geometry from any kind of these objects?
\end{significant}

Observe that the number of ends which go to infinity corresponds with the degree. 

\section{Interim}
%%valued fields

There are many ways to obtain the previous combinatorial pictures:

\subsection{Algebraic Geometry on $\bT$}

We call $(\bT,\oplus,\odot)$ the \term{tropical semifield}. A polynomial $p(\un x)$ gives rise to its \emph{tropicalization}, a piecewise linear function 
$$\Trop(p)\:\bR^n\to\bR.$$
The \term{tropical hypersurface} $V(\Trop(p))$ is the codimension 1 locus in $\bR^n$ where the function is non-linear (corner locus).

\subsection{Valued fields}

\begin{Def}
The field of \term{Puiseux series} over $\bC$ is $\bC\set{\set{t}}$ where the elements are of the form 
$$\al(t)=\sum_{i=k_0}^\infty a_it^{i/n}.$$
The lower bound $k_0$ could be negative and the exponents, as in the case of Laurent series, are rational with bounded denominators.\par 
The \term{valuation} of $\al(t)$ is $\val(\al(t))=\frac{k_0}{n}$, it is the smallest exponent for the natural order of the rational numbers and the coefficient $a_{k_0}$ is called the valuation coefficient.
\end{Def}

If $K$ is the field of Puiseux series, then for a polynomial $p\in K\bonj{\un x}$,
\begin{figure}[h!]
    
\centering
\tikzset{every picture/.style={line width=0.75pt}} %set default line width to 0.75pt        

\begin{tikzpicture}[x=0.75pt,y=0.75pt,yscale=-1,xscale=1]
%uncomment if require: \path (0,300); %set diagram left start at 0, and has height of 300

%Straight Lines [id:da2454102694181276] 
\draw    (235,63) -- (321,63) ;
\draw [shift={(323,63)}, rotate = 180] [color={rgb, 255:red, 0; green, 0; blue, 0 }  ][line width=0.75]    (10.93,-3.29) .. controls (6.95,-1.4) and (3.31,-0.3) .. (0,0) .. controls (3.31,0.3) and (6.95,1.4) .. (10.93,3.29)   ;
%Straight Lines [id:da5779705315012776] 
\draw    (241,113) -- (298,113) ;
\draw [shift={(300,113)}, rotate = 180] [color={rgb, 255:red, 0; green, 0; blue, 0 }  ][line width=0.75]    (10.93,-3.29) .. controls (6.95,-1.4) and (3.31,-0.3) .. (0,0) .. controls (3.31,0.3) and (6.95,1.4) .. (10.93,3.29)   ;
\draw [shift={(241,113)}, rotate = 180] [color={rgb, 255:red, 0; green, 0; blue, 0 }  ][line width=0.75]    (0,5.59) -- (0,-5.59)   ;

% Text Node
\draw (208,53.4) node [anchor=north west][inner sep=0.75pt]    {$K^{n}$};
% Text Node
\draw (329,53.4) node [anchor=north west][inner sep=0.75pt]    {$\mathbb{R}^{n}$};
% Text Node
\draw (202,103.4) node [anchor=north west][inner sep=0.75pt]    {$V( p)$};
% Text Node
\draw (304,99.4) node [anchor=north west][inner sep=0.75pt]    {$\overline{\operatorname{Val}( V( p))}$};
% Text Node
\draw (172,153.4) node [anchor=north west][inner sep=0.75pt]    {$\{\vec{x} :p(\vec{x}) =0\}$};
% Text Node
\draw (302,153.4) node [anchor=north west][inner sep=0.75pt]    {$\operatorname{Trop}( V( p))$};
% Text Node
\draw (210.4,98) node [anchor=north west][inner sep=0.75pt]  [rotate=-270]  {$\subseteq $};
% Text Node
\draw (328.4,98) node [anchor=north west][inner sep=0.75pt]  [rotate=-270]  {$\subseteq $};
% Text Node
\draw (229.6,127) node [anchor=north west][inner sep=0.75pt]  [rotate=-90]  {$=$};
% Text Node
\draw (346.6,127) node [anchor=north west][inner sep=0.75pt]  [rotate=-90]  {$=$};
% Text Node
\draw (172,53.4) node [anchor=north west][inner sep=0.75pt]    {$\operatorname{Val:}$};


\end{tikzpicture}

\end{figure}
and here $\Trop(V(p))$ is the tropical hypersurface for $p$.

\subsection{Amoebas}

Suppose now $p\in\bC[\un x]$ and $V(p)\subseteq\bC^n$. We may consider the function 
$$\log_t\:\bC^n\to\bR^n,\quad (z_1,\dots,z_n)\mapsto(\log_t|z_1|,\dots,\log_t|z_n|).$$
The amoeba is $\log_t(V(p))\subseteq\bR^n$. 

\begin{Ex}
    When $p=x+y-1$ then \red{we have a figure which I can't draw}.
\end{Ex}

Observe that this is the tropical hypersurface associated to $p$. In other words we have that the tropical hypersurface is $\lim_{t\to\infty}\log_t(V(p))$.

\subsection{Degenerations}

Consider a family of \red{of what, what is this family of?! Stuff? Curve in P1xP1 which eventually becomes P2?}
\begin{figure}[h!]
    \centering
    \includegraphics[width=0.5\textwidth]{figs/fig1.1.png}
\end{figure}

It is too early to understand this point of view. We will set everything up to get to it.\par 
In general, the big idea will be to explore and understand these perspectives in the case of plane curves. We want to show how they are equivalent and then recover classical algebraic geometry results in terms of tropical geometry.

\subsection{Tropical Arithmetics}%%%Based on TropicalNumbers.pdf

\subsubsection{Minimizing Tolls}

Consider a set of cities connected by a network of toll-ways:
\begin{figure}[h!]
    \centering
    \includegraphics[width=0.5\textwidth]{figs/fig1.2.png}
\end{figure}
If we only care about minimizing toll expenses, what would be the cheapest way to go from one given city to another? Let us record the information as an incidence matrix. 
$$M=\threebythree{0}{\infty}{2}{x}{0}{y}{\infty}{1}{0}$$
In this matrix, the rows determine the outbound city, while the columns are the destination. Each entry records the cost of a toll and tolls are considered to be infinite when the road does not exist. We can also think of $M$ as recording the cheapest toll to go from one city to another with at most one move.\par 
But if we wanted to find the cheapest way from one city to another in \textbf{two} moves, we could use $M^2$ with standard matrix multiplication. However we must redefine our basic operations as follows:
$$+=\min,\quad\.=+$$
\begin{align*}
    \threebythree{0}{\infty}{2}{x}{0}{y}{\infty}{1}{0}^2&=\threebythree{\min\threebyone{0+0}{\infty+x}{2+\infty}}{\min\threebyone{0+\infty}{\infty+0}{2+1}}{\min\threebyone{0+2}{\infty+y}{2+0}}{\min\threebyone{x+0}{0+x}{y+\infty}}{\min\threebyone{x+\infty}{0+0}{y+1}}{\min\threebyone{x+2}{0+y}{y+0}}{\min\threebyone{\infty+0}{1+x}{0+\infty}}{\min\threebyone{\infty+\infty}{1+0}{0+1}}{\min\threebyone{\infty+2}{1+y}{0+0}}\\
    &=\threebythree{0}{3}{2}{x}{\min(0,y+1)}{\min(x+2,y)}{1+x}{1}{\min(0,1+y)}.
\end{align*}
Observe that $1+y$ can be the minimum in the diagonal when we allow \emph{negative tolls}.
\begin{Rmk}
If we disallow negative tolls, the products $M^n$ eventually stabilize to a matrix whose entries record the cheapest way to get from one city to another in $n$ steps.
\end{Rmk}
This gives us an intuition that minimization problems correspond to linear algebra problems over $(\bT,+,\.)$ which is precisely $(\bR\cup\set{\infty},\min,+)$.

\subsubsection{Forgetting phases}

Recall that any complex number can be written as $z=re^{i\te}$ where $r\geq 0$. Consider the map $T_t\:\bC\to\set{-\infty}\cup\bR,\quad z\mapsto\log_t(r)$.
\begin{figure}[h!]
    \centering
    \includegraphics[width=0.5\textwidth]{figs/fig1.3.png}
\end{figure}
This map is surjective, and this we can see by checking it is right-invertible. Observe that:
$$
\left\lbrace
\begin{aligned}
    &T_t^{-1}(x)=\set{t^xe^{i\te}}\subseteq\bC,\word{for}x\in\bR,\\
    &T_t^{-1}(-\infty)=0.
\end{aligned}
\right.
$$
With this in hand, we wish to define an exotic addition and multiplication on $\set{-\infty}\cup\bR$ using $T_t$. We will dequantize!\par 
We begin with \textbf{hyper-addition}, the output will be a subset of $\set{-\infty}$ so it's not a binary operation by itself. 
$$x\diamondplus_t y\:= T_(T_t^{-1}(x)+T_t^{-1}(y))=\bonj{\log_t(|t^x-t^y|),\log_t(t^x+t^y)}.$$
This is an interval in $\set{-\infty}\cup\bR$, in order to make $\diamondplus_t$ into an operation we take a limit:
\begin{figure}[h!] 
    \centering
\begin{tikzpicture}[x=0.75pt,y=0.75pt,yscale=-1,xscale=1]
%uncomment if require: \path (0,300); %set diagram left start at 0, and has height of 300

%Straight Lines [id:da5156276968518897] 
\draw    (85,62.6) -- (142,62.99) ;
\draw [shift={(144,63)}, rotate = 180.39] [color={rgb, 255:red, 0; green, 0; blue, 0 }  ][line width=0.75]    (10.93,-3.29) .. controls (6.95,-1.4) and (3.31,-0.3) .. (0,0) .. controls (3.31,0.3) and (6.95,1.4) .. (10.93,3.29)   ;

%Straight Lines [id:da8224691621679702] 
\draw    (96,133) -- (144,133.38) ;
\draw [shift={(146,133.4)}, rotate = 180.46] [color={rgb, 255:red, 0; green, 0; blue, 0 }  ][line width=0.75]    (10.93,-3.29) .. controls (6.95,-1.4) and (3.31,-0.3) .. (0,0) .. controls (3.31,0.3) and (6.95,1.4) .. (10.93,3.29)   ;
%Straight Lines [id:da27001319663870027] 
\draw    (65,77) -- (65,118) ;
\draw [shift={(65,120)}, rotate = 270] [color={rgb, 255:red, 0; green, 0; blue, 0 }  ][line width=0.75]    (10.93,-3.29) .. controls (6.95,-1.4) and (3.31,-0.3) .. (0,0) .. controls (3.31,0.3) and (6.95,1.4) .. (10.93,3.29)   ;
%Straight Lines [id:da8909494911629017] 
\draw    (195,83) -- (195,120) ;
\draw [shift={(195,122)}, rotate = 270] [color={rgb, 255:red, 0; green, 0; blue, 0 }  ][line width=0.75]    (10.93,-3.29) .. controls (6.95,-1.4) and (3.31,-0.3) .. (0,0) .. controls (3.31,0.3) and (6.95,1.4) .. (10.93,3.29)   ;

% Text Node
\draw (52,53.4) node [anchor=north west][inner sep=0.75pt]    {$x\diamondplus_t y$};
% Text Node
\draw (40,123.4) node [anchor=north west][inner sep=0.75pt]    {$x\ +_{t} \ y$};
% Text Node
\draw (152,53.4) node [anchor=north west][inner sep=0.75pt]    {$x\diamondplus y=\lim _{t\rightarrow \infty } x\diamondplus_t y$};
% Text Node
\draw (152,123.4) node [anchor=north west][inner sep=0.75pt]    {$x+y=\max( x,y)$};
% Text Node
\draw (67,98.5) node [anchor=west] [inner sep=0.75pt]  [font=\scriptsize]  {$\max$};
% Text Node
\draw (114.5,59.4) node [anchor=south] [inner sep=0.75pt]  [font=\scriptsize]  {$\lim _{t\rightarrow \infty }$};
% Text Node
\draw (121,129.8) node [anchor=south] [inner sep=0.75pt]  [font=\scriptsize]  {$\lim _{t\rightarrow \infty }$};
% Text Node
\draw (197,102.5) node [anchor=west] [inner sep=0.75pt]  [font=\scriptsize]  {$\max$};


\end{tikzpicture}

\end{figure}
\section{Day 2|20230823}
%%Based on 676-Intro
\subsection{Algebraic Geometry on $\bT$}
Let us talk about ways to get into tropical geometry. We will first define the tropical semifield which the base set over which we will do algebraic geometry.

\begin{Def}
    The \term{tropical semifield} is the set $(\bR\cup\set{-\infty})$ equipped with tropical addition and multiplication:
    $$
    \begin{cases}
        x\oplus y=\max(x,y)\\
        x\odot y=x+y
    \end{cases}
    $$
\end{Def}

With this set we can make multivariable polynomials 
$$p(\un x)\:\left(\bR\cup\set{-\infty}\right)^n\to\bR\cup\set{-\infty}$$
which gives rise to their \emph{tropicalization}, a piecewise linear function $\Trop(p)\:\bR^n\to\bR$.

\begin{Ex}
    Consider the polynomial 
    $$p(x,y)=x\oplus y\oplus 0,$$
    its tropicalization is $\Trop(p)(x,y)=\max(x,y,0)$ which indeed is a piecewise linear function from $\bR^2$ to $\bR$.
    \begin{figure}[h!]
        \centering
        \subcaptionbox{$x\oplus y\oplus 0$\label{fig:2.1-LinearTropicalPolynomial}}{\includegraphics[width=0.3\textwidth]{figs/fig2.1-LinearTropicalPolynomial.pdf}}\quad
        \subcaptionbox{Projection onto $xy$-plane\label{fig:2.2-ProjectionLinearTropicalPolynomial}}{\includegraphics[width=0.25\textwidth]{figs/fig2.2-LinearTropicalPolynomialProjected.pdf}}\quad
        \subcaptionbox{Corner locus\label{fig:2.3-CornerLocus}}{\includegraphics[width=0.25\textwidth]{figs/fig2.3-CornerLocus.pdf}}
        %\caption{This is the caption.}
        \label{fig:2.1-and-2.2-and-2.3}
    \end{figure}
    %https://mathematica.stackexchange.com/questions/169777/listplot3d-with-contours-projected-onto-the-xy-plane
    Observe that the surface is not smooth where the planes meet, this is what we will call the \emph{corner locus} or \emph{tropical hypersurface}.
    %In two variables we have a PICTURE. The polynomial $x\oplus y\oplus 0$ is actually $\max(x,y,0)$. This picture is actually the projection of the corner locus. In 3D we can visualize this better.
\end{Ex}

\begin{Def}
    A \term{tropical hypersurface} is the locus of non-linearity of a tropical polynomial.
\end{Def}

\begin{Ex}
If we consider higher degree tropical polynomials, they will become linear in the usual sense. Consider 
$$p(x)=3x^2=3\odot x\odot x=3+x+x=3+2x$$
which is indeed linear.
\end{Ex}
\begin{enumerate}

\item TOPOLISH
    
    \item In the field of rational functions consider the valuation 
    $$\val_0\:\bC(t)\to\bR\cup\set{\infty},\begin{cases}
        0\mapsto \infty\\
        f\mapsto\text{order of vanishing at }0.
    \end{cases}$$
    This order of vanishing is the value $\al$ such that $f/t^\al$ approaches a finite non-zero value.
    \begin{Ex}
        What happens to the order of vanishing when you add two functions? Consider $f=t^2,\ g=t^3$, then $f+g=t^2+t^3$ which has order of vanishing $2$. Observe that $2=\min(2,3)$.
    \end{Ex}
    We have that 
    $$\val(f_1+f_2)\geq\min(\val f_1,\val f_2),\word{and}\val_0(f_1f_2)=\val(f_1)+\val(f_2).$$
    We can do algebraic geometry over this field, let $p(\un x)\in K\bonj{\un x}$. Then the algebraic variety is $X=V(p)=\set{\vec{x}\:p(\vec{x})=0}\subseteq K^n$. Taking the image through the $n$-fold valuation, we will obtain a set in $\left(\bR\cup\set{\infty}\right)^n$. The tropicalization of $X$ is the image via this map.
    \begin{Ex}
        Consider the polynomial in $K\bonj{x,y}$ 
        $$p(x,y)=tx+y+t^2,$$
        then the variety is $X=\set{(x,y)\: tx+y+t^2=0}$ which we can solve to $y=-tx-t^2$.\par 
        If we choose $x=0$ then $y$ becomes $-t^2$. Now we take the valuation of $(0,-t^2)$ and so $(\infty,2)\in\Trop(X)$.
    \end{Ex}
    \item Amoebas let us consider polynomials $p\in\bC[\un x]$ which define an algebraic variety $X=V(p)\subseteq\bC^n$. We can now consider the map which sends every coordinate to the logarithm in base $t$ of its modulus:
    $$\bC^n\to\left(\bR\cup\set{-\infty}\right)^n,\quad (z_1,\dots,z_n)\to(\log_t|z_1|,\dots,\log_t|z_n|).$$
    The image of $X$ under this map is the $t$-amoeba of $X$. If we take the limit as $t\to\infty$ then we get the \emph{spine} of the amoeba. 
    \begin{Ex}
        For $p(x,y)=x+y+1$ the $t$-amoeba looks like PICTURE. If we take the limit as $t\to\infty$ we obtain the previous tropical variety!
    \end{Ex}
    \item We may parametrize any algebraic variety with a time variable, then converting the information to a graph, edges code the information about how fast the node forms related to the length.
\end{enumerate}

%%%%%%%%%%%% Contents end %%%%%%%%%%%%%%%%
\ifx\nextra\undefined
\printindex
\else\fi
\nocite{*}
\bibliographystyle{plain}
\bibliography{bibiTropiGeo.bib}
\end{document} 

