\documentclass[12pt]{memoir}

\def\nsemestre {II}
\def\nterm {Fall}
\def\nyear {2025}
\def\nprofesor {Ignacio Rojas}
\def\nsigla {MATH261}
\def\nsiglahead {Calculus 3}
\def\nextra {HW6}
\def\nlang {ENG}
\input{../../../headerVarillyDiff}


\begin{document}

\section*{Problem 18}
\textit{Repeat problem 17 but traverse the triangle clockwise.}\\
(Problem 17 had the vector field \(\mathbf F(x,y)=\left( 2xy^2,\;2x^2y\right)\) and \(C\) was the triangle with vertices \((0,0),(0,1),(1,0)\).)

We parametrize the paths between the vertices following the orientation
\[
    \begin{cases}
    r_1(t)=(0,t)\To r_1'(t)=(0,1)\\
    r_2(t)=(t,-t+1)\To r_2'(t)=(1,-1)\\
    r_3(t)=(-t+1,0)\To r_3'(t)=(-1,0)
\end{cases}
\]
where all paths have \(0\leq t\leq 1\). Evaluating \(F\) at each path we obtain
\[
    \begin{cases}
    F(r_1(t))=(0,0)\\
    F(r_2(t))=(2(t)(-t+1)^2,2(t)^2(-t+1))=(2 t^3 - 4 t^2 + 2 t,-2t^3+2t^2)\\
    F(r_3(t))=(0,0)
\end{cases}
\]
From this 
\begin{align*}
\oint_T F\.\dd\vec x &= \int_{L_1}F\.\dd\vec x +\int_{L_2}F\.\dd\vec x +\int_{L_3}F\.\dd\vec x \\
&= 0+\int_{0}^1F(r_2(t))\.r_2'(t)\dd t +0 \\
&= \int_{0}^1(2 t^3 - 4 t^2 + 2 t,-2t^3+2t^2)\.(1,-1)\dd t \\
&= \int_{0}^1(2 t^3 - 4 t^2 + 2 t)-(-2t^3+2t^2)\dd t \\
&= \int_{0}^1(4 t^3 - 6 t^2 + 2 t)\dd t \\
&=\frac44-\frac63+\frac22=0.
\end{align*}

Thus our desired integral is 
\[
\boxed{\oint_T F\.\dd\vec x = 0}
\]
\section*{Problem 19}
\textit{Determine whether the vector field \(\mathbf F(x,y)=3x^2\mathbf i + y^3\mathbf j\) is conservative. If it is, compute a potential function.}

Let \(\mathbf F=\left( P,Q\right)\) with \(P=3x^2\) and \(Q=y^3\). Via the conservative criterion, \(\mathbf F\) is conservative iff \(\partial P/\partial y=\partial Q/\partial x\).

Computing:
\[
\frac{\partial P}{\partial y}=0,\qquad
\frac{\partial Q}{\partial x}=0,
\]
so they are equal. Hence \(\mathbf F\) is conservative.

Find a potential \(\varphi(x,y)\) with \(\varphi_x=3x^2\) and \(\varphi_y=y^3\). Integrate \(\varphi_x\) w.r.t.\ \(x\):
\[
\varphi(x,y)=\int 3x^2\,\dd x = x^3 + g(y).
\]
Differentiate with respect to \(y\) and match \(\varphi_y\):
\[
\varphi_y(x,y)=g'(y)=y^3 \implies g(y)=\frac{y^4}{4}+C.
\]
Thus a potential function is
\[
\boxed{\displaystyle \varphi(x,y)=x^3+\frac{y^4}{4}+C.}
\]

\section*{Problem 21}
\textit{Determine whether the vector field \(\mathbf F(x,y)=y e^{x}\mathbf i + x e^{y}\mathbf j\) is conservative. If it is, compute a potential function.}

Let \(\mathbf F=\left( P,Q\right)\) with \(P=y e^{x}\) and \(Q=x e^{y}\). Computing the mixed partials:
\[
\frac{\partial P}{\partial y}=e^{x},\qquad
\frac{\partial Q}{\partial x}=e^{y}.
\]
These are not equal for general \((x,y)\) (they would be equal only on the line \(x=y\)), so \(\mathbf F\) is \emph{not} conservative on any domain containing points with \(x\neq y\). Therefore no global potential function exists.

\[
\boxed{F\text{ is not conservative (since } \partial P/\partial y = e^{x} \ne e^{y} = \partial Q/\partial x\text{).}}
\]

\end{document}