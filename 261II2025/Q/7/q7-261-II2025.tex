\documentclass[12pt]{exam}
\usepackage{amsthm}
\usepackage{bm}
\usepackage{libertine}
\usepackage[utf8]{inputenc}
\usepackage[margin=0.5in]{geometry}
\usepackage{amsmath,amssymb}
\usepackage{multicol}
\usepackage[shortlabels]{enumitem}
\usepackage{siunitx}
\usepackage{physics}
\usepackage{booktabs}
\usepackage{graphicx}
\usepackage{pgfplots}
\usepackage{listings}
\usepackage{tikz}
\usepackage{tikz-3dplot}

\bmdefine{\ii}{i}                       %% cuaternion i
\bmdefine{\jj}{j}                       %% cuaternion j
\bmdefine{\kk}{k}                       %% cuaternion k
\newcommand{\te}{\theta}                %% short for  \theta
\newcommand{\cC}{\mathcal{C}}

\newcommand{\word}[1]{\quad\text{#1}\quad} %% texto intercalado


\pgfplotsset{width=10cm,compat=1.9}
\usepgfplotslibrary{external}
\tikzexternalize

\newcommand{\class}{Math 261-001} % This is the name of the course 
\newcommand{\examnum}{Quiz 7} % This is the name of the assignment
\newcommand{\examdate}{October 24} % This is the due date





\begin{document}
\pagestyle{plain}
\thispagestyle{empty}

\noindent
\textbf{\class}\\
\textbf{\examnum}, \textbf{\examdate} \\

% Name \hfill CSU ID \# \hspace{2.25in}

%\vspace{10 pt}

\setlength{\tabcolsep}{3.5cm} % Default value: 6pt
\renewcommand{\arraystretch}{1.5}
\setlength\extrarowheight{1cm}
\begin{tabular}{ |c|c| } 
 \hline
 Name   & CSU ID \#  \\ 
 \hline
\end{tabular}
% ---
\vspace{10pt}
\iffalse

    \foreach \s in {1,...,5}{
          \choice $P_\s$ has no power 
     }%;
\fi

Be sure to read each question carefully. You must choose and answer \textbf{exactly two} of the four problems.  
If you attempt more than two, only the first two will be graded.  
Write your final answers in the boxes provided. Each problem is worth the same number of points.  
%\textbf{Each problem is accompanied by a figure to help you visualize the region in question.}  

\begin{enumerate} 

\item Ahhh, look at that, a Green's Theorem problem (so relatable, right?).  
You are asked to integrate over the triangle with vertices $(0,0)$, $(1,2)$, and $(0,2)$, traversed counterclockwise. Please do the following:
\begin{itemize}
  \item Give a parametrization of the boundary of the triangle.
  \item In terms of $x$ and $y$, describe the interior of the triangle.
\end{itemize}
\vfill
\begin{flushright}
\framebox(320,125){Boundary:\hspace{9cm}}
\framebox(320,75){Interior:\hspace{9cm}}
\end{flushright}

\item In this problem, $\grad(f)$ is the gradient of a scalar function, $\curl(F)$ is the curl of a vector field $F$, and $\div(F)$ is its divergence. Do the following:
\begin{itemize}
  \item Find $\curl(\curl F)$ for $F(x,y,z)=(e^y,e^z,e^x)$.
  \item Compute $\div(\grad(\div G))$ for $G(x,y,z)=(x^3y,y^3z,z^3x)$.
\end{itemize}
\vfill
\begin{flushright}
\framebox(320,50){$\curl(\curl F)=\phantom{aVENCERAPORQUESDANIFANTON}$}\\[6pt]
\framebox(320,50){$\div(\grad(\div G))=\phantom{aaVENCERAPORkeSDANIFANTON}$}
\end{flushright}

\newpage

\item Consider the conservative vector field 
$$F(x,y,z)=\left(\frac{1}{(x-y)^2}-\frac{1}{(x-z)^2},\frac{1}{(y-z)^2}-\frac{1}{(y-x)^2},\frac{1}{(z-x)^2}-\frac{1}{(z-y)^2}\right).$$
Compute its potential.
\vfill
\begin{flushright}
\framebox(350,50){}
\end{flushright}

\item Consider the vector field $F(x,y)=(0,x)$. 
\begin{itemize}
  \item Compute its two-dimensional curl.
  \item Write down another vector field $G$ with the same curl. (There may be more than one answer.)
  \item Compute the integral $\oint_C F\cdot d\mathbf{x}$, where $C$ is the circle centered at the origin with radius $R$. (Hint: use Green’s Theorem!)
\end{itemize}
\vfill
\begin{flushright}
\framebox(280,50){$\curl(F)=\phantom{aaaaadasdasdasdasdasdasdasdaasdasda}$}\\[6pt]
\framebox(280,50){$G=\phantom{aaaaadasdasdasdasdasdasdasdasdasdasdaasd}$}\\[6pt]
\framebox(280,50){$\oint_C F\cdot d\mathbf{x}=\phantom{aaaaadasdasdasdasdasdasdasdasdasdas}$}
\end{flushright}
\end{enumerate}


\end{document}

