\documentclass[12pt]{exam}
\usepackage{amsthm}
\usepackage{bm}
\usepackage{libertine}
\usepackage[utf8]{inputenc}
\usepackage[margin=0.5in]{geometry}
\usepackage{amsmath,amssymb}
\usepackage{multicol}
\usepackage[shortlabels]{enumitem}
\usepackage{siunitx}
\usepackage{physics}
\usepackage{booktabs}
\usepackage{graphicx}
\usepackage{pgfplots}
\usepackage{listings}
\usepackage{tikz}
\usepackage{tikz-3dplot}

\bmdefine{\ii}{i}                       %% cuaternion i
\bmdefine{\jj}{j}                       %% cuaternion j
\bmdefine{\kk}{k}                       %% cuaternion k
\newcommand{\te}{\theta}                %% short for  \theta
\newcommand{\cC}{\mathcal{C}}

\newcommand{\word}[1]{\quad\text{#1}\quad} %% texto intercalado


\pgfplotsset{width=10cm,compat=1.9}
\usepgfplotslibrary{external}
\tikzexternalize

\newcommand{\class}{Math 261-001} % This is the name of the course 
\newcommand{\examnum}{Quiz 9} % This is the name of the assignment
\newcommand{\examdate}{November 7} % This is the due date





\begin{document}
\pagestyle{plain}
\thispagestyle{empty}

\noindent
\textbf{\class}\\
\textbf{\examnum}, \textbf{\examdate} \\

% Name \hfill CSU ID \# \hspace{2.25in}

%\vspace{10 pt}

\setlength{\tabcolsep}{3.5cm} % Default value: 6pt
\renewcommand{\arraystretch}{1.5}
\setlength\extrarowheight{1cm}
\begin{tabular}{ |c|c| } 
 \hline
 Name   & CSU ID \#  \\ 
 \hline
\end{tabular}
% ---
\vspace{10pt}
\iffalse

    \foreach \s in {1,...,5}{
          \choice $P_\s$ has no power 
     }%;
\fi

Be sure to read each question carefully. You must choose and answer \textbf{exactly two} of the three problems.  
If you attempt more than two, only the first two will be graded.  
Write your final answers in the boxes provided. Each problem is worth the same number of points.  
%\textbf{Each problem is accompanied by a figure to help you visualize the region in question.}  

\begin{enumerate} 

\item Find the flux of the vector field 
\[
F=\left(\tfrac{x}{3},\,\tfrac{y}{3},\,\tfrac{z}{3}\right)
\]
\textbf{out of} the sphere of radius \(R\), assuming the normal vectors point outward.  
(Your final answer should be an explicit number, not an integral.)

\textbf{Solution.}
Parametrize the sphere of radius \(R\) by spherical coordinates:
\[
r(\theta,\phi) = (R\sin\phi\cos\theta,\, R\sin\phi\sin\theta,\, R\cos\phi),
\quad 0\le \theta < 2\pi, \; 0 \le \phi \le \pi.
\]

Compute the tangent vectors:
\[
r_\theta = (-R\sin\phi\sin\theta,\, R\sin\phi\cos\theta,\, 0),
\qquad
r_\phi = (R\cos\phi\cos\theta,\, R\cos\phi\sin\theta,\, -R\sin\phi).
\]

Their cross product gives the oriented normal vector:
\setlength{\tabcolsep}{6pt} % Default value: 6pt
\renewcommand{\arraystretch}{0.5}
\[
r_\theta \times r_\phi 
= \det\begin{pmatrix}
\ii & \jj & \kk\\
 -Rs\phi s\theta & Rs\phi c\theta & 0\\
 Rc\phi c\theta & Rc\phi s\theta & -Rs\phi
\end{pmatrix}=
R^2\sin\phi\,(\sin\phi\cos\theta,\, \sin\phi\sin\theta,\, \cos\phi).
\]
Thus, 
\[
r_\theta \times r_\phi = R^2\sin\phi\, n,
\quad\text{where } n = (\sin\phi\cos\theta,\, \sin\phi\sin\theta,\, \cos\phi)
\]
is the unit outward normal.

Now evaluate \(F(r)\):
\[
F(r(\theta,\phi)) = \frac{1}{3}\big(R\sin\phi\cos\theta,\, R\sin\phi\sin\theta,\, R\cos\phi\big)
= \frac{R}{3}n.
\]

Then
\[
F(r)\cdot (r_\theta \times r_\phi)
= \frac{R}{3} n \cdot (R^2\sin\phi\, n)
= \frac{R^3}{3}\sin\phi.
\]

Compute the flux:
\[
\Phi = \int_0^{2\pi}\!\!\int_0^{\pi}
\frac{R^3}{3}\sin\phi\, d\phi\, d\theta
= \frac{R^3}{3}(2\pi)(2)
= \boxed{\tfrac{4}{3}\pi R^3.}
\]




\item Your dog/cat/pet has one of those lamp things so they don't do something they shouldn't. With some time in your hand, you idealize it as the portion of the cone \(z^2 = x^2 + y^2\) between \(z=2\) and \(z=4\). 
Parametrize this surface as the \emph{ruled surface} joining the circles
\[
r_1(t) = (2\cos t,\,2\sin t,\,2), 
\qquad
r_2(t) = (4\cos t,\,4\sin t,\,4),\word{both in the range} 0\leq t\leq 2\pi.
\]
(Any other parametrization will get you at most half credit.)

\textbf{Solution.}
A ruled surface joining \(r_1\) and \(r_2\) is
\[
r(s,t) = s\,(r_2(t)-r_1(t))+r_1(t), \quad 0 \le s \le 1, \, 0 \le t \le 2\pi.
\]
That is,
\[
r(s,t) = \big((2+2s)\cos t,\, (2+2s)\sin t,\, 2+2s\big).
\]
This parametrization covers the conical band between \(z=2\) and \(z=4\).

\item Let us practice the right-hand rule. Consider the following three surfaces:  
\begin{itemize}
    \item a cylinder of radius \(1\), given by \(r(\theta,z) = (\cos\theta,\, \sin\theta,\, z)\);
    \item a surface obtained by rotating the curve \(y = \sqrt{x}\) about the \(y\)-axis, given by \(r(x,\theta) = (x\cos\theta,\, \sqrt{x},\, x\sin\theta)\);
    \item the graph of the function \(z = \cos(x) + \sin(y)\), given by \(r(x,y) = (x,\, y,\, \cos x + \sin y)\).
\end{itemize}

For each surface, indicate the order to cross the tangent vectors so that the normal vector points \textbf{outward} from the axis of symmetry (for the first two) and \textbf{upward} (for the graph).  
Your answer should be written in the form \(r_u \times r_v\), no computations.

\textbf{Solution.}

\begin{itemize}
    \item Cylinder \(r(\theta,z) = (\cos\theta,\,\sin\theta,\,z)\): outward normal corresponds to \(r_\theta \times r_z\).
    \item Surface of revolution \(r(x,\theta) = (x\cos\theta,\,\sqrt{x},\,x\sin\theta)\): outward from the \(y\)-axis corresponds to \(r_x \times r_\te\).
    \item Graph \(r(x,y) = (x,\,y,\,\cos x + \sin y)\): upward corresponds to \(r_x \times r_y\).
\end{itemize}

\item Consider the vector field
\[
F = \big(y,\, x^2,\, z(x^2 - y^3)^7 \cos(e^{xyz})\big).
\]
Find the flux of \(\operatorname{curl}(F)\) across the surface \(x^2 + y^2 + z^2 = 36\) for \(z \ge 0\), oriented \textbf{outward} (normals pointing away from the origin).  
Use Stokes' theorem and fear not.  
(Your final answer should be an explicit number, not an integral.)(You may use the facts $\int_0^{2\pi}\sin^2(x)\dd x=\pi$, and $\int_0^{2\pi}\cos^3(x)\dd x=0$.)

\textbf{Solution.}
Let \(S\) be the upper hemisphere of radius \(6\).  
Its boundary \(\partial S\) is the circle \(x^2 + y^2 = 36\), \(z=0\), oriented counterclockwise as viewed from above (since normals on the hemisphere point outward).

By Stokes' theorem,
\[
\iint_S (\nabla \times F)\cdot n\, dS
= \oint_{\partial S} F \cdot dr.
\]

On the boundary \(z=0\), the vector field reduces to
\[
F(x,y,0) = (y,\, x^2,\, 0).
\]
Parametrize the boundary by \(r(t) = (6\cos t,\, 6\sin t,\, 0)\), \(0\le t\le 2\pi\).
Then \(dr = (-6\sin t,\, 6\cos t,\, 0)\,dt.\)

Compute:
\[
F\cdot dr = (6\sin t)(-6\sin t) + (36\cos^2 t)(6\cos t)
          = -36\sin^2 t + 216\cos^3 t.
\]

Integrate around the full circle:
\[
\oint F\cdot dr 
= \int_0^{2\pi} \big(-36\sin^2 t + 216\cos^3 t\big)\, dt
= -36 \int_0^{2\pi}\!\!\sin^2 t\, dt + 216 \int_0^{2\pi}\!\!\cos^3 t\, dt.
\]
The second integral vanishes (odd powers of full periods vanish), and
\[
\int_0^{2\pi} \sin^2 t\, dt = \pi.
\]
Thus
\[
\oint_{\partial S} F\cdot dr = -36\pi.
\]

Hence by Stokes' theorem, the flux is
\[
\boxed{-36\pi.}
\]

(The negative sign indicates that the circulation induced by \(F\) points opposite to the hemisphere's outward normal orientation.)


\end{enumerate}


\end{document}

