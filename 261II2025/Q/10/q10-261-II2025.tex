\documentclass[12pt]{exam}
\usepackage{amsthm}
\usepackage{bm}
\usepackage{libertine}
\usepackage[utf8]{inputenc}
\usepackage[margin=0.5in]{geometry}
\usepackage{amsmath,amssymb}
\usepackage{multicol}
\usepackage[shortlabels]{enumitem}
\usepackage{siunitx}
\usepackage{physics}
\usepackage{booktabs}
\usepackage{graphicx}
\usepackage{pgfplots}
\usepackage{listings}
\usepackage{tikz}
\usepackage{tikz-3dplot}

\bmdefine{\ii}{i}                       %% cuaternion i
\bmdefine{\jj}{j}                       %% cuaternion j
\bmdefine{\kk}{k}                       %% cuaternion k
\newcommand{\te}{\theta}                %% short for  \theta
\newcommand{\cC}{\mathcal{C}}

\newcommand{\word}[1]{\quad\text{#1}\quad} %% texto intercalado


\pgfplotsset{width=10cm,compat=1.9}
\usepgfplotslibrary{external}
\tikzexternalize

\newcommand{\class}{Math 261-001} % This is the name of the course 
\newcommand{\examnum}{Quiz 10} % This is the name of the assignment
\newcommand{\examdate}{November 21} % This is the due date





\begin{document}
\pagestyle{plain}
\thispagestyle{empty}

\noindent
\textbf{\class}\\
\textbf{\examnum}, \textbf{\examdate} \\

% Name \hfill CSU ID \# \hspace{2.25in}

%\vspace{10 pt}

\setlength{\tabcolsep}{3.5cm} % Default value: 6pt
\renewcommand{\arraystretch}{1.5}
\setlength\extrarowheight{1cm}
\begin{tabular}{ |c|c| } 
 \hline
 Name   & CSU ID \#  \\ 
 \hline
\end{tabular}
% ---
\vspace{10pt}
\iffalse

    \foreach \s in {1,...,5}{
          \choice $P_\s$ has no power 
     }%;
\fi

Be sure to read each question carefully. You must choose and answer \textbf{exactly two} of the four problems.  
If you attempt more than two, only the first two will be graded.  
Write your final answers in the boxes provided. Each problem is worth the same number of points.  
%\textbf{Each problem is accompanied by a figure to help you visualize the region in question.}  

\begin{enumerate} 

\item Find the flux of the vector field 
\[
F=\left(\tfrac{x}{3},\,\tfrac{y}{3},\,\tfrac{z}{3}\right)
\]
\textbf{out of} the sphere of radius \(R\) assuming the normal vectors point outward, \textbf{by using the Divergence theorem}.
(Your final answer should be an explicit number, not an integral.)

\vfill
\begin{flushright}
\framebox(100,50){}
\end{flushright}

\item Consider the vector field:  
\[
F = \left(x - y^{2024} + 2025z^{-2024}, 2025x^{- 2024} + 2y - z^{2024}, -x^{2024} + 2025y^{- 2024} + 3z\right),
\]
and the solid $M$ bounded by spheres of radii $3$ and $6$ within the first octant, i.e. all $x,y,z\geq 0$.

Use the \textbf{Divergence theorem} to find the flux of this vector field through $S$, the boundary of $M$.  

\vfill
\begin{flushright}
\framebox(100,50){}
\end{flushright}
\newpage


\item Ahhh, you just got a package and ripped off the top face of the box in which it came. Distracted, you decided it was time to calculate a flux integral. Model the box as the unit cube between the origin $(0,0,0)$ and the opposite corner $(1,1,1)$ and remove the plane $z=1$ which is the top face. Let $F$ be the vector field 
$$(x^2+y^2+z^2,2xy+2yz+2zx,-2).$$
Compute the flux integral $\int_{\text{Open Box}}F\cdot N\dd S$ after creatively applying \textbf{the Divergence theorem}.

\vfill
\begin{flushright}
\framebox(100,50){}
\end{flushright}

\item We will derive \textbf{Archimedes' principle of buoyancy}: a body in a uniform density liquid experiences a buoyant force $\mathbf{F}$ equal to the displaced fluid's weight.\par
Recall: Pressure on a surface element is given by 
$$P = \rho g z,\quad\text{where}\quad z=\text{depth from surface}.$$ 
The pressure pushes upwards with a force field 
$$(0,0,\rho g z).$$
Write down the integral expression obtained before applying the \textbf{Divergence theorem} to the field in question and the result you obtain after integrating. You may assume that the volume of the body is $V$. (Your answer should look like $``\int\text{something}=\text{value}''$). 


\vfill
\begin{flushright}
\framebox(300,50){}
\end{flushright}

\end{enumerate}


\end{document}

