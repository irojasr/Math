\documentclass[12pt]{exam}
\usepackage{amsthm}
\usepackage{bm}
\usepackage{libertine}
\usepackage[utf8]{inputenc}
\usepackage[margin=0.5in]{geometry}
\usepackage{amsmath,amssymb}
\usepackage{multicol}
\usepackage[shortlabels]{enumitem}
\usepackage{siunitx}
\usepackage{physics}
\usepackage{booktabs}
\usepackage{graphicx}
\usepackage{pgfplots}
\usepackage{listings}
\usepackage{tikz}
\usepackage{tikz-3dplot}

\bmdefine{\ii}{i}                       %% cuaternion i
\bmdefine{\jj}{j}                       %% cuaternion j
\bmdefine{\kk}{k}                       %% cuaternion k
\newcommand{\te}{\theta}                %% short for  \theta
\newcommand{\cC}{\mathcal{C}}

\newcommand{\word}[1]{\quad\text{#1}\quad} %% texto intercalado


\pgfplotsset{width=10cm,compat=1.9}
\usepgfplotslibrary{external}
\tikzexternalize

\newcommand{\class}{Math 261-001} % This is the name of the course 
\newcommand{\examnum}{Quiz 11} % This is the name of the assignment
\newcommand{\examdate}{December 5} % This is the due date





\begin{document}
\pagestyle{plain}
\thispagestyle{empty}

\noindent
\textbf{\class}\\
\textbf{\examnum}, \textbf{\examdate} \\

% Name \hfill CSU ID \# \hspace{2.25in}

%\vspace{10 pt}

\setlength{\tabcolsep}{3.5cm} % Default value: 6pt
\renewcommand{\arraystretch}{1.5}
\setlength\extrarowheight{1cm}
\begin{tabular}{ |c|c| } 
 \hline
 Name   & CSU ID \#  \\ 
 \hline
\end{tabular}
% ---
\vspace{10pt}
\iffalse

    \foreach \s in {1,...,5}{
          \choice $P_\s$ has no power 
     }%;
\fi

Be sure to read each question carefully. You must choose and answer \textbf{exactly two} of the four problems.  
If you attempt more than two, only the first two will be graded.  
Write your final answers in the boxes provided. Each problem is worth the same number of points.  
%\textbf{Each problem is accompanied by a figure to help you visualize the region in question.}  

\begin{enumerate} 

\item Consider an open-top cylindrical can with radius $r$ and height $h$. If it has surface area $48\pi$, maximize its volume given that constraint.  
Use the method of \textbf{Lagrange multipliers}.


\vfill
\begin{flushright}
\framebox(100,50){}
\end{flushright}


\item Let's say you have a space station with $x > 0$ alien chefs, and each chef can produce $y > 0$ space sandwiches per hour. Assume the cost of operating the station is given by:  
\[
C(x, y) = 2x^2 + 3y,
\]  
where having more chefs increases the cost due to larger living quarters and additional supplies. For a fixed production of $3888$ space sandwiches per hour, find the minimum value of the cost function.  Use the method of \textbf{Lagrange multipliers}.


\vfill
\begin{flushright}
\framebox(100,50){}
\end{flushright}
\newpage

\item Consider the intersection of the cylinder $x^2 + y^2 = 4$ and the plane $2x + 2y + z = 2$.\par
Without solving, set up the system needed to find the points of the intersection closest to the origin.\par
\textbf{Hint:} Use the square of the distance function $f(x, y, z) = x^2 + y^2 + z^2$ and realize that the Lagrange equation is:  
\[
\nabla f = \lambda_1 \nabla g_1 + \lambda_2 \nabla g_2.
\]  


\begin{flushright}
\framebox(200,150){}
\end{flushright}

\item The \textbf{Cobb-Douglas model of production} states that the amount of product $P$ produced depends on the amount of labor $x$ and materials $y$ via:  
\[
P = kx^a y^b,
\]  
where $a,b$ are constants such that $a + b = 1$.  

Assume that labor costs $A$ dollars per unit, and materials cost $B$ dollars per unit. You have a fixed budget of $C$ dollars to spend on labor and materials, such that the cost is distributed as:  
\[
Ax + By = C.
\]  

Use \textbf{Lagrange multipliers} to maximize production $P(x, y)$. Your answer will depend on $a,b,A$ and $B$.

\vfill
\begin{flushright}
\framebox(200,50){}
\end{flushright}

\end{enumerate}


\end{document}

