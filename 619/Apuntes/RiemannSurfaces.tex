\documentclass[12pt]{memoir}

\def\nsemestre {II}
\def\nterm {Fall}
\def\nyear {2024}
\def\nprofesor {Renzo Cavalieri}
\def\nsigla {MATH619}
\def\nsiglahead {Complex Geometry}
\def\nlang {ENG}
%\def\darktheme{}
%\def\nhtml{}
\let\footruleskip\relax %%FADIR

\makeatletter
\ifx \nauthor\undefined
  \def\nauthor{Ignacio Rojas}
\else
\fi

\ifx \nextra \undefined
\ifx \nlang \undefined
\author{Basado en las clases impartidas por \nprofesor \\\small Notas tomadas por \nauthor}
\else
\author{Based on the lectures by \nprofesor \\\small Notes written by \nauthor}
\fi
\else
\author{\nauthor}
\fi
\date{\nterm\ \nyear}

%%%%%%%%%%%%%
%% 1. Pacotes
%%%%%%%%%%%%%

\usepackage{alltt}
\usepackage{amsfonts}
\usepackage{amsmath}
\usepackage{amssymb}
\usepackage{amsthm}
\usepackage{algorithm}
\usepackage[noend]{algpseudocode}
\usepackage{array}
\newcommand\hmmax{0} % default 3
\newcommand\bmmax{0} % default 4 %%tex.se/3676,219310
%\usepackage{bbold}
\usepackage{bm}
\usepackage{booktabs}
%\usepackage{caption}
%\usepackage{cancel}
%\usepackage{dsfont}
\usepackage{esint}
\usepackage{fancyhdr}
\usepackage{graphicx}
\usepackage[utf8]{inputenc}
\usepackage{listings}
\usepackage{mathabx}
\usepackage[cal=euler]{mathalfa}
%\usepackage[cal=euler,frak=euler]{mathalfa} % mathcal (JIRR) precisabamos correr initexmf --mkmaps en cmd JCVDG
\usepackage{mathdots}
\usepackage{mathrsfs}
%\usepackage{mathtools}
\usepackage{microtype}
\usepackage{multicol}
\usepackage{multirow}
\usepackage[theoremfont,largesc,tighter,osf]{newpxtext} %JCV Diff
\let\widering\undefined
%\usepackage[bigdelims,vvarbb]{newpxmath} %JCVDG
%por alguna razón esto afectaba las tildes en \min, \lim y demás
%\usepackage{pdflscape}
\usepackage{pgfplots}
\usepackage{physics}
\usepackage{siunitx}
\usepackage{slashed}
%\usepackage{stmaryrd}
%\SetSymbolFont{stmry}{bold}{U}{stmry}{m}{n}
%\usepackage{subfigure}
\usepackage{subcaption}
\usepackage{tabularx}
\usepackage[breakable,skins]{tcolorbox}
\usepackage{textcomp} %%JCVDG
\usepackage{tikz}
\usepackage{tkz-euclide}
\usepackage[normalem]{ulem}
\usepackage[all]{xy}
\usepackage{imakeidx}
\ifx \nlang \undefined
\usepackage[spanish]{babel}
\else\fi 
\usepackage{wrapfig}

%%%%%%%%%%%%%%%%%%%%
%% 2. Document Setup
%%%%%%%%%%%%%%%%%%%%

\ifx \nextra \undefined
    \ifx \nlang \undefined
    \makeindex[intoc, title=Índice Analítico] %Título de índice analítico
    %El índice general es aquel en el que se indican los capítulos, títulos y subtítulos del libro.
    %Índice onomástico es donde aparece el nombre de personas mencionadas en el texto, por orden alfabético con el número de las páginas donde aparecen.
    %El índice analítico se refiere a los temas y conceptos que aparecen en el libro
    \indexsetup{othercode={\fancyhead[LE]{\emph{Índice Analítico}}}}
    \else
    \makeindex[intoc, title=Index] 
    \indexsetup{othercode={\fancyhead[LE]{\emph{Index}}}}
    \fi
  \usepackage[pdftex,
    hidelinks,
    pdfauthor={\nauthor},
    pdfsubject={Notas: \nsiglahead\ \nsemestre-\nyear},
    pdftitle={Semestre \nsemestre\ - \nsigla},
  pdfkeywords={UCR Costa Rica Matem\'aticas Mate \nsemestre\ \nterm\ \nyear\ \nsiglahead}]{hyperref}
  \title{\nsigla\ --- \nsiglahead}
\else
  \usepackage[pdftex,
     hidelinks,
    pdfauthor={\nauthor},
    pdfsubject={\nextra \nsiglahead\ \nsemestre-\nyear},
    pdftitle={Semestre \nsemestre\ - \nsigla},
  pdfkeywords={UCR Costa Rica Matem\'aticas Mate \nsemestre\ \nterm\ \nyear\ \nsiglahead\ \nextra}]{hyperref}

  \title{\nsigla\ --- \nsiglahead \\ {\Large \nextra}}
  \renewcommand\printindex{}
\fi

\pgfplotsset{compat=1.12}


\pagestyle{fancy}
\setlength{\headheight}{15.72pt} %preceding warning said make it at least this


\ifx \nsiglahead \undefined
\def\nsiglahead{\nsigla}
\fi

\lhead{} %%%empty lhead
\rfoot{\thepage}

\ifx \nextra \undefined
  \chead{
    \ifnum\thepage=1
    \else
      \ifx \nlang \undefined
      \textbf{Notas \nsiglahead\ \nsemestre-\nyear}
      \else
      \textbf{Notes \nsiglahead\ \nsemestre-\nyear}
      \fi
    \fi}
  \rhead{}%\firstxmark} % Top right header
\else
%    \chead{
%    \ifnum\thepage=1
%    \else
%      \textbf{Notas \nsiglahead\ \nsemestre-\nyear \ (\nextra)}
%    \fi}
     \chead{
       \textbf{\nextra\ \nsigla\ \nsemestre-\nyear}
     }
     \rhead{
       \textbf{\nauthor}
     }
\fi
\lfoot{}%\lastxmark} % Bottom left footer
\cfoot{} % Bottom center footer

\usetikzlibrary{arrows.meta}
\usetikzlibrary{decorations.markings}
\usetikzlibrary{decorations.pathmorphing}
\usetikzlibrary{positioning}
\usetikzlibrary{fadings}
\usetikzlibrary{intersections}
\usetikzlibrary{cd}

\ifx \nhtml \undefined
\else
  \renewcommand\printindex{}
  \DisableLigatures[f]{family = *}
  \let\Contentsline\contentsline
  \renewcommand\contentsline[3]{\Contentsline{#1}{#2}{}}
  \renewcommand{\@dotsep}{10000}
  \newlength\currentparindent
  \setlength\currentparindent\parindent

  \newcommand\@minipagerestore{\setlength{\parindent}{\currentparindent}}
  \usepackage[active,tightpage,pdftex]{preview}
  \renewcommand{\PreviewBorder}{0.1cm}

  \newenvironment{stretchpage}%
  {\begin{preview}\begin{minipage}{\hsize}}%
    {\end{minipage}\end{preview}}
  \AtBeginDocument{\begin{stretchpage}}
  \AtEndDocument{\end{stretchpage}}

  \newcommand{\@@newpage}{\end{stretchpage}\begin{stretchpage}}

  \let\@real@section\section
  \renewcommand{\section}{\@@newpage\@real@section}
  \let\@real@subsection\subsection
  \renewcommand{\subsection}{\@ifstar{\@real@subsection*}{\@@newpage\@real@subsection}}
\fi
\ifx \ntrim \undefined
\usepackage[shortlabels]{enumitem} %mfw package order matters por savetrees
\else
  \usepackage{geometry}
  \geometry{
    papersize={379pt, 699pt},
    textwidth=345pt,
    textheight=596pt,
    left=17pt,
    top=54pt,
    right=17pt
  }
  \headwidth=345pt
 \usepackage[extreme]{savetrees}
\fi

\ifx \darktheme\undefined
\else
\pagecolor[rgb]{0.2,0.231,0.302}%{0.23,0.258,0.321}
\color[rgb]{1,1,1}
\fi

\ifx \nextra \undefined
\let\@real@maketitle\maketitle
\renewcommand{\maketitle}{\@real@maketitle\begin{center}\begin{minipage}[c]{0.9\textwidth}\centering\footnotesize 
  \ifx \nlang \undefined
  Estas notas no están respaldadas por los profesores y han sido modificadas (a menudo de manera significativa) después de las clases. No están lejos de ser representaciones precisas de lo que realmente se dio en clase y en particular todos los errores son casi seguramente míos.
  \else 
  Please note that these notes were not provided or endorsed by the lecturer and have been significantly altered after the class. They may not accurately reflect the content covered in class and any errors are solely my responsibility.
  \fi
\end{minipage}\end{center}}
\else
\fi

\def\moverlay{\mathpalette\mov@rlay}
\def\mov@rlay#1#2{\leavevmode\vtop{%
   \baselineskip\z@skip \lineskiplimit-\maxdimen
   \ialign{\hfil$\m@th#1##$\hfil\cr#2\crcr}}}
\newcommand{\charfusion}[3][\mathord]{
    #1{\ifx#1\mathop\vphantom{#2}\fi
        \mathpalette\mov@rlay{#2\cr#3}
      }
    \ifx#1\mathop\expandafter\displaylimits\fi}

%%%%%%%%%%%%%%%%%%%%%%%%%%%%%%
%% 2.1 Some internal machinery
%%%%%%%%%%%%%%%%%%%%%%%%%%%%%%

\makeatletter
\renewcommand{\section}{\@startsection{section}{1}{\z@}%
							 {-3.25ex \@plus -1ex \@minus -.2ex}%
							 {1.5ex \@plus.2ex}%
							 {\normalfont\large\bfseries}}
\renewcommand{\subsection}{\@startsection{subsection}{2}{\z@}%
							 {-3.25ex \@plus -1ex \@minus -.2ex}%
							 {1.5ex \@plus .2ex}%
               {\normalfont\normalsize\bfseries}}
\newcommand*{\defeq}{\!\mathrel{\rlap{%
             \raisebox{0.3ex}{$\m@th\cdot$}}%
             \raisebox{-0.3ex}{$\m@th\cdot$}}%
                    =\!}
\makeatother
\ifx\ntrim\undefined
\newcommand{\coursetitle}{\nsigla: \nsiglahead}
\ifx\nextra\undefined
\pagestyle{ruled}
\makeoddhead{ruled}{\coursetitle}{}{\rightmark}
\else\fi
\settypeblocksize{49pc}{37pc}{*}
\setlrmargins{*}{*}{1.2}
\setulmargins{*}{*}{0.8}
\setheadfoot{16pt}{30pt}
\setheaderspaces{*}{1.5pc}{1}
\setmarginnotes{1pt}{1pt}{1pt}
\checkandfixthelayout

\setlength{\unitlength}{3pt}
\setlength{\hfuzz}{1pt}

\setlength{\fboxsep}{6pt}

\setlength{\footskip}{17pt}

\linespread{1.1}
\else\fi
\renewcommand{\cftdotsep}{\cftnodots} %%% no dots in ToC
\setpnumwidth{2em}  %%% width of page-number box in ToC


\newcommand{\stophere}{\relax} %% can be changed to `\endinput'
% \newcommand{\stophere}{\endinput} %% can be changed to `\relax'


\DeclareRobustCommand{\qned}{\ifmmode
  \else \leavevmode\unskip\penalty9999 \hbox{}\nobreak\hfill \fi
  \quad\hbox{\qnedsymbol}}
\newcommand{\qnedsymbol}{$\boxminus$} %% No-proofs end with `\qned'

\DeclareRobustCommand{\qef}{\ifmmode
  \else \leavevmode\unskip\penalty9999 \hbox{}\nobreak\hfill \fi
  \quad\hbox{\qefsymbol}}
\newcommand{\qefsymbol}{$\lozenge$} %% Examples end with `\qef'
\def\enddefn{\qef\endtrivlist}      %% `\qef' automático en defns
\def\endejem{\qef\endtrivlist}      %% `\qef' automático en ejemplos

\newcommand{\hideqed}{\renewcommand{\qed}{}} %% to suppress `\qed'
\newcommand{\hideqef}{\renewcommand{\qef}{}} %% to suppress `\qef'

% \newcommand{\ldbrack}{\ensuremath{[\mskip-2.5mu[}} %% corchetes [[
% \newcommand{\rdbrack}{\ensuremath{]\mskip-2.5mu]}} %% corchetes ]]

\newcommand{\stroke}{\mathbin|}     %% (for `\bbraket' and such)

\newcommand{\rtri}{\blacktriangleright} %% (for `\marker' and such)
\newcommand{\tribar}{|\mkern-2mu|\mkern-2mu|} %% norma triple: |||


%% Formatting changes:

\renewcommand{\labelitemi}{$\diamond$} %% instead of bullets

\renewcommand{\theenumi}{\alph{enumi}}  %% use lowercase letters
\renewcommand{\labelenumi}{\textup{(\theenumi)}} %% inside parentheses

%%%%%%%%%%%%%%
%% 2.2. Colors
%%%%%%%%%%%%%%

\definecolor{MATLABgreen}{RGB}{28,172,0} % color values Red, Green, Blue
\definecolor{MATLABlila}{RGB}{170,55,241}
\definecolor{dankBlue}{RGB}{51,60,77} % color values Red, Green, Blue
\definecolor{dankBlueLite}{RGB}{82,97,125} % color values Red, Green, Blue
\definecolor{celesUCR}{RGB}{0,192,243}
\definecolor{azulUCR}{RGB}{0,93,164}
\definecolor{verdeUCR}{RGB}{109,192,103}
\definecolor{yelloUCR}{RGB}{255,224,106}

%%%%%%%%%%%%%%%%%%%%%%%%%%%
%% 3. Theorems and suchlike
%%%%%%%%%%%%%%%%%%%%%%%%%%%

\ifx\nlang\undefined

\theoremstyle{plain}
\ifx \nextra \undefined
\newtheorem{Th}{Teorema}[section]      %%% Theorem 1.1.1
\newtheorem{Tmon}[Th]{Teoremón}
\newtheorem{Prop}[Th]{Proposición}     %%% Proposition 1.1.2
\newtheorem{Lem}[Th]{Lema}             %%% Lemma 1.1.3
\newtheorem{Cor}[Th]{Corolario}        %%% Corollary 1.1.4
\else
\newtheorem{Th}{Teorema}               %%% Theorem 1.1.1
\newtheorem{Tmon}{Teoremón}
\newtheorem{Prop}{Proposición}         %%% Proposition 1.1.2
\newtheorem{Lem}{Lema}                 %%% Lemma 3
\newtheorem{Cor}{Corolario}            %%% Corollary 4
\fi
\newtheorem*{nonum-Th}{Teorema}        %%% No-numbered Theorem
\newtheorem*{nonum-Cor}{Corolario}     %%% No-numbered Corollary

\theoremstyle{definition}
\ifx \nextra \undefined
\newtheorem{Def}[Th]{Definición}       %%% Definition 1.1.5
\newtheorem{Ex}[Th]{Ejemplo}           %%% Example 1.1.6
\newtheorem{Ej}[Th]{Ejercicio}         %%% Ejercicio 1.1.7
\else
\newtheorem{Def}{Definición}           %%% Definition 5
\newtheorem{Ex}{Ejemplo}               %%% Example 6
\newtheorem{Ej}{Ejercicio}             %%% Ejercicio 7
\fi
\newtheorem{Hec}[Th]{Hecho}            %%% Hecho 1.1.8
\newtheorem*{nonum-Def}{Definición}    %%% No number Definition
\newtheorem*{nonum-Ex}{Ejemplo}        %%% No number Example
\newtheorem*{nonum-Ej}{Ejercicio}      %%% No number Ejercicio
\newtheorem*{nonum-Hec}{Hecho}         %%% No number Fact


\theoremstyle{remark}
\newtheorem{Rmk}[Th]{Observación}      %%%Remark 1.1.9
\newtheorem*{nonum-Rmk}{Observación}         %%% No number Fact
\newtheorem*{Notn}{Notaci\'on}        %% Notaciones
\newtheorem*{Warn}{Advertencia}       %% Advertencias
\newtheorem*{Qn}{Pregunta}            %% Pregunta

\else

\theoremstyle{plain}
\ifx \nextra \undefined
\newtheorem{Th}{Theorem}[section]      %%% Theorem 1.1.1
\newtheorem{Tmon}[Th]{Teoremón}
\newtheorem{Prop}[Th]{Proposition}     %%% Proposition 1.1.2
\newtheorem{Lem}[Th]{Lemma}             %%% Lemma 1.1.3
\newtheorem{Cor}[Th]{Corollary}        %%% Corollary 1.1.4
\else
\newtheorem{Th}{Theorem}               %%% Theorem 1.1.1
\newtheorem{Tmon}{Teoremón}
\newtheorem{Prop}{Proposition}         %%% Proposition 1.1.2
\newtheorem{Lem}{Lemma}                 %%% Lemma 3
\newtheorem{Cor}{Corollary}            %%% Corollary 4
\fi
\newtheorem*{nonum-Th}{Theorem}        %%% No-numbered Theorem
\newtheorem*{nonum-Cor}{Corollary}     %%% No-numbered Corollary

\theoremstyle{definition}
\ifx \nextra \undefined
\newtheorem{Def}[Th]{Definition}       %%% Definition 1.1.5
\newtheorem{Ex}[Th]{Example}           %%% Example 1.1.6
\newtheorem{Ej}[Th]{Exercise}         %%% Exercise 1.1.7
\else
\newtheorem{Def}{Definition}           %%% Definition 5
\newtheorem{Ex}{Example}               %%% Example 6
\newtheorem{Ej}{Exercise}             %%% Exercise 7
\fi
\newtheorem{Hec}[Th]{Fact}            %%% Fact 1.1.8
\newtheorem*{nonum-Def}{Definition}    %%% No number Definition
\newtheorem*{nonum-Ex}{Example}        %%% No number Example
\newtheorem*{nonum-Ej}{Exercise}      %%% No number Exercise
\newtheorem*{nonum-Hec}{Fact}         %%% No number Fact


\theoremstyle{remark}
\newtheorem{Rmk}[Th]{Remark}      %%%Remark 1.1.9
\newtheorem*{nonum-Rmk}{Remark}         %%% No number Fact
\newtheorem*{Notn}{Notation}        %% Notaciones
\newtheorem*{Warn}{Warning}       %% Warnings
\newtheorem*{Qn}{Question}            %% Question

\fi 

\numberwithin{equation}{section}

\setlength{\parindent}{3ex}

% \renewcommand{\labelitemi}{--}
% \renewcommand{\labelitemii}{$\circ$}
% \renewcommand{\labelenumi}{(\roman{*})}

%\let\stdsection\section
%\renewcommand\section{\newpage\stdsection}

\newcommand\qedsym{\hfill\ensuremath{\square}}
% Strike through
\def\st{\bgroup \ULdepth=-.55ex \ULset}

%%%%%%%%% === My T Color Box === %%%%%%%%%%%%%%

\ifx\nlang\undefined
\ifx \darktheme\undefined
\newtcolorbox{ptcb}{
colframe = black,
colback = white,
breakable,
enhanced
}
\newtcolorbox{ptcbp}{
colframe = black,
colback = white,
coltitle = black,
colbacktitle = black!40,
title = Prueba,
breakable,
enhanced
}
\newtcolorbox{ptcbr}{
colframe = blue,
colback = white,
coltitle = blue,
colbacktitle = blue!40,
title = Respuesta,
breakable,
enhanced
}
\else
\newtcolorbox{ptcb}{
colframe = white,
colback = dankBlue,
colupper = white,
breakable,
enhanced
}
\newtcolorbox{ptcbp}{
colframe = white,
colback = dankBlue,
colupper = white,
coltitle = white,
colbacktitle = dankBlueLite,
title = Prueba,
breakable,
enhanced
}
\newtcolorbox{ptcbr}{
colframe = white,
colback = white,
coltitle = blue,
colbacktitle = blue!40,
title = Respuesta,
breakable,
enhanced
}
\fi

\else
\ifx \darktheme\undefined
\newtcolorbox{ptcb}{
colframe = black,
colback = white,
breakable,
enhanced
}
\newtcolorbox{ptcbp}{
colframe = black,
colback = white,
coltitle = black,
colbacktitle = black!40,
title = Proof,
breakable,
enhanced
}
\newtcolorbox{ptcbr}{
colframe = blue,
colback = white,
coltitle = blue,
colbacktitle = blue!40,
title = Answer,
breakable,
enhanced
}
\else
\newtcolorbox{ptcb}{
colframe = white,
colback = dankBlue,
colupper = white,
breakable,
enhanced
}
\newtcolorbox{ptcbp}{
colframe = white,
colback = dankBlue,
colupper = white,
coltitle = white,
colbacktitle = dankBlueLite,
title = Proof,
breakable,
enhanced
}
\newtcolorbox{ptcbr}{
colframe = white,
colback = white,
coltitle = blue,
colbacktitle = blue!40,
title = Answer,
breakable,
enhanced
}
\fi
\fi


%%%%%%%%% === Listings === %%%%%%%%%%%%%%
\lstset{basicstyle=\ttfamily,breaklines=true}

\lstset{language=Matlab,%
    %basicstyle=\color{red},
    breaklines=true,%
    morekeywords={matlab2tikz},
    keywordstyle=\color{blue},%
    morekeywords=[2]{1}, keywordstyle=[2]{\color{black}},
    identifierstyle=\color{black},%
    stringstyle=\color{MATLABlila},
    commentstyle=\color{MATLABgreen},%
    showstringspaces=false,%without this there will be a symbol in the places where there is a space
    numbers=left,%
    numberstyle={\tiny \color{black}},% size of the numbers
    numbersep=9pt, % this defines how far the numbers are from the text
   % emph=[1]{for,end,break,function,if,elseif,else},emphstyle=[1]\color{blue}, %some words to emphasise
    %emph=[2]{word1,word2}, emphstyle=[2]{style},
}

%%%%%%%%%%%%%%%%%%%%%%%%%%
%% 4. Simple abbreviations
%%%%%%%%%%%%%%%%%%%%%%%%%%

%%% Operator names:

\DeclareMathOperator{\area}{area}
\DeclareMathOperator{\card}{card}
\DeclareMathOperator{\ccl}{ccl}
\DeclareMathOperator{\ch}{ch}
\DeclareMathOperator{\cl}{cl}
\DeclareMathOperator{\coker}{coker}
\DeclareMathOperator{\Conv}{Conv}   %%Convex hull
\DeclareMathOperator{\cosec}{cosec}
\DeclareMathOperator{\cosech}{cosech}
\DeclareMathOperator{\covol}{covol}
\DeclareDocumentCommand\curl{}{\operatorname{curl}} 
\DeclareMathOperator{\diag}{diag}
\DeclareMathOperator{\diam}{diam}
\DeclareMathOperator{\Diff}{Diff}
\DeclareDocumentCommand\div{}{\operatorname{div}} 
\DeclareMathOperator{\energy}{energy}
\DeclareMathOperator{\erfc}{erfc}
\DeclareMathOperator{\Ext}{Ext}
\DeclareMathOperator{\fst}{fst}
\DeclareMathOperator{\Fit}{Fit}
\DeclareMathOperator{\gr}{gr}
\DeclareMathOperator{\hcf}{hcf}
\DeclareMathOperator{\Hilb}{Hilb} %Hilbert scheme
\DeclareMathOperator{\id}{id}
\DeclareMathOperator{\Ind}{Ind}
\DeclareMathOperator{\Int}{Int}
\DeclareMathOperator{\Isom}{Isom}
\DeclareMathOperator{\lcm}{lcm}
\DeclareMathOperator{\length}{length}
\DeclareMathOperator{\Lie}{Lie}
\DeclareMathOperator{\like}{like}
\DeclareMathOperator{\Lk}{Lk}
\DeclareMathOperator{\Maps}{Maps}
\DeclareMathOperator{\mcd}{mcd}
\DeclareMathOperator{\mcm}{mcm}
\DeclareMathOperator{\Min}{Min}
\DeclareMathOperator{\orb}{orb}
\DeclareMathOperator{\ord}{ord}
\DeclareMathOperator{\otp}{otp}
\DeclareMathOperator{\pr}{pr}       %% proyector
\DeclareMathOperator{\poly}{poly}
\DeclareMathOperator{\rel}{rel}
\DeclareMathOperator{\Rad}{Rad}
\DeclareMathOperator*{\res}{res}
\DeclareMathOperator{\Ric}{Ric}
\DeclareMathOperator{\rk}{rk}
\DeclareMathOperator{\Rees}{Rees}
\DeclareMathOperator{\Root}{Root}
\DeclareMathOperator{\rot}{rot}         %% rotacional
\DeclareMathOperator{\spn}{span}
\DeclareMathOperator{\St}{St}
\DeclareMathOperator{\supp}{supp}
\DeclareMathOperator{\Syl}{Syl}
\DeclareMathOperator{\Sym}{Sym}
\DeclareMathOperator{\vol}{vol}

% not-math
\newcommand{\bolds}[1]{{\bfseries #1}}
\newcommand{\cat}[1]{\mathsf{#1}}
\newcommand{\ph}{\,\cdot\,}
\newcommand{\term}[1]{\un{#1}\index{#1}}
\newcommand{\phantomeq}{\hphantom{{}={}}}
\newcommand{\ttt}{\texttt}
\newcommand{\red}[1]{\textcolor{red}{#1}}
\newcommand{\prp}[1]{\textcolor{purple}{#1}}
\newcommand{\blu}[1]{\textcolor{azulUCR}{#1}}
\newcommand{\green}[1]{\textcolor{verdeUCR}{#1}}
\newcommand{\yelo}[1]{\textcolor{yelloUCR}{#1}}
\newcommand{\cele}[1]{\textcolor{celesUCR}{#1}}

%functions
\DeclareMathOperator{\sgn}{sgn}
\newcommand*{\Cdot}{{\raisebox{-0.25ex}{\scalebox{1.5}{$\cdot$}}}}      %% cdot más grande
\newcommand{\ind}{\mathbf{1}}       %%%indicator function
\newcommand{\mm}{\mathfrak{m}}      %%%metric


% Greek letters:

\newcommand{\al}{\alpha}                %% short for  \alpha
\newcommand{\bt}{\beta}                 %% short for  \beta
\newcommand{\Dl}{\Delta}                %% short for  \Delta
\newcommand{\dl}{\delta}                %% short for  \delta
\newcommand{\eps}{\varepsilon}          %% short for  \varepsilon
\newcommand{\Ga}{\Gamma}                %% short for  \Gamma
\newcommand{\ga}{\gamma}                %% short for  \gamma
\newcommand{\kp}{\kappa}                %% short for  \kappa
\newcommand{\La}{\Lambda}               %% short for  \Lambda
\newcommand{\la}{\lambda}               %% short for  \lambda
\newcommand{\Om}{\Omega}                %% short for  \Omega
\newcommand{\om}{\omega}                %% short for  \omega
\newcommand{\Sg}{\Sigma}                %% short for  \Sigma
\newcommand{\sg}{\sigma}                %% short for  \sigma
\newcommand{\Te}{\Theta}                %% short for  \Theta
\newcommand{\te}{\theta}                %% short for  \theta
\newcommand{\ups}{\upsilon}             %% short for  \upsilon
\newcommand{\vf}{\varphi}               %% short for  \varphi
\newcommand{\ze}{\zeta}                 %% short for  \zeta
\newcommand{\vsg}{\varsigma}            %% short for  \varsigma
\newcommand{\vte}{\vartheta}            %% short for  \vartheta

%Boldface letters

\newcommand{\bA}{\mathbb{A}}        %% antisimetrizador
\newcommand{\bB}{\mathbb{B}}        %% bola unitaria
\newcommand{\bC}{\mathbb{C}}    %%% números complejos
\newcommand{\bCP}{\mathbb{CP}}  %%% espacio proyectivo complejo
\newcommand{\bD}{\mathbb{D}}        %% Poincaré disk
\newcommand{\bE}{\mathbb{E}}
\newcommand{\bF}{\mathbb{F}}        %% un cuerpo
\newcommand{\bH}{\mathbb{H}}        %% cuaterniones
\newcommand{\bI}{\mathbb{I}}        %% ideal de zeros
\newcommand{\bK}{\mathbb{K}}            %% ein korper
\newcommand{\bN}{\mathbb{N}}    %%% números naturales
\newcommand{\bP}{\mathbb{P}}        %% números enteros positivos
\newcommand{\bQ}{\mathbb{Q}}    %%% números racionales
\newcommand{\bR}{\mathbb{R}}    %%% números reales
\newcommand{\bRP}{\mathbb{RP}}  %%% espacio proyectivo real
\newcommand{\bS}{\mathbb{S}}    %%% esfera
\newcommand{\bT}{\mathbb{T}}        %% círculo o toro
\newcommand{\bV}{\mathbb{V}}        %% lugar geométrico de ceros
\newcommand{\bZ}{\mathbb{Z}}    %%% números enteros

%Script letters:

\newcommand{\cA}{\mathcal{A}}           %% formas diferenciales
\newcommand{\cB}{\mathcal{B}}           %% una base vectorial
\newcommand{\cC}{\mathcal{C}}           %% otra base vectorial
\newcommand{\cD}{\mathcal{D}}           %% funciones de prueba
\newcommand{\cE}{\mathcal{E}}           %% un modulo proyectivo
\newcommand{\cF}{\mathcal{F}}           %% espacio de Fock
\newcommand{\cG}{\mathcal{G}}           %% funtor de Gelfand
\newcommand{\cH}{\mathcal{H}}           %% espacio de Hilbert
\newcommand{\cI}{\mathcal{I}}           %% un funtor de inclusion
\newcommand{\cJ}{\mathcal{J}}           %% otro funtor
\newcommand{\cK}{\mathcal{K}}           %% otro espacio de Hilbert
\newcommand{\cL}{\mathcal{L}}           %% operadores lineales
\newcommand{\cM}{\mathcal{M}}           %% multiplicadores
\newcommand{\cN}{\mathcal{N}}           %% funciones nulas
\newcommand{\cO}{\mathcal{O}}           %% funciones de crec-to lento
\newcommand{\cP}{\mathcal{P}}           %% una particion
\newcommand{\cR}{\mathcal{R}}           %% funciones representativas
\newcommand{\cQ}{\mathcal{Q}}           %% otra particion
\newcommand{\cS}{\mathcal{S}}           %% funciones de Schwartz
\newcommand{\cT}{\mathcal{T}}           %% una topologia
\newcommand{\cU}{\mathcal{U}}           %% cubrimiento abierto
\newcommand{\cV}{\mathcal{V}}           %% vecindarioas
\newcommand{\cW}{\mathcal{W}}           %% grupo de Weyl
\newcommand{\cZ}{\mathcal{Z}}           %% topología de Zariski

%%% Fraktur letters:

\newcommand{\gA}{\mathfrak{A}}      %% un atlas
\newcommand{\g}{\mathfrak{g}}       %% un álgebra de Lie
\newcommand{\gB}{\mathfrak{B}}      %% otro atlas
\newcommand{\ggl}{\mathfrak{gl}}    %% álg de Lie general lineal
\newcommand{\gsl}{\mathfrak{sl}}    %% álg de Lie especial lineal
\newcommand{\gso}{\mathfrak{so}}    %% álg de Lie especial ortogonal
\newcommand{\gsu}{\mathfrak{su}}    %% álg de Lie especial unitaria
\newcommand{\gX}{\mathfrak{X}}      %% campos vectoriales

%%% Roman letters:

\newcommand{\dR}{\mathrm{dR}}       %% cohomología de de Rham
\newcommand{\rGL}{\mathrm{GL}}      %% grupo general lineal
\newcommand{\rO}{\mathrm{O}}        %% grupo ortogonal
\newcommand{\rSL}{\mathrm{SL}}      %% grupo especial lineal
\newcommand{\rSO}{\mathrm{SO}}      %% grupo ortogonal especial
\newcommand{\rSp}{\mathrm{Sp}}      %% grupo simpléctico
\newcommand{\rSU}{\mathrm{SU}}      %% grupo unitario especial
\newcommand{\rU}{\mathrm{U}}        %% grupo unitario
\newcommand{\rUH}{\mathrm{UH}}      %% cuaterniones unitarias
\newcommand{\rT}{\mathrm{T}}        %% grupo triangular

% Sanserif letters:

\newcommand{\sA}{\mathsf{A}}            %% algebras de Lie A_n
\newcommand{\sB}{\mathsf{B}}            %% grupo como categoria
\newcommand{\sC}{\mathsf{C}}            %% una categoria
\newcommand{\sD}{\mathsf{D}}            %% otra categoria
\newcommand{\sE}{\mathsf{E}}            %% otra categoria mas
\newcommand{\sF}{\mathsf{F}}            %% algebra de Lie F_4
\newcommand{\sG}{\mathsf{G}}            %% algebra de Lie G_2
\newcommand{\sJ}{\mathsf{J}}            %% un poset
\newcommand{\sK}{\mathsf{K}}            %% un poset
\newcommand{\sL}{\mathcal{L}}           %% derivada de Lie
\newcommand{\sN}{\mathsf{N}}            %% categoría con objetos \bN
\newcommand{\sT}{\mathsf{T}}            %% transpuesta

%%% Boldface letters:

\bmdefine{\CC}{C}                       %% C negrilla
\bmdefine{\cc}{c}
%\bmdefine{\dd}{d}                       %% d negrilla
\bmdefine{\ee}{e}                       %% vector e
\bmdefine{\eeps}{\varepsilon}           %% basic form \eps
\bmdefine{\FF}{F}                       %% vector F
\bmdefine{\ff}{f}                       %% vector f
\bmdefine{\ii}{i}                       %% cuaternion i
\bmdefine{\jj}{j}                       %% cuaternion j
\bmdefine{\kk}{k}                       %% cuaternion k
\bmdefine{\lla}{\lambda}                %% sucesion \la
\bmdefine{\mmu}{\mu}                    %% sucesion \mu
\bmdefine{\pp}{p}                       %% vector p
\bmdefine{\qq}{q}                       %% vector q
\bmdefine{\rr}{r}                       %% vector r
\bmdefine{\ssg}{\sigma}                 %% vector \sg
%\bmdefine{\sss}{s}
%\bmdefine{\ttt}{t}
\bmdefine{\VV}{V}                       %% V negrilla
\bmdefine{\xx}{x}                       %% sucesion x
\bmdefine{\xxi}{\xi}                    %% vector \xi
\bmdefine{\yy}{y}                       %% sucesion y
\bmdefine{\zz}{z}                       %% sucesion z

% Matrix groups
\DeclareMathOperator{\GL}{GL}   %%% grupo general lineal
\DeclareMathOperator{\Or}{O}    %%% grupo ortogonal
\DeclareMathOperator{\PGL}{PGL} %%% grupo proyectivo lineal
\DeclareMathOperator{\PSL}{PSL} %%% grupo proyectivo lineal especial
\DeclareMathOperator{\PSO}{PSO} %%% grupo proyectivo ortogonal
\DeclareMathOperator{\PSU}{PSU} %%% grupo proyectivo unitario
\DeclareMathOperator{\SL}{SL}   %%% grupo especial lineal
\DeclareMathOperator{\SO}{SO}   %%% grupo especial ortogonal
\DeclareMathOperator{\SU}{SU}   %%% grupo especial unitario

% Numericc
\newcommand{\argmin}{\text{argm\'in}}
\DeclareMathOperator{\dof}{dof}

%% Brackets
\newcommand{\conj}[1]{\left\lbrace#1\right\rbrace}
\newcommand{\bonj}[1]{\left\lbrack#1\right\rbrack}
\newcommand{\obonj}[1]{\left\rbrack#1\right\lbrack}
\newcommand{\rbonj}[1]{\left\rbrack#1\right\rbrack}
\newcommand{\lbonj}[1]{\left\lbrack#1\right\lbrack}
\newcommand{\snm}[1]{\|#1\|}           %small norma
\newcommand{\nm}[1]{\left\|#1\right\|} %norma pegadita
\newcommand{\pnm}[1]{\biggl|\biggl|#1\biggr|\biggr|}
\let\oldvec=\vec
\renewcommand{\vec}[1]{\mathbf{#1}}
\newcommand\quot[2]{
        \mathchoice
            {% \displaystyle
                \text{\raise1ex\hbox{$#1$}\Big/\lower1ex\hbox{$#2$}}%
            }
            {% \textstyle
                {^{ #1}/_{ #2}}
            }
            {% \scriptstyle
                {^{ #1}/_{ #2}}
            }
            {% \scriptscriptstyle
                {^{ #1}/_{ #2}}
            }
    }
%\newcommand*\quot[2]{{^{\textstyle #1}\big/_{\textstyle #2}}}
\newcommand*\squot[2]{{^{ #1}/_{ #2}}}%%%small quotient
\newcommand{\multinom}[2]{\ensuremath{\left(\kern-.3em\left(\genfrac{}{}{0pt}{}{#1}{#2}\right)\kern-.3em\right)}}

% Probability
\DeclareMathOperator{\Bernoulli}{Bernoulli}
\DeclareMathOperator{\betaD}{beta}
\DeclareMathOperator{\bias}{bias}
\DeclareMathOperator{\binomial}{binomial}
\DeclareMathOperator{\corr}{corr}
\DeclareMathOperator{\cov}{cov}
\DeclareMathOperator{\gammaD}{gamma}
\DeclareMathOperator{\mse}{mse}
\DeclareMathOperator{\multinomial}{multinomial}
\DeclareMathOperator{\Poisson}{Poisson}
\DeclareMathOperator{\Var}{Var}     %%%variance
\DeclareMathOperator{\Cov}{Cov}     %%%Covariance
\renewcommand{\mid}{\;\ifnum\currentgrouptype=16 \middle\fi|\;}

% Combinatorics
\DeclareMathOperator{\ins}{ins}   % insertion tableaux
\DeclareMathOperator{\asc}{asc}   % ascents
\DeclareMathOperator{\rw}{rw}     % reading word
\DeclareMathOperator{\rev}{rev}     % reading word
\DeclareMathOperator{\rect}{rect} % rectification of young tableau
\DeclareMathOperator{\sh}{sh}     % shape of young tableau
\DeclareMathOperator{\std}{std}   % standarization
\DeclareMathOperator{\Fl}{\mathcal{F}\ell}       %% conjunto de Flags
\DeclareMathOperator{\Frob}{Frob} % Frobenius map

% Algebra
\DeclareMathOperator{\Ad}{Ad}       %% acción adjunta
\DeclareMathOperator{\adj}{adj}
\DeclareMathOperator{\Ann}{Ann}     %% aniquilador o anulador de módulos
\DeclareMathOperator{\Ass}{Ass}     %% ideales asociados
\DeclareMathOperator{\Aut}{Aut}
\DeclareMathOperator{\Bl}{\mathcal{B}\!\ell}       %% blowup de un espacio
\DeclareMathOperator{\Char}{char}
\DeclareMathOperator{\codim}{codim}
\DeclareMathOperator{\disc}{disc}
\DeclareMathOperator{\dom}{dom}
\DeclareMathOperator{\End}{End}     %%%space of endomorphisms
\DeclareMathOperator{\Fix}{Fix}
\DeclareMathOperator{\Frac}{Frac}
\DeclareMathOperator{\Gal}{Gal}
\DeclareMathOperator{\gen}{gen}     %%%set generated by...
\DeclareMathOperator{\Gr}{Gr}       %%%Grassmannian
\DeclareMathOperator{\Hom}{Hom}
\DeclareMathOperator{\Hurw}{Hurw}
\DeclareMathOperator{\image}{image}
\DeclareMathOperator{\Mor}{Mor}
\DeclareMathOperator{\Nil}{Nil}
\DeclareMathOperator{\Orb}{Orb}
\DeclareMathOperator{\Pic}{Pic}     %%% grupo de Picard 
\DeclareMathOperator{\Quot}{Quot}
\DeclareMathOperator{\Spec}{Spec}
\DeclareMathOperator{\Stab}{Stab}
\DeclareMathOperator{\Taut}{Taut}

% Analysis
\DeclareMathOperator*{\esssup}{ess\hspace{0.5mm}sup}
\DeclareMathOperator*{\essinf}{ess\hspace{0.5mm}inf}
%\DeclareMathOperator{\Int}{Int}     %%%interior vacilon funcional

\newcommand{\loc}{\text{loc}}
\newcommand{\LB}{\cL_\cB}           %%%bounded linear operator

% Logic
\newcommand{\cleq}{\preccurlyeq}
\newcommand{\cgeq}{\succcurlyeq}

% Others
\renewcommand{\ev}{\operatorname{ev}}     %%%evalutation previously expectation value physics package
\newcommand{\bigcupdot}{\charfusion[\mathop]{\bigcup}{\Cdot}} %%JCVDG
%\renewcommand{\bigcupdot}{\charfusion[\mathop]{\bigcup}{\Cdot}}
\newcommand{\cupdot}{\charfusion[\mathbin]{\cup}{\Cdot}}
\newcommand{\exterior}{\mathchoice{{\textstyle\bigwedge}}{{\bigwedge}}{{\textstyle\wedge}}{{\scriptstyle\wedge}}}
\newcommand{\hol}{\mathfrak{hol}}
\newcommand{\Id}{\mathrm{Id}}
\newcommand{\lie}[1]{\mathfrak{#1}}
\newcommand{\qeq}{\mathrel{``{=}"}}
\newcommand{\wsto}{\stackrel{\mathrm{w}^*}{\to}}
\newcommand{\wt}{\mathrm{wt}}

%\let\Im\relax
%\let\Re\relax

%%% Shorter symbol names:

\newcommand{\bull}{{\scriptstyle\bullet}}  %% vertice en figuras
\newcommand{\del}{\partial}             %% short for  \partial
\newcommand{\downto}{\downarrow}        %% limite a la derecha
\newcommand{\dsp}{\displaystyle}        %% despliegue en texto
\renewcommand{\geq}{\geqslant}          %% mayor o igual (variante)
\newcommand{\hookto}{\hookrightarrow}     %% inclusion arrow
\newcommand{\isom}{\simeq}              %% isomorfismo
\renewcommand{\l}{\ell}                   %% ele cursiva
\renewcommand{\leq}{\leqslant}          %% menor o igual (variante)
\newcommand{\less}{\setminus}           %% set difference
\newcommand{\otto}{\leftrightarrow}     %% bijection
\newcommand{\ox}{\otimes}               %% producto tensorial
\newcommand{\rt}{\triangleleft}         %% un orden parcial
\newcommand{\rteq}{\trianglelefteq}     %% normal subgroup
\newcommand{\up}{{\mathord{\uparrow}}}  %% espinor `up'
\newcommand{\upto}{\uparrow}            %% left hand limit
\newcommand{\w}{\wedge}                 %% producto exterior
\newcommand{\wto}{\rightharpoonup}      %% convergencia debil
\newcommand{\x}{\times}                 %% producto vectorial
\renewcommand{\.}{\Cdot}                %% producto escalar
\renewcommand{\:}{\mathbin{:}}          %% colon in  f: A -> B
\newcommand{\into}{\rightarrowtail}     %% injection arrow
\newcommand{\lr}{\dashv}                %% adjunction
\newcommand{\lt}{\triangleright}        %% a left action
\newcommand{\lteq}{\trianglerighteq}    %% normal supergroup
\newcommand{\nb}{\nabla}                %% homomorfismo de suma
\newcommand{\nisom}{\not\simeq}         %% negacion de isomorfismo
%\newcommand{\oast}{\circledast}         %% variante de * (ya existe en stmaryrd)
\newcommand{\onto}{\twoheadrightarrow}  %% surjection arrow
\newcommand{\opp}{\circ}                %% objeto opuesto
\newcommand{\ottto}{\longleftrightarrow} %% bijection in display
\newcommand{\pullb}{\lrcorner}          %% simbolo de pullback
\newcommand{\pushf}{\ulcorner}          %% simbolo de pushout
\newcommand{\rx}{\rtimes}               %% producto semidirecto
\newcommand{\To}{\Rightarrow}           %% entre funtores
\newcommand{\tofro}{\rightleftarrows}   %% pair of opposed maps
\newcommand{\toto}{\rightrightarrows}   %% pair of parallel maps

\renewcommand{\2}{\flat}                  %% marcador de sucesiones
\newcommand{\3}{\sharp}                 %% marcador de sucesiones
\newcommand{\4}{\natural}               %% marcador de morfismos
% \newcommand{\5}{\diamond}               %% for roots of trees
% \newcommand{\7}{\dagger}                %% adjunto de operador
\newcommand{\8}{\bullet}                %% anonymous degree

%%% Useful abbreviations:

\newcommand{\Coo}{\cC^\infty}         %% funciones suaves
\newcommand{\ctr}{\mathbin{\lrcorner\,}} %% contraction symbol
\newcommand{\nbf}{{\vec\nabla}}     %% short for  \vec\nabla

\newcommand{\as}{\quad\text{cuando}\enspace} %% `cuando' en límites
\newcommand{\bCoo}{{\bC_\infty}}    %% esfera de Riemann
% \newcommand{\bRoo}{{\bR_\infty}}    %% círculo real extendido

%%% Repeated relations:

\newcommand{\cupycup}{\cup\cdots\cup} %% unión repetida
\newcommand{\capycap}{\cap\cdots\cap} %% intersección repetida
\newcommand{\sys}{\subset\cdots\subset}%% subconjunto propio repetido
\newcommand{\subysub}{\subseteq\cdots\subseteq} %%subconjunto repetido
\newcommand{\oxyox}{\otimes\cdots\otimes} %% prod tensorial repetido
\newcommand{\wyw}{\wedge\cdots\wedge} %% producto exterior repetido
\newcommand{\opyop}{\oplus\cdots\oplus} %% suma directa repetida
\newcommand{\xyx}{\times\cdots\times} %% producto directo repetido

%%% Arrows with riders:

\newcommand{\longto}{\mathop{\longrightarrow}\limits}

%%% Small fractions in displays:

\newcommand{\half}{{\mathchoice{\nhalf}{\thalf}{\shalf}{\shalf}}} %%display text script script^2
\newcommand{\happi}{{\tfrac{\pi}{2}}} %% small fraction  \pi/2
\newcommand{\quarter}{\tfrac{1}{4}} %% small fraction  1/4
\newcommand{\nhalf}{\frac{1}{2}}
\newcommand{\shalf}{{\scriptstyle\frac{1}{2}}} %% tiny fraction 1/2
\newcommand{\thalf}{{\tfrac{1}{2}}} %% small fraction  1/2
\renewcommand{\third}{\tfrac{1}{3}}   %% small fraction  1/3 %Hay que renew porque mathabx toma second y third como x'' y x''' por ejemplo

\newcommand{\ihalf}{{\tfrac{i}{2}}} %% small fraction  i/2

%%%%%%%%%%%%%%%%%%%%%%%%%%%%%
%% 5. Commands with arguments
%%%%%%%%%%%%%%%%%%%%%%%%%%%%%

%%% Accent-like commands, abbreviated:

\newcommand{\ov}{\overline}        %% short for  \overline
\newcommand{\un}{\underline}       %% short for  \underline
\newcommand{\wh}{\widehat}          %% short for  \widehat

%%% Separate words in displays:

\newcommand{\word}[1]{\quad\text{#1}\quad} %% texto intercalado

%%% Webpage locator:

\newcommand{\zelda}[1]{$\langle${\footnotesize\texttt{#1}}$\rangle$}

%% Symbol placement:

\newcommand{\pre}[1]{{}^{#1\!}} %% upper left exponent

%%% Proof-part labels:

\newcommand{\Adiff}[2]{\ensuremath{\Ad\,(\mathrm{#1})\Longleftrightarrow
    (\mathrm{#2})}:\enspace}
\newcommand{\Adimp}[2]{\ensuremath{\Ad\,(\mathrm{#1})\Longrightarrow
    (\mathrm{#2})}:\enspace}
\newcommand{\Adit}[1]{\ensuremath{\Ad\,(\mathrm{#1})}:\enspace}

%%% Enclose one argument with delimiters:

\newcommand{\bool}[1]{\llbracket#1\rrbracket} %% condición booleana
\newcommand{\combo}[1]{\operatorname{co}(#1)} %% convex combo
\newcommand{\lin}[1]{\operatorname{lin}\langle#1\rangle} %% `span'
\newcommand{\set}[1]{\{\,#1\,\}}    %% set notation

\newcommand{\floor}[1]{\lfloor#1\rfloor} %% mayor entero <= x
\newcommand{\Set}[1]{\biggl\{\,#1\,\biggr\}} %% set notation (large)
\newcommand{\roof}[1]{\lceil#1\rceil} %% menor entero >= x
\newcommand{\genr}[1]{\left\langle #1\right\rangle}     %% grupo generado por #1

%%% Asides:

\newcommand{\aside}[1]{$\llbracket$\,#1\,$\rrbracket$} % nota lateral
\ifx \nlang \undefined
\newcommand{\hint}[1]{$\llbracket$\,In\-di\-ca\-ci\'on: #1\,$\rrbracket$}
\else
\newcommand{\hint}[1]{$\llbracket$\,Hint: #1\,$\rrbracket$}
\fi 


%%% Matrices:

\newcommand{\onebytwo}[2]{\begin{pmatrix} %% 1 x 2 matrix
  #1 & #2 \end{pmatrix}}
\newcommand{\onebythree}[3]{\begin{pmatrix} %% 1 x 3 matrix
  #1 & #2 & #3 \end{pmatrix}}
\newcommand{\onebyfour}[4]{\begin{pmatrix} %% 1 x 4 matrix
  #1 & #2 & #3 & #4 \end{pmatrix}}
\newcommand{\twobyone}[2]{\begin{pmatrix} %% 2 x 1 matrix
   #1 \\ #2 \end{pmatrix}}
\newcommand{\twobytwo}[4]{\begin{pmatrix} %% 2 x 2 matrix
   #1 & #2 \\ #3 & #4 \end{pmatrix}}
\newcommand{\twobythree}[6]{\begin{pmatrix} %% 2 x 3 matrix
    #1 & #2 & #3\\ #4 & #5 & #6 \end{pmatrix}}
\newcommand{\threebyone}[3]{\begin{pmatrix} %% 3 x 1 matrix
   #1 \\ #2 \\ #3 \end{pmatrix}}
\newcommand{\threebythree}[9]{\begin{pmatrix} %% 3 x 3 matrix
   #1 & #2 & #3 \\ #4 & #5 & #6 \\ #7 & #8 & #9 \end{pmatrix}}
\newcommand{\fourbyone}[4]{\begin{pmatrix} %% 2 x 1 matrix
   #1 \\ #2 \\ #3 \\ #4 \end{pmatrix}}
%\newcommand{\fourbyfour}[16]{\begin{pmatrix} %% 4 x 4 matrix
%  #1 & #2 & #3 & #4\\ #5 & #6 & #7 & #8 \\ #9 & #10 & #11 & #12 \\ #13 & #14 & #15 & #16 \end{pmatrix}}
\newcommand{\nbyn}[9]{\begin{pmatrix} %% 4 x 4 matrix with prefilled entries
  #1 & #2 & \cdots & #3\\ #4 & #5 & \cdots & #6 \\ \vdots & \vdots & \ddots & \vdots \\ #7 & #8 & \cdots & #9 \end{pmatrix}}

%%%%%%%%%%%%%%%%%%%%%%%%%%%%
%% 6. Hyphenation exceptions
%%%%%%%%%%%%%%%%%%%%%%%%%%%%

\hyphenation{auto-va-lor auto-va-lo-res auto-vec-tor auto-vec-to-res
car-di-na-li-dad ce-rra-da ce-rra-do ce-rra-das ce-rra-dos cons-tan-te
cons-tan-tes cons-truc-ci cons-truir con-ti-nua con-ti-nua-mente
con-ti-nuas con-ti-nui-dad con-ti-nuo con-ti-nuos co-rres-pon-den-cia
co-rres-pon-de co-rres-pon-den co-rres-pon-dien-te
co-rres-pon-dien-tes co-va-rian-te cual-quier cual-quiera
cu-bri-mien-to desa-rro-lla-do desa-rro-llar des-pu dia-go-nal
dia-go-na-les di-fe-ren-cia-ble di-fe-ren-cia-bles di-fe-ren-cial
di-fe-ren-cia-les di-fe-ren-te di-fe-ren-tes dis-cre-ta dis-cre-tas
dis-cre-to dis-cre-tos di-vi-si-bi-li-dad di-vi-si-ble ele-men-tal
ele-men-ta-les ele-men-to ele-men-tos equi-va-len-cia equi-va-lente
equi-va-lentes equi-va-rian-te equi-va-rian-tes eu-cli-dia-na
eu-cli-dia-nas eu-cli-dia-no eu-cli-dia-nos Fi-gu-ra Gal-ois
gal-oi-sia-na ge-ne-rada ge-ne-rado ge-ne-ra-dor ge-ne-ra-do-res
ge-ne-ral ge-ne-ra-les ge-ne-ra-li-dad ge-ne-ra-li-za ge-ne-ra-li-zan
ge-ne-ran ge-ne-rar geo-me-tr geo-me-try Ha-da-mard ho-meo-mor-fis-mo
ho-meo-mor-fo idea-les in-de-pen-dien-te in-de-pen-dien-tes
in-va-rian-cia in-va-rian-te in-va-rian-tes li-ne-a-les
li-ne-al-men-te ma-ne-ra me-dian-te mo-der-no nin-gu-no nues-tra
nues-tro nu-me-ra-ble ope-ra-ci ope-ra-cio-nes ope-ra-dor
ope-ra-do-res or-to-go-nal par-ti-cu-lar pro-ce-di-mien-to pro-duc-to
pro-duc-tos pro-pie-dad pro-pie-da-des pro-po-si-ci re-fe-ren-cia
re-fle-xi-va re-fle-xi-vas re-fle-xi-vo re-fle-xi-vos re-so-lu-ble
res-pec-ti-va-men-te res-pec-ti-vo res-pec-ti-vos res-pec-to
sa-tis-fa-ce sepa-ra-ble sepa-ra-bles si-guien-te si-guien-tes
subes-pa-cio subes-pa-cios te-dra-edro te-tra-edros tri-vial
tri-via-les uti-lidad va-lo-res va-ria-ble va-ria-bles va-rie-dad
va-rie-da-des ve-cin-da-rio ve-cin-da-rios vec-to-rial vec-to-ria-les
vice-versa}


%%% TikZ arrows and such

\pgfarrowsdeclarecombine{twolatex'}{twolatex'}{latex'}{latex'}{latex'}{latex'}
\tikzset{->/.style = {decoration={markings,
                                  mark=at position 1 with {\arrow[scale=2]{latex'}}},
                      postaction={decorate}}}
\tikzset{<-/.style = {decoration={markings,
                                  mark=at position 0 with {\arrowreversed[scale=2]{latex'}}},
                      postaction={decorate}}}
\tikzset{<->/.style = {decoration={markings,
                                   mark=at position 0 with {\arrowreversed[scale=2]{latex'}},
                                   mark=at position 1 with {\arrow[scale=2]{latex'}}},
                       postaction={decorate}}}
\tikzset{->-/.style = {decoration={markings,
                                   mark=at position #1 with {\arrow[scale=2]{latex'}}},
                       postaction={decorate}}}
\tikzset{-<-/.style = {decoration={markings,
                                   mark=at position #1 with {\arrowreversed[scale=2]{latex'}}},
                       postaction={decorate}}}
\tikzset{->>/.style = {decoration={markings,
                                  mark=at position 1 with {\arrow[scale=2]{latex'}}},
                      postaction={decorate}}}
\tikzset{<<-/.style = {decoration={markings,
                                  mark=at position 0 with {\arrowreversed[scale=2]{twolatex'}}},
                      postaction={decorate}}}
\tikzset{<<->>/.style = {decoration={markings,
                                   mark=at position 0 with {\arrowreversed[scale=2]{twolatex'}},
                                   mark=at position 1 with {\arrow[scale=2]{twolatex'}}},
                       postaction={decorate}}}
\tikzset{->>-/.style = {decoration={markings,
                                   mark=at position #1 with {\arrow[scale=2]{twolatex'}}},
                       postaction={decorate}}}
\tikzset{-<<-/.style = {decoration={markings,
                                   mark=at position #1 with {\arrowreversed[scale=2]{twolatex'}}},
                       postaction={decorate}}}

\tikzset{circ/.style = {fill, circle, inner sep = 0, minimum size = 3}}
\tikzset{scirc/.style = {fill, circle, inner sep = 0, minimum size = 1.5}}
\tikzset{mstate/.style={circle, draw, blue, text=black, minimum width=0.7cm}}

\tikzset{eqpic/.style={baseline={([yshift=-.5ex]current bounding box.center)}}}
\tikzset{commutative diagrams/.cd,cdmap/.style={/tikz/column 1/.append style={anchor=base east},/tikz/column 2/.append style={anchor=base west},row sep=tiny}}

\definecolor{mblue}{rgb}{0.2, 0.3, 0.8}
\definecolor{morange}{rgb}{1, 0.5, 0}
\definecolor{mgreen}{rgb}{0.1, 0.4, 0.2}
\definecolor{mred}{rgb}{0.5, 0, 0}

\def\drawcirculararc(#1,#2)(#3,#4)(#5,#6){%
    \pgfmathsetmacro\cA{(#1*#1+#2*#2-#3*#3-#4*#4)/2}%
    \pgfmathsetmacro\cB{(#1*#1+#2*#2-#5*#5-#6*#6)/2}%
    \pgfmathsetmacro\cy{(\cB*(#1-#3)-\cA*(#1-#5))/%
                        ((#2-#6)*(#1-#3)-(#2-#4)*(#1-#5))}%
    \pgfmathsetmacro\cx{(\cA-\cy*(#2-#4))/(#1-#3)}%
    \pgfmathsetmacro\cr{sqrt((#1-\cx)*(#1-\cx)+(#2-\cy)*(#2-\cy))}%
    \pgfmathsetmacro\cA{atan2(#2-\cy,#1-\cx)}%
    \pgfmathsetmacro\cB{atan2(#6-\cy,#5-\cx)}%
    \pgfmathparse{\cB<\cA}%
    \ifnum\pgfmathresult=1
        \pgfmathsetmacro\cB{\cB+360}%
    \fi
    \draw (#1,#2) arc (\cA:\cB:\cr);%
}
\newcommand\getCoord[3]{\newdimen{#1}\newdimen{#2}\pgfextractx{#1}{\pgfpointanchor{#3}{center}}\pgfextracty{#2}{\pgfpointanchor{#3}{center}}}

\newcommand\qedshift{\vspace{-17pt}}
\newcommand\fakeqed{\pushQED{\qed}\qedhere}

\def\Xint#1{\mathchoice
   {\XXint\displaystyle\textstyle{#1}}%
   {\XXint\textstyle\scriptstyle{#1}}%
   {\XXint\scriptstyle\scriptscriptstyle{#1}}%
   {\XXint\scriptscriptstyle\scriptscriptstyle{#1}}%
   \!\int}
\def\XXint#1#2#3{{\setbox0=\hbox{$#1{#2#3}{\int}$}
     \vcenter{\hbox{$#2#3$}}\kern-.5\wd0}}
\def\ddashint{\Xint=}
\def\dashint{\Xint-}

\newcommand\separator{{\centering\rule{2cm}{0.2pt}\vspace{2pt}\par}}

\newenvironment{own}{\color{gray!70!black}}{}

\newcommand\makecenter[1]{\raisebox{-0.5\height}{#1}}

\mathchardef\mdash="2D

\newenvironment{significant}{\begin{center}\begin{minipage}{0.9\textwidth}\centering\em}{\end{minipage}\end{center}}
\DeclareRobustCommand{\rvdots}{%
  \vbox{
    \baselineskip4\p@\lineskiplimit\z@
    \kern-\p@
    \hbox{.}\hbox{.}\hbox{.}
  }}
\DeclareRobustCommand\tph[3]{{\texorpdfstring{#1}{#2}}}
\def\BState{\State\hskip-\ALG@thistlm}

\makeatother 

\begin{document}
%\clearpage
\maketitle
%\thispagestyle{empty}
%test
{\small 
\setlength{\parindent}{0em}
\setlength{\parskip}{1em}

This is a topics course on this stuff

\subsubsection*{Requirements}
Knowledge on stuff\par 

\textbf{TO DO:}
\begin{itemize}
    \item Write 10.1
\end{itemize}
}
\newpage
\tableofcontents
%\begin{multicols}{2}
\chapter{Renzo's Complex Projective Exercises}

\section{Set of points of the projective line}

\begin{Ej}
    Show that there is a bijection between the set $\text{Set}\bCP^1$
 and a quotient set of a disjoint union of two copies of $\bC$.
\end{Ej}

\begin{ptcbr}
    Indeed consider our copies of $\bC$ embedded into $\bC^2$ as the lines $\set{x=1}$ and $\set{y=1}$. Then for a line $\l\in\bCP^1$ our map is 
    $$\l\mapsto \l\cap(\text{corresponding line})\mapsto (\text{corresponding coordinate}).$$
    Explicitly, if our line is $[X:Y]$, then the map is $[X:Y]\mapsto X/Y$ on one chart while $Y/X$ on the other.\par 
    Observe that this map is surjective as every point in each copy of $\bC$ is hit by a line of a different slope. The only points which are not hit twice are the origins of both lines. From this, we define the quotient by identifying the coordinates as $x\sim y$ whenever $y=1/x$. Thus our map becomes a bijection at the level of the quotient as we can now properly trace back each point to a particular line. 
\end{ptcbr}
\section{The projective line as a topological space}

\begin{Ej}[Hopf Fibration]
    Show there is a fibration of topological spaces:
    $$S^1\to S^3\to S^2$$
    meaning that there is a surjective continuous function from the three-dimensional sphere to the two-dimensional sphere, and the inverse image of any point is homeomorphic to a circle. This is called the Hopf fibration; notice that while the construction of these maps is rather mysterious in terms of spheres, it becomes transparent when thinking of the two-dimensional sphere as the complex projective line.
\end{Ej}
%https://math.stackexchange.com/questions/4142758/proving-that-the-hopf-fibration-is-a-fiber-bundle
\begin{ptcbr}
We have shown that the map 
$$\pi_2\: S^3_\bR\to\bCP^1,(\text{pt. in }S^3)\mapsto(\text{corresponding line through origin in }\bC^2)$$
is surjective. Also, we have that 
$$\bCP^1\isom \bC\cup\set{\infty}\isom S^2$$
where the first homeomorphism comes from previous discussion and the second one from stereographic projection. This means that we have a map $S^3\to S^2$ which is our candidate for the Hopf map. It remains to be seen that this map is continuous and that the fibers are homeomorphic to $S^1$.\par 
It suffices to show $\pi_2$ is continuous as the rest of the maps are homeomorphisms. To that effect, take an open set $U\subseteq\bCP^1$. This means that in the quotient topology induced by the $\pi_2$ map, $U$ is open whenever $\pi_2^{-1}(U)$ is open. But this proves immediately that $\pi_2$ is continuous as it takes open sets back to open sets.\par 
Now let $\l\in\bCP^1$, we'll analyze what the fibers are:
$$\pi_2^{-1}(\l)=\set{\la z_0\:\ z_0\in\l\cap S^3,\ \la\in\bC}$$
but when restricting to $S^3$, we get the condition that $\la$ can only vary an $S^1$'s worth of values:
$$\pi_2^{-1}(\l)=\set{\la z_0\:\ z_0\in\l\cap S^3,\ |\la|=1}\isom S^1.$$
This means that fibers of our map are homeomorphic to $S^1$ and thus we have the desired fibration structure.
\end{ptcbr}
\newpage
\section{The projective line as a complex manifold}

\begin{Ej}
    Compute 
    $$\phi_{21}\defeq \vf_2\circ\vf_1^{-1}\mid_{\vf_1(U_1\cap U_2)}\: (\vf_1(U_1\cap U_2),x)\to(\bC,y)$$
    and show that it is a holomorphic function on its domain of definition. Show that its inverse is also holomorphic on its domain of definition.
    
    These exercises show that $\bCP^1$
     has the structure of a complex analytic manifold.
    The pairs $(U_i,\vf_i)$
     are called complex charts, the biholomorphic map $\phi_{21}$
    a transition function, and the coordinates $x$
     and $y$
     are called local (or affine) coordinates.
\end{Ej}
\begin{ptcbr}
Recall that the open set $U_1\cap U_2$ is the collection of lines in $\bCP^1$ which are not $x$ or $y$ axes. The image then is all the non-zero $x$ coordinates of the intersection of those lines with $x=1$. Taking those lines through $\vf_2\circ\vf_1^{-1}$ gives us the $y$-coordinates of the intersections of those lines with the $y=1$ line. We get all except the $y$-axis. Computing this for a particular line, if $x_0$ is the intersection with $x=1$, then $\frac{1}{x_0}$ will be the intersection with $y=1$. Therefore, the map $\vf_2\circ\vf_1^{-1}$ is $x_0\mapsto\frac{1}{x_0}$ of non-zero $x_0$. This function and its inverse are holomorphic as the vertical and horizontal lines are excluded from this.
\end{ptcbr}
\section{Functions on the projective line}

\begin{Ej}
    Show that meromorphic functions $f\: \bCP^1\to\bC$
    may be described in two equivalent ways:
    \begin{enumerate}
        \item As the ratio of two homogeneous polynomials of the same degree in the homogeneous coordinates:
        $$f(X:Y)=\frac{P_d(X,Y)}{Q_d(X,Y)}$$
        \item As a rational function in one of the affine coordinates (with no restrictions on the degrees of the polynomials)
        $$f(x)=\frac{p(x)}{q(x)}$$
    \end{enumerate}
   How do you go from one perspective to the other?
\end{Ej}

\begin{ptcbr}
We begin with the second item by claiming that if $f$ is meromorphic has a zero of degree $m$ at $z=z_0$ then we may write $f=(z-z_0)^mg$ where $g$ is meromorphic but has no zeroes at $z_0$. Similarly for poles. This means that we may write $f$ as a product of possibly repeated linear factors over another product of linear factors. These products are the desired polynomials.\par
We may homogenize to obtain the first characterization.\par   
\red{I don't recall how to construct the function in the homogeneous way :(}
\end{ptcbr}
\newpage
\section{Automorphisms of the projective line}

\begin{Ej}
    Prove that, given any two ordered triple of points $P_1,P_2,P_3$
 and $Q_1,Q_2,Q_3$
of the projective line, there exists a unique automorphism $\Phi$
 of the projective line such that $\Phi(P_i)=Q_i$.
Show that it follows that the only automorphism that fixes three points is the identity.
Describe the subgroups of $\Aut(\bCP^1)$
 consisting of automorphisms that fix one, or two points in the projective line.
\end{Ej}

\begin{ptcbr}
    In affine coordinates, we can map any triple to $0,1,\infty$ by considering the function 
    $$z\mapsto\frac{z-p_1}{z-p_3}\left(\frac{p_2-p_3}{p_2-p_1}\right).$$
    This maps $p_1,p_2,p_3$ to $0,1,\infty$ respectively. This is a Möbius transformation, so it is an automorphism of $\bCP^1$. Call it $\vf_P$ and then create $\vf_Q$, the desired function $\Phi$ is $\vf_Q^{-1}\vf_P$.\par
    From this we immediately see that is $P$ is fixed then the function is 
    $$\Phi=\vf_P^{-1}\vf_P=\id.$$
    If $\Phi$ fixed two points we get rotations about the axis passing through those two points. Be it, for example, $0,\infty$ with scalings $z\mapsto \al z$ or $1,-1$ with $z\mapsto 1/z$.\par
    If only one point is fixed, then it is a translation of the line leaving that point fixed. Say for example maps of the form $z\mapsto \frac{az+b}{d}$ leave infinity fixed.
\end{ptcbr}
\section{Maps to projective spaces}

\begin{Ej}
    We define the degree of $F(\bCP^1)$ to be the number of intersections with a general hyperplane in $\bCP^r$.  Prove that if the degree of the polynomials $P_i(X,Y)$ is equal to $d$, then the degree of $F(\bCP^1)$ is less than or equal to $d$. When does the strict inequality hold?
\end{Ej}

\begin{ptcbr}
    
\end{ptcbr}
\newpage
\section{Line bundles on the projective line}

\begin{Ej}
    For $x_0\neq 0$, let $i_1\:\set{x=x_0}\into\cO_{\bCP^1}(d)$ and $i_2\:\set{y=\frac{1}{x_0}}\into\cO_{\bCP^1}(d)$ be the two inclusions of vertical lines. Show that $i_2^{-1}\circ i_1\:\bC\to\bC$ is a linear isomorphism. Observe that the collection of these linear isomorphisms defines a holomorphic function $c_{12}\:\bCP^1\less\set{0,\infty}\to\bC\less 0$.
\end{Ej}

\begin{ptcbr}
    Let's concretely analyze the $i_2^{-1}\circ i_1$ map. This comes out of the line $\set{x=x_0}$ and gives us 
    $$\Im i_1=\set{(x_0,u)\: u\in\bC}\subseteq (\bC^2,(x,u)).$$
    In order to see what $i_2^{-1}$ does, we translate $\Im i_1$ into $(\bC^2,(y,v))$ via
    $$x\mapsto \frac{1}{y},\quad u\mapsto\frac{v}{y^d}\To v=y^du.$$
    So $\Im i_1$ on the other chart is 
    $$\Im i_1=\Set{\left(\frac{1}{x_0},\frac{u}{x_0^d}\right)\: u\in\bC}$$
    and $i_2^{-1}$ returns us $\frac{u}{x_0^d}$. This means that the composition in question is that map $u\mapsto \frac{u}{x_0^d}$ which means that the map is multiplication by $x_0^{-d}$. For a fixed non-zero $x_0$, this is a linear isomorphism of $\bC$.\par
    The collection of such isomorphisms is obtained when we let $x_0$ vary and the function $c_{12}$ given by $x_0\in\bCP^{1}\less\set{0,\infty}\mapsto x_0^{-d}\in\bC\less\set{0}$ is indeed holomorphic.
\end{ptcbr}
\section{Sections of line bundles}

\begin{Ej}
    Show that if $s_0,s_1$ are two sections of the same line bundle, then their ratio is a (meromorphic) function on $\bCP^1$. Show that if $s_0,s_1,\dots,s_r$
 are $(r+1)$ sections of the same line bundle, then they define a map $\bCP^1\to\bCP^r$.
\end{Ej}

\begin{ptcbr}
    Observe that if $z\in\bCP^1$, then $s_0(z),s_1(z)$ lie on the fiber $\pi^{-1}(z)$ which is isomorphic to $\bC$. So taking their ratio on this fiber does produce a complex number. However, we must verify that the ratio is well-defined. Assume we picked another element of the base, $w\in\bCP^1$ and asked about the ratio of $s_0(w)$ with $s_1(w)$. In this case, observe that there's a linear isomorphism between $\pi^{-1}(z)$ and $\pi^{-1}(w)$ which scales all vectors by the same length. This means that $s_i(w)=\al s_i(z)$ for some $\al\in\bC$ and therefore their ratios are the same.\par
    So $s_0/s_1$ does define a meromorphic function on $\bCP^1$ thanks to the linear isomorphisms.\par
    In the same fashion, if we instead have $r+1$ sections, via a same argument we can see that when changing fibers, the sections only change by a scaling which is the same on all entries. So this means that we may write $\bonj{s_0:\dots:s_r}$ as a function to $\bCP^r$.
\end{ptcbr}

\section{Divisors}

\begin{Ej}
    When you multiply two meromorphic functions, what happens to their divisors? If two meromorphic functions produce the same divisor, what can you say about them?
\end{Ej}

\begin{ptcbr}
    If we consider our functions as sections of $\cO(d)$ then,
    $$\displaystyle s_0(x)=\frac{\prod_{i=1}^{m}(x-\al_i)}{\prod_{j=1}^{n}(x-\bt_j)}\word{and}s_1(x)=\frac{\prod_{i=1}^{r}(x-\ga_i)}{\prod_{j=1}^{s}(x-\dl_j)}$$
    then their divisors are 
    $$\div(s_0)=\sum_{i=1}^{m}[\al_i]-\sum_{j=1}^{n}[\bt_j]+(n-m+d)[\infty],\ \div(s_1)=\sum_{i=1}^{r}[\ga_i]-\sum_{j=1}^{s}[\dl_j]+(s-r+d)[\infty].$$
    Where we see the degree at infinity by transitioning via $s(x)\mapsto y^ds(1/y)$.
    Their product is precisely 
    $$s_0(x)s_1(x)=\frac{\prod_{i=1}^{m+r}(x-\tilde\al_i)}{\prod_{j=1}^{n+s}(x-\tilde\bt_j)},$$
    where $\tilde{\al}_i$ corresponds to $\al_i$ when $i$ is between 1 and $m$ and $\ga_i$ from $m+1$ onwards. The same happens for $\tilde{\bt}_j$. In this case the divisor is 
    $$\div(s_0)=\sum_{i=1}^{m+r}[\tilde{\al}_i]-\sum_{j=1}^{n+s}[\tilde{\bt}_j]+((n+s)-(m+r)+d)[\infty]$$
    and the $x=\infty$ still gets the correct coefficient after using the transition function. \red{¿But why don't we get $2[\infty]$? ¿Is it because this only works in $\cO(0)$, the trivial bundle?}\par
    So all non-infinite coefficients do get added and the divisors are summed.\par
    When two meromorphic functions have the same divisor, they should be scalar multiples of each other at most. 
\end{ptcbr}

\begin{Ej}
    Let $s$ be a meromorphic section of a line bundle $\pi\: L\to\bCP^1$, we call the support of $\div(s)$ the set of points that appear with non-zero coefficients in $\div(s)$. Show that there is a natural bijection 
    $$T_s\:\pi^{-1}\bonj{\bCP^1\less\supp(\div(s))}\to\bCP^1\less\supp(\div(s))\x\bC.$$
Meditate on the following fact: the function $T_s$ 
 is an isomorphism of complex manifolds, and in fact an isomorphism of line bundles on the punctured $\bCP^1$
 (we of course did not precisely define these notions, so try and make a guess of what these things should mean). It is called a \emph{trivialization}  of the line bundle $\pi\: L\to\bCP^1$
 on the complement of the support of $\div(s)$.
\end{Ej}

\begin{ptcbr}
    The first intuitive way to define the function is take an isomorphism $\vf$ of the fiber $\pi^{-1}(x)$ with $\bC$. This gives us the map 
    $$T_s(x)=(\pi(x),\vf(x)),$$
    but this map is very non-canonical and ALSO, doesn't depend on $s$. So let us take advantage of a couple of points that we know in the fiber: $x$ and wherever $s$ maps $\pi(x)$ to, $s(\pi(x))$. Outside the support of $s$, $s$ is never zero, so it makes sense to define the quotient $\vf(x)/\vf(s(\pi(x)))$. The desired function is then 
    $$T_s(x)=\left(\pi(x),\frac{\vf(x)}{\vf(s(x))}\right)$$
    which is independent of $\vf$ \red{But, ¿why was this?}, making it natural in that sense. This function is bijective as 
    $$\left(\pi(x),\frac{\vf(x)}{\vf(s(x))}\right)=\left(\pi(y),\frac{\vf(y)}{\vf(s(y))}\right)$$
    implies $x,y$ lie in the same fiber via the first component. This immediately gives us $s(x)=s(y)$ and from this 
    $$\frac{\vf(x)}{\vf(s(x))}=\frac{\vf(y)}{\vf(s(x))}\To \vf(x)=\vf(y)\To x=y.$$
\end{ptcbr}

\section{Distinguished line bundles on the projective line}

\begin{Ej}
    Recall that the very first definition of the set of points of the projective line is that each point corresponds to a line in $\bC^2$. We now want to construct a space $\operatorname{Taut}(\bCP^1)$ whose points correspond to the choice of line in $\bC^2$ together with a point on it:
    $$\text{Set}\left(\Taut(\bCP^1)\right)=\set{(\l,P)\:\ \l\text{ is a line through the origin in }\bC^2,\ P\in\l}.$$
    \begin{enumerate}
        \item Realize $\Taut(\bCP^1)$ as a subspace of $\bCP^1\x\bC^2$.
        \item Show that the first projection restricts to $\Taut(\bCP^1)$ to make it into a line bundle on $\bCP^1$.
        \item Describe the fiber over a point $[\l]\in\bCP^1$.
        \item Show that $\Taut(\bCP^1)\isom \cO(-1)$.
        \item Show that $\Taut(\bCP^1)\isom \Bl_{(0,0)}\bC^2$.
    \end{enumerate}
\end{Ej}

\begin{ptcbr}
    \begin{enumerate}
        \item As mentioned, elements of $\text{Set}\Taut(\bCP^1)$ are $A=(\l,P)$. This immediately defines a map 
        $$\Taut(\bCP^1)\hookto\bCP^1\x\bC^2,\ A\mapsto (\l,P).$$
        \red{I had this question for a bit, but, how can I show that this map is a map of projective varieties?}
        \item The map 
        $$\Taut(\bCP^1)\to\bCP^1,\ (\l,P)\mapsto \l$$
        describes a structure of a line bundle. To show this we must see that for an open neighborhood $U$ of $\l\in\bCP^1$, we have that $\pi^{-1}(U)\isom U\x\bC^1$. We define the map as 
        $$U\x\bC^1\to\pi^{-1}(U),\ (\l,z)\mapsto(\l,zP),\ P\in\l.$$
    \end{enumerate}
\end{ptcbr}

\section{The Picard group of the projective line}

\begin{Ej}
    Show that the set of isomorphism classes of line bundles on $\bCP^1$ forms a group under the operation of tensoring. This group is called the \term{Picard group}, denoted $\Pic(\bCP^1)$. Show $\Pic(\bCP^1)\isom\bZ$.
\end{Ej}

\begin{ptcbr}
    In order to show $\Pic(\bCP^1)$ is a group we must show that there's a closed associative binary operation with an identity and inverses.\par
    It is the case that the tensor product of line bundles over $\bCP^1$ is another line bundle. Associativity of the tensor product gives us our desired property for free. And now for an element $\cO_{\bCP^1}(d)$, we would like to find an identity element $L$ such that $\cO_{\bCP^1}(d)\ox L\isom \cO_{\bCP^1}(d)$. As
    $$\cO_{\bCP^1}(d_1)\ox\cO_{\bCP^1}(d_2)\isom\cO_{\bCP^1}(d_1+d_2)$$
    it must occur that the line bundle we must tensor $\cO_{\bCP^1}(d)$ with is $\cO_{\bCP^1}(0)$, the trivial bundle. Following the same reasoning the inverse element of $\cO_{\bCP^1}(d)$ is $\cO_{\bCP^1}(-d)$.\par
    The map $\vf:d\mapsto\cO_{\bCP^1}(d)$ is a group homomorphism as $d_1-d_2$ gets sent to
    $$\cO_{\bCP^1}(d_1-d_2)=\cO_{\bCP^1}(d_1+(-d_2))\isom\cO_{\bCP^1}(d_1)\ox\cO_{\bCP^1}(-d_2)=\vf(d_1)\ox\vf(d_2)^{-1}.$$
    This map is surjective as every line bundle is $\cO_{\bCP^1}(d)$ for some $d$. And if two line bundles are isomorphic, their spaces of sections must be isomorphic and therefore $d_1=d_2$.
\end{ptcbr}

\section{Sheaf cohomology of the projective line}

\begin{Ej}
    Show that $H^0(\bCP^1,\cO_{\bCP^1}(d))$ is equal to the space of holomorphic (global) sections of $\pi\:\cO_{\bCP^1}(d)\to\bCP^1$.
\end{Ej}

\begin{ptcbr}
    When building the \v{C}ech complex for $\cO_{\bCP^1}(d)$ we can see that $H^0(\bCP^1,\cO_{\bCP^1}(d))\isom\ker\dl$ where
    $$\dl\:\cO_{\bCP^1}(d)(U_0)\oplus\cO_{\bCP^1}(d)(U_1)\to\cO_{\bCP^1}(d)(U_0\cap U_1),\ (s_0,s_1)\mapsto s_{0\mid U_0\cap U_1}-s_{1\mid U_0\cap U_1}.$$
    For our purpose, let us show that global sections correspond to $\ker\dl$. Take a global holomorphic section $s\in\Ga(\bCP^1,\cO_{\bCP^1}(d))$ then we may decompose $s$ by restricting into $U_0,U_1$ to get $(s_{\mid U_0},s_{\mid U_1})$. Mapping this pair through $\dl$ gives us 
    $$\dl(s_{\mid U_0},s_{\mid U_1})=s_{\mid U_0\cap U_1}-s_{\mid U_0\cap U_1}=0.$$
    Which means that any global section is in $\ker(\dl)$. On the other hand, pick $(s_0,s_1)$ local sections such that 
    $$s_{0\mid U_0\cap U_1}=s_{1\mid U_0\cap U_1}.$$
    We may now define a global section as 
    $$s(x)=\left\lbrace
    \begin{aligned}
        &s_0(x),\word{if}x\in U_0,\\
        &s_1(x),\word{if}x\in U_1.
    \end{aligned}
    \right.$$
    The fact that $(s_0,s_1)\in\ker(\dl)$ implies that $s$ is well defined everywhere. Therefore $s$ is our desired global section constructed from $(s_0,s_1)$. This shows that global sections are in correspondence with $H^0$.
\end{ptcbr}

\section{Tautological Bundle}

The graph of the function 
$$F\: \bC^2\less\set{0}\to\bP^1,\ (X,Y)\mapsto[X:Y]$$
is exactly $(X,Y),[S:T]$ such that 
$$F(X,Y)=[S:T]\To\exists\la\neq 0(X=\la S,\ Y=\la T).$$
If we assume that $T\neq 0$, we can divide by $T$ and then get $Y/T=X/S$. This also assumes $S\neq 0$, but ok, the points in the closure of this set is obtaineed by clearing denominators. This leads us to $SY=TX$. Spending some time carefully doing this, if one coordinate is zero and the other one isn't, the equation is still satisfied. Even points $(0,0)$, whatever $S$ and whatever $T$, are also in the closure. This means that $(X,Y)$ belongs to the line $[S:T]$, even if $(X,Y)=(0,0)$, it belongs to every line.


\section{A quick recap}

Originally, we saw $\bCP^1$ as a complex manifold of dimension 1, we studied things by restricting them to charts and reducing problems to complex analysis. Locally around every point we had complex numbers.\par
We like to study functions on the space because this translate geometric data to algebraic data. For example we understand $\bR^2$ by considering the \emph{coordinate functions}! Take a point $P$ where the $x$ function takes value $33$ and the $y$ function $-37$.\par
Bummer, $\bCP^1$ has few functions. Only the constants in fact. This gives us directions of study, meromorphic functions for example give us the whole collection $\bC(x)$. This field of rational functions is a birational invariant of $\bCP^1$, in a sense it's a very rich invariant but it's insensitive to small changes in the space.\par
Otherwise we can consider the local functions. For every open set, consider holomorphic functions on that open set. We have a lot of them, and they are not completely unrelated. There's actually a whole series of connections between these open sets. This is the notion of a sheaf. 
$$\text{regular}\leftrightarrow\text{holomorphic}\word{and}\text{rational}\leftrightarrow\text{meromorphic}.$$
So now, $(\bCP^1,\cO)$ is a locally ringed space which is the datum of a scheme.\par
The third perspective is to consider sections of a line bundle instead of functions! This comes from replacing a function with its graph.
$$F\:X\to Y\leftrightarrow i_{\Ga_F}\: X\to\Ga_F\subseteq X\x Y,$$
we have the same information of a function as $x\mapsto(x,F(x))$. The idea is that we will allow the graph to not live in the product $\bCP^1\x\bC$ but instead on a space which \emph{locally} looks like that. We allow the graph of our \emph{kinda} function to live in spaces that are not necessarilly products $\bCP^1\x\bC$ but locally are products with $\bC$.\par
Sections of a line bundle give us maps to projective space. $r+1$ sections give us a map to $\bP^r$. This gets us to another invariant, the Picard group $\Pic(\bCP^1)$. This is the group of isomorphism classes of line bundles. For a line bundle, we can compute the \v{C}ech cohomology which give us another powerful invariant. Most of our work is to construct such invariants. We don't have a complete set of invariants for algebraic varieties.

\subsection{¿Why do divisors determine the line bundle?}

Given a divisor $D$ we can construct a line bundle $L$ with a section $s$ such that $\div(s)=D$. 

\begin{Ex}
Consider the base $\bCP^1$ and the divisor $D=d[\infty]$. Over $\bCP^1$ we have $\supp(D)=\set{\infty}$ so we build two open sets
$$U_\infty=\text{neighborhood of }\supp(D),\word{and} U_0=\bCP^1\less U_0.$$
In the chart $U_0$ we have the zero section and if there was a non-zero section $s$, then we might as well rescale all the fibers so that the section becomes $s(x)=1$. This is done by rescaling the fiber by $\frac{1}{s(x)}$. About $U_\infty$, we have that our section is $\set{v=y^d}$. So a good transformation would be 
$$y=\frac{1}{x},\ v=uf(x)=u/x^d.$$
\end{Ex}

In general for an arbitrary base of dimension $1$, we have a divisor 
$$D=\sum_{i=1}^n a_iP_i.$$
To construct $L$, we need an open cover of $X$ plus transition functions for any pair of intersecting sets. Let us take then 
$$U_0=\set{p_i}_{i\in\bonj{n}}^c,\ U_i=\text{open disk about }p_i.$$

\red{Ask Simeon for video}

\section{Serre Duality}

We have observed that 

$$\dim(H^0(\bP^1,\cO(d)))=\dim(H^1(\bP^1,\cO(-d-2))).$$

This follows from the fact that $H^0(\bP^1,\cO(d))$ is always isomorphic to the dual of $H^1(\bP^1,\cO(-d)\ox T^\ast_{\bP^1})$. This is also sometimes denoted the canonical divisor of $\bP^1$: 
And a modification if we have $X$ of dimension greater than 1, then we want to write this as
$$H^i(X,\cL)\isom H^{n-i}(X,\cL'\ox K_X)$$
where $K_X\isom\bigwedge^nT^\ast X$ is the canonical sheaf. This amounts to having a perfect pairing (non-degenerate bilinear form), a bilinear map 
$$H^0(\bP^1,\cO(d))\x H^1(\bP^1,\cO(-d)\ox T^\ast_{\bP^1})\to\bC$$
which means that if we represent this map by a basis and a matrix, the matrix is invertible. There is no vector on the left which maps to zero with a vector on the right unless it's zero. It's precisely like a dot product.\par
We have two ingredients to show this:
\begin{enumerate}
    \item We can identity $\bC$ with $H^1(\bP^1,T^\ast_{\bP^1})$. The particular cohomology group is one-dimensional.
    \item We have a natural map from the cartesian product to that $H^1$.
\end{enumerate}
Spelling this out carefully will give us our natural map. Even if we have $T^\ast_{\bP^1}\isom\cO(-2)$ here, we will see it with differential forms in order to apply it to other spaces.\par
In the more general case what we have is a map 
$$H^i(X,\cL_1)\x H^j(X,\cL_2)\to H^{i+j}(X,\cL_1\ox\cL_2).$$
Let's rewrite the \v{C}ech complex for the cotangent bundle:
$$T^\ast_{\bP^1}(U_0)\oplus T^\ast_{\bP^1}(U_\infty)\to T^\ast_{\bP^1}(U_0\cap U_\infty)$$
This is 
$$\bC[x]\dd x\oplus \bC[y]\dd y\to\bC[x,1/x]\dd x$$
and the map acts on monomials as 
$$
\left\lbrace
\begin{aligned}
    &(x^k\dd x,0)\mapsto x^k\dd x\\
    &(0,y^\l\dd y)\mapsto -(-x^{-\l}\dd x/x^2)=x^{-2-\l}\dd x.
\end{aligned}
\right.
$$
At the end of the day $H^1(\bP^1,T^\ast_{\bP^1})$ is represented by the cocycle $\dd x/x$. So $H^1(\bP^1,T^\ast_{\bP^1})=\genr{\dd x/x}_{U_0\cap U_\infty}$. The only monomial we are not catching is $1/x$, every other we've caught.\par
If we did this whole process tensoring with $\cO(-d)\ox-$, then on left we still have functions times $\dd x$ and functions times $\dd y$. The $y$ now transitions as 
$$(0,y^\l\dd y)\mapsto -(-x^{-\l-d}\dd x/x^2)=x^{-2-\l-d}\dd x.$$
So in this case, for example in $d=5$, we will not catch things between $-1$ and $-5-1$:
$$H^1(\bP^1,\cO(-5)\ox T^{\ast}_{\bP^1})\isom\genr{\frac{\dd x}{x},\frac{\dd x}{x^2},\dots,\frac{\dd x}{x^6}}.$$
We don't get $\dd x/x^7$ because it's in the image of the map!\par
Now we would like to show that we have a map 
$$H^0(\bP^1,\cL_1)\x H^1(\bP^1,\cL_2)\to H^1(\bP^1,\cL_1\ox\cL_2)$$
at the level of the \v{C}ech complex:
$$C^0(\bP^1,\cL_1)\x C^1(\bP^1,\cL_2)\to C^1(\bP^1,\cL_1\ox\cL_2).$$
Elements of the first group are $\set{s_0,s_\infty}$ where $s_0\in L_1(U_0)$ and $s_\infty\in L_1(U_\infty)$ and on the other we have $u_{0\infty}\in L_2(U_0\cap U_\infty)$. Given this we want to produce a section $v_{0\infty}$ of $\cL_1\ox\cL_2(U_0\cap U_\infty)$.\par
We can take $s_0$ and restrict it: $s_0\mid_{U_0\cap U_\infty}\.u_{0\infty}$. This gives us a $v_{0\infty}$.\par
Our choice was biased! ¿Why didn't we choose $s_\infty$? The fact that $$\set{s_0,s_\infty}\in\ker d$$
means that $s_0\mid_{U_0\cap U_\infty}-s_\infty\mid_{U_0\cap U_\infty}=0$ which means that on the intersection they're the same.\par
So basically we have 
\begin{gather*}
H^0(\cO(d))\x H^1(\cO(-d)\ox T^\ast)\to H^1(T^\ast)\\
\To \genr{1,x,\dots,x^{d}}_{U_0}\ox\genr{\frac{\dd x}{x},\frac{\dd x}{x^2},\dots,\frac{\dd x}{x^{d+1}}}\to\genr{\frac{\dd x}{x}}.
\end{gather*}
The matrix of this pairing is the identity matrix and then it's obviously the \emph{dot product}. All the other $x^k\dd x=0$.

\chapter{Higher Genus}

\section{Riemann surfaces}

\begin{Def}
A \term{Riemann surface} is a complex analytic manifold of dimension $1$. 
\end{Def}

For every point, there's a neighborhood which is isomorphic to $\bC$ and transition functions are linear isomorphisms of $\bC$.

\begin{Ex}
    The following classes define Riemann surfaces.
    \begin{enumerate}
    \item $\bC$ itself is a Riemann surface with one chart.
    \item Any open set of $\bC$ is a Riemann surface.
    \item A holomorphic function $f\: U\subseteq\bC\to\bC$ defines a Riemann surface by considering $\Ga_f\subseteq\bC^2$. There's only one chart determined by the projection and the inclusion $i_{\Ga_f}$ is its inverse.
    \item Take another holomorphic function $f$, then $\set{f(x,y)=0}$ is a Riemann surface such that 
    $$\text{Sing}(f)=\set{\del_xf=\del_yf=f=0}=\emptyset.$$
    This means that at every point the gradient identifies a normal direction to the level set $f=0$. In particular, there's a well defined tangent line. To show that this is a complex manifold, we will use the inverse function theorem. 
    \item The first compact example is $\bCP^1$.
    \end{enumerate}
\end{Ex}

\section{20241009}

Our first examples of non-compact Riemann surfaces are images of holomorphic functions $\bC^2\to\bC$ such that $\text{Sing}(f)=\emptyset$. These implies that 
$$V(f)=\set{(x,y)\in\bC^2\: f(x,y)=0}$$
is a Riemann surface via the implicit function theorem.

\subsection{Compact Riemann surfaces}

Our first example is $\bCP^1$. But the next one is complex tori $\bC/\La$. Here $\La$ is a non-degenerate lattice:
$$\La= z_1\bZ\oplus z_2\bZ,\quad z_1,z_2\in\bC$$
where these numbers are linearly independent over $\bR$. This quotient is given by the equivalence relation 
$$x\sim y\iff x-y\in\La.$$
Another way to see this is to choose a fundamental domain which is the paralellogram $0,z_1,z_2,z_1+z_2$. Any point in $\bC$ is equivalent to a point inside, the pairs of parallel edges are equivalent and all the vertices are equivalent as well. Essentially what we are doing is building a torus (a real 2-torus) with this paralellogram.\par
This is a Riemann surface because $\bC$ induces a natural atlas via the quotient. For a point $x\in\bC/\La$ we get a chart by:
\begin{enumerate}
    \item Picking a point $z_x\in\bC$ a representative of the equivalence class of $x$.
    \item Then take a neighborhood about $z_x$, and call $U_x$ its image in $\bC/\La$
    \item Consider the inverse of the projection function $U_x\to\bC$ as our chart.
\end{enumerate}
Now pick a $y$ and a representative $z_y$. For elements in the intersection of the neighborhood, transition functions are given by translations.

\begin{Ex}
    Projective plane curves will be our next example. If we let $F\in\pre{h}\bC_d[X,Y,Z]$ be a homogeneous, degree $d$ polynomial such that $\text{Crit}(F)=\emptyset$ in $\bP^2$, then 
    $$V(F)=\set{[X:Y:Z]:\ F(X,Y,Z)=0}$$
    is a Riemann surface. To prove this, we rely on the implicit function theorem. We dehomogenize $F$ and then chart it via 
    $$V(F)\cap U_z=\set{(x,y)\:\ F(x,y,1)=0}.$$
\end{Ex}

\begin{Rmk}
Recall that homogeneous $\text{Crit}(F)$ contains $V(F)$ because $F$ is in the image of the Euler operator $x_i\del_i=d\id$. 
\end{Rmk}

\begin{Ex}
    Complete intersections of $(n-1)$ hypersurfaces in $\bP^n$ are the generalize the previous example. Take $F_i\in\pre{h}\bC_{d_i}[\un x]$ and 
    $$V(F_1,\dots,F_{n-1})=\set{[\un{X}]\in\bP^n\: F_1(\un X)=\dots=F_{n-1}(\un X)=0}$$
    then this is a Riemann surface. There is however a condition on the $F_i$'s, the gradient of $F_i$ at $P$ must give us different directions. This means that the matrix whose columns are $\nb_p F_i$, $i\in\bonj{n-1}$, has full rank. 
\end{Ex}

\begin{Ej}
    Find an expression in terms of determinants for that condition.
\end{Ej}

\begin{Ex}
    As a last example, we have local complete intersections. Consider
    $$C=V(X_0X_3-X_1X_2,X_0X_2-X_1^2,X_1X_3-X_2^2)\subseteq\bP^3,$$
    this is the image of the rational normal curve 
    $$\vf\:\bP^1\to\bP^3,\ [s:t]\mapsto[s^3:s^2t:st^2:t^3].$$
    Choosing only two of the equations, for example the first two, if we set $X_0=X_1=0$ we get a whole line's worth of points because $X_2,X_3$ are free to vary. So this gives us the curve plus a line. As it is the image of a map, then it's a Riemann surface.\par
    If we take any point on the curve besides the intersection point, we do get a complete intersection.
\end{Ex}

\begin{Ej}
    Verify that the image satisfied the polynomial equations.
\end{Ej}

\section{20241011}

Every space is locally approximated by tangent spaces. So giving orientation of manifolds is giving an orientation to all tangent in way that is coherent. And of course, in a chart, the tangent bundle trivializes so then we spread the orientation via fibers. On each individual chart we choose an orientation. But between charts we need to look at transition functions.\par
If for every transition function, the determinant of the Jacobian of the transition is positive implies that all are orientation preserving.

\begin{Lem}
Any Riemann surface is orientable.
\end{Lem}

\begin{ptcbp}
A surface is orientable if and only if for any transition function $\phi_{ij}$ between $U_i,U_j$ we have 
$$\det d\phi_{ij}>0.$$
For a Riemann surface, $\phi_{ij}$ are holomorphic and in particular satisfy Cauchy-Riemann equations. Recall we identity $1=(1,0)$ and $i=(0,1)$ so that $\phi=u+iv$ where $u,v\:\bR^2\to\bR$. \blu{Finish writing derivation for CR eqns via drawing}
$$u_x=v_y\word{and}-u_y=v_x.$$
So for our transition function the Jacobian is 
$$\twobytwo{u_x}{u_y}{v_x}{v_y}=\twobytwo{u_x}{u_y}{-u_y}{u_x}=u_x^2+u_y^2>0.$$
So holomorphicity implies this and therefore any Riemann surface is orientable. 
\end{ptcbp}

Restricting our attention to compact Riemann surfaces we obtain a classification theorem. Non orientable are connected sums of projective planes while orientable ones are connected sums of tori.

\begin{Cor}
Any compact Riemann surface is homeomorphic to a genus $g$ surface.
\end{Cor}

We must interpret connected sum of zero tori as a sphere. Connected sum is surgery from topology. We remove a small open disk from both surfaces and then identifying both boundaries gives us the result.\par
The number of holes we obtain is an important invariant called the \term{genus}. This is a topological invariant.

\begin{Rmk}
    If two R.S. walk about to you with different genera, then they are most certainly non-isomorphic. In genus $1$ there's infinitely many non-isomorphic R.S.
\end{Rmk}

Some questions arise which will be able to answer:

\begin{Qn}
¿Are there Riemann surfaces of any genus?
\end{Qn}

\begin{Qn}
¿Are there general genus Riemann surfaces which are plane projective planes? This is, as the zero locus of polynomials in $\bP^2$.
\end{Qn}

The answer to the first question is yes, and the simplest way to construct one is to construct a hyperelliptic curve of genus $g$. This are built out of equations of the form 
$$y^2=f(x),\quad (x,y)\in\bC^2.$$
Generically this curves admit a degree $2$ map to $\bC$.\par
The second question has a negative answer. This should weird us out as we are defining hyperelliptic curves via a polynomial (this only works if the degree of $f$ is $3$ or $4$). In particular if $C=V(f)$ where $f$ is homogeneous of degree $d$ then 
$$g_C=\binom{d-1}{2}$$
so as not all integers are of the form $\binom{d-1}{2}$, then there's no genus $5$ hyperelliptic curve in $\bP^2$ for example.

\subsection{Families, singular curves and genus}

\begin{Def}
A \term{family of Riemann surfaces} is a space $\gX\xrightarrow{\pi}B$ such that
\begin{enumerate}
    \item For every $b\in B$, $\pi^{-1}(b)$ is a Riemann surface.
\end{enumerate}
\end{Def}

\begin{Ex}
    Consider $F,G\in\pre{h}\bC[X,Y,Z]_d$ then $\la F+\mu G$ is also homogeneous of degree $d$. $V(\la F+\mu G)$ defines a subvariety or a bihomogenous equation (degree $1$ in $\la,\mu$ and degree $d$ in $X,Y,Z$). This is our space $\gX$ in $\bP^1\x\bP^2$ and projecting $\pi_1$ gives us the family.
\end{Ex}

\begin{Ex}
    In a more specficic example we have 
    $$\la\bonj{(Y^2-X^2)Z+X^3}+\mu XZ^2=0.$$
    It is well defined to say whether a point $(\la_0:\mu_0),(X_0:Y_0:Z_0)$ is a solution to the equation above. So it makes sense to ask for points $((\la_0:\mu_0),(X_0:Y_0:Z_0))\in\bP^1\x\bP^2$ solutions of this equation. As the parameters vary we get Riemann surfaces.\par
    However dehomogenizing $(Y^2-X^2)Z+X^3$ gives us $y^2-x^2+x^3$, the nodal curve. The singular locus of this curve contains $(0,0)$, so it's singular at $(0,0)$. About the origin this looks like $\dd y^2=\dd x^2-\dd x^3=\dd x^2$ which is the union of the diagonals. This is a \term{nodal singularity}. It turns out that if we want, we can care only about nodal singularity. If we get a family where curves vary and at some point it degenerates into worse than nodal singularities, then we can replace it with a nodal curve.
\end{Ex}

\section{20241014}

Lat time we were talking about families. We want the thing that \emph{controls} the variation of the Riemann surface to also be a Riemannian manifold. As they vary we would like to form a global object. We shoud think of the variety $B$ to be a parameter space.\par
Our example served to show us that over compact spaces we may have singularities. From a topological point of view, we don't quite know it yet but\dots

\begin{Ex}
A smooth cubic curve in $\bP^2$ has genus 1. As we get closer to zero in the base, one of the loops starts to contract we obtain a node.\par
This pinched torus should have genus $1$. But it's also really close to being a sphere because I've taken two points and pinched them together. For nodal curves the notion of genus becomes two: arithmethic and geoemtric. Arithmethic has to do with algebra and it remains invariant in families. In particular for families resolving the singularity: One of the fibers is our nodal curve and nearby fibers are tori. So this pinched torus has arithmethic genus 1.\par
On the other hand geometric genus stays invariant under birational transformations. In this particular example, there's no birational transformation between the pinched torus and the torus. However there's a birational transformation called normalization which takes the open set of the sphere minus two points to the pinched torus minus the singularity.
\end{Ex}

\subsection{Maps to and from Riemann Surfaces}

We have seen that Riemann surfaces are orientable complex manifolds. 

\begin{Def}
We say that a function $f\: C\to\bC$ is holomorphic at $x$ if at a chart $(U_x,\vf_x)$ we have $(\vf_x^{-1})^\ast(f)$ is holomorphic at $\vf_x(x)$.
\end{Def}

We have a plethora of theorems imported from complex analysis. 

\begin{Prop}
    The following properties hold for holomorphic functions on Riemann surfaces.
\begin{itemize}
    \item Zeroes of holomorphic functions are discrete sets.
    \item If $f$ is non-constant and holomorphic, then it's an open mapping. 
    \item If $C$ is a compact Riemann surface, then $f\: C\to\bC$ being holomorphic implies it's constant. 
\end{itemize}
\end{Prop}

The set of holomorphic functions cannot tell apart two Riemann surfaces.

\begin{ptcbp}
    We will only prove the last item. If $f$ was not constant, $f(C)\subseteq\bC$ would be open. Also, $f(C)$ is compact in $C$. This contradicts the fact that $f$ is non-constant.
\end{ptcbp}

So now we turn our attention to meromorphic functions as holomorphic ones are disappointing. 

\begin{Ex}
    Let us see meromorphic functions on compact Riemann surfaces:
    \begin{enumerate}
        \item Rational functions on $\bCP^1$.
        \item In complex tori: $f\:\bC/\La\to\bC$, we have quotient function. This is function should be a meromorphic function on $\bC$ which is invariant under the torus action. These are called \emph{doubly-periodic}:
        $$f(z+\la)=f(z),\quad\la\in\La.$$
        One such example is \emph{Weierstrass}'s $\wp$ function:
        $$\wp(z)=\frac{1}{z^2}+\sum_{\la\in\La\less\set{0}}\left(\frac{1}{(z-\la)^2}-\frac{1}{\la^2}\right)$$
        which holomorphic outside every lattice point and has poles of order $2$ at any $\la\in\La$. Observe that if $\tilde{\la}\in\La$ then 
        $$\wp(z+\tilde{\la})=\frac{1}{(z+\tilde{\la})^2}+\sum_{\la\in\La\less\set{0}}\left(\frac{1}{(z+\tilde{\la}-\la)^2}-\frac{1}{\la^2}\right)$$
        which is just a reindexing of the previous summation. The $1/z^2$ term appears when we go over the $\la=\tilde{\la}$ term in the sum.\par
        It also occurs that $\wp$ is even. This fact tells us that we get a pole of order $2$ at every point in the lattice, but getting the Laurent expansion tells us that $a_{-1}$, the residue is zero. In other words the expansion looks like
        $$\dots+\frac{a_{-2}}{z^2}+a_0+a_2z^2+\dots.$$
        \item Another meromorphic function is 
        $$\wp'(z)=-2\left(\frac{1}{z^3}+\sum_{\la\in\La\less\set{0}}\frac{1}{(z-\la)^3}\right)$$
        which is now odd, has a pole of order $3$ at every lattice point with expansion 
        $$\frac{a_{-3}}{z^3}+a_1z+\dots$$
        we don't get a $z^{-1}$ term as taking the derivative term by term produces no such term!
        \item Observe that now $(\wp'(z))^2$ has poles of order six while $\wp^3(z)$ does as well! We may choose a constant so that the poles of order $6$ of $(\wp')^2-A\wp^3$ cancel! Our Laurent series look like 
        $$\left(\frac{a_{-6}}{z^6}+\frac{a_{-2}}{z^2}+a_0\right)+A\left(\frac{b_{-6}}{z^6}+\dots\right).$$
        We can also cancel the pole of order 2 with the original Weierstrass function: $(\wp')^2-A\wp^3-B\wp$. With this we have cancelled all of the poles which means that this is holomorphic and therefore constant. This is a functional equation between the Weierstrass function and its derivative. Therefore
        $$(\wp')^2-A\wp^3-B\wp=C$$
        and $(\wp,\wp')\:\bC/\La\less\set{\bonj{\la}}\to\bC^2$ where we map $z\mapsto(x,y)=(\wp(z),\wp'(z))$. What this equation tells us is that the image of this map is contained the locus 
        $$y^2+Ax^3+Bx=C.$$
        We would have to do a bit more work to see that that curve is exactly all of the image.
        \item Our library of examples of Riemann surfaces reduces to projective curves. If $C\subseteq\bP^n$ is a projective curve, then we can obtain any meromorphic function on $C$ by getting a meromorphic function on $\bP^n$ and restricting to $C$, so long as $C$ is not entirely contained in $V(Q)$ where the original function is $P/Q$. Otherwise it becomes the constant function infinity.\par
        In particular take the conic $V(X^2+Y^2+Z^2)=C\subseteq\bP^2$. Take the map $\bP^2\to\bC$ (dotted), $(X:Y:Z)\mapsto X/Z$. Then this function is not defined when $Z=0$, but everywhere else it is. This function is precisely the linear projection onto the first coordinate from $(0:1:0)$. This is really a map from $\bP^2\less\set{(0:1:0)}$ to $\bC$ with exactly two tangents. If we claim that the image of $Z=0$ to the $X$ axis is the point at infinity, this becomes a map $\bP^2\to\bP^1$, becoming our first example of maps \emph{between} Riemann surfaces.
    \end{enumerate}
\end{Ex}

%Surface((1/3 cos(s)(t+t^2+t^3)-sin(s)(t^3-t^2)-1/3 (-2t+t^2+t^3),1/3 (cos(s)-sin(s))(t+t^2+t^3)-(cos(s)+sin(s))(t^3-t^2)+1/6 (-2t+t^2+t^3),1/3 (cos(s)+sin(s))(t+t^2+t^3)-(-cos(s)+sin(s))(t^3-t^2)+1/6 (-2t+t^2+t^3)),t,-1,1,s,0,2pi)

\section{20241016}

Last time we mentioned that compact Riemann surfaces admit no non-constant holomorphic functions. Whereas $\wp$ embeds complex tori into $\bA^2_\bC$. Also projections were meromorphic functions.\par
Today we will start studying functions between Riemann surfaces.

\subsection{Maps between Riemann surfaces}

The notion of holomorphic maps between two Riemann surfaces relies on the charts.

\begin{Def}
    A function $f:C\to D$ is holomorphic at $x$ if 
    $$\psi_xF\vf_x^{-1}\: U_{x}\to U_{f(x)}$$
    is holomorphic at $\vf_x(x)$.\par
    $C\isom D$ when there is a $f^{-1}$ which is also holomorphic.
\end{Def}

This next result in particular tells us hat the degree is not an invariant for isomorphism of Riemann surfaces. If two curves walk up to you and tell you their degrees, you can't differentiate between them just because of that.
\begin{Ex}
    A smooth conic in $\bP^2$ is isomorphic to a line in $\bP^2$. To prove this, we need some facts:
    \begin{itemize}
        \item Any line in $\bP^2$ intersects a conic in 2 points counting multiplicities. This is the fundamental theorem of algebra, but also a case of Bézout's theorem.\par
        To prove this we parametrize the line. Finding the points of intersection amounts to solving a degree two equation in one variable (or homogeneous in 2 variables). Then fundamental theorem of algebra gives us the two local coordinates of the points in the line.
        \item Any two lines in $\bP^2$ intersect at one point. 
    \end{itemize}
    We define our function similar to the stereographic projection. From the conic, we choose a point and project points from the conic into our line. The inverse function takes a point from the line, form another line between the point and our chosen point on the conic and looks at the \emph{other} intersection.
\end{Ex}

\begin{Ej}
    Verify that indeed the aforementioned functions are holomorphic.
\end{Ej}

Isomorphisms occur when two different curves are born and are related by a biholomorphic function. Automorphisms on the other hand occur when we have biholomorphic functions from a curve to itself.

\begin{Def}
    An automorphism of a Riemann surface $C$ is a biholomorphic function $f\: C\to C$.
\end{Def}

\begin{Ex}
    Taking a point outside a conic $C$ and looking at intersections of lines through that point and the conic defines an automorphism.
\end{Ex}

\begin{Ej}
    ¡Verify this!
\end{Ej}

\begin{Ex}
    Consider the elliptic curve
    $$V(Y^2Z-X^3-XZ^2-Z^3),$$
    then the map $\al\:\bP^2\to\bP^2,(X:Y:Z)\mapsto(X:-Y:Z)$ restricts to a map $\al\mid_C\: C\to C$ which is an automorphism.\par
    When dehomogenizing we get an affine chart and a curve symmetric about the $x$-axis. This is the same case as the last map but we've chosen $P=(0:1:0)$. 
\end{Ex}

The idea is to exploit symmetries that define algebraic varieties. If we have even coefficients we probably have this kind of involutions. If $X$ appeared with exponents which are multiples of 3, then we could switch $X$ with $\om X$.

\begin{Qn}
    How do you tell \emph{stuff} (\blu{equality from isom? I didn't hear, ask Ross.})
\end{Qn}

Equality is different from isomorphism, an object is only equal to itself, but is isomorphic to any other object with arrows.\par
Similarly Riemann surfaces are only equal to themselves. If we get a line in $\bP^2$, and we flip it somehow but they are still two different things we want to consider.\par
Two things which are isomorphic are \emph{equal} on the quotient set.\par
Earlier we mentioned that holomorphic functions occur when their local expressions are holomorphic. So the technical lemma that allows us to study functions of Riemann surfaces is:

\begin{Lem}
    If $f\: C\to D$ is holomorphic at $x$, then there exist choices of charts $(U_x,\vf_x)$, $(V_{f(x)},\psi_{f(x)})$ such that the local expression becomes $w=z^k$ where $z$ is the $U$ coordinate and $w$ the $V$ coordinate. Further, $k\in\bN$ is independent of the choice of chart.
\end{Lem}

\begin{Rmk}
    In particular, if $f$ is an isomorphism, $k=1$. That's saying it's the identity on the charts.\par
    If we have two maps then, the local expression of the composition is $\widetilde{w}=z^{k\l}$.
\end{Rmk}

The starting point is that we have a function $f$ which is already holomorphic. We can start with some charts such that the local expression is holomorphic.\par
To make our lives easier assume $\vf_x(x)=\psi_{f(x)}(f(x))=0$ so both points are sent to the origin and $F(0)=0$. In a neighborhood of the origin we have a Taylor expansion which $F$ converges to. Say $k$ is the first term which appears in this expansion. We may write 
$$F(z)=\sum_{i\geq k}a_iz^i=z^k\sum_{i\geq k}a_iz^{i-k}=z^k\sum_{\l\geq 0}\tilde{a}_\l z^\l=z^k\ga(z)$$
and observe that $\tilde{a}_0\neq 0$. This factor is a holomorphic function which is non-zero at the origin. Then there exists a branch of the $k$-th root, let $\widehat{\ga}$ be a choice of a branch of the $k^{\text{th}}$ root of $\ga$ near $z=0$.

\begin{Rmk}
    Recall branches: DRAWING
\end{Rmk}

So let $\tilde{z}=z\widehat{\ga}$, this is a holomorphic change of coordinates. We may choose the composition $(z\widehat{\ga})\vf_x$ as our chart. Then the local expression of our function is $w=\tilde{z}^k$ simply because we picked $F(z)=z^k\ga(z)$ so 
$$\tilde{z}^k=z^k\ga=F=w.$$
\begin{Rmk}
    This proof doesn't work when $k=0$, but proving that entails switch the other chart to the center the constant value.
\end{Rmk}

\section{20241018}

Last time we talked about the technical lemma, its two ingredients were
\begin{enumerate}
    \item Any holomorphic function has a well defined order of vanishing at zero. This is the first non-zero coefficient of the Taylor expansion. This is $k$ such that $a_j=0$ for $j<k$ and $a_k\neq 0$.
    \item If $\ga(z)$ is a holomorphic function such that $\ga(0)\neq 0$ then we can make a choice of a $k^{\text{th}}$ root of $\ga$, $\ga^{1/k}(z)$. There are $k$ holomorphic functions about zero such that we get $\ga$ as their $k^{\text{th}}$ power.
\end{enumerate}

The fact that $k$ is unique comes from the fact that the order of vanishing is unique.

\begin{Def}
    Such $k$ is called the \term{ramification order} of $f\: C\to D$. When $k=1$ we say the map is unramified while for $k>1$, it ramifies.
\end{Def}

\begin{Ej}
    Show that if $k=0$, then our function is constant.
\end{Ej}

\begin{Def}
    For a map of Riemann surfaces $f\: C\to D$ we have:
    \begin{enumerate}
        \item The \term{ramification locus} is $R=\set{x\in C\: k_x>1}\subseteq C$.
        \item The \term{branch locus} is $B=\set{y=f(x)\: x\in R}$.
    \end{enumerate}
\end{Def}

We have that $f(R)=B$ but $f^{-1}(B)\supseteq R$. 

\begin{Th}
The ramification locus is a discrete set.
\end{Th}

\begin{ptcbp}
    Suppose $x\in R$, then there exists an open neighborhood of $C$, $U_x$ where the local expression of $f$ is $w=z^k$. For all points of this neighborhood $y$, we have $k_y=1$. This means that we cannot have accumulation. As we are in a compact set, then our set should be finite.
\end{ptcbp}

\begin{Rmk}
    The function $w=z^k$ has the property that every non-zero point has exactly $k$ preimages. Only zero has one preimage.
\end{Rmk}

\subsection{The degree of a map}

As $C$ is compact, the degree better be finite. This degree should have to do with the covering map property.

\begin{Def}
    The degree of $f\: C\to D$ is 
    $$\deg(f)=|f^{-1}(y)|,\quad y\in D\less B.$$
\end{Def}

This notion is well defined as the number of inverse images of a covering is constant. Take a point not in $B$, count its inverse image suppose 37. Then the set of points with 37 images is a whole open set. The set of points \emph{without} 37 inverse images is also an open set. Every point of $D\less B$ has either or not 37 inverse images. As $D$ is connected, we are done.\par
Pick 
$$y_0\in D\less B,\quad d_0=|f^{-1}(y_0)|$$
Consider 
$$U_0=\set{y\in D\less B\:\ |f^{-1}(y)|=d_0},\quad U_1=\set{(\dots)|f^{-1}(y)|\neq d_0}$$
Observe that $U_0\cup U_1= D\less B$ and $U_0\cap U_1=\emptyset$. $U_0,U_1$ are both open and closed so as $D\less B$ is connected, $U_0\neq \emptyset$ we conclude $U_1=\emptyset$ and $U_0=D\less B$.\par
We would like to think of the ramification order as a multiplicity.

\begin{Lem}
    If $f\:\ C\to D$ is holomorphic of degree $d$ and $y\in D$, then
$$\sum_{x\in f^{-1}(y)}k_x=d.$$
\end{Lem}

The fact that the degree is well defined partitucularly implies that if $f\: C\to \bC$ then it has as many zeroes as poles. Recall this is the same as having a holomorphic function to $\bP^1$. So this now becomes an honest map between Riemann surfaces to which we apply the lemma to:
$$\sum_{x\in f^{-1}(0)}k_x=\sum_{x\in f^{-1}(\infty)}k_x=d.$$

\subsection{A preview on hyperelliptic curves}

We would like to construct Riemann surfaces of any genus. 

\begin{enumerate}
    \item We consider an affine plane curve
    $$\set{y^2=p(x)}$$
    where $p(x)$ has distinct roots and degree $2g+1$. This is not compact, but if we could add infinity points then we fail as compacting in $\bP^2 we get a lot of singularities$.
    \item We may compactify in $\cO(g+1)$ by adding \emph{only one point}.
    \item By a topological argument, this is a 2-to-1 covering of a sphere.
\end{enumerate}

\section{20241021}

A hyperelliptic curve 
$$C=\Set{y^2=\prod_{i=1}^{2g-1}(x-a_i)}\subseteq\bA^2_\bC$$
admits a map $\pi$ to $\bA^1_\bC$ which is the projection of the first coordinate of degree 2. For most $x=x_0$, 
$$\pi^{-1}(x_0)=\set{(x_0,y_0)\:\ y_0\text{ solves }y^2=\prod(x_0-a_i)}$$
and when the product is non-zero we have two choices for $y_0$. Branch points are $(a_i)_{i\in[2g+1]}$ while the ramification locus is the set of points $(a_i,0)$.\par
This is not a compact Riemann surface, if we would compactify this in projective space then we get 
$$Y^2=\prod_{i=1}^{2g-1}(X-a_iZ)$$
which doesn't work as we will get a singularity at $(0:1:0)$. Let's try to stick it into $\cO(d)$ for some $d$.

\begin{center}
    % https://tikzcd.yichuanshen.de/#N4Igdg9gJgpgziAXAbVABwnAlgFyxMJZARgBoAGAXVJADcBDAGwFcYkQBhEAX1PU1z5CKMsWp0mrdgB1pEWjABOjLGBjAO3HnxAZseAkXKkATOIYs2iELIBGABQB6xbf31CjpMTQtTrdgEFnV10BA2FkEy9zSSsbaQBjAHkACigAShC9QUMUKKofWJlpWyCTAH1gO00ecRgoAHN4IlAAM0UIAFskYxAcCCQAVhoceixGdgALCAgAaxD2rqGRgcQAFhGxietpuYWO7sQyPtWNvq2pmfneNoOelaRj0fHLvZuQRcPe-qQAZk2XtZIGoQDRGPRbDBGPYwh5rIosA1JjhQRJLOwTEd9ksjg9EFEQODIdDYbkQAikSjCujrJiXADtuACGx3p8-niCc9GbtrpRuEA
\begin{tikzcd}
    & C \arrow[r, hook] \arrow[d, hook] \arrow[ld, "2:1"'] & \bA^2_{\bC} \arrow[d, hook] \\
\bA^1 \arrow[d, hook] & \overline{C} \arrow[r, hook] \arrow[ld, "2:1"']      & \cO(d)                      \\
\bP^1                 &                                                      &                            
\end{tikzcd}

\end{center}


If we pullback the equation via transition functions we get 
$$\tilde{y}^2=\left(\prod_{i=1}^{2g-1}(1-a_i\tilde{x})\right)\tilde{x}^{2d-2g-1}.$$

At $\tilde{x}=0$, ¿what do we see? It depends on $d$:
\begin{itemize}
    \item If $d\leq g$, then for $\tilde{x}=0$ we get $\tilde{y}=\infty$. This not a problem as long as
    \item $d\geq g+1$, but let's point out that when $d\geq g+2$ we get a polynomial which vanishes at zero. The partial derivatives also vanish so we get a singular point.
    \item So when $d=g+1$ our equation has the form 
    $$\tilde{y}^2=\left(\prod_{i=1}^{2g-1}(1-a_i\tilde{x})\right)\tilde{x}.$$
    When doing the partial derivatives on $\tilde{y}$ is at zero, at $(0,0)$ we don't get singularities.
\end{itemize}

The projection map extends and only one point lies over $(0,0)$ which is the ramification point.\par
In conclusion 
$$C\subseteq\ov C\subseteq\cO(g+1)$$ 
sits as a smooth Riemann surface and 
\begin{center}
    % https://tikzcd.yichuanshen.de/#N4Igdg9gJgpgziAXAbVABwnAlgFyxMJZABgBpiBdUkANwEMAbAVxiRAB12BjAeQAoA5gGoAjAEoQAX1LpMufIRQjyVWoxZtOAIwAKAPREB9bQGEpMkBmx4CRMiNX1mrRCDOTVMKAPhFQAMwAnCABbJAAmahwIJDIQHDosBjYACwgIAGtzAOCwxDjopGU1Z012NCxskCDQiKiYxGKnDVdOCsNgThCsKEMTaU4AAnDGkGoGOi0YBh05G0UQQKwBFJwpCkkgA
\begin{tikzcd}
    \cO(g+1) \arrow[r, "\pi"]                             & \bP^1_\bC \\
    C \arrow[u, hook] \arrow[ru, "{\pi_{\mid_C},\ 2:1}"'] &          
    \end{tikzcd}
\end{center}
where the map has $(2g+2)$ ramification and branch points.

\subsection{Topological details}

We have $2g+2$ points and $\bP^1$ and we have a mysterious curve $C$ and what we know is that we have a $2:1$ map that ramifies precisely over those points and for every other point we have two preimages.\red{AdD FIGURE}\par
Locally, about ramification points the local expression of our map is $w=z^2$. Observe that when going about a ramification point we go from an ordinary point to another (one preimage to the other). By joining pairs of branch points we disconnect the cover. And now that cover, we'd like to put back the segments we took to actually obtain $C$. We will add more segments than needed and identify them.\par
We don't have to worry about orientability or anything because we started with a Riemann surface. 

\subsection{Riemann's Existence Theorem}

The moment that we had a $2:1$ map from an affine curve to $\bC$, we could've said that there was a way to complete a map to $\bP^1$. Then via Riemann-Hurwitz we could've said that there was even-ness.\par
Suppose we have a compact Riemann surface $Y$ and a non-compact topological surface $X_0$ (hasn't met the complex numbers yet, charts are $(\bR^2,\vf)$ where those are continuous functions).\par
Somehow we have a covering map $X_0\to Y_0\defeq Y\less\set{y_1,\dots,y_b}$.

\begin{significant}
    If a topological surface walks up to you and it almost covers a Riemann surface, ¿what can you say about me?
\end{significant}

\begin{Th}[$R\exists T$]
    If $X_0$ is a non-compact topological surface and $Y$ is a compact Riemann surface with $Y_0=Y\less\set{y_1,\dots,y_b}$ with a covering map
    $$f\: X_0\to Y_0$$
    then there exists a unique way to add a finite number of points to $X_0$ and to give a Riemann surface structure to the resulting space so that $f$ extends to a map of compact Riemann surfaces.
\end{Th}

In books, the theorem states that a diagram exists:

\begin{center}
    % https://tikzcd.yichuanshen.de/#N4Igdg9gJgpgziAXAbVABwnAlgFyxMJZABgBpiBdUkANwEMAbAVxiRAA0B9YkAX1PSZc+QigCM5KrUYs27PgJAZseAkTJip9Zq0QgAmtwWCVIohM3Vtsvfr5SYUAObwioAGYAnCAFskZEBwIJAAmfg9vP0QJQODEAGZwkC9ff2ogpBicOiwGNgALCAgAa2NkyND0uPj0nLy9QpL7XiA
\begin{tikzcd}
    X_0 \arrow[d] \arrow[r, hook] & X \arrow[d] \\
    Y_0 \arrow[r, hook]           & Y          
    \end{tikzcd}
\end{center}

On $Y_0$ we already have complex structure. The inverse image of a disk gives us a homeomorphism. That's how we get charts for all points of $X_0$. This gives a Riemann surface structure to $X_0$ in kind of a dumb way. The fun part is extending this to $X$.

\section{20241023}

\subsection{Some topology}

The fundamental group is a functor from pointed topological spaces to the category of groups:
$$\pi_1\:\cat{Top}_{\text{ptd}}\to\cat{Grp},\ (X,x_0)\mapsto \pi_1(X,x_0).$$
\begin{center}
    % https://tikzcd.yichuanshen.de/#N4Igdg9gJgpgziAXAbVABwnAlgFyxMJZABgBpiBdUkANwEMAbAVxiRAAoANUgDwH1iAShABfUuky58hFGQCMVWoxZt2ATVIBPAcLETseAkTnlF9Zq0QgAOtbRY+crrx2jxIDAenHSC6uZUrW3tHdS1XEUUYKABzeCJQADMAJwgAWyQyEBwIJBMlCzZEtyTUjMR8nKQAZj0QFPTM6irEACY6hvLW5tzEav9lSxs7BydE3QoRIA
\begin{tikzcd}
    {(X,x_0)} \arrow[d, "f"] \arrow[r] & {\pi_1(X,x_0)} \arrow[d, "\pi_1(f)"] \\
    {(Y,y_0)} \arrow[r]                & {\pi_1(Y,y_0)}                      
    \end{tikzcd}
\end{center}
The fundamental group detects holes by making walks. For functions on topological spaces, we get group homomorphisms via pushforwards.\par
Now if $f\:(X,x_0)\to(Y,y_0)$ is a covering map we get
\begin{itemize}
    \item $\pi_1(f)$ is injective. In particular $\Im \pi_1(f)\leq\pi_1(Y,y_0)$. The degree of the covering is 
    $$\deg(f)=[\pi_1(Y,y_0):\Im \pi_1(f)]$$
    \item If $(X,x_0)\sim (S^1,x_0)$ then $\pi_1(X,x_0)\isom\bZ$. The fundamental group measures gow many times we go around the circle. There is exactly one isomorphism class of covers of degree $k$ of $S^1$ corresponding to the subgroup $k\bZ\subseteq\bZ$. This is just the map $e^{i\te}\mapsto e^{i k\te}$.
    \item If we have a covering map $f\: (X,x_0)\to (Y,y_0)$ and a continuous surjective function $g\:(Z,z_0)\to(Y,y_0)$ and we ask when does the map $g$ lift, because $f$ is a local homeomorphism, we choose a particular preimage. The obstruction is loops not returning to $x_0$. It is precisely finding a loop in $Z$ which maps to a loop $Y$ that doesn't return to a loop in $X$. This will not occur when 
    $$\pi_1(g)(\pi_1(Z,z_0))\leq\pi_1(f)(\pi_1(X,x_0)).$$
\end{itemize}

\begin{Qn}
    ¿Why is the morphism injective? It boils down to the homotopy lifting property.
\end{Qn}

\begin{Qn}
    ¿If $Z$ is a covering do we get the property for free? No, take the circle, the identity and a line.
\end{Qn}

\subsection{Riemann's Existence Theorem}

\begin{Th}
    Suppose $Y$ is a compact Riemann surface, $X_0$ is a topological surface and the map
    $$f\: X_0\to Y$$
    is continuous and $f(X_0)=Y\less\set{y_1,\dots,y_b}$. Then there exists a unique compact Riemann surface $X$ such that we have a commutative diagram:
\begin{center}
    % https://tikzcd.yichuanshen.de/#N4Igdg9gJgpgziAXAbVABwnAlgFyxMJZABgBpiBdUkANwEMAbAVxiRAA0B9YkAX1PSZc+QigCM5KrUYs27PgJAZseAkQlip9Zq0QgAmnykwoAc3hFQAMwBOEALZIyIHBCQAmattl6rIagx0AEYwDAAKQqqiIDZYpgAWOArWdo6Izq5IEi50WAxs8RAQANb+0jpsADqV+Dh0ySC2DlnUmYie5T4g1XgMsMBWvEa8QA
\begin{tikzcd}
    X_0 \arrow[rd, "f"'] \arrow[r, "\iota", hook] & X \arrow[d, "\tilde{f}"] \\
                                                  & Y                       
    \end{tikzcd}
\end{center}
such that 
\begin{itemize}
    \item $\iota(X_0)=X\less\set{x_1,\dots,x_n}$.
    \item The map $\tilde{f}$ is a map of compact Riemann surfaces.
\end{itemize}
\end{Th}

If you have topological information about a function that somehow can be completed to a ramified covering, then it is possible to do this in a unique way.\par
We will complete $X_0$ and then give an atlas. For a point not in the image, the idea is that we will build a chart via covering. 

\begin{ptcbp}
    (sketch)
    \begin{enumerate}
        \item For every $x\in X_0$, we can get a chart by finding a small neighborhood $V_x$ upstairs such that $f_{\mid_{V_x}}$ is a homeomorphism and then compose with a chart of $f(x)$.
        \item If $U_{i}$ is a neighborhood of one of our points $y_i$, which is homotopic to $S_1$ then 
        $$f^{-1}(U_i)=\bigcup_j V_j$$
        where $V_j$ form a connected covering of $U_i$ then $V_j\sim S^1$ and 
        $$\pi_1(f_{\mid_{V_j}})=1\mapsto k_i\:\ \bZ\to\bZ.$$
        \item Compose with the chart about $y_b$
        $$\vf_{y_i}\circ f_{\mid_{V_j}}$$
        which lifts to a 1-to-1 map $\psi$ via the cover $w=z^{k_j}\:\bC\less 0\to\bC\less0$.
        \item We can add a point to each $V_j$, declare $\psi(P)=0$, and give a topology such that $\psi$ extends to a homeomorphism to its image.
     \end{enumerate}
     With this we've simultaneously completed $X_0$ and given an atlas. In this atlas, the local expression of $f$ is already given in normal form.
\end{ptcbp}

\begin{Qn}
    ¿Is there a way to say why $X$ is unique with respect to this property?
\end{Qn}

Completion is unique, but the complex structure could be different. By getting another complex structure we can see that the new atlas is compatible with the one that we built. 

\section{20241025}

\subsection{Monodromy representations}

A holomorphic map of compact Riemann surfaces $f:X\to Y$ together with $y_0\in Y\less B$ and a labeling of $f^{-1}(y)$ by the set $[d]$ is equivalent information to:
$$\vf_f\: \pi_1(Y\less B,y_0)\to S_d$$
A group homeomorphism.\par

\subsubsection{From the compact Riemann surfaces map}

We can lift the loop $\ga$, there's $d$ distinct choices to lift the loop.

Only the loops that lift to loops are the ones that live in the fundamental group of $X$. We want to construct a map, so $\ga$ gives us $d$ lifts. We associate the permutation 
$$i\mapsto (\text{endpoint of }\ga_i).$$
This corresponds to a permutation of $[d]$. This is a bijective function because we can walk the paths backwards. We can reverse the parametrization of $\ga$ thus switching the starting and endpoints of the lift.

\subsubsection{From a permutation to map}

Our goal now is to construct a topological ramified cover which we apply $R\exists T$ to. If the map ramifying at $b_1,b_2,b_3$ exists, then $f$

\begin{Ej}
    Take a map $f:S\to\bP^1$ which ramifies at $b_1,b_2,b_3$
\end{Ej}

\begin{Rmk}
    If we change the labeling of the points of $f^{-1}(y)$, the effect on the monodromy is that it gets post-composed by an inner automorphism of $S_d$, a conjugation. In order to make a more canonical statement, we mod out by inner automorphisms.\par
    If $b$ is a branch point such that $f^{-1}(b)$ consists of points $x_1,\dots,x_n$ with ramification orders $k_1,\dots,k_n$ and $\rho_b$ is a loop winding once about $b$. Then 
    $\vf_f(\rho_b)$ has cycle type $(k_1,\dots,k_n)$. There's a correspondence between algebraic information and the monodromy representation.
\end{Rmk}

\section{20241028}

Last time we talked about the monodromy representation which works via loop lifting. The class of a loop $[\ga]$ is $i\mapsto$ endpoint of $\ga_i$.\par
¿What can we say about this group homomorphism?
\begin{itemize}
    \item If $b$ is a branch point such that $f^{-1}(b)=\set{x_1,\dots,x_n}$ with ramification profile $k_1,\dots,k_n$ and $\rho_b$ is a small loop winding once around $b$, then $\vf_f(\rho_b)$ is a permutation of cycle type $\set{k_1,\dots,k_n}$. 
\end{itemize}

\begin{Ex}
    Suppose $b$ has two inverse images, $x_1,x_2$ with $k_1=3$ and $k_2=1$. $x_1$ is a ramification point and $x_2$ is unramified. About $b$ we have a local chart such that the local expression of the covering map above the local chart, the map is $w=z_1^3$ on one component and $w=z_2$ in the other.\par
    The way that these power functions work is by seeing that the inverse image of a segment is 3 segments or just another segment.\par
    If we take a loop winding around $b$, the loop comes in through the $(z_2,w)$ chart so as we go about we get a fixed point.\par
    On the other hand in the $(z_1,w)$ chart, we come in one direction and exit in another. This is corresponds thus to a $3$-cycle.
 \end{Ex}

 \begin{Ex}
    Consider a map $S\to\bP^1$ whose branch locus is $\set{b}$ and giving rise to a (bogus) monodromy representation such that 
    $$\vf_f(\rho_b)=(12)\in S_2.$$
    The fundamental group of $\bP^1\less\set{\text{pt}}$ is $\set{\id}$ so there's no group homeomorphism in heaven or earth which would send $\id\mapsto (12)$.\par
    Visually, if we take a segment $b$ to $Q$ and then removing the segment we get a 2-gon. The cover restricted to this shape is two copies of the shape. Above $b$ we only have one point, and above $Q$ we have $P_1\neq P_2$ as it's not a branch point.\par
    ¿What happens if we lift a loop winding about $b$? We need to switch the charts. This gives us a certain edge identification, in particular $P_1$ should be glued to $P_2$. In one case, we make $Q$ another branch point while on the other we make $P_1,P_2$ infinitely close.
 \end{Ex}

 This in particular shows that we can't have a map to $\bP^1$ with one branch point. 

 \subsection{Discrete data}

 \begin{significant}
    ¿What is the discrete data associated to a map of Riemann surfaces?
 \end{significant}

 There's data that varires continuously as the complex manifold structure. In the case of complex tori, that could be visualized via the lattice structure. The discrete data is, for example, the genus.\par
 For a map, the discrete data we've seen is 
 \begin{enumerate}
    \item Degree.
    \item Genera of $X$ and $Y$.
    \item Branch points and ramification profile.
 \end{enumerate}

 If we fix all of this discrete data, if we want a map of certain degree from surface of genus $g_1$ to $g_2$ and declare points with ramification profile, then we have a finite number of such maps.

 \begin{Rmk}
    Fixing all the data before plus the complex manifold structure on $Y$, then there are finitely many maps $f:X\to Y$ satisfying those conditions.
 \end{Rmk}

 This is because each of these maps give rise to a monodromy representation. Those morphisms are finite as there's a finite number of generators. The monodromy representation allows us to convert an analytic problem to \red{something I didn't understand}. This is the beginning of Hurwitz theory.

 \begin{significant}
    Now assume we choose all the invariants at random. ¿Will there be a map which satisfies them? ¿Are there constraints to choose them?
 \end{significant}

 The answer is no, there is one elementary constraint, the Riemann-Hurwitz formula. The more complicated constraints are open problems.

 \begin{Th}[Riemann and Hurwitz's formula]
    For a map $f\: X\to Y$ of compact Riemann surfaces, then the discrete invariants satisfy 
    $$2g_X-2=\deg(f)(2g_Y-2)+\sum_{x\in X}(k_x-1).$$
    We may also call $\nu_x=k_x-1$, the ramification index of $f$ at $x$.
 \end{Th}

 Observe that $2g_X-2=-\chi_X$, the Euler characteristic which suggests how we will prove it.

 \begin{Qn}[Open]
    Characterize when discrete data satisfying the Riemann-Hurwitz formula still does not admit the existence of any mapping. 
 \end{Qn}

 \begin{Ex}
    The simplest example of discrete data satisfying Riemann-Hurwitz but no the existence of a mapping is $g_Y=0$, the degree of the map is $4$ and we have $3$ branch points $b_1,b_2,b_3$ with ramification type $(2,2),(2,2)$ and $(3,1)$ and $g_X$ is zero.\par
    In the case of the formula we have 
    $$-2=4(-2)+((1+1)+(1+1)+(2+0))$$
    so that the formula works.\par
    If such a map existed, there would a monodromy representation. The fundamental group of $\bP^1$ without three points is $\gen(p_1,p_2,p_3\: p_3p_2p_1=e)$. We want $p_1\mapsto$ a $(2,2)$ cycle, $p_2$ as well but $p_3$ goes to a $3$-cycle.\par
    $p_1,p_2$ should map in the Klein 4-subgroup of $S_4$. However $p_3$ should be the inverse of their product but it can't be in the Klein $4$ group as it's a 3 cycle.
 \end{Ex}

 \begin{Rmk}
    There are 3 cows, you lasso them, but you pull the lasso about the world and get it back to you.
 \end{Rmk}

\section{20241030}

\subsection{The Riemann-Hurwitz formula}

For a map $f\: X\to Y$ of compact Riemann surfaces:

$$2g_X-2=\deg(f)(2g_Y-2)+\sum_{x\in X}(k_x-1).$$
This is a necessary but not sufficient condition for a map $f$ to exist. The first counterexample comes from $S_4$. Before we prove the Riemann-Hurwitz formula let's observe the following. Call $S$ the sum, $S$ is the total amount of ramification of the cover.
\begin{itemize}
    \item Observe that $S$ is always even. 
    \item Dig back from the mist of your memory and consider the hyperelliptic curve $y^2=\prod_{i\in I}(x-a_i)\subseteq\bA^2$ which gives us a $2:1$ map to $\bA^1$. $\bA^2$ is $\bP^1$ minus a point and $f$ is a map of degree $2$.\par
    By the $R\exists T$, there's a curve $\ov C$ which is our hyperelliptic curve adding one or two points and a map to $\bP^1$. The total amount of ramification that we see on our affine part is $2g-1$ so that the point at $\infty$ is one ramification point. 
    \item There are no maps from surfaces of lower genus to surfaces of higher genus. In other words, it always holds that $g_X\geq g_Y$.
    \item We can check, as a sanity check, that Riemann-Hurwitz holds for hyperelliptic curves. If $C\to\bP^1$ is a hyperelliptic map, then 
    $$2g_C-2=2(-2)+S\To S=2g+2.$$
    \item The last thing to observe is that the only maps of degree $\deg(f)\geq 2$ from curves of genus $g$ to curves of same genus happen when $g=0$ or $1$. If $g=1$ then they are unramified. These maps are isogenies. 
\end{itemize}

Let us prove the Riemann-Hurwitz formula topologically with the Euler characteristic.

\begin{ptcbp}
    The Euler chracteristic of $X$ will be computed in $2$ different ways to obtain the Riemann-Hurwitz formula. The first way is the identification polygon: $2-2g_X$.\par
    Now we find a cell decomposition of $X$ and do $V-E+F$. The way we are going to obtain such a cell decomposition of $X$ is to start at $Y$ and then look at $X$. Pick a graph on $Y$ such that its complement is a finite union of topological disks being a little bit careful. We want to make sure the graph doesn't ignore branch points. So our branch points will be vertices. Choose $V$ then, so that $B\subseteq V$.\par
    What we care about is that all the red points are vertices. And then we consider $f^{-1}$ of the graph. This is some kind of graph in $X$ which we claim is a good graph in the previous sense. Observe that $f^{-1}(G^\sC)=f^{-1}(G)^\sC$ because $(Y\less G)\cap B=\emptyset$ then $f_{\mid f^{-1}(Y\less G)}$ then $Y\less G=\coprod (\text{disks})$. So $X\less f^{-1}(G)=\coprod (\text{disks})$. So we can compute the Euler characteristic of $X$ using this graph.\par
    Call $\tilde{G}$ the upper graph. The number of faces of $\tilde{G}$ is $\deg(f)F_Y$. For edges it's the same story, it doesn't contian a branch point so the map is a degree $d$ covering. The number of edges is $\deg(f)E_Y$.\par
    The interesting case happens with vertices. For a non-branch point, we have $d$ distinct vertices. Above a branch point we get a ramification point with order $k_r$. So we should see $k_r$ points! But we only see one. So we've lost $k_r-1$ points from the $d$ that we should have. So for vertices 
    $$V_{\tilde{G}}=\deg(f)V_G-\sum_{x\in X}\nu_x.$$
    Taking the alternating sum we get 
    $$V_{\tilde{G}}-E_{\tilde{G}}+F_{\tilde{G}}=\deg(f)V_G-\sum_{x\in X}\nu_x-\deg(f)E_G+\deg(f)F_G.$$ 
\end{ptcbp}

The quantity $2g-2$ is the degree of the divisor of any meromorphic 1-form on our curve so that's why the formula is written like that. Eventually we will use the Riemann-Hurwitz formula to prove that same fact. 

\subsection{Applications of Riemann-Hurwitz}

We will see the genus-degree formula and the Riemann automorphism bound.

\begin{Prop}
    If $C\subseteq\bP^2$ is a smooth projective curve of degree $d$ ($C$ is the zero set of a homogeneous polynomial of degree $d$), then $g_C=\binom{d-1}{2}$.
\end{Prop}

Let's sanity check this:
\begin{itemize}
    \item If $\deg F=1$ then $C$ is a line which is isomorphic to $\bCP^1$. Then it has genus 0 which coincides with $\binom{1-1}{2}=0$.
    \item For $\deg F=2$, we have conics which are all isomorphic $\bCP^1$. 
    \item If $\deg F=3$, then we have cubics. We've seen some cubics arising coming from embedding the Weierstrass function. Any cubic is equivalent to a Weierstrass form which comes from a torus.
\end{itemize}
In the first two cases we get $\binom{0}{2}=\binom{1}{2}=0$ whereas for degree $3$ we have $\binom{2}{2}=1$.\par
Besides the Riemann-Hurwitz formula, the other ingredient to prove the genus degree formula is Bézout's theorem. Curves of degree $c,d$ intersect in $cd$ points counting multiplicities. Recall that Bézout generalizes the fundamental theoem of algebra.\par
This intersection multiplicity should persist across continuous deformations of our curve. If we let 
$$D_t=V\left(t\prod_{i=1}^{d}(a_iX+b_iY+c_iZ)+(1-t)D\right)$$
and assume that $C\cap D_t$ is independent of $t$ then $t=0$ we get $C\cap D$ and at $t=1$ we get $C\cap d$ lines.

\section{20241106}

\subsection{Genus-degree formula}

Using Bézout's theorem and the Riemann-Hurwitz formula we will prove the genus-degree formula.

\begin{ptcbp}
    One way to get a map is to project. Without loss of generality let us project from $P=[0:1:0]$ so that we will project our curve $C=V(f)$ into $\bP^1\isom\set{Y=0}$. This defines a map 
    $$\pi_P\:C\to\bP^1$$
    where the intersecting lines generically intersect $C$ at $d$ points.
    \begin{center}
        ¿How do we detect intersections of not multiplicity one?
    \end{center}
    This is with the tangents, such can be detected when $\del_Yf=0$. So we may characterize the set of ramification points for $\pi_P$ as $V(f)\cap V\left(\pdv{F}{Y}\right)$. The degree of $f_Y$ is $d-1$, so by Bézout's thereom 
    $$|R|=d(d-1).$$
    Now applying Riemann-Hurwitz:
    $$2g(C)-2=d(0-2)+|R|\To g(C)=\binom{d-1}{2}.$$
\end{ptcbp}

Another way to see this, without using the Riemann-Hurwitz formula is by
\begin{enumerate}
    \item Knowing that arithmethic genus is invariant under deformations,
    \item and knowing that smooth curves obey $a.g.=g.g.$
\end{enumerate}
From this, it is also possible to deduce that genus-degree formula.

\begin{Rmk}
    Here $g_A=h^1(\cO_C)$ and $h^0(\cO_C)=1$ because of the global sections of $C$. And we know that $\chi(\cO_C)$ is invariant, therefore $h^1$ here is invariant.\par
    From this consider the family
    $$\la F+\mu\left(\prod_1^dL_i\right)$$
    we can think of this as a family of curves parametrized over $\bP^1$ by $[\la:\mu]$. For $\mu=0$ we get $F$ and $\la=0$ returns the union of $d$ lines.\par
    The genus of this union of lines is the number of holes we create. From one to two lines we get no holes, but intersecting three gives us a hole. The fourth intersects each line in one point so it adds more genus. The pattern is $0,0,1,1+2,1+2+3,\dots$ which is precisely the binomial coefficient.
\end{Rmk}

\subsection{Hurwitz' Theorem}

\textbf{On the automorphisms of compact Riemann surfaces of genus higher than 2.}\par

\begin{Th}
    For a compact Riemann surface $C$ with genus $g\geq 2$ and $G\leq \Aut(C)$ then 
    $$|G|\leq 84(g-1).$$
\end{Th}

If we consider a group acting on $C$, then we can consider the projection to the orbit space $C/G$. This is a map of compact Riemann surfaces.\par
The degree of this map is $|G|$ and the key point is that if we have a point $r_1\in R(\pi)$ (a ramification point) of order $k_1$ and $r_2$ is such that $\pi(r_1)=\pi(r_2)$, then $k_1=k_2$ and $r_2$ is also a ramification point.\par
In other words, for a map obtained as quotient of a group action, all the other inverse image will have the same ramification index. This follows from the orbit-stabilizer theorem for finite group actions. This is because $k_1=|\Stab(r_1)|$. If we have a point fixed by $g,h$, then near that point we have $2$ points exchanged by those $g,h$ and that's the same picture as \red{something}.\par
As a consequence, for every $b\in B$ (branch point on the image), we have $\frac{|G|}{k_b}$ ramification points, each of order $k_b$.

\begin{Rmk}
    If $|G|$ has order $6$ then, we can have $3$ points of order $2$ or $2$ of order $3$.
\end{Rmk}

Now by Riemann-Hurwitz:
\begin{align*}
2g(C)-2&=|G|(2g(C/G)-2)+\sum_{b\in B}\frac{|G|}{k_b}(k_b-1)\\
&=|G|\left((2g(C/G)-2)+\sum_{b\in B}\left(1-\frac{1}{k_b}\right)\right)
\end{align*}

\begin{center}
    ¿How small can the sum be?
\end{center}

Every $k_b\geq 2$ so $1-\frac{1}{k_b}\geq\half$. So the sum is bounded between $\half$ and $1$. If we have exactly $1$ branch point.\blu{correct this}\par
If the sum is $1$ then $|B|=2$ with $k_1=k_2=2$. For $S\leq 2$ then either $|B|=2$ or $|B|=3$ and we have $\vec{k}=(2,2,k)$ or $(2,3,k\leq 6)$ or $(3,3,3)$. Or when $|B|=4$, $\vec{k}=(2,2,2,2)$.\par
Finally if $S>2$ then $S\geq 2+\frac{1}{42}$, there's a finite number of things to check $|B|=3$ and $\vec{k}=(2,3,7)$.\par
Returning to Riemann-Hurwitz:
$$|G|=\frac{2g(C)-2}{2g(C/G)-2+S}$$
The cases are 
\begin{itemize}
    \item When $g(C/G)$, the denominator is $S-2$, so $S>2$ because $|G|>0$ and $g(C)\geq 2$. In this case $S\geq 2+\frac{1}{42}$.
    In particular $S-2\geq \frac{1}{42}$. Then $|G|\leq (2g-2)/(1/42)=84(g-1)$. Every other case gives us something worse.
    \item If $g(C/G)=1$ then the denominator is $S$ and $S$ is always bigger than $\half$ so 
    $$|G|\leq (2g-2)/(1/2)=4(g-1).$$
    \item Finally if $g(C/G)\geq 2$ then denominator is always at least $2$ so that $|G|\leq g-1$.
\end{itemize}
This just elementary maths and Riemann-Hurwitz and it's quite interesting how this allows us to talk about automorphisms.\par
Next time we will talk about degrees of divisors and differential forms.

\section{20241108}

\subsection{Canonical Divisors}

Today we will compute the degree of the canonical divisor of $C$ of genus $g$ and then compute global holomorphic 1 forms on hyperelliptic curves.\par

\begin{Rmk}
    Observe that the Riemann Hurwitz formula is very topological and the consequences we are getting are quite a bit deep.
\end{Rmk}

\begin{Def}
    A holomorphic differential 1-form on a compact Riemann surface $C$ is a holomorphic section of $T^\ast C$, the cotangent bundle. 
\end{Def}

In practice we look at local expressions.

\begin{enumerate}
    \item If $(U,\vf)$ is a chart on $C$ with some local coordinate $z$, the local coordinate expression of any $1$-form on $U$ is $\om= f(z)\dd z$ with $f$ holomorphic.\par
    $\dd z$ is a basis vector on any cotangent space. The way to relate this concrete picture to the abstract picture is that if we have an open set over which the cotangent line bundle being trivialized. 
    \item If we have a change of coordinates to $(\tilde{U},\tilde{\vf})$ with local coordiante $\tilde{z}$ and $z=\Phi(\tilde{z})$, then the differential form transitions via the chain rule 
    $$\dd z=\Phi'(\tilde{z})\dd\tilde{z}.$$
\end{enumerate}

If we want to go from local to global, we need an atlas for the Riemann surface and find things which are compatible everywhere. In practice we want global results without having to use the definition. 

\begin{Def}
    If $\om$ is a holomorphic $1$-form on a curve $C$ of genus $g$, then 
    $$\div(\om)=\sum_{x\in C}\ord_X(\om)x.$$
\end{Def}

\begin{Rmk}
    All of this can be done with meromorphic instead of holomorphic sections. 
\end{Rmk}

If $\om_1,\om_2$ are two 1-forms on $C$ then $\om_1/\om_2$ is a function on $C$. The ratio is a complex number, which doesn't change by isomorphism. As meromorphic functions have as many zeroes and poles, so in particular $\deg(\om_1/\om_2)$ is constant. Since $\deg(\div(f))=0$, then 
$$\deg\div(\om_1)=\deg\div(\om_2).$$
The canonical divisors are the equivalence class of divisors of meromorphic $1$-forms of $C$.

\begin{Th}
    The degree of the canonical divisor is $2g_C-2$.
\end{Th}

\begin{Ex}
    Observe that in $\bCP^1$ we have $T^\ast \bCP^1=\cO(-2)$ then $\deg(K_{\bCP^1})=-2=2\.0-2$.
\end{Ex}

The degree doesn't depend on the complex manifold structure but only on the genus. This has to do more with the topology, it's the Euler class of the cotangent formula.\par
In fact proving this in another way implies that we can rewrite the Riemann-Hurwitz formula via canonical divisors.\par
We will also use a lemma which will be proven later:

\begin{Lem}
    For any Riemann surface $C$, there exists a function $C\to\bCP^1$.
\end{Lem}

This is because we have proven facts about the degree of the canonical bundle over $\bP^1$.

\begin{ptcbp}
    Suppose we have a map of Riemann surfaces $\pi\:X\to Y$ and a meromorphic $1$-form $\om_Y$ on $Y$. We can pullback $\pi^\ast(\om_Y)$ to $X$.\par
    We can use the computation of degree in $Y$ and the pullback to compute the degree of any meromorphic $1$-form in $X$.
    \begin{center}
        ¿How can we relate zeroes and poles of $\om_Y$ to those of $\pi^\ast\om_Y$?
    \end{center}
    Locally if we have $(U_x,x)$ and $\pi_{\mid U_x}\: (U_x,x)\to(U_y,y)$ and our local expression of $\om_Y$ is $f(y)\dd y$. We want to pull it back and not choose random coordinates. We want coordinates such that the local expression of $\pi$ is $y=x^k$. Now lets focus on the point $y=0$, and its inverse image $x=0$.\par
    The zeroes and poles of $\om_Y$ are given by the order of vanishing of $f$ at $y=0$. Recall 
    $$f(y)=a_ny^n+a_{n+1}y^{n+1}+\dots\To\ord\om_{Y\mid y=0}=n.$$
    Pulling back makes us substitute 
    $$\pi^\ast\om_Y=kf(x^k)x^{k-1}\dd x$$
    so that the zeroes and poles at $0$ is $kn+(k-1)$. We now compute the degree of the divisor of $\pi^\ast\om_Y$, this is 
    \begin{align*}
    &\sum_{x\in X}\ord_x(\pi^\ast\om_Y)\\
    =&\sum_{y\in Y}\sum_{x\in \pi^{-1}(y)}\ord_x(\pi^\ast\om_Y)\\
=&\sum_{y\in Y}\sum_{x\in\pi^{-1}(y)}k_x\ord_y(\om_Y)+(k_x-1)\\
=&\sum_{y\in Y}\left(\sum_{x\in\pi^{-1}(y)}k_x\right)\ord_y(\om_Y)+\sum_{y\in Y}\sum_{x\in\pi^{-1}(y)}(k_x-1)\\
=&d\deg\div\om_Y+\sum_{x\in X}\nu_x
    \end{align*}
    Via our lemma, 
    $$\deg K_C=-2d+\sum\nu_x$$
    and Riemann-Hurwitz tells us 
    $$2g_C-2=d(-2)+\sum\nu_x.$$
\end{ptcbp}

Every meromorphic 1 form gives us a divisor of the same degree, so we only need one. We might as well get a pullback.\par
Next time we will compute differential forms on hyperelliptic curves, but we will do it next time.

\section{20241111}

Last time we defined the canonical divisor $K_C$ as the equivalence class of any meromorphic $1$-form on $C$. We proved 
$$\deg(K_C)=2g-2$$
where $C$ was a compact Riemann surface of genus $g$. We deduced this using the Riemann-Hurwitz formula noticing how zeroes and poles of the pullback of $\om$ relate to zeroes and poles of the actual $\om$.\par
We will use this degree fact to find a $g$ dimensional space of holomorphic $1$-forms on a hyperelliptic curve of genus $g$. After we prove the Riemann-Roch formula, we will see that these are \emph{all} of the holomorphic $1$-forms of $C$.\par
Reviewing the setup is that we start with a hyperelliptic curve of genus $g$. This means that our curve is 
$$C=V\left(y^2=\prod_{i=1}^{2g+1}(x-a_i)\right),\quad a_i\neq 0\quad\text{all distinct}.$$
We can compactify this curve to add the point at infinity. So now, we would like to describe some differential forms on our curve $C$. There's a very large open set where we can describe a lot of holomorphic differential forms. Then we have to check what conditions are imposed when transitioning to open sets where we have the missing points. In particular consider the open set
$$C\less R$$
this is $C$ without the $2g+1$ \red{red} points. Observe that for any point in $C\less R$ we can find a neighborhood where $x$ is a local coordinate. We can write a differential from as $f(x)\dd x$. Not only that but if $f$ is holomorphic, we get a holomorphic $1$-form. Instead of $f$, let's check monomial by monomial and check $x^n\dd x$. Let's also chug in a $\frac1y$ because it doesn't vanish. We want to know, which of these will survive into neighborhoods of \red{red} points and the point at infinity.
\begin{itemize}
    \item In a neighborhood of any red point, $(a_i,0)$ we can use $y$ as a local coordinate. To transition $\dd x$ to $\dd y$ we use the equation 
    $$y^2=p(x)\To 2y\dd y=p'(x)\dd x\To 2\frac{\dd y}{p'(x)}=\frac{\dd x}{y}.$$
    From this the transition of our differential form is 
    $$\om=\frac{2x^n}{p'(x)}\dd y.$$
    This gives us infinitely many $1$-forms holomorphic everywhere except possibly at infinity.
    \item Checking infinity amounts to asking what is the divisor of $\om$. The zeroes are when $x=0$, we get two points $Q_+,Q_-$. $x$ vanishes with order $1$ at each of these two points so $x^n$ does it with order $n$. Then the divisor is 
    $$\div(\om)=n(\bonj{Q_+}+\bonj{Q_-})+\text{¿?}$$
    but there's no poles anywhere. The last point is $\bonj{\infty}$ and in order to get degree $2g-2$ we have
    $$\div(\om)=n(\bonj{Q_+}+\bonj{Q_-})+(2g-2-2n)\bonj{\infty}.$$
    Thus $\om$ will be holomorphic if 
    $$2g-2-2n\geq 0\iff n\leq g-1$$
    and our $g$ differential forms will be 
    $$\gen\left(\frac{\dd x}{y},x\frac{\dd x}{y},x^2\frac{\dd x}{y},\dots,x^{g-1}\frac{\dd x}{y}\right)$$
    which gives us a $g$ dimensional vector space of holomorphic $1$-forms.
\end{itemize}

With this, we have the all of our applications of Riemann-Hurwitz. For the remainder of the week, we will talk about sheaves of sections. 

\subsection{Line bundles and more}

This is very important in algebraic geometry. There's three equivalent objects, one in algebra, one in geometry and one in combinatorics. Let's discuss what they are and their relationship. Assume our base variety $B$ is a compact Riemann surface. 
\begin{enumerate}
    \item In geometry we have line bundles, $\pi\:L\to B$ such that $\pi^{-1}(b)\isom\bC$ and it's locally trivial. Transition functions are holomorphic. Recall that locally trivial means that the bundle restricted to neighborhoods returns a trivial bundle.
    \item In algebra there is the notion of invertible sheaves. These are locally free sheaves of rank $1$ on our space $B$. If $\cL$ is our sheaf, then:
    \begin{itemize}
        \item For every open set $\cL(U)$ is an $\cO_B$-module.
        \item For every $b\in B$, $\cL_b\isom\cO_{B,b}$.
    \end{itemize}
    \item Finally in combinatorics these are what we call linear series $|D|$. These are equivalence classes of divisors on $B$. 
\end{enumerate}

The first connection makes algebra come to life. If we have a line bundle we look at sections of line bundles. If $L\xrightarrow{\pi}B$ is our line bundle, then $\cL$ is defined as the sheaf of sections of $L$. In other words, for every open set $U$ we have
$$\cL(U)\defeq \set{s\:U\to L\mid s\text{ holomorphic, }\pi s=1_U}.$$
The stalk condition says that for a small enough open set, the sheaf of sections is the sheaf of regular function. 

\section{20241113}

Last time we saw that if we start from line bundles, we can naturally get a sheaf out of it. This is simply the sheaf of local sections of the line bundle.\par
Today we will go the other way, if we have an invertible sheaf, we will get a line bundle.

\begin{Def}
    $\cL$ is a \term{invertible sheaf} if it's locally free of rank one. This is, $\cL$ is an $\cO_C$-module and for every $x$ there's a neighborhood $U_x$ such that $\cL_{\mid U_x}\isom \cO_{C\mid U_x}$. 
\end{Def}

Let's now construct the way from invertible sheaves to line bundles.

\begin{ptcb}
Pick a cover $\cU=(U_\al)_{\al\in A}$ of $C$ such that
\begin{itemize}
    \item $\cL_{\mid U_\al}\isom\cO_{C\mid U_\al}$ for all $\al$.
    \item There are no non-trivial quadruple intersections. (see the exercise \ref{ej-triangulable-rs})
    \item If $U_\al\cap U_\bt\cap U_\ga\neq\emptyset$ then their intersection is inside a chart.
\end{itemize}
Our goal is to define $L$, the line bundle, as 
$$L\defeq \coprod_{\al\in A}quot{U_\al\x\bC}{\sim}.$$
Pick $U_\al\cap U_\bt\neq\emptyset$, then 
$$\cL(U_\al)\xrightarrow[\isom]{\phi_\al}U_\al\x\bC$$
and the same for $\bt$. Then we have restrictions $r_\al,r_\bt$ to $\cL(U_\al\cap U_\bt)$. If we have sections $s_\al,s_\bt$ then we have 
$$r_\al s_\al=r_\bt s_\bt,$$
because of the third condition. We can think of $s_\al$, via the isomorphism, as a function $\phi_\al s_\al$. So now if we look at $U_\al\cap U_\bt$, the fact that $r_\al s_\al=r_\bt s_\bt$ implies that $\phi_\al s_\al$ matches with $\phi_\bt s_\bt$ over the intersection $U_\al\cap U_\bt$. We thus define an equivalence $(x_\al,P_\al)\sim(x_\bt,P_\bt)$ when 
\begin{enumerate}
    \item $x_\al=x_\bt$ in $C$.
    \item $P_\al=\frac{\phi(s_\al)(x_\al)}{\phi(s_\bt)(x_\bt)}P_\bt.$
\end{enumerate}
Even if we pick a point where the section is zero, the order of zero is going to cancel out so it is well defined. There's still things to worry about. 
\begin{itemize}
    \item ¿Have we defined a line bundle?
    \item If yes, ¿does our definition depend on our choices?
\end{itemize}
Assume we have a line bundle, we won't talk about choosing a different cover, that's for a qualifying exam. If we pick different sections, then they must differ by multiplication by another function so that when dividing they go away. In principle, it's just some abstract section.\par
But if it wasn't a line bundle, pick a $P\in U_\al\cap U_\bt\cap U_\ga$. Then there's three fibers; with our construction we've defined isomorphisms $F_{\bt\al},F_{\ga\bt}$ and $F_{\al\ga}$. In the quotient, we are only going to see one copy of $C$, but ¿what if by doing the three maps we get a non-trivial automorphism? We wouldn't like to quotient out by it.\par
By our construction 
$$F_{\bt\al}\circ F_{\ga\bt}\circ F_{\al\ga}=\id_\bC$$
which is the cocycle condition. This holds because we will have three quotients 
$$P_\al=\frac{\phi(s_\al)(x_\al)}{\phi(s_\bt)(x_\bt)}P_\bt=\frac{\phi(s_\al)(x_\al)}{\phi(s_\bt)(x_\bt)}\frac{\phi(s_\bt)(x_\bt)}{\phi(s_\ga)(x_\ga)}P_\ga=\dots\To P_\al=P_\al.$$
    Note that 
    $$F_{\bt\al}\defeq\frac{\phi(s_\al)(x_\al)}{\phi(s_\bt)(x_\bt)}$$
    for $x_\al=x_\bt\in U_\al\cap U_\bt$ is an element of $\cO_C^\ast(U_\al\cap U_\bt)$. The collection $\set{F_{\bt\al}}$ gives a \v{C}ech cochain for the degree $1$ \v{C}ech complex for $\cO^\ast$. The cocycle condition means that $d(\dots)=0$.\par
    From this we have that the data of invertible sheaves is equivalent to the data of an element in $H^1(C,\cO^\ast_C)$. 
\end{ptcb}

\begin{Ej}\label{ej-triangulable-rs}
    Any Riemann surface can be covered by a open cover such that there's no non-trivial quadruple intersections. In particular this holds for topological surfaces which can be triangulated.
\end{Ej}

\subsection{From geometry to combinatorics}

We would like to construct a divisor, or a set of divisors from a line bundle. Naturally we would look at divisors of sections. Recall:
\begin{significant}
    From any meromorphic section $s$ of a line bundle $L\xrightarrow{\pi}C$, we get a divisor
    $$\div(s)\defeq\sum_{x\in C}\ord_x(s)\bonj{x}.$$
\end{significant}
One way to rephrase this is:
\begin{itemize}
    \item There is a set of divisors $\set{D\in\div(C)\: D\text{ lin. equiv. to }\div(s)}$ which means that $D-\div(s)=\div(f)$ is the divisor of a rational function $f$ on $C$.
\end{itemize}

\begin{Rmk}
    If we embed the curve $C$ on the plane $\bP^2$, then the natural way to make linearly equivalent divisors is to take two lines $L_1,L_2$ and then points of intersection of one line with $C$ with one color and the other with another. So $\blu{D}\sim\red{D}$ because $\blu{D}-\red{D}$ is the divisor of the function $f_2/f_1$ where $f_2$ is a function for line $2$ and $f_1$ with line $L_1$. And there's a sense in whihc this is not just an example, but really the prototypical thing.\par
    If we start with a line bundle and we have three sections\dots
\end{Rmk}

\subsection{From combinatorics to geometry}

\begin{Prop}
    Given a divisor $D\in\div(C)$, we can construct a line bundle $\cO_C(D)$ with a section whose divisor is $D$.
\end{Prop}

This gives a bit more information than just a line bundle, it gives us also the information of a \emph{distinguished} section (a class of equivalence of sections such that\dots).

\begin{Ex}
    Let's illustrate this with one point. Let's construct $\cO_C(\bonj{p})$. In this case take $p\in C$ with the open sets
    $$U_p,\word{a neighborhood, and}C\less\set{p}.$$
    Their intersection is just a punctured disk $U_p\less\set{p}$. As there's no $p$ in $C\less\set{p}$, then we may have a transition function $w=\frac{1}{z}$. What we want in essence is to glue the section $\set{v=1}$ to the section $\set{u=z}$, so if we want that to happen we need $v=\frac{1}{z}u$.\par
    This is truly a local construction because we could've taken $U_p\less\set{p}$ instead of $C\less\set{p}$ and then extended the trivial bundle. It's also possible to do so for $\cO_C(n\bonj{p})$ by taking $v=\frac{1}{z^n}u$.
\end{Ex}

\section{20241115}

Our objective for today is to obtain a subsheaf $\cL_B$ of the constant sheaf of meromorphic sections(\dots)\par
In particular we also get a map 
$$\cL\to\cM(C)$$
which for every open set $W$(\dots)and the fact that this is an injective function(\dots)meromorphic functions are defined up to closed subsets, if we throw away a closed subset of points the meromorphic function (\dots)
\begin{significant}
    ¿How do we characterize meromorphic functions in $\cO_C(D)(W)$?
\end{significant}

A key point is that such meromorphic functions must have (\dots) If $p\in C$ If it's nonzero, when we divide by $s$ we get a zero of order $n$. These are all meromorphic functions such that when we add $\div(s)$, we get what's called an \emph{effective divisor} (all coefficients are positive).\par
The quick way to say this is that $\cO_C(D)(W)$ is 
$$\set{f\in\cO_C(D)\:\ \div(f)\mid_{W}+\div(s)\mid_W\geq 0}.$$
Let's connect with our comfort zone $\bP^1$:

\begin{Ex}
    Consider the point $p=[0:1]\in\bP^1$. (\dots) together with the choice (\dots)  
    $$(\cO_{\bP^1}(1),x)\xrightarrow{??}\cO_D(?)?$$
    $\cO_{\bP^1}(1)(\bP^1)=\gen(X,Y)$ and $\gen(1,\frac{1}{X})$ 
\end{Ex}

Her the messy thing is that the section that vanishes at zero corresponds to a constant section. It's a major confusing point that when we write $\cO_C(D)$ we mean the line bundle and in other cases it means the linear series. 

\begin{Rmk}
    Simeon almost cried because of something and Renzo argues that's always a good moment to cry.
\end{Rmk}

If $L\xrightarrow{\pi}C$ has a $(n+1)$-dimensional space of sections \red{unreadable}.

\begin{enumerate}
    \item Pick a basis $s_0,\dots,s_n$ for $\cL(C)$ (or if we'd rather for $\cO_C(D)$ where $D=\div(s_0)$.) Then 
    $$\vf_k(x)=(s_0(x),s_1(x),\dots,s_n(x))$$
    The map is well-defined when at least one of the sections is non-zero. (For curves this is not a problem but it can become one when the dimension is greater than 2.) So a line bundle is \emph{base-point free} if \red{something} doesn't vanish. If all sections vanish at one point, then it's a base point.\par
    Notice that it's important that this is considered a point in $\bP^n$ for the map to be well-defined. Once again this is because of the choice of section and the isomorphisms with charts. 
    \item A coordinate free description of the map starts by not choosing a basis. Then we can think of $\bP^n$ as the projectivization of functions from $\cL(C)$ to $\bC$ in the sense that
    $$\vf_k\: C\to\bP(\Hom(\cL(C),\bC)),\ x\mapsto \eps_x$$
    where $\eps_x$ is the evaluation $s\mapsto s(x)$. To get a complex number here, take any isomorphism you want between the fiber $\cL_x$ and $\bC$. Depending on our choice the map will be different, but the projectivization won't change!
\end{enumerate}

If we take $s_i$ to be the restriction \red{my bag almost fell down off of the bus}

\section{20241118}

Last time we saw that given a divisor $D$, $\cO_C(D)\subseteq\cM(C)$ (meromorphic sections of $C$) as a subsheaf is isomorphic to the sheaf of sections of the line bundle associated to $D$. Then the fact that 
$$H^0(C,\cO_C(D))=\set{\vf\in\cM(C)\:\ \div\vf+D\geq 0}$$
implies that we can build a map 
$$\vf_D\: C\to\bP^{h^0-1},\quad x\mapsto(s_0(x):\dots:s_n(x))$$
and $(s_i)$ also form a basis of $H^0$.

\begin{Rmk}
    We have that:
    \begin{enumerate}
        \item If $\deg(D)<0$, then $H^0(C,\cO_C(D))=\set{0}$.
        Saying that the degree of the divisor is negative is imposing more zeroes than poles that we are allowing. But no meromorphic section has a different number of zeroes than poles. 
        \item The degree of $\vf(C)$ in $\vf_D:C\to\bP^{N}$, where $N\defeq h^0-1$, is equal to $\deg(D)$. This is because
        $$\deg(\vf_D(C))=\deg(\vf_D^\ast(H))=\deg(\div(s))=\deg(D)$$
        where $s$ is a section of $\cO_C(D)$.
        \item In order to get a map $C\to\bP^1$ it suffices to find a divisor such that $h^0\geq 2$. If $V\leq H^0(C,\cO_C(D))$ then we have the diagram:
        \begin{center}
            % https://tikzcd.yichuanshen.de/#N4Igdg9gJgpgziAXAbVABwnAlgFyxMJZABgBpiBdUkANwEMAbAVxiRAGEQBfU9TXfIRQBGclVqMWbADrSARgAUAesAAWS4gAoAIgEoAtMK7deIDNjwEio4ePrNWiELMUrZULAFtNANQNHucRgoAHN4IlAAMwAnCE8kMhAcCCQAJmp7KSdZGkiAfR8QagY6ORgGBX5LIRBorBDVHBMo2PjEUSSUxHSJBxlpHBgADxxgNFiAK2MeFriE6mSkDszHZ2lcvO1AriA
\begin{tikzcd}
    C \arrow[rd, "\vf_V"'] \arrow[r, "\vf_D"] & \bP^{h^0(D)-1} \arrow[d, "\text{proj}"] \\
                                              & \bP^{\dim(V)-1}                        
    \end{tikzcd}
        \end{center}
    \end{enumerate}
\end{Rmk}

\begin{Ex}
    Let's construct this map for our favorite Riemann surface $\bP^1$. Take the divisor $D=0+\infty$. We want meromorphic functions with a single pole at $0$ or $\infty$. This means that 
    $$H^0(C,\cO_C(D))=\Set{\frac{at^2+bt+c}{t}}=\gen(t,1,1/t)$$
    where $t$ is an affine coordinate about $0$. Then the map $\vf_D\:\bP^1\to\bP^2$ such that $t\mapsto [t:1:1/t]=[t^2:t:1]$. The image of the map $\vf_D(\bP^1)$ is equal to the conic $V(XZ-Y^2)$. In particular observe that the image of the point $t=0$ is $[0:0:1]$ and $t=\infty$ maps to $[1:0:0]$. Also interesting is $t=1$ maps to $[1:1:1]$.\par
    To exemplify the phenomenom, let's pick a divisor:
    $$D'=[0]+[\infty]-[1]$$
    Certainly a function with a zero of order $1$ and poles at $0,\infty$ still is in $H^0(C,\cO_C(D))$. This means that 
    $$H^0(C,\cO_C(D'))\Set{\frac{(t-1)(at+b)}{t}}=gen\left((t-1),\frac{(t-1)}{t}\right)$$
    and in terms of coordinates of $\bP^2$ we have $X-Y,Y-Z$. In terms of our diagram, this is
    \begin{center}
        % https://tikzcd.yichuanshen.de/#N4Igdg9gJgpgziAXAbVABwnAlgFyxMJZABgBpiBdUkANwEMAbAVxiRAB12AjABQD0AjCAC+pdJlz5CKAeSq1GLNp159gAJmEixIDNjwEisgfPrNWiDt37ABW4fJhQA5vCKgAZgCcIAWyRkIDgQSOrUZkqWnDQeAPrAACIA5FrUDHRcMAw8EgbSIF5YzgAWONqePv6IskEhiGEK5mzIABoAtACaiO0AWhQgaRlZOfpSbAwwHmWiFX4B1MFINREWVjGxCSIUwkA
\begin{tikzcd}
    \bP^1 \arrow[rd, "\vf_{D'}"'] \arrow[r, "\vf_D"] & \bP^{2} \arrow[d, "{[X-Y:X-Z]}"] \\
                                                     & \bP^{1}                         
    \end{tikzcd}
    \end{center}
    And we obtain a map of degree one because the degree of the divisor is $1$. It can also be seen because every line, when projecting the conic from $\vf_D(1)$, intersects at one other point!
\end{Ex}

\begin{Rmk}
    
One of the reasons this is called lienar system is because divisors obtained from effective divisor are obtained from hyperplanes. The coordinate sections $X,Y,Z$ give us $2[0],2[\infty]$ and $[0]+[\infty]$. Linear combinations of $X,Y,Z$ give us hyperplanes. 
\end{Rmk}

\begin{Ex}
    ¿What if instead we chose $V=\gen(t,1/t)$? This coordinates are nothing but $X$ and $Z$ so our map is $[X:Y:Z]\mapsto [X:Z]$. Since this point is no longer on the curve, ¡this map has degree $2$ now! This is becuase $t\mapsto [t:1/t]=[t^2:1]$.
    \begin{center}
        % https://tikzcd.yichuanshen.de/#N4Igdg9gJgpgziAXAbVABwnAlgFyxMJZABgBpiBdUkANwEMAbAVxiRAB12AjABQD0AjCAC+pdJlz5CKAeSq1GLNp159gAJmEixIDNjwEisgfPrNWiDt37ABW4fJhQA5vCKgAZgCcIAWyRkIDgQSOrUZkqWnDQeAPrAACIA5FrUDHRcMAw8EgbSIF5YzgAWONqePv6IskEhiGEK5mzIABoAtACaiO0AWhQgaRlZOfpSbAwwHmWiFX4B1MFINREWVjGxCeUg3nOIgYv11JlgUEhtACwAnOGKq9FxAGoDIOmZ2bljloUl0zo7VTUDpo-pUlgs6sDZgDwaEjjATkgAOwADhuTUsrUQfREFGEQA
\begin{tikzcd}
    \bP^1 \arrow[rd, "\vf_{D'}"'] \arrow[r, "\vf_D"] \arrow[rd, "\vf_V"', bend right=49] & \bP^{2} \arrow[d, "{\ast}"] \arrow[d] \arrow[d] \arrow[d, "{[X:Z]}", bend left=69] \\
                                                                                         & \bP^{1}                                                                                
    \end{tikzcd}
    \end{center}
\end{Ex}

\subsection{Riemann-Roch and Serre Duality}

\begin{Th}[Riemann-Roch]
    Assume $C$ is a compact Riemann surface of genus $g$, with $D$ a divisor on $C$. Then 
    $$h^0(D)-h^1(D)=\deg D+1-g.$$
\end{Th}

¡This is how many holomorphic sections and holomorphic 1-cocycles controlled by topological and arithmethic information! If we correct holomorphic sections in the sense of Euler characteristic, then we can control the number.

\begin{Th}[Serre's Duality]
    Under the conditions of the previous theorem
    $$H^0(C,\cO_C(D))\isom (H^1(C,\cO_C(K_C-D)))'$$
    so in particular $h^0(D)=h^1(K_C-D)$.
\end{Th}

Instead of proving them, ¡let's use them! Some consequences of these results are:

\begin{enumerate}
    \item The dimension of the space of global section of the canonical bundle is $g$: $h^0(K_C)=g$. We should think of this as an analogous of Gauss-Bonnet. Local to global is controlled by the topology of the space. Plugging in $K_C$ into Riemann-Roch:
    $$h^0(K_C)-h^1(K_C)=2g-2+1-g$$
    and then apply Serre's duality.
    \item Right now we know 1 curve of genus zero. ¿How do we know it's the only one? ¿What if there's another exotic complex structure? The next result is 
    \begin{significant}
        Every curve of genus 0 is isomorphic to $\bP^1$.
    \end{significant}
    Pick a divisor of degree $1$, $D=[p]$. ¿What does RR tell us?
    $$h^0([p])-h^1([p])=2.$$
    We also know $h^1$ is zero, because by Serre duality that is $h^0(K_C-[p])$ so that $K_C-[p]$ is a divisor of degree $-1$ as in genus zero $\deg(K_C)=-2$. We started today by observing that negative degree implies not having global sections. So we found $h^0([p])=2$. This gives us a function 
    $$\vf_pC\to\bP^1,\quad x\mapsto(s_0(x),s_1(x))$$
    and this map is of degree $1$ so it's one-to-one. All maps of compact Riemann surfaces are onto. In normal form the local expression of every point is $w=z$ which is invertible. Thus we have a biholomorphism.
    \item If the genus is strictly greater than $0$, $D=[p]$ is such that $h^0([p])=1$. ¿Why? Well it's at least one because of constant functions. But if it were bigger than one we would get two linearly independent sections and we could construct a map to $\bP^1$. The existence of this map would mean that a sphere is homeomorphic to a $g$-holed surface.
\end{enumerate}
%%%%%%%%%%%% Contents end %%%%%%%%%%%%%%%%

\ifx\nextra\undefined
\printindex
\else\fi
\nocite{*}
\bibliographystyle{plain}
\bibliography{bibiRiemannSurfaces.bib}
\end{document} 

