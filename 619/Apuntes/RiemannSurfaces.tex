\documentclass[12pt]{memoir}

\def\nsemestre {II}
\def\nterm {Fall}
\def\nyear {2024}
\def\nprofesor {Renzo Cavalieri}
\def\nsigla {MATH619}
\def\nsiglahead {Complex Geometry}
\def\nlang {ENG}
%\def\darktheme{}
%\def\nhtml{}
\let\footruleskip\relax %%FADIR
\input{../../headerVarillyDiff}

\begin{document}
%\clearpage
\maketitle
%\thispagestyle{empty}
%test
{\small 
\setlength{\parindent}{0em}
\setlength{\parskip}{1em}

This is a topics course on this stuff

\subsubsection*{Requirements}
Knowledge on stuff\par 

\textbf{TO DO:}
\begin{itemize}
    \item Write 10.1
    \item Write 11.3
    \item Write 12.4
\end{itemize}
}
\newpage
\tableofcontents
%\begin{multicols}{2}
\chapter{Renzo's Complex Projective Exercises}

\section{Set of points of the projective line}

\begin{Ej}
    Show that there is a bijection between the set $\text{Set}\bCP^1$
 and a quotient set of a disjoint union of two copies of $\bC$.
\end{Ej}

\begin{ptcbr}
    Indeed consider our copies of $\bC$ embedded into $\bC^2$ as the lines $\set{x=1}$ and $\set{y=1}$. Then for a line $\l\in\bCP^1$ our map is 
    $$\l\mapsto \l\cap(\text{corresponding line})\mapsto (\text{corresponding coordinate}).$$
    Explicitly, if our line is $[X:Y]$, then the map is $[X:Y]\mapsto X/Y$ on one chart while $Y/X$ on the other.\par 
    Observe that this map is surjective as every point in each copy of $\bC$ is hit by a line of a different slope. The only points which are not hit twice are the origins of both lines. From this, we define the quotient by identifying the coordinates as $x\sim y$ whenever $y=1/x$. Thus our map becomes a bijection at the level of the quotient as we can now properly trace back each point to a particular line. 
\end{ptcbr}
\section{The projective line as a topological space}

\begin{Ej}[Hopf Fibration]
    Show there is a fibration of topological spaces:
    $$S^1\to S^3\to S^2$$
    meaning that there is a surjective continuous function from the three-dimensional sphere to the two-dimensional sphere, and the inverse image of any point is homeomorphic to a circle. This is called the Hopf fibration; notice that while the construction of these maps is rather mysterious in terms of spheres, it becomes transparent when thinking of the two-dimensional sphere as the complex projective line.
\end{Ej}
%https://math.stackexchange.com/questions/4142758/proving-that-the-hopf-fibration-is-a-fiber-bundle
\begin{ptcbr}
We have shown that the map 
$$\pi_2\: S^3_\bR\to\bCP^1,(\text{pt. in }S^3)\mapsto(\text{corresponding line through origin in }\bC^2)$$
is surjective. Also, we have that 
$$\bCP^1\isom \bC\cup\set{\infty}\isom S^2$$
where the first homeomorphism comes from previous discussion and the second one from stereographic projection. This means that we have a map $S^3\to S^2$ which is our candidate for the Hopf map. It remains to be seen that this map is continuous and that the fibers are homeomorphic to $S^1$.\par 
It suffices to show $\pi_2$ is continuous as the rest of the maps are homeomorphisms. To that effect, take an open set $U\subseteq\bCP^1$. This means that in the quotient topology induced by the $\pi_2$ map, $U$ is open whenever $\pi_2^{-1}(U)$ is open. But this proves immediately that $\pi_2$ is continuous as it takes open sets back to open sets.\par 
Now let $\l\in\bCP^1$, we'll analyze what the fibers are:
$$\pi_2^{-1}(\l)=\set{\la z_0\:\ z_0\in\l\cap S^3,\ \la\in\bC}$$
but when restricting to $S^3$, we get the condition that $\la$ can only vary an $S^1$'s worth of values:
$$\pi_2^{-1}(\l)=\set{\la z_0\:\ z_0\in\l\cap S^3,\ |\la|=1}\isom S^1.$$
This means that fibers of our map are homeomorphic to $S^1$ and thus we have the desired fibration structure.
\end{ptcbr}
\newpage
\section{The projective line as a complex manifold}

\begin{Ej}
    Compute 
    $$\phi_{21}\defeq \vf_2\circ\vf_1^{-1}\mid_{\vf_1(U_1\cap U_2)}\: (\vf_1(U_1\cap U_2),x)\to(\bC,y)$$
    and show that it is a holomorphic function on its domain of definition. Show that its inverse is also holomorphic on its domain of definition.
    
    These exercises show that $\bCP^1$
     has the structure of a complex analytic manifold.
    The pairs $(U_i,\vf_i)$
     are called complex charts, the biholomorphic map $\phi_{21}$
    a transition function, and the coordinates $x$
     and $y$
     are called local (or affine) coordinates.
\end{Ej}
\begin{ptcbr}
Recall that the open set $U_1\cap U_2$ is the collection of lines in $\bCP^1$ which are not $x$ or $y$ axes. The image then is all the non-zero $x$ coordinates of the intersection of those lines with $x=1$. Taking those lines through $\vf_2\circ\vf_1^{-1}$ gives us the $y$-coordinates of the intersections of those lines with the $y=1$ line. We get all except the $y$-axis. Computing this for a particular line, if $x_0$ is the intersection with $x=1$, then $\frac{1}{x_0}$ will be the intersection with $y=1$. Therefore, the map $\vf_2\circ\vf_1^{-1}$ is $x_0\mapsto\frac{1}{x_0}$ of non-zero $x_0$. This function and its inverse are holomorphic as the vertical and horizontal lines are excluded from this.
\end{ptcbr}
\section{Functions on the projective line}

\begin{Ej}
    Show that meromorphic functions $f\: \bCP^1\to\bC$
    may be described in two equivalent ways:
    \begin{enumerate}
        \item As the ratio of two homogeneous polynomials of the same degree in the homogeneous coordinates:
        $$f(X:Y)=\frac{P_d(X,Y)}{Q_d(X,Y)}$$
        \item As a rational function in one of the affine coordinates (with no restrictions on the degrees of the polynomials)
        $$f(x)=\frac{p(x)}{q(x)}$$
    \end{enumerate}
   How do you go from one perspective to the other?
\end{Ej}

\begin{ptcbr}
We begin with the second item by claiming that if $f$ is meromorphic has a zero of degree $m$ at $z=z_0$ then we may write $f=(z-z_0)^mg$ where $g$ is meromorphic but has no zeroes at $z_0$. Similarly for poles. This means that we may write $f$ as a product of possibly repeated linear factors over another product of linear factors. These products are the desired polynomials.\par
We may homogenize to obtain the first characterization.\par   
\red{I don't recall how to construct the function in the homogeneous way :(}
\end{ptcbr}
\newpage
\section{Automorphisms of the projective line}

\begin{Ej}
    Prove that, given any two ordered triple of points $P_1,P_2,P_3$
 and $Q_1,Q_2,Q_3$
of the projective line, there exists a unique automorphism $\Phi$
 of the projective line such that $\Phi(P_i)=Q_i$.
Show that it follows that the only automorphism that fixes three points is the identity.
Describe the subgroups of $\Aut(\bCP^1)$
 consisting of automorphisms that fix one, or two points in the projective line.
\end{Ej}

\begin{ptcbr}
    In affine coordinates, we can map any triple to $0,1,\infty$ by considering the function 
    $$z\mapsto\frac{z-p_1}{z-p_3}\left(\frac{p_2-p_3}{p_2-p_1}\right).$$
    This maps $p_1,p_2,p_3$ to $0,1,\infty$ respectively. This is a Möbius transformation, so it is an automorphism of $\bCP^1$. Call it $\vf_P$ and then create $\vf_Q$, the desired function $\Phi$ is $\vf_Q^{-1}\vf_P$.\par
    From this we immediately see that is $P$ is fixed then the function is 
    $$\Phi=\vf_P^{-1}\vf_P=\id.$$
    If $\Phi$ fixed two points we get rotations about the axis passing through those two points. Be it, for example, $0,\infty$ with scalings $z\mapsto \al z$ or $1,-1$ with $z\mapsto 1/z$.\par
    If only one point is fixed, then it is a translation of the line leaving that point fixed. Say for example maps of the form $z\mapsto \frac{az+b}{d}$ leave infinity fixed.
\end{ptcbr}
\section{Maps to projective spaces}

\begin{Ej}
    We define the degree of $F(\bCP^1)$ to be the number of intersections with a general hyperplane in $\bCP^r$.  Prove that if the degree of the polynomials $P_i(X,Y)$ is equal to $d$, then the degree of $F(\bCP^1)$ is less than or equal to $d$. When does the strict inequality hold?
\end{Ej}

\begin{ptcbr}
    
\end{ptcbr}
\newpage
\section{Line bundles on the projective line}

\begin{Ej}
    For $x_0\neq 0$, let $i_1\:\set{x=x_0}\into\cO_{\bCP^1}(d)$ and $i_2\:\set{y=\frac{1}{x_0}}\into\cO_{\bCP^1}(d)$ be the two inclusions of vertical lines. Show that $i_2^{-1}\circ i_1\:\bC\to\bC$ is a linear isomorphism. Observe that the collection of these linear isomorphisms defines a holomorphic function $c_{12}\:\bCP^1\less\set{0,\infty}\to\bC\less 0$.
\end{Ej}

\begin{ptcbr}
    Let's concretely analyze the $i_2^{-1}\circ i_1$ map. This comes out of the line $\set{x=x_0}$ and gives us 
    $$\Im i_1=\set{(x_0,u)\: u\in\bC}\subseteq (\bC^2,(x,u)).$$
    In order to see what $i_2^{-1}$ does, we translate $\Im i_1$ into $(\bC^2,(y,v))$ via
    $$x\mapsto \frac{1}{y},\quad u\mapsto\frac{v}{y^d}\To v=y^du.$$
    So $\Im i_1$ on the other chart is 
    $$\Im i_1=\Set{\left(\frac{1}{x_0},\frac{u}{x_0^d}\right)\: u\in\bC}$$
    and $i_2^{-1}$ returns us $\frac{u}{x_0^d}$. This means that the composition in question is that map $u\mapsto \frac{u}{x_0^d}$ which means that the map is multiplication by $x_0^{-d}$. For a fixed non-zero $x_0$, this is a linear isomorphism of $\bC$.\par
    The collection of such isomorphisms is obtained when we let $x_0$ vary and the function $c_{12}$ given by $x_0\in\bCP^{1}\less\set{0,\infty}\mapsto x_0^{-d}\in\bC\less\set{0}$ is indeed holomorphic.
\end{ptcbr}
\section{Sections of line bundles}

\begin{Ej}
    Show that if $s_0,s_1$ are two sections of the same line bundle, then their ratio is a (meromorphic) function on $\bCP^1$. Show that if $s_0,s_1,\dots,s_r$
 are $(r+1)$ sections of the same line bundle, then they define a map $\bCP^1\to\bCP^r$.
\end{Ej}

\begin{ptcbr}
    Observe that if $z\in\bCP^1$, then $s_0(z),s_1(z)$ lie on the fiber $\pi^{-1}(z)$ which is isomorphic to $\bC$. So taking their ratio on this fiber does produce a complex number. However, we must verify that the ratio is well-defined. Assume we picked another element of the base, $w\in\bCP^1$ and asked about the ratio of $s_0(w)$ with $s_1(w)$. In this case, observe that there's a linear isomorphism between $\pi^{-1}(z)$ and $\pi^{-1}(w)$ which scales all vectors by the same length. This means that $s_i(w)=\al s_i(z)$ for some $\al\in\bC$ and therefore their ratios are the same.\par
    So $s_0/s_1$ does define a meromorphic function on $\bCP^1$ thanks to the linear isomorphisms.\par
    In the same fashion, if we instead have $r+1$ sections, via a same argument we can see that when changing fibers, the sections only change by a scaling which is the same on all entries. So this means that we may write $\bonj{s_0:\dots:s_r}$ as a function to $\bCP^r$.
\end{ptcbr}

\section{Divisors}

\begin{Ej}
    When you multiply two meromorphic functions, what happens to their divisors? If two meromorphic functions produce the same divisor, what can you say about them?
\end{Ej}

\begin{ptcbr}
    If we consider our functions as sections of $\cO(d)$ then,
    $$\displaystyle s_0(x)=\frac{\prod_{i=1}^{m}(x-\al_i)}{\prod_{j=1}^{n}(x-\bt_j)}\word{and}s_1(x)=\frac{\prod_{i=1}^{r}(x-\ga_i)}{\prod_{j=1}^{s}(x-\dl_j)}$$
    then their divisors are 
    $$\div(s_0)=\sum_{i=1}^{m}[\al_i]-\sum_{j=1}^{n}[\bt_j]+(n-m+d)[\infty],\ \div(s_1)=\sum_{i=1}^{r}[\ga_i]-\sum_{j=1}^{s}[\dl_j]+(s-r+d)[\infty].$$
    Where we see the degree at infinity by transitioning via $s(x)\mapsto y^ds(1/y)$.
    Their product is precisely 
    $$s_0(x)s_1(x)=\frac{\prod_{i=1}^{m+r}(x-\tilde\al_i)}{\prod_{j=1}^{n+s}(x-\tilde\bt_j)},$$
    where $\tilde{\al}_i$ corresponds to $\al_i$ when $i$ is between 1 and $m$ and $\ga_i$ from $m+1$ onwards. The same happens for $\tilde{\bt}_j$. In this case the divisor is 
    $$\div(s_0)=\sum_{i=1}^{m+r}[\tilde{\al}_i]-\sum_{j=1}^{n+s}[\tilde{\bt}_j]+((n+s)-(m+r)+d)[\infty]$$
    and the $x=\infty$ still gets the correct coefficient after using the transition function. \red{¿But why don't we get $2[\infty]$? ¿Is it because this only works in $\cO(0)$, the trivial bundle?}\par
    So all non-infinite coefficients do get added and the divisors are summed.\par
    When two meromorphic functions have the same divisor, they should be scalar multiples of each other at most. 
\end{ptcbr}

\begin{Ej}
    Let $s$ be a meromorphic section of a line bundle $\pi\: L\to\bCP^1$, we call the support of $\div(s)$ the set of points that appear with non-zero coefficients in $\div(s)$. Show that there is a natural bijection 
    $$T_s\:\pi^{-1}\bonj{\bCP^1\less\supp(\div(s))}\to\bCP^1\less\supp(\div(s))\x\bC.$$
Meditate on the following fact: the function $T_s$ 
 is an isomorphism of complex manifolds, and in fact an isomorphism of line bundles on the punctured $\bCP^1$
 (we of course did not precisely define these notions, so try and make a guess of what these things should mean). It is called a \emph{trivialization}  of the line bundle $\pi\: L\to\bCP^1$
 on the complement of the support of $\div(s)$.
\end{Ej}

\begin{ptcbr}
    The first intuitive way to define the function is take an isomorphism $\vf$ of the fiber $\pi^{-1}(x)$ with $\bC$. This gives us the map 
    $$T_s(x)=(\pi(x),\vf(x)),$$
    but this map is very non-canonical and ALSO, doesn't depend on $s$. So let us take advantage of a couple of points that we know in the fiber: $x$ and wherever $s$ maps $\pi(x)$ to, $s(\pi(x))$. Outside the support of $s$, $s$ is never zero, so it makes sense to define the quotient $\vf(x)/\vf(s(\pi(x)))$. The desired function is then 
    $$T_s(x)=\left(\pi(x),\frac{\vf(x)}{\vf(s(x))}\right)$$
    which is independent of $\vf$ \red{But, ¿why was this?}, making it natural in that sense. This function is bijective as 
    $$\left(\pi(x),\frac{\vf(x)}{\vf(s(x))}\right)=\left(\pi(y),\frac{\vf(y)}{\vf(s(y))}\right)$$
    implies $x,y$ lie in the same fiber via the first component. This immediately gives us $s(x)=s(y)$ and from this 
    $$\frac{\vf(x)}{\vf(s(x))}=\frac{\vf(y)}{\vf(s(x))}\To \vf(x)=\vf(y)\To x=y.$$
\end{ptcbr}

\section{Distinguished line bundles on the projective line}

\begin{Ej}
    Recall that the very first definition of the set of points of the projective line is that each point corresponds to a line in $\bC^2$. We now want to construct a space $\operatorname{Taut}(\bCP^1)$ whose points correspond to the choice of line in $\bC^2$ together with a point on it:
    $$\text{Set}\left(\Taut(\bCP^1)\right)=\set{(\l,P)\:\ \l\text{ is a line through the origin in }\bC^2,\ P\in\l}.$$
    \begin{enumerate}
        \item Realize $\Taut(\bCP^1)$ as a subspace of $\bCP^1\x\bC^2$.
        \item Show that the first projection restricts to $\Taut(\bCP^1)$ to make it into a line bundle on $\bCP^1$.
        \item Describe the fiber over a point $[\l]\in\bCP^1$.
        \item Show that $\Taut(\bCP^1)\isom \cO(-1)$.
        \item Show that $\Taut(\bCP^1)\isom \Bl_{(0,0)}\bC^2$.
    \end{enumerate}
\end{Ej}

\begin{ptcbr}
    \begin{enumerate}
        \item As mentioned, elements of $\text{Set}\Taut(\bCP^1)$ are $A=(\l,P)$. This immediately defines a map 
        $$\Taut(\bCP^1)\hookto\bCP^1\x\bC^2,\ A\mapsto (\l,P).$$
        \red{I had this question for a bit, but, how can I show that this map is a map of projective varieties?}
        \item The map 
        $$\Taut(\bCP^1)\to\bCP^1,\ (\l,P)\mapsto \l$$
        describes a structure of a line bundle. To show this we must see that for an open neighborhood $U$ of $\l\in\bCP^1$, we have that $\pi^{-1}(U)\isom U\x\bC^1$. We define the map as 
        $$U\x\bC^1\to\pi^{-1}(U),\ (\l,z)\mapsto(\l,zP),\ P\in\l.$$
    \end{enumerate}
\end{ptcbr}

\section{Tautological Bundle}

The graph of the function 
$$F\: \bC^2\less\set{0}\to\bP^1,\ (X,Y)\mapsto[X:Y]$$
is exactly $(X,Y),[S:T]$ such that 
$$F(X,Y)=[S:T]\To\exists\la\neq 0(X=\la S,\ Y=\la T).$$
If we assume that $T\neq 0$, we can divide by $T$ and then get $Y/T=X/S$. This also assumes $S\neq 0$, but ok, the points in the closure of this set is obtaineed by clearing denominators. This leads us to $SY=TX$. Spending some time carefully doing this, if one coordinate is zero and the other one isn't, the equation is still satisfied. Even points $(0,0)$, whatever $S$ and whatever $T$, are also in the closure. This means that $(X,Y)$ belongs to the line $[S:T]$, even if $(X,Y)=(0,0)$, it belongs to every line.


\section{A quick recap}

Originally, we saw $\bCP^1$ as a complex manifold of dimension 1, we studied things by restricting them to charts and reducing problems to complex analysis. Locally around every point we had complex numbers.\par
We like to study functions on the space because this translate geometric data to algebraic data. For example we understand $\bR^2$ by considering the \emph{coordinate functions}! Take a point $P$ where the $x$ function takes value $33$ and the $y$ function $-37$.\par
Bummer, $\bCP^1$ has few functions. Only the constants in fact. This gives us directions of study, meromorphic functions for example give us the whole collection $\bC(x)$. This field of rational functions is a birational invariant of $\bCP^1$, in a sense it's a very rich invariant but it's insensitive to small changes in the space.\par
Otherwise we can consider the local functions. For every open set, consider holomorphic functions on that open set. We have a lot of them, and they are not completely unrelated. There's actually a whole series of connections between these open sets. This is the notion of a sheaf. 
$$\text{regular}\leftrightarrow\text{holomorphic}\word{and}\text{rational}\leftrightarrow\text{meromorphic}.$$
So now, $(\bCP^1,\cO)$ is a locally ringed space which is the datum of a scheme.\par
The third perspective is to consider sections of a line bundle instead of functions! This comes from replacing a function with its graph.
$$F\:X\to Y\leftrightarrow i_{\Ga_F}\: X\to\Ga_F\subseteq X\x Y,$$
we have the same information of a function as $x\mapsto(x,F(x))$. The idea is that we will allow the graph to not live in the product $\bCP^1\x\bC$ but instead on a space which \emph{locally} looks like that. We allow the graph of our \emph{kinda} function to live in spaces that are not necessarilly products $\bCP^1\x\bC$ but locally are products with $\bC$.\par
Sections of a line bundle give us maps to projective space. $r+1$ sections give us a map to $\bP^r$. This gets us to another invariant, the Picard group $\Pic(\bCP^1)$. This is the group of isomorphism classes of line bundles. For a line bundle, we can compute the \v{C}ech cohomology which give us another powerful invariant. Most of our work is to construct such invariants. We don't have a complete set of invariants for algebraic varieties.

\subsection{¿Why do divisors determine the line bundle?}

Given a divisor $D$ we can construct a line bundle $L$ with a section $s$ such that $\div(s)=D$. 

\begin{Ex}
Consider the base $\bCP^1$ and the divisor $D=d[\infty]$. Over $\bCP^1$ we have $\supp(D)=\set{\infty}$ so we build two open sets
$$U_\infty=\text{neighborhood of }\supp(D),\word{and} U_0=\bCP^1\less U_0.$$
In the chart $U_0$ we have the zero section and if there was a non-zero section $s$, then we might as well rescale all the fibers so that the section becomes $s(x)=1$. This is done by rescaling the fiber by $\frac{1}{s(x)}$. About $U_\infty$, we have that our section is $\set{v=y^d}$. So a good transformation would be 
$$y=\frac{1}{x},\ v=uf(x)=u/x^d.$$
\end{Ex}

In general for an arbitrary base of dimension $1$, we have a divisor 
$$D=\sum_{i=1}^n a_iP_i.$$
To construct $L$, we need an open cover of $X$ plus transition functions for any pair of intersecting sets. Let us take then 
$$U_0=\set{p_i}_{i\in\bonj{n}}^c,\ U_i=\text{open disk about }p_i.$$

\red{Ask Simeon for video}

\section{Serre Duality}

We have observed that 

$$\dim(H^0(\bP^1,\cO(d)))=\dim(H^1(\bP^1,\cO(-d-2))).$$

This follows from the fact that $H^0(\bP^1,\cO(d))$ is always isomorphic to the dual of $H^1(\bP^1,\cO(-d)\ox T^\ast_{\bP^1})$. This is also sometimes denoted the canonical divisor of $\bP^1$: 
And a modification if we have $X$ of dimension greater than 1, then we want to write this as
$$H^i(X,\cL)\isom H^{n-i}(X,\cL'\ox K_X)$$
where $K_X\isom\bigwedge^nT^\ast X$ is the canonical sheaf. This amounts to having a perfect pairing (non-degenerate bilinear form), a bilinear map 
$$H^0(\bP^1,\cO(d))\x H^1(\bP^1,\cO(-d)\ox T^\ast_{\bP^1})\to\bC$$
which means that if we represent this map by a basis and a matrix, the matrix is invertible. There is no vector on the left which maps to zero with a vector on the right unless it's zero. It's precisely like a dot product.\par
We have two ingredients to show this:
\begin{enumerate}
    \item We can identity $\bC$ with $H^1(\bP^1,T^\ast_{\bP^1})$. The particular cohomology group is one-dimensional.
    \item We have a natural map from the cartesian product to that $H^1$.
\end{enumerate}
Spelling this out carefully will give us our natural map. Even if we have $T^\ast_{\bP^1}\isom\cO(-2)$ here, we will see it with differential forms in order to apply it to other spaces.\par
In the more general case what we have is a map 
$$H^i(X,\cL_1)\x H^j(X,\cL_2)\to H^{i+j}(X,\cL_1\ox\cL_2).$$
Let's rewrite the \v{C}ech complex for the cotangent bundle:
$$T^\ast_{\bP^1}(U_0)\oplus T^\ast_{\bP^1}(U_\infty)\to T^\ast_{\bP^1}(U_0\cap U_\infty)$$
This is 
$$\bC[x]\dd x\oplus \bC[y]\dd y\to\bC[x,1/x]\dd x$$
and the map acts on monomials as 
$$
\left\lbrace
\begin{aligned}
    &(x^k\dd x,0)\mapsto x^k\dd x\\
    &(0,y^\l\dd y)\mapsto -(-x^{-\l}\dd x/x^2)=x^{-2-\l}\dd x.
\end{aligned}
\right.
$$
At the end of the day $H^1(\bP^1,T^\ast_{\bP^1})$ is represented by the cocycle $\dd x/x$. So $H^1(\bP^1,T^\ast_{\bP^1})=\genr{\dd x/x}_{U_0\cap U_\infty}$. The only monomial we are not catching is $1/x$, every other we've caught.\par
If we did this whole process tensoring with $\cO(-d)\ox-$, then on left we still have functions times $\dd x$ and functions times $\dd y$. The $y$ now transtitions as 
$$(0,y^\l\dd y)\mapsto -(-x^{-\l-d}\dd x/x^2)=x^{-2-\l-d}\dd x.$$
So in this case, for example in $d=5$, we will not catch things between $-1$ and $-5-1$:
$$H^1(\bP^1,\cO(-5)\ox T^{\ast}_{\bP^1})\isom\genr{\frac{\dd x}{x},\frac{\dd x}{x^2},\dots,\frac{\dd x}{x^6}}.$$
We don't get $\dd x/x^7$ because it's in the image of the map!\par
Now we would like to show that we have a map 
$$H^0(\bP^1,\cL_1)\x H^1(\bP^1,\cL_2)\to H^1(\bP^1,\cL_1\ox\cL_2)$$
at the level of the \v{C}ech complex:
$$C^0(\bP^1,\cL_1)\x C^1(\bP^1,\cL_2)\to C^1(\bP^1,\cL_1\ox\cL_2).$$
Elements of the first group are $\set{s_0,s_\infty}$ where $s_0\in L_1(U_0)$ and $s_\infty\in L_1(U_\infty)$ and on the other we have $u_{0\infty}\in L_2(U_0\cap U_\infty)$. Given this we want to produce a section $v_{0\infty}$ of $\cL_1\ox\cL_2(U_0\cap U_\infty)$.\par
We can take $s_0$ and restrict it: $s_0\mid_{U_0\cap U_\infty}\.u_{0\infty}$. This gives us a $v_{0\infty}$.\par
Our choice was biased! ¿Why didn't we choose $s_\infty$? The fact that $$\set{s_0,s_\infty}\in\ker d$$
means that $s_0\mid_{U_0\cap U_\infty}-s_\infty\mid_{U_0\cap U_\infty}=0$ which means that on the intersection they're the same.\par
So basically we have 
\begin{gather*}
H^0(\cO(d))\x H^1(\cO(-d)\ox T^\ast)\to H^1(T^\ast)\\
\To \genr{1,x,\dots,x^{d}}_{U_0}\ox\genr{\frac{\dd x}{x},\frac{\dd x}{x^2},\dots,\frac{\dd x}{x^{d+1}}}\to\genr{\frac{\dd x}{x}}.
\end{gather*}
The matrix of this pairing is the identity matrix and then it's obviously the \emph{dot product}. All the other $x^k\dd x=0$.

\chapter{Higher Genus}

\section{Riemann surfaces}

\begin{Def}
A \term{Riemann surface} is a complex analytic manifold of dimension $1$. 
\end{Def}

For every point, there's a neighborhood which is isomorphic to $\bC$ and transition functions are linear isomorphisms of $\bC$.

\begin{Ex}
    The following classes define Riemann surfaces.
    \begin{enumerate}
    \item $\bC$ itself is a Riemann surface with one chart.
    \item Any open set of $\bC$ is a Riemann surface.
    \item A holomorphic function $f\: U\subseteq\bC\to\bC$ defines a Riemann surface by considering $\Ga_f\subseteq\bC^2$. There's only one chart determined by the projection and the inclusion $i_{\Ga_f}$ is its inverse.
    \item Take another holomorphic function $f$, then $\set{f(x,y)=0}$ is a Riemann surface such that 
    $$\text{Sing}(f)=\set{\del_xf=\del_yf=f=0}=\emptyset.$$
    This means that at every point the gradient identifies a normal direction to the level set $f=0$. In particular, there's a well defined tangent line. To show that this is a complex manifold, we will use the inverse function theorem. 
    \item The first compact example is $\bCP^1$.
    \end{enumerate}
\end{Ex}

\section{20241009}

Our first examples of non-compact Riemann surfaces are images of holomorphic functions $\bC^2\to\bC$ such that $\text{Sing}(f)=\emptyset$. These implies that 
$$V(f)=\set{(x,y)\in\bC^2\: f(x,y)=0}$$
is a Riemann surface via the implicit function theorem.

\subsection{Compact Riemann surfaces}

Our first example is $\bCP^1$. But the next one is complex tori $\bC/\La$. Here $\La$ is a non-degenerate lattice:
$$\La= z_1\bZ\oplus z_2\bZ,\quad z_1,z_2\in\bC$$
where these numbers are linearly independent over $\bR$. This quotient is given by the equivalence relation 
$$x\sim y\iff x-y\in\La.$$
Another way to see this is to choose a fundamental domain which is the paralellogram $0,z_1,z_2,z_1+z_2$. Any point in $\bC$ is equivalent to a point inside, the pairs of parallel edges are equivalent and all the vertices are equivalent as well. Essentially what we are doing is building a torus (a real 2-torus) with this paralellogram.\par
This is a Riemann surface because $\bC$ induces a natural atlas via the quotient. For a point $x\in\bC/\La$ we get a chart by:
\begin{enumerate}
    \item Picking a point $z_x\in\bC$ a representative of the equivalence class of $x$.
    \item Then take a neighborhood about $z_x$, and call $U_x$ its image in $\bC/\La$
    \item Consider the inverse of the projection function $U_x\to\bC$ as our chart.
\end{enumerate}
Now pick a $y$ and a representative $z_y$. For elements in the intersection of the neighborhood, transition functions are given by translations.

\begin{Ex}
    Projective plane curves will be our next example. If we let $F\in\pre{h}\bC_d[X,Y,Z]$ be a homogeneous, degree $d$ polynomial such that $\text{Crit}(F)=\emptyset$ in $\bP^2$, then 
    $$V(F)=\set{[X:Y:Z]:\ F(X,Y,Z)=0}$$
    is a Riemann surface. To prove this, we rely on the implicit function theorem. We dehomogenize $F$ and then chart it via 
    $$V(F)\cap U_z=\set{(x,y)\:\ F(x,y,1)=0}.$$
\end{Ex}

\begin{Rmk}
Recall that homogeneous $\text{Crit}(F)$ contains $V(F)$ because $F$ is in the image of the Euler operator $x_i\del_i=d\id$. 
\end{Rmk}

\begin{Ex}
    Complete intersections of $(n-1)$ hypersurfaces in $\bP^n$ are the generalize the previous example. Take $F_i\in\pre{h}\bC_{d_i}[\un x]$ and 
    $$V(F_1,\dots,F_{n-1})=\set{[\un{X}]\in\bP^n\: F_1(\un X)=\dots=F_{n-1}(\un X)=0}$$
    then this is a Riemann surface. There is however a condition on the $F_i$'s, the gradient of $F_i$ at $P$ must give us different directions. This means that the matrix whose columns are $\nb_p F_i$, $i\in\bonj{n-1}$, has full rank. 
\end{Ex}

\begin{Ej}
    Find an expression in terms of determinants for that condition.
\end{Ej}

\begin{Ex}
    As a last example, we have local complete intersections. Consider
    $$C=V(X_0X_3-X_1X_2,X_0X_2-X_1^2,X_1X_3-X_2^2)\subseteq\bP^3,$$
    this is the image of the rational normal curve 
    $$\vf\:\bP^1\to\bP^3,\ [s:t]\mapsto[s^3:s^2t:st^2:t^3].$$
    Choosing only two of the equations, for example the first two, if we set $X_0=X_1=0$ we get a whole line's worth of points because $X_2,X_3$ are free to vary. So this gives us the curve plus a line. As it is the image of a map, then it's a Riemann surface.\par
    If we take any point on the curve besides the intersection point, we do get a complete intersection.
\end{Ex}

\begin{Ej}
    Verify that the image satisfied the polynomial equations.
\end{Ej}

\section{20241011}

Every space is locally approximated by tangent spaces. So giving orientation of manifolds is giving an orientation to all tangent in way that is coherent. And of course, in a chart, the tangent bundle trivializes so then we spread the orientation via fibers. On each individual chart we choose an orientation. But between charts we need to look at transition functions.\par
If for every transtition function, the determinant of the Jacobian of the transition is positive implies that all are orientation preserving.

\begin{Lem}
Any Riemann surface is orientable.
\end{Lem}

\begin{ptcbp}
A surface is orientable if and only if for any transition function $\phi_{ij}$ between $U_i,U_j$ we have 
$$\det d\phi_{ij}>0.$$
For a Riemann surface, $\phi_{ij}$ are holomorphic and in particular satisfy Cauchy-Riemann equations. Recall we identity $1=(1,0)$ and $i=(0,1)$ so that $\phi=u+iv$ where $u,v\:\bR^2\to\bR$. \blu{Finish writing derivation for CR eqns via drawing}
$$u_x=v_y\word{and}-u_y=v_x.$$
So for our transition function the Jacobian is 
$$\twobytwo{u_x}{u_y}{v_x}{v_y}=\twobytwo{u_x}{u_y}{-u_y}{u_x}=u_x^2+u_y^2>0.$$
So holomorphicity implies this and therefore any Riemann surface is orientable. 
\end{ptcbp}

Restricting our attention to compact Riemann surfaces we obtain a classification theorem. Non orientable are connected sums of projective planes while orientable ones are connected sums of tori.

\begin{Cor}
Any compact Riemann surface is homeomorphic to a genus $g$ surface.
\end{Cor}

We must interpret connected sum of zero tori as a sphere. Connected sum is surgery from topology. We remove a small open disk from both surfaces and then identifying both boundaries gives us the result.\par
The number of holes we obtain is an important invariant called the \term{genus}. This is a topological invariant.

\begin{Rmk}
    If two R.S. walk about to you with different genera, then they are most certainly non-isomorphic. In genus $1$ there's infinitely many non-isomorphic R.S.
\end{Rmk}

Some questions arise which will be able to answer:

\begin{Qn}
¿Are there Riemann surfaces of any genus?
\end{Qn}

\begin{Qn}
¿Are there general genus Riemann surfaces which are plane projective planes? This is, as the zero locus of polynomials in $\bP^2$.
\end{Qn}

The answer to the first question is yes, and the simplest way to construct one is to construct a hyperelliptic curve of genus $g$. This are built out of equations of the form 
$$y^2=f(x),\quad (x,y)\in\bC^2.$$
Generically this curves admit a degree $2$ map to $\bC$.\par
The second question has a negative answer. This should weird us out as we are defining hyperelliptic curves via a polynomial (this only works if the degree of $f$ is $3$ or $4$). In particular if $C=V(f)$ where $f$ is homogeneous of degree $d$ then 
$$g_C=\binom{d-1}{2}$$
so as not all integers are of the form $\binom{d-1}{2}$, then there's no genus $5$ hyperelliptic curve in $\bP^2$ for example.

\subsection{Families, singular curves and genus}

\begin{Def}
A \term{family of Riemann surfaces} is a space $\gX\xrightarrow{\pi}B$ such that
\begin{enumerate}
    \item For every $b\in B$, $\pi^{-1}(b)$ is a Riemann surface.
\end{enumerate}
\end{Def}

\begin{Ex}
    Consider $F,G\in\pre{h}\bC[X,Y,Z]_d$ then $\la F+\mu G$ is also homogeneous of degree $d$. $V(\la F+\mu G)$ defines a subvariety or a bihomogenous equation (degree $1$ in $\la,\mu$ and degree $d$ in $X,Y,Z$). This is our space $\gX$ in $\bP^1\x\bP^2$ and projecting $\pi_1$ gives us the family.
\end{Ex}

\begin{Ex}
    In a more specficic example we have 
    $$\la\bonj{(Y^2-X^2)Z+X^3}+\mu XZ^2=0.$$
    It is well defined to say whether a point $(\la_0:\mu_0),(X_0:Y_0:Z_0)$ is a solution to the equation above. So it makes sense to ask for points $((\la_0:\mu_0),(X_0:Y_0:Z_0))\in\bP^1\x\bP^2$ solutions of this equation. As the parameters vary we get Riemann surfaces.\par
    However dehomogenizing $(Y^2-X^2)Z+X^3$ gives us $y^2-x^2+x^3$, the nodal curve. The singular locus of this curve contains $(0,0)$, so it's singular at $(0,0)$. About the origin this looks like $\dd y^2=\dd x^2-\dd x^3=\dd x^2$ which is the union of the diagonals. This is a \term{nodal singularity}. It turns out that if we want, we can care only about nodal singularity. If we get a family where curves vary and at some point it degenerates into worse than nodal singularities, then we can replace it with a nodal curve.
\end{Ex}

\section{20241014}

Lat time we were talking about families. We want the thing that \emph{controls} the variation of the Riemann surface to also be a Riemannian manifold. As they vary we would like to form a global object. We shoud think of the variety $B$ to be a parameter space.\par
Our example served to show us that over compact spaces we may have singularities. From a topological point of view, we don't quite know it yet but\dots

\begin{Ex}
A smooth cubic curve in $\bP^2$ has genus 1. As we get closer to zero in the base, one of the loops starts to contract we obtain a node.\par
This pinched torus should have genus $1$. But it's also really close to being a sphere because I've taken two points and pinched them together. For nodal curves the notion of genus becomes two: arithmethic and geoemtric. Arithmethic has to do with algebra and it remains invariant in families. In particular for families resolving the singularity: One of the fibers is our nodal curve and nearby fibers are tori. So this pinched torus has arithmethic genus 1.\par
On the other hand geometric genus stays invariant under birational transformations. In this particular example, there's no birational transformation between the pinched torus and the torus. However there's a birational transformation called normalization which takes the open set of the sphere minus two points to the pinched torus minus the singularity.
\end{Ex}

\subsection{Maps to and from Riemann Surfaces}

We have seen that Riemann surfaces are orientable complex manifolds. 

\begin{Def}
We say that a function $f\: C\to\bC$ is holomorphic at $x$ if at a chart $(U_x,\vf_x)$ we have $(\vf_x^{-1})^\ast(f)$ is holomorphic at $\vf_x(x)$.
\end{Def}

We have a plethora of theorems imported from complex analysis. 

\begin{Prop}
    The following properties hold for holomorphic functions on Riemann surfaces.
\begin{itemize}
    \item Zeroes of holomorphic functions are discrete sets.
    \item If $f$ is non-constant and holomorphic, then it's an open mapping. 
    \item If $C$ is a compact Riemann surface, then $f\: C\to\bC$ being holomorphic implies it's constant. 
\end{itemize}
\end{Prop}

The set of holomorphic functions cannot tell apart two Riemann surfaces.

\begin{ptcbp}
    We will only prove the last item. If $f$ was not constant, $f(C)\subseteq\bC$ would be open. Also, $f(C)$ is compact in $C$. This contradicts the fact that $f$ is non-constant.
\end{ptcbp}

So now we turn our attention to meromorphic functions as holomorphic ones are disappointing. 

\begin{Ex}
    Let us see meromorphic functions on compact Riemann surfaces:
    \begin{enumerate}
        \item Rational functions on $\bCP^1$.
        \item In complex tori: $f\:\bC/\La\to\bC$, we have quotient function. This is function should be a meromorphic function on $\bC$ which is invariant under the torus action. These are called \emph{doubly-periodic}:
        $$f(z+\la)=f(z),\quad\la\in\La.$$
        One such example is \emph{Weierstrass}'s $\wp$ function:
        $$\wp(z)=\frac{1}{z^2}+\sum_{\la\in\La\less\set{0}}\left(\frac{1}{(z-\la)^2}-\frac{1}{\la^2}\right)$$
        which holomorphic outside every lattice point and has poles of order $2$ at any $\la\in\La$. Observe that if $\tilde{\la}\in\La$ then 
        $$\wp(z+\tilde{\la})=\frac{1}{(z+\tilde{\la})^2}+\sum_{\la\in\La\less\set{0}}\left(\frac{1}{(z+\tilde{\la}-\la)^2}-\frac{1}{\la^2}\right)$$
        which is just a reindexing of the previous summation. The $1/z^2$ term appears when we go over the $\la=\tilde{\la}$ term in the sum.\par
        It also occurs that $\wp$ is even. This fact tells us that we get a pole of order $2$ at every point in the lattice, but getting the Laurent expansion tells us that $a_{-1}$, the residue is zero. In other words the expansion looks like
        $$\dots+\frac{a_{-2}}{z^2}+a_0+a_2z^2+\dots.$$
        \item Another meromorphic function is 
        $$\wp'(z)=-2\left(\frac{1}{z^3}+\sum_{\la\in\La\less\set{0}}\frac{1}{(z-\la)^3}\right)$$
        which is now odd, has a pole of order $3$ at every lattice point with expansion 
        $$\frac{a_{-3}}{z^3}+a_1z+\dots$$
        we don't get a $z^{-1}$ term as taking the derivative term by term produces no such term!
        \item Observe that now $(\wp'(z))^2$ has poles of order six while $\wp^3(z)$ does as well! We may choose a constant so that the poles of order $6$ of $(\wp')^2-A\wp^3$ cancel! Our Laurent series look like 
        $$\left(\frac{a_{-6}}{z^6}+\frac{a_{-2}}{z^2}+a_0\right)+A\left(\frac{b_{-6}}{z^6}+\dots\right).$$
        We can also cancel the pole of order 2 with the original Weierstrass function: $(\wp')^2-A\wp^3-B\wp$. With this we have cancelled all of the poles which means that this is holomorphic and therefore constant. This is a functional equation between the Weierstrass function and its derivative. Therefore
        $$(\wp')^2-A\wp^3-B\wp=C$$
        and $(\wp,\wp')\:\bC/\La\less\set{\bonj{\la}}\to\bC^2$ where we map $z\mapsto(x,y)=(\wp(z),\wp'(z))$. What this equation tells us is that the image of this map is contained the locus 
        $$y^2+Ax^3+Bx=C.$$
        We would have to do a bit more work to see that that curve is exactly all of the image.
        \item Our library of examples of Riemann surfaces reduces to projective curves. If $C\subseteq\bP^n$ is a projective curve, then we can obtain any meromorphic function on $C$ by getting a meromorphic function on $\bP^n$ and restricting to $C$, so long as $C$ is not entirely contained in $V(Q)$ where the original function is $P/Q$. Otherwise it becomes the constant function infinity.\par
        In particular take the conic $V(X^2+Y^2+Z^2)=C\subseteq\bP^2$. Take the map $\bP^2\to\bC$ (dotted), $(X:Y:Z)\mapsto X/Z$. Then this function is not defined when $Z=0$, but everywhere else it is. This function is precisely the linear projection onto the first coordinate from $(0:1:0)$. This is really a map from $\bP^2\less\set{(0:1:0)}$ to $\bC$ with exactly two tangents. If we claim that the image of $Z=0$ to the $X$ axis is the point at infinity, this becomes a map $\bP^2\to\bP^1$, becoming our first example of maps \emph{between} Riemann surfaces.
    \end{enumerate}
\end{Ex}
%%%%%%%%%%%% Contents end %%%%%%%%%%%%%%%%

\ifx\nextra\undefined
\printindex
\else\fi
\nocite{*}
\bibliographystyle{plain}
\bibliography{bibiRiemannSurfaces.bib}
\end{document} 

