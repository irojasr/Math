\documentclass[a4paper,12pt,final]{book}
\usepackage[margin=1in]{geometry}
 
\usepackage{fontspec}
\usepackage{xltxtra}
\usepackage{microtype}

\usepackage{amsmath}
 
\defaultfontfeatures{Ligatures={Required, Common, Contextual, TeX}}
\setmainfont{FreeSerif}
 
\usepackage{setspace}
\linespread{1} % equivalent to MS Word's 1.5 spacing

 
\begin{document}
It's been a whole semester full of learning experiences, both my students and myself have made the effort to get to this point and we've made it. In this moment of reflection and introspection I can think about \emph{the could've been's}, the \emph{oh I should've that differently}, or the \emph{thinking about it, that actually was good} moments which have molded my opinions and expectations up to this point. My original thoughts are similar to the actual ones, and in the end, my calling remains the same. I strive to give my students the best of me, the difference now is that I feel accompanied, so it is \emph{our collective duty} to give it our best.\par 
My mind still is centered in the idea that the \textbf{best way to learn mathematics }is 
$$\text{Receive knowledge}\Rightarrow\text{Abstract it}\Rightarrow\text{Find analogies}.$$
This is the summarized version of what I previously called the ``\emph{tried and true}'' method. However throughout this semester I've been able to incorporate principles of equity and community which have reinforced this process.
\end{document}