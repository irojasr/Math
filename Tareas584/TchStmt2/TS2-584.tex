\documentclass[a4paper,12pt,final]{book}
\usepackage[margin=1in]{geometry}
 
\usepackage{fontspec}
\usepackage{xltxtra}
\usepackage{microtype}

\usepackage{amsmath}
 
\defaultfontfeatures{Ligatures={Required, Common, Contextual, TeX}}
\setmainfont{FreeSerif}
 
\usepackage{setspace}
\linespread{1} % equivalent to MS Word's 1.5 spacing

 
\begin{document}
It's been a whole semester full of learning experiences, both my students and myself have made the effort to get to this point and we've made it. In this moment of reflection and introspection I can think about \emph{the could've been's}, the \emph{oh I should've that differently}, or the \emph{thinking about it, that actually was good} moments which have molded my opinions and expectations up to this point.\par 
My original thoughts are similar to the actual ones, and in the end, my calling remains the same. This is something that I realized when I started thinking to myself \emph{why did I enjoy the education seminar as I do?} I can finally give a concrete answer now: \emph{because I realized that my love for mathematics compels me to share it with my students}. For this, I strive to give my students the best of me. And the difference now is that I feel accompanied, so it is not only my duty, it is \textbf{our collective duty} to give it our best.  \par 
My mind still is centered in the idea that the \textbf{best way to learn mathematics }is 
$$\textbf{Receive knowledge}\Rightarrow\textbf{Abstract it}\Rightarrow\textbf{Find analogies}.$$
This is the summarized version of what I previously called the ``\emph{tried and true}'' method. However throughout this semester I've been able to incorporate principles of community and sympathy between students which have reinforced this process, they responded positively to these approaches. The key procedures I followed were:
\begin{itemize}
    \itemsep=-0.5em
    \item \textbf{Waiting} in between questions for my students to finish their writing and think about what they just wrote. 
    \item Asking \textbf{other students}, the ones with lower grades, what did they about what we were talking. 
    \item Switching groups a couple of more times during the semester and suggesting they \textbf{exchanged their contacts}. 
    \item Opening myself a bit and asking them about \textbf{how was their semester going}. 
\end{itemize}
Not all of my students continue to go to class. 
Fot the ones that still do, success is shown in a small increase in their grades, and also by their attention and attitude towards the class. 
Particularly, directing the attention towards students who often did not participate, made them more eager to do so and showed other students that the class was not solely composed of themselves. 
In turn, this helped construct a \textbf{sense of community} among the students. After noticing more students, apart from the ones who consistently tend to answer, engage more with the material and talk more to their own classmates, I believe I have succeeded in building a type of community.\par 
Change is hard, especially when one has grown accustomed to a certain set of practices. In this vein, my belief about teaching has little changes from the last time, \textbf{anyone can learn mathematics} and \textbf{mathematics is a collaborative effort}. I tried to prove that my students could \textbf{fend for themselves}, and I wanted to believe it, but in the end it was a fruitless attempt. However I have learned that I can now help my students according to those beliefs and what I've been taught.\par
When saying the \emph{anyone} can do it, I thought anyone had the capacity, but I wasn't actively enabling them to do it. 
Now I actually have strategies, not only when openly asking questions to my class, to help those students who might not feel confident.
Another strategy which I found useful in this sense was to reach to the quiet student in a subgroup when they were doing groupwork, even if the correct answer had already been mentioned. There was a notable difference in their expressions when I asked what their opinion or thoughts were.\par 
To help students \emph{collaborate}, during the last two exchanges of subgroups I encouraged them to exchange contacts, and also I tried to group them according to their preferences. On one side, this made students particularly happy, and on the other, it helped them engage with the material among themselves.\par 
Good results were not the case when making students learn \emph{for themselves}. I found it very difficult to help students when trying to help them \emph{discover} the material on their own. Doing this requires more preparation which I must admit I didn't have, I almost felt like building the train-track while the train was running. Luckily for my future students and myself, the coming semester I will be teaching the same course again, and during the break it is my objective to construct and design \emph{richer tasks} for the days in which I intend to have students discover topics on their own.\par 
Talking about my mental image of a \textbf{good professor}, I can talk about the positive side without much comparisons to the bad experiences I had. It is very refreshing to be able to describe what good teaching should look like without referencing what the not-as-good teachers did. Not only that but in general, my views have been extended in the sense that good teaching doesn't explicitly require a professor. However beginning with this idea, my idea of a good professor is one that has the following qualities:
\begin{itemize}
    \itemsep=-0.5em
    \item Is \textbf{sincere and honest}, and tells students a truth even if it might harsh to listen. 
    \item Good professors \textbf{lead} and do not \textbf{show-off}. Students are lead along the path of knowledge. But it's not holding their hands all the way and not through a paved road all the way. 
    \item They \textbf{empathize} with students without transgressing their personal boundaries. 
\end{itemize}
These personal qualities along with organization in class are what defines a good professor. But as I mentioned, I no longer believe only in good teaching through a professor. Good teaching also requires that students themselves engage with the topic. This means that students will go on their own and a have a tiny spark of curiosity which will itch them to think on their own. Good teaching promotes such moments through comments from the professor, from the students' own classmates or from fun exercises which make the student \emph{think}. In essence, to my original thoughts of good teaching I've added a component of personal work which I believe students must do themselves.\par 
I still want my students to go out of my class and think \emph{this was enjoyable} or after having piqued their interest with some example, I hope that they search for it on the internet. The knowledge I'm giving my students right now is not something that they will use immediately nor on a daily basis on their jobs. But the way to obtain that knowledge, \textbf{the way to think} is what I want them to extract out my class. As most of my students are engineers, I want them to think like one. What they will be doing most of the time is to solve problems. They might involve something tangentially related to the class and for that it's important that they know how find \emph{analogies} between that and what they know. It's not about being the ultimate mathematician, it's about finding and recognizing analogous patterns between what they will be doing and what they currently know. If in addition I can show them the joys of what it truly means to do mathematics, then I will say that that is all that I want them to leave with. With a bit of my own love towards this beautiful subject.\par 
Finally, I'd like to thank \textbf{you}, not only for reading, but also for having the enthusiasm to impart on us your teachings throughout the semester. I will strive to improve myself based on what I've learned and never stop. I know that even when one thinks that \emph{this} is the end, it is not. I leave with one my favorite quotes which has driven me for a long time. 

\begin{flushright}
    \emph{So\~n\'e que pod\'ia.}\\
    \emph{Y pude.}\\
    \emph{Vamos por m\'as.}\\
    Ignacio Rojas
\end{flushright}
\end{document}