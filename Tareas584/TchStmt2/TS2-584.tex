\documentclass[a4paper,12pt,final]{book}
\usepackage[margin=1in]{geometry}
 
\usepackage{fontspec}
\usepackage{xltxtra}
\usepackage{microtype}

\usepackage{amsmath}
 
\defaultfontfeatures{Ligatures={Required, Common, Contextual, TeX}}
\setmainfont{FreeSerif}
 
\usepackage{setspace}
\linespread{1} % equivalent to MS Word's 1.5 spacing

 
\begin{document}
It's been a whole semester full of learning experiences, both my students and myself have made the effort to get to this point and we've made it. In this moment of reflection and introspection I can think about \emph{the could've been's}, the \emph{oh I should've that differently}, or the \emph{thinking about it, that actually was good} moments which have molded my opinions and expectations up to this point. My original thoughts are similar to the actual ones, and in the end, my calling remains the same. I strive to give my students the best of me, the difference now is that I feel accompanied, so it is \textbf{our collective duty} to give it our best.\par 
My mind still is centered in the idea that the \textbf{best way to learn mathematics }is 
$$\textbf{Receive knowledge}\Rightarrow\textbf{Abstract it}\Rightarrow\textbf{Find analogies}.$$
This is the summarized version of what I previously called the ``\emph{tried and true}'' method. However throughout this semester I've been able to incorporate principles of community and sympathy between students which have reinforced this process, they responded positively to these approaches. The key procedures I followed were:
\begin{itemize}
    \itemsep=-0.5em
    \item \textbf{Waiting} in between questions for my students to finish their writing and think about what they just wrote. 
    \item Asking \textbf{other students}, the ones with lower grades, what did they about what we were talking. 
    \item Switching groups a couple of more times during the semester and suggesting they \textbf{exchanged their contacts}. 
    \item Opening myself a bit and asking them about \textbf{how was their semester going}. 
\end{itemize}
Not all of my students continue to go to class. 
Fot the ones that still do, success is shown in a small increase in their grades, and also by their attention and attitude towards the class. 
Particularly, directing the attention towards students who often did not participate, made them more eager to do so and showed other students that the class was not solely composed of themselves. 
In turn, this helped construct a \textbf{sense of community} among the students. After noticing more students, apart from the ones who consistently tend to answer, engage more with the material and talk more to their own classmates, I believe I have succeeded in building a type of community.\par 
My belief about teaching is no different than it was last time, \textbf{anyone can learn mathematics} and \textbf{mathematics is a collaborative effort}. The difference now, is that I can now help my students according to those beliefs thanks to what I've learned.
When saying the \emph{anyone} can do it, I thought anyone had the capacity, but I wasn't actively enabling them to do it. 
Now I actually have strategies, not only when openly asking questions to my class, to help those students who might not feel confident.
Another strategy which I found useful in this sense was to reach to the quiet student in a subgroup when they were doing groupwork, even if the correct answer had already been mentioned. There was a notable difference in their expressions when I asked what their opinion or thoughts were.\par 
To help students \emph{collaborate}, during the last two switchings of subgroups I encouraged them to exchange contacts, and also I tried to group them according to their preferences. On one side, this made students particularly happy, and on the other, it helped them engage with the material among themselves. 
\end{document}