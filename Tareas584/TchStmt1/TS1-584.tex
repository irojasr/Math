\documentclass[a4paper,12pt,final]{book}
\usepackage[margin=1in]{geometry}
 
\usepackage{fontspec}
\usepackage{xltxtra}
\usepackage{microtype}
 
\defaultfontfeatures{Ligatures={Required, Common, Contextual, TeX}}
\setmainfont{FreeSerif}
 
\usepackage{setspace}
\linespread{1} % equivalent to MS Word's 1.5 spacing

 
\begin{document}
 This essay is a reflection on my learning process, as well as some of my classmates' from the graduate program, and my old classmates from Costa Rica.\par 
I began by questioning my own learning process which I cam summarize into 5 main points:
\begin{enumerate}
    \itemsep=-0.4em
    \item Receive an explanation about the topic with very concrete and basic examples. 
    \item Convert the concrete examples into abstract ones with instructions from the professor. 
    \item Practice with other basic examples while being guided.
    \item Start questioning myself in search of a general pattern while solving examples on my own.
    \item Comprehend the pattern and become able to produce my own examples and questions.
\end{enumerate}
This process didn't include evaluation but when there's a professor, that issue must be considered. After asking around my desk, then my office, and internationally, I noticed most of the experiences were similar. Sincerely my dataset is pretty biased however, I feel that anyone else that I could ask would describe a similar experience.\par 
I concluded that this way of learning mathematics is the ``\emph{tried and true}'' way of doing it. Being in grad school or having a job is already a measure of success, and since all the people I asked were in one of those two sets, it follows that \textbf{this way of teaching produces good results}. I also know that I didn't measure the quality of the contents of what they learned. Even though I consider that part of the ``best learning process'' is the quality of the information received and retained, I had to assume that for everyone I asked, the quality was implicitly decent at least.\par 
Even though my belief is that the ``\emph{tried and true}'' way is effective, I must admit that I have my doubts. I can't confirm for certain that this is the best; through my studies I have had some tribulations which, in some cases, made the experience not as good in some sense. We will explore these experiences later in this essay. For now, let us talk about my own practices.\par 
%To talk about my practices, I first was asked about did my beliefs influenced my practices\footnote[1]{I must admit that this question's wording confused me and I had to read it multiple times.} This led me to question which were my beliefs, and then expand upon them to see how I was managing my their learning process according to my thoughts. I can summarize this beliefs, once again, in several points:
Regarding \textbf{the ``\emph{tried and true}'' method}, I believe it \textbf{is the best method to learn mathematics}. I tend to use more examples while giving lectures instead of reciting the result and staying mainly in the theoretical side of things. This is combined with working sessions in which the students work in groups. While walking around to check on my students' progress, as soon as I see a mistake I point it out to them.\par 
After finishing, I review the exercises on the board while providing examples of \emph{slippery slopes} which I might've noticed some of my students fell into. During this explanations I tend to pause and ask for questions. If no one asks directly, I try to seek a student who's watching the board closely and ask them to share their thoughts. And lastly I hand them additional resources so they might study on their own.\par 
Along this lines, I believe that \textbf{everyone can learn mathematics.} There's no type of background that can make a person incapable of learning mathematics. Not any type of discrimination; it doesn't even matter if someone doesn't know how to count or if they are a math olympic medalist. I try to treat everybody in the same way regardless of their capacities. Any student can ask any question such as ``\emph{can you repeat?}'', ``\emph{how did you go from here to there?}'', ``\emph{which was the formula?}'', or even , ``\emph{how will I apply this in the future?}''\par 
No question is out of place and I strive to create an environment where students will feel safe about asking questions.\par 
Even if studying on one's own time is important, I firmly believe that \textbf{mathematics is a collaborative effort}. In that sense, greater progress is achieved when people work together. My students work among themselves and I encourage them to compare their processes to see how they did it and in some cases explain to one another.\par 
The resources which I give them: the webpages with exercises and solutions, video-lectures from other professors, or other books which explain the subjects that I'm teaching in another way are tacit ways of collaborating.\par 
There's also an important point to talk about regarding help, and that is the ``\emph{impostor syndrome}''. Students who seek help shouldn't feel ashamed about it, looking up an exercise on the web, understanding it and writing it in their own words does not take their merit away. Those are still theirs even if they looked up an exercise. I remind them of this fact because I have felt it numerous times as well.\par 
All of the parts of my learning processes have been thoroughly influenced by my own professors. My ideal of what good mathematics teaching looks like comes from them, and also it is formed by the complement of what my bad professors did. Implicitly collective opinions are precisely what orients me to call a professor good or bad and therefore measure the success of my claims.\par 
Most of my good professors had several personal qualities and teaching practices which make them stand out. Personally, their attitude is one the most important characteristics. Towards the subject itself, and the students as well. A good professor is enthusiastic about what they are teaching. They are also modest and accessible, not egocentric and arrogant.\par
These qualities can be expressed through nice handwriting, with even-tempered voice when responding and while lecturing, and timing exposition to go right with the schedule.\par
There is one professor who marked me because he helped me realize I had to continue my maturing process in math. The only class which I failed was the one that he taught me, and the value which I learned from this professor was honesty. Many times, professors will spoil students to some points, but this one didn't. Not only that, he also is honest when stating one's progress. This is something which I mean to do with my students.\par 
I want my students to know that I am a honest person who will not hand them an A in the course for free. That I will evaluate them justly without bias.\par 
Mathematics is the discipline which I love and want to dedicate my life to, I get a feeling of amazement many times when I see new topics and discover new patterns within the ones I already knew. This curiosity and enjoyment is part of what I try to show when I am teaching. My hope is that my students will feel the same or at least similar to me during the course.\par 
Sadly I've been in classes with professors which I regret having, so I understand when my students form their opinions about mathematicians. Those might be negative opinions, but my hope is that I can clear that image a little bit. Not all of the mathematicians are grumpy and egotistical. In the end, I want them to feel satisfied about having taken the course with myself.\par 
As a final remark, I'd like to add that their experiences outside the class are as important as the lectures which I give them. I wish that my students feel motivated about working on a problems which piques their curiosity, that they will work on it and then ask about it with their classmates or myself. In the end I strive to give my students the experiences which I missed and improve on what was already an enjoyable and profitable.

\end{document}