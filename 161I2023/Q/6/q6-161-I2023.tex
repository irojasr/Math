%----------------------------------------------------------------------------------------
%	PACKAGES AND OTHER DOCUMENT CONFIGURATIONS
%----------------------------------------------------------------------------------------

\documentclass[12pt]{article}
\usepackage[spanish]{babel} %Tildes
\usepackage[extreme]{savetrees} %Espaciado e interlineado. Comentar si no gusta el interlineado.
\usepackage[utf8]{inputenc} %Encoding para tildes
\usepackage[breakable,skins]{tcolorbox} %Cajitas
\usepackage{fancyhdr} % Se necesita para el título arriba
\usepackage{lastpage} % Se necesita para poner el número de página
\usepackage{amsmath,amsfonts,amssymb,amsthm} %simbolos y demás
\usepackage{mathabx} %más símbolos
\usepackage{physics} %simbolos de derivadas, bra-ket.
\usepackage{multicol}
\usepackage[customcolors]{hf-tikz}
\usepackage[shortlabels]{enumitem}
\usepackage{tikz}
\usetikzlibrary{patterns}
\usepackage{siunitx}

%\def\darktheme
%%%%%%%%% === Document Configuration === %%%%%%%%%%%%%%

\pagestyle{fancy}
\setlength{\headheight}{14.49998pt} %NO MODIFICAR
\setlength{\footskip}{14.49998pt} %NO MODIFICAR

\ifx \darktheme\undefined

\lhead{Math161S1} % Nombre de autor
\chead{\textbf{Quiz 6 - Solutions}} % Titulo
\rhead{Name:\hspace*{5cm}}%\firstxmark} 
\lfoot{}%\lastxmark}
\cfoot{}
\rfoot{Page \thepage\ of\ \pageref{LastPage}} %A la derecha saldrá pág. 6 de 9. 
\else
\pagenumbering{gobble}
\pagecolor[rgb]{0,0,0}%{0.23,0.258,0.321}
\color[rgb]{1,1,1}
\fi

%%%%%%%%% === My T Color Box === %%%%%%%%%%%%%%

\ifx \darktheme\undefined
\newtcolorbox{ptcb}{
colframe = black,
colback = white,
breakable,
enhanced
}
\newtcolorbox{ptcbP}{
colframe = black,
colback = white,
coltitle = black,
colbacktitle = black!40,
title = Practice,
breakable,
enhanced
}

\else
\newtcolorbox{ptcb}{
colframe = white,
colback = black,
colupper = white,
breakable,
enhanced
}
\newtcolorbox{ptcbP}{
colframe = white,
colback = black,
colupper = white,
coltitle = white,
colbacktitle = black,
title = Practice,
breakable,
enhanced
}
\fi

%%%%%%%%% === Tikz para matrices === %%%%%%%%%%%%%%

\tikzset{
  style green/.style={
    set fill color=green!50!lime!60,
    set border color=white,
  },
  style cyan/.style={
    set fill color=cyan!90!blue!60,
    set border color=white,
  },
  style orange/.style={
    set fill color=orange!80!red!60,
    set border color=white,
  },
  row/.style={
    above left offset={-0.15,0.31},
    below right offset={0.15,-0.125},
    #1
  },
  col/.style={
    above left offset={-0.1,0.3},
    below right offset={0.15,-0.15},
    #1
  }
}

%%%%%%%%% === Theorems and suchlike === %%%%%%%%%%%%%%

\theoremstyle{plain}
\newtheorem{Th}{Theorem}  %%% Theorem 1.1
\newtheorem*{nTh}{Theorem}             %%% No-numbered Theorem
\newtheorem{Prop}[Th]{Proposition}     %%% Proposition 1.2
\newtheorem{Lem}[Th]{Lemma}             %%% Lemma 1.3
\newtheorem*{nLem}{Lemma}               %%% No-numbered Lemma
\newtheorem{Cor}[Th]{Corollary}        %%% Corollary 1.4
\newtheorem*{nCor}{Corollary}          %%% No-numbered Corollary

\theoremstyle{definition}
\newtheorem*{Def}{Definition}       %%% Definition 1.5
\newtheorem*{nonum-Def}{Definition}    %%% No number Definition
\newtheorem*{nEx}{Example}             %%% No number Example
\newtheorem{Ex}[Th]{Example}           %%% Example
\newtheorem{Ej}[Th]{Exercise}         %%% Exercise
\newtheorem*{nEj}{Exercise}           %%% No number Excercise
\newtheorem*{Not}{Notation}       %%% Definition 1.5

\theoremstyle{remark}
\newtheorem*{Rmk}{Remark}      %%%Remark 1.6

%\numberwithin{equation}{section}

\setlength{\parindent}{3ex}

%%====== Useful macros: =======%%%

\DeclareMathOperator{\gen}{gen}     %%%set generated by...
\DeclareMathOperator{\Rng}{Rng}     %%%rangomat
\DeclareMathOperator{\Nul}{Nul}     %%%rangomat
\DeclareMathOperator{\Proy}{Proy}   %%%proyección
\DeclareMathOperator{\id}{id}       %%%identity operator

\newcommand{\al}{\alpha}            %%%short for \alpha
\newcommand{\la}{\lambda}           %%%short for \lambda
\newcommand{\sg}{\sigma}            %%%short for \sigma
\newcommand{\te}{\theta}                %% short for  \theta
\renewcommand{\l}{\ell}

\newcommand{\thickhat}[1]{\mathbf{\hat{\text{$#1$}}}}
\newcommand{\ii}{\vu{\imath}}
\newcommand{\jj}{\vu{\jmath}}
\newcommand{\kk}{\thickhat{k}}

\newcommand{\bC}{\mathbb{C}}        %%%complex numbers
\newcommand{\bN}{\mathbb{N}}        %%%natural numbers
\newcommand{\bP}{\mathbb{P}}        %%%polynomials
\newcommand{\bR}{\mathbb{R}}        %%%real numbers
\newcommand{\bZ}{\mathbb{Z}}        %%%integer numbers
\newcommand{\cB}{\mathcal{B}}       %%%basis
\newcommand{\cC}{\mathcal{C}}       %%%basis
\newcommand{\cM}{\mathcal{M}}       %%%matrix family

\newcommand{\sT}{\mathsf{T}}        %%%traspuesta

\renewcommand{\geq}{\geqslant}      %%%(to save typing)
\renewcommand{\leq}{\leqslant}      %%%(to save typing)
\newcommand{\x}{\times}             %%%product
\renewcommand{\:}{\colon}           %%%colon in  f: A -> B
\newcommand{\isom}{\simeq}              %% isomorfismo

\newcommand{\un}[1]{\underline{#1}}
\newcommand{\half}{\frac12}

\newcommand*{\Cdot}{{\raisebox{-0.25ex}{\scalebox{1.5}{$\cdot$}}}}      %% cdot más grande
\renewcommand{\.}{\Cdot}                %% producto escalar

\newcommand{\twobyone}[2]{\begin{pmatrix} %% 2 x 1 matrix
  #1 \\ #2 \end{pmatrix}}
  \newcommand{\twobytwo}[4]{\begin{pmatrix} %% 2 x 2 matrix
    #1 & #2 \\ #3 & #4 \end{pmatrix}}
    \newcommand{\twobythree}[6]{\begin{pmatrix} %% 2 x 3 matrix
        #1 & #2 & #3\\ #4 & #5 & #6 \end{pmatrix}}
\newcommand{\threebyone}[3]{\begin{pmatrix} %% 3 x 1 matrix
  #1 \\ #2 \\ #3 \end{pmatrix}}
  \newcommand{\threebytwo}[6]{\begin{pmatrix} %% 3 x 1 matrix
    #1 & #2\\ #3 & #4\\ #5&#6 \end{pmatrix}}
\newcommand{\threebythree}[9]{\begin{pmatrix} %% 3 x 3 matrix
  #1 & #2 & #3 \\ #4 & #5 & #6 \\ #7 & #8 & #9 \end{pmatrix}}

\newcommand{\To}{\Rightarrow}

\newcommand{\vaf}{\overrightarrow}

\newcommand{\set}[1]{\{\,#1\,\}}    %% set notation
\newcommand{\Set}[1]{\biggl\{\,#1\,\biggr\}} %% set notation (large)
\newcommand{\red}[1]{\textcolor{red}{#1}}
\newcommand{\blu}[1]{\textcolor{blue}{#1}}

%----------------------------------------------------------------------------------------
%	ARTICLE CONTENTS
%----------------------------------------------------------------------------------------

\begin{document}
%\begin{multicols}{2}

\begin{Ej}
 Consider a region $R$ bounded by the curves
 $$y=x,\quad y=-x,\quad\text{and}\quad x=1,$$
 additionally the region has density $\rho(y)=e^y$. Now suppose we rotate the region about the axis $y=-4$. Do the following:
  \begin{enumerate}[i)]
    \itemsep=-0.4em
    \item Draw the region in question.
    \item Draw the solid of revolution obtained after rotation.
    \item Which method should we use to find the volume of this shape?
    \item Find the bounds of the region. Label them either as $a\leq x\leq b$ or $c\leq y\leq d$.
    \item Find the \texttt{GREATER} and \texttt{LOWER} curves by writing their equations.
    \item Find the parameters (\emph{either $R,r$ or $r,h$}) used to build your area function. 
    \item With the previous information, write out the integral which represents the mass of the solid obtained.
  \end{enumerate}
\end{Ej}
\begin{ptcb}
\begin{enumerate}[i)]
  \itemsep=-0.4em 
  \item See diagram.
  \item See diagram.\vspace{2cm}
  \item As we have a rotation about a $y=b$ line, we can either use shells in $y$ or rings in $x$. The density is in terms of $y$ so we should use shells in $y$.
  \item The regions starts at $y=-1$ and ends at $y=1$. So $-1\leq y\leq 1$.
  \item We must divide the region into two pieces:
  $$
  \left\lbrace
  \begin{aligned}
    &\text{GREATER: } x=1\\
    &\text{LOWER1: } x=-y\quad\text{for}\quad-1\leq y\leq 0
  \end{aligned}
  \right.
  \quad\text{and}\quad
  \left\lbrace
  \begin{aligned}
    &\text{GREATER: } x=1\\
    &\text{LOWER2: } x=y\quad\text{for}\quad 0\leq y\leq 1
  \end{aligned}
  \right.
  $$
  \item As we are using shells we must find $r,h$. These are:
  $$  \left\lbrace
  \begin{aligned}
    & h_1=\text{GREATER}-\text{LOWER1}=[1-(-y)]\quad\text{for}\quad -1\leq y\leq 0\\
    & h_2=\text{GREATER}-\text{LOWER2}=[1-(y)]\quad\text{for}\quad 0\leq y\leq 1\\
    & r=\text{dist}(\text{axis},\text{coordinate})=y-(-4)
  \end{aligned}
  \right.
  $$
  \item Adding this facts together we get 
  $$m=\int_{-1}^02\pi(y+4)(1+y)e^y\dd y+\int_0^12\pi(y+4)(1-y)e^y\dd y.$$
\end{enumerate}
\end{ptcb}

\begin{Ej}
 Consider the tank formed after rotating the curve $y=x^3$ with $0\leq x\leq 1$ about the axis $x=0$. Suppose tank is filled with \emph{radioactive waste} with density $\rho(y)=100+25y^2$. Do the following:
 \begin{enumerate}[i)]
  \itemsep=-0.4em 
  \item Draw a cross-section of the tank.
  \item Make a diagram of an infinitesimal slice of fluid and label the height and the radius accordingly.
  \item Write a expression that describes the radius of the infinitesimal cylinder in question.
  \item Write an expression for weight of the water at any particular height. 
  \item Suppose a tube of length $1$ sits at the top of the tank. Write the bounds of the work integral given this.
  \item With the previous information, write an integral expression for the work required to pump out water from the tank.
 \end{enumerate}
\end{Ej}

\begin{ptcb}
  \vspace*{7cm} 
\end{ptcb}

%\end{multicols}
\end{document} 