%----------------------------------------------------------------------------------------
%	PACKAGES AND OTHER DOCUMENT CONFIGURATIONS
%----------------------------------------------------------------------------------------

\documentclass[12pt]{article}
\usepackage[spanish]{babel} %Tildes
\usepackage[extreme]{savetrees} %Espaciado e interlineado. Comentar si no gusta el interlineado.
\usepackage[utf8]{inputenc} %Encoding para tildes
\usepackage[breakable,skins]{tcolorbox} %Cajitas
\usepackage{fancyhdr} % Se necesita para el título arriba
\usepackage{lastpage} % Se necesita para poner el número de página
\usepackage{amsmath,amsfonts,amssymb,amsthm} %simbolos y demás
\usepackage{mathabx} %más símbolos
\usepackage{physics} %simbolos de derivadas, bra-ket.
\usepackage{multicol}
\usepackage[customcolors]{hf-tikz}
\usepackage[shortlabels]{enumitem}
\usepackage{tikz}
\usetikzlibrary{patterns}
\usepackage{siunitx}

%\def\darktheme
%%%%%%%%% === Document Configuration === %%%%%%%%%%%%%%

\pagestyle{fancy}
\setlength{\headheight}{14.49998pt} %NO MODIFICAR
\setlength{\footskip}{14.49998pt} %NO MODIFICAR

\ifx \darktheme\undefined

\lhead{Math161S2} % Nombre de autor
\chead{\textbf{Quiz 5}} % Titulo
\rhead{Name:\hspace*{5cm}}%\firstxmark} 
\lfoot{}%\lastxmark}
\cfoot{}
\rfoot{Page \thepage\ of\ \pageref{LastPage}} %A la derecha saldrá pág. 6 de 9. 
\else
\pagenumbering{gobble}
\pagecolor[rgb]{0,0,0}%{0.23,0.258,0.321}
\color[rgb]{1,1,1}
\fi

%%%%%%%%% === My T Color Box === %%%%%%%%%%%%%%

\ifx \darktheme\undefined
\newtcolorbox{ptcb}{
colframe = black,
colback = white,
breakable,
enhanced
}
\newtcolorbox{ptcbP}{
colframe = black,
colback = white,
coltitle = black,
colbacktitle = black!40,
title = Practice,
breakable,
enhanced
}

\else
\newtcolorbox{ptcb}{
colframe = white,
colback = black,
colupper = white,
breakable,
enhanced
}
\newtcolorbox{ptcbP}{
colframe = white,
colback = black,
colupper = white,
coltitle = white,
colbacktitle = black,
title = Practice,
breakable,
enhanced
}
\fi

%%%%%%%%% === Tikz para matrices === %%%%%%%%%%%%%%

\tikzset{
  style green/.style={
    set fill color=green!50!lime!60,
    set border color=white,
  },
  style cyan/.style={
    set fill color=cyan!90!blue!60,
    set border color=white,
  },
  style orange/.style={
    set fill color=orange!80!red!60,
    set border color=white,
  },
  row/.style={
    above left offset={-0.15,0.31},
    below right offset={0.15,-0.125},
    #1
  },
  col/.style={
    above left offset={-0.1,0.3},
    below right offset={0.15,-0.15},
    #1
  }
}

%%%%%%%%% === Theorems and suchlike === %%%%%%%%%%%%%%

\theoremstyle{plain}
\newtheorem{Th}{Theorem}  %%% Theorem 1.1
\newtheorem*{nTh}{Theorem}             %%% No-numbered Theorem
\newtheorem{Prop}[Th]{Proposition}     %%% Proposition 1.2
\newtheorem{Lem}[Th]{Lemma}             %%% Lemma 1.3
\newtheorem*{nLem}{Lemma}               %%% No-numbered Lemma
\newtheorem{Cor}[Th]{Corollary}        %%% Corollary 1.4
\newtheorem*{nCor}{Corollary}          %%% No-numbered Corollary

\theoremstyle{definition}
\newtheorem*{Def}{Definition}       %%% Definition 1.5
\newtheorem*{nonum-Def}{Definition}    %%% No number Definition
\newtheorem*{nEx}{Example}             %%% No number Example
\newtheorem{Ex}[Th]{Example}           %%% Example
\newtheorem{Ej}[Th]{Exercise}         %%% Exercise
\newtheorem*{nEj}{Exercise}           %%% No number Excercise
\newtheorem*{Not}{Notation}       %%% Definition 1.5

\theoremstyle{remark}
\newtheorem*{Rmk}{Remark}      %%%Remark 1.6

%\numberwithin{equation}{section}

\setlength{\parindent}{3ex}

%%====== Useful macros: =======%%%

\DeclareMathOperator{\gen}{gen}     %%%set generated by...
\DeclareMathOperator{\Rng}{Rng}     %%%rangomat
\DeclareMathOperator{\Nul}{Nul}     %%%rangomat
\DeclareMathOperator{\Proy}{Proy}   %%%proyección
\DeclareMathOperator{\id}{id}       %%%identity operator

\newcommand{\al}{\alpha}            %%%short for \alpha
\newcommand{\la}{\lambda}           %%%short for \lambda
\newcommand{\sg}{\sigma}            %%%short for \sigma
\newcommand{\te}{\theta}                %% short for  \theta
\renewcommand{\l}{\ell}

\newcommand{\thickhat}[1]{\mathbf{\hat{\text{$#1$}}}}
\newcommand{\ii}{\vu{\imath}}
\newcommand{\jj}{\vu{\jmath}}
\newcommand{\kk}{\thickhat{k}}

\newcommand{\bC}{\mathbb{C}}        %%%complex numbers
\newcommand{\bN}{\mathbb{N}}        %%%natural numbers
\newcommand{\bP}{\mathbb{P}}        %%%polynomials
\newcommand{\bR}{\mathbb{R}}        %%%real numbers
\newcommand{\bZ}{\mathbb{Z}}        %%%integer numbers
\newcommand{\cB}{\mathcal{B}}       %%%basis
\newcommand{\cC}{\mathcal{C}}       %%%basis
\newcommand{\cM}{\mathcal{M}}       %%%matrix family

\newcommand{\sT}{\mathsf{T}}        %%%traspuesta

\renewcommand{\geq}{\geqslant}      %%%(to save typing)
\renewcommand{\leq}{\leqslant}      %%%(to save typing)
\newcommand{\x}{\times}             %%%product
\renewcommand{\:}{\colon}           %%%colon in  f: A -> B
\newcommand{\isom}{\simeq}              %% isomorfismo

\newcommand{\un}[1]{\underline{#1}}
\newcommand{\half}{\frac12}

\newcommand*{\Cdot}{{\raisebox{-0.25ex}{\scalebox{1.5}{$\cdot$}}}}      %% cdot más grande
\renewcommand{\.}{\Cdot}                %% producto escalar

\newcommand{\twobyone}[2]{\begin{pmatrix} %% 2 x 1 matrix
  #1 \\ #2 \end{pmatrix}}
  \newcommand{\twobytwo}[4]{\begin{pmatrix} %% 2 x 2 matrix
    #1 & #2 \\ #3 & #4 \end{pmatrix}}
    \newcommand{\twobythree}[6]{\begin{pmatrix} %% 2 x 3 matrix
        #1 & #2 & #3\\ #4 & #5 & #6 \end{pmatrix}}
\newcommand{\threebyone}[3]{\begin{pmatrix} %% 3 x 1 matrix
  #1 \\ #2 \\ #3 \end{pmatrix}}
  \newcommand{\threebytwo}[6]{\begin{pmatrix} %% 3 x 1 matrix
    #1 & #2\\ #3 & #4\\ #5&#6 \end{pmatrix}}
\newcommand{\threebythree}[9]{\begin{pmatrix} %% 3 x 3 matrix
  #1 & #2 & #3 \\ #4 & #5 & #6 \\ #7 & #8 & #9 \end{pmatrix}}

\newcommand{\To}{\Rightarrow}

\newcommand{\vaf}{\overrightarrow}

\newcommand{\set}[1]{\{\,#1\,\}}    %% set notation
\newcommand{\Set}[1]{\biggl\{\,#1\,\biggr\}} %% set notation (large)
\newcommand{\red}[1]{\textcolor{red}{#1}}
\newcommand{\blu}[1]{\textcolor{blue}{#1}}

%----------------------------------------------------------------------------------------
%	ARTICLE CONTENTS
%----------------------------------------------------------------------------------------

\begin{document}
%\begin{multicols}{2}

\begin{Ej}
  Consider a \emph{wing of a plane} with density $\rho(x)=\frac{1}{x+1}$. The wing of the plane is bounded by the curves
  $$y=0,\quad x=0,\quad x=3,\quad\text{and}\quad\text{a line }L\text{ connecting } (0,1)\ \text{and}\ (3,1/3).$$
  \begin{enumerate}[i)]
    \itemsep=-0.4em 
    \item Make a drawing which represents the wing in question.
    \item Use the point slope formula to determine an equation for the line $L$. 
    \item If the wing is flat, in which order should we setup the integral to find its mass? $\dd x$ or $\dd y$?
    \item Indicate in your diagram the bounds of integration. Write them as well as $a\leq x\leq b$ or $c\leq y\leq d$ depending on your choice for order of integration.
    \item Find the \texttt{GREATER} and \texttt{LOWER} curves. Label them.
    \item Use the previous information to find the mass of the wing in question.
  \end{enumerate}
\end{Ej}
\begin{ptcb}
  \begin{enumerate}[i)]
    \itemsep=-0.4em 
    \item See diagram.
    \item The points in question are $(0,1)$ and $(3,1/3)$ so the slope of the line is $m=\frac{1/3-1}{3-0}=\frac{-2}{9}$. From this we get 
    $$y=\frac{-2}{9}x+b\To b=0(-2/9)+1=1\To y=\frac{-2}{9}x+1.$$
    \item Since the density is in $x$ we should use a $\dd x$ integral.
    \item See diagram and $0\leq x \leq 3$.
    \item The greater one is $y=\frac{-2}{9}x+1$ and the lower one is $y=0$.
    \item The mass of the wing will be 
    $$\int_0^3\frac{1}{x+1}\left(\frac{-2}{9}x+1-0\right)\dd x.$$
  \end{enumerate}
\end{ptcb}

\begin{Ej}
  Consider the region in the $1^{\text{st}}$ quadrant enclosed by the curves
$$y=x-2,\quad\text{and}\quad x=3.$$
  Now suppose we rotate the region about the axis $x=2$. Do the following:
  \begin{enumerate}[i)]
    \itemsep=-0.4em
    \item Draw the region in question.
    \item Draw the solid of revolution obtained after rotation.
    \item Which 2 methods can we use to find the volume of this shape? Recall the possibilities are rings/shells on $x/y$.
    \item Given your method of choice, find the bounds of the region. Label them either as $a\leq x\leq b$ or $c\leq y\leq d$.
    \item Find the \texttt{GREATER} and \texttt{LOWER} curves. Label them.
    \item Use the previous information find the pair of parameters $R,r$ or $r,h$ given your choice of method. Label them.
    \item Construct the area function of your method. Label it.
    \item With the previous information, find the volume of the shape in question.
  \end{enumerate}
\end{Ej}

\begin{ptcb}
  \begin{enumerate}[i)]
    \itemsep=-0.4em 
    \item See diagram.
    \item See diagram.
    \item As the region is being rotated about $x=2$ we can use shells in $x$ or rings in $y$.
    \item The bounds are in $x$: $2\leq x\leq 3$ or in $y$: $0\leq y\leq 1$. 
    \item In $\dd x$ order, 
    $$\text{Greater: }y=x-2,\quad\text{and}\quad\text{Lower: }y=0.$$
    While in $\dd y$ order 
    $$\text{Greater: }x=3,\quad\text{and}\quad\text{Lower: }x=y+2.$$
    \item In terms of rings in $y$ we have 
    $$R=(3)-(2),\quad\text{and}\quad r=(3)-(y+2).$$
    While using shells in $x$ we have 
    $$h=(x-2)-(0),\quad\text{and}\quad r=(x)-(2).$$
    \item The area functions are 
    $$A(y)=\pi\left\lbrace(3-2)^2-(3-(y+2))^2\right\rbrace,\quad\text{and}\quad A(x)=2\pi(x-2)(x-2).$$
    \item The volume of the shape is 
    $$V=\int_0^1\pi\left\lbrack(3-2)^2-(3-(y+2))^2\right\rbrack\dd y=\int_{2}^32\pi(x-2)(x-2)\dd x.$$
  \end{enumerate}
\end{ptcb}

%\end{multicols}
\end{document} 