\documentclass[12pt]{memoir}

\def\nsemestre {I}
\def\nterm {Spring}
\def\nyear {2024}
\def\nprofesor {Amit Patel}
\def\nsigla {MATH571}
\def\nsiglahead {Topology 2}
\def\nextra {P}
\def\nlang {ENG}
\def\ntrim{}
\input{../../headerVarillyDiff}
\title{Sheaf Cohomology of Line Bundles over $\bP^1$: Intuition and Examples.}
\author{Ignacio Rojas}
\date{Spring, 2024}
\begin{document}
\bgroup
\renewcommand\thesection{\arabic{section}}
\renewcommand{\thefigure}{\arabic{figure}}
\maketitle
%\vspace*{-2.5em}
\begin{abstract}
    %\vspace*{-1.5em}
    mfwmfw
    \end{abstract}
    
    \begin{flushleft}
    \small
    \emph{Keywords}:
    my face when.
    
    \emph{MSC classes}:  Primary \texttt{1}; Secondary \texttt{2,3}.
    \end{flushleft}
    \section{Introduction}

    Intuitively a line bundle is a collection of lines with additional properties. Take a point $b$ in a base variety $B$ and place a copy of $\bC$ above it. Over each point of the variety we have a copy of the complex line, so that's where we get the line bundle. Specifically:

    \begin{Def}
        A \term{line bundle} over a base $B$ is a map $\pi\: L\to B$ with the following properties:
        \begin{itemize}
            \item There's an open cover $(U_i)_{i\in I}$ of $B$ such that 
            $$\pi^{-1}(U_i)\isom U_i\x\bC$$
            where we call $\phi_i\:\pi^{-1}(U)\to U\x\bC$ the isomorphism.
            \item For $b\in U_i\cap U_j$, the composition
            $$\set{b}\x\bC\xrightarrow[]{\phi_i^{-1}}\pi^{-1}(b)\xrightarrow[]{\phi_j}\set{b}\x\bC$$
            is a linear isomorphism. This map, $\phi_j\phi_i^{-1}$, is multiplication by a nonzero scalar $\la_b$.
        \end{itemize}
        A \term{fiber} over $b\in B$ of the line bundle is $\pi^{-1}(b)$.
    \end{Def}
    
    \begin{figure}[h!]
        \centering
        \includegraphics[width=0.7\textwidth, trim= 0.725cm 19.25cm 11.625cm 2.25cm,clip]{../figs/figLineBundleDefn.pdf}
        \caption{Line Bundle with the two properties}
        \label{fig:1-LineBdlExample}
    \end{figure} 

    \begin{Rmk}
        Despite the global definition of line bundles, they can also be defined locally through charts. This is achieved by specifying the open cover of the base variety and providing the transition functions between the charts.
    \end{Rmk}
    
    \begin{Rmk}
        The first property describes the \emph{local triviality} of a line bundle, indicating that around each point, the fibers are isomorphic to $\mathbb{C}$. This may not be true when considered globally. The second property ensures the compatibility of these local trivializations, maintaining a consistent linear structure. This allows different parts of the bundle to exhibit varying linear behaviors while maintaining a coherent structure.
    \end{Rmk}

    \begin{Ex}
    Over any base $B$ we have the \term{trivial line bundle} which is 
    $$\pi\: B\x\bC\to B,\quad (b,z)\mapsto b.$$
    The transition functions are all identity maps.
    \end{Ex} 

    %https://math.stackexchange.com/questions/742121/tangent-bundle-of-s1-is-diffeomorphic-to-the-cylinder-s1-times-bbbr
    %https://math.stackexchange.com/questions/933168/computing-transition-function-of-tangent-bundle-sn
    \begin{Ex}
        Consider the tangent bundle to the unit circle $S^1$. The points on the circle can be parametrized as 
        \[
        (\cos(t), \sin(t)), \quad t \in \lbonj{0, 2\pi}.
        \]
        The tangent vector at any point on the circle must be perpendicular to the radius at that point. This tangent vector can be represented as:
        \[
        \lambda (-\sin(t), \cos(t)), \quad \lambda \in \mathbb{R}.
        \]
        Consequently, the tangent bundle to the circle is isomorphic to $S^1 \times \mathbb{R}.$
        The transition functions can be defined through the Jacobian!
    \end{Ex}
    

    %https://math.stackexchange.com/questions/1398410/m%C3%B6bius-band-as-line-bundle-over-s1-starting-from-the-cocycles
    \begin{Ex}
        The M\"obius band can be seen as a line bundle over the circle. First identify $S^1$ as $\bonj{0,1}/0\sim 1$ and now consider the charts 
        $$U=S_1\less\set{0},\quad V=S^1\less\set{1/2}\word{with}\phi=\id,\quad\psi=1-\id.$$
        Even though all fibers are lines, this is a non trivial line bundle \emph{due to the transition functions}.
    \end{Ex}

    %https://math.stackexchange.com/questions/4777074/tangent-spaces-of-bbb-p1?noredirect=1&lq=1
    %https://math.stackexchange.com/questions/4389163/holomorphic-tangent-bundle-over-projective-line
    \begin{Ex}
        The tangent bundle of $\bP^1$ is a line bundle defined via two charts $(\bC^2,x,\del_x)$, $(\bC^2,y,\del_y)$. The chain rule gives us the relation
        $$\pdv{y}=\pdv{x}{y}\pdv{x}=\left(\pdv{y}\frac{1}{y}\right)\pdv{x}=\frac{-1}{y^2}\pdv{x}=-x^2\pdv{x}$$
        which lets us conclude that the transition function is 
        $$(x,\del_x)\mapsto(1/y,-y^2\del_y).$$
    \end{Ex}
    \subsection{Sections of Line Bundles}

    \begin{Def}
        A \term{section} of a line bundle $L$ over $B$ is a map $s\:B\to L$ with $\pi s=\id_B$. 
    \end{Def}
    
    Sections can be defined locally on open sets $U\subseteq B$ or globally when they are defined everywhere on $B$. Intuitively, sections single out points in fibers. For every $b\in B$, $s(b)$ is a point on the fiber $\pi^{-1}(b)$.

    \begin{Ex}
        In a line bundle, every fiber is a copy of $\bC$. And independent of the transition maps, zero is fixed. We can thus consider the \term{zero section} of the bundle as 
        $$s(b)=(b,0)\in\pi^{-1}(b).$$
    \end{Ex}

    The space of sections of a line bundle naturally has the structure of a vector space. This is why, sometimes, elements of certain varieties are referred to as sections of certain line bundles. 

    \section{Line Bundles over $\bP^1$}

    We begin by introducing a family of complex manifolds.

\begin{Def}
    For $d\in\bZ$, the manifold $\cO_{\bP^1}(d)$ (or $\cO(d)$) is defined by two charts and a transition function:
    $$(\bC^2,(x,u))\xrightarrow[v=u/x^d]{y=1/x}(\bC^2,(y,v)).$$
    This transition function is  $(y,v)=\left(\frac{1}{x},\frac{u}{x^d}\right)$ with inverse $\left(\frac{1}{y},\frac{v}{y^d}\right)$.
\end{Def}

We could also regard this set as $\bC^2/\sim$ where the equivalence relation is described via the transition function. $\cO(d)$ comes with a natural projection onto $\bP^1$: $(x,u)\mapsto x$ which allows to see $\cO(d)$ as a line bundle over $\bP^1$. 

\begin{Ex}
The tangent bundle over $\bP^1$ can be realized as $\cO(2)$. 
\end{Ex}

Moreover, \emph{any} line bundle over $\bP^1$ is of the form $\cO(d)$ for some $d$.  

\begin{Prop}
The space of sections of $\cO(d)$ is
\begin{itemize}
    \item The space of polynomials of degree $d$ in one affine coordinate, or homogenous polynomials of degree $d$ in both coordinates when $d\geq 0$.
    \item When $d<0$ the space of sections is trivial.
\end{itemize}
\end{Prop}

It's also possible to talk of \term{meromorphic sections} over $\cO(d)$ as rational functions of degree $d$. From this we can now see\dots

\section{$\cO(d)$ as a sheaf}

To define a sheaf we take an open set $U\subseteq\bP^1$ and consider the space of meromorphic sections, $\cO(d)(U)$, of $\cO(d)$ which are \emph{holomorphic} on $U$. This defines a sheaf over $\bP^1$. 

\begin{Ex}
Let's take $d=4$ and let's consider the meromorphic sections over $U_0=\set{Y\neq 0}$:
$$s(X:Y)=\frac{(X^2+XY+Y^2)^3}{(XY)^2},\word{and} \tilde{s}(X:Y)=\frac{X^7+X^4Y^3+Y^7}{Y^3}.$$
In this chart, we can take $Y=1$ via projective equivalence, and we get 
$$\frac{(x^2+x+1)^3}{x^2},\word{and}\frac{x^7+x^4+1}{1}.$$
The first section is \emph{not} holomorphic on $U_0$, but the second one is!
\end{Ex}

\begin{Prop}
On the charts of $\bP^1$, $U_0=\set{Y\neq 0}$, $U_1=\set{X\neq 0}$, we have 
$$\cO(d)(U_0)\isom\bC[x],\quad\cO(d)(U_1)\isom\bC[y],\quad\cO(d)(U_0\cap U_1)\isom\bC[x,1/x].$$
\end{Prop}

%https://en.wikipedia.org/wiki/Coherent_sheaf_cohomology
%https://math.stanford.edu/~vakil/216blog/FOAGaug2922public.pdf
\subsection{Sheaf cohomology}

Our calculation will de done using \v{C}ech cohomology, as in our case with \(\mathcal{O}(d)\), it's isomorphic to the corresponding sheaf cohomology.
We first consider the \v{C}ech complex associated to $\cO(d)$:
$$0\to O(d)(U)\oplus\cO(d)(V)\xrightarrow[]{\dd}O(d)(U\cap V)\to 0$$
The action of the coboundary map is given by 
$$\dd(s,\tilde{s})=s_{U\cap V}-\tilde{s}_{U\cap V}$$

%Write a bit about Sheaf to Cech cohom
%Make sure to understand coboundary map, write to Renzo
%Concretely find the cohomology 0 and 1
%If possible, find cellulation of P1 to do cellular sheaf cohomology

%%%%%%%%%%%% Contents end %%%%%%%%%%%%%%%%
\ifx\nextra\undefined
\printindex
\else\fi
\nocite{*}
\bibliographystyle{plain}
\bibliography{bibiProyTopp2.bib}
\end{document}