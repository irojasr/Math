\documentclass[12pt]{memoir}

\def\nsemestre {I}
\def\nterm {Spring}
\def\nyear {2024}
\def\nprofesor {Amit Patel}
\def\nsigla {MATH571}
\def\nsiglahead {Topology 2}
\def\nextra {P}
\def\nlang {ENG}
\def\ntrim{}
\input{../../headerVarillyDiff}
\title{Sheaf Cohomology of Line Bundles over $\bP^1$: Intuition and Examples.}
\author{Ignacio Rojas}
\date{Spring, 2024}
\begin{document}
\bgroup
\renewcommand\thesection{\arabic{section}}
\renewcommand{\thefigure}{\arabic{figure}}
\maketitle
%\vspace*{-2.5em}
\begin{abstract}
    %\vspace*{-1.5em}
    mfwmfw
    \end{abstract}
    
    \begin{flushleft}
    \small
    \emph{Keywords}:
    my face when.
    
    \emph{MSC classes}:  Primary \texttt{1}; Secondary \texttt{2,3}.
    \end{flushleft}
    \section{Introduction}

    Intuitively a line bundle is a collection of lines with additional properties. Take a point $b$ in a base variety $B$ and place a copy of $\bC$ above it. Over each point of the variety we have a copy of the complex line, so that's where we get the line bundle. Specifically:

    \begin{Def}
        A \term{line bundle} over a base $B$ is a map $\pi\: L\to B$ with the following properties:
        \begin{itemize}
            \item There's an open cover $(U_i)_{i\in I}$ of $B$ such that 
            $$\pi^{-1}(U_i)\isom U_i\x\bC$$
            where we call $\phi_i\:\pi^{-1}(U)\to U\x\bC$ the isomorphism.
            \item For $b\in U_i\cap U_j$, the composition
            $$\set{b}\x\bC\xrightarrow[]{\phi_i^{-1}}\pi^{-1}(b)\xrightarrow[]{\phi_j}\set{b}\x\bC$$
            is a linear isomorphism. This map, $\phi_j\phi_i^{-1}$, is multiplication by a nonzero scalar $\la_b$.
        \end{itemize}
        A \term{fiber} over $b\in B$ of the line bundle is $\pi^{-1}(b)$.
    \end{Def}
    
    \begin{figure}[h!]
        \centering
        \includegraphics[width=0.7\textwidth, trim= 0.725cm 19.25cm 11.625cm 2.25cm,clip]{../figs/figLineBundleDefn.pdf}
        \caption{Line Bundle with the two properties}
        \label{fig:1-LineBdlExample}
    \end{figure} 

    The first property states that the bundle is \emph{locally trivial}. Around a point, all the fibers upstairs basically look the same. This may not hold globally though! The second property states that those local trivializations' vector space structure is compatible. This can allow for one part of the bundle to look like a certain copy of $\bC$ while another one in a totally (but still linear) different manner and they'll be compatible.

    \begin{Ex}
    Over any base $B$ we have the \term{trivial line bundle} which is 
    $$\pi\: B\x\bC\to B,\quad (b,z)\mapsto b.$$
    The transition functions are all identity maps.
    \end{Ex} 

    %https://math.stackexchange.com/questions/742121/tangent-bundle-of-s1-is-diffeomorphic-to-the-cylinder-s1-times-bbbr
    %https://math.stackexchange.com/questions/933168/computing-transition-function-of-tangent-bundle-sn
    \begin{Ex}
        The tangent bundle of the unit circle $S^1$ is $S^1\x\bR$. Observe that points in the circle are of the form 
        $$(\cos(t),\sin(t)),\quad t\in\lbonj{0,2\pi},$$
        while the tangent space to this point is generated by $(-\sin(t),\cos(t))$. So, the tangent space is 
        $$\set{s(-\sin(t),\cos(t))}_{s\in\bR}$$
        from which we can see that the tangent bundle is indeed $S^1\x\bR$. For this bundle, the transition functions in this case are given by the Jacobian!
    \end{Ex}

    %https://math.stackexchange.com/questions/1398410/m%C3%B6bius-band-as-line-bundle-over-s1-starting-from-the-cocycles
    \begin{Ex}
        The M\"obius band can be seen as a line bundle over the circle via charts whose transition functions are $\id$ and $1-\id$. Even though all fibers are lines, this is a non trivial line bundle \emph{due to the transition functions}.
    \end{Ex}
    \subsection{Sections of Line Bundles}

    \begin{Def}
        A \term{section} of a line bundle $L$ over $B$ is a map $s\:B\to L$ with $\pi s=\id_B$. 
    \end{Def}
    
    Sections can be defined locally on open sets $U\subseteq B$ or globally when they are defined everywhere on $B$. Intuitively, sections single out points in fibers. For every $b\in B$, $s(b)$ is a point on the fiber $\pi^{-1}(b)$.

    \begin{Ex}
        In a line bundle, every fiber is a copy of $\bC$. And independent of the transition maps, zero is fixed. We can thus consider the \term{zero section} of the bundle as 
        $$s(b)=(b,0)\in\pi^{-1}(b).$$
    \end{Ex}

    The space of sections of a line bundle naturally has the structure of a vector space. This is why, sometimes, elements of certain varieties are referred to as sections of certain line bundles. 

    \section{Line Bundles over $\bP^1$}

    We begin by introducing a family of complex manifolds.

\begin{Def}
    For $d\in\bZ$, the manifold $\cO_{\bP^1}(d)$ is defined by two charts and a transition function:
    $$(\bC^2,(x,u))\xrightarrow[v=u/x^d]{y=1/x}(\bC^2,(y,v)).$$
    This transition function is  $(y,v)=\left(\frac{1}{x},\frac{u}{x^d}\right)$ with inverse $\left(\frac{1}{y},\frac{y}{v^d}\right)$.
\end{Def}

We could also regard this set as $\bC^2/\sim$ where the equivalence relation is described via the transition function. $\cO_{\bP^1}(d)$ comes with a natural projection onto $\bP^1$: $(x,u)\mapsto x$ which allows to see $\cO_{\bP^1}(d)$ as a line bundle over $\bP^1$. 

%%Pensar si eliminar ejemplos de S1, ver que secciones de O(d) son polinomios o nada explicar como O(d) es un haz y hacer el calculo de cohomologia.

    
%%%%%%%%%%%% Contents end %%%%%%%%%%%%%%%%
\ifx\nextra\undefined
\printindex
\else\fi
\nocite{*}
\bibliographystyle{plain}
\bibliography{bibiProyTopp2.bib}
\end{document}