\documentclass[12pt]{memoir}

\def\nsemestre {I}
\def\nterm {Spring}
\def\nyear {2025}
\def\nprofesor {Clayton Shonkwiler}
\def\nsigla {MATH670}
\def\nsiglahead {Differential Geometry}
\def\nextra {P}
\def\nlang {ENG}
\def\ntrim{}
\input{../../headerVarillyDiff}
\title{Euler Characteristics of Toric Varieties via Localization}
\author{Ignacio Rojas}
\date{Spring, 2025}
\begin{document}
\bgroup
\renewcommand\thesection{\arabic{section}}
\renewcommand{\thefigure}{\arabic{figure}}
\maketitle
%\vspace*{-2.5em}
\iffalse
\begin{abstract}
    %\vspace*{-1.5em}
    Our approach for computation involves calculating an integral of the Euler class of that manifold. But if that wasn't enough, the calculation will go through a localization process. This is formally known as Atiyah-Bott localization.\par
    To our effect, we will look at the action of a torus on our manifold and derive its cohomology by considering only the fixed loci of the action. However, this cohomology is \emph{equivariant cohomology}, which means that it remembers the structure of the torus action on our manifold. Thus, our manifold becomes an orbifold.\par
    Through this analysis, we will see that the Euler characteristic of toric varieties depends only on the number of torus fixed points it contains.
    \end{abstract}
    
    \begin{flushleft}
    \small
    \emph{Keywords}:
    Euler characteristic, Euler class, Betti number, toric variety, fixed loci, equivariant cohomology, Atiyah-Bott localization.
    
    %\emph{MSC classes}:  Primary \texttt{1}; Secondary \texttt{2,3}.
    \end{flushleft}
    \section{Premier}

    First of all, my interest in this project arises from the fact that I have to study localization and equivariant cohomology for my own research in the moduli space of stable maps. So developing a better intuition through examples in this project would benefit me.\par
    The main skeleton of this project goes like this:
    \begin{itemize}
        \item Explain what Euler characteristic is and how it can be computed as an integral of a manifold's Euler class.
        \item Introduce the definition of orbifold along with equivariant cohomology.
        \item Introduce the Atiyah-Bott isomorphism theorem. 
        \item Utilize the theorem to find the Euler characteristic of toric varieties such as $\bP^n$, $\bP^1\x\bP^1$ and $\Hilb^n(\bC^2)$.
    \end{itemize}
    I consider this project to be related to the class as it's another viewpoint on manifolds, and in this case, adding group actions to them which is a very natural thing to do. %\cite{BigMirrorSymmetryBook}
\fi
\begin{abstract}
    The Euler characteristic is an invariant of manifolds which can be computed as the alternating sum of its Betti numbers. In this project, we approach this calculation by integrating the manifold's Euler class. Atiyah-Bott localization will help us to refine the process.\par
    Our varieties come equipped with a torus action so we would like a cohomology which remembers this structure. This leads to equivariant cohomology, and in our cases, there will loci of our varieties which will remain fixed. Through this analysis, we will achieve our objective to demonstrate that the Euler characteristic of toric varieties depends solely on the number of torus-fixed points they contain.
    \end{abstract}
    \smallskip
    \begin{flushleft}
        \small
        \emph{Keywords}: Euler characteristic, Euler class, Betti numbers, toric variety, fixed loci, equivariant cohomology, Atiyah-Bott localization.
       \emph{MSC classes}:  Primary \texttt{57S12}; Secondary \texttt{14F43,55N91}.
    \end{flushleft}
    \section{Premier}
    
    This project arises from my interest in localization techniques and equivariant cohomology, particularly in relation to my research on the moduli space of stable maps. Developing a deeper intuition for these concepts through concrete examples will be valuable for my broader studies.
    
    The structure of this project is as follows:
    \begin{itemize}
        \item Define the Euler characteristic and realize it as the integral of the Euler class of a manifold.
        \item Introduce equivariant cohomology and the Atiyah-Bott localization theorem.
        \item Apply this theorem to compute the Euler characteristic of toric varieties, including $\mathbb{P}^n$, $\mathbb{P}^1 \times \mathbb{P}^1$, and $\text{Hilb}^n(\mathbb{C}^2)$.
    \end{itemize}
    
    This project aligns with the course by offering an alternative perspective on manifolds, by viewing group actions as another part of their study. Through this approach, we gain a new way to calculate invariants and insight into algebraic geometry.\par
    I intend to write a paper to develop this project. ¿Will we be doing presentations at the end of the course? 
    
    
%%%%%%%%%%%% Contents end %%%%%%%%%%%%%%%%
\ifx\nextra\undefined
\printindex
\else\fi
\nocite{*}
\bibliographystyle{plain}
\bibliography{bibiProyGeoDiff.bib}
\end{document}