\documentclass[12pt]{memoir}

\def\nsemestre {I}
\def\nterm {Spring}
\def\nyear {2024}
\def\nprofesor {Renzo Cavalieri}
\def\nsigla {DN}
\def\nsiglahead {Doctoral Notebook}
\def\nlang {ENG}
%\def\ntrim{}
%\def\darktheme{}
\let\footruleskip\relax %%FADIR
\input{../headerVarillyDiff}
\newcommand{\Mickey}{\pre{\raisebox{-1mm}{\scalebox{1.3}{\ensuremath\bullet}}}\hspace{-1.8 mm}{\raisebox{-1.65mm}{\scalebox{2.1}{\ensuremath\bullet}}}^{\hspace{-2.1mm}\raisebox{-1mm}{\scalebox{1.3}{\ensuremath\bullet}}}}
\newcommand{\Submickey}{\pre{\scalebox{0.75}{\ensuremath\bullet}}\hspace{-0.9mm}{\raisebox{-0.35mm}{\scalebox{1.1}{\ensuremath\bullet}}}^{\hspace{-1.1mm}\raisebox{-0.1mm}{\scalebox{0.75}{\ensuremath\bullet}}}}
\author{\nauthor}
\begin{document}
%\clearpage
%\thispagestyle{empty}
{\small 
\setlength{\parindent}{0em}
\setlength{\parskip}{1em}

This is my doctoral notebook where I will add clean information regarding whatever I'm learning about at the moment. It should serve as a starting point for writing. ¿Writing what? You may ask, I don't know.
}
\newpage
\tableofcontents
%\begin{multicols}{2}
\chapter{A Study of \emph{The Green Book} and the Moduli of Curves}

\section{Introduction and Prologue of \emph{The Green Book}}

The main objective of the green book is to prove the formula for the number $N_d$ of rational curves of degree $d$ passing through $3d-1$ points in general position in $\bP_\bC^1$. Let's begin by unwrapping some concepts:

\begin{Def}
A \term{projective curve} $\cC$ is the zero locus of points in $\bP^2_k$ which satisfy a homogeneous polynomial equation. Formally, for a homogeneous polynomial $f\in k[X,Y,Z]$, the projective curve determined by $f$ is
$$V(f)=\set{p\in\bP^2_k\:\ f(p)=0}.$$
If $f$ has degree $d$, then the curve $\cC$ is said to be a \term{curve of degree $d$}.
\end{Def}

\begin{Ex}
Consider the polynomial 
$$f(X,Y,Z)=X-Y-Z.$$
Inside the affine plane $\set{Z=1}$, this contains all the points of the form $(X:X-1:1)$. This is the line $y=x-1$ in $\bA^2$. But it also contains the point at infinity $(1:1:0)$. The degree 1 curve being described here is a projective line.
\end{Ex}

\begin{Ex}
    The degree 2 curve described by the equation $XY-Z^2=0$ is an affine hyperbola containing two points at infinity $(1:0:0)$ and $(0:1:0)$. 
\end{Ex}

\begin{Def}
A \term{parametrization} of a curve $\cC$ is a generically injective function $$\phi\:\bP^1_k\to\bP^2_k,\ (S:T)\mapsto(P(S:T),Q(S:T),R(S:T)),\quad P,Q,R\in k[S,T]_h.$$
A projective plane curve admitting a parametrization is called a \term{rational curve}.
\end{Def}

\begin{Ex}
    The line $X-Y-Z=0$ can be parametrized with $\phi(S:T)=(S,T,S-T)$. \red{¿Is the other curve rational?}
\end{Ex}

\begin{Ex}
    Degree $d$ curves with a $d-1$-tuple point are rational. As they can be parametrized by a line passing through the singular point. 
\end{Ex}

\subsection{The dimension of maps from $\bP^1$ to $\bP^2$ of degree $d$}

The number $3d-1$ sounds like an arbitrary number. It certainly did to me at least; this number corresponds to the dimension of the space of maps from $\bP^1$ to $\bP^2$ of degree $d$. There's this very important question, 
\begin{significant}
¿which vector space is the space of maps from $\bP^1$ to $\bP^2$  of degree $d$?
\end{significant}

\begin{Prop}\label{prop-dimension-maps-P1-to-P2}
The aforementioned space has dimension $3d-1$.
\end{Prop}

\begin{ptcbp}
A map $F:\bP^1\to\bP^2$ is defined via homogeneous, degree $d$ polynomials. This means that 
$$F(s:t)=(X:Y:Z)=(F_1(s:t),F_2(s:t),F_3(s:t)),$$
where each $F_i$ is a homogeneous degree $d$ polynomial. Explicitly we may write 
$$F_j(s:t)=\sum_{i=0}^da_is^{d-i}t^{i}=a_0s^d+a_1s^{d-1}t+\dots+a_{d-1}st^{d-1}+a_dt^d$$
which allows us to see that every $F_j$ has $d+1$ degrees of freedom. But we have to take of changes in the input and output spaces:
\begin{itemize}
    \item $3$ dimensions off for $\Aut(\bP^1)=\PGL_2$.
    \item $1$ dimension off for projective quotients: $(X:Y:Z)=\la(X:Y:Z)$.
\end{itemize}
This leaves us with $3d+3-3-1=3d-1$ dimensions. 
\end{ptcbp}

There's another way to prove this by counting the general number of degree $d$ curves and then making sure they are rational. For this we need the \term{genus-degree formula}.

\begin{Prop}\label{prop-genus-degree-formula}
    A projective curve of degree $d$ has genus $\binom{d-1}{2}$.
\end{Prop}

The proof of the genus-degree formula will be written down at a later point when we have to talk about Bézout's theorem. For now, the second proof of the dimension question:

\begin{ptcbp}
    Consider a general degree $d$ curve defined by a homogeneous polynomial $F$. Such a polynomial can be written as a combination of monomials $X^{a}Y^{b}Z^{c}$ where $a+b+c=d$. So to count the number of monomials, we must find the number of triples $(a,b,c)$ of non-negative integers whose sum is $d$. This is precisely 
    $$\multinom{3}{d}=\binom{3+d-1}{d}=\binom{d+2}{d}=\binom{d+2}{2},$$
    and we have to take off $1$ dimension due to projective quotients. Thus the dimension of the space of degree $d$ curves is precisely $N\:=\binom{d+2}{2}-1$.\par 
    Consider now the universal curve over $\bP^N$
    $$\cU=\set{F=\sum_{i+j+k=d}a_Ix^iy^jz^k=0}\subseteq\bP^N\x\bP^2.$$
    such that the curve $\set{F=0}$ is the fiber above the point $a_I\in\bP^N$. The family's dimension is precisely $N+1$, there's two ways to see this:
    \begin{itemize}
        \item We have all the dimensions of $\bP^N$ plus the dimension of the curve so that's $N+1$.
        \item The universal curve lives inside a $(N+2)$-dimensional space and it's a hypersurface, as it is defined via one equation. So we add one codimension to get $N+1$.
    \end{itemize}
    Inside $\cU$ we have the singular locus
    $$\text{Sing}=\cU\cap\set{\del_xF=0}\cap\set{\del_yF=0},$$
    where each of this conditions impose a dimension 1 restriction so this whole singular locus has dimension $N+1-2=N-1$. Projecting down to $\bP^N$ we get the set of points in $\bP^N$ whose fibers correspond to curves with \emph{at least} one singularity $\pi(\text{Sing})$. This set still has dimension $N-1$ inside of $\bP^N$ which means that adding a singularity adds one codimension.\par 
    Continuing this process we find smaller subsets corresponding to curves with more nodes and adding each node means adding one codimension. For the curve to be rational, we must add $g=\binom{d-1}{2}$ nodes in total, so that we add $g$ codimensions. The set of curves with $g$ singularities forms an open set inside of the rational curves which means that the dimension of the set of rational curves is
    $$\binom{d+2}{2}-1-\binom{d-1}{2}=(d+1)+d+(d-1)-1=3d-1.$$
\end{ptcbp}

\begin{Rmk}
    Recall $\multinom{n}{k}$ is the number of ways that I can distribute $k$ cookies amongst $n$ friends.
\end{Rmk}

\begin{Ej}
Learn why removing geometric genus reduces the dimension of the space of curves.
\end{Ej}

\begin{ptcb}
    \red{It has to do with resolution or normalization of singularities. Clarify and add the picture}
\end{ptcb}

The whole idea is to use the moduli space of maps from $\bP^1$ to $\bP^r$, $\ov M_{0,3d-1}(\bP^r,d)$, to show the formula. Isomorphism classes inside this set look like classes of bundles. And the formula is derived from intersection theory of this space. 

\subsection{Quadruplets of points}

I consider myself lucky to already know what $M_{0,4}$ is (it's the set of genus 0 Riemann surfaces with 4 distinct marked points). The notion of 
$$Q=\set{\text{quadruplets of distinct points in }\bP^1}$$
is introduced to alleviate the posterior definition of the moduli space. The set of quadruplets of points can actually be viewed as 
$$Q=(\bP^1)^4\less\Dl,$$
where $\Dl$ is the set of diagonals. This means that if we have $\vec x=(x_1,\dots,x_4)$, then $\Dl=\set{x_i=x_j}$ for some $i,j$. So indeed $Q$ is the set of distinct quadruplets. 

\begin{Rmk}
    When I find it convenient, points in $(\bP^1)^n$ will be denoted $\vec x$, but I'll mostly forget and I'll just call them $x$ without acknowledging that they are arrays.
\end{Rmk}

\begin{Ej}
Show that $Q$ is an affine algebraic variety. \hint{It's a similar argument to proving that $\bA^1\less\set{0}$ is an algebraic variety by considering $\bC[x,y]/\gen(xy-1)$.}
\end{Ej}

\begin{ptcb}
    The set $\Dl$ is a \emph{divisor}, we can see that it is 
    $$\Dl=V(x_1-x_2)\cup V(x_1-x_3)\cupycup V(x_3-x_4)=V\left(\prod_{\substack{i,j\in\bonj{4}\\ i<j}}(x_i-x_j)\right).$$
    Call this polynomial $f$, then $Q=\bA^4\less V(f)$ which can be seen as $V(tf-1)\subseteq \bA^{n+1}$.
    \end{ptcb}

\begin{Rmk}
    Recall affine algebraic varieties are those who are $\Spec$ of someone. In particular, $Q$ is the spectrum of the quotient of $\bC[x_1,\dots,x_4]$ by the ideal generated by the product.
\end{Rmk}

Ahh, you've dug the hole for yourself in this one\dots

\begin{Ej}
    Show that indeed $\bA^1\less\set{0}$ is an affine algebraic variety.
\end{Ej}

The set $Q$ is \emph{tautologically} a moduli space for quadruplets. In the easiest of terms, every element in $Q$ corresponds to a quadruplet of distinct points.\par
In the same way you go up to $M_{0,4}\isom\bP^1\less\set{0,1,\infty}$, look at a point $[\la:\mu]$ and find a $\bP^1$ with 4 marked points $(0,1,\infty,\la/\mu)$, you can go up to $Q$ and look at a point $\vec{p}=(p_1,\dots,p_4)$ and see that it \emph{tautologically} corresponds to the point $\vec p=(p_1,\dots,p_4)$. ¡The parameter is the quadruplet itself!\par 
It is claimed that $Q$ is a \emph{fine} moduli space \emph{because} it carries a universal family. In my mind, this notion of \emph{fineness}\footnote{\textbf{What is the difference between finesse and fineness?}
Finesse refers to the skill and cleverness someone shows in the way they deal with a situation or problem. Fineness refers to a thing's quality of being fine—for example, the fineness of print (that is, how small the letters are) or the fineness of one wire in comparison to another (that is, how thin they are).} is the same as representability of the moduli space as a functor. 

\begin{Ej}
    ¿Is the fineness the same as representability as a functor? Also, ¿does having a universal family guarantee that a moduli space as a functor be representable?
\end{Ej}

\subsubsection{The family business}

Intuitively, a family of pointed Mickies $\Mickey$ is a diagram:
\begin{figure}[h!]
    \centering
\begin{tikzcd}
    E \arrow[d, "\pi"']                     \\
    B \arrow[u, "\sigma_i"', bend right=49]
    \end{tikzcd}
    %\caption{}
    \label{fig-example-family-diagram}
\end{figure}
where $B$ is called the \term{base variety}, $E$ is more usually than not $\set{\Mickey}\x B$ in genus 0 (But in higher genus and some genus 0 cases like the Hirzebruch surface, it's not) and $\sg_i$ are sections which single out the important points in each Mickey. For each $b\in B$, the fiber over $b$, $\pi^{-1}(b)$ is isomorphic to a particular Mickey.\par 
For a family to be \emph{universal}, it is my understanding that the base variety should be the moduli space of Mickies itself. So the universal family should be 

\begin{figure}[h!]
    \centering
\begin{tikzcd}
    E \arrow[d, "\pi"']                     \\
    M_{\Submickey} \arrow[u, "s_i"', bend right=49]
    \end{tikzcd}
    %\caption{}
    \label{fig-example-universal-family-diagram}
\end{figure}

such that every fiber $\pi^{-1}(x)$ is the corresponding Mickey and $s_i(x)$ is the $i^{\text{th}}$ characteristic of the corresponding Mickey but seen in the upstairs Mickey.

\begin{Def}
    A \term{family of quadruplets} in $\bP^1$ over a base variety $B$ is a family of pointed $\bP^1$'s with 4 sections $\sg_i$ singling out the points in each $\bP^1$. Diagramatically: 
    \begin{center}
    \begin{tikzcd}
        B\x\bP^1 \arrow[d, "\pi"']                     \\
        B \arrow[u, "\sigma_i(\x 4)"', bend right=49]
        \end{tikzcd}
    \end{center}
    so a fiber over a point $b\in B$ is a copy of $\bP^1$ with four points marked via the map $\ssg=(\sg_1,\dots,\sg_4)$.
\end{Def}

From this, the universal family over $Q$ is the family of quadruplets over $Q$ as a base. The section $\sg_i$ is given by the $i^{\text{th}}$ projection mapping $\pi_i\: Q\to\bP^1$ which singles out the $i^{\text{th}}$ point of the quadruple.

\begin{Ej}
    The \emph{universal} family enjoys the \emph{universal} property that any other family of quadruples is induced from it via pullback. Explain how this happens and prove that the universal family indeed has this universal property.
\end{Ej}

\begin{ptcb}
    Let us begin by considering a family of quadruplets $\pi\: B\x\bP^1\to B$ along with its four sections $\sg_i$. We can build a map $\ssg$ which is the $\kp$ map we are looking for in this case from $B$ to $Q$ by considering all the sections:
    $$\ssg\:B\to Q,\ b\mapsto(\sg_1(b),\sg_2(b),\sg_3(b),\sg_4(b)).$$
    In order to create the pullback family, we look at the fiber of $\ssg(b)$ on the universal family of $Q$: $Q\x\bP^1$. To construct the pullback, we build it fiber by fiber.\par 
    For every $b\in B$, the fiber will be a copy of the fiber of $\ssg(b)$ but pasted on top of $B$ and the sections will be the pullback of the sections of $Q$ via $\ssg$:
    \begin{center}
\begin{tikzcd}
    B\x\bP^1 \arrow[rdd, "\pi"'] \arrow[rr, "\exists?\phi", dashed] &                                                                                               & B\x_Q(Q\x\bP^1) \arrow[ldd, "\pi"] \\
                                                                    &                                                                                               &                                    \\
                                                                    & B \arrow[luu, "\sg_i(\x 4)", bend left=49] \arrow[ruu, "\ssg^\ast(s_i)(\x4)"', bend right=49] &                                   
    \end{tikzcd}
    \end{center}
Finally, we are in need of the base morphism $\phi$. Observe that this $\phi$ we are looking for is the identity map on the fibers. It takes fibers to fibers, the points of the quadruple to the corresponding \emph{same} points but in the other fiber, and it's invertible. It follows that $\phi$ is an isomorphism of families over $B$ which means that the original family and the one induced via pullback are equivalent.\par
In terms of the diagram for fibered products what we have is the following:
\begin{center}
    % https://tikzcd.yichuanshen.de/#N4Igdg9gJgpgziAXAbVABwnAlgFyxMJZABgBpiBdUkANwEMAbAVxiRACEAdTgD24CMACgD0AjCAC+pdJlz5CKUaVFVajFmy68A+gEUAFLu59OQsQEpJ0kBmx4CRAEzLV9Zq0QgjvASPFSZO3knUkdXdQ8vK0C5B0VQ8PdNSVUYKABzeCJQADMAJwgAWyQlEBwIJAAWajcNT240LBBqBjp+GAZBWXsFEDysdIALHGiQfKKq6nKkAGYaiLZuAGs0ZpBW9s7u4M9+oZGAsYLixGcyisQ5tST6zkbR8ZPS6dPDx6Qyc5KWrDBIqDocEGaTWtUi3BgPCwcBwcAA-A1Bk03scPlMLtUQO0wFAkABaGafDYdLpBOJ9AbDUELW73FETRCfF5nbG4y7ECQUCRAA
\begin{tikzcd}
    B\x\bP^1 \arrow[rd, "\exists?\phi", dashed] \arrow[rdd, "\pi"', bend right] \arrow[rrd, bend left] &                                             &                           \\
                                                                                                       & B\x_Q(Q\x\bP^1) \arrow[d, "\pi"'] \arrow[r] & Q\x\bP^1 \arrow[d, "\pi"] \\
                                                                                                       & B \arrow[r, "\kp"']                         & Q                        
    \end{tikzcd}
\end{center}
$B\x\bP^1$ play the role of the new object which has morphisms to the already existing ones, and the pullback or fibered product is the universal object with this property.
\end{ptcb}

We have mentioned the idea of base morphisms, but formally\dots

\begin{Def}
    A \term{base morphism} between two families $E\xrightarrow[]{\pi}B$, $F\xrightarrow[]{\la}B$ is a map $\phi\:E\to F$ which makes the following diagram commute.
    \begin{center}
\begin{tikzcd}
E \arrow[rd, "\pi"'] \arrow[rr, "\phi"] &   & F \arrow[ld, "\la"] \\
                                        & B &                    
\end{tikzcd}
    \end{center}
\end{Def}

This does not add intuition to our understanding, but let us unravel the definition. The diagram commutes when 
$$\pi=\la\phi,$$
and we would like to see how fibers behave. So take $y$ in the fiber $\pi^{-1}(x)$,
$$y\in\pi^{-1}(x)\xrightarrow[]{\phi}\phi(y).$$
Now observe that if we map this element down to $B$ we get, via the commuting relationship
$$\la\phi(y)=\pi(y)=x.$$
This means that $\phi(y)\in\la^{-1}(x)$. And as our element was arbitrary, every fiber gets mapped to another fiber. In the case of base isomorphisms we have correspondence among the fibers, and so an automorphism of families is basically a rearrangement of fibers.
Now in the case of sections:
\begin{center}
\begin{tikzcd}
    E \arrow[rd, "\pi"] \arrow[rr, "\sg(x)\mapsto\phi(\sg(x))"] &                                                             & F \arrow[ld, "\la"'] \\
                                                                & B \arrow[lu, "\sg", bend left] \arrow[ru, "s"', bend right] &                     
    \end{tikzcd}
\end{center}
This diagram commutes when $s=\phi\sg$. So for a point $\sg(x)\in\pi^{-1}(x)$ 

\chapter{A Project on Sheaf Cohomology of Line Bundles over $\bP^1$}

\section{Is Bundle like a Family?}

The notion of a family of lines over a base variety $B$ is a map $\pi\: B\x\bC\to\bC$ where $\pi$ is the projection, so that over each point $b\in B$, the fiber of $\pi$ is a copy of the fixed $\bC$. 

\begin{Def}
    A \term{line bundle} over a base $B$ is a map $\pi\: L\to B$ with the following properties:
    \begin{enumerate}[i)]
        \item There's an open cover $(U_i)_{i\in I}$ of $B$ such that 
        $$\pi^{-1}(U_i)\isom U_i\x\bC$$
        where we call $\phi_i\:\pi^{-1}(U)\to U\x\bC$ the isomorphism. This means that the fiber is isomorphic to $\bC$.
        \item For $b\in U_i\cap U_j$, the composition
        $$\set{b}\x\bC\xrightarrow[]{\phi_i^{-1}}\pi^{-1}(b)\xrightarrow[]{\phi_j}\set{b}\x\bC$$
        is a linear isomorphism. This map, $\phi_j\phi_i^{-1}$, is multiplication by a nonzero scalar $\la_b$.
    \end{enumerate}
\end{Def}

Comparing this with the notion of family, we have \emph{local triviality} and the vector space structure between fibers is compatible. This means a line bundle is a locally trivial family of complex lines. 

\begin{figure}[h!]
    \centering
    \includegraphics[width=0.8\textwidth, trim= 0.725cm 19.25cm 11.625cm 2.25cm,clip]{figs/figLineBundleDefn.pdf}
    \caption{Line Bundle with the two properties}
    \label{fig:2.1-LineBdlExample}
\end{figure} 

\begin{Def}
    A \term{section} of a line bundle $L$ over $B$ is a map $s\:B\to L$ with $\pi s=\id_B$. 
\end{Def}

Sections can be defined locally on open sets $U\subseteq B$ or globally when they are defined everywhere on $B$. Intuitively, a section singles out points in fibers. For every $b\in B$, $s(b)$ is a point on the fiber $\pi^{-1}(b)$.
\subsection*{The Fun Part is not the sets, it's the\dots}

\textbf{Base morphisms} of a families are maps between total spaces which carry fibers onto fibers and distiguished points to distinguished points. For the case of line bundles we have just a tiny bit more.
    
\begin{Def}
    A \term{morphism of line bundles} is a map $f\: L_1\to L_2$ which makes the following diagram commute:
    \begin{center}
        \begin{tikzcd}
            L_1 \arrow[rdd, "\pi_1"'] \arrow[rr, "f"] && L_2 \arrow[ldd, "\pi_2"] \\&&\\& B&
            \end{tikzcd}
    \end{center}
    It must hold that on fibers, the map $f$ restricts to a \textbf{linear map}.
\end{Def}

Unwrapping the definition a bit, we have that the diagram commutes when $\pi_1=f\pi_2$. So the question is, where does a point in a fiber, $x\in\pi^{-1}_1(b)$, map to? We would like it to be in the corresponding fiber of $L_2$.\par 
For $f(x)\in\pi_2^{-1}(x)$, it must occur that $\pi_2(f(x))=b$. But we have 
$$\pi_1(x)=\pi_2(f(x))\word{and}\pi_1(x)=b,$$
so it holds that 
$$f\left(\pi^{-1}_1(b)\right)\subseteq \pi_2^{-1}(b).$$
Similarly for sections for sections, the diagram commutes when $s_2=fs_1$. But this means that for $b\in B$, 
$$s_2(b)=f(s_1(b)),$$
so distinguished points of fibers get sent to the corresponding distinguished points.

\section{Line Bundles over $\bP^1$}

We begin by introducing a family of complex manifolds.

\begin{Def}
    The manifold $\cO_{\bP^1}(d)$ is defined by two charts and a transition function:
    $$(\bC^2,(x,u))\xrightarrow[v=u/x^d]{y=1/x}(\bC^2,(y,v)).$$
    This transition function is  $(y,v)=\left(\frac{1}{x},\frac{u}{x^d}\right)$ with inverse $\left(\frac{1}{y},\frac{y}{v^d}\right)$.
\end{Def}
We could also regard this set as $\bC^2$ under the equivalence relation described via the transition function.

\subsubsection{No other line bundles}
$\cO_{\bP^1}(d)$ comes with a natural projection onto $\bP^1$ 
$$(x,u)\mapsto x$$
This allows us to see $\cO_{\bP^1}(d)$ as a line bundle, because every fiber $\pi^{-1}(x)$ is isomorphic to $\bC$. When $x$ is non-zero we get a copy of $\bC$ on both charts, but when $x=0$ or $\infty$, the line is only on one of the charts.\par 
Spoiling ourselves of the fun\footnote{Because it'd be so much fun to prove this are all the line bundles.}, we claim that all line bundles over $\bP^1$ are of the form $\cO_{\bP^1}(d)$ for an integer $d$. 
%%%%%%%%%%%% Contents end %%%%%%%%%%%%%%%%
\ifx\nextra\undefined
\printindex
\else\fi
\nocite{*}
\bibliographystyle{plain}
\bibliography{bibiDoctoralNotebook.bib}
%https://anddil.github.io/teaching/ %moduli of curves and maps
%https://www.math.colostate.edu/~renzo/teaching/Toric18/Linebundles.pdf %Renzo LBS
\end{document} 

