\documentclass[12pt]{memoir}

\def\nsemestre {I}
\def\nterm {Spring}
\def\nyear {2024}
\def\nprofesor {Renzo Cavalieri}
\def\nsigla {DN}
\def\nsiglahead {Doctoral Notebook}
\def\nlang {ENG}
%\def\ntrim{}
%\def\darktheme{}
\input{../headerVarillyDiff}
\author{\nauthor}
\begin{document}
%\clearpage
%\thispagestyle{empty}
{\small 
\setlength{\parindent}{0em}
\setlength{\parskip}{1em}

This is my doctoral notebook where I will add clean information regarding whatever I'm learning about at the moment. It should serve as a starting point for writing. ¿Writing what? You may ask, I don't know.
}
\newpage
\tableofcontents
%\begin{multicols}{2}
\chapter{A Study of \emph{The Green Book} and the Moduli of Curves}

\section{Introduction and Prologue of \emph{The Green Book}}

The main objective of the green book is to prove the formula for the number $N_d$ of rational curves of degree $d$ passing through $3d-1$ points in general position in $\bP_\bC^1$. Let's begin by unwrapping some concepts:

\begin{Def}
A \term{projective curve} $\cC$ is the zero locus of points in $\bP^2_k$ which satisfy a homogeneous polynomial equation. Formally, for a homogeneous polynomial $f\in k[X,Y,Z]$, the projective curve determined by $f$ is
$$V(f)=\set{p\in\bP^2_k\:\ f(p)=0}.$$
If $f$ has degree $d$, then the curve $\cC$ is said to be a \term{curve of degree $d$}.
\end{Def}

\begin{Ex}
Consider the polynomial 
$$f(X,Y,Z)=X-Y-Z.$$
Inside the affine plane $\set{Z=1}$, this contains all the points of the form $(X:X-1:1)$. This is the line $y=x-1$ in $\bA^2$. But it also contains the point at infinity $(1:1:0)$. The degree 1 curve being described here is a projective line.
\end{Ex}

\begin{Ex}
    The degree 2 curve described by the equation $XY-Z^2=0$ is an affine hyperbola containing two points at infinity $(1:0:0)$ and $(0:1:0)$. 
\end{Ex}

\begin{Def}
A \term{parametrization} of a curve $\cC$ is a generically injective function $$\phi\:\bP^1_k\to\bP^2_k,\ (S:T)\mapsto(P(S:T),Q(S:T),R(S:T)),\quad P,Q,R\in k[S,T]_h.$$
A projective plane curve admitting a parametrization is called a \term{rational curve}.
\end{Def}

\begin{Ex}
    The line $X-Y-Z=0$ can be parametrized with $\phi(S:T)=(S,T,S-T)$. \red{¿Is the other curve rational?}
\end{Ex}

\begin{Ex}
    Degree $d$ curves with a $d-1$-tuple point are rational. As they can be parametrized by a line passing through the singular point. 
\end{Ex}

\subsection{The dimension of maps from $\bP^1$ to $\bP^2$ of degree $d$}

The number $3d-1$ sounds like an arbitrary number. It certainly did to me at least; this number corresponds to the dimension of the space of maps from $\bP^1$ to $\bP^2$ of degree $d$. There's this very important question, 
\begin{significant}
¿which vector space is the space of maps from $\bP^1$ to $\bP^2$  of degree $d$?
\end{significant}

\begin{Prop}
The aforementioned space has dimension $3d-1$.
\end{Prop}

\begin{ptcbp}
A map $F:\bP^1\to\bP^2$ is defined via homogeneous, degree $d$ polynomials. This means that 
$$F(s:t)=(X:Y:Z)=(F_1(s:t),F_2(s:t),F_3(s:t)),$$
where each $F_i$ is a homogeneous degree $d$ polynomial. Explicitly we may write 
$$F_j(s:t)=\sum_{i=0}^da_is^{d-i}t^{i}=a_0s^d+a_1s^{d-1}t+\dots+a_{d-1}st^{d-1}+a_dt^d$$
which allows us to see that every $F_j$ has $d+1$ degrees of freedom. But we have to take of changes in the input and output spaces:
\begin{itemize}
    \item $3$ dimensions off for $\Aut(\bP^1)=\PGL_2$.
    \item $1$ dimension off for projective quotients: $(X:Y:Z)=\la(X:Y:Z)$.
\end{itemize}
This leaves us with $3d+3-3-1=3d-1$ dimensions. 
\end{ptcbp}

There's another way to prove this by counting the general number of degree $d$ curves and then making sure they are rational. For this we need the \term{genus-degree formula}.

\begin{Prop}\label{prop-genus-degree-formula}
    A projective curve of degree $d$ has genus $\binom{d-1}{2}$.
\end{Prop}

The proof of the genus-degree formula will be written down at a later point when we have to talk about Bézout's theorem. For now, the second proof of the dimension question:

\begin{ptcbp}
    Consider a general degree $d$ curve defined by a homogeneous polynomial $F$. Such a polynomial can be written as a combination of monomials $X^{a}Y^{b}Z^{c}$ where $a+b+c=d$. So to count the number of monomials, we must find the number of triples $(a,b,c)$ of non-negative integers whose sum is $d$. This is precisely 
    $$\multinom{3}{d}=\binom{3+d-1}{d}=\binom{d+2}{d}=\binom{d+2}{2},$$
    and we have to take off $1$ dimension due to projective quotients.\par 
    \red{Hold on, how did we reduce dimension by removing arithmethic genus?} But in essence what happens is that 
    $$\binom{d+2}{2}-1-\binom{d-1}{2}=(d+1)+d+(d-1)-1=3d-1.$$
\end{ptcbp}

\begin{Rmk}
    Recall $\multinom{n}{k}$ is the number of ways that I can distribute $k$ cookies amongst $n$ friends.
\end{Rmk}

The whole idea is to use the moduli space of maps from $\bP^1$ to $\bP^r$, $\ov\cM_{0,3d-1}(\bP^r,d)$, to show the formula. Isomorphism classes inside this set look like classes of bundles. And the formula is derived from intersection theory of this space. 
%%%%%%%%%%%% Contents end %%%%%%%%%%%%%%%%
\ifx\nextra\undefined
\printindex
\else\fi
\nocite{*}
\bibliographystyle{plain}
\bibliography{bibiDoctoralNotebook.bib}
\end{document} 

