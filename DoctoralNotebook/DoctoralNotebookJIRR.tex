\documentclass[12pt]{memoir}

\def\nsemestre {I}
\def\nterm {Spring}
\def\nyear {2024}
\def\nprofesor {Renzo Cavalieri}
\def\nsigla {DN}
\def\nsiglahead {Doctoral Notebook}
\def\nlang {ENG}
%\def\ntrim{}
%\def\darktheme{}
%\let\footruleskip\relax %%FADIR
\input{../headerVarillyDiff}
\newcommand{\Mickey}{\pre{\raisebox{-1mm}{\scalebox{1.3}{\ensuremath\bullet}}}\hspace{-1.8 mm}{\raisebox{-1.65mm}{\scalebox{2.1}{\ensuremath\bullet}}}^{\hspace{-2.1mm}\raisebox{-1mm}{\scalebox{1.3}{\ensuremath\bullet}}}}
\newcommand{\Submickey}{\pre{\scalebox{0.75}{\ensuremath\bullet}}\hspace{-0.9mm}{\raisebox{-0.35mm}{\scalebox{1.1}{\ensuremath\bullet}}}^{\hspace{-1.1mm}\raisebox{-0.1mm}{\scalebox{0.75}{\ensuremath\bullet}}}}
\author{\nauthor}
\begin{document}
%\clearpage
%\thispagestyle{empty}
{\small 
\setlength{\parindent}{0em}
\setlength{\parskip}{1em}

This is my doctoral notebook where I will add clean information regarding whatever I'm learning about at the moment. It should serve as a starting point for writing. ¿Writing what? You may ask, I don't know.

\red{TO DO 1}
\begin{itemize}
    \item Finish reading stratification of $M_{0,n}$
    \item Write ex 6.2
    \item Re-read Zvonkine paper now with new knowledge
    \item Summarize Borcherds Chow Ring and Chern Classes DONE 20240921
\end{itemize}
\red{TO DO 2}
\begin{itemize}
    \item Read Theorem 3.1 in MattAndRenzo plus examples afterwards. Understand and write summary.
    \item Add new bibliographies, Chow's paper, Borcherds lectures.
\end{itemize}
}
\newpage
\tableofcontents
%\begin{multicols}{2}
\chapter{A Study of the Introduction and Prologue of \emph{The Green Book}}

\section{Really quickly: The Introduction}

The main objective of the green book is to prove the formula for the number $N_d$ of rational curves of degree $d$ passing through $3d-1$ points in general position in $\bP_\bC^1$. Let's begin by unwrapping some concepts:

\begin{Def}
A \term{projective curve} $\cC$ is the zero locus of points in $\bP^2_k$ which satisfy a homogeneous polynomial equation. Formally, for a homogeneous polynomial $f\in k[X,Y,Z]$, the projective curve determined by $f$ is
$$V(f)=\set{p\in\bP^2_k\:\ f(p)=0}.$$
If $f$ has degree $d$, then the curve $\cC$ is said to be a \term{curve of degree $d$}.
\end{Def}

\begin{Ex}
Consider the polynomial 
$$f(X,Y,Z)=X-Y-Z.$$
Inside the affine plane $\set{Z=1}$, this contains all the points of the form $(X:X-1:1)$. This is the line $y=x-1$ in $\bA^2$. But it also contains the point at infinity $(1:1:0)$. The degree 1 curve being described here is a projective line.
\end{Ex}

\begin{Ex}
    The degree 2 curve described by the equation $XY-Z^2=0$ is an affine hyperbola containing two points at infinity $(1:0:0)$ and $(0:1:0)$. 
\end{Ex}

\begin{Def}
A \term{parametrization} of a curve $\cC$ is a generically injective function $$\phi\:\bP^1_k\to\bP^2_k,\ (S:T)\mapsto(P(S:T),Q(S:T),R(S:T)),\quad P,Q,R\in k[S,T]_h.$$
A projective plane curve admitting a parametrization is called a \term{rational curve}.
\end{Def}

\begin{Ex}
    The line $X-Y-Z=0$ can be parametrized with $\phi(S:T)=(S,T,S-T)$. \red{¿Is the other curve rational?}
\end{Ex}

\begin{Ex}
    Degree $d$ curves with a $d-1$-tuple point are rational. As they can be parametrized by a line passing through the singular point. 
\end{Ex}

\subsection{The dimension of maps from $\bP^1$ to $\bP^2$ of degree $d$}

The number $3d-1$ sounds like an arbitrary number. It certainly did to me at least; this number corresponds to the dimension of the space of maps from $\bP^1$ to $\bP^2$ of degree $d$. There's this very important question, 
\begin{significant}
¿which vector space is the space of maps from $\bP^1$ to $\bP^2$  of degree $d$?
\end{significant}

\begin{Prop}\label{prop-dimension-maps-P1-to-P2}
The aforementioned space has dimension $3d-1$.
\end{Prop}

\begin{ptcbp}
A map $F:\bP^1\to\bP^2$ is defined via homogeneous, degree $d$ polynomials. This means that 
$$F(s:t)=(X:Y:Z)=(F_1(s:t),F_2(s:t),F_3(s:t)),$$
where each $F_i$ is a homogeneous degree $d$ polynomial. Explicitly we may write 
$$F_j(s:t)=\sum_{i=0}^da_is^{d-i}t^{i}=a_0s^d+a_1s^{d-1}t+\dots+a_{d-1}st^{d-1}+a_dt^d$$
which allows us to see that every $F_j$ has $d+1$ degrees of freedom. But we have to take of changes in the input and output spaces:
\begin{itemize}
    \item $3$ dimensions off for $\Aut(\bP^1)=\PGL_2$.
    \item $1$ dimension off for projective quotients: $(X:Y:Z)=\la(X:Y:Z)$.
\end{itemize}
This leaves us with $3d+3-3-1=3d-1$ dimensions. 
\end{ptcbp}

There's another way to prove this by counting the general number of degree $d$ curves and then making sure they are rational. For this we need the \term{genus-degree formula}.

\begin{Prop}\label{prop-genus-degree-formula}
    A projective curve of degree $d$ has genus $\binom{d-1}{2}$.
\end{Prop}

The proof of the genus-degree formula will be written down at a later point when we have to talk about Bézout's theorem. For now, the second proof of the dimension question:

\begin{ptcbp}
    Consider a general degree $d$ curve defined by a homogeneous polynomial $F$. Such a polynomial can be written as a combination of monomials $X^{a}Y^{b}Z^{c}$ where $a+b+c=d$. So to count the number of monomials, we must find the number of triples $(a,b,c)$ of non-negative integers whose sum is $d$. This is precisely 
    $$\multinom{3}{d}=\binom{3+d-1}{d}=\binom{d+2}{d}=\binom{d+2}{2},$$
    and we have to take off $1$ dimension due to projective quotients. Thus the dimension of the space of degree $d$ curves is precisely $N\:=\binom{d+2}{2}-1$.\par 
    Consider now the universal curve over $\bP^N$
    $$\cU=\set{F=\sum_{i+j+k=d}a_Ix^iy^jz^k=0}\subseteq\bP^N\x\bP^2.$$
    such that the curve $\set{F=0}$ is the fiber above the point $a_I\in\bP^N$. The family's dimension is precisely $N+1$, there's two ways to see this:
    \begin{itemize}
        \item We have all the dimensions of $\bP^N$ plus the dimension of the curve so that's $N+1$.
        \item The universal curve lives inside a $(N+2)$-dimensional space and it's a hypersurface, as it is defined via one equation. So we add one codimension to get $N+1$.
    \end{itemize}
    Inside $\cU$ we have the singular locus
    $$\text{Sing}=\cU\cap\set{\del_xF=0}\cap\set{\del_yF=0},$$
    where each of this conditions impose a dimension 1 restriction so this whole singular locus has dimension $N+1-2=N-1$. Projecting down to $\bP^N$ we get the set of points in $\bP^N$ whose fibers correspond to curves with \emph{at least} one singularity $\pi(\text{Sing})$. This set still has dimension $N-1$ inside of $\bP^N$ which means that adding a singularity adds one codimension.\par 
    Continuing this process we find smaller subsets corresponding to curves with more nodes and adding each node means adding one codimension. For the curve to be rational, we must add $g=\binom{d-1}{2}$ nodes in total, so that we add $g$ codimensions. The set of curves with $g$ singularities forms an open set inside of the rational curves which means that the dimension of the set of rational curves is
    $$\binom{d+2}{2}-1-\binom{d-1}{2}=(d+1)+d+(d-1)-1=3d-1.$$
\end{ptcbp}

\begin{Rmk}
    Recall $\multinom{n}{k}$ is the number of ways that I can distribute $k$ cookies amongst $n$ friends.
\end{Rmk}

\begin{Ej}
Learn why removing geometric genus reduces the dimension of the space of curves.
\end{Ej}

\begin{ptcb}
    \red{It has to do with resolution or normalization of singularities. Clarify and add the picture}
\end{ptcb}

The whole idea is to use the moduli space of maps from $\bP^1$ to $\bP^r$, $\ov M_{0,3d-1}(\bP^r,d)$, to show the formula. Isomorphism classes inside this set look like classes of bundles. And the formula is derived from intersection theory of this space. 

\section{Quadruplets of Points}

I consider myself lucky to already know what $M_{0,4}$ is (it's the set of genus 0 Riemann surfaces with 4 distinct marked points). The notion of 
$$Q=\set{\text{quadruplets of distinct points in }\bP^1}$$
is introduced to alleviate the posterior definition of the moduli space. The set of quadruplets of points can actually be viewed as 
$$Q=(\bP^1)^4\less\Dl,$$
where $\Dl$ is the set of diagonals. This means that if we have $\vec x=(x_1,\dots,x_4)$, then $\Dl=\set{x_i=x_j}$ for some $i,j$. So indeed $Q$ is the set of distinct quadruplets. 

\begin{Rmk}
    When I find it convenient, points in $(\bP^1)^n$ will be denoted $\vec x$, but I'll mostly forget and I'll just call them $x$ without acknowledging that they are arrays.
\end{Rmk}

\begin{Ej}
Show that $Q$ is an affine algebraic variety. [Hint: It's a similar argument to proving that $\bA^1\less\set{0}$ is an algebraic variety by considering $\bC[x,y]/\gen(xy-1)$.]
\end{Ej}

\begin{ptcb}
    The set $\Dl$ is a \emph{divisor}, we can see that it is 
    $$\Dl=V(x_1-x_2)\cup V(x_1-x_3)\cupycup V(x_3-x_4)=V\left(\prod_{\substack{i,j\in\bonj{4}\\ i<j}}(x_i-x_j)\right).$$
    Call this polynomial $f$, then $Q=\bA^4\less V(f)$ which can be seen as $V(tf-1)\subseteq \bA^{n+1}$.
    \end{ptcb}

\begin{Rmk}
    Recall affine algebraic varieties are those who are $\Spec$ of someone. In particular, $Q$ is the spectrum of the quotient of $\bC[x_1,\dots,x_4]$ by the ideal generated by the product.
\end{Rmk}

Ahh, you've dug the hole for yourself in this one\dots

\begin{Ej}
    Show that indeed $\bA^1\less\set{0}$ is an affine algebraic variety.
\end{Ej}

The set $Q$ is \emph{tautologically} a moduli space for quadruplets. In the easiest of terms, every element in $Q$ corresponds to a quadruplet of distinct points.\par
In the same way you go up to $M_{0,4}\isom\bP^1\less\set{0,1,\infty}$, look at a point $[\la:\mu]$ and find a $\bP^1$ with 4 marked points $(0,1,\infty,\la/\mu)$, you can go up to $Q$ and look at a point $\vec{p}=(p_1,\dots,p_4)$ and see that it \emph{tautologically} corresponds to the point $\vec p=(p_1,\dots,p_4)$. ¡The parameter is the quadruplet itself!\par 
It is claimed that $Q$ is a \emph{fine} moduli space \emph{because} it carries a universal family. In my mind, this notion of \emph{fineness}\footnote{\textbf{What is the difference between finesse and fineness?}
Finesse refers to the skill and cleverness someone shows in the way they deal with a situation or problem. Fineness refers to a thing's quality of being fine—for example, the fineness of print (that is, how small the letters are) or the fineness of one wire in comparison to another (that is, how thin they are).} is the same as representability of the moduli space as a functor. 

\begin{Ej}
    ¿Is the fineness the same as representability as a functor? Also, ¿does having a universal family guarantee that a moduli space as a functor be representable?
\end{Ej}

\subsection{The Family Business}

Intuitively, a family of pointed Mickies $\Mickey$ is a diagram:
\begin{figure}[h!]
    \centering
\begin{tikzcd}
    E \arrow[d, "\pi"']                     \\
    B \arrow[u, "\sigma_i"', bend right=49]
    \end{tikzcd}
    %\caption{}
    \label{fig-example-family-diagram}
\end{figure}
where $B$ is called the \term{base variety}, $E$ is more usually than not $\set{\Mickey}\x B$ in genus 0 (But in higher genus and some genus 0 cases like the Hirzebruch surface, it's not) and $\sg_i$ are sections which single out the important points in each Mickey. For each $b\in B$, the fiber over $b$, $\pi^{-1}(b)$ is isomorphic to a particular Mickey.\par 
For a family to be \emph{universal}, it is my understanding that the base variety should be the moduli space of Mickies itself. So the universal family should be 

\begin{figure}[h!]
    \centering
\begin{tikzcd}
    U \arrow[d, "\pi"']                     \\
    M_{\Submickey} \arrow[u, "s_i"', bend right=49]
    \end{tikzcd}
    %\caption{}
    \label{fig-example-universal-family-diagram}
\end{figure}

such that every fiber $\pi^{-1}(x)$ is the corresponding Mickey and $s_i(x)$ is the $i^{\text{th}}$ characteristic of the corresponding Mickey but seen in the upstairs Mickey.

\begin{Rmk}\label{rmk-intuition-universal-family}
    The idea behind universality is that this family $U\to M_{\Submickey}$ contains \emph{every possible Mickey}. All the possible Mickeys are coded as equivalence classes into points of $M_{Submickey}$, so when looking at fibers upstairs in $U$ there's not a Mickey we're missing.
\end{Rmk}

\begin{Def}
    A \term{family of quadruplets} in $\bP^1$ over a base variety $B$ is a family of pointed $\bP^1$'s with 4 sections $\sg_i$ singling out the points in each $\bP^1$. Diagramatically: 
    \begin{center}
    \begin{tikzcd}
        B\x\bP^1 \arrow[d, "\pi"']                     \\
        B \arrow[u, "\sigma_i(\x 4)"', bend right=49]
        \end{tikzcd}
    \end{center}
    so a fiber over a point $b\in B$ is a copy of $\bP^1$ with four points marked via the map $\ssg=(\sg_1,\dots,\sg_4)$.
\end{Def}

From this, the universal family over $Q$ is the family of quadruplets over $Q$ as a base. The section $\sg_i$ is given by the $i^{\text{th}}$ projection mapping $\pi_i\: Q\to\bP^1$ which singles out the $i^{\text{th}}$ point of the quadruple.

\begin{Ej}\label{ejer-universality-of-quadruplets-no-projective-equivalence}
    The \emph{universal} family enjoys the \emph{universal} property that any other family of quadruples is induced from it via pullback. Explain how this happens and prove that the universal family indeed has this universal property.
\end{Ej}

\begin{ptcb}
    Let us begin by considering a family of quadruplets $\pi\: B\x\bP^1\to B$ along with its four sections $\sg_i$. We can build a map $\ssg$ which is the $\kp\: B\to Q$ map we are looking for in this case from $B$ to $Q$ by considering all the sections:
    $$\ssg\:B\to Q,\ b\mapsto(\sg_1(b),\sg_2(b),\sg_3(b),\sg_4(b)).$$
    In order to create the pullback family, we look at the fiber of $\ssg(b)$ on the universal family of $Q$: $Q\x\bP^1$. To construct the pullback, we build it fiber by fiber.\par 
    For every $b\in B$, the fiber will be a copy of the fiber of $\ssg(b)$ but pasted on top of $B$ and the sections will be the pullback of the sections of $Q$ via $\ssg$:
    \begin{center}
\begin{tikzcd}
    B\x\bP^1 \arrow[rdd, "\pi"'] \arrow[rr, "\exists?\phi", dashed] &                                                                                               & B\x_Q(Q\x\bP^1) \arrow[ldd, "\pi"] \\
                                                                    && \\
& B \arrow[luu, "\sg_i(\x 4)", bend left=49] \arrow[ruu, "\ssg^\ast(s_i)(\x4)"', bend right=49] &                                   
    \end{tikzcd}
    \end{center}
Finally, we are in need of the base morphism $\phi$. Observe that this $\phi$ we are looking for is the identity map on the fibers. It takes fibers to fibers, the points of the quadruple to the corresponding \emph{same} points but in the other fiber, and it's invertible. It follows that $\phi$ is an isomorphism of families over $B$ which means that the original family and the one induced via pullback are equivalent.\par
In terms of the diagram for fibered products what we have is the following:
\begin{center}
    % https://tikzcd.yichuanshen.de/#N4Igdg9gJgpgziAXAbVABwnAlgFyxMJZABgBpiBdUkANwEMAbAVxiRACEAdTgD24CMACgD0AjCAC+pdJlz5CKUaVFVajFmy68A+gEUAFLu59OQsQEpJ0kBmx4CRAEzLV9Zq0QgjvASPFSZO3knUkdXdQ8vK0C5B0VQ8PdNSVUYKABzeCJQADMAJwgAWyQlEBwIJAAWajcNT240LBBqBjp+GAZBWXsFEDysdIALHGiQfKKq6nKkAGYaiLZuAGs0ZpBW9s7u4M9+oZGAsYLixGcyisQ5tST6zkbR8ZPS6dPDx6Qyc5KWrDBIqDocEGaTWtUi3BgPCwcBwcAA-A1Bk03scPlMLtUQO0wFAkABaGafDYdLpBOJ9AbDUELW73FETRCfF5nbG4y7ECQUCRAA
\begin{tikzcd}
    B\x\bP^1 \arrow[rd, "\exists?\phi", dashed] \arrow[rdd, "\pi"', bend right] \arrow[rrd, bend left] &                                             &                           \\
                                                                                                       & B\x_Q(Q\x\bP^1) \arrow[d, "\pi"'] \arrow[r] & Q\x\bP^1 \arrow[d, "\pi"] \\
                                                                                                       & B \arrow[r, "\kp"']                         & Q                        
    \end{tikzcd}
\end{center}
$B\x\bP^1$ play the role of the new object which has morphisms to the already existing ones, and the pullback or fibered product is the universal object with this property.
\end{ptcb}

\begin{Rmk}
    Now, returning to our previous observation \ref{rmk-intuition-universal-family}, the universal family enjoys the universal property \emph{because it contains all of the possible fibers}. The $\kp$ map literally looks at a point $b$ in the base, asks which is the fiber above it and then points to that fiber's equivalence class in the moduli space. ¡Such a map's existence is guaranteed because the moduli space has (the equivalence class of) all the fibers!
\end{Rmk}

We have mentioned the idea of base morphisms, but formally\dots

\begin{Def}
    A \term{base morphism} between two families $E\xrightarrow[]{\pi}B$, $F\xrightarrow[]{\la}B$ is a map $\phi\:E\to F$ which makes the following diagram commute.
    \begin{center}
\begin{tikzcd}
E \arrow[rd, "\pi"'] \arrow[rr, "\phi"] &   & F \arrow[ld, "\la"] \\
                                        & B &                    
\end{tikzcd}
    \end{center}
\end{Def}

This does not add intuition to our understanding, but let us unravel the definition. The diagram commutes when 
$$\pi=\la\phi,$$
and we would like to see how fibers behave. So take $y$ in the fiber $\pi^{-1}(x)$,
$$y\in\pi^{-1}(x)\xrightarrow[]{\phi}\phi(y).$$
Now observe that if we map this element down to $B$ we get, via the commuting relationship
$$\la\phi(y)=\pi(y)=x.$$
This means that $\phi(y)\in\la^{-1}(x)$. And as our element was arbitrary, every fiber gets mapped to another fiber. In the case of base isomorphisms we have correspondence among the fibers, and so an automorphism of families is basically a rearrangement of fibers.
Now in the case of sections:
\begin{center}
\begin{tikzcd}
    E \arrow[rd, "\pi"] \arrow[rr, "\sg(x)\mapsto\phi(\sg(x))"] &                                                             & F \arrow[ld, "\la"'] \\
                                                                & B \arrow[lu, "\sg", bend left] \arrow[ru, "s"', bend right] &                     
    \end{tikzcd}
\end{center}
This diagram commutes when $s=\phi\sg$. So, ¿does a point $\sg(x)\in\pi^{-1}(x)$ get sent to another \emph{special point} on a fiber of $x$, or does it go to another fiber? Observe that if we map $\sg(x)$ through $\phi$ we get 
$$\phi(\sg(x))=s(x)$$
but also $\la(s(x))=x$, so it happens that $s(x)\in\la^{-1}(x)$ and not in another fiber.

Intuitively, isomorphisms of families are morphisms of families which also happen to be bijective. This helps us now because the next step is looking at\dots

\section{Quadruplets along with $\bP^1$}

The Green Book goes along with the notion of quadruplets but now up to equivalence. Two quadruplets are equivalent when they differ by a $\bP^1$ automorphism term-by-term. Families of quadruplets will be projectively equivalent when there is a base-isomorphism between them.

\begin{Ex}
    The following quadruplets are equivalent:
    \begin{itemize}
        \item $(0,1,\infty,8)$ and $(0,2,\infty,16)$ via the M\"obius transformation $z\mapsto 2z$. 
        \item $(-7/8,0,\infty,1/8)$ and $(0,7/8,\infty,1)$ via $z\mapsto z+7/8$.
        \item $(\infty,1,0,1/9)$ and $(0,3,\infty,27)$ via $z\mapsto 3/z$.
    \end{itemize}
    Observe that their cross-ratio 
    $$(a,b,c,d)\mapsto\frac{(a-b)(c-d)}{(a-d)(c-b)}$$
    is presevered by the transformation. These respectively are: $1/8,7/8$ and $1/9$.
\end{Ex}

The book's definition of $M_{0,4}$ is then 
$$M_{0,4}\defeq\set{\text{quadruplets of }\bP^1}/\text{projective equivalence}.$$
In a similar fashion to $Q$, we can construct a universal family for $M_{0,4}$. The fibers are once again copies of $\bP^1$ imbued with the corresponding quadruplet, it's \emph{tautological}. The family is thus 
\begin{center}
    \begin{tikzcd}
        M_{0,4}\x\bP^1 \arrow[d, "\pi"']                     \\
        M_{0,4} \arrow[u, "\tau_i(\x 4)"', bend right=49]
        \end{tikzcd}
    \end{center}
where $\tau_1,\tau_2$ and $\tau_3$ single out $0,1$ and $\infty$ in each fiber. The last section is \emph{like} a diagonal section. It is given by the inclusion map of $M_{0,4}$ seen as the diagonal $(z,z)$ of $M_{0,4}\x\bP^1$.

\begin{Th}
The universal family over $M_{0,4}$ also induces every other family of quadruplets up to projective equivalence via pullback. 
\end{Th}

\begin{ptcbp}
    Indeed as in the case of the exercise \ref{ejer-universality-of-quadruplets-no-projective-equivalence}, we start with a family of quadruplets 
    \begin{center}
        \begin{tikzcd}
            B\x\bP^1 \arrow[d, "\pi"']                     \\
            B \arrow[u, "\sigma_i(\x 4)"', bend right=49]
            \end{tikzcd}
        \end{center}
    and show that there exists a base isomorphism between this family and the pullback of the universal family via a certain map.\par 
    Step 1 is to construct the map $\kp$:
    \begin{center}
        % https://tikzcd.yichuanshen.de/#N4Igdg9gJgpgziAXAbVABwnAlgFyxMJZABgBpiBdUkANwEMAbAVxiRACEAdTgD24CMACgD0AjCAC+pdJlz5CKMqKq1GLNu0nSQGbHgJFRpZdXrNWiEAFkA+sDIAWCZJUwoAc3hFQAMwBOEAC2SGQgOBBIRqrmbNxoWCDUDHT8MAyCsvoKIH5Y7gAWOFq+AcGIUeFIAEymahYg3DA8WHA4cAD83ADWaC4SQA
\begin{tikzcd}
    B\x\bP^1 \arrow[d, "\pi"'] &           \\
    B \arrow[r, "\exists?\kp"] & {M_{0,4}}
    \end{tikzcd}
\end{center}

    Intuitively, the map looks at a point $b\in B$, asks the family for its corresponding quadruplet and then normalizes it via a M\"obius transformation.\par 
    Formally, for $b\in B$ we map 
    $$b\mapsto \ssg(b)\defeq (\sg_1(b),\sg_2(b),\sg_3(b),\sg_4(b))$$
    which gives us the associated quadruplet. To normalize it we take the M\"obius transformation 
    $$\la(z)=\frac{(z-\sg_2(b))(\sg_3(b)-\sg_4(b))}{(z-\sg_4(b))(\sg_3(b)-\sg_2(b))}$$
    and apply it repeatedly to our quadruplet as 
    $$\lla(p,q,r,s)=(\la(p),\la(q),\la(r),\la(s)).$$
    Call $t\defeq\la(\sg_1(b))$ the cross-ratio of our quadruplet. The map $\kp$ is thus 
    $$\kp(b)\defeq (\lla\circ\ssg)(b)=(t,0,1,\infty).$$
    Actually, this is still a quadruplet, we are missing the quotient map which sends that quadruplet to its equivalence class inside $M_{0,4}$. But for all the effects, that is our desired map.\par 
    As this is a quadruplet inside $M_{0,4}$, we now have the \emph{canonical} map from $B$ to $M_{0,4}$. With it, we will pullback the universal family and show it's isomorphic to the original family over $B$.\par 
    To construct a fiber of the pullbacked family we take the fiber over $\kp(b)$ and attach it to our original point $b$. In this case the fiber is 
    $$\pi^{-1}(\kp(b))=\set{[t,0,1,\infty]}\x\bP^1$$
    so that'll be the same fiber over $b\in B$.\red{In the same way that I can define a bundle via fibers and transition functions, is it possible to define a family via fibers and\dots and\dots and what?}. 
\end{ptcbp}
\chapter{A Project on Sheaf Cohomology of Line Bundles over $\bP^1$}


\section{Is Bundle like a Family?}

The notion of a family of lines over a base variety $B$ is a map $\pi\: B\x\bC\to\bC$ where $\pi$ is the projection, so that over each point $b\in B$, the fiber of $\pi$ is a copy of the fixed $\bC$. 

\begin{Def}
    A \term{line bundle} over a base $B$ is a map $\pi\: L\to B$ with the following properties:
    \begin{enumerate}[i)]
        \item There's an open cover $(U_i)_{i\in I}$ of $B$ such that 
        $$\pi^{-1}(U_i)\isom U_i\x\bC$$
        where we call $\phi_i\:\pi^{-1}(U)\to U\x\bC$ the isomorphism. This means that the fiber is isomorphic to $\bC$.
        \item For $b\in U_i\cap U_j$, the composition
        $$\set{b}\x\bC\xrightarrow[]{\phi_i^{-1}}\pi^{-1}(b)\xrightarrow[]{\phi_j}\set{b}\x\bC$$
        is a linear isomorphism. This map, $\phi_j\phi_i^{-1}$, is multiplication by a nonzero scalar $\la_b$.
    \end{enumerate}
\end{Def}

Comparing this with the notion of family, we have \emph{local triviality} and the vector space structure between fibers is compatible. This means a line bundle is a locally trivial family of complex lines. 

\begin{figure}[h!]
    \centering
    \includegraphics[width=0.8\textwidth, trim= 0.725cm 19.25cm 11.625cm 2.25cm,clip]{figs/figLineBundleDefn.pdf}
    \caption{Line Bundle with the two properties}
    \label{fig:2.1-LineBdlExample}
\end{figure} 

\begin{Def}
    A \term{section} of a line bundle $L$ over $B$ is a map $s\:B\to L$ with $\pi s=\id_B$. 
\end{Def}

Sections can be defined locally on open sets $U\subseteq B$ or globally when they are defined everywhere on $B$. Intuitively, a section singles out points in fibers. For every $b\in B$, $s(b)$ is a point on the fiber $\pi^{-1}(b)$.
\subsection*{The Fun Part is not the sets, it's the\dots}

\textbf{Base morphisms} of a families are maps between total spaces which carry fibers onto fibers and distiguished points to distinguished points. For the case of line bundles we have just a tiny bit more.
    
\begin{Def}
    A \term{morphism of line bundles} is a map $f\: L_1\to L_2$ which makes the following diagram commute:
    \begin{center}
        \begin{tikzcd}
            L_1 \arrow[rdd, "\pi_1"'] \arrow[rr, "f"] && L_2 \arrow[ldd, "\pi_2"] \\&&\\& B&
            \end{tikzcd}
    \end{center}
    It must hold that on fibers, the map $f$ restricts to a \textbf{linear map}.
\end{Def}

Unwrapping the definition a bit, we have that the diagram commutes when $\pi_1=f\pi_2$. So the question is, where does a point in a fiber, $x\in\pi^{-1}_1(b)$, map to? We would like it to be in the corresponding fiber of $L_2$.\par 
For $f(x)\in\pi_2^{-1}(x)$, it must occur that $\pi_2(f(x))=b$. But we have 
$$\pi_1(x)=\pi_2(f(x))\word{and}\pi_1(x)=b,$$
so it holds that 
$$f\left(\pi^{-1}_1(b)\right)\subseteq \pi_2^{-1}(b).$$
Similarly for sections for sections, the diagram commutes when $s_2=fs_1$. But this means that for $b\in B$, 
$$s_2(b)=f(s_1(b)),$$
so distinguished points of fibers get sent to the corresponding distinguished points.

\section{Line Bundles over $\bP^1$}

We begin by introducing a family of complex manifolds.

\begin{Def}
    The manifold $\cO_{\bP^1}(d)$ is defined by two charts and a transition function:
    $$(\bC^2,(x,u))\xrightarrow[v=u/x^d]{y=1/x}(\bC^2,(y,v)).$$
    This transition function is  $(y,v)=\left(\frac{1}{x},\frac{u}{x^d}\right)$ with inverse $\left(\frac{1}{y},\frac{y}{v^d}\right)$.
\end{Def}
We could also regard this set as $\bC^2$ under the equivalence relation described via the transition function.

\subsubsection{No other line bundles}
$\cO_{\bP^1}(d)$ comes with a natural projection onto $\bP^1$ 
$$(x,u)\mapsto x$$
This allows us to see $\cO_{\bP^1}(d)$ as a line bundle, because every fiber $\pi^{-1}(x)$ is isomorphic to $\bC$. When $x$ is non-zero we get a copy of $\bC$ on both charts, but when $x=0$ or $\infty$, the line is only on one of the charts.\par 
Spoiling ourselves of the fun\footnote{Because it'd be so much fun to prove this are all the line bundles.}, we claim that all line bundles over $\bP^1$ are of the form $\cO_{\bP^1}(d)$ for an integer $d$. 

\chapter{Renzo's Complex Projective Exercises}

\section{Set of points of the projective line}

\begin{Ej}
    Show that there is a bijection between the set $\text{Set}\bCP^1$
 and a quotient set of a disjoint union of two copies of $\bC$.
\end{Ej}

\begin{ptcbr}
    Indeed consider our copies of $\bC$ embedded into $\bC^2$ as the lines $\set{x=1}$ and $\set{y=1}$. Then for a line $\l\in\bCP^1$ our map is 
    $$\l\mapsto \l\cap(\text{corresponding line})\mapsto (\text{corresponding coordinate}).$$
    Explicitly, if our line is $[X:Y]$, then the map is $[X:Y]\mapsto X/Y$ on one chart while $Y/X$ on the other.\par 
    Observe that this map is surjective as every point in each copy of $\bC$ is hit by a line of a different slope. The only points which are not hit twice are the origins of both lines. From this, we define the quotient by identifying the coordinates as $x\sim y$ whenever $y=1/x$. Thus our map becomes a bijection at the level of the quotient as we can now properly trace back each point to a particular line. 
\end{ptcbr}
\section{The projective line as a topological space}

\begin{Ej}[Hopf Fibration]
    Show there is a fibration of topological spaces:
    $$S^1\to S^3\to S^2$$
    meaning that there is a surjective continuous function from the three-dimensional sphere to the two-dimensional sphere, and the inverse image of any point is homeomorphic to a circle. This is called the Hopf fibration; notice that while the construction of these maps is rather mysterious in terms of spheres, it becomes transparent when thinking of the two-dimensional sphere as the complex projective line.
\end{Ej}
%https://math.stackexchange.com/questions/4142758/proving-that-the-hopf-fibration-is-a-fiber-bundle
\begin{ptcbr}
We have shown that the map 
$$\pi_2\: S^3_\bR\to\bCP^1,(\text{pt. in }S^3)\mapsto(\text{corresponding line through origin in }\bC^2)$$
is surjective. Also, we have that 
$$\bCP^1\isom \bC\cup\set{\infty}\isom S^2$$
where the first homeomorphism comes from previous discussion and the second one from stereographic projection. This means that we have a map $S^3\to S^2$ which is our candidate for the Hopf map. It remains to be seen that this map is continuous and that the fibers are homeomorphic to $S^1$.\par 
It suffices to show $\pi_2$ is continuous as the rest of the maps are homeomorphisms. To that effect, take an open set $U\subseteq\bCP^1$. This means that in the quotient topology induced by the $\pi_2$ map, $U$ is open whenever $\pi_2^{-1}(U)$ is open. But this proves immediately that $\pi_2$ is continuous as it takes open sets back to open sets.\par 
Now let $\l\in\bCP^1$, we'll analyze what the fibers are:
$$\pi_2^{-1}(\l)=\set{\la z_0\:\ z_0\in\l\cap S^3,\ \la\in\bC}$$
but when restricting to $S^3$, we get the condition that $\la$ can only vary an $S^1$'s worth of values:
$$\pi_2^{-1}(\l)=\set{\la z_0\:\ z_0\in\l\cap S^3,\ |\la|=1}\isom S^1.$$
This means that fibers of our map are homeomorphic to $S^1$ and thus we have the desired fibration structure.
\end{ptcbr}
\newpage
\section{The projective line as a complex manifold}

\begin{Ej}
    Compute 
    $$\phi_{21}\defeq \vf_2\circ\vf_1^{-1}\mid_{\vf_1(U_1\cap U_2)}\: (\vf_1(U_1\cap U_2),x)\to(\bC,y)$$
    and show that it is a holomorphic function on its domain of definition. Show that its inverse is also holomorphic on its domain of definition.
    
    These exercises show that $\bCP^1$
     has the structure of a complex analytic manifold.
    The pairs $(U_i,\vf_i)$
     are called complex charts, the biholomorphic map $\phi_{21}$
    a transition function, and the coordinates $x$
     and $y$
     are called local (or affine) coordinates.
\end{Ej}
\begin{ptcbr}
Recall that the open set $U_1\cap U_2$ is the collection of lines in $\bCP^1$ which are not $x$ or $y$ axes. The image then is all the non-zero $x$ coordinates of the intersection of those lines with $x=1$. Taking those lines through $\vf_2\circ\vf_1^{-1}$ gives us the $y$-coordinates of the intersections of those lines with the $y=1$ line. We get all except the $y$-axis. Computing this for a particular line, if $x_0$ is the intersection with $x=1$, then $\frac{1}{x_0}$ will be the intersection with $y=1$. Therefore, the map $\vf_2\circ\vf_1^{-1}$ is $x_0\mapsto\frac{1}{x_0}$ of non-zero $x_0$. This function and its inverse are holomorphic as the vertical and horizontal lines are excluded from this.
\end{ptcbr}
\section{Functions on the projective line}

\begin{Ej}
    Show that meromorphic functions $f\: \bCP^1\to\bC$
    may be described in two equivalent ways:
    \begin{enumerate}
        \item As the ratio of two homogeneous polynomials of the same degree in the homogeneous coordinates:
        $$f(X:Y)=\frac{P_d(X,Y)}{Q_d(X,Y)}$$
        \item As a rational function in one of the affine coordinates (with no restrictions on the degrees of the polynomials)
        $$f(x)=\frac{p(x)}{q(x)}$$
    \end{enumerate}
   How do you go from one perspective to the other?
\end{Ej}

\begin{ptcbr}
We begin with the second item by claiming that if $f$ is meromorphic has a zero of degree $m$ at $z=z_0$ then we may write $f=(z-z_0)^mg$ where $g$ is meromorphic but has no zeroes at $z_0$. Similarly for poles. This means that we may write $f$ as a product of possibly repeated linear factors over another product of linear factors. These products are the desired polynomials.\par
We may homogenize to obtain the first characterization.\par   
\red{I don't recall how to construct the function in the homogeneous way :(}
\end{ptcbr}
\newpage
\section{Automorphisms of the projective line}

\begin{Ej}
    Prove that, given any two ordered triple of points $P_1,P_2,P_3$
 and $Q_1,Q_2,Q_3$
of the projective line, there exists a unique automorphism $\Phi$
 of the projective line such that $\Phi(P_i)=Q_i$.
Show that it follows that the only automorphism that fixes three points is the identity.
Describe the subgroups of $\Aut(\bCP^1)$
 consisting of automorphisms that fix one, or two points in the projective line.
\end{Ej}

\begin{ptcbr}
    In affine coordinates, we can map any triple to $0,1,\infty$ by considering the function 
    $$z\mapsto\frac{z-p_1}{z-p_3}\left(\frac{p_2-p_3}{p_2-p_1}\right).$$
    This maps $p_1,p_2,p_3$ to $0,1,\infty$ respectively. This is a Möbius transformation, so it is an automorphism of $\bCP^1$. Call it $\vf_P$ and then create $\vf_Q$, the desired function $\Phi$ is $\vf_Q^{-1}\vf_P$.\par
    From this we immediately see that is $P$ is fixed then the function is 
    $$\Phi=\vf_P^{-1}\vf_P=\id.$$
    If $\Phi$ fixed two points we get rotations about the axis passing through those two points. Be it, for example, $0,\infty$ with scalings $z\mapsto \al z$ or $1,-1$ with $z\mapsto 1/z$.\par
    If only one point is fixed, then it is a translation of the line leaving that point fixed. Say for example maps of the form $z\mapsto \frac{az+b}{d}$ leave infinity fixed.
\end{ptcbr}
\section{Maps to projective spaces}

\begin{Ej}
    We define the degree of $F(\bCP^1)$ to be the number of intersections with a general hyperplane in $\bCP^r$.  Prove that if the degree of the polynomials $P_i(X,Y)$ is equal to $d$, then the degree of $F(\bCP^1)$ is less than or equal to $d$. When does the strict inequality hold?
\end{Ej}

\begin{ptcbr}
    
\end{ptcbr}
\newpage
\section{Line bundles on the projective line}

\begin{Ej}
    For $x_0\neq 0$, let $i_1\:\set{x=x_0}\into\cO_{\bCP^1}(d)$ and $i_2\:\set{y=\frac{1}{x_0}}\into\cO_{\bCP^1}(d)$ be the two inclusions of vertical lines. Show that $i_2^{-1}\circ i_1\:\bC\to\bC$ is a linear isomorphism. Observe that the collection of these linear isomorphisms defines a holomorphic function $c_{12}\:\bCP^1\less\set{0,\infty}\to\bC\less 0$.
\end{Ej}

\begin{ptcbr}
    Let's concretely analyze the $i_2^{-1}\circ i_1$ map. This comes out of the line $\set{x=x_0}$ and gives us 
    $$\Im i_1=\set{(x_0,u)\: u\in\bC}\subseteq (\bC^2,(x,u)).$$
    In order to see what $i_2^{-1}$ does, we translate $\Im i_1$ into $(\bC^2,(y,v))$ via
    $$x\mapsto \frac{1}{y},\quad u\mapsto\frac{v}{y^d}\To v=y^du.$$
    So $\Im i_1$ on the other chart is 
    $$\Im i_1=\Set{\left(\frac{1}{x_0},\frac{u}{x_0^d}\right)\: u\in\bC}$$
    and $i_2^{-1}$ returns us $\frac{u}{x_0^d}$. This means that the composition in question is that map $u\mapsto \frac{u}{x_0^d}$ which means that the map is multiplication by $x_0^{-d}$. For a fixed non-zero $x_0$, this is a linear isomorphism of $\bC$.\par
    The collection of such isomorphisms is obtained when we let $x_0$ vary and the function $c_{12}$ given by $x_0\in\bCP^{1}\less\set{0,\infty}\mapsto x_0^{-d}\in\bC\less\set{0}$ is indeed holomorphic.
\end{ptcbr}
\section{Sections of line bundles}

\begin{Ej}
    Show that if $s_0,s_1$ are two sections of the same line bundle, then their ratio is a (meromorphic) function on $\bCP^1$. Show that if $s_0,s_1,\dots,s_r$
 are $(r+1)$ sections of the same line bundle, then they define a map $\bCP^1\to\bCP^r$.
\end{Ej}

\begin{ptcbr}
    Observe that if $z\in\bCP^1$, then $s_0(z),s_1(z)$ lie on the fiber $\pi^{-1}(z)$ which is isomorphic to $\bC$. So taking their ratio on this fiber does produce a complex number. However, we must verify that the ratio is well-defined. Assume we picked another element of the base, $w\in\bCP^1$ and asked about the ratio of $s_0(w)$ with $s_1(w)$. In this case, observe that there's a linear isomorphism between $\pi^{-1}(z)$ and $\pi^{-1}(w)$ which scales all vectors by the same length. This means that $s_i(w)=\al s_i(z)$ for some $\al\in\bC$ and therefore their ratios are the same.\par
    So $s_0/s_1$ does define a meromorphic function on $\bCP^1$ thanks to the linear isomorphisms.\par
    In the same fashion, if we instead have $r+1$ sections, via a same argument we can see that when changing fibers, the sections only change by a scaling which is the same on all entries. So this means that we may write $\bonj{s_0:\dots:s_r}$ as a function to $\bCP^r$.
\end{ptcbr}

\section{Divisors}

\begin{Ej}
    When you multiply two meromorphic functions, what happens to their divisors? If two meromorphic functions produce the same divisor, what can you say about them?
\end{Ej}

\begin{Ej}
    Let $s$ be a meromorphic section of a line bundle $\pi\: L\to\bCP^1$, we call the support of $\div(s)$ the set of points that appear with non-zero coefficients in $\div(s)$. Show that there is a natural bijection 
    $$T_s\:\pi^{-1}\bonj{\bCP^1\less\supp(\div(s))}\to\bCP^1\less\supp(\div(s))\x\bC.$$
Meditate on the following fact: the function $T_s$ 
 is an isomorphism of complex manifolds, and in fact an isomorphism of line bundles on the punctured $\bCP^1$
 (we of course did not precisely define these notions, so try and make a guess of what these things should mean). It is called a \emph{trivialization}  of the line bundle $\pi\: L\to\bCP^1$
 on the complement of the support of $\div(s)$.
\end{Ej}

\chapter{Incursion into Pseudo-Stable Land}

\section{The Chow ring}

This summary is based on Richard E. Borcherds' introduction to the Chow ring in Youtube.

For a non-singular variety \( V \), we define the \emph{Chow ring} \( A^\ast(V) \), whose elements \emph{correspond} to subvarieties of \( V \), and the product reflects the intersection of these subvarieties. The ring is graded by codimension:
\[
A^\ast(V) = \bigoplus_i A^i(V)
\]
where \( A^i(V) \) consists of classes of subvarieties with codimension \( i \). Ideally, the intersection of a codimension \( m \) subvariety \( X \) and a codimension \( n \) subvariety \( Y \) would yield a subvariety of codimension \( m+n \). However, this does not always hold. Imagine a hyperplane $H$ intersected with itself. So there's a complication in defining this product.

\subsection{Cycles and Intersection Numbers}

An initial attempt to resolve starts by defining \emph{cycles}, which are formal sums of subvarieties. Specifically, we define the group of codimension \( i \) cycles as:
\[
A^i(V) = \left\langle X \mid X \subseteq V, \ \text{closed subvariety}, \ \codim(X) = i \right\rangle
\]
For two cycles \( X \in A^i(V) \) and \( Y \in A^j(V) \), we aim to define their product as:
\[
X \cap Y = \sum_{Z} i(X, Y; Z) Z
\]
where the sum runs over the irreducible components \( Z\) of \( X \cap Y \), and \( i(X, Y; Z) \) is an \emph{intersection number}, representing the multiplicity of the intersection at \( Z \). However, defining this intersection number precisely poses significant challenges.

\subsection{Rational Equivalence and the Chow Group}

To address the ambiguities in defining intersections, particularly when subvarieties do not intersect transversally, we use the notion of \emph{rational equivalence}. Cycles are considered equivalent if their difference is the divisor of a rational function on a subvariety of dimension \( j+1 \). This leads to the following definition:

\begin{Def}
The \( i \)-th \emph{Chow group} \( A^i(V) \) of a non-singular variety \( V \) consists of equivalence classes of codimension \( i \) cycles, where two cycles are equivalent if their difference is a principal divisor, i.e., the zero set of a rational function.
\end{Def}

The \emph{Chow ring} is the direct sum over all Chow groups:
\[
A^\ast(V) = \bigoplus_i A^i(V)
\]
The intersection product on the Chow ring is well-defined:
\[
[X] \cap [Y] = \sum_{[Z]} i(X, Y; Z) [Z]
\]
where \( [X] \), \( [Y] \), and \( [Z] \) denote rational equivalence classes of cycles.

\subsection{Example and Further Considerations}

\begin{Lem}[Chow's moving lemma (1956)]
Given two algebraic cycles $X,Y$ in $V$ a non-singular variety, there is another cycle $Y'$ rationally equivalent to $Y$ on $V$ such that $X$ and $Y'$ intersect \emph{properly}. \red{Look for better reference than wikipedia}.
\end{Lem}

Let's illustrate this by considering the next example.

\begin{Ex}
    Consider the surface $S=\Bl_{\text{pt}}\bP^2$. We have a copy of $\bP^1$ as the exceptional divisor $E$. If we ask what is $E\cap E$, then we have to move $E$ slightly.\par
    This is impossible as $E$ can't be deformed. This is not a counterexample to Chow's lemma but to see this we have be a bit subtle. Consider a line $A$ which we deform to pass about the exceptional divisor, call it $B$. This means that 
    $$A\sim B\cup E\To E\sim A-B.$$
    $A-B$ has a well-defined intersection with $E$ because $A$ doesn't meet $B$ and $B$ has a transversal intersection with $E$. Thus $(A-B)\cap E$ is well defined as we've deformed $E$ into a cycle with negative coefficients.\par
    Observe that if we deform $E$ to something which has transversal intersection with itself, we will acquire negative coefficients because 
    $$E\cap E=(-1)[\text{pt}].$$
    So we will acquire negative coefficients when turning $E$ into a well-behaved cycle.
\end{Ex}

In essence, Chow's lemma doesn't say we can deform subvarieties, it says we can deform cycles and even if we start with one with positive coefficients, we may end up with one with negative coefficients.\par 

Now if $V$ is a singular variety, we may end up with extra complications.

\begin{Ex}
    Suppose we take $V$ to be a cone and two subvarieties $X,Y$ being a hyperbola and a line through the singularity.
\red{Add figure}
    If we intersect $X\cap Y$ we get just one point without troubles, but let us slide $X$ so that $X$ becomes a \emph{double line} through the singularity. So it might the case that $X\sim 2Z$ where $Z$ is the corresponding line. It must happen as well that 
    $$2Z\cap Y=[\text{pt}]\To Z\cap Y =\half[\text{pt}].$$
    We can still get intersections on varieties with singularities but with rational coefficients.
\end{Ex}

This means that it certainly does make sense to consider the Chow ring of $\ov{M}_{g,n}$.

\begin{Qn}
¿Is this why there appears a $\frac{1}{24}$ in some places when doing calculations with $\la$-classes?
\end{Qn}

Fianlly, lets consider examples of Chow groups and Chow rings. In general finding the Chow ring is very hard, but we have
\begin{itemize}
    \item The $0^{\text{th}}$ Chow group, $A^0(V)\isom\bZ$, is generated by $[V]$. This is the class of the whole variety or \emph{the fundamental class}. This is the multiplicative identity in the Chow ring as $X\cap V=X$. 
    \item $A^1(V)$ contains hypersurfaces modulo linear equivalence. This are basically divisors up to linear equivalence which amounts to the Picard group of $V$, $\Pic(V)$. Already for an elliptic curve, $\Pic(V)$ is uncountable.
\end{itemize}

\begin{Ex}
    If $V=\bP^2$, then $A^{i}(V)\isom\bZ$ for $0\leq i\leq 2$. This is because we can decompose $\bP^2$ as 
    $$\bP^2=\text{pt.}\cup\text{line}\cup\text{plane}$$
    In general $A^\ast(\bP^n)\isom\bZ[H]/\genr{H}^{n+1}$ where $H$ represents the class of a hyperplane in $A^1(\bP^n)$.\par
    ¡This is the same case for the Grassmannian! Its Chow ring is also easy to describe as we may decompose the Grassmannian into affine spaces.
\end{Ex}

\begin{Qn}
    ¿What is the Chow ring of the Grassmannian?
\end{Qn}

\begin{Rmk}
    Some of this intersection numbers can also be calculated with Schubert calculus, and they happen to coincide with Littlewood-Richardson coefficients.    
\end{Rmk}

\section{Chern classes}

\subsection{A quick cheat sheet}

Let us begin with the Chern class cheat sheet${}^{\text{TM}}$. This based off of my meeting with Renzo on 20240917. First assume $E\xrightarrow{\pi}B$ is a rank $r$ vector bundle. We have the following:

\begin{itemize}
    \item $c_i(E)\neq 0$ whenever $0\leq i\leq r$.
    \item $c_i(E)$ has degree $i$ in the Chow ring. This means $c_i(E)\in A^i(E)$. (¿or $A^i(B)$?)
    \item $c_0(E)=1$, or in words, it's the fundamental class. It's usually the case that we rescale in order for this to be exactly $1$.
    \item If we define 
    $$c_{\text{tot}}=c_0+c_1+\dots+c_r$$
    and we have a short exact sequence of vector bundles 
    $$0\to F\to E\to Q\to 0,$$
    then $c_{\text{tot}}(E)=c_{\text{tot}}(F)\.c_{\text{tot}}(Q)$. In particular we have 
    $$c_1(E)=c_1(F)+c_1(Q).$$
    \item If $E$ is a line bundle $L$, i.e. rank 1, then 
    $$c_1(L)=[\div(s)],$$
    this is the class of the divisor of a meromorphic section.
    \item $c_1$ commutes with pullbacks.
\end{itemize}

For more, check out \href{https://math.stackexchange.com/questions/989147/quick-question-chern-classes-of-sym-wedge-hom-and-tensor}{this math.se post}\footnote{\href{math.stackexchange.com/q/989147/}{\ttt{math.stackexchange.com/q/989147/}}} or \href{https://rigtriv.wordpress.com/2009/11/03/chern-classes-part-1/}{here}\footnote{\href{https://rigtriv.wordpress.com/2009/11/03/chern-classes-part-1/}{\ttt{https://rigtriv.wordpress.com/2009/11/03/chern-classes-part-1/}}}.

\subsection{A Historical Note}

Continuing with Richard Borcherds' lecture, we focus on how to obtain Chern classes from a vector bundle $E\xrightarrow{\pi}V$ over a non-singular variety $V$. These are certain elements of $A^i(V)$

    On the side of historical notes, in differential geometry, Chern classes take values in the cohomology ring of $V$ when seen as a complex manifold. The Chow ring is related to the cohomology ring via a homeomorphism 
    $$A^i(V)\to H_{2n-2i}(V)\to H^{2i}(V)$$
    where the first map is taking a cycle to a cycle, and then applying Poincaré duality.
    
    \begin{Rmk}
        Personally, I believe it would be more accurate to say that the Chow ring is more closely related to homology groups than to cohomology. This is because cohomology deals with functions mapping cycles to $\bR$ while homology is primarily concerned with formal groups of cycles. 
    \end{Rmk}

    This map is not injective in general as $A^i(V)$ is hopelessly huge in general. For an elliptic curve $E$, it happens that 
    $$A^{1}(E)\word{is uncountable, but}H^{2}(E)=\bZ.$$
    It is also not onto as nothing in the Chow ring maps to $H^{2i+1}(V)$ for any $i$. We may refine our question to 
    $$\text{¿is}\quad A^i(V)\to H^{2i}(V)\quad\text{onto?}$$ 
    And the answer remains negative. The image of this map is subtle, and Hodge established that
    $$H^{2i}(V,\bR)\isom\bigoplus H^{p,q}(X)\word{and}\Im(A^i)\subseteq H^{i,i}(V).$$
    So the previous question can be further refined to, 
    $$\text{¿is}\quad \Im(A^i)=H^{i,i}(V)\cap H^\ast(X,\bZ)?$$
    Once again not positive. Atiyah and Hirzebruch found a torsion element in the cohomology that is not in the image of $A^i(V)$. This leads us to the Hodge conjecture which remains unsolved. 

\subsection{Characteristic Classes}

\begin{Ex}
    For a line bundle $L\xrightarrow[]{\pi}B$, take a section $f$. So at each point $b\in B$, $f$ picks out a point in the fiber $\pi^{-1}(b)$. Usually the set of zeroes of $f$ is a codimension $1$ cycle of $B$.
    \begin{center}
        \red{add drawing}
    \end{center}
    This cycle lives in $H_{n-1}(B)$ where $\dim B=n$. Via Poincaré duality, this gives us an element in $H^1(B)$. This is the first Stiefel-Whitney class of $L$.\par
    If we take $B=S^1$, we have two obvious line bundles, $S^1\x\bR$ and the Möbius bundle. 
    \begin{center}
        \red{add drawing}
    \end{center}
    In the case of the trivial bundle, the zeroes of a non-zero section will always be even, while in the case of the Möbius bundle, it'll always be odd. This means that at the level of $H^{1}(S^1,\bZ/2\bZ)$ we always get the $0$ and $1$ elements respectively. So we can distinguish this line bundles by counting the zeroes of a generic section.
\end{Ex}

The idea of characteristic classes is a generalization of this ideas. If we have a vector bundle, we can look at a section's zeroes and they will give us homology (or cohomology) classes.

\subsection{A First Step into Complex Land}

Chern classes on a complex vector bundle will take values in $A^*(V)$ (or $H^\ast(V)$ if we are doing differential geometry).\par
Consider a complex line bundle $L\xrightarrow[]{\pi}V$ over a non-singular variety or complex manifold $V$. We take the set of zeroes of a section $f$. This has complex codimension $1$ in $V$ and means that we have an element of $A^1(V)$, or in the real case, codimension $2$ with it being an element of $H_{2n-2}(V)\isom H^2(V)$.\par 
So the first Chern class is given by the cycle of zeroes of a section. We denote it $c_1(V)$.

\begin{Ex}
    ¿What is $c_1$ of $\cO(-1)$, the tautological bundle of $\bP^1$? We know that $\cO(-1)$ possesses no non-zero sections. But this is not a problem, as we may use the fact that 
    $$c_1(L\ox L')=c_1(L)+c_1(L')$$
    so we can tensor $\cO(-1)\ox\cO(d)$ for a sufficiently large $d$. This will be a line bundle with enough sections (\emph{hopefully}) so that $c_1$ makes sense here. In this case we define 
    $$c_1(\cO(-1))=c_1(\cO(-1)\ox\cO(d))-c_1(\cO(d)).$$
\end{Ex}

\subsubsection{More analytically, but not algebraically}
Another way to get $c_1$, we have the sequence of sheaves 
$$0\to\un{\bZ}\to\cO\xrightarrow{\exp}\cO^\ast\to 1$$
which only works in complex analytic geometry. So in terms of the long exact sequence of cohomology we have 
$$\dots\to H^1(V,\cO^\ast)\xrightarrow{c_1}H^2(X,\bZ)\to\dots$$
where the first term \emph{classifies} line bundles and the connecting morphism is more or less the first Chern class. 

\subsection{Higher Chern classes}

Returning to our vector bundle $E\xrightarrow{\pi}V$ of rank $r>1$, we follow Grothendieck's approach. We can associate $E$ with the projective bundle $P(E)$ where the typical fiber is isomorphic to $\bP^{r-1}$. Now $P(E)$ has itself a line bundle $\cO(1)$ leading us to a Chern class $c_1(\cO(1))\in A^1(P(E))$ which is not in $A^1(V)$. Thus, we must relate $A^\ast(P(E))$ with $A^\ast(V)$.\par
Call $\xi=c_1(\cO(1))$ to simplify notation. The Chow ring $A^\ast(P(E))$ forms a free, rank $r$ $A^\ast(V)$-module with basis $\set{1,\xi,\dots,\xi^{r-1}}$. This modularity arises from a similar reasoning to the construction of the Chow ring of $\bP^n$, which corresponds to the case when $V$ is a point. Now the element $\xi^r$ can be expressed as a linear combination of the basic elements: 
$$\xi^r=c_1\xi^{r-1}-c_2\xi^{r-2}+\dots\pm c_r,\word{for some}c_i\in A^i(V).$$
¡These coefficients are the Chern classes in $A^\ast(V)$! The class $c_r$ can be described easily. It is represented by a cycle of the zeroes of a section of $E$. The zeroes of that section will generally have codimension $r$ and we will be able to represent it by $c_r$.

\section{Examples and Theorem 3.1}

Theorem 3.1 in MattAndRenzo deals with the Mumford formula. This is the product of the total Chern class of the Hodge bundle over $\ov{M}_{g,n}$ with the one of the dual. The formula itself is 
$$(1+\hat{\la}_1+\dots+\hat\la_g)(1-\hat{\la}_1+\dots+(-1)^g\hat\la_g)=\sum_{i=0}^g\frac{1}{i!}\cG_\ast^i\left(\prod_{j=1}^i(\psi_{\star j}-\psi_{\8 j})\right).$$

To prove the formula we will do examples first. The first case that we deal with is in $\ov M_{1,n}$ for a fixed $n\geq 1$.

\begin{Ex}
    The pseudo stable Mumford formula in this case states:
$$(1+\hat{\la}_1)(1-\hat{\la}_1)=\frac{1}{0!}\cG_\ast^0\left(\id\right)+\frac{1}{1!}\cG_\ast(\psi_{\star 1}-\psi_{\8 1}).$$
    Let us analyze the left side. By definition, the pseudo-stable lambda class is 
    $$\hat\la_j=\cT^\ast(\la_j)=\sum_{i=0}^j\frac{1}{i!}\cG_\ast^i(p_0^\ast(\la_{j-i}))$$
    so in the case of $\hat\la_1$ we have 
    $$\hat\la_1=\sum_{i=0}^1(\dots)=\cG_\ast^0(p_0^\ast(\la_{1-0}))+\cG_\ast^1(p_0^\ast(\la_{1-1}))$$
    where $\cG^0$ is the identity map and $\la_0$ is the fundamental class of $A^\ast(\ov M_{g,n})$. So using this we have 
    \begin{align*}
        \hat\la_1&=\cG_\ast^0(p_0^\ast(\la_{1}))+\cG_\ast^1(p_0^\ast(\la_{0}))\\
        &=\la_1+\cG_\ast^1(\id)
    \end{align*}
    and expanding the product we get 
    \begin{align*}
        (1+\hat{\la}_1)(1-\hat{\la}_1)&=(1+\la_1+\cG_\ast^1(\id))(1-\la_1-\cG_\ast^1(\id))\\
        &=(\La_1(1)+\cG_\ast^1(\id))(\La_1(-1)-\cG_\ast^1(\id))\\
        &=\La_1(1)\La_1(-1)-\La_1(1)\cG_\ast^1(\id)+\cG_\ast^1(\id)\La_1(-1)-\left(\cG_\ast^1(\id)\right)^2.
    \end{align*}
    Here, $\La_1$ represents the total Chern class. We may analyze term by term this expression:
    \begin{itemize}
        \item The product $\La_1(1)\La_1(-1)=1$ is the usual Mumford formula. 
        \item For the case $\La_1(1)\cG_\ast^1(\id)$ what we have is $(1+\la_1)\cG_\ast^1(\id)$.
    \end{itemize}
    We now arrive to the question of what is $\cG_\ast^1(\id)$ so let's take a step back and remember how $\cG^1$ works as a map:
    $$\cG^1\: \ov M_{(1-1),n+1}\x\ov M_{1,1}\to \ov M_{1,n},$$
    and recall that the $\id$ in the argument is $p_0^\ast(\la_0)$ where the $\la_0$ comes from the Chow ring of $\ov M_{0,n+1}$. In particular, it is the fundamental class of the space. Pulling it back just makes it part of an ordered pair and then pushing it forwards attaches it to a copy of an elliptic curve. Graphically:
    \begin{figure}[h]
        \centering
        \includegraphics[width=0.5\textwidth, clip]{figs/GluingMapActionEx}
    \end{figure}
    This means that the pushforward of the fundamental class through $\cG^1$ is the class of curves described by the dual graph on the right of the diagram.\footnote{This sentence sounds a bit fishy. I just remember what you said about ``I can't pick a concrete representative''.}\par
    From this, we have 
    \begin{align*}
        (1+\la_1)\cG_\ast^1(\id)&=\cG_\ast^1(\id)+\la_1\cG_\ast^1(\id)\\
        &=\cG_\ast^1(\id)+\cG_\ast^1(p_1^\ast(\la_1))+\cG_\ast^1(p_0^\ast(\la_1))\\
        &=\cG_\ast^1(\id)+\cG_\ast^1(p_1^\ast(\la_1))+0.
    \end{align*}
    This quantity should be equal to $\cG^1_\ast(p_1^\ast(\La_1(1)))$, and expanding this we have
    \begin{align*}
        \cG^1_\ast(p_1^\ast(\La_1(1)))&=\cG^1_\ast(p_1^\ast(1+\la_1))\\
        &=\cG^1_\ast(p_1^\ast(1))+\cG^1_\ast(p_1^\ast(\la_1))\\
        &=\cG^1_\ast(1)+\cG^1_\ast(p_1^\ast(\la_1)),
    \end{align*}
    so both quantities are equal.\newpage
    Diagramatically we have 
    \begin{figure}[h]
        \centering
        \includegraphics[width=0.5\textwidth, clip]{figs/DiagramGenus1TotChernG1PB}
    \end{figure}\\
    The calculation for $\cG^1_\ast(\id)\La_1(-1)$ is similar but different signs. Finally calculating $\cG_\ast^1(\id)^2$ amounts to finding a self intersection number.  
\end{Ex}

\begin{Qn}
    I remember how to find $E\cap E$ in $\Bl_{\text{pt}}\bP^2$ by deforming a cycle that goes through $E$. ¿Is the technique to find $\cG_\ast^1(\id)\cap \cG_\ast^1(\id)$ similar?
\end{Qn}
%%%%%%%%%%%% Contents end %%%%%%%%%%%%%%%%
\ifx\nextra\undefined
\printindex
\else\fi
\nocite{*}
\bibliographystyle{plain}
\bibliography{bibiDoctoralNotebook.bib}
%https://anddil.github.io/teaching/ %moduli of curves and maps
%https://www.math.colostate.edu/~renzo/teaching/Toric18/Linebundles.pdf %Renzo LBS
\end{document} 

