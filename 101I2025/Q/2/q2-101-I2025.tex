\documentclass[12pt]{exam}
\usepackage{amsthm}
\usepackage{libertine}
\usepackage[utf8]{inputenc}
\usepackage[margin=1in]{geometry}
\usepackage{amsmath,amssymb}
\usepackage{multicol}
\usepackage[shortlabels]{enumitem}
\usepackage{siunitx}
\usepackage{cancel}
\usepackage{graphicx}
\usepackage{pgfplots}
\usepackage{listings}
\usepackage{tikz}



\pgfplotsset{width=10cm,compat=1.9}
\usepgfplotslibrary{external}
\tikzexternalize

\newcommand{\class}{Math 101-002} % This is the name of the course 
\newcommand{\examnum}{Quiz 2} % This is the name of the assignment
\newcommand{\examdate}{February 6} % This is the due date





\begin{document}
\pagestyle{plain}
\thispagestyle{empty}

\noindent
\textbf{\class}\\
\textbf{\examnum}, \textbf{\examdate} \\

% Name \hfill CSU ID \# \hspace{2.25in}

%\vspace{10 pt}

\setlength{\tabcolsep}{3.5cm} % Default value: 6pt
\renewcommand{\arraystretch}{1.5}
\setlength\extrarowheight{1cm}
\begin{tabular}{ |c|c| } 
 \hline
 Name   & CSU ID \#  \\ 
 \hline
\end{tabular}
% ---
\vspace{10pt}

Be sure to read each question fully and carefully. Multiple choice answer bubbles must be fully filled in.  There is space to the right of each multiple choice question to show work, if your work is correct you can get points even with an incorrect multiple choice answer.  


\iffalse

    \foreach \s in {1,...,5}{
          \choice $P_\s$ has no power 
     }%;
\fi


\begin{enumerate} 

\item Consider the Weighted Voting Scheme $[q:9,7,1]$ formed by players $P_1, P_2$ and $P_3$.

\begin{enumerate}
\item What are the minimum and maximum possible values of $q$? (4 points) [Hint: Recall that the quota lies between half the total votes and the total votes.]
\begin{checkboxes}
    \choice 7
    \choice 9
    \choice 11
    \choice 17
    \choice 23
\end{checkboxes}
\vfill
\item Which value of $q$ results in a dictator for this scheme? (2 points)
\begin{checkboxes}
    \foreach \s in {7,...,11}{
        \choice \s
   }%;
\end{checkboxes}
\vfill
\item For the previous value, who is a dictator, why? (2 points)
\begin{checkboxes}
    \choice $P_1$ is a dictator because $P_1$ has enough votes to pass a motion single-handedly.
    \choice $P_1$ is a dictator because no motion can pass without their votes.
    \choice Both $P_1$ and $P_2$ are dictators because they have enough votes to pass motions together.
    \choice Both $P_1$ and $P_2$ are dictators because no motion can pass without both of their votes.
\end{checkboxes}
\vfill
\item For the previous value, who has no power, why? (2 points)
\begin{checkboxes}
    \choice Both $P_2$ and $P_3$ have no power because they can pass a motion together.
    \choice Both $P_2$ and $P_3$ have no power because $P_1$ is a dictator.
    \choice All players have no power because $P_1$ is a dictator.
    \choice All players have no power because no one can pass a motion single-handedly.
\end{checkboxes}
\vfill\newpage
\item For which value of $q$ is there exactly one player with veto power? (2 points)
\begin{checkboxes}
    \foreach \s in {8,...,12}{
        \choice \s
   }%;
\end{checkboxes}
\vfill
\item For the previous value, which player has veto power, why? (2 points)
\begin{checkboxes}
    \choice $P_1$ has veto power because $P_1$ has enough votes to pass a motion single-handedly.
    \choice $P_1$ has veto power because $P_2$ and $P_3$ together have only $8$ votes.
    \choice $P_2$ has veto power because $P_2$ has enough votes to pass a motion single-handedly.
    \choice $P_2$ has veto power because $P_1$ and $P_3$ together have only $10$ votes.
\end{checkboxes}
\vfill
\item Which values $q$ guarantee that \emph{two players} will have veto power? (2 points)
\begin{checkboxes}
    \choice 7
    \choice 9
    \choice 11
    \choice 15
    \choice 17
\end{checkboxes}
\vfill
\item For the previous values, which players have veto power, why? (2 points)
\begin{checkboxes}
    \choice Both $P_1$ and $P_3$ have veto power because no motion can pass without at least one of their votes. 
    \choice Both $P_1$ and $P_2$ have veto power because no motion can pass without at least one of their votes. 
    \choice Both $P_1$ and $P_2$ have veto power because no motion can pass without both of their votes. 
    \choice Both $P_1$ and $P_3$ have veto power because no motion can pass without both of their votes. 
\end{checkboxes}
\vfill
\item Which value of $q$ guarantees that all players have veto power? (2 points)
\begin{checkboxes}
    \choice 7
    \choice 9
    \choice 11
    \choice 15
    \choice 17
\end{checkboxes}
\end{enumerate}
\vfill

\newpage

\item Assume you're on a family trip with your two uncles $U$ and two cousins $C$ (you are also represented by the letter $C$). Family trip decisions are decided by a majority of the votes (that is, at least three people must vote Yes), but at least one uncle must vote Yes (that is, the three children don't have enough weight to carry the motion).\par
Using a scheme $[q:U,U,C,C,C]$ find the smallest possible values for $q,U$ and $C$ such that the conditions are held. 




\end{enumerate}
\end{document}

