\documentclass[12pt]{exam}
\usepackage{amsthm}
\usepackage{libertine}
\usepackage[utf8]{inputenc}
\usepackage[margin=1in]{geometry}
\usepackage{amsmath,amssymb}
\usepackage{multicol}
\usepackage[shortlabels]{enumitem}
\usepackage{siunitx}
\usepackage{booktabs}
\usepackage{graphicx}
\usepackage{pgfplots}
\usepackage{listings}
\usepackage{tikz}



\pgfplotsset{width=10cm,compat=1.9}
\usepgfplotslibrary{external}
\tikzexternalize

\newcommand{\class}{Math 101-002} % This is the name of the course 
\newcommand{\examnum}{Exam 2} % This is the name of the assignment
\newcommand{\examdate}{March 13} % This is the due date





\begin{document}
\pagestyle{plain}
\thispagestyle{empty}

\noindent
\textbf{\class}\\
\textbf{\examnum}, \textbf{\examdate} \\

% Name \hfill CSU ID \# \hspace{2.25in}

%\vspace{10 pt}

\setlength{\tabcolsep}{3.5cm} % Default value: 6pt
\renewcommand{\arraystretch}{1.5}
\setlength\extrarowheight{1cm}
\begin{tabular}{ |c|c| } 
 \hline
 Name   & CSU ID \#  \\ 
 \hline
\end{tabular}
% ---
\vspace{10pt}

Be sure to read each question fully and carefully. Multiple choice answer bubbles must be fully filled in.  There is space to the right of each multiple choice question to show work, if your work is correct you can get points even with an incorrect multiple choice answer.  


\iffalse

    \foreach \s in {1,...,5}{
          \choice $P_\s$ has no power 
     }%;
\fi


\begin{enumerate} 

\item For questions \ref{firstQnSec1} through \ref{lastQnSec1} consider the following graph:\par

\textcolor{red}{ADD Drawing}

\begin{enumerate}
    \item \label{firstQnSec1} What is the degree of the vertex $A$? (2 points)
    \begin{checkboxes}
        \choice choices choices
    \end{checkboxes}
    \vfill
    \item Which edges are bridges (cut-edges)? (2 points)
    \begin{checkboxes}
        \choice it's the skyfall (music cue)
    \end{checkboxes}
    \vfill
    \item How many vertices does this graph have? (2 points)
    \begin{checkboxes}
        \choice meik et beter
    \end{checkboxes}
    \vfill
    \item \label{lastQnSec1} Find the number of edges in this graph:
    \begin{checkboxes}
        \choice es el skaifol
    \end{checkboxes}
    \vfill
\end{enumerate}
\item For questions \ref{firstQnSec2} through \ref{lastQnSec2} consider the following graph:\par

\textcolor{red}{ADD Drawing}

\begin{enumerate}
\item \label{firstQnSec2} If each person gets a vote per each \$1000 they put into the fund, write down the weighted voting scheme for this setting assuming Markus gets $M$ votes: (3 points)\vspace{1em}
$$[24:M,\hspace{1cm},\hspace{1cm},\hspace{1cm}]=[24:M,12,7,4]$$
\vfill
\newpage
\item Find the minimum and maximum values for Markus' share of votes $M$, \underline{pick 2 options}: (4 points)
\begin{checkboxes}
   \choice 7
   \choice 12 (CORRECT)
   \choice 17
   \choice 19
   \choice 23
   \choice 24 (CORRECT)
\end{checkboxes}
\vfill
\item Which value of $M$ results in someone being a dictator? (2 points)
\begin{checkboxes}
    \choice 7
   \choice 12
   \choice 17
   \choice 19
   \choice 23
   \choice 24 (CORRECT)
\end{checkboxes}
\vfill
\item Using the value of $M$ you chose in the previous item, who is a dictator, why? (2 points)
\begin{checkboxes}
    \choice (CORRECT) Markus is a dictator because they have enough votes to pass a motion single-handedly.
    \choice Markus and Natalie are dictators because any motion can pass without their votes.
    \choice Natalie is a dictator because the other people can pass motions without them.
    \choice Both Markus and Natalie are dictators because no motion can pass without both of their votes.
\end{checkboxes}
\vfill
\item Recall a player in a Weighted Voting Scheme has \emph{no power} when they have no say in the outcome of the voting. For there to be exactly one player with no power, the value of $M$ must be between... \underline{Pick 2 options}: (4 points)
\begin{checkboxes}
    \choice 7
    \choice 12
    \choice 17 (CORRECT)
    \choice 19 (CORRECT)
    \choice 23
    \choice 24
\end{checkboxes}
\vfill
\newpage
\item For the previous values of $M$ who is the player with no power? (2 points)
\begin{checkboxes}
    \choice Oscar has no power because no motion can pass without their votes.
    \choice Oscar has no power because both Markus and Natalie or Markus and Pauline can pass resolutions on their own meaning Oscar's vote doesn't influence the decision.
    \choice Pauline has no power because they can pass motions single-handedly.
    \choice (CORRECT) Pauline has no power because both Markus and Natalie or Markus and Oscar can pass resolutions on their own meaning Pauline's vote doesn't influence the decision.
\end{checkboxes}
\vfill
\item Which values of $M$ result in players (can be one, or more than one) with veto power? \underline{Pick 2 options}: (4 points)
\begin{checkboxes}
    \choice 7
   \choice 12 (CORRECT)
   \choice 17
   \choice 19
   \choice 23 (CORRECT)
   \choice 24
\end{checkboxes}
\vfill
\item Which value of $M$ result in \emph{exactly two} players with veto power? (2 points)
\begin{checkboxes}
    \choice 7
    \choice 12 (CORRECT)
    \choice 17
    \choice 19
    \choice 23
    \choice 24
\end{checkboxes}
\vfill
\item \label{lastQnSec2} For the previous value, which players have veto power, why? (2 points)
\begin{checkboxes}
    \choice Both Natalie and Oscar have veto power because all motions can pass without their consideration.
    \choice (CORRECT) Both Markus and Natalie have veto power because no motion can pass without both of their votes.
    \choice Both Oscar and Pauline have veto power because no coalition can pass any motion at all.
    \choice Both Natalie and Pauline have veto power because they need the support of all the players to pass a motion.
\end{checkboxes}
\vfill
\end{enumerate}
\newpage

\item In this exercise we will explore graphs with Euler and Hamilton walks or circuits. Follow the instructions and complete each task as asked:
\begin{itemize}
    \item Explain the difference between a walk and a circuit of a graph. (4 points)
    \item Explain the difference between Eulerian and Hamiltonian paths or circuits. (4 points)
    \item Draw a graph (doesn't need to be very big) which contains an Euler tour but not an Euler circuit. (4 points)
    \item Draw a graph which contains a Hamilton tour but doesn't have an Euler tour. (4 points)
    \item Draw a graph which contains an Euler circuit but not a Hamilton tour. (4 points)
    \item Extra: ¿Can you draw a graph with an Euler circuit and a Hamilton tour but not a Hamilton circuit? (4 extra points)
    \item Extra: ¿Can you draw a graph with an Euler tour and a Hamilton circuit but not a Euler circuit? (4 extra points)
\end{itemize}


Sol:
\begin{enumerate}
    \item A walk starts and ends at different places whereas a circuit begins and ends at the same place.
    \item Eulerian means that edges are the object of interest to traverse, while Hamiltonian means that the vertices are the traversed ones.
    \item Any path graph.
    \item A path graph with any pair of middle vertices connected.
    \item Two triangles joined at a vertex, say a ribbon.
    \item The ribbon graph again.
    \item A cycle with more than four vertices with any two non-adjacent vertices connected.
\end{enumerate}


\end{enumerate}
\end{document}

