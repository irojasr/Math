\documentclass[12pt]{exam}
\usepackage{amsthm}
\usepackage{libertine}
\usepackage[utf8]{inputenc}
\usepackage[margin=1in]{geometry}
\usepackage{amsmath,amssymb}
\usepackage{multicol}
\usepackage[shortlabels]{enumitem}
\usepackage{siunitx}
\usepackage{booktabs}
\usepackage{graphicx}
\usepackage{tikz}
\usepackage{pgfplots}

\newcommand{\class}{Math 101-002}
\newcommand{\examnum}{Exam 2 Makeup}
\newcommand{\examdate}{Summer 2025}

\begin{document}
\pagestyle{plain}
\thispagestyle{empty}

\noindent
\textbf{\class}\\
\textbf{\examnum}, \textbf{\examdate} \\

\vspace{10pt}

\textbf{Name/CSU ID:}

\vspace{10pt}

Be sure to read each question fully and carefully. There is space to the right of each multiple choice question to show work. If your work is correct, you can receive credit even with an incorrect choice.

\begin{questions}

\section*{Section 1: Euler Graphs and Graph Analysis (Ch. 5)}

\question[10] Consider the graph $G$ with vertices $\{A, B, C, D\}$ and the following edges and weights:

\begin{center}
\begin{tabular}{ccccccc}
\textbf{Edge} & A--B & B--C & C--D & D--A & B--D & A--C \\
\hline
\textbf{Weight} & 3 & 4 & 5 & 2 & 1 & 6 \\
\end{tabular}
\end{center}

\begin{parts}

\part[2] Draw the graph in question.

\vspace{3cm}
\part[2] How many vertices of odd degree does this graph have?

\begin{choices}
\choice 0
\choice 2
\choice 4
\choice 6
\end{choices}

\part[2] Does this graph have an Euler circuit?
\begin{choices}
\choice Yes
\choice No
\end{choices}

\part[4] Draw an Eulerization of this graph.

\vspace{2cm}
\end{parts}
\newpage
\question[4] Draw a graph with:
\begin{itemize}
\item an Euler path but not an Euler circuit.
\item a Hamilton circuit.
\end{itemize}

\newpage

\section*{Section 2: Hamilton Graphs and Optimization (Ch. 6)}

\question Consider the following complete weighted graph of 5 cities:

\begin{center}
\begin{tabular}{c|ccccc}
 & A & B & C & D & E \\
\hline
A & - & 3 & 4 & 2 & 7 \\
B & 3 & - & 5 & 6 & 3 \\
C & 4 & 5 & - & 4 & 5 \\
D & 2 & 6 & 4 & - & 6 \\
E & 7 & 3 & 5 & 6 & - \\
\end{tabular}
\end{center}

\begin{parts}
\part[3] Draw the graph and label all weights.

\vspace{4cm}

\part[3] Apply the Nearest-Neighbor algorithm starting at A. Where does the walk end, and what is the total weight?

\vspace{4cm}

\part[3] Use the Cheapest-Link algorithm to find a Hamilton circuit. What is its total weight?

\vspace{4cm}
\end{parts}
\newpage
\question[6] Explain the difference between a Hamilton path and an Euler path. Give an example of a real-world situation best modeled by each.

\vspace{8cm}

\question[6] You are designing roads between six town centers: A, B, C, D, E, and F. \emph{Each town must be connected so that you can get from any town to any other}.

\begin{parts}
\part[3] What type of structure are you building? (Be specific in graph-theoretic terms.)

\vspace{3cm}

\part[3] What algorithm could help you minimize the total travel time between towns? Briefly explain how it works.

\vfill
\end{parts}
\newpage
\section*{Discussion (10 points)}

\question[10] You along with friends are planning a summer road trip across five cities. You want to visit each city exactly once and return to where you started. However, you don't trust GPS: you'll plan the route by hand. 

Discuss whether you should model this as a Hamiltonian or Eulerian circuit. What algorithms could you use to plan a route? If one friend insists on stopping at the cheapest gas station in each town, would that affect your route? What limitations might arise in the modeling of this real-life problem?

\vspace{8cm}

\end{questions}

\end{document}
