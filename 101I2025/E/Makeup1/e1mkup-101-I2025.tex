\documentclass[12pt]{exam}
\usepackage{amsthm}
\usepackage{libertine}
\usepackage[utf8]{inputenc}
\usepackage[margin=1in]{geometry}
\usepackage{amsmath,amssymb}
\usepackage{multicol}
\usepackage[shortlabels]{enumitem}
\usepackage{siunitx}
\usepackage{booktabs}
\usepackage{graphicx}
\usepackage{tikz}
\usepackage{pgfplots}

\newcommand{\class}{Math 101-002}
\newcommand{\examnum}{Exam 1 Makeup}
\newcommand{\examdate}{Summer 2025}

\begin{document}
\pagestyle{plain}
\thispagestyle{empty}

\noindent
\textbf{\class}\\
\textbf{\examnum}, \textbf{\examdate} \\

\vspace{10pt}

\textbf{Name/CSU ID:}

\vspace{10pt}

Be sure to read each question fully and carefully. There is space to the right of each multiple choice question to show work. If your work is correct, you can receive credit even with an incorrect choice.
\begin{questions}

\section*{Section 1: Voting Systems and Fairness (Ch. 1)}

\question[10] For the following preference schedule, answer the following questions:

\begin{center}
\begin{tabular}{cccc}
\toprule
\# Voters & 10 & 8 & 7 \\
\midrule
1st & P & Q & R \\
2nd & Q & R & P \\
3rd & R & P & Q \\
\bottomrule
\end{tabular}
\end{center}

\begin{parts}
\part[2] Who wins under the plurality method?
\begin{choices}
\choice P
\choice Q
\choice R
\end{choices}

\part[3] Use the Borda count to compute the total score for each candidate. Who wins?

\vspace{2cm}

\part[3] Is there a Condorcet candidate? If so, who?

\vspace{2cm}

\part[2] Suppose we use the plurality-with-elimination method. Who is eliminated first?

\begin{choices}
\choice P
\choice Q
\choice R
\end{choices}
\vfill
\end{parts}
\newpage
\question[4] Explain the difference between a Condorcet candidate and a majority winner. Can a candidate be one without being the other?

\vspace{9cm}

\question[4] Arrow’s Impossibility Theorem states that no voting system can satisfy all fairness criteria for three or more candidates. List and briefly explain two of the fairness criteria.
\vfill
\newpage
\section*{Section 2: Weighted Voting (Ch. 2)}

\question[10] Consider the weighted voting system $[q : 7, 5, 3, 1]$ with players $P_1$ through $P_4$.

\begin{parts}
\part[2] What is the smallest value of $q$ such that there is no dictator?
\begin{choices}
\choice 6
\choice 9
\choice 11
\choice 16
\end{choices}

\part[2] For the value of $q=12$, is any player a dummy player?

\begin{choices}
\choice Yes: $P_1$
\choice Yes: $P_2$ and $P_4$
\choice Yes: $P_3$ and $P_4$
\choice No player is a dummy
\end{choices}

\part[3] Under the value of $q=10$, is there a player with veto power? Explain.

\vspace{4cm}

\part[3] In the coalition $\{P_1, P_2, P_4\}$ for the value $q=12$, is $P_4$ a critical player? Explain.

\vspace{4cm}
\end{parts}
\newpage
\question[4] Explain the difference between a player with veto power and a dictator in a weighted voting system.

\vspace{7cm}

\question[6] A council has four players with weights $[q : 10, 7, 4, 2]$ and quota $q = 14$.

\begin{parts}
\part[3] List all winning coalitions.

\vspace{6cm}

\part[3] For each winning coalition, identify the critical players.

\vfill
\end{parts}
\newpage
\section*{Discussion (10 points)}

\question[10] Consider a student council where larger departments are assigned more votes than smaller departments. Do you think a weighted voting system is a fair way to make decisions in this context? Refer to concepts such as veto power, dummy players, and coalition-building in your answer.

\vspace{8cm}

\end{questions}

\end{document}
