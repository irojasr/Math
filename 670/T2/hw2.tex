\documentclass[11pt]{article}
\usepackage{fullpage}
\usepackage{amsmath,amsthm, amssymb, amsfonts, amscd}
\usepackage[mathscr]{eucal}
\usepackage{graphicx}
\usepackage{psfrag}
\usepackage[usenames,dvipsnames]{color}
\usepackage{subfigure}

%       Theorem environments

%% \theoremstyle{plain} %% This is the default
\newtheorem{theorem}{Theorem}[section]
\newtheorem{corollary}[theorem]{Corollary}
\newtheorem{lemma}[theorem]{Lemma}
\newtheorem{proposition}[theorem]{Proposition}
\newtheorem{ax}{Axiom}
\newtheorem{conjecture}[theorem]{Conjecture}

\theoremstyle{definition}
\newtheorem{definition}{Definition}[section]

\theoremstyle{definition}
\newtheorem{remark}[theorem]{Remark}


\newcommand{\ext}[1]{%
    {\def\tmp{#1}
    \ifx\tmp\empty
        {\textstyle\bigwedge}
    \else
        {\textstyle\bigwedge\!\!^{#1}}
    \fi}}
%Math Definitions


\newcommand{\Q}{{\mathbb Q}}
\newcommand{\Z}{{\mathbb Z}}
\newcommand{\R}{{\mathbb R}}
\newcommand{\C}{{\mathbb C}}
\newcommand{\N}{{\mathbb N}}

\newcommand{\RP}{{\mathbb{RP}}}
\newcommand{\CP}{{\mathbb{CP}}}



\newcommand{\sgn}{\operatorname{sgn}}
\newcommand{\dx}{\,\mathrm{d}x}
\newcommand{\dy}{\mathrm{d}y}
\newcommand{\dz}{\mathrm{d}z}
\newcommand{\inv}{\operatorname{inv}}
\newcommand{\I}{\mathbf{i}}
\newcommand{\J}{\mathbf{j}}
\newcommand{\K}{\mathbf{k}}
\newcommand{\SO}{\operatorname{SO}}
\newcommand{\dt}{\operatorname{dt}}
\newcommand{\tw}{\operatorname{tw}}
\newcommand{\ds}{\operatorname{ds}}
\newcommand{\st}{\operatorname{st}}
\newcommand{\sech}{\operatorname{sech}}
\newcommand{\range}{\operatorname{range}}
\newcommand{\nul}{\operatorname{null}}
\newcommand{\spa}{\operatorname{span}}
\newcommand{\quat}{\mathbb{H}}
\renewcommand{\Re}{\operatorname{Re}}
\renewcommand{\Im}{\operatorname{Im}}
\newcommand{\from}{\co\!\!}


\newcommand{\Field}{\mathbb{F}}

\usepackage{xcolor}
\definecolor{linkblue}{HTML}{003d73}
\definecolor{linkgreen}{HTML}{006161}
\definecolor{linkred}{HTML}{a11950}
\usepackage{hyperref}
\hypersetup{
	pdftitle={Math 670 HW \#2},
	pdfauthor={Clayton Shonkwiler},
	pdfsubject={differential geometry},
	pdfkeywords={differential geometry, homework, Math 670},
	colorlinks=true,
	linkcolor=linkblue,
	citecolor=linkgreen,
	urlcolor=linkred
}
% \pagestyle{empty}
\def\co{\colon}

\begin{document}
\begin{center}
{\Large\textbf{Math 670 HW \#2}}\\
Due 11:59 PM Friday, March 28
\end{center}




\begin{enumerate}	
	
	\item Let $A: V \to W$ be a linear map between vector spaces.
	\begin{enumerate}
		\item Show that the induced map $\ext{k}(V) \to \ext{k}(W)$ is well-defined by
		\[
			v_1 \wedge \ldots \wedge v_k \mapsto Av_1 \wedge \ldots \wedge Av_k
		\]
		(extending linearly to sums).
		
		\item Show that the map $A^*: W^* \to V^*$ defined by $(A^\ast(\eta))(v) := \eta(A(v))$ determines a map $\ext{k}(W^*) \to \ext{k}(V^*)$.
		
		\item Show that, if $V$ is an $n$-dimensional vector space, then the map $\ext{n}(V) \to \ext{n}(V)$ is multiplication by $\det A$.
	\end{enumerate}
	
	
	\item Show that the vectors $v_1, \ldots , v_k \in V$ are linearly independent if and only if $v_1 \wedge \ldots \wedge v_k \neq 0$ as an element of $\ext{k}(V)$.
	
	\item We say that an element of $\ext{k}(V)$ is \emph{decomposable} if it can be written as $v_1 \wedge \ldots \wedge v_k$.
	\begin{enumerate}
		\item Suppose $v,w,x,y \in V$. Find necessary and sufficient conditions for $v \wedge w + x \wedge y \in \ext{2}(V)$ to be decomposable.
		
		\item Show that $\omega \in \ext{2}(\R^4)$ is decomposable if and only if $\omega \wedge \omega = 0$.
	\end{enumerate}

	\item Let $V$ be an $n$-dimensional inner product space. We can extend the inner product from $V$ to all of $\ext{}(V)$ by setting the inner product of homogeneous elements of different degrees equal to zero and by letting
	\[
		\langle w_1 \wedge \ldots \wedge w_k, v_1 \wedge \ldots \wedge v_k \rangle = \det \left(\langle w_i, v_j \rangle \right)_{i,j}
	\]
	and extending bilinearly. 
	
	Since $\ext{n}(V)$ is a one-dimensional real vector space, $\ext{n}(V) - \{0\}$ has two components. An \emph{orientation} on $V$ is a choice of component of $\ext{n}(V) - \{0\}$. If $V$ is an oriented inner product space, then there is a linear map $\star: \ext{}(V) \to \ext{}(V)$ called the star map, which is defined by requiring that for any orthonormal basis $e_1, \ldots , e_n$ for $V$,
	\begin{align*}
		\star(1) = \pm e_1 \wedge \ldots \wedge e_n,  & \qquad \star(e_1 \wedge \ldots \wedge e_n) = \pm 1, \\
		\star(e_1 \wedge \ldots \wedge e_k) & = \pm e_{k+1} \wedge \ldots \wedge e_n,
	\end{align*}
	where in each case we take ``$+$'' if $e_1 \wedge \ldots \wedge e_n$ is in the preferred component of $\ext{n}(V)$ and we take ``$-$'' otherwise. Notice that $\star: \ext{k}(V) \to \ext{n-k}(V)$.
	
	\begin{enumerate}
		\item Prove that if $e_1, \ldots , e_n$ is an orthonormal basis for $V$, then the $e_{i_1} \wedge \ldots \wedge e_{i_k}$ with $1 \leq i_1 < \ldots < i_k \leq n$ and $1 \leq k \leq n$ give an orthonormal basis for $\ext{}(V)$.
		\item Prove that, as a map $\ext{k}(V) \to \ext{k}(V)$, $\star \star = (-1)^{k(n-k)}$.
		\item Prove that, for $\omega, \eta \in \ext{k}(V)$, their inner product is given by
		\[
			\langle \omega, \eta \rangle = \star (\omega \wedge \star \eta) = \star(\eta \wedge \star \omega).
		\]
	\end{enumerate}
	
	\item Let $M^n$ be a closed manifold (i.e., a compact manifold without boundary) and let $\omega \in \Omega^1(M)$ so that $\omega_p \neq 0$ for all $p \in M$ (i.e., for all $p$, there exists $v \in T_pM$ so that $\omega_p(v) \neq 0$). Show that $\omega$ is not exact.
	
		 
\end{enumerate}


	
	


\end{document}
