\documentclass[12pt]{memoir}

\def\nsemestre {I}
\def\nterm {Spring}
\def\nyear {2025}
\def\nprofesor {Clayton Shonkwiler}
\def\nsigla {MATH670}
\def\nsiglahead {Differential Geometry}
\def\nextra {HW2}
\def\nlang {ENG}
\input{../../headerVarillyDiff}

\begin{document}
%\begin{multicols}{2}

\begin{Ej}
    Let $A: V \to W$ be a linear map between vector spaces.
	\begin{enumerate}
		\item Show that the induced map $\exterior^k(V) \to \exterior^k(W)$ is well-defined by
		\[
			v_1 \wedge \ldots \wedge v_k \mapsto Av_1 \wedge \ldots \wedge Av_k
		\]
		(extending linearly to sums).
		
		\item Show that the map $A^*: W^* \to V^*$ defined by $(A^\ast(\eta))(v) := \eta(A(v))$ determines a map $\exterior^k(W^\ast) \to \exterior^k(V^\ast)$.
		
		\item Show that, if $V$ is an $n$-dimensional vector space, then the map $\exterior^n(V) \to \exterior^n(V)$ is multiplication by $\det A$.
	\end{enumerate}
\end{Ej}

\begin{ptcbr}
    To prove well-definedness of a map, it suffices to take two representatives of the same class and see that they map to the same place.
    \begin{enumerate}
        \item Consider then, without loss of generality,
        $$v_1\w v_2\wyw v_k=-(v_2\w v_1\wyw v_k).$$
        This second element we can reinterpret as 
        $$(-v_2)\w v_1\wyw v_k.$$
        Applying $\exterior^k(A)$ to this we get
        $$
        \left\lbrace
        \begin{aligned}
            &Av_1\w Av_2\wyw Av_k\\
            &A(-v_2)\w Av_1\wyw Av_k
        \end{aligned}
        \right.
        $$
        and using the fact that $A$ is linear we get 
        \begin{align*}
        A(-v_2)\w Av_1\wyw Av_k&=-(Av_2\w Av_1\wyw Av_k)\\
        &=Av_1\w Av_2\wyw Av_k
        \end{align*}
        and this is the desired representation of the image. \red{FINISH MULTILINEAR} This allows to see that $\exterior^k(A)$ is well-defined.
        \item The map $A^\ast$ does indeed define a map from the exterior powers, namely $\exterior^k(A^\ast)$. \red{FINISH}
        \item \red{NO IDEA}
    \end{enumerate}
\end{ptcbr}

\begin{Ej}
    Show that the vectors $v_1, \ldots , v_k \in V$ are linearly independent if and only if $v_1 \wedge \ldots \wedge v_k \neq 0$ as an element of $\exterior^k(V)$.
\end{Ej}

\begin{ptcbr}
    Assume that $\set{v_1,\dots,v_k}$ is linearly dependent, then if $\set{v_1,\dots,v_\l}$ is a maximally independent set, we may write any $v_i$ with $\l<i\leq k$ as a linear combination of $\set{v_1,\dots,v_l}$.\par
    This means that 
    \begin{align*}
        v_1\wyw v_k&=v_1\wyw v_{\l+1}\wyw v_k\\
        &=v_1\wyw \sum_{i=1}^kc_iv_i\wyw v_k\\
        &=\sum_{i=1}^kc_i(v_1\wyw v_i\wyw v_k)
    \end{align*}
    and all the summands will be zero as we will find repeated $v_i's$ in each term.
\end{ptcbr}

\begin{Ej}
    We say that an element of $\exterior^k(V)$ is \emph{decomposable} if it can be written as $v_1 \wedge \ldots \wedge v_k$.
	\begin{enumerate}
		\item Suppose $v,w,x,y \in V$. Find necessary and sufficient conditions for $v \wedge w + x \wedge y \in \exterior^2(V)$ to be decomposable.
		
		\item Show that $\omega \in \exterior^{2}(\bR^4)$ is decomposable if and only if $\omega \wedge \omega = 0$.
	\end{enumerate}
\end{Ej}

\begin{Ej}
    Let $V$ be an $n$-dimensional inner product space. We can extend the inner product from $V$ to all of $\exterior(V)$ by setting the inner product of homogeneous elements of different degrees equal to zero and by letting
	\[
		\langle w_1 \wedge \ldots \wedge w_k, v_1 \wedge \ldots \wedge v_k \rangle = \det \left(\langle w_i, v_j \rangle \right)_{i,j}
	\]
	and extending bilinearly. 
	
	Since $\exterior^n(V)$ is a one-dimensional real vector space, $\exterior^n(V) - \{0\}$ has two components. An \emph{orientation} on $V$ is a choice of component of $\exterior^n(V) - \{0\}$. If $V$ is an oriented inner product space, then there is a linear map $\star: \exterior(V) \to \exterior(V)$ called the star map, which is defined by requiring that for any orthonormal basis $e_1, \ldots , e_n$ for $V$,
	\begin{align*}
		\star(1) = \pm e_1 \wedge \ldots \wedge e_n,  & \qquad \star(e_1 \wedge \ldots \wedge e_n) = \pm 1, \\
		\star(e_1 \wedge \ldots \wedge e_k) & = \pm e_{k+1} \wedge \ldots \wedge e_n,
	\end{align*}
	where in each case we take ``$+$'' if $e_1 \wedge \ldots \wedge e_n$ is in the preferred component of $\exterior^n(V)$ and we take ``$-$'' otherwise. Notice that $\star: \exterior^k(V) \to \exterior^{n-k}(V)$.
	
	\begin{enumerate}
		\item Prove that if $e_1, \ldots , e_n$ is an orthonormal basis for $V$, then the $e_{i_1} \wedge \ldots \wedge e_{i_k}$ with $1 \leq i_1 < \ldots < i_k \leq n$ and $1 \leq k \leq n$ give an orthonormal basis for $\exterior(V)$.
		\item Prove that, as a map $\exterior^k(V) \to \exterior^k(V)$, $\star \star = (-1)^{k(n-k)}$.
		\item Prove that, for $\omega, \eta \in \exterior^k(V)$, their inner product is given by
		\[
			\langle \omega, \eta \rangle = \star (\omega \wedge \star \eta) = \star(\eta \wedge \star \omega).
		\]
	\end{enumerate}
\end{Ej}

\begin{Ej}
    Let $M^n$ be a closed manifold (i.e., a compact manifold without boundary) and let $\omega \in \Omega^1(M)$ so that $\omega_p \neq 0$ for all $p \in M$ (i.e., for all $p$, there exists $v \in T_pM$ so that $\omega_p(v) \neq 0$). Show that $\omega$ is not exact.
\end{Ej}
\end{document} 