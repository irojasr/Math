\documentclass[12pt]{memoir}

\def\nsemestre {I}
\def\nterm {Spring}
\def\nyear {2025}
\def\nprofesor {Clayton Shonkwiler}
\def\nsigla {MATH670}
\def\nsiglahead {Differential Geometry}
\def\nextra {P}
\def\nlang {ENG}
\def\ntrim{}
\input{../../headerVarillyDiff}
\title{Euler Characteristics of Toric Varieties via Localization}
\author{Ignacio Rojas}
\date{Spring, 2025}
\begin{document}
\bgroup
\renewcommand\thesection{\arabic{section}}
\renewcommand{\thefigure}{\arabic{figure}}
\maketitle

\begin{abstract}
    The Euler characteristic is an invariant of manifolds which can be computed as the alternating sum of its Betti numbers. In this project, we approach this calculation by integrating the manifold's Euler class. Atiyah-Bott localization will help us to refine the process.\par
    Our varieties come equipped with a torus action so we would like a cohomology which remembers this structure. This leads to equivariant cohomology, and in our cases, there will loci of our varieties which will remain fixed. Through this analysis, we will achieve our objective to demonstrate that the Euler characteristic of toric varieties depends solely on the number of torus-fixed points they contain.
    \end{abstract}
    \smallskip
    \begin{flushleft}
        \small
        \emph{Keywords}: Euler characteristic, Euler class, Betti numbers, toric variety, fixed loci, equivariant cohomology, Atiyah-Bott localization.
       \emph{MSC classes}:  Primary \texttt{57S12}; Secondary \texttt{14F43,55N91}.
    \end{flushleft}
    \section{Premier}
    
    This project arises from my interest in localization techniques and equivariant cohomology, particularly in relation to my research on the moduli space of stable maps. Developing a deeper intuition for these concepts through concrete examples will be valuable for my broader studies.
    
    The structure of this project is as follows:
    \begin{itemize}
        \item Define the Euler characteristic and realize it as the integral of the Euler class of a manifold.
        \item Introduce equivariant cohomology and the Atiyah-Bott localization theorem.
        \item Apply this theorem to compute the Euler characteristic of toric varieties, including $\mathbb{P}^n$, $\mathbb{P}^1 \times \mathbb{P}^1$, and $\text{Hilb}^n(\mathbb{C}^2)$.
    \end{itemize}
    
    This project aligns with the course by offering an alternative perspective on manifolds, by viewing group actions as another part of their study. Through this approach, we gain a new way to calculate invariants and insight into algebraic geometry.
    
\section{Manifolds and Euler characteristic}

%https://en.wikipedia.org/wiki/Chern–Gauss–Bonnet_theorem

\begin{Def}
For a manifold $M$, call its $i^{\text{th}}$ \term{Betti number}
$$b_i=\dim H_i(M),$$
the rank of $M$'s $i^{\text{th}}$ homology group.
The \term{Euler characteristic} of the manifold $M$ is defined as 
$$\chi(M)=\sum_{i=0}^\infty(-1)^ib_i.$$
\end{Def}

Observe that this definition generalizes the usual definition of Euler characteristic for graphs:

\begin{enumerate}
    \item Graphs as CW complex
    \item Chern Gauss Bonnet for euler char of tangent bundle, talk a bit about tangent bundle.
    \item Prove CGB
\end{enumerate}


%%%%%%%%%%%% Contents end %%%%%%%%%%%%%%%%
\ifx\nextra\undefined
\printindex
\else\fi
\nocite{*}
\bibliographystyle{plain}
\bibliography{bibiProyGeoDiff.bib}
\end{document}