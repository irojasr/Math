\documentclass[12pt]{memoir}

\def\nsemestre {I}
\def\nterm {Spring}
\def\nyear {2025}
\def\nprofesor {Clayton Shonkwiler}
\def\nsigla {MATH670}
\def\nsiglahead {Differential Geometry}
\def\nextra {P}
\def\nlang {ENG}
\def\ntrim{}
\input{../../headerVarillyDiff}
\title{Euler Characteristics of Toric Varieties via Localization}
\author{Ignacio Rojas}
\date{Spring, 2025}
\begin{document}
\bgroup
\renewcommand\thesection{\arabic{section}}
\renewcommand{\thefigure}{\arabic{figure}}
\maketitle

\begin{abstract}
    The Euler characteristic is an invariant of manifolds which can be computed as the alternating sum of its Betti numbers. In this project, we approach this calculation by integrating the manifold's Euler class. Atiyah-Bott localization will help us to refine the process.\par
    Our varieties come equipped with a torus action so we would like a cohomology which remembers this structure. This leads to equivariant cohomology, and in our cases, there will loci of our varieties which will remain fixed. Through this analysis, we will achieve our objective to demonstrate that the Euler characteristic of toric varieties depends solely on the number of torus-fixed points they contain.
    \end{abstract}
    \smallskip
    \begin{flushleft}
        \small
        \emph{Keywords}: Euler characteristic, Euler class, Betti numbers, toric variety, fixed loci, equivariant cohomology, Atiyah-Bott localization.
       \emph{MSC classes}:  Primary \texttt{57S12}; Secondary \texttt{14F43,55N91}.
    \end{flushleft}
    \section{Premier}
    
    This project arises from my interest in localization techniques and equivariant cohomology, particularly in relation to my research on the moduli space of stable maps. Developing a deeper intuition for these concepts through concrete examples will be valuable for my broader studies.
    
    The structure of this project is as follows:
    \begin{itemize}
        \item Define the Euler characteristic and realize it as the integral of the Euler class of a manifold.
        \item Introduce equivariant cohomology and the Atiyah-Bott localization theorem.
        \item Apply this theorem to compute the Euler characteristic of toric varieties, including $\mathbb{P}^n$, $\mathbb{P}^1 \times \mathbb{P}^1$, and $\text{Hilb}^n(\mathbb{C}^2)$.
    \end{itemize}
    
    This project aligns with the course by offering an alternative perspective on manifolds, by viewing group actions as another part of their study. Through this approach, we gain a new way to calculate invariants and insight into algebraic geometry.
    
\section{Manifolds and Euler characteristic}

%https://en.wikipedia.org/wiki/Chern–Gauss–Bonnet_theorem

\begin{Def}
For a manifold $M$, call its $i^{\text{th}}$ \term{Betti number}
$$b_i=\dim H_i(M),$$
the rank of $M$'s $i^{\text{th}}$ homology group.
The \term{Euler characteristic} of the manifold $M$ is defined as 
$$\chi(M)=\sum_{i=0}^\infty(-1)^ib_i.$$
\end{Def}

Observe that this definition generalizes the usual definition of Euler characteristic for graphs:

\begin{Ex}
    Consider a planar graph $G$. We may construct a 2-dimensional CW complex by taking:
    \begin{itemize}
        \item 0-cells as vertices,
        \item 1-cells as edges, and
        \item 2-cells as faces. We must also consider the \emph{exterior face to the graph}.
    \end{itemize} 
    In this case we have that 
    $$b_0=|V|,\quad b_1=|E|,\quad b_2=|F|,\word{and} b_i=0,\ i\geq 3.$$
    Adding up the Betti numbers as in the characteristic computation we obtain 
    $$\chi(G)=|V|-|E|+|F|$$
    which corresponds to Euler's polyhedron formula. This quantity is $2$ and aligns with $\chi(S^2)=2$ as homology is homotopy-invariant.
\end{Ex}

%% https://web.archive.org/web/20100524152105/http://www.math.upenn.edu/~alina/GaussBonnetFormula.pdf
% https://mathoverflow.net/questions/84521/on-the-generalized-gauss-bonnet-theorem
% https://mathoverflow.net/questions/73450/top-chern-class-euler-characteristic EXXPPPLLL
% https://www.maths.ed.ac.uk/~v1ranick/papers/li4.pdf T4.1.9

Another way to compute the Euler characteristic is via Chern's generalization of the Gauss-Bonnet theorem which is the main tool we intend to use in this exploration.

\begin{Th}
Suppose $M$ is a compact and oriented manifold without boundary of real dimension $2n$. Then 
$$\int_Me(TM)=\chi(M),$$
where $TM$ is the tangent bundle of $M$ and $e(TM)\in H^{2n}(M)$ is its Euler class.
\end{Th}

Chern's original proof goes along the following lines:
\begin{itemize}
    \item First show that $\pi^\ast(e(TM))$ is an exact form. The map $\pi$ is the projection $\pi\: TM\to M$. Then there is a form $\vf\in H^{2n-1}(TM)$ such that 
    $$d\vf=\pi^\ast(e(TM)).$$
    %%% https://math.stackexchange.com/questions/4845049/vector-fields-as-a-section-of-the-tangent-bundle
    \item Then is $X$ is a vector field (a section of the tangent bundle) on $M$, it has only isolated zeroes and singularities. If $S\subseteq M$ is its set of singularities we may further realize as a section
    $$X\:M\less S\to TM.$$
    Chern proved that $\del X(M\less S)\in H_{2n-1}(TM)$.
    \item Then the integral of the Euler class can be manipulated into
    $$\int_Me(TM)=\int_{M\less S}X^\ast(d\vf)=\int_{X(M\less S)}\dd\vf=\int_{\del X(M\less S)}\vf$$
    where Stokes is applied in the last step.
    \item Finally, this last integral can be realized as the sum of indices of $X$, which by Poincaré-Hopf is precisely the Euler characteristic.
\end{itemize}

\subsection{Really quickly: The Euler class}

Vector bundles $E\xrightarrow[]{\pi}B$ carry certain information through their Chern classes. These are elements in $A^i(B)$, the $i^{\text{th}}$ Chow group of $B$, which we may interpret via
$$A^i(B)\to H_{2n-2i}(B)\to H^{2i}(B)$$
where the first map takes cycles to cycles and then we're applying Poincaré duality.

\begin{Def}
    For a vector bundle $E\xrightarrow[]{\pi}B$ of rank $r\geq 1$, its \term{Euler class} is
    $$e(E)\defeq c_r(E)=\bonj{\div(s)},$$
    the class of a divisor of a section. 
\end{Def}

\begin{Rmk}
In other words, we may think of the Euler class as the top Chern class of a vector bundle.
\end{Rmk}


\begin{enumerate}
    \item Comment that Chern classes exist as cohomology elements of $B$ in $E\to B$. 
    \item Define euelr class as top chern class.
\end{enumerate}


%%%%%%%%%%%% Contents end %%%%%%%%%%%%%%%%
\ifx\nextra\undefined
\printindex
\else\fi
\nocite{*}
\bibliographystyle{plain}
\bibliography{bibiProyGeoDiff.bib}
\end{document}