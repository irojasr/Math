\documentclass[12pt]{memoir}

\def\nsemestre {I}
\def\nterm {Spring}
\def\nyear {2025}
\def\nprofesor {Clayton Shonkwiler}
\def\nsigla {MATH670}
\def\nsiglahead {Differential Geometry}
\def\nextra {P}
\def\nlang {ENG}
\def\ntrim{}
\input{../../headerVarillyDiff}
\title{Euler Characteristics of Toric Varieties via Localization}
\author{Ignacio Rojas}
\date{Spring, 2025}
\begin{document}
\bgroup
\renewcommand\thesection{\arabic{section}}
\renewcommand{\thefigure}{\arabic{figure}}
\maketitle

\begin{abstract}
    The Euler characteristic is an invariant of manifolds which can be computed as the alternating sum of its Betti numbers. In this project, we approach this calculation by integrating the manifold's Euler class. Atiyah-Bott localization will help us to refine the process.\par
    Our varieties come equipped with a torus action so we would like a cohomology which remembers this structure. This leads to equivariant cohomology, and in our cases, there will loci of our varieties which will remain fixed. Through this analysis, we will achieve our objective to demonstrate that the Euler characteristic of toric varieties depends solely on the number of torus-fixed points they contain.
    \end{abstract}
    \smallskip
    \begin{flushleft}
        \small
        \emph{Keywords}: Euler characteristic, Euler class, Betti numbers, toric variety, fixed loci, equivariant cohomology, Atiyah-Bott localization.
       \emph{MSC classes}:  Primary \texttt{57S12}; Secondary \texttt{14F43,55N91}.
    \end{flushleft}
    \section{Premier}
    
    This project arises from my interest in localization techniques and equivariant cohomology, particularly in relation to my research on the moduli space of stable maps. Developing a deeper intuition for these concepts through concrete examples will be valuable for my broader studies.
    
    The structure of this project is as follows:
    \begin{itemize}
        \item Define the Euler characteristic and realize it as the integral of the Euler class of a manifold.
        \item Introduce equivariant cohomology and the Atiyah-Bott localization theorem.
        \item Apply this theorem to compute the Euler characteristic of toric varieties, including $\mathbb{P}^n$, $\mathbb{P}^1 \times \mathbb{P}^1$, and $\text{Hilb}^n(\mathbb{C}^2)$.
    \end{itemize}
    
    This project aligns with the course by offering an alternative perspective on manifolds, by viewing group actions as another part of their study. Through this approach, we gain a new way to calculate invariants and insight into algebraic geometry.
    
\section{Manifolds and Euler characteristic}

%https://en.wikipedia.org/wiki/Chern–Gauss–Bonnet_theorem

\begin{Def}
For a manifold $M$, call its $i^{\text{th}}$ \term{Betti number}
$$b_i=\dim H_i(M),$$
the rank of $M$'s $i^{\text{th}}$ homology group.
The \term{Euler characteristic} of the manifold $M$ is defined as 
$$\chi(M)=\sum_{i=0}^\infty(-1)^ib_i.$$
\end{Def}

Observe that this definition generalizes the usual definition of Euler characteristic for graphs:

\begin{Ex}
    Consider a planar graph $G$. We may construct a 2-dimensional CW complex by taking:
    \begin{itemize}
        \item 0-cells as vertices,
        \item 1-cells as edges, and
        \item 2-cells as faces. We must also consider the \emph{exterior face to the graph}.
    \end{itemize} 
    In this case we have that 
    $$b_0=|V|,\quad b_1=|E|,\quad b_2=|F|,\word{and} b_i=0,\ i\geq 3.$$
    Adding up the Betti numbers as in the characteristic computation we obtain 
    $$\chi(G)=|V|-|E|+|F|$$
    which corresponds to Euler's polyhedron formula. This quantity is $2$ and aligns with $\chi(S^2)=2$ as homology is homotopy-invariant.
\end{Ex}

%% https://web.archive.org/web/20100524152105/http://www.math.upenn.edu/~alina/GaussBonnetFormula.pdf
% https://mathoverflow.net/questions/84521/on-the-generalized-gauss-bonnet-theorem
% https://mathoverflow.net/questions/73450/top-chern-class-euler-characteristic EXXPPPLLL
% https://www.maths.ed.ac.uk/~v1ranick/papers/li4.pdf T4.1.9

Another way to compute the Euler characteristic is via Chern's generalization of the Gauss-Bonnet theorem which is the main tool we intend to use in this exploration.

\begin{Th}
Suppose $M$ is a compact and oriented manifold without boundary of real dimension $2n$. Then 
$$\int_Me(TM)=\chi(M),$$
where $TM$ is the tangent bundle of $M$ and $e(TM)\in H^{2n}(M)$ is its Euler class.
\end{Th}

Chern's original proof goes along the following lines:
\begin{itemize}
    \item First show that $\pi^\ast(e(TM))$ is an exact form. The map $\pi$ is the projection $\pi\: TM\to M$. Then there is a form $\vf\in H^{2n-1}(TM)$ such that 
    $$d\vf=\pi^\ast(e(TM)).$$
    %%% https://math.stackexchange.com/questions/4845049/vector-fields-as-a-section-of-the-tangent-bundle
    \item Then is $X$ is a vector field (a section of the tangent bundle) on $M$, it has only isolated zeroes and singularities. If $S\subseteq M$ is its set of singularities we may further realize as a section
    $$X\:M\less S\to TM.$$
    Chern proved that $\del X(M\less S)\in H_{2n-1}(TM)$.
    \item Then the integral of the Euler class can be manipulated into
    $$\int_Me(TM)=\int_{M\less S}X^\ast(d\vf)=\int_{X(M\less S)}\dd\vf=\int_{\del X(M\less S)}\vf$$
    where Stokes is applied in the last step.
    \item Finally, this last integral can be realized as the sum of indices of $X$, which by Poincaré-Hopf is precisely the Euler characteristic.
\end{itemize}

\subsection{Really quickly: Undefined to defined}

Vector bundles $E\xrightarrow[]{\pi}B$ carry certain information through their Chern classes. These are elements in $A^i(B)$, the $i^{\text{th}}$ Chow group of $B$, which we may interpret via
$$A^i(B)\to H_{2n-2i}(B)\to H^{2i}(B)$$
where the first map takes cycles to cycles and then we're applying Poincaré duality.

\begin{Def}
    For a vector bundle $E\xrightarrow[]{\pi}B$ of rank $r\geq 1$, its \term{Euler class} is
    $$e(E)\defeq c_r(E)=\bonj{\div(s)},$$
    the class of a divisor of a section. 
\end{Def}

\begin{Rmk}
In other words, we may think of the Euler class as the top Chern class of a vector bundle.
\end{Rmk}

\begin{Ex}
    The tangent bundle to a manifold carries an Euler class which is the class of a divisor of a section. In the case of the tangent bundle, a section is a vector field! So in essence, finding the value of $\int_Me(TM)$ becomes a matter of counting the zeroes and poles of a vector field over $M$.
\end{Ex}

\begin{Ex}
    Consider the vector field $F$ over the sphere $S^2$ given by 
    $$F(x,y,z)=(zx,zy,1-z^2).$$
    This is a section of $TS^2$ and 
    $$\bonj{\div(F)}=\bonj{\kk}+\bonj{-\kk}$$
    corresponds to the zeroes of this vector field. This means that 
    $$e(TS^2)=2\bonj{\text{pt.}}$$
    and so 
    $$\chi(S^2)=\int_{S^2}2\bonj{\text{pt.}}=2$$
    which coincides with our established notion.
\end{Ex}

\begin{Rmk}
    For more examples on vector field ideas, say for the torus, check out \href{https://mathoverflow.net/questions/153961/constructing-a-vector-field-with-given-zeros-on-a-torus}{this math.of post} or \href{https://math.stackexchange.com/questions/213901/vector-fields-on-torus}{this math.se post} or \href{https://math.stackexchange.com/questions/3604214/non-vanishing-vector-fields-on-the-2-torus}{this other one math.se}. Also check \href{https://mathoverflow.net/questions/97449/computing-the-euler-characteristic-of-the-complex-projective-plane-using-differe}{this math.ov for projective plane}.
\end{Rmk}

My reference for the definition of the integral comes form Fulton and Pandharipande \cite{FPNotes}.

\begin{Def}[\cite{FPNotes} pg. 2]
    For a complete variety, $c\in A^\ast(X)$ and $\bt\in A_k(X)$ then 
    $$\int_\bt c=\deg(c_k,\bt)$$
    where $c_k$ is the component of $c$ in $A^k(X)$ and $(c_k,\bt)$ is the evaluation of $c_k$ on $\bt$ giving us a zero cycle. When $V$ is a closed, pure-dimensional subvariety of $X$, then we write 
    $$\int_V c\word{instead of}\int_{\bonj{V}}c.$$
\end{Def}

It is part of my goal to concile this definition and the algebraic definition of the Euler class with our differential-geometric points of view.

\subsection{More differential-geometric}
%%https://www.hiroleetanaka.com/pdfs/2014-fall-230a-lecture-26-gauss-bonnet-chern.pdf
%%Ver refs en https://ncatlab.org/nlab/show/Euler+class
We must take a small detour into concepts which we haven't defined yet in class.

\subsubsection{Definitions, definitions, definitions\dots}

The concept of an affine connection arises from the necessity of defining a derivative of a vector field $X$ in the \emph{direction of another vector field} $Y$.

\begin{Def}
    An \term{affine connection} over a manifold $M$ is bilinear map 
    $$\nbf\:\gX(M)^2\to\gX(M),\quad (X,Y)\mapsto \nbf_XY$$
    which is $\Coo(M)$-linear in $X$ and satisfies a Leibniz rule in $Y$:
    \begin{gather*}
        \nbf_{fX}Y=f\nbf_XY\\
        \nbf_{X}(fY)=(Xf)Y+f\nb_XY\\
    \end{gather*}
    for any $f\in\Coo(M)$.
\end{Def}

Now on the other hand, topologists have the notion of distance and metric too intertwined. For our purposes, these are subtly different.

\begin{Def}
    A \term{Riemmannian manifold} is a pair $(M,g)$ formed by a manifold $M$ and a \term{Riemmannian metric} $g$ over it. This is symmetric and positive-definite map:
    $$g\:\gX(M)^2\to\Coo(M).$$
\end{Def}

\begin{Rmk}
    In essence, $g_p$ is a real inner product over $T_pM$ depending smoothly on the point $p$.    
\end{Rmk}

These two concepts come hand-in-hand, we say that\dots

\begin{Def}
    An affine connection $\nbf$ is \un{compatible with the metric $g$} if its parallel transport is \emph{isometric}. This means that if $\ga$ is a smooth curve and $X,Y$ are \emph{parallel} over $\ga$, then 
    $$g(X,Y)(\ga(t))=g_{\ga(t)}(X_{\ga(t)},Y_{\ga(t)})$$
    is a constant function of $t$.
\end{Def}

Finally, before defining an important tool we will just say that\dots

\begin{Def}
    An affine connection $\nbf$ is \term{torsion-free} if 
    $$\nbf_XY-\nbf_YX=[X,Y].$$
    We call $T_\nbf=\nbf_XY-\nbf_YX-[X,Y]$ the torsion of the connection $\nbf$.
\end{Def}

\begin{Th}
    Over a Riemmannian manifold $(M,g)$ there exists a unique affine connection which is 
    \begin{itemize}
        \item compatible with the metric $g$, and
        \item torsion-free.
    \end{itemize}
    Such a connection is called the \term{Levi-Civita connection} of $M$ associated to the metric $g$.
\end{Th}

We're almost there, and we might think first about defining a kind of second derivative for vector fields, something like
$$\nbf_X\nbf_YZ.$$
Its local expression does include second derivatives, however it is \emph{not tensorial}! It's possible to modify this idea in order to have a tensor.

\begin{Def}
    For an affine connection $\nbf$ over $M$, we define the \term{curvature operator} as 
    $$R(X,Y)\defeq \nbf_X\nbf_Y-\nbf_Y\nbf_X-\nbf_{[X,Y]}.$$
\end{Def}

So close now, just one last concept:

\begin{Def}
    For a skew-symmetric matrix $A$ of size $2n$, we can define its \term{Pfaffian} as 
    $$\frac{1}{2^nn!}\sum_{\sg\in S_{2n}}\sgn(\sg)\prod_{k=1}^{n}a_{\sg(2i-1)\sg(2i)}.$$
\end{Def}

\subsection{What we came for\dots}
%Ojo Hiroleetanaka ss 4
The Euler class can be defined alternatively in terms of differential forms. 

\begin{Def}
    For a Riemmannian manifold $(M,g)$, we define the \term{Euler form} as 
    $$e(R)=\frac{1}{(2\pi)^n}\Pf(R)$$
    where $R$ is the associated curvature form of the Levi-Civita connection of $M$. The cohomology class of $e(R)$ is the Euler class.
\end{Def}

\begin{Rmk}
    Let us observe two things:
    \begin{enumerate}
        \item First, by virtue of the Pfaffian and the fact that the curvature form is a $2$-form, the Euler class is a $2n$-form. Which means its a top form.
        \item The cohomology class of $e(R)$ is independent of $\nbf$ or $g$ and its a closed form.
    \end{enumerate}
\end{Rmk}

\begin{Qn}
How is it that my Euler class and this Euler class are related?
\end{Qn}

\section{Equivariant cohomology and localization}

Manifolds usually don't come by themselves, like in the case of homogenous spaces, some manifolds have a lot of symmetries. These can be expressed by a group action on the manifold. We would like a cohomology theory which retains information on the group action!

\begin{Ex}[A na\"ive approach]
    Consider the $S^1$ action $\bCP^1$ given by $u\. z=uz$. This action has two fixed points, $0$ and $\infty$. Observe also that 
    $$u\.z=z\iff u=1.$$
    If we were to define the a cohomology which retains information on the group action (equivariant cohomology), we could say 
    $$H_{S^1}^\ast(\bCP^1)\defeq H^\ast(\quot{\bCP^1}{S^1}).$$
    However the orbit space $\quot{\bCP^1}{S^1}$ is the same as a closed interval which means it has trivial cohomology.
\end{Ex}

Instead of considering the cohomology of the orbit space $M/G$, which doesn't retain information on the group action, we should look for an alternive which does.

\subsection{The Borel construction}

The main idea for this concept is that homotopy equivalent spaces have the same cohomology. Suppose $G$ acts on $M$, let us create a space $EG$, a \emph{classifying space}, with the following properties:

\begin{enumerate}
    \item The right action $EG\. G$ is free. ($\forall x(\Stab(x)=0)$)
    \item $EG$ is contractible. 
    \item There exists a unique $EG$ up to homotopy. ($EG$ satisfies a universal property in a category of $G$-spaces)
\end{enumerate}

This sounds a bit risky to ask, because questions may arise. But let's avoid them for now, instead observe that 
$$M\x EG\isom M$$
as $EG$ is contractible! 

\begin{Def}
    We call the \term{orbit space}\footnote{This is now overloading the previous definition of orbit space $M/G$.} of $M$ the quotient
    $$M_G\defeq\quot{M\x EG}{(g\.x,y)\sim(x,y\.g)}.$$
    From this we define the \term{equivariant cohomology} of $M$ as 
    $$H_G^\ast(M)\defeq H^\ast(M_G).$$
\end{Def}

\begin{Ex}[Cohomology of a point]
We know that the usual cohomology of a point is trivial, but let's check two examples to see what changes.
\begin{enumerate}
    \item First consider the (trivial) action of $\bZ$ on a point. In this case we have
    $$E\bZ=\bR\word{with}x\.n=x+n.$$
    This is a free action and $\bR$ is contractible\footnote{You'll have to trust me on the fact that $\bR$ is unique up to homotopy on this one.}. Find the classifying space isn't very bad:
    $$\pt_\bZ=\quot{\bR}{x\sim x+n}\isom S^1$$
    so that 
    $$H_\bZ^\ast(\pt)=H^\ast(S^1)=\quot{\bZ[t]}{t^2}.$$
    \item Now let's take a bigger group, say $U(1)$, but for our purposes let's call it $T$ as in torus. The classifying space here is 
    $$ET=\bC^\infty\less\set{0},\word{with}\al\.\un{z}=(\al z_i)_i.$$
    The action takes a sequence of complex numbers and scalar-multiplies it by $\al\in T$. 
    %%https://mathoverflow.net/questions/198/how-do-you-show-that-s-infty-is-contractible
    This action is free, and we may see that $\bC^\infty\less\set{0}\isom S^\infty$. The infinite sphere is contractible by arguments out of my scope. And certainly, this classifying space is unique. But now, the quotient in question is 
    $$\pt_T=\quot{\bC^\infty\less\set{0}}{\un{z}\sim \al\un{z}}\isom\bP^\infty.$$
    The cohomology now is 
    $$H^\ast_{T}\pt=H^\ast\bP^\infty=\bC[t].$$
    From this example we can extend the calculation to see that for an $n$-dimensional torus $T^n$ we have 
    $$H^\ast_{T^n}\pt=H^\ast(\bP^\infty)^n=\bC[t_1,\dots,t_n]$$
    by the K\"unneth formula.
\end{enumerate}
\end{Ex}

Questions remain for me such as\dots

\begin{Qn}
    What happens when $G$ is a symmetric group $S_n$, or a finite group $\bZ/n\bZ$? Even more, what if $G$ is a matrix group, or an exceptional group such as the Mathieu group\footnote{At the time of writing, Ignacio hasn't read Classifying Spaces of Sporadic Groups by Benson and Smith.}?  
\end{Qn}

\begin{Rmk}
    One can see that the idea of constructing the cohomology of the orbit space goes haywire as soon as our space is not a point. For $\bP^1$ one has to find
    $$H^\ast\left(\quot{T^2\x\bP^2}{\sim}\right)$$
    which becomes unsurmountably hard.
\end{Rmk}

To solve this issue we ask for help with the\dots

\subsection{Atiyah-Bott localization theorem}

\begin{Th}[Atiyah and Bott, 1984]
    If $G\.M$ is an action and $F_k\subseteq M$ are the fixed loci of the action $G\.F_k=F_k$, then there exists an isomorphism of cohomologies
    $$H^\ast_G(M)\isom \bigoplus_kH^\ast_G(F_k)$$
    where the inclusion maps $i_k\: F_k\into M$ induce the morphisms:
    $$\un{i}^\ast\:H^\ast_G(M)\to\bigoplus_kH^\ast_G(F_k),$$
    component-wise this is the pullback of each $i_k$. And on the other direction it's
 $$\frac{i_\ast}{e(N_{\.\mid M})}\:\oplus H^\ast_G(F_k)\to H^\ast_G(M),$$ 
 where $N_{Y\mid X}$ is the normal bundle $Y\subseteq X$.
\end{Th}

To say that we're using a localization technique to find cohomology is to apply the Atiyah-Bott theorem.

\begin{Ex}[Projective line cohomology via localization]
First, let's clearly define the action of $T^2=(\bC\less\set{0})^2$ on $\bP^1$. For $\un{\al}\in T^2$ and $[X,Y]\in\bP^1$ we have
$$\un{\al}\.[X,Y]\defeq \bonj{\frac{X}{\al_1},\frac{Y}{\al_2}}\footnote{I know this is an unorthodox choice, but it's so that the weights of a certain representation are aligned properly. I'm already too traumatized to do it the \emph{correct} way.}.$$
Then, the only fixed points of this action are $0=[0:1]$ and $\infty=[1:0]$:
$$\un{\al}\.[0:1]=\bonj{0:\frac{1}{\al_2}}=[0:1],\word{and}\un{\al}\.[1:0]=\bonj{\frac{1}{\al_1}:0}=[1:0].$$
Proving that there's no more fixed points amounts to a linear algebra exercise. Applying Atiyah-Bott we now have that 
\begin{gather*}
    H_{T^2}^\ast(\bP^1)\isom H_{T^2}^\ast([0:1])\oplus H_{T^2}^\ast([1:0])\\
    \To\quot{\bC[t_1,t_2,H]}{(H-t_1)(H-t_2)}\footnote{Believe me on this one. This is the actual cohomology.}\isom \bC[t_1,t_2]\oplus\bC[t_1,t_2].
\end{gather*}
But the question is, how does this isomorphism work? It suffices to see where the generators go. On the left, we have the generators $t_1, t_2$ and $H$ representing two hyperplane classes in each copy of $\bP^\infty$ and $H$ which represents the hyperplane class of $\bP^1$ as a bundle over a point. Mapping these classes we get
$$\un i^\ast\left\lbrace
\begin{aligned}
    &t_1\mapsto (t_1,t_1),\\
    &t_2\mapsto (t_2,t_2),\\
    &H\mapsto (t_1,t_2).
\end{aligned}
\right.$$
Whereas the generators on the right are the classes of the points $[0]=(1,0)$ and $[\infty]=(0,1)$.
\end{Ex}
\begin{enumerate}
    \item Terminar AB de P1
    \item Ver ejemplos de variedades toricas con AB
\end{enumerate}
%%%%%%%%%%%% Contents end %%%%%%%%%%%%%%%%
\ifx\nextra\undefined
\printindex
\else\fi
\nocite{*}
\bibliographystyle{plain}
\bibliography{bibiProyGeoDiff.bib}
\end{document}