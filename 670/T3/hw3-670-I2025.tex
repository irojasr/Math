\documentclass[12pt]{memoir}

\def\nsemestre {I}
\def\nterm {Spring}
\def\nyear {2025}
\def\nprofesor {Clayton Shonkwiler}
\def\nsigla {MATH670}
\def\nsiglahead {Differential Geometry}
\def\nextra {HW3}
\def\nlang {ENG}
\input{../../headerVarillyDiff}

\begin{document}
%\begin{multicols}{2}

\begin{Ej}
    Let $G$ be a Lie group. 
	\begin{enumerate}
		\item Show that the set of right-invariant vector fields on $G$ forms a Lie algebra with bracket given by the Lie bracket of vector fields. Note that the right-invariant vector fields form a vector space which is isomorphic to $T_eG$.
		
		\item Let $\inv: G \to G$ be given by $\inv(g) = g^{-1}$. Prove that if $X$ is a left-invariant vector field on $G$, then $\dd \inv(X)$ is a right-invariant vector field whose value at $e$ is $-X_e$.
		
		\item Prove that the map $-d\inv$ from left-invariant vector fields to right-invariant vector fields is a Lie algebra isomorphism. (The point is that we could just have well chosen to interpret the Lie algebra of $G$ as the right-invariant vector fields rather than the left-invariant ones.)
	\end{enumerate}
\end{Ej}


%%%general
%https://math.uchicago.edu/~may/REU2016/REUPapers/Mandel.pdf

%%% a
%MASO https://math.stackexchange.com/questions/2686570/prove-that-the-identification-of-t-e-g-with-the-space-of-right-invariant-vecto
%https://math.stackexchange.com/questions/4803039/question-about-lie-bracket-and-right-invariant-vector-fields
%https://math.stackexchange.com/questions/295758/showing-that-left-invariant-vector-fields-commute-with-right-invariant-vector-fi
%https://math.stackexchange.com/questions/4573185/proving-vector-space-isomorphism-of-left-invariant-vector-fields-and-tangent-spa?rq=1

%%%b
%https://www.reddit.com/r/learnmath/comments/1435v54/about_rightleft_invariant_vector_fields_and_lie/
%https://bohr.physics.berkeley.edu/classes/250/f15/notes/liegroup.pdf
%https://mathoverflow.net/questions/55679/lie-bracket-of-invariant-vector-fields

\iffalse
First let us recall that a left-invariant vector field is an $X\in\cX(G)$ such that 
		$$\dd L_gX=X,\word{i.e.}\left(\dd L_g\right)_h (X(h)) = X(gh).$$ 
		So in a similar fashion a right-invariant vector field should be an $X\in\cX(G)$ such that $\dd R_gX=X$. This is, for $h\in G$
		$$\left(\dd R_g\right)_h (X(h)) = X(hg)$$
		where the action of $R_g$ is 
		$$R_g(h)=hg,\word{for}h\in G.$$
		Call $\g^R$ the set of right-invariant vector fields, it suffices to show that the Lie bracket of two right-invariant vector fields is also right-invariant. To that effect we employ Lemma 3.3.8, we have that 
		$$\dd R_g([X,Y](h))=[X,Y](R_g(h))=[X,Y](hg).$$
\fi

I will allow myself just a quick refresher. If we have a diffeomorphism $F\: M\to N$, then there is a pushforward onto tangent spaces:
$$F_\ast\mid_x\:T_x M\to T_{y}N,\quad y=F(x).$$
This in turn can map vector fields into vector fields via the rule
$$(F_\ast\mid_x)(X\mid_x)=(F_\ast X)\mid_{y}.$$
Recall a left-invariant vector field is $X\in\g$ defined by 
$$\dd L_gX=X,\word{equivalently}\left(\dd L_g\right)_h (X(h)) = X(gh),\quad h\in G.$$ 
In regards to the previous terms this means that
$$(L_g)_\ast X=X,\word{or}((L_g)_\ast\mid_h)(X(h))=((L_g)_\ast X)\mid_{L_g(h)}=X(gh),$$
for $h\in G$. So this means that a right-invariant vector field arises from the right action of $G$ on itself given by $R_g(h)=hg$. The right-invariant vector fields form a set $\g^R$ determined by the rule
$$(R_g)_\ast X=X\iff ((R_g)_\ast\mid_h)(X(h))=X(hg).$$
Restating finally lemma 3.3.8 for $M=N=G$ and $f=R_g$ we have the key ingredient for the first part: 
$$(\dd R_g)_h([X,Y](h))=[X,Y](R_g(h))=[X,Y](hg).$$
\begin{ptcbr}
	\begin{enumerate}
		\item To prove that $\g^R$ is a Lie algebra, it suffices to see that the Lie bracket of vector fields obeys that same right-invariance. Let $g,h\in G$ and take $X^R,Y^R\in\g^R$, we wish to see that 
		$$((R_g)_\ast\mid_h)([X^R,Y^R](h))=[X^R,Y^R](hg).$$
		Lemma 3.3.8 from the notes asserts that this is the case and thus, we have the desired result.
		\item The map $\dd\inv$ is the pushforward of the map $g\mapsto g^{-1}$ in $G$. Assume then that $X\in\g$, applying $\dd\inv$ to our left-invariance relation we have
		\begin{align*}
			&(L_g)_\ast X=X\\
			\To&\inv_\ast(L_g)_\ast X=\inv_\ast X\\
			\To&(R_{g^{-1}})_\ast\inv_\ast X=\inv_\ast X
		\end{align*}
		so that $\inv_\ast X$ is a right-invariant vector field. The implication from second to third line comes from the fact that 
		$$\inv\circ L_g=R_{g^{-1}}\circ\inv$$
		and then pushforwarding from $G$ to $\g$.\par
		I couldn't quite piece the value at the identity. But we have the following:
		\begin{itemize}
			\item Every right-invariant vector field $X^R$'s values at any point $g\in G$ is determined via the formula 
			$$X^R(g)=(\dd R_g)_e(X^R(e)),$$
			i.e. by $X^R$'s value at the identity.
			\item Same business happens for LIVF's: 
			$$X(g)=(\dd L_g)_e(X(e)).$$
		\end{itemize}
		As $\dd \inv X$ is a RIVF, we have 
		$$(\inv_\ast X)(g)=((R_g)_\ast)_e((\inv_\ast X)(e)).$$
		Via our previous relation this is 
		$$((R_g)_\ast)_e((\inv_\ast X)(e))= \inv_\ast ( (L_{g^{-1}})_\ast)_e X(e)=\inv_\ast X(g^{-1}).$$
		Now, consider the diagram
		\begin{center}
			% https://tikzcd.yichuanshen.de/#N4Igdg9gJgpgziAXAbVABwnAlgFyxMJZABgBpiBdUkANwEMAbAVxiRAHEQBfU9TXfIRQBGclVqMWbTjz7Y8BImWHj6zVohAAdLQHNuvEBnmCioldTVTNO-V3Ewou+EVAAzAE4QAtkjIgcCCQAJktJDW0tABksVmoGOgAjGAYABX4FIRAPLF0ACxwDdy9fRH9ApFEJdTYdLDAaEHiklPSTRU0GGDdC2RBPHxDqCsQAZjCamy16mgB9HTo4QubktIzTTRz83sMB0qqR0fsuIA
\begin{tikzcd}
	G \arrow[d, "X"'] \arrow[r, "\inv"] & G \arrow[d] \\
	\g \arrow[r, "\inv_\ast"']             & \g         
	\end{tikzcd}
		\end{center}
		Assuming the diagram commutes, then 
		$$\inv_\ast(X(g))=X(\inv(g))=X(g^{-1}).$$
		This is the only relation I can find between $\inv_\ast X$ and $X$ itself. As others only have either $X$'s or $\inv_\ast X$. Returning this to the relation we get
		$$((R_g)_\ast)_e((\inv_\ast X)(e))=X(g)$$
		as $\inv$ is involutive. Setting $g=e$ doesn't lead me anywhere reasonable, as I can't exactly see where to go from here. \red{Where does the minus sign come from?}
		\item Finally, let us assume the second item. It suffices to see that the Lie bracket is preserved. To that effect, take two vector fields $X,Y$ and observe that 
		$$-\inv_\ast[X,Y]=[X,Y]^R=[X^R,Y^R]=[-\inv_\ast X,-\inv_\ast Y].$$
	\end{enumerate}
\end{ptcbr}
\begin{Ej}
    Consider the special orthogonal group $\SO(3)$ of all $3 \times 3$ matrices $B$ such that 
	\[
		B B^T = I \qquad \text{and} \qquad \det B = 1.
	\]
	We saw in section 3.4 that
	\[
		A_1 = \begin{bmatrix} 0 & 0 & 0 \\ 0 & 0 & -1 \\ 0 & 1 & 0 \end{bmatrix}, A_2 = \begin{bmatrix} 0 & 0 & 1 \\ 0 & 0 & 0 \\ -1 & 0 & 0 \end{bmatrix}, A_3 = \begin{bmatrix} 0 & -1 & 0 \\ 1 & 0 & 0 \\ 0 & 0 & 0 \end{bmatrix}
	\]
	gives a basis for $T_I\SO(3)$ so that $[A_1, A_2]=A_3$, $[A_2,A_3]=A_1$, and $[A_3,A_1]=A_2$. Notice also that $A_3 = \gamma_3'(0)$, where
	\[
		\gamma_3(t) = \begin{bmatrix} \cos t & -\sin t & 0 \\ \sin t & \cos t & 0 \\ 0 & 0 & 1 \end{bmatrix},
	\]
	and there are similar curves whose tangents at the identity give $A_1$ and $A_2$.
	
	Let $V_1$, $V_2$, and $V_3$ be the corresponding left-invariant vector fields on $\SO(3)$; i.e., $V_i(I)=A_i$.
	
	Let $\alpha_i$ be the dual basis of left-invariant 1-forms and compute their exterior derivatives. 
\end{Ej}
%%%%general
%https://mathoverflow.net/questions/81247/what-is-the-structure-of-so3-and-its-lie-algebra
%https://web.williams.edu/Mathematics/it3/texts/so3.pdf
%https://pvangoor.github.io/mathematics/2025/02/15/special_orthogonal_so3.html
\begin{ptcbr}
In my attempt to answer this question I've been reminded of the fact that $\gso(3)'=\gso(3)$. I must admit I don't know exactly how to apply this fact.\par
On the other hand, I would like to follow example 2.7.6 but I'm completely lost on how to do it. Say I have the 1-form $\al_1$ and I want to find
$$\dd\al_1(V_2,V_3)=\dots=\cL_{V_2}\al_1(V_3).$$
But the calculation doesn't go exactly like that example. I need to discuss it.
\end{ptcbr}
\begin{Ej}
    Let $G$ be a compact Lie group and assume $\langle \cdot , \cdot \rangle$ is an $\operatorname{Ad}$-invariant inner product on $\mathfrak{g}$ (an $\operatorname{Ad}$-invariant inner product on $\mathfrak{g}$ is one that satisfies $\langle X,Y \rangle = \langle \operatorname{Ad}_g X, \operatorname{Ad}_g Y \rangle$ for all $g \in G$ and for any $X,Y \in \mathfrak{g}$). 
	
	Define $\tau_e: \mathfrak{g} \times \mathfrak{g} \times \mathfrak{g} \to \bR$ by
	\[
		\tau(X,Y,Z) = \langle [X,Y], Z\rangle.
	\]
	\begin{enumerate}
		\item Show that $\tau_e$ is alternating. Since $\tau_e$ is clearly multilinear, this means $\tau_e$ can be identified with an element of $\bigwedge^3(\mathfrak{g}^*)$.
		\item Extend $\tau_e$ to a left-invariant 3-form on $G$ in the usual way: for each $g \in G$, define $\tau_g := L_{g^{-1}}^* \tau_e$. Prove that $\tau \in \Omega^3(G)$ is bi-invariant (Hint: feel free to use the fact that a left-invariant form is bi-invariant if and only if it is conjugation-invariant). The bi-invariant 3-form $\tau$ is called the \emph{fundamental 3-form} of the Lie group $G$.
		\item Explicitly compute the fundamental 3-form of $\SO(3)$ in terms of the $\alpha_i$ from the previous problem. 
	\end{enumerate}
\end{Ej}

%%%c
%https://math.stackexchange.com/questions/17856/visualizing-the-fundamental-group-of-so3
\begin{ptcbr}
\begin{enumerate}
	\item It's clear that if $X=Y$ then $\tau$ is zero, but when $Z=X$ we have 
	$$\langle[X,Y],X\rangle=\langle\Ad_g[X,Y],\Ad_gX\rangle.$$
	Expanding the adjoint map and using bilinearity we get
	$$\langle gXYg^{-1},gXg^{-1}\rangle-\langle gYXg^{-1},gXg^{-1}\rangle.$$
	Applying the $\Ad$ invariance again doesn't lead anywhere. \red{I can't quite figure out where to proceed.}
\end{enumerate}
I didn't give myself the opportunity to try the other parts and didn't understand how to apply the hint :(
\end{ptcbr}
\end{document} 