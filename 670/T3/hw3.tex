\documentclass[11pt]{article}
\usepackage{fullpage}
\usepackage{amsmath,amsthm, amssymb, amsfonts, amscd}
\usepackage[mathscr]{eucal}
\usepackage{graphicx}
\usepackage{psfrag}
\usepackage[usenames,dvipsnames]{color}
\usepackage{subfigure}

%       Theorem environments

%% \theoremstyle{plain} %% This is the default
\newtheorem{theorem}{Theorem}[section]
\newtheorem{corollary}[theorem]{Corollary}
\newtheorem{lemma}[theorem]{Lemma}
\newtheorem{proposition}[theorem]{Proposition}
\newtheorem{ax}{Axiom}
\newtheorem{conjecture}[theorem]{Conjecture}

\theoremstyle{definition}
\newtheorem{definition}{Definition}[section]

\theoremstyle{definition}
\newtheorem{remark}[theorem]{Remark}


\newcommand{\ext}[1]{%
    {\def\tmp{#1}
    \ifx\tmp\empty
        {\textstyle\bigwedge}
    \else
        {\textstyle\bigwedge\!\!^{#1}}
    \fi}}
%Math Definitions


\newcommand{\Q}{{\mathbb Q}}
\newcommand{\Z}{{\mathbb Z}}
\newcommand{\R}{{\mathbb R}}
\newcommand{\C}{{\mathbb C}}
\newcommand{\N}{{\mathbb N}}

\newcommand{\RP}{{\mathbb{RP}}}
\newcommand{\CP}{{\mathbb{CP}}}



\newcommand{\sgn}{\operatorname{sgn}}
\newcommand{\dx}{\,\mathrm{d}x}
\newcommand{\dy}{\mathrm{d}y}
\newcommand{\dz}{\mathrm{d}z}
\newcommand{\inv}{\operatorname{inv}}
\newcommand{\I}{\mathbf{i}}
\newcommand{\J}{\mathbf{j}}
\newcommand{\K}{\mathbf{k}}
\newcommand{\SO}{\operatorname{SO}}
\newcommand{\dt}{\operatorname{dt}}
\newcommand{\tw}{\operatorname{tw}}
\newcommand{\ds}{\operatorname{ds}}
\newcommand{\st}{\operatorname{st}}
\newcommand{\sech}{\operatorname{sech}}
\newcommand{\range}{\operatorname{range}}
\newcommand{\nul}{\operatorname{null}}
\newcommand{\spa}{\operatorname{span}}
\newcommand{\quat}{\mathbb{H}}
\renewcommand{\Re}{\operatorname{Re}}
\renewcommand{\Im}{\operatorname{Im}}
\newcommand{\from}{\co\!\!}


\newcommand{\Field}{\mathbb{F}}

\usepackage{xcolor}
\definecolor{linkblue}{HTML}{003d73}
\definecolor{linkgreen}{HTML}{006161}
\definecolor{linkred}{HTML}{a11950}
\usepackage{hyperref}
\hypersetup{
	pdftitle={Math 670 HW \#3},
	pdfauthor={Clayton Shonkwiler},
	pdfsubject={differential geometry},
	pdfkeywords={differential geometry, homework, Math 670},
	colorlinks=true,
	linkcolor=linkblue,
	citecolor=linkgreen,
	urlcolor=linkred
}
% \pagestyle{empty}
\def\co{\colon}

\begin{document}
\begin{center}
{\Large\textbf{Math 670 HW \#3}}\\
Due 11:59 PM Friday, April 18
\end{center}




\begin{enumerate}	
	
	\item Let $G$ be a Lie group. 
	\begin{enumerate}
		\item Show that the set of right-invariant vector fields on $G$ forms a Lie algebra with bracket given by the Lie bracket of vector fields. Note that the right-invariant vector fields form a vector space which is isomorphic to $T_eG$.
		
		\item Let $\inv: G \to G$ be given by $\inv(g) = g^{-1}$. Prove that if $X$ is a left-invariant vector field on $G$, then $d\inv(X)$ is a right-invariant vector field whose value at $e$ is $-X_e$.
		
		\item Prove that the map $-d\inv$ from left-invariant vector fields to right-invariant vector fields is a Lie algebra isomorphism. (The point is that we could just have well chosen to interpret the Lie algebra of $G$ as the right-invariant vector fields rather than the left-invariant ones.)
	\end{enumerate}
	
	\item Consider the special orthogonal group $\SO(3)$ of all $3 \times 3$ matrices $B$ such that 
	\[
		B B^T = I \qquad \text{and} \qquad \det B = 1.
	\]
	We saw in section 3.4 that
	\[
		A_1 = \begin{bmatrix} 0 & 0 & 0 \\ 0 & 0 & -1 \\ 0 & 1 & 0 \end{bmatrix}, A_2 = \begin{bmatrix} 0 & 0 & 1 \\ 0 & 0 & 0 \\ -1 & 0 & 0 \end{bmatrix}, A_1 = \begin{bmatrix} 0 & -1 & 0 \\ 1 & 0 & 0 \\ 0 & 0 & 0 \end{bmatrix}
	\]
	gives a basis for $T_I\SO(3)$ so that $[A_1, A_2]=A_3$, $[A_2,A_3]=A_1$, and $[A_3,A_1]=A_2$. Notice also that $A_3 = \gamma_3'(0)$, where
	\[
		\gamma_3(t) = \begin{bmatrix} \cos t & -\sin t & 0 \\ \sin t & \cos t & 0 \\ 0 & 0 & 1 \end{bmatrix},
	\]
	and there are similar curves whose tangents at the identity give $A_1$ and $A_2$.
	
	Let $V_1$, $V_2$, and $V_3$ be the corresponding left-invariant vector fields on $\SO(3)$; i.e., $V_i(I)=A_i$.
	
	Let $\alpha_i$ be the dual basis of left-invariant 1-forms and compute their exterior derivatives. 
	
	
	\item Let $G$ be a compact Lie group and assume $\langle \cdot , \cdot \rangle$ is an $\operatorname{Ad}$-invariant inner product on $\mathfrak{g}$ (an $\operatorname{Ad}$-invariant inner product on $\mathfrak{g}$ is one that satisfies $\langle X,Y \rangle = \langle \operatorname{Ad}_g X, \operatorname{Ad}_g Y \rangle$ for all $g \in G$ and for any $X,Y \in \mathfrak{g}$). 
	
	Define $\tau_e: \mathfrak{g} \times \mathfrak{g} \times \mathfrak{g} \to \R$ by
	\[
		\tau(X,Y,Z) = \langle [X,Y], Z\rangle.
	\]
	\begin{enumerate}
		\item Show that $\tau_e$ is alternating. Since $\tau_e$ is clearly multilinear, this means $\tau_e$ can be identified with an element of $\bigwedge^3(\mathfrak{g}^*)$.
		\item Extend $\tau_e$ to a left-invariant 3-form on $G$ in the usual way: for each $g \in G$, define $\tau_g := L_{g^{-1}}^* \tau_e$. Prove that $\tau \in \Omega^3(G)$ is bi-invariant (Hint: feel free to use the fact that a left-invariant form is bi-invariant if and only if it is conjugation-invariant). The bi-invariant 3-form $\tau$ is called the \emph{fundamental 3-form} of the Lie group $G$.
		\item Explicitly compute the fundamental 3-form of $\SO(3)$ in terms of the $\alpha_i$ from the previous problem. 
	\end{enumerate}
		
		 
\end{enumerate}


	
	


\end{document}
