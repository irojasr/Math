\documentclass[11pt]{article}
\usepackage{fullpage}
\usepackage{amsmath,amsthm, amssymb, amsfonts, amscd}
\usepackage[mathscr]{eucal}
\usepackage{graphicx}
\usepackage{psfrag}
\usepackage[usenames,dvipsnames]{color}
\usepackage{subfigure}

%       Theorem environments

%% \theoremstyle{plain} %% This is the default
\newtheorem{theorem}{Theorem}[section]
\newtheorem{corollary}[theorem]{Corollary}
\newtheorem{lemma}[theorem]{Lemma}
\newtheorem{proposition}[theorem]{Proposition}
\newtheorem{ax}{Axiom}
\newtheorem{conjecture}[theorem]{Conjecture}

\theoremstyle{definition}
\newtheorem{definition}{Definition}[section]

\theoremstyle{definition}
\newtheorem{remark}[theorem]{Remark}


\newcommand{\ext}[1]{%
    {\def\tmp{#1}
    \ifx\tmp\empty
        {\textstyle\bigwedge}
    \else
        {\textstyle\bigwedge\!\!^{#1}}
    \fi}}
%Math Definitions


\newcommand{\Q}{{\mathbb Q}}
\newcommand{\Z}{{\mathbb Z}}
\newcommand{\R}{{\mathbb R}}
\newcommand{\C}{{\mathbb C}}
\newcommand{\N}{{\mathbb N}}

\newcommand{\RP}{{\mathbb{RP}}}
\newcommand{\CP}{{\mathbb{CP}}}



\newcommand{\sgn}{\operatorname{sgn}}
\newcommand{\dx}{\,\mathrm{d}x}
\newcommand{\dy}{\mathrm{d}y}
\newcommand{\dz}{\mathrm{d}z}
\newcommand{\inv}{\operatorname{inv}}
\newcommand{\I}{\mathbf{i}}
\newcommand{\J}{\mathbf{j}}
\newcommand{\K}{\mathbf{k}}
\newcommand{\SO}{\operatorname{SO}}
\newcommand{\dt}{\operatorname{dt}}
\newcommand{\tw}{\operatorname{tw}}
\newcommand{\ds}{\operatorname{ds}}
\newcommand{\st}{\operatorname{st}}
\newcommand{\sech}{\operatorname{sech}}
\newcommand{\range}{\operatorname{range}}
\newcommand{\nul}{\operatorname{null}}
\newcommand{\spa}{\operatorname{span}}
\newcommand{\quat}{\mathbb{H}}
\renewcommand{\Re}{\operatorname{Re}}
\renewcommand{\Im}{\operatorname{Im}}
\newcommand{\from}{\co\!\!}
\newcommand{\Fl}{\operatorname{F\ell}}
\newcommand{\orthog}{\operatorname{O}}


\newcommand{\Field}{\mathbb{F}}

\usepackage{xcolor}
\definecolor{linkblue}{HTML}{003d73}
\definecolor{linkgreen}{HTML}{006161}
\definecolor{linkred}{HTML}{a11950}
\usepackage{hyperref}
\hypersetup{
	pdftitle={Math 670 HW \#4},
	pdfauthor={Clayton Shonkwiler},
	pdfsubject={differential geometry},
	pdfkeywords={differential geometry, homework, Math 670},
	colorlinks=true,
	linkcolor=linkblue,
	citecolor=linkgreen,
	urlcolor=linkred
}
% \pagestyle{empty}
\def\co{\colon}

\begin{document}
\begin{center}
{\Large\textbf{Math 670 HW \#4}}\\
Due 11:59 PM Friday, May 2
\end{center}




\begin{enumerate}	
	
	\item (Exercise 3.12.11) Show that 
	\[
		\Fl(d_1, \dots, d_k) \cong \orthog(n)/(\orthog(n_1) \times \dots \times \orthog(n_k)),
	\]
	where $n_1 = d_1$ and $n_i = d_i - d_{i-1}$ for $i=2, \dots , k$. (In other words, the $n_i$ are the jumps in dimension as we go up the flag.)
		
	\item Let $M$ be a manifold with an affine connection $\nabla$. Suppose $\alpha \from I \to M$ is a constant curve; that is, $\alpha(t) = p$ for all $t \in I$. Let $V$ be a vector field along $\alpha$, meaning that $V(t) \in T_{\alpha(t)}M = T_p M$ just gives a curve in the tangent space $T_pM$. Show that $\frac{DV}{dt} = V'(t)$; that is, the covariant derivative agrees with the usual derivative in this case, regardless of what $\nabla$ is.
	
	\item (Exercise 4.3.4) Show that an affine connection $\nabla$ is compatible with a Riemannian metric $g$ on $M$ if and only if, for any vector fields $V$ and $W$ along a smooth curve $\alpha \from I \to M$, we have
	\[
		\left.\frac{d}{dt}\right|_{t=t_0} g_{\alpha(t)}(V(t),W(t)) = g_{\alpha(t_0)} \left(\frac{DV}{dt},W\right) + g_{\alpha(t_0)} \left(V,\frac{DW}{dt}\right).
	\]
	In other words, for compatible connections we can use the usual product rule to differentiate the inner product.
		
		 
\end{enumerate}


	
	


\end{document}
